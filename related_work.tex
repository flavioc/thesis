\section{Declarative Programming}

Many programming models have been developed in order to make parallel programs
both easier to write and reason about. The most popular examples of such
paradigms are \emph{logic programming} and \emph{functional programming}.  In
logic languages such as Prolog, researchers took advantage of the
non-determinism of proof-search to evaluate subgoals in parallel with models
such as \emph{or-parallelism} and
\emph{and-parallelism}~\cite{Gupta:2001:PEP:504083.504085}.  In functional
languages, the stateless nature of computation allows multiple expressions to
evaluate safely in parallel.  This has been initially explored in several
languages, such as NESL~\cite{Blelloch:1996:PPA:227234.227246} or
Id~\cite{Nikhil93anoverview}, and later implemented in more modern languages
such as Haskell~\cite{Chakravarty07dataparallel}.

Recently, there has been an increasing interest in declarative and data-centric
languages. MapReduce~\cite{Dean:2008:MSD:1327452.1327492}, for instance, is a
popular data-centric programming model that is optimized for large clusters. The
scheduling and data sharing model is very simple: in the \emph{map phase}, data
is transformed at each node and the result reduced to a final result in the
\emph{reduce phase}. In order to facilitate the writing of programs over large
datasets, SQL-like languages such as
PigLatin~\cite{Olston:2008:PLN:1376616.1376726} have been developed. PigLatin
builds on top of MapReduce and allows the programmer to write complex data-flow
graphs, raising the abstraction and ease of programmability of MapReduce
programs. An alternative to PigLatin/MapReduce is
Dryad~\cite{Isard:2007:DDD:1272996.1273005} that allows programmers to design
arbitrary computation patterns using DAG abstractions. It combines computational
vertices with communication channels (edges) that are automatically scheduled to
run on multiple computers/cores.

\subsection{Origins of LM}

LM is a direct descendant of Meld, a logic programming language developed by
Ashley-Rollman et
al.~\cite{ashley-rollman-iclp09,ashley-rollman-derosa-iros07wksp} in the context
of the Claytronics project~\cite{goldstein-computer05}. Meld is a language
suited for programming massively dynamic distributed systems made of modular
robots. While mutable state is not supported by Meld, Meld performs \emph{state
management} on the persistent facts by keeping a consistent database of facts
whenever there is a change in the axioms. If an axiom is no longer true,
everything derived from that axiom is retracted. Likewise, when a fact becomes
true, the database is immediately updated to take the new logical fact into
account. To take advantage of these state management facilities, Meld supports
\emph{sensing} and \emph{action} facts. Sensing facts are axioms derived from
the state of the world (e.g., temperature, new neighbor node) and action facts
are facts that have an effect on the world (e.g., move), changing the underlying
sensing facts.

Meld was inspired in the P2 system~\cite{Loo-condie-garofalakis-p2}, which
includes a logic programming language called NDlog for writing network
algorithms declaratively. Many ideas about state management were already present
in NDlog.  NDlog is essentially a Datalog~\cite{Ullman:1990:PDK:533142} based
language with a few extensions for declarative networking. Datalog has been
traditionally used in deductive databases, but is now being increasingly used in
other fields such as sensor nets~\cite{Chu:2007:DID:1322263.1322281} and cloud
computing~\cite{alvaro:boom}.

\subsection{Provability}

Many techniques and formal systems have been devised to help reason about
parallel programs.  One such example is the
Owicki-Gries~\cite{Owicki:1976:VPP:360051.360224} deductive system for proving
properties about imperative parallel programs (deadlock detection, termination,
etc). It extends Hoare logic with a stronger set axioms such as parallel
execution, critical section and auxiliary variables. The formal system can be
successfully used in small imperative programs, although using it on languages
such as C is difficult since they do not restrict the use of shared variables.

Some formal systems do not build on top of a known programming paradigm, but
instead create an entirely new formal system for describing concurrent systems.
Process calculus such as $\pi$-calculus~\cite{Milner:1999:CMS:329902} is a good
example of this.  The $\pi$-calculus describes the interactions between
processes through the use of channels for communication. Interestingly, channels
can also be transmitted as messages, allowing for changes in the network of
processes.  Given two processes, $\pi$-calculus is able to prove that they
behave the same through the use of bi-simulation equivalence.

Another interesting model is Mobile UNITY~\cite{Roman97anintroduction}. The
basic UNITY~\cite{UNITY} model assumes that statements could be executed
non-deterministically in order to create parallelism. This principle is applied
to prove properties about the system.  Mobile UNITY transforms UNITY by adding
locations to processes and removing the nondeterministic aspect from local
processes. Processes could then communicate or move between locations.

The Meld language, as a logic programming language, has been used to produce
proofs of correctness. Meld program code is amenable to mechanized analysis via
theorem checkers such as Twelf~\cite{twelf}, a logic system designed for
analyzing program logics and logic program implementations.  For instance, a
meta-module based shape planner program was proven to be
correct~\cite{dewey-iros08,ashley-rollman-iclp09} under the assumption that
actions derived by the program are always successfully applied in the outside
world.  While the fault tolerance aspect is lax, the planner will always reach
the target shape in finite time.  The sketch of the proof is presented in Dewey
et al.~\cite{dewey-iros08}

