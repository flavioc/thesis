%for a more compact document, add the option openany to avoid
%starting all chapters on odd numbered pages
\documentclass[12pt]{cmuthesis}
% This is a template for a CMU thesis.  It is 18 pages without any content :-)
% The source for this is pulled from a variety of sources and people.
% Here's a partial list of people who may or may have not contributed:
%
%        bnoble   = Brian Noble
%        caruana  = Rich Caruana
%        colohan  = Chris Colohan
%        jab      = Justin Boyan
%        josullvn = Joseph O'Sullivan
%        jrs      = Jonathan Shewchuk
%        kosak    = Corey Kosak
%        mjz      = Matt Zekauskas (mattz@cs)
%        pdinda   = Peter Dinda
%        pfr      = Patrick Riley
%        dkoes = David Koes (me)

% My main contribution is putting everything into a single class files and small
% template since I prefer this to some complicated sprawling directory tree with
% makefiles.

% some useful packages
\usepackage{times}
\usepackage{fullpage}
\usepackage{graphicx}
\usepackage{enumitem}
\usepackage{amsmath}
\usepackage{url}
\usepackage{fancyvrb}
\usepackage{verbatim}
\usepackage{caption}
\usepackage{subcaption}
\usepackage{multirow}
\usepackage{floatrow}
\usepackage{proof-dashed}
\usepackage{amsthm}
\usepackage{thmtools}
\usepackage{mathtools}
\usepackage{etoolbox}
\usepackage[]{algorithm2e}
\usepackage[numbers,sort]{natbib}
\usepackage[backref,pageanchor=true,plainpages=false, pdfpagelabels, bookmarks,bookmarksnumbered,
%pdfborder=0 0 0,  %removes outlines around hyper links in online display
]{hyperref}
\graphicspath{{./figures/}{./benchmarks/data/}{./benchmarks/}}
\usepackage{latexsym}
\usepackage{amssymb}            % for \multimap (-o)
\usepackage{stmaryrd}           % for \binampersand (&), \bindnasrepma (\paar)

\newcommand{\m}[1]{\mathsf{#1}}
\newcommand{\f}[1]{\framebox{#1}}

\newcommand{\eph}{\mathit{eph}}
\newcommand{\pers}{\mathit{pers}}
\newcommand{\um}[1]{\underline{\m{#1}}}

\newcommand{\seq}{\vdash}
\newcommand{\semi}{\mathrel{;}}
\newcommand{\lequiv}{\mathrel{\dashv\vdash}}

% symbols of linear logic
\newcommand{\lolli}{\multimap}
\newcommand{\tensor}{\otimes}
\newcommand{\with}{\mathbin{\binampersand}}
\newcommand{\paar}{\mathbin{\bindnasrepma}}
\newcommand{\one}{\mathbf{1}}
\newcommand{\zero}{\mathbf{0}}
\newcommand{\bang}{{!}}
\newcommand{\whynot}{{?}}
\newcommand{\bilolli}{\mathrel{\raisebox{1pt}{\ensuremath{\scriptstyle\circ}}{\lolli}}}
% \oplus, \top, \bot

\newcommand{\selector}[4]{[\; #1 \Rightarrow #2; \; #3 \;] \lolli #4}
\newcommand{\comprehension}[3]{\{ \; #1; \; #2; \; #3 \; \}}
\newcommand{\aggregate}[6]{[\; #1 \Rightarrow #2; \; #3; \; #4; \; #5; \; #6 \;]}

\newcommand{\taba}{\;\;\;}

\newcommand{\mz}{\m{match} \;}
\newcommand{\tab}[0]{\;\;\;\;}
\newcommand{\dz}{\m{derive} \;}
\newcommand{\comp}[0]{\m{comp} \;}
\newcommand{\az}{\m{apply} \;}
\newcommand{\doz}{\m{run} \;}
\newcommand{\seqnocut}[3]{#1 ; #2 \Rightarrow #3}
\newcommand{\defeq}{\buildrel\triangle\over =}
\newcommand{\seqx}[3]{#1 ; #2 \implies #3}
\newcommand{\compr}[1]{\m{def} \; #1}

\newcommand{\mo}{\m{match}_1 \;}
\newcommand{\cont}{\m{cont} \;}
\newcommand{\contc}{\m{contc} \;}
\newcommand{\done}{\m{derive}_1 \;}
\newcommand{\doo}{\m{run}_1 \;}
\newcommand{\mc}[0]{\m{match}_c \; }
\newcommand{\dall}[0]{\m{fix} \; }
\newcommand{\strans}[0]{\m{strans} \;}
\newcommand{\dc}{\m{derive}_c \;}
\newcommand{\ao}{\m{apply}_1 \;}


\makeatletter
\def\thm@space@setup{%
  \thm@preskip=15pt \thm@postskip=15pt
}
\makeatother
\makeatletter
\newtheoremstyle{indented}
  {15pt}% space before
  {0pt}% space after
  {\parshape 1 0em \linewidth}% body font
  {0pt}% indent
  {\bfseries}% header font
  {.}% punctuation
   {\newline}% after theorem header
  {}% header specification (empty for default)
\makeatother

\theoremstyle{indented}

% Approximately 1" margins, more space on binding side
%\usepackage[letterpaper,twoside,vscale=.8,hscale=.75,nomarginpar]{geometry}
%for general printing (not binding)
\usepackage[letterpaper,twoside,vscale=.8,hscale=.75,nomarginpar,hmarginratio=1:1]{geometry}
\newtheorem{corollary}{Corollary}
\newtheorem{definition}{Definition}
\newtheorem{lemma}{Lemma}
%\newtheorem{theorem}{Theorem}
\declaretheorem{theorem}
\newtheorem{invariant}{Invariant}
\AtBeginEnvironment{theorem}{\parindent0pt}
\AtBeginEnvironment{lemma}{\parindent0pt}
\AtBeginEnvironment{definition}{\parindent0pt}
\AtBeginEnvironment{corollary}{\parindent0pt}
\AtBeginEnvironment{invariant}{\parindent0pt}

% Provides a draft mark at the top of the document. 
\draftstamp{\today}{}

\begin{document} 
\frontmatter

%initialize page style, so contents come out right (see bot) -mjz
\pagestyle{empty}

\title{{\it \huge PhD Thesis}\\
{\bf Linear Logic and Coordination for Parallel Programming}}
\author{Fl\'{a}vio Manuel Fernandes Cruz}
\date{\today}
\Year{2015}
\trnumber{}

\committee{
Umut Acar\\
Luis Barbosa\\
Seth Goldstein\\
Carlos Guestrin\\
Frank Pfenning\\
Ricardo Rocha\\\
}

\support{}
\disclaimer{}

% copyright notice generated automatically from Year and author.
% permission added if \permission{} given.

\keywords{linear logic, parallel programming, logic programming}

\maketitle

\pagestyle{plain} % for toc, was empty

%% Obviously, it's probably a good idea to break the various sections of your thesis
%% into different files and input them into this file...

Parallel programming is known to be difficult to apply, exploit and reason
about. Programs written using low level parallel constructs tend to be
problematic to understand and debug. Declarative programming is a step towards
the right direction because it moves the details of parallelization from the
programmer to the runtime system. However, this paradigm leaves the programmer
with little opportunities to coordinate the execution of the program, resulting
in suboptimal declarative programs.  We propose a new declarative programming
language, called Linear Meld~(LM), that provides a solution to this problem by
supporting data-driven dynamic coordination mechanisms that are semantically
equivalent to regular computation.

LM is a logic programming language designed for programs that operate on graphs
and supports coordination and structured manipulation of mutable state.
Coordination is achieved through two mechanisms: (i) coordination facts, which
allow the programmer to control how computation is scheduled and how data is
laid out, and (ii) thread facts, which allow the programmer to reason about the
state of the underlying parallel architecture.  The use of coordination allows the
programmer to combine the inherent implicit parallelism of LM with a form of
declarative explicit parallelism provided by coordination, allowing the
development of complex coordinated programs that run faster than regular
programs. Furthermore, since coordination is indistinguishable from regular
computation, it allows the programmer to reason both about the problem at hand
and also about parallel execution.

We have written several graph algorithms, search algorithms and machine learning
algorithms in LM. For some programs, we have written informal proofs of
correctness to show that programs are easily proven correct. We have also
engineered a compiler and runtime system that is able to run LM programs on
multicore architectures with decent performance and scalability.


%\begin{acknowledgments}
%My advisor is cool.
%\end{acknowledgments}



\tableofcontents
\listoffigures
\listoftables
\renewcommand{\listtheoremname}{List of Equations}
\listoftheorems

\mainmatter

%% Double space document for easy review:
%\renewcommand{\baselinestretch}{1.66}\normalsize

% The other requirements Catherine has:
%
%  - avoid large margins.  She wants the thesis to use fewer pages, 
%    especially if it requires colour printing.
%
%  - The thesis should be formatted for double-sided printing.  This
%    means that all chapters, acknowledgements, table of contents, etc.
%    should start on odd numbered (right facing) pages.
%
%  - You need to use the department standard tech report title page.  I
%    have tried to ensure that the title page here conforms to this
%    standard.
%
%  - Use a nice serif font, such as Times Roman.  Sans serif looks bad.
%
% Other than that, just make it look good...

\chapter{Introduction}
Multicore architectures have become more widespread recently and are forcing the
development of new software methodologies that enable developers to take
advantage of increasing processing power through parallelism. However, parallel
programming is difficult, usually because programs are written in imperative and
stateful programming languages that make use of low level synchronization
primitives such as locks, mutexes and barriers. This tends to make the task of
managing multi threaded execution complicated and error-prone, resulting in race
hazards and deadlocks. In the future, \emph{many-core} processors will make this
task even more daunting.

Past developments in parallel and distributed programming have given birth to
several programming models. At one end of the spectrum are the lower-level
programming abstractions such as \emph{message passing} (e.g.,
MPI~\cite{gabriel04-open-mpi}) and \emph{shared memory} (e.g.,
Pthreads~\cite{Butenhof:1997:PPT:263953} or
OpenMP~\cite{Chapman-2007-UOP-1370966}). While such abstractions give a lot of
control to the programmer and provide excellent performance, these APIs are hard
to use and debug, which makes it difficult to prove that a program is correct,
for instance. On the opposite end, we have many declarative programming
models~\cite{Blelloch:1996:PPA:227234.227246} that can exploit some form of
implicit parallelism. Implicit parallelism allows the runtime system to
automatically exploit parallelism by deciding which tasks to run in parallel.
However, it is not always obvious which tasks to run in parallel and which tasks
to run sequentially. This important issue, known as the \emph{granularity
problem}, needs to be handled well because creating too many parallel tasks will
make the program run very slowly since there is a significant overhead for
creating parallel tasks. It is then fundamental that there is a good mapping
between tasks and available processors.  Another different, but related problem,
is that declarative programming paradigms offer little to no control to the
programmer over how parallel execution is scheduled or how data is laid out,
making it hard to improve efficiency. Even if the runtime system reasonably
solves the granularity problem, there is a lack of specific information about
the program that a compiler cannot easily deduce. Furthermore, the program's
data layout is also critical for performance since poor data locality will
degrade performance even if the task granularity is optimal.  If the programmer
could provide such information, then execution would improve in terms of run
time, memory usage, or scalability.

In the context of the Claytronics project~\cite{goldstein-computer05},
Ashley-Rollman et al.~\cite{ashley-rollman-iclp09,
ashley-rollman-derosa-iros07wksp} created Meld, a logic programming language
suited to program massively distributed systems made of modular robots with a
dynamic topology. Meld programs can derive actions that are used by the
robots to act on the outside world. The distribution of computation is done by
first partitioning the program state across the robots and then by making the
computation local to the robot. Because Meld programs are sets of logical
clauses, they are more amenable to proofs than programs written using
lower-level programming abstractions.

In this thesis, we present Linear Meld (LM), a new language for parallel
programming over graph data structures that extends the original Meld
programming language with linear logic and coordination. Linear logic gives the
language a structured way to manage state, allowing the programmer to assert and
retract logical facts.  While the original Meld sees a running program as an
ensemble of robots, LM sees the program as a graph of node data structures,
where each node performs computation independently of other nodes and is able to
communicate with its neighborhood of nodes. Using the graph as the main program
abstraction, LM also solves the granularity program by allowing nodes to be
grouped into tasks that can be ran in a single thread of control.

LM introduces a new mechanism, called coordination, that is semantically
equivalent to regular computation and allows the programmer to reason about
parallel execution. Coordination introduces the concept of \emph{coordination
facts}, which are logical facts used for scheduling and data partitioning
purposes, and \emph{thread facts}, which allow the programmer to reason about
the state of the underlying parallel architecture. The use of these new
facilities moves the LM language from the paradigm of implicit parallelism to
some form of declarative explicit parallelism, but without the pitfalls of
imperative parallel programming. In turn, this makes LM a novel declarative
language that allows the programmer to optionally control how execution and data
is managed by the execution system.

Our main goal with LM is to efficiently execute provably correct declarative
graph-based programs on multi core machines. To show this, we wrote many
graph-based algorithms, proved program correctness for some programs and
developed a compiler and runtime system where we have seen good experimental
results. Finally, we have also used coordination in some programs and we were
able to see interesting improvements in terms of run time, memory usage and
scalability.

\section{Thesis Statement}


We propose Linear Meld (LM), a new linear logic programming language, designed
to write parallel graph based programs on multicores.  We argue that LM is a
superior declarative programming model because it not only automatically
parallelizes programs, but, in cases where it is necessary, it allows the
programmer to control parallel scheduling and placement of data to further
improve the program run time and scalability.  Since LM is based on solid
logical foundations, we argue that we can also prove interesting properties of
programs, including correctness and termination. We will prove our thesis
through five major contributions:

\begin{itemize}
   
   \item Linear Logic

   We integrated linear logic into our language, so that program state can be
   encoded naturally. The original Meld was fully based on classical logic where
   everything that is derived is true forever. Linear logic turns some facts
   into resources that will be consumed when a rule is applied.  To the best of
   our knowledge, LM is the first linear logic based language implementation
   that attempts to solve real world problems.

   % need to talk about thread facts here.

   \item Coordination
   
   LM offers execution control to the programmer through the use of coordination
   facts to make the program faster and more scalable. These coordination
   facts change how the runtime system schedules computation and partitions data
   and is semantically equivalent to standard computation facts.
   We used the concept of \emph{action facts} and \emph{sensing facts} o coordinate the execution of
   programs.  We can increase the priority of certain nodes during runtime
   according to the state of the computation and to the state of the runtime in
   order to make better scheduling decisions so that programs can run faster.
   
   \item Provability
   
   Since LM is a logic programming language, we leveraged the logical
   foundations of the language to show how to prove correctness and termination of a few
   programs. We also show that coordination facts do not change those
   correctness proofs of programs but only improve run time, scalability or
   memory usage.

   \item Multicore Parallelism
   
   We divide the logical facts across all the nodes of the graph. Since the
   logical rules only make use of facts from a node, computation can be
   performed locally, without reference from other nodes of the graph.
   We envision the application as a communicating graph data structure where
   each processing unit performs work on a different subset of the graph, thus
   enabling concurrency. This is an advantage of LM since we can run programs on
   many different types of distributed systems as long as the underlying runtime
   system uses the appropriate communication facilities.

   \item Experimental Results

   We have implemented a compiler and a virtual machine prototype from
   scratch that executes on multicore machines.  We have implemented programs
   such as belief propagation, belief propagation with residual splash,
   PageRank, graph coloring, N queens, shortest path, diameter estimation,
   MapReduce, game of life, quick-sort, neural network training, among others.
   Our experimental results show that our particular implementation provides
   good scalability with at least 24 cores.
   
\end{itemize}




\chapter{Related Work}

In this section, we explore programming languages and programming models that
allow coordination and/or scheduling of computation and/or processing units.

\subsection{Programming Languages}

Many programming languages follow what is called the coordination
paradigm~\cite{Papadopoulos98coordinationmodels}. This form of distributed
programming divides execution in two parts: \emph{computation}, where the actual
computation is performed, and \emph{coordination}, which deals with
communication and cooperation between processing units. This paradigm attempts
to clearly distinguish between these two parts by providing abstractions for
coordination in an attempt to provide architecture and system-independent forms
of communication.

We can identify two main types of coordination models:

\begin{description}
   \item[Data-Driven:]
   
   In a data-driven model, the state of the computation depends on both the data
   being received or transmitted by the processes and the current configuration
   of the coordinated processes. The coordinated process is not only responsible
   for reading and manipulating the data but is also responsible for
   coordinating itself and/or other processes. Each process must intermix the
   coordination directives provided by the coordination model with the
   computation code. While these directives have a very clear interface, it is
   in the programmer's responsibility to use them correctly.

   \item[Task-Driven:]
   
   In this model, the coordination code is more cleanly separated from the
   computation code. While in data-driven models, the content of the data
   exchanged by the processes will affect how the processes coordinate with each
   other, in a task-driven model, the process behavior depends only on the
   coordination patterns that are setup before hand. This means that the
   computation component is defined as a black box and there are clearly defined
   interfaces for input/output. These interfaces are usually defined as a
   full-fledged coordination language and not as simple directives present in
   the data-driven models.  \end{description}

Linda~\cite{linda} is probably the most famous coordination model. Linda
implements a data-driven coordination model and features a \emph{tuple space}
that can be manipulated using the following coordination directives:
\texttt{out(t)} writes a tuple \texttt{t} into the tuple space; \texttt{in(t)}
reads a tuple using the template \texttt{t}; \texttt{rd(t)} retrieves a copy of
the tuple \texttt{t} from the tuple space; and \texttt{eval(p)} puts a process
\texttt{p} in the tuple space and executes it in parallel.  Linda processes do
not need to know the identity of other processes because processes only
communicate through the tuple space.  Linda can be implemented on top of many
popular languages by simply creating a communication and storage mechanism for
the tuple space and then adding the directives as a language library.

Another early coordination language is Delirium~\cite{Delirium}. Unlike Linda,
which is embedded into another language, Delirium actually embeds operators
written in other languages inside the Delirium language. The advantages of
Delirium are improved abstraction and easier debugging because sequential
operators are isolated from the coordination language.

Linda and Delirium are limited in the sense that the programmer can only
coordinate the scheduling of processing units, while placement of data is left
to the implementation. LM differs from those languages because coordination acts
on data instead of processing units. The abstraction is then raised by
considering data and algorithmic aspects of the program instead of focusing on
how processing units are used. Furthermore, LM is both a coordination language
and a computation language and there is no distinction between the two
components.

The original Meld~\cite{ashley-rollman-iclp09} can also be seen as a kind of
data-driven coordination language. The important distinction is that in Meld
there's no explicit coordination directives. When Meld rules are activated they
may derive facts that need to be sent to a neighboring robot. In turn, this will
activate computation on the neighbor. Robot communication is implemented by
\emph{localizing} the program rules and then by creating \emph{communication
rules}.

The LM language also implements communication rules, however it goes further
because some facts, action facts, can change how the processing units schedule
nodes to be executed, namely, which node is to be computed next, which may in
turn change the program's final result. This result in a more complete
inter-play between coordination code and data.

\subsection{Programming Models}

The Galois~\cite{Pingali:2011:TPA:1993316.1993501} programming model implements
autonomous scheduling by default, where activities may be rolled back in case of
conflict. However, it is possible to employ a concrete scheduling strategy for
coordinating parallel execution in order to improve execution and avoid
conflicts.  First, there is \emph{compile-time coordination}, where the
scheduling ordered is computed during compilation and is pre-defined before the
program is executed. Secondly, there is \emph{runtime coordination}, where the
order of activities is computed during execution. The execution of the algorithm
proceeds in rounds: first, a set of non-conflicting activities is computed and
then executed by applying the operator; conflicting activities are postponed to
the next round. The third and last scheduling strategy is \emph{just-in-time
coordination} where the order of activities is defined by the underlying data
structure where the operator is applied (for instance, computing on a graph
may depend on its topology).

In the context of the Galois model, Nguyen et al.~\cite{nguyen11} expanded the
concept of runtime coordination with the introduction of a flexible approach to specify
scheduling policies for Galois programs. This approach was motivated by the fact
that some algorithms can be executed faster if computations use better
scheduling strategies. The scheduling language specifies 3 main scheduler types:
\texttt{FIFO} (First-In First-Out), \texttt{LIFO} (Last-In First-Out) and
\texttt{OrderedByMetric} (order activities by some metric). These schedulers can
be composed and synthesized without requiring users to write complex concurrent
code.

Elixir~\cite{Prountzos:2012:ESS:2384616.2384644} is a domain specific language
that builds on top of the Galois and allows easy specification of scheduling
strategies.  The main idea behind Elixir is that the user should be able to
specify how operator application is scheduled and the framework will compile
this high level specification to low level code using the provided scheduling
specification. One of the motivating examples is the Single Source Shortest Path
program that can be specified using multiple scheduling specifications,
generating different well-known shortest path algorithms such as the
Dijkstra or Bellman-Ford algorithm. Unlike the work of Nguyen et
all.~\cite{nguyen11}, Elixir does not allow graph mutations.

Halide~\cite{Ragan-Kelley:2013:HLC:2491956.2462176} is a language and compiler
for image processing pipelines with the goal of optimizing parallelism, locality
and re-computation. Halide decouples the algorithm definition from its execution
strategy, allowing the compiler to find which execution strategy may be the best
for optimizing for locality and parallelism. The language allows the programmer
to specify the scheduling strategy, allowing the programmer to decide the order
of computations, what intermediate results need to be stored, how to split the
data among processing units and how to use vectorization and the well-known
sliding window mechanism. However, the compiler is able to use stochastic search
to find good schedules for Halide pipelines. Notably, experimental results
indicate that automatic search sometimes leads to better execution than
hand-written code.

In contrast to the previous systems, LM stands alone in making coordination
(both scheduling and partitioning) a first-class programming construct and
semantically equivalent to computation. Furthermore, LM distinguishes itself by
supporting data-driven dynamic coordination, particularly for irregular data
structures. Elixir and Galois do not support coordination for data partitioning,
and, in Elixir, the coordination specification is separated from computation,
limiting the programmability of coordination. Compared to LM, Halide is
targeted for regular applications and therefore only supports compile time
coordination.


\chapter{Linear Meld}

Linear Meld (LM) is a forward chaining logic programming language where a
program is defined by a \emph{database of facts} and by a set of
\emph{derivation rules}. LM supports structured manipulation of mutable state
through the use of linear logic~\cite{girard-87} by supporting two types of
logical facts, namely, \emph{persistent facts} (which cannot be retracted) and
\emph{linear facts} (which can be retracted).

LM programs are evaluated as follows. Initially, we populate the database with
the program's initial facts and then determine which derivation rules can be
applied by using the facts from the database. Once a rule is applied, we derive
new facts, which are then added to the database. If a rule uses linear facts,
they are consumed and thus deleted from the database. The program stops when we
reach \emph{quiescence}, that is, when we can no longer obtain new facts through
rule derivation.

LM sees the database of facts as a graph data structure where each node contains
facts belonging to that node. The database of facts can also be seen as a
collection of node objects and each node contains several attributes represented
as facts that describe the node. Derivation rules are restricted so that they
are only allowed to manipulate facts belonging to the same node. This
restriction allows nodes of the graph to compute independently of other nodes
since they can only perform computation with their current attributes.
Communication between nodes arises when a rule derives a fact that belongs to
another node. These design decisions allow LM to be concurrent but also make LM
easier to implement.

\section{A Taste Of LM}

In order to understand how LM programs are written, we now present and discuss
three elementary LM programs.

\subsection{First Example: Message Routing}\label{section:language:message}

Figure~\ref{code:language:message} shows the first LM program, a message routing
program that simulates message transmission through a network of nodes.
Lines~\ref{line:language:message_pred1}-\ref{line:language:message_pred2}
declare the predicates used in the program's rules. Note that the first argument
of every predicate must be typed as \code{node} because the first argument
indicates where the fact lives in the graph. Predicate \code{edge} is a
\emph{persistent predicate} while \code{message} and \code{processed} are
\emph{linear predicates}. Persistent predicates model facts that are never
retracted from the database, while linear predicates model linear facts which
are retracted when used in rules. To improve readability of LM rules, persistent
predicates are preceded with a \code{!} symbol. Predicate \code{edge} represents
the connections between nodes, predicate \code{message} contains the message
content and the route list, and predicate \code{processed} keeps count of the
number of messages routed at each node. Along with the type, a predicate
argument can also be named (e.g., see \code{Neighbor} and \code{Content}) for
documentation purposes.

\begin{figure}[h!]
\begin{LineCode}[commandchars=\*\{\}]
type edge(node, node Neighbor).*label{line:language:message_pred1}*hfill// Predicate declaration
type linear message(node, string Content, list node Routing).
type linear processed(node, int Total).*label{line:language:message_pred2}

message(A, Content, [B | L]),*label{line:language:message_first1}*hfill// Rule 1
!edge(A, B),
processed(A, N)
   -o processed(A, N + 1),
      message(B, Content, L).*label{line:language:message_first2}

message(A, Content, []),*label{line:language:message_second1}*hfill// Rule 2
processed(A, N)
   -o processed(A, N + 1).*label{line:language:message_second2}
\end{LineCode}
\mycap{LM code for routing messages in a graph.}
\label{code:language:message}
\end{figure}

The message routing program in Fig.~\ref{code:language:message} implements two
rules. An LM rule has the form $L_1, \cdots, L_n \mathtt{-o} \; R_1, \cdots,
R_m$, where $L_1, \cdots, L_n$ is the \emph{left-hand side}~(LHS) of the rule
and $R_1, \cdots, R_m$ is \emph{right-hand side}~(RHS) of the rule. The meaning
of a rule is then as follows: if facts $L_1, \cdots, L_n$ are present in the
database then consume all the facts $L_i$ that are linear facts and derive the
facts $R_1, \cdots, R_m$ from the RHS. Note that in the rules, the LHS of each
rule uses only facts from the same node (represented by \code{A} in both rules
of Fig.~\ref{code:language:message}), but the rule's RHS may derive facts that
belong to other nodes (case of \code{B} in the first rule) if the variable is
instantiated in the LHS.

The first rule
(lines~\ref{line:language:message_first1}-\ref{line:language:message_first2})
grabs the head node \code{B} in the route list (third argument of \code{message}
using the syntax \code{[B | L]} to represent the head and tail of a list) and
ensures that a communication edge exists (through \code{edge(A,~B)}). If so,
the number of processed messages is increased by consuming \code{processed(A,~N)}
and deriving \code{processed(A,~N~+~1)}, along with a new message to
the head node \code{B}.  When the route list is empty, the message has reached
its destination and thus it is simply consumed (second rule in lines
\ref{line:language:message_second1}-\ref{line:language:message_second2}).

The initial facts of the program are presented in
Fig.~\ref{code:language:message_facts}. The \code{edge} facts describe the
structure of the graph, where the node in the first argument has a direct
connection with the node in the second argument. Node literals are represented
using the syntax \code{@N}, where \code{N} is an non-negative number. We also
declare the message that needs to be routed with the content \code{"hello
world"} and a route list with the nodes \code{@3} and \code{@4}.

\begin{figure}[h!]
\begin{LineCode}[commandchars=\*\{\}]
!edge(@1, @2). !edge(@2, @3).
!edge(@3, @4). !edge(@1, @3).
processed(@1, 0). processed(@2, 0).
processed(@3, 0). processed(@4, 0).
message(@1, "hello world", [@3, @4]).
\end{LineCode}
\mycap{Initial facts for the message routing program. There is only one message ("hello
world") to route through nodes \code{@3} and \code{@4}.}
\label{code:language:message_facts}
\end{figure}

In Fig.~\ref{fig:message_trace} we present an execution trace of the message
routing program. The database is represented as a graph structure where the
edges represent the \code{edge} initial facts. To simplify the figure, we
dropped the first argument of each fact since the first argument corresponds to
the node where the fact is placed. In Fig.~\ref{fig:message_trace}(a) the
database is initialized with the program's initial facts. Note that the initial
\code{message} fact is instantiated at node \code{@1}. After applying rule 1, we
get the database represented in Fig.~\ref{fig:message_trace}(b), where the
message has been derived at node \code{@3}. After applying rule 1 again, the
message is then routed to node \code{@4} (Fig.~\ref{fig:message_trace}(c))
where it will be consumed (Fig.~\ref{fig:message_trace}(d)).

\newcommand{\visitsize}{0.38}

\begin{figure}[h]
        \centering
        \begin{subfigure}[b]{\visitsize\textwidth}
                \includegraphics[width=\textwidth]{figures/message/message_trace1}
                \mycap{Initial database.}
                \label{fig:message_trace1}
        \end{subfigure}%
        ~
        \begin{subfigure}[b]{\visitsize\textwidth}
                \includegraphics[width=\textwidth]{figures/message/message_trace2}
                \mycap{After applying rule 1 at node \code{@1}.}
                \label{fig:message_trace2}
        \end{subfigure}\\
        \begin{subfigure}[b]{\visitsize\textwidth}
                \includegraphics[width=\textwidth]{figures/message/message_trace3}
                \mycap{After applying rule 1 at node \code{@3}.}
                \label{fig:message_trace3}
        \end{subfigure}%
        ~
        \begin{subfigure}[b]{\visitsize\textwidth}
                  \includegraphics[width=\textwidth]{figures/message/message_trace4}
                  \mycap{After applying rule 2 at node \code{@4}.}
                  \label{fig:message_trace4}
          \end{subfigure}
        \mycap{An execution trace for the message routing program. The message "hello
        world" travels from node \code{@1} to node \code{@4}.}\label{fig:message_trace}
\end{figure}

The attentive reader will wonder how much concurrency is available in this
particular routing implementation. For a single message, there is no concurrency
because the message follows a pre-determined path as it travels from node to
node. However, concurrency arises when the program needs to route multiple
messages. The amount of concurrency is then dependent on the messages routing
lists being non-overlapping. If messages travel in different paths, then there
is more concurrency because more nodes are routing messages at the same time.


\subsection{Second Example: Key/Value Dictionary}\label{sec:language:key_value}

\begin{figure}[ht]
\begin{Verbatim}[numbers=left,fontsize=\codesize,commandchars=\*\{\}]
type left(node, node Child).*label{line:language:dict_header1}*hfill // Predicate declaration
type right(node, node Child).
type linear value(node, int Key, string Value).
type linear replace(node, int Key, string Value).*label{line:language:dict_header2}

replace(A, K, New),*label{line:language:dict_first1}*hfill// Rule 1: we found our key
value(A, K, Old)
   -o value(A, K, New).*label{line:language:dict_first2}

replace(A, RKey, RValue),*label{line:language:dict_second1}*hfill// Rule 2: go left
value(A, Key, Value),
RKey < Key,
!left(A, B)
   -o value(A, Key, Value),
      replace(B, RKey, RValue).*label{line:language:dict_second2}

replace(A, RKey, RValue),*label{line:language:dict_third1}*hfill// Rule 3: go right
value(A, Key, Value),
RKey > Key,
!right(A, B)
   -o value(A, Key, Value),
      replace(B, RKey, RValue). *label{line:language:dict_third2}
\end{Verbatim}
\mycap{LM program for replacing a key's value in a BST dictionary.}
\label{code:language:btree_replace}
\end{figure}

Our second example, shown in Fig.~\ref{code:language:btree_replace}, implements
the key update operation for a binary search tree~(BST) represented as a
key/value dictionary. Each LM node represents a binary tree node. We first
declare all the predicates in
lines~\ref{line:language:dict_header1}-\ref{line:language:dict_header2}.
Predicates \code{left} and \code{right} represent the child nodes of each BST
node. Linear predicate \code{value} stores the key/value pair of a node, while
the linear predicate \code{replace} represents an update operation where the key
in the second argument is to be updated to the value in the third argument.

The algorithm uses three rules for the three possible cases of updating a key's
value. The first rule
(lines~\ref{line:language:dict_first1}-\ref{line:language:dict_first2}) performs
the update by removing \code{replace(A, K, New)} and \code{value(A, K, Old)} and
deriving \code{value(A, K, New)}. The second rule
(lines~\ref{line:language:dict_second1}-\ref{line:language:dict_second2})
recursively picks the left branch for the update operation by deleting
\code{replace(A, RKey, RValue)} and re-deriving it at node \code{B}. Similarly,
third rule
(lines~\ref{line:language:dict_third1}-\ref{line:language:dict_third2})
recursively descends the right branch. The derivation of \code{replace} facts on
node \code{B} can also be seen as a implicit \emph{message passing} from
\code{A} to \code{B}, since \code{B} is a different node than \code{A} and the
LHS of each rule can only manipulate facts from the same node.

The initial facts of the program are presented in
Fig.~\ref{code:language:btree_replace_initial} and describe the initial binary
tree configuration, including keys and values, and the \code{replace(@1, 6,
"x")} fact instantiated at the root node \code{@1} that manifests the intent to
change the value of key 6 to "x".

\begin{figure}[ht]
\begin{Verbatim}[numbers=left,fontsize=\codesize,commandchars=\*\{\}]
!left(@1, @2).
!right(@1, @3).
!left(@2, @4).
!right(@2, @5). 
!left(@3, @6).
!right(@3, @7).

value(@1, 3, "a").
value(@2, 1, "b").
value(@3, 5, "c").
value(@4, 0, "d").
value(@5, 2, "e").
value(@6, 4, "f").
value(@7, 6, "g").

// Update key 6 to value "x".
replace(@1, 6, "x").
\end{Verbatim}
\mycap{Initial facts for replacing a key's value in a BST dictionary.}
\label{code:language:btree_replace_initial}
\end{figure}

Figure~\ref{fig:language:btree_trace} represents the trace of the algorithm. The program
database is partitioned by the seven nodes using the first argument of each
fact. In Fig.~\ref{fig:language:btree_trace}~(a), we present the database filled with the
program's initial facts. Next, we follow the right branch using rule 3 since $6
> 3$ (Fig.~\ref{fig:language:btree_trace}~(b)).  We use the same rule again in
Fig.~\ref{fig:language:btree_trace}~(c) where we finally reach the key 6. Here, we apply
rule 1 and \code{value(@7, 6, "g")} is updated to \code{value(@7, 6, "x")}.

\begin{figure}[h]
        \centering
        \begin{subfigure}[b]{0.5\textwidth}
                \includegraphics[width=\textwidth]{figures/btree/btree_trace1}
                \mycap{Initial database.}
                \label{fig:language:btree_trace1}
        \end{subfigure}%
        ~
        \begin{subfigure}[b]{0.5\textwidth}
                \includegraphics[width=\textwidth]{figures/btree/btree_trace2}

                \mycap{After applying rule 3 at node \code{@1}.}

                \label{fig:language:btree_trace2}
        \end{subfigure}\\
        \begin{subfigure}[b]{0.5\textwidth}
                \includegraphics[width=\textwidth]{figures/btree/btree_trace3}
                \mycap{After applying rule 3 at node \code{@3}.}
                \label{fig:language:btree_trace3}
        \end{subfigure}%
        ~
        \begin{subfigure}[b]{0.5\textwidth}
                  \includegraphics[width=\textwidth]{figures/btree/btree_trace4}
                  \mycap{After applying rule 1 at node \code{@7}.}
                  \label{fig:language:btree_trace4}
          \end{subfigure}
        \mycap{An execution trace for the binary tree dictionary
           algorithm. The first argument of each fact was dropped and the
           address of the node was placed beside it.}\label{fig:language:btree_trace}
\end{figure}

Like the message routing example shown in the previous section, the amount of
concurrency present in this example depends on the number of \code{replace}
facts. If there are many \code{replace} operations to perform on different parts
of the BST, then there is more concurrency. On the other hand, if only a few BST
nodes are being updated, then concurrency is reduced. This is the same behavior
that one would expect from an parallel implementation of the BST data structure
in an imperative language.


\subsection{Third Example: Graph Visit}

\begin{figure}[h!]
\begin{Verbatim}[numbers=left,fontsize=\codesize,commandchars=\*\[\]]
type edge(node, node).*hfill// Predicate declaration
type linear visit(node).
type linear unvisited(node).
type linear visited(node).

visit(A), *label[line:language:visit_first1]*hfill // Rule 1: visit node
unvisited(A) -o
   visited(A),
   {B | !edge(A, B) -o visit(B)}.*label[line:language:visit_first2]*label[line:language:visit_comprehension]

visit(A),*label[line:language:visit_second1]*hfill // Rule 2: node already visited
visited(A)
   -o visited(A).*label[line:language:visit_second2]
\end{Verbatim}
  \mycap{LM code for the graph visit program.}
  \label{code:language:visit}
\end{figure}

Our third example, shown in Fig.~\ref{code:language:visit}, presents another LM
program that, for a given graph of nodes, performs a visit to all nodes
reachable from node \code{@1}. The first rule of the program
(lines~\ref{line:language:visit_first1}-\ref{line:language:visit_first2}) visits
node \code{A} for the first time: fact \code{visited(A)} is derived and a
\emph{comprehension} construct is used to go through all the edge facts and then
derive \code{visit(B)} at each neighbor node \code{B}. The second rule of the
program
(lines~\ref{line:language:visit_second1}-\ref{line:language:visit_second2})
applies when the node is already visited more than once: we keep the
\code{visited} fact and delete \code{visit}. The initial facts shown in
Fig.~\ref{code:language:visit_initial} use the \code{visit(@1)} fact to start
the program.

\begin{figure}[h!]
\begin{Verbatim}[numbers=left,fontsize=\codesize,commandchars=\*\[\]]
!edge(@1, @2).
!edge(@1, @4).
!edge(@2, @3).
!edge(@2, @4).
!edge(@2, @1).
!edge(@3, @2).
!edge(@4, @1).
!edge(@4, @2).

unvisited(@1).
unvisited(@2).
unvisited(@3).
unvisited(@4).

visit(@1).
\end{Verbatim}
  \mycap{Initial facts for the graph visit program. Nodes reachable from node \code{@1} will be marked as \code{visited}.}
  \label{code:language:visit_initial}
\end{figure}

Fig.~\ref{fig:exec_trace} shows a possible execution trace for the visit
program. After applying the first rule at node \code{@1} we get the database in
Fig~\ref{fig:exec_trace}~(b).  Note that node \code{@1} is now visited and nodes
\code{@2} and \code{@4} now have a \code{visit} fact. At this point, we could
either apply rule 1 at node \code{@2} or at node \code{@4}.  For this specific
trace, we apply the rule at node \code{@2}, resulting in
Fig.~\ref{fig:exec_trace}~(c). Node \code{@4} now has two \code{visit} facts,
thus we can apply rule 1 followed by rule 2, therefore consuming both
\code{visit} facts and deriving \code{visited}. In addition, we can also apply
rule 1 at node \code{@3} to reach the state of Fig.~\ref{fig:exec_trace}~(d).

\begin{figure}[h]
        \centering
        \begin{subfigure}[b]{0.45\textwidth}
                \includegraphics[width=\textwidth]{figures/visit/trace1}
                \mycap{Initial database.}
                \label{fig:exec_trace1}
        \end{subfigure}%
        ~ %add desired spacing between images, e. g. ~, \quad, \qquad etc.
          %(or a blank line to force the subfigure onto a new line)
        \begin{subfigure}[b]{0.45\textwidth}
                \includegraphics[width=\textwidth]{figures/visit/trace2}
                \mycap{After applying rule 1 at node \code{@1}.}
                \label{fig:exec_trace2}
        \end{subfigure}\\
        \begin{subfigure}[b]{0.45\textwidth}
                \includegraphics[width=\textwidth]{figures/visit/trace3}
                \mycap{After applying rule 1 at node \code{@2}.}
                \label{fig:exec_trace3}
        \end{subfigure}%
        ~ %add desired spacing between images, e. g. ~, \quad, \qquad etc.
         %(or a blank line to force the subfigure onto a new line)
        \begin{subfigure}[b]{0.45\textwidth}
                  \includegraphics[width=\textwidth]{figures/visit/trace4}

                  \mycap{After applying rule 1 and 2 at node \code{@4} and
                  rule 1 at node \code{@3}.}

                  \label{fig:exec_trace4}
          \end{subfigure}
        \mycap{A possible execution trace for the visit
           program. Note that the \code{edge} facts were omitted for simplicity.}
        \label{fig:exec_trace}
\end{figure}

The graph visit has the potential to have plenty of concurrency. Once the visit
starts from the initial node, it expands to other nodes, allowing several nodes
to derive rules concurrently. At some point, the amount of concurrency is
reduced because more and more nodes have been visited. The level of concurrency
depends on the structure of the graph. In the worst case, if the graph is a
chain of nodes then the program becomes effectively sequential.  In the best
case, if the graph is densily connected (each node connects to most nodes in the
graph), then it is possible to run the program on most nodes at the same time.

The goal of introducing higher-level declarative languages is to facilitate
reasoning about the properties of a program. In the case of the visit program,
the most important goal is to prove that, if the graph is connected, then all
the nodes will become \code{visited}, regardless of the order in which we apply
the rules. First, we define a visit graph:

\begin{definition}[Visit graph]
A visit graph is an ordered pair $G = (N, E)$ comprising a set $N$ of nodes together
with a set $E$ of edges. Set $E$ contains pairs $(A, B)$ that correspond to
facts $\bang \mathtt{edge}(A, B)$. For every pair $(A, B) \in E$ there is also a
pair $(B, A) \in E$, representing an undirected edge.
\end{definition}

To prove the correctness property of the program, we first define the main
\emph{invariant} of the program:

\begin{invariant}[Node state]
A node is either \code{visited} or \code{unvisited} but not both.
\end{invariant}

\begin{proof}

From the initial facts we check if all nodes start as \code{unvisited}. This is
clearly true for the previous example. Rule 1 changes a node from
\code{unvisited} to \code{visited}, while rule 2 keeps the node \code{visited},
proving the invariant.

\end{proof}

Invariants are conditions that never change, no matter which rules are derived
during execution. With this invariant, it is now possible to classify nodes of
the graph $G$ according to their state:

\begin{definition}[Node sets] \code{visited} nodes are contained in set $V$,
while \code{unvisited} nodes are in set $U$. From the node state invariant, we
know that $V \cup U = N$ and $V \cap U = \emptyset$.
\end{definition}

We can now prove an important lemma about sets $V$ and $U$:

\begin{invariant}[Visited set]

After each rule derivation, visited set $V$ always increases or stays the same
size. The inverse is true for set $U$.

\end{invariant}
\begin{proof}
Initially, $V = \emptyset$ and $U = N$.
By rule 1, $V$ increases by 1 while $U$ decreases by 1. With rule 2, set
membership remains unchanged.
\end{proof}

In turn, since set membership changes from $U$ to $V$, we now prove the
following:

\begin{lemma}[Edge visits]\label{lemma:language:edge_visits}
The program generates at most one \code{visit} per directed edge and for a node
$a \in N$ that receives a \code{visit} fact, then for all $b \in N$ where $(a,
b) \in E$, exactly one \code{visit} fact is generated at $b$.
\end{lemma}
\begin{proof}
From the visited set invariant, we know that once nodes become members of set $V$,
they no longer return to set $U$, therefore rule 1 applies once per
node. This rule generates a \code{visit} fact per neighbor node.
\end{proof}

In order to prove that all the nodes in the graph are visited, we need to make
sure that the graph is connected.

\begin{definition}[Connected graph]
A connected graph is a graph where every pair of nodes has a path between them.
\end{definition}

Finally, we prove that all nodes will become \code{visited}.

\begin{theorem}[Graph visit correctness]
If graph $G$ is connected, set $V$ will eventually include all nodes in $N$,
while $U = \emptyset$.
\end{theorem}
\begin{proof}
Proof by induction.

\begin{itemize}

   \item Base case: initial fact \code{visit(@1)} adds node \code{@1} to $V$.
      By Lemma~\ref{lemma:language:edge_visits}, a \code{visit} fact is generate
      for all edges of \code{@1}.

   \item Inductive case: assume visited set $V'$ and unvisited set $U'$.
  Since the graph
   is connected, there must be a node $a \in V'$ that is connected to a node $b
   \in U'$. Using the Edge visits lemma, a $\mathtt{visit}(b)$ fact is generated,
   swapping $b$ from $U'$ to $V'$.
\end{itemize}

Eventually, set $V$ will include all nodes in $N$. Otherwise, there would be
unreachable nodes in the graph and that would be a contradiction since the graph
is connected.
\end{proof}



\section{Types and Locality}

Each fact is an association between a \emph{predicate} and a tuple of values. A
predicate is a pair with a name and a tuple of types (the argument types). LM
rules are type-checked using the predicate declarations in the header of the
program. LM has a simple type system that includes the following basic types:
\emph{node}, \emph{int}, \emph{float}, \emph{string}, \emph{bool}. The following
structured types are also supported: \emph{list} $X$, for lists of type $X$;
\emph{struct} $X_1, \ldots, X_n$, for composite values made of $n$ elements; and
\emph{array} $X$, for arrays of type $X$.

LM allows the definition of new type names from simpler types using the declaration
\code{type simple-type new-type} in the header of the program. The type
\code{new-type} can then be used as any other type. Note that LM uses
\emph{structural equivalence} to check if two types are the same, therefore
\code{simple-type} and \code{new-type} are type equivalent.

Type checking LM programs is straightforward due to its simple type system and
mandatory predicate declarations. For each rule, the variables found in the LHS
are mapped to types based on their use on atomic proposition arguments. Some
constraints of the form \code{X = expression} that force an equality between
\code{X} and \code{expression} may actually represent an \emph{assignment} if
\code{X} is not defined by any LHS atomic proposition. In this case, all the
variables in \code{expression} must be typed and \code{X} is assigned the value
of \code{expression} during run time.  Any variable used in the RHS of the rule
must be defined in the LHS, because otherwise derived facts would not be
\emph{grounded}, that is, some arguments would be undefined or uncomputable.
For comprehensions and aggregates, type checking is identical, however, the LHS
of each construct must declare explicitly the variables in scope.

Another important component of type checking is \emph{locality checking}. The
first argument of each atomic proposition in the LHS must use the same variable
in order to enforce locality and allow concurrency. This \emph{home variable} is
always typed as a \emph{node} and represents a node in the program's graph. In
the rule's RHS, other home variables are allowed, as long as they have been
defined in the LHS. For comprehensions and aggregates, the LHS must use the
same home argument as the rule's LHS.

\section{LM Syntax}
Table~\ref{tbl:language:ast} shows the abstract syntax for rules in LM.  A LM
program $Prog$ consists of a list of derivation rules $\Sigma$ and a database
$D$.  Each derivation rule $R$ can be written as $LHS \lolli RHS$ with the
meaning described in Section~\ref{section:language:message}.  Rules without an
LHS are allowed in LM and they are called \emph{initial facts}
(lines~\ref{line:language:dict_axioms1}-\ref{line:language:dict_axioms2} in
Fig.~\ref{code:language:btree_replace}).

\newcommand{\sop}[0]{\Vert}

\begin{table}[h]
\centering
\begin{tabular}{ l l c l }
  Program & $Prog$ & $::=$ & $\Sigma, D$ \\
  List Of Rules & $\Sigma$ & $::=$ & $\cdot \; \sop \; \Sigma, R$\\
  Database & $D$ & $::=$ & $\Gamma; \Delta$ \\
  Rule & $R$ & $::=$ & $LHS \lolli RHS \; \sop \; \forall_{x}. R \; \sop \;
  \selector{S}{y}{LHS}{RHS}$ \\
  LHS Expression & $LHS$ & $::=$ & $L \; \sop \; P \; \sop \; C \; \sop \; LHS,
  LHS \; \sop \; \exists_{x}. LHS \; \sop \; \one$\\
  RHS Expression & $RHS$ & $::=$ & $L \; \sop \; P \; \sop \; RHS, RHS \; \sop
  \; EE \; \sop \; CE \; \sop \; AE \; \sop \; \one$\\
  
  Linear Atomic Proposition & $L$ & $::=$ & $l(\hat{x})$\\
  Persistent Atomic Prop. & $P$ & $::=$ & $\bang p(\hat{x})$\\
  Constraint & $C$ & $::=$ & $c(\hat{x})$ \\
  Selector Operation & $S$ & $::=$ & $\mathtt{min} \; \sop \; \mathtt{max} \;
  \sop \; \mathtt{random}$\\
  
  Exists Expression & $EE$ & $::=$ & $\existsc{\widehat{x}}{SRHS}$ \\
  Comprehension & $CE$ & $::=$ & $\comprehension{\widehat{x}}{SLHS}{SRHS}$ \\

  Aggregate & $AE$ & $::=$ & $\aggregate{A}{y}{\widehat{x}}{SLHS}{SRHS_1}{SRHS_2}$ \\
  Aggregate Operation & $A$ & $::=$ & $\mathtt{min} \; \sop \; \mathtt{max} \; \sop \;
\mathtt{sum} \; \sop \; \mathtt{count} \; \sop \; \mathtt{collect}$ \\
  
  Sub-LHS & $SLHS$ & $::=$ & $L \; \sop \; P \; \sop \; SLHS, SLHS \; \sop \; \exists_{x}. SLHS$\\
  Sub-RHS & $SRHS$ & $::=$ & $L \; \sop \; P \; \sop \; SRHS, SRHS \; \sop \; \one$\\
  
  Known Linear Facts & $\Delta$ & $::=$ & $\cdot \; \sop \; \Delta, l(\hat{t})$ \\
  Known Persistent Facts & $\Gamma$ & $::=$ & $\cdot \; \sop \; \Gamma, \bang p(\hat{t})$ \\
\end{tabular}
\caption{Core abstract syntax of LM.}\label{tbl:language:ast}
\end{table}

The LHS of a rule, $LHS$, may contain linear ($L$) and persistent ($P$)
\emph{atomic propositions} and constraints ($C$). Atomic propositions are
template facts that instantiate variables (from facts in the database) such as
\code{visit(A)} in line~\ref{line:language:visit_second} in
Fig.~\ref{code:language:visit}. Variables can be used again in the LHS for
matching and also in the RHS when instantiating facts.  Constraints are boolean
expressions that must be true in order for the rule to be derived. Constraints
use variables from fact expressions and are built using a small functional
language that includes mathematical operations, boolean operations, external
functions and literal values.

The RHS of a rule, $RHS$, contains linear ($L$) and persistent ($P$)
\emph{atomic propositions} which are uninstantiated facts. The RHS can also have
\emph{exists expressions} ($EE$), \emph{comprehensions} ($CE$) and
\emph{aggregates} ($AE$). All those expressions may use all the variables
instantiated in the rule's LHS and are explained in detail in
Section~\ref{section:language:expressions}.

\subsubsection{Graph visit using abstract syntax}\label{visit:ast}

In order to show how programs are represented using the abstract syntax
presented in Table~\ref{tbl:language:ast}, we present the 2 rules in the graph
visit program shown in Fig.~\ref{code:language:visit}:

\nopagebreak

\begin{align}
\forall_A. \mathtt{visit}(A), \; \mathtt{unvisited}(A) \lolli & \;
\mathtt{visited}(A), \; \{ B; \; \bang \mathtt{edge}(A, B); \;
\mathtt{visit}(B)\}\\
\forall_A. \mathtt{visit}(A), \; \mathtt{visited}(A) \lolli & \;
\mathtt{visited}(A)
\end{align}

Finally, initial facts are represented using an empty LHS:

\nopagebreak

\begin{align}
\one \lolli & \bang \mathtt{edge}(@1, @2) \\
\one \lolli & \bang \mathtt{edge}(@2, @3) \\
\one \lolli & \bang \mathtt{edge}(@1, @4) \\
\one \lolli & \bang \mathtt{edge}(@2, @4) \\
\one \lolli & \bang \mathtt{unvisited}(@1)  \\
\one \lolli & \bang \mathtt{unvisited}(@2) \\
\one \lolli & \bang \mathtt{unvisited}(@3) \\
\one \lolli & \bang \mathtt{unvisited}(@4) \\
\one \lolli & \bang \mathtt{visit}(@1)
\end{align}

\subsection{Predicates And Facts}

Each fact is an association between a \emph{predicate} and a tuple of values. A
predicate is a pair with a name and a tuple of types (the argument types). LM
rules are type-checked using the predicate declarations in the header of the
program. LM has a simple type system that includes the following simples types:
\emph{node}, \emph{int}, \emph{float}, \emph{string}, \emph{bool}. Recursive
types such as \emph{list X} and \emph{pair X; Y} are also allowed.

LM allows definition of new type names from simpler types using the declaration
\code{type simple-type new-type.} in the header of the program. The type
\code{new-type} can then be used as any other type. Note that LM uses
\emph{structural equivalence} to check if two types are the same, therefore
\code{simple-type} and \code{new-type} are type equivalent.

Node subtypes can be introduced using the declaration \code{node subnode.},
allowing the programmer to use nodes with separate predicates.  The
\emph{subnode} type becomes a subtype of \emph{node}, that is, $subnode <:
node$. If initial facts are declared using the form \code{f(A)}, where \code{A} is a
\emph{subnode}, then the axiom is selectively instantiated on nodes with type
\emph{subnode}. Otherwise, if \code{A} was a \emph{node} then \code{f(A)} would
be instantiated in every node.

\subsection{RHS Expressions}\label{section:language:expressions}

\subsubsection{Selectors}

When a rule's LHS is instantiated using facts from the database, facts are picked
non-deterministically. While our system uses an implementation dependent order
for efficiency reasons, sometimes it is important to sort facts by one of the
arguments because linearity imposes commitment during rule derivation. The
abstract syntax for this construct is $\selector{S}{y}{LHS}{RHS}$, where $S$ is
the selection operation and $y$ is the variable in $LHS$ that represents
the value to be selected according to $S$. An example using concrete syntax is
as follows:

\begin{Verbatim}[fontsize=\codesize]
[min => W | !edge(A, B), weight(A, B, W)]
                              -o picked(A, B, W).
\end{Verbatim}

In this case, we order the \code{weight} facts by \code{W} in ascending order
and then try to match them. Other operations available are \code{max} and
\code{random} (to force no pre-defined order at the implementation level).

\subsubsection{Exists Expression}

Exist constructs ($EE$) are based on the linear logic construct of the same name
and are used to create new node addresses. We can use the new address to
instantiate new facts for the new node.  The following example illustrates the
use of the exists construct, where we derive \code{perform-work} at a new node
\code{B}.

\begin{Verbatim}[fontsize=\codesize]
work(A, W) -o exists B. (perform-work(B, W)).
\end{Verbatim}

\subsubsection{Comprehensions}

Sometimes we need to consume a linear fact and then immediately generate several
facts depending on the contents of the database. To solve this particular need,
we created the concept of comprehensions, which are sub-rules that are applied
with all possible combinations of facts from the database. In a comprehension
$\comprehension{\widehat{x}}{SLHS}{SRHS}$, $\widehat{x}$ is a list of variables,
$SLHS$ is the comprehension's left-hand side and $SRHS$ is the right-hand side.
$SLHS$ is used to generate all possible combinations for $SRHS$, according to
the facts in the database. We have already seen an example of comprehensions in
the visit program (Fig.~\ref{code:language:visit}
line~\ref{line:language:visit_comprehension}). Here, we match \code{!edge(A, B)}
using all the combinations available in the database and for each combination we
derive \code{visit(B)}.

\subsubsection{Aggregates}

Another useful feature in logic programs is the ability to reduce several facts
into a single fact. LM features aggregates ($AE$), a special kind of sub-rule
that works somewhat like comprehensions. In the abstract syntax
$\aggregate{A}{y}{\widehat{x}}{SLHS}{SRHS_1}{SRHS_2}$, $A$ is the aggregate
operation, $\widehat{x}$ is the list of variables introduced in $SLHS$, $SRHS_1$
and $SRHS_2$ and $y$ is the variable in $SLHS$ that represents the values to be
aggregated using $A$. Like comprehensions, we use $\widehat{x}$ to try all the
combinations of $SLHS$, but, in addition to deriving $SRHS_1$ for each
combination, we aggregate the values represented by $y$ into a new $y$ variable
that is used to derive $SRHS_2$.

To better understand aggregates, let's consider a database with the following
facts and a rule:

\begin{Verbatim}[fontsize=\codesize]
price(@1, 3).
price(@1, 4).
price(@1, 5).
count-prices(@1).
count-prices(A) -o [sum => P; . | price(A, P) -o 1 -> total(A, P)].
\end{Verbatim}

By applying the rule, we consume \code{count-prices(@1)} and derive the
aggregate which consumes all the \code{price(@1, P)} facts.  These are summed
and \code{total(@1,~12)} is derived. LM provides several aggregate operations,
including the \code{min} (minimum value), \code{max} (maximum value),
\code{sum} (add all numbers), \code{count} (count combinations) and
\code{collect} (collect items into a list).

\section{Operational Semantics}

As said before, the first argument of every predicate must be typed as a
\emph{node}.  For distribution purposes, the LHS of all rules can only use facts
from the same node in order to avoid concurrency issues. However, the facts in
the rule's RHS may refer to other nodes, as long as those nodes are instantiated
in the LHS. We drew some inspiration from the Linda system~\cite{linda}
mentioned early on, where the tuple space contains the data and is used by the
processors to communicate and do computation.  This differs from imperative
languages, since in those languages data and computation are two separate
entities.

The execution is performed at the node level and happens non-deterministically
(i.e., any node can be picked to run). This means that the programmer cannot
expect that facts coming from different nodes will be considered as a whole or
partially since the process is non-deterministic. The operational semantics
promises that rule derivations are performed atomically, therefore if a rule
sends many facts to a node then the target node will receive them all at once.
Under these restrictions, computation can then be parallelized by processing
many nodes concurrently.

Each rule in LM has a defined priority that is inferred from its position in the
source file.  Rules at the beginning of the file have higher priority. At the
node level, we consider all the new facts that have been not consider before to
create a priority queue of \emph{candidate rules}.  The queue of candidate rules
is then applied (by priority) and updated as new facts are derived or consumed.
Section~\ref{sec:implementation:rule_engine} gives details in how our
implementation manages the set of candidate rules.


\section{Applications}
In this section, we present solutions to well-known problems. We start with
straightforward graph-based problems such as bipartitness checking and two
versions of the PageRank program. Next, we present the LM version of the
QuickSort algorithm, which from a first impression may not fit well under the
programming paradigm offered by LM. Informal correctness and termination proofs
are also included to further show that such important properties are relatively
easy to prove for programs written in LM.

\subsection{Bipartiteness Checking}

The problem of checking if a graph is bipartite can be seen as a 2-color graph
coloring problem.  The code for this algorithm is shown in
Fig.~\ref{language:code:bichecking}. All nodes in the graph start as
\texttt{uncolored},
because they do not have a color yet. The axiom \texttt{visit(@1, 1)} is
instantiated at node \texttt{@1} (line 9) in order to color it with color 1.

If a node is \texttt{uncolored} and needs to be marked with a color \texttt{P}
then the rule in lines 11-12 is applied. We consume the \texttt{uncolored} fact
and derive a \texttt{colored(A, P)} to effectively color the node with
\texttt{P}. We also derive \texttt{visit(B, next(P))} in neighbor nodes to color
them with the other color. Line 

The coloring can fail if a node is already colored with a color \texttt{P} and
needs to be colored with a different color (line 15) or if it has already failed
(line 16).

\begin{figure}[h!]
\begin{Verbatim}[numbers=left,fontsize=\scriptsize]
type route edge(node, node).
type linear visit(node, int).
type linear uncolored(node).
type linear colored(node, int).
type linear fail(node).

fun next(int X) : int = if X <> 1 then 1 else 2 end.

visit(@1, 1).

visit(A, P), uncolored(A)
   -o {B | !edge(A, B) | visit(B, next(P))}, colored(A, P).

visit(A, P), colored(A, P) -o colored(A, P).
visit(A, P1), colored(A, P2), P1 <> P2 -o fail(A).
visit(A, P), fail(A) -o fail(A).
\end{Verbatim}
  \caption{Bipartiteness Checking program.}
  \label{language:code:bichecking}
\end{figure}

\subsubsection{Proof Of Correctness}

In order to show that the code in Fig.~\ref{language:code:bichecking} works as
intended, we first setup some invariants that hold throughout the execution of
the program. Assume that the set of nodes in the graph is represented as $N$.

\begin{invariant}[Node state]
Set of nodes $N$ is partitioned into 4 different states that represent the 4
possible states that a node can be in, namely:

\begin{itemize}
   \item $U$ (\texttt{uncolored} nodes)
   \item $F$ (\texttt{fail} nodes)
   \item $C_{true}$ (\texttt{colored(A, 1)} nodes)
   \item $C_{false}$ (\texttt{colored(A, 2)} nodes)
\end{itemize}
\end{invariant}
\begin{proof}
Initially, all nodes start in set $U$. All the 4 rules of the programs either
keep the node in the same set or exchange the node with another set.
\end{proof}

A bipartite graph is one where in every edge $a \leftrightarrow b$, there is a
valid assignment that makes $a$ member of set $C_{true}$ or $C_{false}$ and node
$b$ member of either $C_{false}$ or $C_{true}$ respectively.

\begin{lemma}[Bipartiteness
   Convergence]\label{language:lemma:bipartite_convergence}
   We now reason from the application of the program rules. After each
   application of an inference rule, one of the following will happen:

   \begin{itemize}
      \item Set $U$ will decrease and set $C_{true}$ or $C_{false}$ will
         increase, with a potential increase in the number of \texttt{visit}
         facts.
      \item Set $C_{true}$ or $C_{false}$ will stay the same, while the number
         of \texttt{visit} facts will be reduced.

      \item Set $C_{true}$ or $C_{false}$ will decrease and set $F$ will
         increase, while the number of \texttt{visit} facts will be reduced.

      \item Set $F$ will stay the same, while the number of \texttt{visit} facts
         decreases.
   \end{itemize}

\end{lemma}
\begin{proof}
Directly from the rules.
\end{proof}

From this invariant, it can be inferred that set $U$ never increases in size
and in a node transition from \texttt{uncolored} to \texttt{colored}, the
database may increase in size. For every other rule application, the database of
facts always decreases. This also means that the program will eventually
terminate, since it is limited by the number of \texttt{visit} facts that can be
generated.

\begin{theorem}[Bipartiteness Correctness]
If the graph is connected and bipartite then the nodes will be partitioned into
sets $C_{true}$ and $C_{false}$, while sets $F$ and $U$ are empty.
\end{theorem}
\begin{proof}
   By induction, we prove that uncolored nodes become part of either $C_{true}$
   and $C_{false}$ and, if there is an edge between nodes in the two sets then
   they have different colors.

   In the base case, we start with empty sets but node \texttt{@1} is made
   member of $C_{true}$. Rule 1 sends \texttt{visit} facts to the neighbors of
   \texttt{@1}, forcing them to be members of $C_{false}$.

   In the inductive case, we have sets $C'_{true}$ and $C'_{false}$ with some
   nodes already colored. From Lemma~\ref{language:lemma:bipartite_convergence},
   we know that $U$ always decreases. Since the graph is bipartite, events 3 and
   4 never happen since there is a possible partitioning of nodes. With event 1,
   we have set $C_{true} = C'_{true}, n$, (or $C_{false}$) where $n$ is the
   node and with event 2, the sets remain the same. Since the graph is
   connected, every node will be colored, therefore event 1 will happen for
   every node of the graph.
\end{proof}

\subsection{PageRank}

PageRank~\cite{Page:2001:MNR} is a well known graph algorithm that is used to
compute the relative relevance of web pages.  The code for a synchronous version
of the algorithm is shown in Fig.~\ref{code:pagerank}.  As the name indicates,
the pagerank is computed for a certain number of iterations. The initial
pagerank is the same for every page and is initialized in the first rule (lines
15-16) along with an accumulator.

The second rule of the program (lines 17-19) propagates a newly computed
pagerank value to all neighbors. The fact \texttt{neighbor-pagerank} informs the
neighbor node about the pagerank value of node \texttt{A} for iteration
\texttt{Iter + 1}. For every iteration, each node will accumulate the all the
\texttt{neighbor-pagerank} facts into the \texttt{accumulator} fact (lines
27-28). When all inbound neighbor pagerank values are accumulated, the third
rule (lines 21-25) is fired and a pagerank value is derived for iteration
\texttt{Iter}.

\begin{figure}[h!]
\begin{Verbatim}[numbers=left,fontsize=\scriptsize]
type outbound(node, node, float).
type linear pagerank(node, float, int).
type numInbound(node, int).
type linear accumulator(node, float Acc, int Left, int Iteration).
type linear neighbor-pagerank(node, node Neighbor, float Rank, int Iteration).
type linear start(node).

const damping = 0.85. // probability of user following a link in the current page.
const iterations = str2int(@arg1). // iterations to compute.
const pages = @world. // number of pages in the graph.

start(A).

start(A), !numInbound(A, T)
   -o accumulator(A, 0.0, T, 1), pagerank(A, 1.0 / float(pages), 0).

pagerank(A, V, Iter), // propagate new pagerank value
Iter < iterations
   -o {B, W | !outbound(A, B, W) | neighbor-pagerank(B, A, V * W, Iter + 1)}.

accumulator(A, Acc, 0, Iter), // new pagerank
!numInbound(A, T),
V = damping + (1.0 - damping) * Acc,
Iter <= iterations
   -o pagerank(A, V, Iter), accumulator(A, 0.0, T, Iter + 1).
	
neighbor-pagerank(A, B, V, Iter), accumulator(A, Acc, T, Iter)
   -o accumulator(A, Acc + V, T - 1, Iter).
\end{Verbatim}
\caption{Synchronous PageRank program.}
\label{language:code:pagerank}
\end{figure}

\subsubsection{Asynchronous PageRank}

We also have an asynchronous version of the algorithm that trades correctness
with convergence speed since it does not synchronize between iterations.
Figure~\ref{language:code:async_pagerank} shows the LM code for this particular
version, where two major differences can be observed: (1) there is a linear fact
\texttt{neighbor-pagerank} containing the most up-to-date pagerank value of a
neighbor node; (2) a new \texttt{update} fact that forces the node to re-compute
its pagerank by processing the currently available \texttt{neighbor-pagerank}
facts. Rules in lines 13-21 update the \texttt{neighbor-pagerank} values, while
rule in lines 23-29 asynchronously update the current pagerank value. This last
rule derives multiple \texttt{new-neighbor-rank} that is used to inform the
neighbor about the new pagerank value.

\begin{figure}[h!]
\begin{Verbatim}[numbers=left,fontsize=\scriptsize]
type outbound(node, node, float).
type linear pagerank(node, float, int).
type numInbound(node, int).
type linear neighbor-pagerank(node, node Neighbor, float Rank, int Iteration).
type linear new-neighbor-rank(node, node Neighbor, float Rank, int Iteration).
type linear update(A).
type linear sum-ranks(node, float).

pagerank(A, 1.0 / float(pages), 0).
update(A).
neighbor-pagerank(A, B, 1.0 / float(pages), 0). // pagerank of B is ...

// save incoming pagerank value.
new-neighbor-rank(A, B, New, Iteration),
neighbor-pagerank(A, B, Old, OldIteration),
Iteration > OldIteration
   -o neighbor-pagerank(A, B, New, Iteration).
new-neighbor-rank(A, B, New, Iteration),
neighbor-pagerank(A, B, Old, OldIteration),
Iteration <= OldIteration
   -o neighbor-pagerank(A, B, Old, OldIteration).

sum-ranks(A, Acc),
NewRank = damping/float(pages) + (1.0 - damping) * Acc,
pagerank(A, OldRank, Iteration)
      -o pagerank(A, NewRank, Iteration + 1),
         {B, W, Delta, Iter | !outbound(A, B, W), Delta = fabs(NewRank -
               OldRank) * W | new-neighbor-rank(B, A, NewRank, Iteration + 1),
               if Delta > bound then update(B) end}.

update(A), update(A) -o update(A).

update(A),
!numInbound(A, T)
   -o [sum => V | B, W, Val, Iter | neighbor-pagerank(A, B, Val, Iter)
         V = Val/float(T) | neighbor-pagerank(A, B, Val, Iter) | sum-ranks(A, V)].
\end{Verbatim}
\caption{Asynchronous PageRank program.}
\label{language:code:async_pagerank}
\end{figure}

\subsubsection{Proof Of Correctness}

To build the proof of correctness, we must again prove several program
invariants. This will help us prove that this partifcular program is similar to
a computation on a nonnegative matrix of of unit spectral radius, which has been
proven that it converges~\cite{DBLP:journals/corr/abs-cs-0606047,
Lubachevsky:1986:CAA:4904.4801}.

\begin{invariant}[Page Invariant]
Each page/node has a single \texttt{pagerank(A, Value, Iteration)} and:
\begin{itemize}
   \item For each outbound link, a single \texttt{\bang outbound(A, B, W)}.
   \item For each inbound link, a single \texttt{neighbor-pagerank(A, B, V, Iter)}.
   \item For each \texttt{\bang outbound(A, B, W)}, a \texttt{neighbor-pagerank(A,
      B, V, Iter}.
\end{itemize}
\end{invariant}

\begin{proof}

All axioms validate the 3 conditions of the variant. Note that the third
condition is also validated by the axioms, although not all axioms are shown in
the code.

In relation to rule application:

\begin{itemize}
   \item Rule 1: inbound link re-derived.
   \item Rule 2: inbound link re-derived.
   \item Rule 3: \texttt{pagerank/2} re-derived.
   \item Rule 4: Nothing happens.
   \item Rule 5: inbound links re-derived in the comprehension.
\end{itemize}
\end{proof}

\begin{lemma}[Neighbor rank lemma]
Given a fact \texttt{neighbor-pagerank(A, B, V, Iter)} and a set of facts
\texttt{new-neighbor-rank(A, B, New, Iter2)}, we end up with a single
\texttt{neighbor-pagerank(A, B, V', Iter')}, where \texttt{Iter} is the greater of
\texttt{Iter} or all of \texttt{Iter2'}.
\end{lemma}
\begin{proof}
By induction on the number of \texttt{new-neighbor-rank} facts.

Base case: \texttt{neighbor-pagerank} remains.

Inductive case: given one \texttt{new-neighbor-rank} fact:

\begin{itemize}
   \item Rule 1: the new iteration is older and thus \texttt{neighbor-pagerank}
   is replaced. By applying induction, we know that we will select either the
   new best iteration or a better iteration from the remaining set of
   \texttt{new-neighbor-rank} facts.
   \item Rule 2: the new iteration is not older and we keep the old
   \texttt{neighbor-pagerank} fact. By induction, we select the best from either
   the current iteration or some other (from the set).
\end{itemize}
\end{proof}

\begin{lemma}[Update lemma]
Given at least 1 \texttt{update/1} fact, rule 7 will run.
\end{lemma}
\begin{proof}
By induction on the number of \texttt{update} facts.

Base case: rule 5 will run.

Inductive case: rule 4 will run first because it has a higher priority, reducing
the number of \texttt{update} facts by one. By induction, we know that by
using the remaining \texttt{update} facts, rule 7 will run.
\end{proof}

\begin{lemma}[Pagerank update lemma]
(1) Given at least one \texttt{update} fact, the \texttt{pagerank(A, $V_{I}$,
I)} fact will be updated to become \texttt{pagerank(A, $V_{I + 1}$, I +
1)}, where \texttt{$V_{I + 1} = damping / P + (1.0 - damping)\sum_{B,
I} (W_{B} \times  N_{I,B})$}.

where $W_B = 1.0/T$ ($T$ from \texttt{\bang numInbound(A, $T$)})
and $N_{I,B}$ from \texttt{neighbor-pagerank(A, B, $N_{I, B}$, $I$)}.

(2) For all \texttt{B} outbound nodes (represented using \texttt{\bang outbound(A, B,
W)}, a \texttt{new-neighbor-rank(B, A, $V_{I+1}$, $I + 1$)} is generated.

(3) For all \texttt{B} outbound nodes (represented using \texttt{\bang outbound(A, B,
W)}), a \texttt{update(B)} is generated if 
$fabs(V_{I + 1} - V_{I}) \times W > bound$.
\end{lemma}
\begin{proof}
Using the Update lemma, rule 5 will necessarily run.

It derives \texttt{sum-ranks(A, $\sum_{B, I} (W_B \times N_{I,B})$)} and
fulfills (3).

\texttt{sum-ranks/2} will necessarily fire rule 6,
computing $V_{I+1}$ and updating \texttt{pagerank}. (2) and (3) are fulfilled
through the comprehension of rule 6.
\end{proof}

\begin{invariant}[New neighbor rank equality]
All \texttt{new-neighbor-rank(A, B, V, I)} facts are generated from a corresponding
\texttt{pagerank(B, V, I)} fact, therefore the iteration of any
\texttt{new-neighbor-rank} is at least the same or less than the iteration of
the current pagerank.
\end{invariant}
\begin{proof}
No axioms to prove.

\begin{itemize}
   \item Rule 3: true, new fact is generated.
   \item Rule 6: the fact is kept.
\end{itemize}
\end{proof}

\begin{invariant}[Neighbor rank equality]
All \texttt{neighbor-pagerank(A, B, V, I)} facts have one corresponding
\texttt{pagerank(B, V, I)} fact and the iteration of the \texttt{neighbor-pagerank}
is the same or less than the current iteration of the corresponding
\texttt{pagerank}.
\end{invariant}
\begin{proof}
By analyzing axioms and rules.

Axioms: true.

Rule cases:

\begin{itemize}
   \item Rule 1: uses \texttt{new-neighbor-rank} fact (use new neighbor rank
         equality invariant).
   \item Rule 2: same fact is re-derived.
\end{itemize}
\end{proof}

\begin{theorem}[Pagerank convergence]
The program will compute the pagerank of all nodes that is within \texttt{bound} error
of an asynchronous pagerank computation.
\end{theorem}
\begin{proof}

Using the program axioms, we start with the same pagerank value for all nodes.
The \texttt{\bang outbound(A, B, W)} fact forms a $n \times n$ square matrix (number
of nodes) and is the so-called "Google Matrix".  All the initial pagerank values
can be seen as a vector that adds up to $1$.

The pagerank computation from the "Pagerank update lemma" computes $V_{I + 1} =
damping / P + (1.0 - damping)\sum_{B, I'} (W_{B} \times N_{I',B})$, where $I' <=
I$
(from Neighbor rank equality invariant).

Consider that each node contains a column $G_i$ of the Google matrix. The
pagerank computation can then be represented as: \newline


$V_{I + 1} = G_i fix([N_{I_1, B_1}, ..., N_{I_p, B_p}])$ \hfill (1) \\


Where $p$ is the number of inbound links and $N_{I_j, B_j}$ is the value of
the \texttt{neighbor-pagerank(A, $B_j$, $N_{I_j, B_j}$, $I_j$)}. The $fix()$
function takes the neighbor vector and expands it with zeros corresponding to
nodes that are not inbound links.

From~\cite{DBLP:journals/corr/abs-cs-0606047, Lubachevsky:1986:CAA:4904.4801} we
know that equation (1) will improve (converge) the pagerank value, given that some new
neighbor pagerank values are sent to node $i$ and by the fact that $G_i$ is a
nonnegative matrix of unit spectral radius. Let's use induction by assuming that there
is at least one \texttt{update/1} fact that
schedules a node to improve its pagerank. We want to prove that such fact will
not only improve the node's pagerank but also the pagerank vector.
If the pagerank vector is now close enough (within \texttt{bound}), then the
program will terminate.

\begin{itemize}
   \item Base case: Since we have an \texttt{update} fact as an axiom, rule 7
   will compute a new pagerank (Pagerank update lemma) for all nodes that is
   improved (from equation (1)). From these updates, a new \texttt{update}
   fact is generated that correspond to nodes that have inbound links from the
   source node and need to update their pagerank. These \texttt{update} facts
   may not be generated if the pagerank vector is close enough to its real
   value.

   \item The induction hypothesis tells us that there is at least one node that
   has an \texttt{update} fact. From pagerank update lemma, this
   generates \texttt{new-neighbor-rank} facts if the new value differs
   significantly from the older value. When this happens and using the ``Neighbor
   rank lemma'', the target node will update its \texttt{neighbor-pagerank} fact
   with the newest iteration and then execute a valid pagerank computation that
   brings the pagerank vector close to its solution.

\end{itemize}

\end{proof}


\subsection{Quick-Sort}

The quick-sort algorithm is a divide and conquer sorting algorithm that works by splitting
a list of items into two sublists and then recursively sorting the two sublists.
To split the list, we pick a pivot element and put the items that are smaller than the pivot
into the first sublist and items greater than the pivot into the second list.

The quick-sort algorithm is interesting because it does not map immediately to the graph-based
model of LM. Our approach considers that the program starts with a single node where
the initial list is located. Then we split the list as usual and create two nodes
that will recursively sort the sublists. Interestingly, this will create a tree
that will look similar to a call tree in a functional language.

Fig.~\ref{code:quicksort} presents the code for the quick-sort algorithm in LM.
For each sublist to sort, we start with a \texttt{down} fact that must be (eventually)
transformed into an \texttt{up} fact, where the sublist in the \texttt{up} fact is sorted.
In line 11 we start with the initial list at node \texttt{@0}. Lines 13-16 will immediately
sort the list when the number of items is very small. Otherwise, we apply the rule in line 17.
\texttt{buildpivot} will first split the list using the pivot \texttt{X} using rules in
lines 23-26. When there is nothing more to split, we apply the rule in lines 19-21
that uses an exist construct to create nodes \texttt{B} and \texttt{C}. The sublists
are then sent to these nodes using \texttt{down} facts. Note, however, that we also
derive \texttt{back} facts, that will be used to send the sorted list back using the rule
in line 40.

When the sublists are finally sorted, we get two \texttt{sorted} facts that will match
against \texttt{waitpivot} in the rule located in lines 28-31. The sorted sublists
are appended and then an \texttt{up} fact is finally derived (line 37).

\begin{figure}[h!]
\small\begin{Verbatim}[numbers=left]
type route back(node, node).
type linear down(node, list int).
type linear up(node, list int).
type linear sorted(node, node, list int).
type linear buildpivot(node, list int, int, list int, list int).
type linear waitpivot(node, node, node, int).
type linear append(node, list int, list int).
type linear reverse(node, list int, list int, list int).
type linear reverse2(node, list int, list int).

down(@0, tosort).

down(A, []) -o up(A, []).
down(A, [X]) -o up(A, [X]).
down(A, [X, Y]), X < Y -o up(A, [X, Y]).
down(A, [X, Y]), X >= Y -o up(A, [Y, X]).
down(A, [X | L]) -o buildpivot(A, L, X, [], []).

buildpivot(A, [], X, Smaller, Greater)
   -o exists B, C. (back(B, A), down(B, Smaller),
            back(C, A), down(C, Greater), waitpivot(A, B, C, X)).

buildpivot(A, [Y | L], X, Smaller, Greater), Y <= X
   -o buildpivot(A, L, X, [Y | Smaller], Greater).
buildpivot(A, [Y | L], X, Smaller, Greater), Y > X
   -o buildpivot(A, L, X, Smaller, [Y | Greater]).
   
waitpivot(A, NodeSmaller, NodeGreater, Pivot),
sorted(A, NodeSmaller, Smaller),
sorted(A, NodeGreater, Greater)
   -o append(A, Smaller, [Pivot | Greater]).

append(A, L1, L2) -o reverse(A, L1, L2, []).

reverse(A, [], L2, L3) -o reverse2(A, L3, L2).
reverse(A, [X | L], L2, L3) -o reverse(A, L, L2, [X | L3]).
reverse2(A, [], Result) -o up(A, Result).
reverse2(A, [X | L1], L2) -o reverse2(A, L1, [X | L2]).

up(A, L), back(A, B) -o sorted(B, A, L).
\end{Verbatim}
  \caption{Quick-Sort program.}
  \label{code:quicksort}
\end{figure}
\normalsize




\section{Related Work}\label{section:language:related}
\subsection{Graph-Based Programming Models}

Many programming systems have designed for writing graph-based programs.  Some
good examples are the Dryad, Pregel, GraphLab, Ligra, Grace, Galois and
SociaLite systems.

The Dryad system~\cite{Isard:2007:DDD:1272996.1273005} combines computational
vertices with communication channels (edges) to form a data-flow graph. The
program is scheduled to run on multiple processors or cores and data is
partitioned during runtime. Routines that run on computational vertices are
sequential, with no synchronization. Dryad is better suited for a wider range of
problem domains than LM, however Dryad is not a language but a framework upon
which languages such as LM could be built.

The Pregel system~\cite{Malewicz:2010:PSL:1807167.1807184} is also graph based,
although programs have a more strict structure. They must be represented as a
sequence of iterations where each iteration is composed of computation and
message passing. Pregel is especially suited for large graphs since it aims to
scale to large architectures. LM does not impose iteration restrictions on
programs, however it is not as well suited for large graphs as Pregel.

GraphLab~\cite{GraphLab2010} is a C++ framework for developing parallel machine
learning algorithms. While Pregel uses message passing, GraphLab allows nodes to
have read/write access to different scopes through different concurrent access
models in order to balance performance and data consistency. While some programs
only require access to the local node's data, others may need to update edge
information. Each consistency model will provide different guarantees that are
better adapted to some algorithms. GraphLab also provides different schedulers
that dictate the order in which nodes are computed. The downside of GraphLab is
that GraphLab has a steep learning curve and programs must be written in C++,
requiring a lot of boilerplate code. LM programs tend to be smaller and easier to
reason about since they are written at a higher abstraction level.

Ligra~\cite{Shun:2013:LLG:2517327.2442530} is a lightweight framework for large
scale graph processing on a single multicore machine. Ligra exploits the fact
that most huge graph datasets available today can be made to fit in the main
memory of commodity servers. Ligra is a simple framework that exposes two main
interfaces: \texttt{EdgeMap} and \texttt{VertexMap}. The former applies a
function to a subset of edges of the graph, while the latter applies a function
to a subset of vertices. The functions passed as arguments are applied to either
a single edge or a single vertex and the user must ensure that the function can
be executed in parallel. The framework allows the use of
\emph{Compare-and-Swap}~(CAS) instructions when implementing functions in order
to avoid race conditions.

Grace~\cite{wang:asynchronous} is another graph-based framework for multicore
machines. Unlike Ligra, Grace programs are implemented from the point of view of
a vertex. Each vertex and edge can be customized with different data depending
on the target application. By default, programs are executed iteratively: for
each iteration the vertex program reads incoming edge messages, performs
computation and sends messages to the outbound edges. Since iterative programs
require synchronization after each iteration, Grace allows the user to relax
these constraints and implement customizable execution policies by implementing
code for describing which vertices are allowed to run and in which order. The
order is dictated by assigning a \emph{scheduling priority value}.

Galois~\cite{Pingali:2011:TPA:1993316.1993501} is a parallel programming model
with irregular applications based on graphs, trees and sets. A Galois parallel
algorithm is viewed as a parallel application of an \emph{operator} over an
irregular data structure which generate \emph{activities} on the data structure.
Such operator may, for instance, be applied to the node of the graph in order to
change its data or change the structure of its neighborhood, allowing for data
structure changes. In Galois, \emph{active elements} are nodes of the graph
where computation needs to be performed. Each operator performed on an active
element needs to acquire the \emph{neighborhood}, which are the nodes connected
to the active element that are involved in the operator's computation.
Furthermore, operators are selected for execution according to some specific
\emph{ordering}. From the point of view of the programmer, the active elements
are represented in a work-list, while operators can be implemented on top of
work-list's iterators. Galois supports speculative execution by allowing
operator rollback when an operator requires a node that is held by another
operator.

Ligra, Grace and Galois are not programming language but frameworks built on top
of other programming languages such as C and Java. Naturally, programs written
in these frameworks need to take into account several implementation details
such as \emph{compare-and-set} instructions, work-lists, and scheduling order.
LM programs are more abstract since reasoning is performed around logical rules
and logical facts which are much closer to the problem domain of graph-based
programs.

SociaLite~\cite{Seo:2013:DSD:2556549.2556572} is a Datalog-based language for
writing algorithms for social-network analysis. SociaLite takes Datalog programs
and compiles them to efficient distributed Java code. The language places some
restrictions on rules in order to make distribution possible, however, the
programmer is free to tell the system how to shard database's facts. Like LM,
these restrictions deal with the first argument of each fact, however, the first
argument is not related to a graph abstraction but is instead related to fact
distribution, allowing programmers to optimize how messages are sent between
machines.

\subsection{Sensor Network Programming Languages}

Many programming languages have been designed to help developers write programs
that run on sensor networks, which are networks that can be represented as
graphs.  Programming languages such as
Hood~\cite{Whitehouse:2004:HNA:990064.990079},
Tinydb~\cite{Madden:2005:TAQ:1061318.1061322} or
Regiment~\cite{Newton:2007:RMS:1236360.1236422} have been proposed, providing
support for data collection and aggregation over the network. These systems
assume that the network remains static and nodes stay in place.

Other languages such as Pleiades~\cite{Kothari:2007:REP:1250734.1250757},
LDP~\cite{4543691} or Proto~\cite{Beal:2006:IEE:1137236.1137354} go beyond
static networks and support dynamic reconfiguration. In Pleiades, the
programmer writes an application from the point of view of the whole
sensor network and the compiler transforms it into code that can be run on
each individual node.  LDP is a language derived from a method for
distributed debugging, that allows it to efficiently detect conditions on
variably-sized groups of nodes. It is based on the tick model, generating
a new set of condition matchers throughout the ensemble on each tick.
Like Pleiades and LDP, Proto also compiles global programs into locally
executed code.

Finally, the original Meld~\cite{ashley-rollman-iclp09} is also a language
designed for dynamic networks, namely, ensembles of robots. As we have seen
before, Meld is a logic programming language where programs are sets of logical
rules that infer over the state of the ensemble. Meld supports action facts and
sensing facts, which allow the robots to act on the world or sense the world,
respectively. Like Pleiades programs, Meld programs are also written from the
point of the view of the whole ensemble.

\subsection{Constraint Handling Rules}

Since LM is a bottom-up linear logic programming language, it also shares
similarities with Constraint Handling Rules
(CHR)~\cite{Betz:2005kx,Betz:2013:LBA:2422085.2422086}.  CHR is a concurrent
committed-choice constraint language used to write constraint solvers. A CHR
program is a set of rules and a set of constraints. Constraints can be consumed
or generated during the application of rules. Unlike LM, in CHR there is no
concept of rule priorities, but there is an extension to CHR that supports
them~\cite{DeKoninck:2007:URP:1273920.1273924}. Finally, there is also a CHR
extension that adds persistent constraints and it has been proven to be sound
and complete~\cite{DBLP:journals/corr/abs-1007-3829} in relation to the standard
formulation of CHR.

\subsection{Graph Transformation Systems}

Graph Transformation Systems (GTS)~\cite{Ehrig:2004vn}, commonly used to model
distributed systems, perform rewriting of graphs through a set of graph
productions. GTS also introduces concepts of concurrency, where it may be
possible to apply several transformations at the same time. In principle, it
should be possible to model LM programs as graph transformations: we directly
map the LM graph of nodes to GTS's initial graph and consider logical facts as
nodes that are connected to LM's nodes. Each LM rule is then a graph production
that manipulates the node's neighbors (the database) or sends new facts to other
nodes. On the other hand, it is also possible to embed GTS inside
CHR~\cite{Raiser:2011:AGT:1972935.1972938}.

\section{Chapter Summary}

In this chapter, we gave an overview of the LM language, including its syntax
and operational semantics. We also explained how to write programs using all the
facilities provided by LM, including linear facts, comprehensions, and
aggregates. We also explained how to informally prove the correctness of several
LM programs.


\chapter{Logical Foundations}
This chapter provides an overview of the proof theoretic basis behind LM
and the dynamic semantics of the language. First, we will present the subset of
linear logic from which LM is built on. Second, we present the high level
dynamic semantics - how rules are evaluated and node communication - followed by
the low level dynamics, a close representation of how the virtual machine runs.
Finally, we prove that the low level dynamic semantics are sound in relation to
the high level dynamic semantics.

\section{Linear Logic}

Logic, as \emph{classically} understood, treats true propositions as
\emph{persistent truth}. When a persistent proposition is needed to prove other
propositions, it can be reused as many times as we wish because it is true
indefinitely. This is also true in the \emph{constructive} or
\emph{intuitionistic} school of logic.  Linear logic is a \emph{substructural
logic} (lacks weakening and contraction) developed by
Girard~\cite{Girard95logic:its} extends \emph{persistent logic} with linear
propositions which can be understood as ephemeral resources that can be used
only once to prove other propositions.  Naturally, linear logic is well
suited for modeling computing systems that deal with state, of which LM is
one of them.  Traditional forward-chaining logic programming languages like
Datalog only use persistent logic, however many ad-hoc
extensions~\cite{Liu98extendingdatalog,Ludascher95alogical} have been devised
in to support state updates, but most are extra-logical which makes it harder
to reason about programs. LM uses linear logic as its foundation, therefore
state updates are natural to the language.

In linear logic, truth is treated as a resource that is consumed once used. For
instance, in the graph visit program in Fig.~\ref{code:visit}, the
\texttt{unvisited(A)} and \texttt{visit(A)} linear facts are consumed in order
to prove \texttt{visit(A)} and the comprehension. If those facts were
persistent, then the rule would make no sense, because the node would be
\texttt{visited} and \texttt{unvisited} at the same time!

\subsection{Sequent calculus of \fragment}

We now describe the linear logic fragment used as a basis for LM.  Note that in
this thesis we follow the intuitionistic approach and use the sequent
calculus~\cite{gen35} to specify the logic. Our initial sequent is written as
$\Psi; \seqx{\Gamma}{\Delta}{C}$ and can be read as "assuming persistent
resources $\Psi$ and linear resources $\Delta$ then $C$ is true".  More
specifically, $\Psi$ is the typing context, $\Gamma$ is a multi-set of
persistent resources, $\Delta$ is a multi-set of linear resources while $C$ is
the proposition we want to prove.

We first have the \emph{simultaneous conjunction} $A \otimes B$ that packages
linear resources together. In the right rule, $A \otimes B$ is true if both $A$
and $B$ are true, and, in the left rule, it is possible to split $A \otimes B$
apart.

\[
\infer[\otimes R]
{\Psi ; \seqx{\Gamma}{\Delta, \Delta'}{A \otimes B}}
{\Psi ; \seqx{\Gamma}{\Delta}{A} & \Psi ; \seqx{\Gamma}{\Delta}{B}}
\tab
\infer[\otimes L]
{\Psi ;\seqx{\Gamma}{\Delta, A \otimes B}{C}}
{\Psi ; \seqx{\Gamma}{\Delta, A, B}{C}}
\]

Next, we have the \emph{additive conjunction} $A \with B$ that allows us to
select between $A$ or $B$. In the right rule we must prove $A$ and $B$ using
the same resources, while in the left rule, we can select one of the
resources.

\[
\infer[\with R]
{\Psi ; \seqx{\Gamma}{\Delta}{A \with B}}
{\Psi ; \seqx{\Gamma}{\Delta}{A} & \Psi ; \seqx{\Gamma}{\Delta}{B}}
\]

\[
\infer[\with L_1]
{\Psi ; \seqx{\Gamma}{\Delta, A \with B}{C}}
{\Psi ; \seqx{\Gamma}{\Delta, A}{C}}
\tab
\infer[\with L_2]
{\Psi ; \seqx{\Gamma}{\Delta, A \with B}{C}}
{\Psi ; \seqx{\Gamma}{\Delta, B}{C}}
\]


To express inference, we introduce the \emph{linear implication} connective
written as $A \lolli B$. For the right rule, we prove $A \lolli B$ by assuming
$A$ and then proving $B$, while in the left rule, we obtain $B$ by using some
linear resources to prove $A$.

\[
\infer[\lolli R]
{\Psi ; \seqx{\Gamma}{\Delta}{A \lolli B}}
{\Psi ; \seqx{\Gamma}{\Delta, A}{B}}
\tab
\infer[\lolli L]
{\seqx{\Gamma}{\Delta, \Delta', A \lolli B}{C}}
{\Psi ; \seqx{\Gamma}{\Delta}{A} &
   \Psi ; \seqx{\Gamma}{\Delta', B}{C}}
\]

Next, we introduce persistent resources written as $\bang A$. For the right
rule, we prove $\bang A$ by proving it without any linear resources. Likewise,
to use a persistent resource, we simply drop the $
\bang$. There is also a $\m{copy}$ rule that moves persistent resources from
$\Gamma$ to $\Delta$. Remember that $\Gamma$ contains persistent resources.

\[
\infer[\bang R]
{\Psi ; \seqx{\Gamma}{\cdot}{\bang A}}
{\Psi ; \seqx{\Gamma}{\cdot}{A}}
\tab
\infer[\bang L]
{\Psi ; \seqx{\Gamma}{\Delta, \bang A}{C}}
{\Psi ; \seqx{\Gamma, A}{\Delta}{C}}
\tab
\infer[\m{copy}]
{\Psi ; \seqx{\Gamma, A}{\Delta}{C}}
{\Psi ; \seqx{\Gamma, A}{\Delta, A}{C}}
\]

Another useful connective is the \emph{multiplicative unit} of the $\otimes$
connective. It is written as $\one$ and is best understood as something that
does not need any resource to be proven.

\[
\infer[\one R]
{\Psi ; \seqx{\Gamma}{\cdot}{\one}}
{}
\tab
\infer[\one L]
{\Psi ; \seqx{\Gamma}{\Delta, \one}{C}}
{\Psi ; \seqx{\Gamma}{\Delta}{C}}
\]

Next, we introduce the \emph{quantification} connectives, namely \emph{universal
quantification} $\forall_{n:\tau}. A$ and \emph{existencial quantification}
$\exists_{n:\tau}. A$. These connectives use the typing context $\Psi$ because
they can introduce and read terms from the context. The right and left duals of
those two connectives are dual.

\[
\infer[\forall R]
{\Psi ; \seqx{\Gamma}{\Delta}{\forall_{n:\tau}. A}}
{\Psi, m:\tau ; \seqx{\Gamma}{\Delta}{A\{m/n\}}}
\tab
\infer[\forall L]
{\Psi ; \seqx{\Gamma}{\Delta, \forall_{n:\tau}. A}{C}}
{\Psi \vdash M : \tau & \Psi ; \seqx{\Gamma}{\Delta, A\{M/n\}}{C}}
\]

\[
\infer[\exists R]
{\Psi ; \seqx{\Gamma}{\Delta}{\exists_{n: \tau}. A}}
{\Psi \vdash M : \tau &
   \Psi ; \seqx{\Gamma}{\Delta}{A \{M/n\}}}
\tab
\infer[\exists L]
{\Psi ; \seqx{\Gamma}{\Delta, \exists_{n:\tau}. A}{C}}
{\Psi, m:\tau ; \seqx{\Gamma}{\Delta, A\{m/n\}}{C}}
\]

Finally, we complete the linear logic system with the \emph{cut rules} and the
\emph{identity rule}:

\[
\infer[cut_A]
{\Psi ; \seqx{\Gamma}{\Delta, \Delta'}{C}}
{\Psi ; \seqx{\Gamma}{\Delta}{A} & \Psi ; \seqx{\Gamma}{\Delta', A}{C}}
\tab
\infer[cut\bang_A]
{\Psi ; \seqx{\Gamma}{\Delta}{C}}
{\Psi ; \seqx{\Gamma}{\cdot}{A} & \Psi ; \seqx{\Gamma, A}{\Delta}{C}}
\]

\[
\infer[id_A]
{\Psi ; \seqx{\Gamma}{A}{A}}
{}
\]


\subsection{From the \fragment sequent calculus to LM}


\begin{table*}
\begin{center}
\resizebox{16cm}{!}{
    \begin{tabular}{ | l | l || l | l | l |}
    \hline
    Connective                   & Description                                      & LM Syntax                                  & LM Place     & LM Example                                  \\ \hline \hline
    $\emph{fact}(\hat{x})$       & Linear facts.                                    & $fact(\hat{x})$                               & Body or Head    & \texttt{path(A, P)}                            \\ \hline
    $\bang \emph{fact}(\hat{x})$ & Persistent facts.                                & $\bang fact(\hat{x})$                         & Body or Head    & \texttt{$\bang$edge(X, Y, W)}                  \\ \hline
    $1$                          & Represents rules with an empty head.             & $1$                                           & Head            & \texttt{1}                                     \\ \hline
    $A \otimes B$                & Connect two expressions.                         & $A, B$                                        & Body and Head   & \texttt{path(A, P), edge(A, B, W)}             \\ \hline
    $\forall x. A$               & To represent variables defined inside the rule.  & Please see $A \lolli B$                       & Rule            & \texttt{path(A, B) $\lolli$ reachable(A, B)}   \\ \hline
    $\exists x. A$               & Instantiates new node variables.  & $\existsc{\widehat{x}}{B}$                  & Head            & \texttt{exists A.(path(A, P))}                 \\ \hline
    $A \lolli B$                 & $\lolli$ means "linearly implies".               & $A \lolli B$                                  & Rule            & \texttt{path(A, B) $\lolli$ reachable(A, B)}   \\
                                 & $A$ is the body and $B$ is the head.             &                                               &                 &                                                \\ \hline
    $\iters{name}{\widehat{V}}$               & For comprehensions
    ($\widehat{V}$ is empty).  & $\comprehension{\widehat{x}}{A}{B}$  & Head            & \texttt{\{B | !edge(A, B) | visit(B)\}}        \\
                                 & For aggregates ($\widehat{V}$ accumulates).          &                                               &                 &                                                \\ \hline
    \end{tabular}
}
\end{center}
\caption{Connectives from \fragment and their use in LM.}
\label{table:linear}
\end{table*}

The connections between LM and \fragment presented in the previous section
are somewhat obvious. We summarize the connection connectives of the system and
the LM syntax in Table~\ref{table:linear}. As an example, we translate the
abstract syntax of the first rule of the graph visit program shown
in~\ref{visit:ast} to a proposition in \fragment:

\begin{align}
\forall_A. (\mathtt{visit}(A) \otimes \mathtt{unvisited}(A) \lolli
   \mathtt{visited}(A) \otimes \itersz{comp} A)
\end{align}

The translation is fairly straightforward, except for the comprehension. Each
comprehension of a LM program must be assigned to an unique name and its
corresponding terms. For the iterative definition $comp$, it is defined as
following:

\begin{align}
\iterz{comp}{0} A & \defeq \one\\
\iterz{comp}{N} A & \defeq \forall_B.((\bang \mathtt{edge}(A, B) \lolli
\mathtt{visit}(B)) \otimes \iterz{comp}{N-1} A)
\end{align}

Notice that the argument list of the iterative definition is being used
to pass around terms from outside the definition, in this case, the variable
$A$. However, the argument list can also be used to implement aggregates.
Recall the aggregate example shown before:

{\small
\begin{Verbatim}
count-prices(A) -o [sum => P | . | price(A, P) | 1 | total(A, P)].
\end{Verbatim}
}

This rule is translated into a linear logic proposition as shown next:

\begin{align}
\forall A. (\mathtt{count-prices}(A) \lolli \iters{agg} A \; 0)
\end{align}

The iterative definition $\mathtt{agg}$ is defined as follows:

\begin{align}
\iterz{agg}{0} A \; P' & \defeq \mathtt{total}(A, P') \\
\iterz{agg}{N} A \; P' & \defeq \forall_P. ((\mathtt{price}(A, P) \lolli \one)
      \otimes \iterz{agg}{N-1} A \; (P' + P))
\end{align}

\section{High Level Dynamic Semantics}


In this section, we present the high level dynamic~(HLD) semantics of LM.  HLD
formalizes the mechanism of matching rules and deriving new facts.  HLD
semantics present a simplified overview of the dynamics of the language that are
closer to \fragment (Section~\ref{sec:fragment}) presented before.  than the
implementation principles of our virtual machine. The low level dynamic~(LLD)
semantics are much closer to a real implementation and represents the
operational semantics of the language.

Note that both HLD and LLD do not model the use of variable bindings when
matching facts from the database. The formalization of bindings tends to
complicate the formal system and it is not necessary for a good understanding of
the system. Instead, we assume that all facts of type $\emph{fact}(\hat{x})$ do
not have the argument $\hat{x}$.

Starting from \fragment presented earlier, we consider $\Gamma$ and $\Delta$ the database
of our program. $\Gamma$ contains the database of persistent facts while $\Delta$ the database of linear
facts. We assume that the rules of the program are persistent linear implications of the form
$\bang (A \lolli B)$ that can be used several times. However, we do not put the rules in the $\Gamma$
context but in a separate context $\Phi$.

The main idea of the dynamic semantics is to ignore the right side of the
sequent calculus and use \emph{chaining} and \emph{inversion} on the $\Delta$
and $\Gamma$ contexts so that we only have atomic facts (e.g., the database of
facts).  To apply rules we use
\emph{focusing}~\cite{Andreoli92logicprogramming} on one of the derivation rules
in $\Phi$. Note that in the focusing process we assume that all the atoms
(facts) are positive thus the chaining proceeds in a \emph{forward chaining}
fashion.

\subsection{Step}\label{sec:step_hld}

Operationally, LM proceeds in \emph{steps}. A step happens at some node $i$ and
proceeds by picking one rule to apply, matching the body of the rule against the
database, removing all those facts from the database and then deriving all the
constructs in the head of the rule. We assume the existence of $n$ nodes in the
program and that $\Delta$ and $\Gamma$ are split into $\Delta_1, \dotsc, \Delta_n$
and $\Gamma_1, \dotsc, \Gamma_n$ respectively. After each step, the database of
each fact is updated accordingly.

Steps are defined as $\stepz \Gamma; \Delta; \Phi \Longrightarrow \Gamma';
\Delta'$, where $\Gamma'$ and $\Delta'$ is the new database. The step rule is as
follows:

{\footnotesize
\[
\infer[\stepz]
{\stepz [\Gamma_1 .. \Gamma_i .. \Gamma_n]; [\Delta_1 .. \Delta_i ..
   \Delta_n]; \Phi \Longrightarrow [\Gamma_1, \Gamma'_1; .. \Gamma_i,
   \Gamma'_i; .. \Gamma_n, \Gamma'_n]; [\Delta_1, \Delta'_1; .. (\Delta_i -
         \Xi'), \Delta'_i; .. \Delta_n, \Delta'_n]}
{\doz \Gamma_i; \Delta_i; \Phi \rightarrow \Xi'; \Delta'_1 .. \Delta'_n;
   \Gamma'_1 .. \Gamma'_n}
\]
}


\subsection{Application}

A step is performed through $\doz \Gamma; \Delta; \Phi \rightarrow \Xi';
\Delta'; \Gamma'$.  $\Gamma$, $\Delta$ and $\Phi$ have the meaning explained
before, while $\Xi'$, $\Delta'$ and $\Gamma'$ are output multi-sets from
applying one of the rules in $\Phi$. $\Xi'$ is the set of consumed linear
resources, $\Delta'$ is the set of derived linear facts and $\Gamma'$ is the set
of derived persistent facts. Note that for HLD semantics there is no concept of
rule priority, therefore a rule is picked non-deterministically.

The judgment $\az \Gamma ; \Delta ; A \lolli B \rightarrow \Xi'; \Delta';
\Gamma'$ will attempt to apply the derivation rule $A \lolli B$. To do this, it
splits the $\Delta$ context into $\Delta_1$ and $\Delta_2$, namely the set of
linear resources consumed to match the body of the rule ($\Delta_1$) and the
remaining linear facts ($\Delta_2$).  Again, the set of resources needed to
match the body of the rule is guessed. LLD semantics will deterministically
calculate $\Delta_1$.

\[
\infer[\az \m{rule}]
{\az \Gamma ; \Delta_1, \Delta_2 ; A \lolli B \rightarrow \Xi' ; \Delta' ; \Gamma'}
{\mz \Gamma ; \Delta_1 \rightarrow A & \dz \Gamma ; \Delta_2; \Delta_1; \cdot ; \cdot ; B \rightarrow \Xi' ; \Delta' ; \Gamma'}
\]

\[
\infer[\doz \m{rule}]
{\doz \Gamma ; \Delta ; R, \Phi \rightarrow \Xi' ; \Delta' ; \Gamma'}
{\az \Gamma ; \Delta ; R \rightarrow \Xi' ; \Delta' ; \Gamma'}
\]


\subsection{Match}

The $\mz \Gamma ; \Delta \rightarrow C$ judgment uses the right ($R$) rules of
\fragment in order to match (prove) the body $C$ using $\Gamma$
and $\Delta$. We must consume all the linear facts in the multi-set $\Delta$
when matching $C$. The context $\Gamma$ may be used to match persistent terms in
$C$ but such facts are never consumed since they are persistent.

\[
\infer[\mzname \one]
{\mz{\Gamma}{\cdot}{\one}}
{}
\]
\[
\infer[\mzname p]
{\mz{\Gamma}{p}{p}}
{}
\tab
\infer[\mzname \bang p]
{\mz{\Gamma, p}{\cdot}{\bang p}}
{}
\]

\[
\infer[\mzname \otimes]
{\mz{\Gamma}{\Delta_1, \Delta_2}{A \otimes B}}
{\mz{\Gamma}{\Delta_1}{A} & \mz{\Gamma}{\Delta_2}{B}}
\]


\subsection{Derivation}

After successfully matching the body of the rule, we next derive the head of the
rule. The derivation judgment has the form $\dz \Gamma ; \Delta ; \Xi ; \Gamma_1
; \Delta_1 ; \Omega \rightarrow \Xi'; \Delta'; \Gamma'$ with the following
meaning:

\begin{enumerate}

   \item[$\Gamma$] the multi-set of persistent resources in the database;

   \item[$\Delta$] the multi-set of linear resources in the database not yet
   consumed;

   \item[$\Xi$] the multi-set of linear resources that have been consumed while
   matching the body of the rule, matching comprehensions or aggregates;

   \item[$\Gamma_1$] the multi-set of persistent facts that have been derived
   using the current rule;

   \item[$\Delta_1$] the multi-set of linear facts that have been derived using
   the current rule;

   \item[$\Omega$] an ordered list contain the terms of the head of rule that
   still need to be derived. We start with the head of the rule $B$ that is
   continuously deconstructed to derive all the facts of the rule;

   \item[$\Xi'$] the consumed linear facts to apply this rule;

   \item[$\Delta'$] the derived linear facts;

   \item[$\Gamma'$] the derived persistent facts.

\end{enumerate}

The following derivation rules are a direct translation from \fragment:

\[
\infer[\dz p]
{\dz \Gamma ; \Delta ; \Xi ; \Gamma_1 ; \Delta_1 ; p, \Omega \rightarrow \Xi' ; \Delta' ; \Gamma'}
{\dz \Gamma ; \Delta ; \Xi ; \Gamma_1 ; p, \Delta_1 ; \Omega \rightarrow \Xi' ; \Delta' ; \Gamma'}
\]

\[
\infer[\dz \bang p]
{\dz \Gamma ; \Delta ; \Xi ; \Gamma_1 ; \Delta_1 ; \bang p, \Omega \rightarrow \Xi' ; \Delta' ; \Gamma'}
{\dz \Gamma ; \Delta ; \Xi ; \Gamma_1, p ; \Delta_1 ; \Omega \rightarrow \Xi' ; \Delta' ; \Gamma'}
\]

\[
\infer[\dz \otimes]
{\dz \Gamma ; \Delta ; \Xi ; \Gamma_1 ; \Delta_1 ; A \otimes B, \Omega \rightarrow \Xi' ; \Delta' ; \Gamma'}
{\dz \Gamma ; \Delta ; \Xi ; \Gamma_1 ; \Delta_1 ; A, B, \Omega \rightarrow \Xi' ; \Delta' ; \Gamma'}
\]

\[
\infer[\dz \one]
{\dz \Gamma ; \Delta ; \Xi ; \Gamma_1; \Delta_1 ; 1, \Omega \rightarrow \Xi' ; \Delta' ; \Gamma'}
{\dz \Gamma ; \Delta ; \Xi ; \Gamma_1; \Delta_1 ; \Omega \rightarrow \Xi' ; \Delta' ; \Gamma'}
\]

\[
\infer[\dz end]
{\dz \Gamma ; \Delta ; \Xi' ; \Gamma' ; \Delta' ; \cdot \rightarrow \Xi' ; \Delta' ; \Gamma'}
{}
\]


We did not include the aggregates here because they are similar to comprehensions.
The starting rule for deriving comprehensions is $\dz comp^*$. It
deterministically picks a number $N$ that then can be unfolded $N$ times to get
$A \lolli B$. The HLD semantics do not take into account the contents of the
database to determine how many times a comprehension should be applied.

\[
\infer[\dz \m{comp}^*]
{\dz \Gamma ; \Delta ; \Xi ; \Gamma_1 ; \Delta_1 ; \compsz{A}{B}, \Omega \rightarrow \Xi' ; \Delta' ; \Gamma'}
{\dz \Gamma ; \Delta ; \Xi ; \Gamma_1 ; \Delta_1 ; \compz{N}{A}{B}, \Omega \rightarrow \Xi' ; \Delta' ; \Gamma'}
\]

\[
\infer[\dz \m{comp}^N]
{\dz \Gamma ; \Delta ; \Xi ; \Gamma_1 ; \Delta_1 ; \compz{N}{A}{B}, \Omega \rightarrow \Xi' ; \Delta' ; \Gamma'}
{\dz \Gamma ; \Delta ; \Xi ; \Gamma_1 ; \Delta_1 ; \compunfold{N-1}{A}{B}, \Omega \rightarrow \Xi' ; \Delta' ; \Gamma'}
\]

\[
\infer[\dz \m{comp}^0]
{\dz \Gamma ; \Delta ; \Xi ; \Gamma_1 ; \Delta_1 ; \compz{0}{A}{B}, \Omega \rightarrow \Xi' ; \Delta' ; \Gamma'}
{\dz \Gamma ; \Delta ; \Xi ; \Gamma_1 ; \Delta_1 ; \compunfoldz, \Omega \rightarrow \Xi' ; \Delta' ; \Gamma'}
\]


These rules mirror the rules for the iterative definition in \fragment.  Rules
for aggregates are similar:

\[
\infer[\dz \m{agg}^*]
{\dz \Gamma ; \Delta ; \Xi ; \Gamma_1 ; \Delta_1 ; \aggsz{A}{B}{C}, \Omega \rightarrow \Xi' ; \Delta' ; \Gamma'}
{\dz \Gamma ; \Delta ; \Xi ; \Gamma_1 ; \Delta_1 ; \aggz{N}{A}{B}{C}{0}, \Omega \rightarrow \Xi' ; \Delta' ; \Gamma'}
\]

{\small
\[
\infer[\dz \m{agg}^N]
{\dz \Gamma ; \Delta ; \Xi ; \Gamma_1 ; \Delta_1 ; \aggz{N}{A}{B}{C}{V}, \Omega \rightarrow \Xi' ; \Delta' ; \Gamma'}
{\dz \Gamma ; \Delta ; \Xi ; \Gamma_1 ; \Delta_1 ; \aggunfold{N-1}{A}{B}{C}{V}, \Omega \rightarrow \Xi' ; \Delta' ; \Gamma'}
\]
}

\[
\infer[\dz \m{agg}^0]
{\dz \Gamma ; \Delta ; \Xi ; \Gamma_1 ; \Delta_1 ; \aggz{0}{A}{B}{C}{V}, \Omega \rightarrow \Xi' ; \Delta' ; \Gamma'}
{\dz \Gamma ; \Delta ; \Xi ; \Gamma_1 ; \Delta_1 ; \aggunfoldz{C}{V}, \Omega \rightarrow \Xi' ; \Delta' ; \Gamma'}
\]


Finally, because the comprehensions and aggregates create implications $A \lolli
B$, we add a final derivation rule $\dz \lolli$:

\[
\infer[\dzname \lolli]
{\dz{\Gamma}{\Pi}{\Delta_a, \Delta_b}{\Xi}{\Gamma_1}{\Delta_1}{A \lolli B,
   \Omega}{\outsem}}
{\mz{\Gamma}{\Delta_a}{A} & \dz{\Gamma}{\Pi}{\Delta_b}{\Xi, \Delta_a}
   {\Gamma_1}{\Delta_1}{B, \Omega}{\outsem}}
\]

\[
\infer[\dzname \forall]
{\dz{\Gamma}{\Pi}{\Delta}{\Xi}{\Gamma_1}{\Delta_1}{\forall_x. A,
   \Omega}{\outsem}}
{\dz{\Gamma}{\Pi}{\Delta}{\Xi}{\Gamma_1}{\Delta_1}{A\{V/x\},
   \Omega}{\outsem}}
\]



\section{Low Level Dynamic Semantics}
The Low Level Dynamic~(LLD) semantics remove all the non-deterministic choices
in the previous dynamics and makes them deterministic. The new semantics will do
the following:

\begin{itemize}

   \item Match rules by priority order;

   \item Determine the set of linear facts needed to match either the rule's LHS
      or the LHS of comprehensions/aggregates without guessing;

   \item Apply as many comprehensions as the database allows.

   \item Apply as many aggregates as the database allows.

\end{itemize}

While the implementation presented in Chapter~\ref{chapter:local} follows the
LLD semantics, there are several optimizations not implemented in LLD, such as:

\begin{itemize}
   \item Indexing: the implementation uses indexing for looking up facts using a
      specific argument;
   \item Better candidate rules: when selecting a rule to execute, the
      implementation filters out rules which do not have enough facts to be
      derived;
   \item Multiple rule derivation: the LLD semantics only execute one rule at
      the time, while the implementation is able to derive a rule multiple times
      when there are no conflicting rule;
   \item Matching and substitution: in the implementation, matching is done
      implicitly using variables and comparisons, while LLD uses the $\Psi$
      context to hold substitutions.
\end{itemize}

The complete set of inference rules for the LLD semantics are presented in
Appendix~\ref{sec:lld}.

LLD is specified as an \emph{abstract machine} and is represented as a sequence
of state transitions of the form $\trans{S_1}{S_2}$. HLD had many different
proof trees for a given triplet $\Gamma; \Delta; \Phi$ because HLD allows
choices to be made during the inference rules. For instance, in HLD any rule
could be selected to be executed. In LLD there is only one state sequence
possible for a given $\Gamma; \Delta; \Phi$ since there is no guessing involved.
LLD semantics present a complete step by step mechanism that is needed to
correctly evaluate a LM program. For instance, when LLD tries to apply a rule,
it will check if there are enough facts in the database and backtrack until a
rule can be applied.

\subsection{Application}

LLD shares exactly the same inputs and outputs as HLD. The inputs correspond to
the $\Gamma$ and $\Delta$ fact contexts and the list of rules $\Phi$, while the
outputs correspond to the newly asserted facts in $\Gamma'$ and $\Delta'$ and
the retracted facts which are put in the $\Xi'$ context.

The first difference between LLD and HLD start when picking a rule to derive.
Instead of guessing, LLD treats the list of rules as a stack and picks the first
rule $R_1$ to execute (the rule with the highest priority). The remaining rules
are stored as a \emph{continuation}. If $R_1$ cannot be matched because there
are not enough facts in the database, we backtrack and use the rule continuation
to pick the next rule and so on, until one rule can be successfully applied.

The machine starts with a database $(\Gamma; \Delta)$ and a list of rules
$\Phi$. The initial state is always $\dostate{\Delta}{\Phi}{\Gamma}{\Pi}$.
We start by picking the first rule $R_1$ from $\Phi$:


\[
\trans{\dostate{\Delta}{R_1, \Phi}{\Gamma}{\Pi}}
{\appstate{\cdot}{\Delta}{\Phi}{\Pi}{\Gamma}{R}} \tag{select rule}
\]


If, after trying all the rules, there are no remaining candidate rules, the
machine enters into the $\m{next}$ state, which means that no more rules are
possible for this node and the machine should perform local computation on
another node.


\[
\trans{\dostate{\Delta}{\cdot}{\Gamma}{\Pi}}
   {\failstate{\Gamma}{\Delta}} \tag{fail}
\]


In order to try a particular rule, we either need to unfold the $\forall$
connective, by adding its variable to the $\Psi$ context, or, initiate the matching
process when reaching the $\lolli$ connective. The variables in the $\Psi$
context, which are initially assigned to an unknown value $\_$, will later be
assigned to a concrete value as the matching process goes forward.


\[
   \trans{\appstate{\Psi}{\Delta}{\Phi}{\Pi}{\Gamma}{\forall_{x : \tau}. A}}
   {\appstate{\Psi, x : \_ : \tau}{\Delta}{\Phi}{\Pi}{\Gamma}{A}}
                                                             \tag{open rule}
\]


\[
   \trans{\appstate{\Psi}{\Delta}{\Phi}{\Pi}{\Gamma}{A \lolli B}}
   {\matstateb{A \lolli B}{(\Delta; \Phi)}{\cdot}{\Gamma}{\Delta}{A}{\cdot \rightarrow
   \one}{\Psi}} \tag{init
                                                            rule}
\]



\subsection{Continuation Frames}

The most interesting aspects introduced by the LLD machine are the
\emph{continuation frame} and the \emph{continuation stack}. A continuation
frame acts as a choice point that is created during rule matching whenever we
try to match a fact expression against the database.  The frame considers all
the facts relevant to the expression given the current context $\Psi$.

The frame contains enough state to resume the matching process at the time of
its creation, therefore we can easily backtrack to the choice point and select
the next candidate fact from the database.  We keep the continuation frames in a
continuation stack for backtracking purposes. If, at some point there are no
candidate facts because the current variable assignments are not usable, we
update the top frame to try the next candidate fact. If all candidates are
exhausted, we pop the top frame and continue with the next available frame.

By using this match mechanism, we determine which facts need to be used to match
a rule.  Our LM implementation works like LLD, by iterating over the available
facts at each choice point and then committing to the rule if the matching
process succeeds. However, while the implementation only attempts to match rules
when the database has all the facts required by the rule's LHS, LLD is more
na\"{i}ve in this aspect because it tries all rules in order.


\subsection{Structure of Continuation Frames}

We have two continuation frame types, depending on the type of the candidate
facts.\footnote{All continuation frames have an implicit $\Psi$ context that
models variable assignments, including variable names, values and their
locations in the terms. This is important if we want to model variable
assignments and matchings.}

\subsubsection{Linear Continuation Frames}

There are two types of continuation frames. Linear frames use the form
$\lframe{\Delta}{\Delta''}{p(\widehat{x})}{\Omega; \Psi}{\Delta'}{\Omega'}$, where:

\begin{description}

   \item[$p(\widehat{x})$] atomic proposition that created this
      frame. The predicate for the proposition is $p$;

   \item[$\Delta$] multi-set of linear facts that are not of predicate $p$ plus
      all the other candidate facts of the predicate $p$ we have already
      tried, including a fact $p$, which is the current candidate fact;

   \item[$\Delta''$] facts of predicate $p$ that match $p(\widehat{x})$ which we
      haven't tried yet. It is a multi-set of linear facts;


   \item[$\Omega$] ordered list of remaining terms needed to match;

   \item[$\Delta'$] multi-set of linear facts we have consumed to reach this point;

   \item[$\Omega'$] terms matched already using $\Delta'$ and $\Gamma$;
   \item[$\Psi$] dictionary of variable assignments (includes variable and
      value).
\end{description}

\subsubsection{Persistent Continuation Frame}

Persistent frames are slightly different since they only need to keep track of
remaining persistent candidates. They are structured as
$\pframe{\Gamma''}{\Delta}{\bang
   p(\widehat{x})}{\Omega; \Psi}{\Delta'}{\Omega'}$:

\begin{description}
   \item[$\bang p(\widehat{x})$] persistent atomic proposition that created
      this frame;
   \item[$\Gamma''$] remaining candidate facts that match $\bang p(\widehat{x})$;
   \item[$\Delta$] multi-set of linear facts not consumed yet;

   \item[$\Omega$] ordered list of terms needed to match past this
   frame;

   \item[$\Delta'$] multi-set of linear facts consumed up-to this frame;
   \item[$\Omega'$] terms matched up-to this point using $\Delta'$ and $\Gamma$;
   \item[$\Psi$] dictionary of variable assignments (includes variable and value).
\end{description}


\subsection{Match}\label{sec:lld_body_match}

The matching state for the LLD machine uses the continuation stack to try
different combinations of facts until a match is achieved.  The state is
structured as $\matstate{A \lolli
   B}{\rulestk}{\lstack{C}}{\Gamma}{\Delta}{\Omega}{\Delta' \rightarrow
      \Omega'}$, where:

\begin{description}
   \item[$A \lolli B$] rule being matched: $A$ is the rule's LHS and $B$ the RHS;

   \item[$\rulestk$] rule continuation to be used if the current rule fails.
   Contains the original $\Delta_N$ and the rest of the rules $\Phi$;

   \item[$\lstack{C}$] ordered list of frames representing the continuation
   stack used for matching $A$;

   \item[$\Delta$] multi-set of linear facts still available to complete the
   matching process;

   \item[$\Omega$] ordered list of deconstructed RHS terms to match;

   \item[$\Delta'$] multi-set of linear facts from the original $\Delta_N$ that
   were already consumed ($\Delta', \Delta = \Delta_N$);

   \item[$\Omega'$] parts of $A$ already matched. They are in the form $P_1
   \otimes \dotsb \otimes P_n$. The idea is to use term equivalence and the fact
   that $\feq{\Omega, \Omega'}{A}$ to justify $\mz{\Gamma}{\Delta'}{A}$ when the
   matching process completes.

\end{description}

Not shown in the matching state is the context $\Psi$ that maps variables to
values. At the start of matching, the $\widehat{x}$ variables are set as
\emph{undefined}. Matching then uses facts from $\Delta$ and $\Gamma$ to match
the terms of the rule's LHS represented as $\Omega$. During the process
continuation frames are pushed into $\lstack{C}$ and if the matching process
fails, we use $\lstack{C}$ to restore the process using different candidate
facts. New facts also update the variables in the $\Psi$ context by assigning
them concrete values.

\subsubsection{Linear fact expression}

The first 2 state transitions are used when the head of $\Omega$ is a linear fact
expression $p$.

In the first transition we find $p_1$ and $\Delta''$ as facts from the database
that match $p$'s hidden and partially initialized arguments.  Context $\Delta''$
is stored in the second argument of the new continuation frame but is passed
along with $\Delta$ since the facts have not been consumed yet (just $p_1$).

The second transition deals with the case where there are no candiate facts and
thus a different machine state is used for enabling backtracking.

Note that the proposition $p_1, \Delta'' \prec p$ indicates that facts
$\Delta'', p_1$ satisfy the constraints of $p$ while $\Delta \npreceq p$
indicates that no fact in $\Delta$ satisfies $p$. Both propositions use the
omitted variable context $\Psi$ in order to replace the variables of $p$.


\begin{multline}
\transx{\matstateb{A \lolli B}{\rulestk}{\lstack{C}}{\Gamma}{\Delta, p_1,
\Delta''}{p(\widehat{x}),
   \Omega}{\Delta' \rightarrow \Omega'}{\Psi}}
{\matstateb{A \lolli B}{\rulestk}{\lframe{\Delta,
p_1}{\Delta''}{p(\widehat{x})}{\Omega; \m{extend}(\Psi, \theta)}{\Delta'}{\Omega'}, \lstack{C}}{\Gamma}{\Delta,
   \Delta''}{\Omega}{\Delta', p_1 \rightarrow \Omega' \otimes
      p(\widehat{x}\theta)}{\m{extend}(\Psi, \theta)}} \\
   \;\;\; (p_1,
   \Delta'' \prec p(\widehat{x}) \;\;\; \Delta \npreceq p(\widehat{x}))
   \tag{match p ok}
\end{multline}

\begin{align}
   \trans{\matstate{A \lolli
   B}{\rulestk}{\lstack{C}}{\Gamma}{\Delta}{p(\widehat{x}),
   \Omega}{\Delta' \rightarrow \Omega'}}
{\contstate{A \lolli B}{\rulestk}{\lstack{C}}{\Gamma}} \;\;\; (\Delta \npreceq
p(\widehat{x})) \tag{match p fail}
\end{align}


\subsubsection{Persistent fact expressions}

The next 2 state transitions are used when the head of $\Omega$ contains a
persistent fact expression $\bang p$. They are identical to the previous 2
transitions but they deal with the persistent context $\Gamma$.


\[
\infer[\mo \bang p~\m{first}]
{\mo \Gamma, p_1, \Gamma'' ; \Delta; \Xi; \bang p, \Omega; H; \cdot; \lstack{R}
   \rightarrow \outsem}
{
   \begin{gathered}
      p_1, \Gamma'' \prec \bang p \\
      \mo \Gamma, p_1, \Gamma'' ; \Delta; \Xi; \Omega;
      H; [\Gamma''; \Delta; \bang p ; \Omega; \Xi; \cdot; \cdot]; \lstack{R} \rightarrow \outsem
   \end{gathered}
}
\]

\[
\infer[\mo \bang p~\m{on}~q]
{\mo \Gamma, p_1, \Gamma'' ; \Delta; \Xi; \bang p, \Omega; H; f, \lstack{C};
   \lstack{R}
   \rightarrow \outsem}
{
   \begin{gathered}
      p_1, \Gamma'' \prec \bang p \\
      f = (\Delta_{old}; \Delta'_{old};
         q; \Omega_{old}; \Xi_{old}; \Lambda; \Upsilon) \\
      \mo \Gamma, p_1,
         \Gamma'' ; \Delta; \Xi; \Omega; H; [\Gamma''; \Delta; \bang p ; \Omega; \Xi; q,
      \Lambda; \Upsilon], f, \lstack{C}; \lstack{R} \rightarrow \outsem
   \end{gathered}
}
\]


\[
\infer[\mo \bang p~\m{on}~\bang q]
{\mo \Gamma, p_1, \Gamma'' ; \Delta; \Xi; \bang p, \Omega; H; f, \lstack{C};
   \lstack{R}
   \rightarrow \outsem}
{
   \begin{gathered}
      p_1, \Gamma'' \prec \bang p \\
      f = [\Gamma_{old}; \Delta_{old}; \bang q; \Omega_{old}; \Xi_{old}; \Lambda; \Upsilon] \\
      \mo \Gamma, p_1, \Gamma'' ; \Delta; \Xi; \Omega; H; [\Gamma''; \Delta;
      \bang p ; \Omega; \Xi; \Lambda; q, \Upsilon], f, \lstack{C}; \lstack{R} \rightarrow \outsem
   \end{gathered}
}
\]

\[
\infer[\mo \bang p~\m{fail}]
{\mo \Gamma ; \Delta; \Xi; \bang p, \Omega; H; \lstack{C}; \lstack{R} \rightarrow \outsem}
{\Gamma \npreceq \bang p & \cont \lstack{C}; H; \lstack{R}; \Gamma \rightarrow \outsem}
\]


\subsubsection{Other expressions}

Finally, we have the transitions that deconstruct the other LHS terms and a
final transition to initiate the RHS derivation.


\begin{align}
\trans{\matstate{A \lolli B}{\rulestk}{\lstack{C}}{\Gamma}{\Delta}{\one,
   \Omega}{\Delta' \rightarrow \Omega'}}
{\matstate{A \lolli B}{\rulestk}{\lstack{C}}{\Gamma}{\Delta}{\Omega}{\Delta'
   \rightarrow \Omega'}} \tag{match $\one$}
\end{align}

\begin{align}
\trans{\matstate{A \lolli B}{\rulestk}{\lstack{C}}{\Gamma}{\Delta}{X \otimes Y,
   \Omega}{\Delta' \rightarrow \Omega'}}
{\matstate{A \lolli B}{\rulestk}{\lstack{C}}{\Gamma}{\Delta}{X, Y,
   \Omega}{\Delta' \rightarrow \Omega;}} \tag{match $\otimes$}
\end{align}

\begin{align}
\trans{\matstate{A \lolli
   B}{\rulestk}{\lstack{C}}{\Gamma}{\Delta}{\cdot}{\Delta' \rightarrow \Omega'}}
{
   \derstatex{\Gamma}{\Delta}{\Delta'}{\cdot}{\cdot}{B}
} \tag{match end}
\end{align}


\subsection{Backtracking}\label{sec:lld_match_cont}

The backtracking state of the machine reads the top of the continuation stack
$\lstack{C}$ and restores the matching process with a different candidate fact
from the continuation frame. The state is written as $\contstate{A \lolli
B}{\rulestk}{\lstack{C}}{\Gamma}$, where:

\begin{description}
   \item[$A \lolli B$] the rule being matched;
   \item[$\rulestk$] next available rules if the current rule does not match;
   the rule continuation;
   \item[$\lstack{C}$] the continuation stack for matching $A$;
\end{description}

\subsubsection{Linear continuation frames}

The next two state transitions handle linear continuation frames on the top of the
continuation stack. The first transition selects the next candidate fact $p_1$ from the
second argument of the linear frame and updates the frame. Otherwise, if we have
no more candidate facts, we pop the continuation frame and backtrack to the
remaining continuation stack.

\begin{align}
\trans{\contstate{A \lolli B}{\rulestk}{\lframe{\Delta}{p_2,
   \Delta''}{p}{\Omega}{\Delta'}{\Omega'}, \lstack{C}}{\Gamma}}
{
   \matstate{A \lolli B}{\rulestk}{\lframe{\Delta,
      p_2}{\Delta''}{p}{\Omega}{\Delta'}{\Omega'},
   \lstack{C}}{\Gamma}{\Delta}{\Omega}{\Delta', p_2 \rightarrow \Omega' \otimes p}}
   \tag{next p}
\end{align}

\begin{align}
\trans{\contstate{A \lolli
   B}{\rulestk}{\lframe{\Delta}{\cdot}{p}{\Omega}{\Delta'}{\Omega'},
      \lstack{C}}{\Gamma}}
{
   \contstate{A \lolli B}{\rulestk}{\lstack{C}}{\Gamma}} \tag{next frame}
\end{align}


\subsubsection{Persistent continuation frames}

We also have the same two kinds of inference rules for persistent continuation
frames.

\[
\infer[\cont \bang p~\m{next}]
{\cont [p_1, \Gamma'; \Delta; \bang p, \Omega; \Xi; \Lambda; \Upsilon],
   \lstack{C}; H; \lstack{R};
   \Gamma \rightarrow \outsem}
{\mo \Gamma; \Delta; \Xi; \Omega; H; [\Gamma'; \Delta; \bang p, \Omega; \Xi;
   \Lambda; \Upsilon], \lstack{C}; \lstack{R} \rightarrow \outsem}
\]

\[
\infer[\cont \bang p~\m{no~more}]
{\cont [\cdot; \Delta; \bang p, \Omega; \Xi; \Lambda; \Upsilon], \lstack{C}; H;
   \lstack{R}; \Gamma
   \rightarrow \outsem}
{\cont \lstack{C}; H; \lstack{R}; \Gamma \rightarrow \outsem}
\]


\subsubsection{Empty continuation stack}

Finally, if the continuation stack is empty, we simply force execution to try
the next inference rule in $\Phi$.

\[
\infer[\cont \m{next~rule}]
{\cont \cdot; H; (\Phi, \Delta); \Gamma \rightarrow \Xi'; \Delta'; \Gamma'}
{\doo \Gamma; \Delta; \Phi \rightarrow \Xi'; \Delta'; \Gamma'}
\]


\subsection{Derivation}

Once the list of terms $\Omega$ of the LHS is exhausted, we derive the rule's
RHS. The derivation state simply iterates over $B$, the rule's RHS, and derives
terms into the corresponding new contexts. The state is represented as
$\derstatex{\Gamma}{\Delta}{\Xi}{\Gamma_1}{\Delta_1}{\Omega}$ with the following
meaning:

\begin{enumerate}
   \item[$\Gamma$] set of persistent facts;

   \item[$\Delta$] multi-set of remaining liner facts;

   \item[$\Xi$] multi-set of linear facts consumed up-to this point;

   \item[$\Gamma_1$] set of persistent facts derived;

   \item[$\Delta_1$] multi-set of linear facts derived;

   \item[$\Omega$] remaining terms to derive as an ordered list. We start with
   $B$ if the original rule is $A \lolli B$.

\end{enumerate}

\subsubsection{Fact templates}

When deriving either $p$ or $\bang p$ we have the following two inference rules:

{\footnotesize
\[
\infer[\done p]
{\done \Gamma ; \Delta; \Xi; \Gamma_1 ; \Delta_1; p, \Omega \rightarrow \Xi'; \Delta'; \Gamma'}
{\done \Gamma ; \Delta; \Xi; \Gamma_1 ; p, \Delta_1; \Omega \rightarrow \Xi'; \Delta'; \Gamma'}
\tab
\infer[\done \bang p]
{\done \Gamma ; \Delta ; \Xi; \Gamma_1 ; \Delta_1; \bang p, \Omega \rightarrow \Xi'; \Delta'; \Gamma'}
{\done \Gamma ; \Delta ; \Xi; \Gamma_1, p; \Delta_1; \Omega \rightarrow \Xi'; \Delta'; \Gamma'}
\]
}


\subsubsection{RHS deconstruction}

The following two inference rules deconstruct the RHS list $\Omega$ from terms
created using either $\one$ or $\otimes$.

\[
\infer[\done 1]
{\done \Gamma; \Delta; \Xi; \Gamma_1 ; \Delta_1; 1, \Omega \rightarrow \outsem}
{\done \Gamma; \Delta; \Xi; \Gamma_1 ; \Delta_1; \Omega \rightarrow \outsem}
\tab
\infer[\done \otimes]
{\done \Gamma ; \Delta; \Xi; \Gamma_1; \Delta_1; A \otimes B, \Omega \rightarrow
   \outsem}
{\done \Gamma ; \Delta; \Xi; \Gamma_1; \Delta_1; A, B, \Omega \rightarrow
   \outsem}
\]


\subsubsection{Aggregates}

We also have a transition for aggregates. The aggregate starts with a set of
values $\widehat{V}$ and an accumulator initialized as $\cdot$. The second state
initiates the matching process of the LHS $A$ of the aggregate (explained in
the next section).

\[
\infer[\done \m{agg}]
{\done \Gamma; \Delta ; \Xi; \Gamma_1; \Delta_1; \aggsz{A}{B}{C}, \Omega
   \rightarrow \outsem}
{\ma \Gamma; \Delta; \Xi; \Gamma_1; \Delta_1; \cdot; A ; \cdot; \cdot;
   \aggsz{A}{B}{C}; \Omega; \Delta; \cdot \rightarrow \outsem}
\]


\subsubsection{Successful rule}

Finally, if the ordered list $\Omega$ is exhausted, then the whole execution
process is done.  Note how the output arguments match the input arguments of the
$\done$judgment.

\[
\infer[\done \m{end}]
{\done \Gamma; \Delta; \Xi; \Gamma_1; \Delta_1; \cdot \rightarrow \Xi; \Delta_1; \Gamma_1}
{}
\]


\subsection{Aggregates}

The most intricate part of the derivation process is processing comprehensions
and aggregates. For both of them, we need to perform as many derivations as the
database allows, therefore we need to deterministically check the contents of
the database until no more derivations are possible.  The matching process is
then similar to the process used for matching the body of the rule presented in
Section~\ref{sec:lld_body_match}, however we use two continuation stacks,
$\lstack{C}$ and $\lstack{P}$. In $\lstack{P}$, we put all the initial
persistent frames and in $\lstack{C}$ we put the first linear frame and then
everything else.

In order to reuse the stacks $\lstack{C}$ and $\lstack{P}$, we need to update
them by removing all the frames in $\lstack{C}$ pushed after the first linear
continuation frame.  If we tried to use those frames, we would assumed that the
linear facts used by the other frames were still in the database, but that is
not true because they have been consumed during the first application of the
comprehension.  For example, if the body is $\bang \mathtt{a(X)} \otimes
\mathtt{b(X)} \otimes \mathtt{c(X)}$ and the continuation stack has three frames
(one per fact), we cannot backtrack to the frame of $\mathtt{c(X)}$ because, at
that point, the matching process was assuming that the previous \texttt{b(X)}
linear fact was still available.  Moreover, we also need to remove the consumed
linear facts from the frames of \texttt{b(X)} and $\bang$\texttt{a(X)} in order
to make the stack fully consistent with the new database. We will see later on
how to do that.

Each aggregate derivation also needs to accumulate a list of values for each
combination. Once all combinations are performed, then the main head of the
aggregate is derived using the combined value.

The matching state for aggregates is 
$\matstatea{\Delta_N}{\lstack{C};
   \lstack{P}}{\Gamma}{\Delta}{\Omega}{\Delta' \rightarrow \Omega'}{\Sigma}$

\begin{enumerate}
   \item[$\Omega_N$] ordered list of remaining terms of the head of the rule to
   be derived;

   \item[$\Delta_N$] multi-set of linear facts that were still available after
   matching the body of the rule and all the previous aggregates. Note that
   $\Delta, \Xi = \Delta_N$;

   \item[$\Xi$] multi-set of linear facts used during the matching process of
   the body of the rule and all the previous aggregates;

   \item[$\Gamma_{1}$] set of persistent facts derived up to this point in the
   head of the rule;

   \item[$\Delta_{1}$] multi-set of linear facts derived up to this point in
   the head of the rule;

   \item[$\Delta'$] multi-set of linear facts consumed up to this point;

   \item[$\Omega'$] terms matched using $\Delta'$ up to this point;

   \item[$\m{agg}$] aggregate that is being matched;

   \item[$\Sigma$] the list of aggregated values;

   \item[$\lstack{C}$] continuation stack that contains both linear and persistent
   frames. The first frame must be linear;

   \item[$\lstack{P}$] initial part of the continuation stack with only persistent
   frames;

   \item[$\Delta$] multi-set of linear facts remaining up to this point in the
   matching process;

   \item[$\Omega$] ordered list of terms that need to be matched for the
   comprehension to be applied.

\end{enumerate}

Since aggregates accumulate values (from specific variables), we extend the
$\Psi$ context to include triplets $x : M : \tau$ (variable, term and type)
instead of pairs $M : \tau$ in order to be able to retrieve bound variables.
Remember that $\Psi$ is used for the quantification connectives in the sequent
calculus.

\subsubsection{Linear fact expressions}

The following two transitions deal with the case when there is a linear
fact expression in the body of the aggregate.

\input{lld/agg-match-p}

\subsubsection{Persistent fact expressions}

The transitions for dealing with persistent facts are similar to the previous
ones.

\input{lld/agg-match-bang-p}

\subsubsection{Deconstruct body}

\input{lld/agg-match-other}


\subsubsection{Successful match}

When the aggregate body finally matches, we retrieve the term for variable $x$
(the aggregate variable) and add it to the list $\Sigma$.

\input{lld/agg-match-end}

\subsubsection{Continuation stack update}

As we said before, to update the continuation stacks, we need remove to all the
frames except the first linear frame and remove the consumed linear facts from
the remaining frames so that they are still valid for the next application of
the aggregate.  The judgment that updates the stack has the form
$\fixstatea{\Delta}{\Xi; \Delta'}{\lstack{C};
   \lstack{P}}{\Gamma}{\Sigma}$, where:

\begin{enumerate}
   \item[$\Omega_N$] ordered list of remaining terms of the head of the rule to
   be derived;
   \item[$\Delta$] multi-set of linear facts that were still available after
   matching the body of the rule and the body of the aggregate;
   \item[$\Xi$] multi-set of linear facts used during the matching process of
   the body of the rule and all the previous aggregates;
   \item[$\Delta'$] multi-set of linear facts consumed by the aggregate body;
   \item[$\Gamma_{1}$] set of persistent facts derived by the head of the rule
   and all the previous aggregates;
   \item[$\Delta_{1}$] multi-set of linear facts derived by the head of the
   rule and all the previous aggregates;
   \item[$\m{agg}$] the current aggregate;
   \item[$\Sigma$] list of accumulated values;
   \item[$\lstack{C}, \lstack{P}$] continuation stacks for the comprehension;
   \item[$\Gamma$] set of usable persistent facts.
\end{enumerate}

\subsubsection{Remove linear continuation frames}

To remove all linear continuation frames except the first one, we simply go
through all the frames in the stack $\lstack{C}$ until only one frame remains.
This last frame and stack $\lstack{P}$ are then updated by removing $\Delta'$
from its contexts.

\input{lld/agg-fix}

\subsubsection{Aggregate backtracking}

If the aggregate match fails, we need to backtrack to the next candidate fact.
The backtracking state 
has the form
$\contstatea{\Delta_N}{\lstack{C} ; \lstack{P}}{\Gamma}{\Sigma}$, where:

\begin{enumerate}
   \item[$\Omega_N$] ordered list of remaining terms of the head of the rule to
   be derived;
   \item[$\Delta_N$] multi-set of linear facts that were still available after
   matching the body of the rule and the body of the aggregate;
   \item[$\Xi$] multi-set of linear facts used during the matching process of
   the body of the rule and all the previous aggregates;
   \item[$\Gamma_{1}$] set of persistent facts derived by the head of the rule
   and all the previous aggregates;
   \item[$\Delta_{1}$] multi-set of linear facts derived by the head of the
   rule and all the previous aggregates;
   \item[$\m{agg}$] the current aggregate;
   \item[$\Sigma$] list of accumulated values.
   \item[$\lstack{C}, \lstack{P}$] continuation stacks for the comprehension;
   \item[$\Gamma$] set of usable persistent facts.
\end{enumerate}

\paragraph{Using the $\lstack{C}$ stack}

The following 4 state transitions use the $\lstack{C}$ stack, the stack where the
first continuation frame is linear, to perform backtracking.

\input{lld/agg-cont-c}

\paragraph{Using the $\lstack{P}$ stack}

The following 2 state transitions rules use the $\lstack{P}$ stack instead, the stack where all
continuation frames are persistent.

\input{lld/agg-cont-p}

\paragraph{Aggregate done}

If both the $\lstack{C}$ and $\lstack{P}$ stacks are empty, backtracking is
impossible and the aggregate is done. The final head of the aggregate is then
derived along with the rest of the rule's head.

\input{lld/agg-cont-end}

\subsubsection{Aggregate Derivation}

After updating the continuation stacks, the subhead of the aggregate is derived.
The derivation state has the form
$\derstatea{\Delta}{\Xi}{\Gamma_1}{\Delta_1}{\Sigma}{\lstack{C};
   \lstack{P}}{\Omega}$, where:

\begin{enumerate}
   \item[$\Omega_N$] ordered list of remaining terms of the head of the rule to
   be derived;
   \item[$\Delta$] multi-set of remaining linear facts that can be used for
   the next aggregate applications.
   \item[$\Xi$] multi-set of linear facts consumed both by the body of the rule
   and previous aggregate applications;
   \item[$\Gamma_1$] set of persistent facts derived by the head of the rule,
   previous aggregates and current derivation;
   \item[$\Delta_1$] multi-set of linear facts derived by the head of the rule,
   previous aggregates and current derivation;
   \item[$\m{agg}$] current aggregate symbol;
   \item[$\Sigma$] accumulated list of values of the aggregate;
   \item[$\lstack{C}, \lstack{P}$] new continuation stacks;
   \item[$\Gamma$] set of persistent facts;
   \item[$\Omega$] ordered list of terms to derive.
\end{enumerate}

\input{lld/agg-der}

This completes the specification of LLD.



\section{Soundness Proof}

Now that we have presented both the HLD and LLD semantics, we are in position to
start building our soundness theorem.  The soundness theorem proves that if a
rule was successfully derived in the LLD semantics then it can also be derived
in the HLD semantics. Since the HLD semantics are so close to linear logic, we
prove that our language has a determined, correct, proof search behavior when
executing programs. However, the completeness theorem cannot be proven since LLD
lacks the non-determinism inherent in HLD.

First and foremost, we need to prove some auxiliary theorems and definitions
that will be used during the soundness theorem. Note that $\outsem$ is a
short-hand for the output contexts of LLD and HLD.

\subsection{Term equivalence}

The first definition defines the equality between two multi-sets of terms.  Two
multi-sets $A$ and $B$ are equal, $\feq{A}{B}$, when they have the same
constituent atoms.

\[
\infer[\equiv p]
{\feq{p, A}{p, B}}
{\feq{A}{B}}
\tab
\infer[\equiv \bang p]
{\feq{\bang p, A}{\bang p, B}}
{\feq{A}{B}}
\tab
\infer[\equiv 1~L]
{\feq{1, A}{B}}
{\feq{A}{B}}
\tab
\infer[\equiv 1~R]
{\feq{A}{1, B}}
{\feq{A}{B}}
\]

\[
\infer[\equiv \cdot]
{\feq{\cdot}{\cdot}}
{}
\tab
\infer[\equiv \otimes~L]
{\feq{A \otimes B, C}{D}}
{\feq{A, B, C}{D}}
\tab
\infer[\equiv \otimes~R]
{\feq{A}{B \otimes C, D}}
{\feq{A}{B, C, D}}
\]

\begin{theorem}[Match equivalence]
If two multi-sets are equivalent, $\feq{A_1, \dotsc, A_n}{B_1, \dotsc, B_m}$,
   and we can match $A_1 \otimes \dotsb \otimes A_n$ in HLD such that $\mz
   \Gamma ; \Delta \rightarrow A_1 \otimes \dotsb \otimes A_n$ then $\mz \Gamma;
   \Delta \rightarrow B_1 \otimes \dotsb \otimes B_m$ is also true.
\end{theorem}
\begin{proof}
By straightforward induction on the first assumption.
\end{proof}

\subsection{Well-formed continuation frames}

We now define the concept of a well-formed frame given initial linear and
persistent contexts and a term $A$ that needs to be matched.

\begin{definition}[Well-formed frame]

Consider a triplet $A; \Gamma; \Delta_{N}$ where $A$ is a term, $\Gamma$ is a
multi-set of persistent resources and $\Delta_{N}$ a multi-set of linear
resources. A frame $f$ is well-formed iff:

\begin{enumerate}[leftmargin=*]
   \item Linear frame $f = (\Delta, p_1; \Delta'; \Xi_1, \dotsc, \Xi_m; p;
         \Omega_1, \dotsc, \Omega_n; \Lambda_1, \dotsc, \Lambda_m; \Upsilon_1,
         \dotsc, \Upsilon_k)$

   \begin{enumerate}
      \item $\feq{p, \Omega_1, \dotsc, \Omega_n, \Lambda_1, \dotsc, \Lambda_m,
         \Upsilon_1, \dotsc, \Upsilon_k}{A}$ (the remaining terms and already
               matched terms are equivalent to the initial body $A$);
      \item $\mz \Xi_1, \dotsc, \Xi_m \rightarrow \Lambda_1 \otimes \dotsb \otimes
      \Lambda_m$ and $\mz \Xi_i \rightarrow \Lambda_i$ for every $i$;

      \item $\Delta, \Delta', \Xi, p_1 = \Delta_{N}$ (available facts, candidate
            facts for $p$, consumed facts and the linear fact used for $p$,
            respectively, are the same as the initial $\Delta_{N}$);

      \item $\mz \Gamma; \cdot \rightarrow \Upsilon_1 \otimes \dotsb \otimes
      \Upsilon_k$ (past persistent facts can be matched with $\Gamma$).

   \end{enumerate}
   \item Persistent frame $f = [\Gamma'; \Delta; \Xi_1, \dotsc, \Xi_m; \bang p;
         \Omega_1, \dotsc, \Omega_n; \Lambda_1, \dotsc, \Lambda_m; \Upsilon_1,
         \dotsc, \Upsilon_k]$

      \begin{enumerate}
         \item $\feq{\bang p, \Omega_1, \dotsc, \Omega_n, \Lambda_1, \dotsc,
                     \Lambda_m, \Upsilon_1, \dotsc, \Upsilon_k}{A}$;
         \item $\mz \Xi_1, \dots, \Xi_m \rightarrow \Lambda_1 \otimes \dotsb \otimes
                     \Lambda_m$ and $\mz \Xi_i \rightarrow \Lambda_i$ for every $i$;
         \item $\Delta, \Xi = \Delta_{N}$;
         \item $\mz \Gamma; \cdot \rightarrow \bang p \otimes \Upsilon_1 \otimes
                     \dotsb \otimes \Upsilon_k$;
         \item $\Gamma' \subset \Gamma$ (remaining candidates are a subset of
                     $\Gamma$).
      \end{enumerate}
\end{enumerate}
\end{definition}


\begin{definition}[Well-formed stack]
A continuation stack $C$ is well-formed iff every frame is well-formed.
\end{definition}

Given the previous definitions, we can now define what it means for a matching
judgment to be well-formed.

\begin{definition}[Well-formed body match]

$\mo \Gamma; \Delta; \Xi; \Omega; H; C; R \rightarrow \outsem$ is well-formed in relation to a triplet $A; \Gamma; \Delta_{N}$ iff:

\begin{itemize}[leftmargin=*]
   \item $\Delta, \Xi = \Delta_{N}$
   \item $C$ is well-formed in relation to $A; \Gamma; \Delta_{N}$ and:
   \begin{itemize}[leftmargin=\secondm]
      \item If $C = \cdot$
   
      $\feq{\Omega}{A}$.
   
      \item If $C = (\Delta_a, p_1; \Delta_b; \Xi''; p; \Omega'; \Lambda_1,
            \dotsc, \Lambda_m; \Upsilon_1, \dotsc, \Upsilon_k), C'$
   
      \begin{itemize}[leftmargin=\thirdm]
         \item $\feq{\Omega'}{\Omega}$;
         \item $p_1 \in \Xi$ and $\mz \Gamma; p_1 \rightarrow p$;
         \item $\Xi = \Xi'', p_1$;
         \item $\Delta = \Delta_a, \Delta_b$.
      \end{itemize}
      \item If $C = [\Gamma'; \Delta''; \Xi''; \bang p; \Omega'; \Lambda_1,
      \dotsc, \Lambda_m; \Upsilon_1, \dotsc, \Upsilon_k], C'$
      \begin{itemize}[leftmargin=\thirdm]
         \item $\feq{\Omega}{\Omega'}$;
         \item $\Xi = \Xi''$;
         \item $\Delta = \Delta''$.
      \end{itemize}
   \end{itemize}
\end{itemize}

\end{definition}

\begin{definition}[Well-formed comprehension match]
$\mc \Gamma; \Delta; \Xi_N; \Gamma_{N1}; \Delta_{N1}; \Xi; \Omega; C; P;
\compsz{A}{B}; \Omega_N; \Delta_N \rightarrow \outsem$ is
well-formed in relation to a triplet $A; \Gamma; \Delta_{N}$ iff:

\begin{itemize}[leftmargin=*]
   \item $P$ is composed solely of persistent frames.
   \item $C$ is composed of either linear or persistent frames, but the first
   frame is linear.
   \item $\Delta, \Xi = \Delta_{N}$
   \item $C$ and $P$ are well-formed in relation to $A; \Gamma; \Delta_{N}$ and
   follow the same rules presented before in "Well-formed body match" as a stack
   $C, P$.
\end{itemize}
\end{definition}

\begin{definition}[Well-formed aggregate match]
$\ma \Gamma; \Delta; \Xi_N; \Gamma_{N1}; \Delta_{N1}; \Xi; \Omega; C; P;
\aggsz{A}{B}{C}; \Omega_N; \Delta_N; T \rightarrow \outsem$ is
well-formed in relation to a triplet $A; \Gamma; \Delta_{N}$ iff the rules in
"Well-formed comprehension match" also apply.

\end{definition}


\subsection{Soundness of matching}

The soundness theorem will be proven into two main steps. First, we prove that
performing a rule match at LLD is sound in relation to HLD and then we prove
that the derivation of head terms in LLD is also sound.

In order to prove the soundness of matching, we want to reconstitute a valid
$\mz$in HLD from a valid $\mo$in LLD. However, LLD may fail during matching,
therefore our body match lemma needs to handle unsuccessful matches. In order to
be able to use induction, we must assume a matching proposition $\mo$that
already contains some continuation frames in stack $C$ that is well-formed in
relation to the rule's body $A$ and initial database.

Our lemma also needs to apply to our continuation judgment $\contlld$, because when inverting some of
the matching assumptions, we get a continuation proposition. Apart from an unsuccessful match, we deal
with two situations during a successful match: (1) we succeed without needing to backtrack to a frame
in stack $C$ or (2) we need to backtrack to a frame in $C$. The complete lemma is stated and proven below.

\begin{lemma}[Body match soundness]\label{thm:body_match}
   
Given a rule $A \lolli H$, consider a triplet $T = A; \Gamma; \Delta_{N}$ and a context $\Delta_{N} = \Delta_1, \Delta_2, \Xi$.

If $\mo \Gamma; \Delta_1, \Delta_2; \Xi; \Omega; H; C; R \rightarrow \outsem$ is well-formed in relation to $T$ then either:

\begin{itemize}[leftmargin=*]
   \item Match fails:
   \begin{itemize}[leftmargin=\secondm]
      \item $\cont \cdot; H; R; \Gamma \rightarrow \outsem$
   \end{itemize}

   \item Match succeeds with no backtracking to frames of stack $C$:
   \begin{itemize}[leftmargin=\secondm]
      \item $\mz \Gamma; \Xi, \Delta_2 \rightarrow A$
      \item $\mo \Gamma; \Delta_1; \Xi, \Delta_2; \cdot; H; C'', C; R
         \rightarrow \outsem$ (well-formed in relation to $T$)
      \item $\mo \Gamma; \Delta_1; \Xi, \Delta_2; \Omega; H; C; (\cdot, \Delta_N) \rightarrow \outsem$ (well-formed in relation to $T$)
   \end{itemize}

   \item Match succeeds with backtracking to a linear frame:
   \begin{itemize}[leftmargin=\secondm]
      \item $\mz \Gamma; \Xi_1, \dotsc, \Xi_m, p_2, \Xi_c \rightarrow A$
      \item $\exists_{f \in C}. f = (\Delta_a; \Delta_{b_1}, p_2, \Delta_{b_2}; p;
            \Omega_1, \dotsc, \Omega_k; \Xi_1 .. \Xi_m; \Lambda_1, \dotsc,
            \Lambda_m; \Upsilon_1, \dotsc, \Upsilon_n)$

      \item $C = C', f, C''$

      \item $f$ turns into $f' = (\Delta_a, \Delta_{b_1}, p_2;
            \Delta_{b_2}; p; \Omega_1, \dotsc, \Omega_k; \Xi_1, \dotsc, \Xi_m;
            \Lambda_1, \dotsc, \Lambda_m; \Upsilon_1, \dotsc, \Upsilon_n)$

      \item $\mo \Gamma; \Delta_c; \Xi_1, \dotsc, \Xi_m, p_2, \Xi_c; \cdot; H;
            C''', f', C''; R \rightarrow \outsem$ (well-formed in
                  relation to $T$)
      \item $\Delta_c = (\Delta_1, \Delta_2, \Xi) - (\Xi_1, \dotsc, \Xi_m, p_2, \Xi_c)$
   \end{itemize}

   \item Match succeeds with backtracking to a persistent frame:
   \begin{itemize}[leftmargin=\secondm]
      \item $\mz \Gamma; \Xi_1, \dotsc, \Xi_m, \Delta_{c_2} \rightarrow A$
      \item $\exists_{f \in C}. = f = [\Gamma_1, p_2, \Gamma_2; \Delta_{c_1}, \Delta_{c_2}; \Xi_c; \bang
         p; \Omega_1, \dotsc, \Omega_k; \Lambda_1, \dotsc, \Lambda_m;
         \Upsilon_1, \dotsc, \Upsilon_n]$
      \item $C = C', f, C''$
      \item $f$ turns into $f' = [\Gamma_2; \Delta_{c_1}, \Delta_{c_2}; \Xi_1, \dotsc,
         \Xi_m; \bang p; \Omega_1, \dotsc, \Omega_k; \Lambda_1, \dotsc,
         \Lambda_m; \Upsilon_1, \dotsc, \Upsilon_n]$
      \item $\mo \Gamma; \Delta_{c_1}; \Xi_1, \dotsc, \Xi_m, \Delta_{c_2};
         \cdot; H; C'', f', C''; R \rightarrow \outsem$ (well-formed in
            relation to $T$)
      \item $\Delta_{c_1}, \Delta_{c_2} = (\Delta_1, \Delta_2,
            \Xi) - (\Xi_1, \dotsc, \Xi_m)$
   \end{itemize}
\end{itemize}

If $\cont C; H; R; \Gamma \rightarrow \outsem$ and $C$ is well-formed in relation to $T$ then either:

\begin{itemize}[leftmargin=*]
   \item Match fails:
   \begin{itemize}[leftmargin=\secondm]
      \item $\cont \cdot; H; R; \Gamma \rightarrow \outsem$
   \end{itemize}

   \item Match succeeds with backtracking to a linear frame:
   \begin{itemize}[leftmargin=\secondm]
      \item $\mz \Gamma; \Xi_1, \dotsc, \Xi_m, p_2, \Xi_c \rightarrow A$
      \item $\exists_{f \in C}. f = (\Delta_a; \Delta_{b_1}, p_2, \Delta_{b_2}; p;
            \Omega_1, \dotsc, \Omega_k; \Xi_1 .. \Xi_m; \Lambda_1, \dotsc,
            \Lambda_m; \Upsilon_1, \dotsc, \Upsilon_n)$

      \item $C = C', f, C''$

      \item $f$ turns into $f' = (\Delta_a, \Delta_{b_1}, p_2;
            \Delta_{b_2}; p; \Omega_1, \dotsc, \Omega_k; \Xi_1, \dotsc, \Xi_m;
            \Lambda_1, \dotsc, \Lambda_m; \Upsilon_1, \dotsc, \Upsilon_n)$

      \item $\mo \Gamma; \Delta_c; \Xi_1, \dotsc, \Xi_m, p_2, \Xi_c; \cdot; H;
            C''', f', C''; R \rightarrow \outsem$ (well-formed in
                  relation to $T$)
      \item $\Delta_c = (\Delta_1, \Delta_2, \Xi) - (\Xi_1, \dotsc, \Xi_m, p_2, \Xi_c)$
   \end{itemize}

   \item Match succeeds with backtracking to a persistent frame:
   \begin{itemize}[leftmargin=\secondm]
      \item $\mz \Gamma; \Xi_1, \dotsc, \Xi_m, \Delta_{c_2} \rightarrow A$
      \item $\exists_{f \in C}. = f = [\Gamma_1, p_2, \Gamma_2; \Delta_{c_1}, \Delta_{c_2}; \Xi_c; \bang
         p; \Omega_1, \dotsc, \Omega_k; \Lambda_1, \dotsc, \Lambda_m;
         \Upsilon_1, \dotsc, \Upsilon_n]$
      \item $C = C', f, C''$
      \item $f$ turns into $f' = [\Gamma_2; \Delta_{c_1}, \Delta_{c_2}; \Xi_1, \dotsc,
         \Xi_m; \bang p; \Omega_1, \dotsc, \Omega_k; \Lambda_1, \dotsc,
         \Lambda_m; \Upsilon_1, \dotsc, \Upsilon_n]$
      \item $\mo \Gamma; \Delta_{c_1}; \Xi_1, \dotsc, \Xi_m, \Delta_{c_2};
         \cdot; H; C'', f', C''; R \rightarrow \outsem$ (well-formed in
            relation to $T$)
      \item $\Delta_{c_1}, \Delta_{c_2} = (\Delta_1, \Delta_2,
            \Xi) - (\Xi_1, \dotsc, \Xi_m)$
   \end{itemize}
\end{itemize}
\end{lemma}

\begin{proof}
   Proof by mutual induction. In $\mo$on the size of $\Omega$ and on $\contlld$, first on the size of the second argument of the frame ($\Delta''$ and $\Gamma''$) and then on the size of the stack $C$. Sub-cases:
   
\begin{itemize}[leftmargin=*]
   \item $\mo p~\m{first}$, $\mo p~\m{on}~q$, $\mo p~\m{on}~\bang q$, $\mo \bang p~\m{first}$ $\mo \bang p~\m{on}~q$, $\mo \bang p~\m{on}~\bang q$, $\mo \otimes$
   
   When inverting the assumption, the well-formedness of the stack and match are
   proven straightforwardly using the well-formedness of the assumption and the
   match equivalence theorem. The induction hypothesis is then applied
   straightforwardly.
   
   \item $\mo \m{end}$
   
   Trivial.
   
   \item $\mo p~\m{fail}$, $\mo \bang p~\m{fail}$
   
   Invert the assumption and apply induction hypothesis on the $\cont$assumption.
   
   \item $\cont \m{next}~\m{rule}$
   
   Match fails.
   
   \item $\cont p~\m{next}, \cont \bang p~\m{next}$
   
   When inverting the assumption, we get a $\mo$proposition that is trivially
   proven to be well-formed in relation to $T$. Using induction hypothesis on
   this assumption, we have 3 sub-cases:
   
   \begin{itemize}[leftmargin=\secondm]
      \item Match fails: trivial.
      \item Match succeeds with no backtracking: the frame that we updated is the successful frame to backtrack to.
      \item Match succeeds with backtracking: $f \in C$ from the new assumption is the frame we need.
   \end{itemize}
   
   \item $\cont p~\m{no}~\m{more}$, $\cont \bang p~\m{no}~\m{more}$
   
   Invert the assumption to apply the induction hypothesis.
\end{itemize}
\end{proof}

For the induction hypothesis to be applicable in in Lemma~\ref{thm:body_match} there must be
a relation between the judgments $\mo$and $\contlld$.
We can define a lexicographic ordering $A \prec B$, meaning that proposition $A$ has a smaller proof than proposition $B$ (potentially $A$ is sub-proof of $B$),
or alternatively, $A$ is "executed later" than $B$.
The specific ordering is as follows:

\begin{enumerate}[leftmargin=*]
   \item $\cont C; H; R; \Gamma \rightarrow \outsem \prec \cont C', C; H; R; \Gamma \rightarrow \outsem$\\
   The continuation must use the top of the stack $C'$ before using $C$;
   \item $\cont C', (\Delta, \Delta_1; \Delta_2; \Xi; p; \Omega; \Lambda; \Upsilon), C; H; R; \Gamma \rightarrow \outsem$\\
   \hspace*{1cm}$\prec \cont C'', (\Delta; \Delta_1, \Delta_2; \Xi; p; \Omega; \Lambda; \Upsilon), C; H; R; \Gamma \rightarrow \outsem$\\
   A continuation frame with more candidates has more steps to do than a frame with less candidates;
   \item $\cont C', [\Gamma_1; \Delta; \Xi; \bang p; \Omega; \Lambda; \Upsilon], C; H; R; \Gamma \rightarrow \outsem$\\
   \hspace*{1cm} $\prec \cont C'', [\Gamma_2, \Gamma_3; \Delta; \Xi; \bang p; \Omega; \Lambda; \Upsilon], C; H; R; \Gamma \rightarrow \outsem$\\
      Same as the previous one;
   \item $\cont C; H; R; \Gamma \rightarrow \outsem \prec \mo \Gamma; \Delta; \Xi; \Omega; H; C', C; R \rightarrow \outsem$\\
   Same as (1);
   \item $\mo \Gamma; \Delta; \Xi; \Omega; H; C; R \rightarrow \outsem \prec \cont C', C; H; R; \Gamma \rightarrow \outsem$\\
   Same as the previous one;
   \item $\mo \Gamma; \Delta''; \Xi''; \Omega'; H; C', C; R \rightarrow \outsem \prec \mo \Gamma; \Delta; \Xi; \Omega; H; C; R \rightarrow \outsem$ as long as $\Omega' \prec \Omega$\\
   Adding continuation frames to the stack makes the proof smaller as long as $\Omega$ is also smaller; 
   \item $\mo \Gamma; \Delta; \Xi; \Omega; H; C', (\Delta, \Delta_1; \Delta_2;
         \Xi; p; \Omega; \Lambda; \Upsilon), C; R \rightarrow \outsem$\\
   \hspace*{1cm} $\prec \mo \Gamma; \Delta''; \Xi''; \Omega'; C'', (\Delta; \Delta_1, \Delta_2; \Xi; p; \Omega; \Lambda; \Upsilon), C; R \rightarrow \outsem$\\
   Same as (2);
   \item $\mo \Gamma; \Delta; \Xi; \Omega; H; C', [\Gamma_1; \Delta; \Xi; \bang p; \Omega; \Lambda; \Upsilon], C; R \rightarrow \outsem$\\
   \hspace*{1cm} $\prec \mo \Gamma; \Delta''; \Xi''; \Omega'; C'', [\Gamma_2, \Gamma_3; \Delta; \Xi; \bang p; \Omega; \Lambda; \Upsilon], C; R \rightarrow \outsem$\\
   Same as (3).
\end{enumerate}

\subsection{Soundness of derivation}

Proving that the derivation of the head of the rule is sound is trivial except
for comprehensions and aggregates. LLD deterministically computes the number of
available comprehensions to apply while HLD "guesses" the number and then
performs the derivation. In the next two sections, we show how to prove the
soundness of comprehensions and aggregates. The strategy for proving for proving
both is identical due to their inherient similarities.

\subsection{Comprehension soundness}

Proving that deriving a comprehension in LLD is sound in relation to HLD is
built from 4 results: (1) proving that matching the body of a comprehension is
sound in relation to HLD; (2) proving that updating the continuation stacks
makes them suitable for use in the next comprehension applications; (3) proving
that deriving the head of the comprehension is sound in relation to HLD; (4)
proving that we can apply as many comprehensions as the database allows.

\begin{lemma}[Comprehension body match]\label{thm:comprehension_body_match}
Given a comprehension $AB = \compsz{A}{B}$, consider a triplet $T = A; \Gamma; \Delta_{N}$ and a context $\Delta_{N} = \Delta_1, \Delta_2, \Xi$.

If $\mc \Gamma; \Delta_1, \Delta_2; \Xi_N; \Gamma_{N1}; \Delta_{N1}; \Xi;
\Omega; C; P; \compsz{A}{B}; \Omega_N; \Delta_N \rightarrow \outsem$ is well-formed in relation to $T$ then either:

   \begin{itemize}[leftmargin=*]
      \item Match fails:
      \begin{itemize}[leftmargin=\secondm]
         \item $\done \Gamma; \Delta_N; \Xi_N; \Gamma_{N1}; \Delta_{N1}; \Omega_N \rightarrow \outsem$
      \end{itemize}
      
      \item Match succeeds with no backtracking to frames of stack $C$ or $P$
      ($C \neq \cdot$):

      \begin{itemize}[leftmargin=\secondm]
         \item $\mz \Gamma; \Delta_2 \rightarrow A$
         \item $\mc \Gamma; \Delta_1; \Xi_N; \Gamma_{N1}; \Delta_{N1}; \Xi,
            \Delta_2; \cdot; C', C; P; AB; \Omega_N; \Delta_N
            \rightarrow \outsem$ (well-formed in relation to
                  $T$)
      \end{itemize}

      \item Match succeeds with no backtracking to frames of stack $P$ ($C =
            \cdot$):
      \begin{itemize}[leftmargin=\secondm]
         \item $\mz \Gamma; \Delta_2 \rightarrow A$
         \item $\mc \Gamma; \Delta_1; \Xi_N; \Gamma_{N1}; \Delta_{N1}; \Xi,
            \Delta_2; \cdot; C'; P', P; AB; \Omega_N; \Delta_N
            \rightarrow \outsem$ (well-formed in relation to
                  $T$)
      \end{itemize}

      \item Match succeeds with backtracking to a linear continuation frame in
      stack $C$ ($C \neq \cdot$):

      \begin{itemize}[leftmargin=\secondm]
         \item $\mz \Gamma; \Xi_1, \dotsc, \Xi_m, p_2, \Xi_c$
         \item $\exists_{f \in C}. f = (\Delta_a; \Delta_{b_1}, p_2,
               \Delta_{b_2}; p; \Xi_1, \dotsc, \Xi_m; \Omega_1, \dotsc,
               \Omega_k; \Lambda_1, \dotsc, \Lambda_m; \Upsilon_1, \dotsc,
               \Upsilon_n)$
         \item $C = C', f, C''$
         \item $f$ turns into $f' = (\Delta_a, \Delta_{b_1}, p_2;
               \Delta_{b_2}; p; \Xi_1, \dotsc, \Xi_m;
               \Omega_1, \dotsc, \Omega_k; \Lambda_1, \dotsc, \Lambda_m;
               \Upsilon_1, \dotsc, \Upsilon_n)$
         \item $\mc \Gamma; \Delta_c; \Xi_N; \Gamma_{N1}; \Delta_{N1}; \Xi_1,
            \dotsc, \Xi_m, p_2, \Xi_c; \cdot; C''', f', C''; P;
            AB; \Omega_N; \Delta_N \rightarrow \outsem$ (well-formed in relation to $T$)
         \item $\Delta_c = (\Delta_1, \Delta_2, \Xi) - (\Xi_1, \dotsc, \Xi_m,
               p_2, \Xi_c)$
      \end{itemize}

      \item Match succeeds with backtracking to a persistent continuation frame
      in stack $C$ ($C \neq \cdot$):
      \begin{itemize}[leftmargin=\secondm]
         \item $\mz \Gamma; \Delta_{c_2}, \Xi_1, \dotsc, \Xi_m \rightarrow A$
         \item $\exists_{f \in C}. f = [\Gamma_1, p_2, \Gamma_2; \Delta_{c_1},
            \Delta_{c_2}; \Xi_1, \dotsc, \Xi_m; \bang p; \Omega_1, \dotsc, \Omega_k;
            \Lambda_1, \dotsc, \Lambda_m; \Upsilon_1, \dotsc, \Upsilon_n]$
         \item $C = C', f, C''$
         \item $f$ turns into $f' = [\Gamma_2; \Delta_{c_1}, \Delta_{c_2};
            \Xi_1, \dotsc, \Xi_m; \bang p; \Omega_1, \dotsc, \Omega_k; \Lambda_1,
            \dotsc, \Lambda_m; \Upsilon_1, \dotsc, \Upsilon_n]$
         \item $\mc \Gamma; \Delta_{c_1}; \Xi_N; \Gamma_{N1}; \Delta_{N1};
            \Delta_{c_2}, \Xi_1, \dotsc, \Xi_m; \cdot; C''', f', C''; P;
            AB; \Omega_N; \Delta_N \rightarrow \outsem$ (well-formed in relation to $T$)
         \item $\Delta_{c_1}, \Delta_{c_2} = (\Delta_1, \Delta_2, \Xi) - (\Xi_1, \dotsc, \Xi_m)$
      \end{itemize}

      \item Match succeeds with backtracking to a persistent continuation frame
      in stack $P$ ($C = \cdot$):
      \begin{itemize}[leftmargin=\secondm]
         \item $\mz \Gamma; \Delta_{c_2}, \Xi_1, \dotsc, \Xi_m \rightarrow A$
         \item $\exists_{f \in P}. f = [\Gamma_1, p_2, \Gamma_2; \Delta_{c_1}, \Delta_{c_2};
      \Xi_1, \dotsc, \Xi_m; \bang p; \Omega_1, \dotsc, \Omega_k; \Lambda_1,
      \dotsc, \Lambda_m; \Upsilon_1, \dotsc, \Upsilon_n]$
         \item $P = P', f, P''$
         \item $f$ turns into $f' = [\Gamma_2; \Delta_{c_1},
            \Delta_{c_2}; \Xi_1, \dotsc, \Xi_m; \bang p; \Omega_1, \dotsc,
            \Omega_k; \Lambda_1, \dotsc, \Lambda_m; \Upsilon_1, \dotsc, \Upsilon_n]$
         \item $\mc \Gamma; \Delta_{c_1}; \Xi_N; \Gamma_{N1}; \Delta_{N1};
         \Delta_{c_2}, \Xi_1, \dotsc, \Xi_m; \cdot; C'; P''', f', P'';
         AB; \Omega_N; \Delta_N \rightarrow \outsem$
         (well-formed in relation to $T$)
         \item $\Delta_{c_1}, \Delta_{c_2} = (\Delta_1, \Delta_2, \Xi) - (\Xi_1, \dotsc,
               \Xi_m)$
      \end{itemize}
   \end{itemize}
   
If $\contc \Gamma; \Delta_{N}; \Xi_{N}; \Gamma_{N1}; \Delta_{N1}; C; P;
AB; \Omega_N \rightarrow \outsem$ and $C$ and $P$ are
well-formed in relation to $T$ then either:

\begin{itemize}[leftmargin=*]
   \item Match fails:
   \begin{itemize}[leftmargin=\secondm]
      \item $\done \Gamma; \Delta_N; \Xi_N; \Gamma_{N1}; \Delta_{N1}; \Omega_N \rightarrow \outsem$
   \end{itemize}

   \item Match succeeds with backtracking to a linear continuation frame in
   stack $C$ ($C \neq \cdot$):

   \item Match succeeds: $\mz \Gamma; \Delta_x \rightarrow A$ (where
         $\Delta_x \subseteq \Delta_N$) and either:
   \begin{itemize}[leftmargin=\secondm]
      \item $\mz \Gamma; \Xi_1, \dotsc, \Xi_m, p_2, \Xi_c$
      \item $\exists_{f \in C}. f = (\Delta_a; \Delta_{b_1}, p_2,
            \Delta_{b_2}; p; \Xi_1, \dotsc, \Xi_m; \Omega_1, \dotsc,
            \Omega_k; \Lambda_1, \dotsc, \Lambda_m; \Upsilon_1, \dotsc,
            \Upsilon_n)$
      \item $C = C', f, C''$
      \item $f$ turns into $f' = (\Delta_a, \Delta_{b_1}, p_2;
            \Delta_{b_2}; p; \Xi_1, \dotsc, \Xi_m;
            \Omega_1, \dotsc, \Omega_k; \Lambda_1, \dotsc, \Lambda_m;
            \Upsilon_1, \dotsc, \Upsilon_n)$
      \item $\mc \Gamma; \Delta_c; \Xi_N; \Gamma_{N1}; \Delta_{N1}; \Xi_1,
         \dotsc, \Xi_m, p_2, \Xi_c; \cdot; C''', f', C''; P;
         AB; \Omega_N; \Delta_N \rightarrow \outsem$ (well-formed in relation to $T$)
      \item $\Delta_c = (\Delta_1, \Delta_2, \Xi) - (\Xi_1, \dotsc, \Xi_m,
            p_2, \Xi_c)$
   \end{itemize}

   \item Match succeeds with backtracking to a persistent continuation frame
   in stack $C$ ($C \neq \cdot$):
   \begin{itemize}[leftmargin=\secondm]
      \item $\mz \Gamma; \Delta_{c_2}, \Xi_1, \dotsc, \Xi_m \rightarrow A$
      \item $\exists_{f \in C}. f = [\Gamma_1, p_2, \Gamma_2; \Delta_{c_1},
         \Delta_{c_2}; \Xi_1, \dotsc, \Xi_m; \bang p; \Omega_1, \dotsc, \Omega_k;
         \Lambda_1, \dotsc, \Lambda_m; \Upsilon_1, \dotsc, \Upsilon_n]$
      \item $C = C', f, C''$
      \item $f$ turns into $f' = [\Gamma_2; \Delta_{c_1}, \Delta_{c_2};
         \Xi_1, \dotsc, \Xi_m; \bang p; \Omega_1, \dotsc, \Omega_k; \Lambda_1,
         \dotsc, \Lambda_m; \Upsilon_1, \dotsc, \Upsilon_n]$
      \item $\mc \Gamma; \Delta_{c_1}; \Xi_N; \Gamma_{N1}; \Delta_{N1};
         \Delta_{c_2}, \Xi_1, \dotsc, \Xi_m; \cdot; C''', f', C''; P;
         AB; \Omega_N; \Delta_N \rightarrow \outsem$ (well-formed in relation to $T$)
      \item $\Delta_{c_1}, \Delta_{c_2} = (\Delta_1, \Delta_2, \Xi) - (\Xi_1, \dotsc, \Xi_m)$
   \end{itemize}

   \item Match succeeds with backtracking to a persistent continuation frame
   in stack $P$ ($C = \cdot$):
   \begin{itemize}[leftmargin=\secondm]
      \item $\mz \Gamma; \Delta_{c_2}, \Xi_1, \dotsc, \Xi_m \rightarrow A$
      \item $\exists_{f \in P}. f = [\Gamma_1, p_2, \Gamma_2; \Delta_{c_1}, \Delta_{c_2};
   \Xi_1, \dotsc, \Xi_m; \bang p; \Omega_1, \dotsc, \Omega_k; \Lambda_1,
   \dotsc, \Lambda_m; \Upsilon_1, \dotsc, \Upsilon_n]$
      \item $P = P', f, P''$
      \item $f$ turns into $f' = [\Gamma_2; \Delta_{c_1},
         \Delta_{c_2}; \Xi_1, \dotsc, \Xi_m; \bang p; \Omega_1, \dotsc,
         \Omega_k; \Lambda_1, \dotsc, \Lambda_m; \Upsilon_1, \dotsc, \Upsilon_n]$
      \item $\mc \Gamma; \Delta_{c_1}; \Xi_N; \Gamma_{N1}; \Delta_{N1};
      \Delta_{c_2}, \Xi_1, \dotsc, \Xi_m; \cdot; C'; P''', f', P'';
      AB; \Omega_N; \Delta_N \rightarrow \outsem$
      (well-formed in relation to $T$)
      \item $\Delta_{c_1}, \Delta_{c_2} = (\Delta_1, \Delta_2, \Xi) - (\Xi_1, \dotsc,
            \Xi_m)$
   \end{itemize}
   
\end{itemize}
\end{lemma}

Proving that matching the body of a comprehension is sound in relation to HLD
follows the structure of the Lemma~\ref{thm:body_match}. The lemma uses mutual
induction on the recursive judgments $\mc$and $\contc$and considers the three
possible results of matching: failure, success with no backtracking and success
with backtracking.

In order to apply a comprehension again, we need to reuse the continuation
stacks. However, in order to use $C$ and $P$ safely, we need to prove that $C$
will have at most one updated linear continuation frame and $P$ will have all
its frames updated to account the consumption of the facts from the previous
application of the comprehension.

We first prove some auxiliary theorems.

\begin{theorem}[Full stack update]\label{thm:stack_update}
If $\strans \Xi; P; P'$ then $P'$ will be the transformation of stack $P$ where
every frame $f \in P$, where $f = [\Gamma'; \Delta_N; \cdot; \bang p; \Omega; \cdot;
      \Upsilon])$, will turn into $f' = [\Gamma'; \Delta_N - \Xi; \cdot;
      \bang p; \Omega; \cdot; \Upsilon]$, where $f' \in P'$.
\end{theorem}
\begin{proof}
Straightforward induction on the size of $P$.
\end{proof}

\begin{theorem}[From update to derivation]\label{thm:from_update_to_derivation}
If $\fix \Gamma; \Xi_N; \Gamma_{N1}; \Delta_{N1}; \Xi; C; P; AB; \Omega_N;
\Delta_N \rightarrow \outsem$ then\\
\texttab$\dc \Gamma; \Xi_N, \Xi;
\Gamma_{N1}; \Delta_{N1}; B; C' ; P'; AB; \Omega_N; (\Delta_N - \Xi) \rightarrow
\outsem$, where:

\begin{itemize}[leftmargin=*]
   \item If $C = \cdot$ then $C' = \cdot$

   \item If $C = C_1, (\Delta_a; \Delta_b; \cdot; p; \Omega; \cdot; \Upsilon)$
   then $C' = (\Delta_a - \Xi; \Delta_b - \Xi; \cdot; p; \Omega; \cdot;
         \Upsilon)$

   \item $P'$ is the transformation of stack $P$, where for every frame $f \in
   P$ of the form $[\Gamma'; \Delta_N; \cdot; \bang p; \Omega; \cdot; \Upsilon]$
   will turn into $f' = [\Gamma';\Delta_N-\Xi;\cdot;\bang p;\Omega;\cdot;\Upsilon]$

\end{itemize}
\end{theorem}
\begin{proof}
Use induction on the size of the stack $C$ to get the result of $C'$ and then
apply Theorem~\ref{thm:stack_update} to get $P'$.
\end{proof}


Now we prove that a match of a comprehension's body implies the start of a
derivation of the comprehension's head with correct continuation stacks. Note
that $\Omega = \cdot$ in $\matchlldc$, so there is nothing left to match.

\begin{corollary}[Match to derivation]\label{thm:match_to_derivation}
If $\mc \Gamma; \Delta; \Xi_N; \Gamma_{N1}; \Delta_{N1}; \Xi; \cdot; B; C; P;
AB;\Omega_N; \Delta_N \rightarrow \outsem$ then\\
\texttab$\dc \Gamma; \Xi_N, \Xi; \Gamma_{N1}; \Delta_{N1}; B; C'; P'; AB; \Omega_N; (\Delta_N - \Xi) \rightarrow \outsem$ where:
   
\begin{itemize}[leftmargin=*]
   \item If $C = \cdot$ then $C' = \cdot$
   \item If $C = C_1, (\Delta_a; \Delta_b; \cdot; p; \Omega; \cdot; \Upsilon)$ then $C' = (\Delta_a - \Xi; \Delta_b - \Xi; \cdot; p; \Omega; \cdot; \Upsilon)$ then \linebreak $C' = (\Delta_a - \Xi; \Delta_b - \Xi; \cdot; p; \Omega; \cdot; \Upsilon)$
   \item $P'$ is the transformation of stack $P$, where for every frame $f \in
   P$ of the form $[\Gamma'; \Delta_N; \cdot; \bang p; \Omega; \cdot; \Upsilon]$
   will turn into $f' = [\Gamma';\Delta_N-\Xi;\cdot;\bang p;\Omega;\cdot;\Upsilon]$
\end{itemize}
\end{corollary}

\begin{proof}
Invert the assumption and then apply Theorem~\ref{thm:from_update_to_derivation}.
\end{proof}


\paragraph{Comprehension Derivation}

We also need to prove that deriving the head of a comprehension is sound in
relation to HLD.  With the results of the next theorem we can reuse the
continuation stacks to start the comprehension process all over again, but now
with a non-empty continuation stack.

\begin{theorem}[Comprehension derivation soundness]\label{thm:comprehension_derivation}
If $\dc \Gamma; \Delta; \Xi_N; \Gamma_{N1}; \Delta_{N1}; \Omega_1, \dotsc, \Omega_n; C; P;
AB; \Omega_N; \Delta_N \rightarrow \outsem$ then:

\begin{itemize}[leftmargin=*]
   \item $\dc \Gamma; \Delta; \Xi_N; \Gamma_{N1}, \Gamma_1, \dotsc, \Gamma_n; \Delta_{N1},
   \Delta_1, \dotsc, \Delta_n; \cdot; C; P; AB; \Omega_N; \Delta_N \rightarrow
   \outsem$;

   \item $\forall_{\Omega_x}.($ if $\dz \Gamma; \Delta; \Xi_N;
   \Gamma_{N1}, \Gamma_1, \dotsc, \Gamma_n; \Delta_{N1}, \Delta_1, \dotsc,
   \Delta_n; \Omega_x \rightarrow \outsem$ then

   $\dz \Gamma; \Delta; \Xi_N; \Gamma_{N1}; \Delta_{N1}; \Omega_1, \dotsc,
   \Omega_n, \Omega_x \rightarrow \outsem)$.

\end{itemize}
\end{theorem}

\begin{proof}
Straightforward induction on $\Omega_1, \dotsc, \Omega_n$.
\end{proof}

The second result of this theorem is the soundness result we need because it will allow us to reconstruct the derivation tree in HLD.


\paragraph{Multiple Comprehension Derivation} We are interested in proving that
if we start with a given comprehension match $\matchlldc$ then we can apply the
comprehension several times.

\begin{theorem}[Multiple comprehension derivation]\label{thm:multiple_comprehension_derivation}
Consider a triplet $T = A; \Gamma; \Delta_{N}$ and a comprehension $AB =
\compsz{A}{B}$. Assume that there exists $n \geq 0$ applications of $AB$
where the $i_{th}$ application is related to the following contexts:
\begin{description}
   \item[$\Delta_i$]: context of derived linear facts;
   \item[$\Gamma_i$]: context of derived persistent facts;
   \item[$\Xi_i$]: context of consumed linear facts.
\end{description}

Since each application consumes $\Xi_i$ then the initial context $\Delta_N =
\Delta, \Xi_1, \dotsc, \Xi_n$. We now define the two main implications of the
theorem.

\begin{itemize}[leftmargin=*]
   \item Assume that $\Delta_N = \Delta_a, \Delta_b$, $\Delta_b =
   \Delta'_b, p_1$ and there is a frame $f = (\Delta_a, p_1; \Delta'_b; \cdot;
         p; \Omega; \cdot; \Upsilon)$.

   If $\mc \Gamma; \Delta_a, \Delta'_b; \Xi_N; \Gamma_{N1}; \Delta_{N1}; p_1;
   \Omega; f; P; AB; \Omega_N; \Delta, \Xi_1, \dotsc, \Xi_n \rightarrow \outsem$ (well-formed in relation to $T$) then:

   \begin{itemize}[leftmargin=\secondm]
      \item $n$ comprehensions are derived:\\
      $\done \Gamma; \Delta_N; \Xi_N, \Xi_1, \dotsc, \Xi_n; \Gamma_{N1},
      \Gamma_1, \dotsc, \Gamma_n; \Delta_{N1}, \Delta_1, \dotsc, \Delta_n; \Omega_N \rightarrow \outsem$
      \item $n$ $\mz$propositions for the $n$ comprehension matches:
      \begin{itemize}[leftmargin=\thirdm]
         \item $\mz \Gamma; \Xi_1 \rightarrow A$
         \item $\dots$
         \item $\mz \Gamma; \Xi_n \rightarrow A$
      \end{itemize}
      \item $n$ derivation implications for HLD: \\
      $\forall_{\Omega_x}.($ if $\dz \Gamma; \Delta, \Xi_{i+1}, \dotsc, \Xi_{n}; \Xi_N, \Xi_1,
            \dotsc, \Xi_i; \Gamma_{N1}, \Gamma_1, \dotsc, \Gamma_i; \Delta_{N1},
            \Delta_1, \dotsc, \Delta_i; \Omega_x \rightarrow \outsem$ then $\dz \Gamma; \Delta, \Xi_{i+1}, \dotsc, \Xi_{n}; \Xi_N, \Xi_1,
            \dotsc,
            \Xi_i; \Gamma_{N1}, \Gamma_1, \dotsc, \Gamma_{i-1}; \Delta_{N1},
            \Delta_1, \dotsc, \Delta_{i-1}; B, \Omega_x \rightarrow \outsem)$
   \end{itemize}

   \item If $\mc \Gamma; \Delta_N; \Xi_N; \Gamma_{N1}; \Delta_{N1}; \cdot; \Omega;
      \cdot; P; AB; \Omega_N; \Delta, \Xi_1, \dotsc, \Xi_n \rightarrow \outsem$ (well-formed in relation to $T$) then:

   \begin{itemize}[leftmargin=\secondm]
      \item $n$ comprehensions are derived:\\
      $\done \Gamma; \Delta_N; \Xi_N, \Xi_1, \dotsc, \Xi_n; \Gamma_{N1},
      \Gamma_1, \dotsc, \Gamma_n; \Delta_{N1}, \Delta_1, \dotsc, \Delta_n; \Omega_N \rightarrow \outsem$

      \item $n$ $\mz$propositions for the $n$ comprehension matches:
      \begin{itemize}[leftmargin=\thirdm]
         \item $\mz \Gamma; \Xi_1 \rightarrow A$
         \item \dots
         \item $\mz \Gamma; \Xi_n \rightarrow A$
      \end{itemize}

      \item $n$ derivation implications for HLD: \\
      $\forall_{\Omega_x}.($ if $\dz \Gamma; \Delta, \Xi_{i+1}, \dotsc, \Xi_{n}; \Xi_N, \Xi_1,
            \dotsc, \Xi_i; \Gamma_{N1}, \Gamma_1, \dotsc, \Gamma_i; \Delta_{N1},
            \Delta_1, \dotsc, \Delta_i; \Omega_x \rightarrow \outsem$ then $\dz \Gamma; \Delta, \Xi_{i+1}, \dotsc, \Xi_{n}; \Xi_N, \Xi_1,
            \dotsc,
            \Xi_i; \Gamma_{N1}, \Gamma_1, \dotsc, \Gamma_{i-1}; \Delta_{N1},
            \Delta_1, \dotsc, \Delta_{i-1}; B, \Omega_x \rightarrow \outsem)$
   \end{itemize}

\end{itemize}
   
\end{theorem}
\begin{proof}

By mutual induction, first on either the size of $\Delta'_b$ (second argument of
the linear continuation frame) or $\Gamma'$ (second argument of the
persistent frame in $P$) and then on the size of $C, P$.  We only show
how to prove the first implication since the second implication is proven
in a similar way.

$\mc \Gamma; \Delta_a, \Delta'_b; \Xi_N; \Gamma_{N1}; \Delta_{N1}; p_1;
\Omega; f; P; AB; \Omega_N; \Delta, \Xi_1, \dotsc, \Xi_n \rightarrow \outsem$ \hfill (1) assumption\\

By applying Lemma~\ref{thm:comprehension_body_match} to (1), we get:

\begin{itemize}[leftmargin=*]
   \item Failure:
   
   $\done \Gamma; \Delta_N; \Xi_N; \Gamma_{N1}; \Delta_{N1}; \Omega_N
   \rightarrow \outsem$ \hfill (2) from lemma, thus $n = 0$\\
   
   \item Success with no backtracking to frames of stack $C$ or $P$:
   
      $\mz \Gamma; \Xi_1 \rightarrow A$ \hfill (2) from lemma \\

      $\Xi_1 = \Xi'_1, p_1$ \hfill (3) from the well-formedness of (1) \\
      $f = (\Delta_a, p_1; \Delta'_b; \cdot; p; \Omega; \cdot; \Upsilon)$ \\

      $\mc \Gamma; \Delta, \Xi_2, \dotsc, \Xi_n; \Xi_N; \Gamma_{N1};
            \Delta_{N1}; p_1, \Xi'_1; \cdot; C', f; P; AB; \Omega_N; \Delta_N \rightarrow
            \outsem$ \\
      \dots \hfill (4) from lemma (1) \\

      $f' = (\Delta_a, p_1 - \Xi_1; \Delta_b - \Xi_1; \cdot; p; \Omega; \cdot;
            \Upsilon)$ \\

      $\dc \Gamma; \Xi_N, \Xi_1; \Gamma_{N1}; \Delta_{N1}; B; f'; P'; AB;
            \Omega_N; \Delta, \Xi_2, \dotsc, \Xi_n \rightarrow \outsem$ \\
      \dots \hfill (5) using Corollary~\ref{thm:match_to_derivation} on (4) \\

      $\dc \Gamma; \Xi_N, \Xi_1; \Gamma_{N1}, \Gamma_1; \Delta_{N1}, \Delta_1;
            \cdot; f'; P; AB; \Omega_N; \Delta, \Xi_2, \dotsc, \Xi_n \rightarrow \outsem$
      \\ \dots \hfill (6) applying Theorem~\ref{thm:comprehension_derivation} on (5)

      $\forall_{\Omega_x}. ($ if $\dz \Gamma; \Delta, \Xi_2, \dotsc, \Xi_n; \Xi_N, \Xi_1;
            \Gamma_{N1}, \Gamma_1; \Delta_{N1}, \Delta_1; \Omega_x \rightarrow
            \outsem$ then \\ \hspace*{0.5cm} $\dz \Gamma;
            \Delta, \Xi_2, \dotsc, \Xi_n; \Xi_N, \Xi_1; \Gamma_{N1}; \Delta_{N1}; B, \Omega_x
            \rightarrow \outsem)$ \hfill (7) from
      Theorem~\ref{thm:comprehension_derivation} on (5) \\

      $\contc \Gamma; \Delta, \Xi_2, \dotsc, \Xi_n; \Xi_N, \Xi_1; \Gamma_{N1},
         \Gamma_1; \Delta_{N1}, \Delta_1; f'; P'; AB; \Omega_N
         \rightarrow \outsem$\\ \dots \hfill (8) inversion of (6) \\
        
        By inverting (8) we either fail (thus $n = 1$) or we get a new match.
        For the latter case, we apply mutual induction to get the remaining $n -
        1$ comprehensions.
      
   \item With backtracking to the linear continuation frame of stack $C$:
      
      $\mz \Gamma; \Xi_1 \rightarrow A$ \hfill (2) from lemma \\

      $f = (\Delta_a, p_1; \Delta'_b; \cdot; p; \Omega; \cdot; \Upsilon)$ \hfill (3) frame to backtrack to \\
      turns into $f' = (\Delta_a, p_1, \Delta'''_b, p_2; \Delta''_b; \cdot; p; \Omega; \cdot; \Upsilon)$ \hfill (4) resulting frame \\

      $\mc \Gamma; \Delta, \Xi_2, \dotsc, \Xi_n; \Xi_N; \Gamma_{N1};
\Delta_{N1}; p_2, \Xi'_1; \cdot; C', f'; P; AB; \Omega_N; \Delta_N \rightarrow
\outsem$\\ \dots \hfill (5) from lemma (1) \\
      
      Use the same approach as the case with no backtracking.
      
   \item With backtracking to a persistent continuation frame of stack $P$:

      $\mz \Gamma; \Xi_1 \rightarrow A$ \hfill (2) from lemma \\

      $f = [\Gamma''_1, p_2, \Gamma''_2; \Delta_N; \cdot; \bang p; \Omega; \cdot; \Upsilon]$ \hfill (4) from theorem \\
      turns into $f' = [\Gamma''_2; \Delta_N; \cdot; \bang p; \Omega; \cdot;
      \Upsilon]$ \hfill (5) from theorem \\

      $\mc \Gamma; \Delta, \Xi_2, \dotsc, \Xi_n; \Xi_N; \Gamma_{N1};
\Delta_{N1}; \Xi_1; \cdot; C'; P', f', P''; AB; \Omega_N; \Delta_N \rightarrow
\outsem$ \\ \dots \hfill (6) from theorem \\
         
      Use the same approach as the case with no backtracking.
      
\end{itemize}
\end{proof}

For this theorem, we derive three important propositions for HLD: (1) the final
derivation proposition; (2) the matching propositions for each comprehension
application; (2) derivation implications to get from (1) to a derivation
judgment without any derivations of the comprehension. However, the theorem
starts from an initial stack with frames and the comprehension process starts
with an empty stack. We need another theorem that gives us one application of
the comprehension plus the other $n$ that we get from this theorem.

\begin{lemma}[All comprehensions]\label{thm:comprehension}
Consider a triplet $T = A; \Gamma; \Delta_{N}$ and a comprehension $AB =
\compsz{A}{B}$. Assume that there exists $n \geq 0$ applications of $AB$
where the $i_{th}$ application is related to the following contexts:
\begin{description}
   \item[$\Delta_i$]: context of derived linear facts;
   \item[$\Gamma_i$]: context of derived persistent facts;
   \item[$\Xi_i$]: context of consumed linear facts.
\end{description}

Since each application consumes $\Xi_i$ then the initial context $\Delta_N =
\Delta, \Xi_1, \dotsc, \Xi_n$.

If $\mc \Gamma; \Delta, \Xi_1, \dotsc, \Xi_n;
\Xi_N; \Gamma_{N1}; \Delta_{N1}; \cdot; A; \cdot; \cdot; AB; \Omega_N;
\Delta, \Xi_1, \dotsc, \Xi_n \rightarrow \outsem$ (well-formed in
relation to $T$) then:

\begin{itemize}[leftmargin=*]
   \item $n$ comprehensions are derived:\\
   $\done \Gamma; \Delta_N; \Xi_N, \Xi_1, \dotsc, \Xi_n; \Gamma_{N1},
   \Gamma_1, \dotsc, \Gamma_n; \Delta_{N1}, \Delta_1, \dotsc, \Delta_n; \Omega_N \rightarrow \outsem$
   \item $n$ $\mz$propositions for the $n$ comprehension matches:
   \begin{itemize}[leftmargin=\secondm]
      \item $\mz \Gamma; \Xi_1 \rightarrow A$
      \item $\dots$
      \item $\mz \Gamma; \Xi_n \rightarrow A$
   \end{itemize}
   \item $n$ derivation implications for HLD: \\
   $\forall_{\Omega_x}.($ if $\dz \Gamma; \Delta, \Xi_{i+1}, \dotsc, \Xi_n; \Xi_N, \Xi_1,
         \dotsc, \Xi_i; \Gamma_{N1}, \Gamma_1, \dotsc, \Gamma_i; \Delta_{N1},
         \Delta_1, \dotsc, \Delta_i; \Omega_x \rightarrow \outsem$ then $\dz \Gamma; \Delta, \Xi_{i+1}, \dotsc, \Xi_n; \Xi_N, \Xi_1,
         \dotsc,
         \Xi_i; \Gamma_{N1}, \Gamma_1, \dotsc, \Gamma_{i-1}; \Delta_{N1},
         \Delta_1, \dotsc, \Delta_{i-1}; B, \Omega_x \rightarrow \outsem)$
\end{itemize}
\end{lemma}

\begin{proof}
Apply Lemma~\ref{thm:comprehension_body_match} to get two sub-cases:
   
\begin{itemize}[leftmargin=*]
   \item Match fails:
   
   
   $\done \Gamma; \Delta_N; \Xi_N; \Gamma_{N1}; \Delta_{N1}; \Omega_N
   \rightarrow \outsem$\\
   \dots \hfill (1) no comprehension application was possible ($n = 0$)\\
   
   \item Match succeeds:
   
   $\mc \Gamma; \Xi_2, \dotsc, \Xi_n; \Xi_N; \Gamma_{N1}; \Delta_{N1}; \Xi_1; \cdot; C; P; AB; \Omega_N; \Delta_N \rightarrow \outsem$
   
   \dots \hfill (1) result from Lemma~\ref{thm:comprehension_body_match}
   
   $\mz \Gamma; \Xi_1 \rightarrow A$
   \hfill (2) also from Lemma~\ref{thm:comprehension_body_match}
   
   $\dc \Gamma; \Xi_N, \Xi_1; \Gamma_{N1}; \Delta_{N1}; B; C'; P'; AB;
   \Omega_N; \Delta, \Xi_2, \dotsc, \Xi_n \rightarrow \outsem$
   
   \dots \hfill (3) applying Corollary~\ref{thm:match_to_derivation} on (1)
   
   $\dc \Gamma; \Xi_N, \Xi_1; \Gamma_{N1}, \Gamma_1; \Delta_{N1}, \Delta_1;
   \cdot; C'; P'; AB; \Omega_N; \Delta, \Xi_2, \dotsc, \Xi_n \rightarrow \outsem$
   
   \dots \hfill (4) using Theorem~\ref{thm:comprehension_derivation} on (3)\\
   
   $\forall_{\Omega_x}. ($ if $\dz \Gamma; \Delta, \Xi_2, \dotsc, \Xi_n; \Xi_N, \Xi_1; \Gamma_{N1}, \Gamma_1; \Delta_{N1}, \Delta_1; \Omega_x \rightarrow \outsem$ then
   
    \hspace*{0.5cm} $\dz \Gamma; \Delta, \Xi_2, \dotsc, \Xi_n; \Xi_N, \Xi_1; \Gamma_{N1};
    \Delta_{N1}; B, \Omega_x \rightarrow \outsem)$ \\ \dots \hfill (5)
   from the theorem applied in (4)\\
   
   $\contc \Gamma; \Delta, \Xi_2, \dotsc, \Xi_n; \Xi_N, \Xi_1; \Gamma_{N1},
   \Gamma_1; \Delta_{N1}, \Delta_1; C'; P'; AB; \Omega_N \rightarrow \outsem$
   
   \dots \hfill (6) inversion of (5)\\
   
   Invert (6) to get either $n = 1$ application of the comprehension or apply Theorem~\ref{thm:multiple_comprehension_derivation} to the inversion to get the remaining $n-1$. 
\end{itemize}
\end{proof}

If the previous lemma, the comprehension is applied for as many times as the
database allows. We now have to map these $n$ applications to HLD by rebuilding
the proof tree for these $n$ matches and derivations and then using
$n$ when "guessing" the number of iterative definitions in HLD.

\subsection{Soundness of derivation}

We are finally ready to prove that the derivation of terms of the head of a rule
is sound in relation to HLD.

\begin{lemma}[Head derivation soundness]\label{thm:head_derivation_soundness}
If $\done \Gamma; \Delta_N; \Xi_N; \Gamma_{N1}; \Delta_{N1}; \Omega \rightarrow \outsem$ then
$\dz \Gamma; \Delta_N; \Xi_N; \Gamma_{N1}; \Delta_{N1}; \Omega \rightarrow \outsem$.
\end{lemma}

\begin{proof}\label{sec:derivation_theorem} Induction on $\Omega$. Most of the
sub-cases can be proven using the induction hypothesis or by straightforward
rule inference. The sub-cases for the comprehensions and aggregates are
complicated and are proved beflow.

\paragraph{Comprehensions} Apply Lemma~\ref{thm:comprehension} on the assumption
to get $n$ applications of the comprehension. Assume that 
$\Delta_N = \Delta, \Xi_1, \dotsc, \Xi_n$, where $\Xi_i$ are the facts consumed
and $\Gamma_i, \Delta_i$ the facts derived by the $i^{th}$ application.
The lemma proves the following:

\begin{itemize}[leftmargin=*]
   \item $\done \Gamma; \Delta; \Xi_N, \Xi_1, \dotsc, \Xi_n; \Gamma_{N1},
   \Gamma_1, \dotsc, \Gamma_n; \Delta_{N1}, \Delta_1, \dotsc, \Delta_n;
\Omega_N \rightarrow \outsem$ \hfill (1)
   \item $n$ propositions $\mz \Gamma; \Xi_i \rightarrow A$ \hfill (2)
   \item $n$ implications\\
   $\forall_{\Omega_x}.($ if $\dz \Gamma; \Delta, \Xi_{i+1}, \dotsc,
         \Xi_{n}; \Xi_N, \Xi_1,
         \dotsc, \Xi_i; \Gamma_{N1}, \Gamma_1, \dotsc, \Gamma_i; \Delta_{N1},
         \Delta_1, \dotsc, \Delta_i; \Omega_x \rightarrow \outsem$ then $\dz \Gamma; \Delta, \Xi_{i+1}, \dotsc, \Xi_n; \Xi_N, \Xi_1,
         \dotsc,
         \Xi_i; \Gamma_{N1}, \Gamma_1, \dotsc, \Gamma_{i-1}; \Delta_{N1},
         \Delta_1, \dotsc, \Delta_{i-1}; B, \Omega_x \rightarrow \outsem)$ \hfill (3) \\
\end{itemize}

\noindent From (1) we apply the inductive hypothesis since $\Omega$ gets
smaller:\\
$\dz \Gamma; \Delta; \Xi_N, \Xi_1, \dotsc, \Xi_n; \Gamma_{N1}, \Gamma_1,
\dotsc, \Gamma_n; \Delta_{N1}, \Delta_1, \dotsc, \Delta_n; \Omega \rightarrow
\outsem$ \\

\noindent Since we are building the proof tree backwards, starting from the final
derivation result, we first need to derive $\compz{0}{A}{B}$ by applying rules
$\dz \one$ and $\dz \m{comp}^0$:\\
$\dz \Gamma; \Delta; \Xi_N, \Xi_1, \dotsc, \Xi_n; \Gamma_{N1}, \Gamma_1,
\dotsc, \Gamma_n; \Delta_{N1}, \Delta_1, \dotsc, \Delta_n; \compz{0}{A}{B}, \Omega \rightarrow
\outsem$
\\

\noindent From result (4), we first rebuild the matching and derivation process of the
$n^{th}$ comprehension.  Using the $n^{th}$ implication (3) on (5):

\noindent $\dz \Gamma; \Delta, \Xi_n; \Xi_N, \Xi_1, \dotsc, \Xi_{n-1}; \Gamma_{N1}, \Gamma_1,
\dotsc, \Gamma_{n-1}; \Delta_{N1}, \Delta_1, \dotsc, \Delta_{n-1}; B, \compz{0}{A}{B},
\Omega \rightarrow \outsem$ \\

\noindent Using $\dz \lolli$ and the matching proposition (2) on (6), the $A \lolli B$
implication is reconstructed:

\noindent $\dz \Gamma; \Delta, \Xi_n; \Xi_N, \Xi_1, \dotsc, \Xi_{n-1}; \Gamma_{N1},
   \Gamma_1, \dotsc, \Gamma_{n-1}; \Delta_{N1}, \Delta_1, \dotsc, \Delta_{n-1};
A \lolli B, \compz{0}{A}{B}, \Omega \rightarrow \outsem$ \\

\noindent Now, $\compz{1}{A}{B}$ is rebuilt by applying $\dz \otimes$ followed by $\dz
\m{comp}^N$:

\noindent $\dz \Gamma; \Delta, \Xi_n; \Xi_N, \Xi_1, \dotsc, \Xi_{n-1}; \Gamma_{N1},
\Gamma_1, \dotsc, \Gamma_{n-1}; \Delta_{N1}, \Delta_1, \dotsc, \Delta_{n-1};
\compz{1}{A}{B}, \Omega \rightarrow \outsem$ \\

\noindent Steps (5) through (8) are then applied $n-1$ times to get:

\noindent $\dz \Gamma; \Delta, \Xi_1, \dotsc, \Xi_n; \Xi_N; \Gamma_{N1}; \Delta_{N1};
\compz{n}{A}{B}, \Omega \rightarrow \outsem$ \\

\noindent Finally, to construct the conclusion and finish the proof, $\dz \m{comp}^*$ needs to
be applied:

\noindent $\dz \Gamma; \Delta, \Xi_1, \dotsc, \Xi_n; \Xi_N; \Gamma_{N1}; \Delta_{N1};
\compsz{A}{B}, \Omega \rightarrow \outsem$ \\

\noindent This finishes the sub-case for comprehensions.

\end{proof}

\section{Summary}

In this chapter we presented the proof theoretic foundations of LM.  First, we
introduced \fragment, the linear logic fragment that supports LM. We then
presented HLD, the high level dynamic semantics that was created by interpreting
the linear logic fragment using focusing and chaining. Next, we designed LLD,
the low level dynamic semantics that mimics the execution of rules in our
virtual machine minus small details.  Finally, we proved that LLD is sound
in relation to HLD, thus showing a connection from LLD to \fragment.



\chapter{Coordination}\label{chapter:coordination}
LM has no natural matching of nodes and computation to threads since nodes are a
program abstraction and part of the program's logic. We view the set of nodes as
a graph data structure $G = (V, E)$ with nodes $V$ and edges $E$ where threads
$T$ perform work. A thread is able to process any node in $V$, although a node
cannot be computed by more than one thread at the same time. It is possible to
specify many scheduling policies in order to compute the program in $G$.

The original Meld language was implemented as an ensemble programming language,
targeting modular robotic systems such as
Claytronics~\cite{ashley-rollman-derosa-iros07wksp}. In such systems, there is a
natural matching between computation and processing units, since each robot is
represented by a node. This distribution of data leaves little choice to be made
in the division of computation to the various nodes.

For LM, we have decided to partition the nodes $G$ into $T$ sub-graphs, that
are then assigned and processed by a thread. In the case where the subgraph of a
given thread has no more work available (no more rules to run) then the thread
is allowed to steal nodes from another thread and update its own sub-graph.

However, there are many scheduling details that are left undefined. How should a
thread schedule the computation of its sub-graph? Is node stealing beneficial to
all programs? What is the best sub-graph partitioning for a given LM program?
The answer to all these questions is \emph{coordination}, a mechanism that we
introduce to allow the programmer to specify custom scheduling and node
partitioning policies. This is an important functionality because LM uses linear
logic and thus the order in which nodes are scheduled can impact the performance
and even the results of the program. Note that when using classical logic (or
persistent logic), the computation order does not matter and the end result is
always the same since the program is strictly monotonic.

\section{Rationale}

In order to justify the introduction of coordination, we present the Single
Source Shortest Path~(SSSP), a concise program that can take advantage of custom
scheduling policies to improve its performance. The SSSP program starts (lines
1-3) with the declaration of the predicates. The first predicate, \texttt{edge},
is a persistent predicate that describes the relationship between the nodes of
the graph, where the third argument represents the weight of the edge.  The
program computes the shortest distance from node \texttt{@1} to all other nodes
in the graph. Every node has a \texttt{shortest} fact that is improved with new
\texttt{relax} facts.  Lines 5-9 declare the axioms of the program:
\texttt{edge} facts describe the graph; \texttt{shortest(A, +00, [])} is the
initial shortest distance (infinity) for all nodes; and \texttt{relax(@1, 0,
   [@1])} starts the algorithm by setting the distance from \texttt{@1} to
\texttt{@1} to be 0.

\begin{figure}[ht]
\begin{Verbatim}[numbers=left]
type route edge(node, node, int).
type linear shortest(node, int, list int).
type linear relax(node, int, list int).

!edge(@1, @2, 3). !edge(@1, @3, 1).
!edge(@3, @2, 1). !edge(@3, @4, 5).
!edge(@2, @4, 1).
shortest(A, +00, []).
relax(@1, 0, [@1]).

shortest(A, D1, P1), D1 > D2, relax(A, D2, P2)
   -o shortest(A, D2, P2),
      {B, W | !edge(A, B, W) | relax(B, D2 + W, P2 ++ [B])}.

shortest(A, D1, P1), D1 <= D2, relax(A, D2, P2)
   -o shortest(A, D1, P1).
\end{Verbatim}
\caption{Single Source Shortest Path program code.}
\label{code:shortest_path_program}
\end{figure}

\begin{figure}
\begin{center}
   \begin{subfigure}[b]{0.4\textwidth}
      \includegraphics[width=\textwidth]{figures/sssp/shortest2}
   \end{subfigure}
   \begin{subfigure}[b]{0.4\textwidth}
      \includegraphics[width=\textwidth]{figures/sssp/shortest3}
   \end{subfigure}
   \begin{subfigure}[b]{0.4\textwidth}
      \includegraphics[width=\textwidth]{figures/sssp/shortest8}
   \end{subfigure}
\end{center}
\caption{Graphical representation of the SSSP program. (a) represents the
   program after propagating initial distance at node \texttt{@1}, followed by
   (b) where the first rule is applied in node \texttt{@2}. (c)
   represents the state of the final program, where all the shortest paths
   have been computed.}
\label{fig:shortest_path_program}
\end{figure}

The first rule of the program (lines 11-14) reads as following: if the current
\texttt{shortest} path \texttt{P1} with distance \texttt{D1} is larger than a
new path \texttt{relax} with distance \texttt{D2}, then replace the current
shortest path with \texttt{D2}, delete the new \texttt{relax} path and propagate
new paths to the neighbors (lines 13-14).  The comprehension iterates over the
edges of node \texttt{A} and derives a new \texttt{relax} fact for each node
\texttt{B} with the distance \texttt{D2 + W}, where \texttt{W} is the weight of
the edge. For example, in Fig.~\ref{fig:shortest_path_program}~(a) we apply rule
1 in node \texttt{@1} where two new \texttt{relax} facts are derived at node
\texttt{@2} and \texttt{@3}. Fig.~\ref{fig:shortest_path_program}~(b) is the
result after applying the same rule but at node \texttt{2}.

The second rule of the program (lines 16-17) retracts a \texttt{relax} fact
that has a longer distance than the current shortest distance stored in
\texttt{shortest}.

There are many opportunities for custom scheduling in the SSSP program. For
instance, after applying rule 1 in Fig.~\ref{fig:shortest_path_program}~(a), it
is possible to either apply rules in either node \texttt{@2} or node
\texttt{@3}. This decision depends largely on implementation factors such as
node partitioning and number of threads in the system.  Still, it is easy to
prove that no matter the scheduling used, the final result, as presented in
Fig.~\ref{fig:shortest_path_program}~(c), is achieved.

The SSSP program is concise and declarative but its performance depends on the
order in which nodes are executed. If nodes with greater distances are
prioritized over other nodes, the program will generate more \texttt{relax}
facts since it will take longer to reach the shortest distances. From
Fig.~\ref{fig:shortest_path_program}, it is clear that the best scheduling is
the following: \texttt{@1}, \texttt{@3}, \texttt{@2} and then \texttt{@4}, where
only 4 \texttt{relax} facts are generated. If we had decided to process nodes in
order \texttt{@1}, \texttt{@2}, \texttt{@4}, \texttt{@3}, \texttt{@4},
\texttt{@2}, then 6 \texttt{relax} facts would have been generated.  The optimal
solution for SSSP is to schedule the node with the shortest distance, which is
essentially the Dijkstra shortest path algorithm~\cite{Dijkstra}. Note how it is
possible to change the nature of the algorithm by simply changing the order of
node computation, but still retain the declarative nature of the program.

\section{Types of Facts}

LM introduces the concept of coordination that allows the programmer to write
code that changes how the runtime system schedules and partitions node across
threads of execution. Beyond the distinction between linear and persistent
facts, LM further classifies facts into 3 categories: \emph{computation} facts,
\emph{structural} facts and \emph{coordination} facts.
Predicates are also classified accordingly.

Computation facts are regular facts used to represent the program state. In
Fig.~\ref{code:shortest_path_program}, \texttt{relax} and \texttt{shortest} are
all computation facts.

Structural facts describe information about the connections between the nodes in
the graph.  In the example of Fig.~\ref{code:shortest_path_program},
\texttt{edge} facts are structural since the corresponding \texttt{edge}
predicate is marked as a \texttt{route} predicate. Note that structural facts
can also be seen as computation facts since they are heavily used in the program
logic.

\emph{Coordination facts} allow the programmer to change how the run time
schedules nodes and how it partitions the nodes among threads of execution.
Coordination facts can be used in either the body of the rule, the head of the
rule or both.  This allows scheduling and partition decisions to be made based
on the state of the program and on the state of the underlying machine.  In this
fashion, we keep the language declarative because we reason logically about the
state of execution, without the need to introduce extra-logical operators into
the language that would introduce significant issues when proving properties
about programs.

Coordination facts are further classified into two kinds of facts:
\emph{sensing} and \emph{action} facts. Sensing facts are used to sense
information about the underlying runtime system, including node placement and
node scheduling.  Action facts are used to apply a coordination operations on
the runtime system.

%%%%%%%%%%%%%%%%%%%%%%%%%%%%%%%%%%%%%%%%%%%%%%%%%%%%%%%%%%%%%%%%%%%%%%%%%%%

Sensing facts are facts about the current state of the runtime system, such as
the placement of nodes in the CPU and scheduling information. In the original
Meld, sensing facts were used to get information about the outside world, like
temperature, touch data, neighborhood status, etc.

Action facts are linear facts which are consumed when the corresponding action
is performed.  In the original Meld, they were used to make the robots perform
actions in the outside world.  For LM we use them to change information about
the program state in the user interface. For example, when we want to change the
color of nodes or the label of edges, we just derive a new action fact and the
action is performed in the interface.  A more important use of action facts is
to change the order in which nodes are evaluated in the runtime system. It is
possible to give hints to the virtual machine in order to prioritize the
computation of some nodes.

With sensing facts and action facts, we can write \emph{meta-rules} that will
sense the state of the runtime system and then apply decisions in order to
improve execution speed.  In some situations, this set of rules can be added to
the program without any modifications to the original rules.

\section{Scheduling Facts}

We can use action facts to change the order in which nodes are evaluated by
adding \emph{priorities}. Node priority works at the worker level so that each
worker can make a local decision about which node of the graph to run next.
Note that, by default, nodes are picked arbitrarily (a FIFO approach is used).

The following list presents the action facts available to manipulate the
scheduling decisions of the system:

\begin{description}

   \item[\texttt{type linear action set-priority(node, float)}]: This sets the
      priority of a node. If the node already has some priority, we only change
      the priority if the new one is higher priority. The programmer can decide
      if priorities are to be ordered in ascending or descending order.

   \item[\texttt{type linear action add-priority(node, float)}]: This gets the
      current node's priority and increases or decreases it.

   \item[\texttt{type linear action schedule-next(node)}]: The work will fetch
      the highest priority node's priority $P$ from its set of nodes and set the
      action's argument node's priority as $P + 1.0$. If using the priorities in
      ascending order, we pick the lowest priority and subtract $1.0$.

   \item[\texttt{type linear action unset-priority(node)}]: Removes the
      priority, if any, of a given node.

   \item[\texttt{type linear action stop-program(node)}]: Immediately stops the
      execution of the whole program.

\end{description}

When the highest priority node is picked up for execution, its priority is reset
to 0 (the default priority value). This means that the programmer must set the
node's priority again if he wants to prioritize that node.

We intend to add more action facts in the near future. For example, we want the
programmer to be able to place specific nodes in workers. This will permit good
use of memory locality by forcing certain computations to be performed in the
same worker.

Sensing facts provide information about node placement and node priority. We can
use those facts to express coordination policies. LM provides the following two
sensing facts:

\begin{description}

   \item[\texttt{type linear cpu-id(node, node, int)}]: The third argument
      indicates the worker's ID where the node of the second argument is
      currently running.

   \item[\texttt{type linear priority(node, node, float)}]: The third argument
      is the current priority of the node in the second argument.

\end{description}

Note that when sensing facts are consumed, they are re-derived automatically,
except if the programmer explicitly tells the compiler otherwise. 

\subsection{Global Directives}

We also provide a few global coordination statements:

\begin{description}

   \item[\texttt{priority @order ORDER.}] \texttt{ORDER} can be either
      \texttt{asc} or \texttt{desc}. This defines if node's are to be selected
      by the smallest or the greatest priority, respectively.

   \item[\texttt{priority @initial P.}] The \texttt{initial} statement informs
      the runtime system that all nodes must start with priority $P$.
      Alternatively, the programmer can define an \texttt{set-priority(A, P)}
      axiom.

   \item[\texttt{priority @static.}] The \texttt{static} priority tells the
      runtime system that the partition of nodes among workers is to be used
      until the end of program. 

   %\item[\texttt{priority @cluster TYPE.}] Define what type of graph clustering
   %to use. \texttt{TYPE} can be either \texttt{static}, \texttt{bfs} or
   %\texttt{random}.

\end{description}

\section{Partitioning Facts}

\section{Programs}

To better understand how coordination directives work, we next present some programs that
take advantage of them.

\subsection{Belief Propagation}

\section{Advantages of Coordination}

Randomized and approximation algorithms can obtain significant benefits from
coordination directives because although the final program results will not be
exact, they follow important statistical properties and can be computed faster.
Examples of such programs are PageRank~\cite{Lubachevsky:1986:CAA:4904.4801} and
Loopy Belief Propagation~\cite{Gonzalez+al:aistats09paraml}.

\section{Summary}

In this chapter we presented the current set of coordination directives,
implemented as sensing and action facts. The use of such facilities allows the
programmer to write derivation rules that change how the runtime system
schedules computation thus improving the executing time and possibly the final
program results. As future work, we intend to extend the set of available
directives and write additional programs using coordination.


\chapter{Thread Linear Meld}
In the previous chapter, we introduced new language facilities that can be used
by the programmer to coordinate execution. While these facilities retain the
implicit parallelism of the language, they do not allow the programmer to fully
reason about the underlying parallel architecture since the only reasoning
allowed relates to node partitioning and movement between threads. In principle,
it should be advantageous to reason about thread state, that is, to perform rule
inference about facts stored on each thread and allow threads to communicate and
coordinate between them depending on their current state. This would introduce a
kind of explicit parallelism into the implicit parallel model of LM.
However, this explicit parallelism should remain declarative in order to be easy
sto prove properties about the thread's state.

\section{Motivational Example: Graph Searching}
Consider the problem of checking if a set of nodes $S$ in a graph $G$ is
reachable from an arbitrary node $N$. An obvious solution to this problem is to
start at $N$, gather all the neighbor nodes into a list and then recursively
visit all those reachable nodes, until $S$ is covered. This reduces to a problem
of performing a breadth or depth-first search on graph $G$. However, this
solution is sequential and does not have much concurrency.  An alternative
solution to the problem is to recursively propagate the search to all neighbors
and aggregate the results in the node where the search started.  The code for
this later solution is shown in Fig.~\ref{code:threads:reach_simple}.

\begin{figure}[h]
\begin{Verbatim}[numbers=left,fontsize=\codesize,commandchars=*\#\&]
type int id.*hfill// Type declaration
type list int reach-list.

type edge(node, node).*hfill// Predicate declaration
type value(node, int).
type linear search(node, id, reach-list).
type linear do-search(node, id, node, reach-list).
type linear lookup(node, id, reach-list, int Val).
type linear new-lookup(node, id, int Val).
type linear visited(node, id).

search(A, Id, ToReach)*label#line:threads:reach_lookup1&*hfill// Rule 1: initialize search
   -o do-search(A, Id, A, ToReach),
      lookup(A, Id, ToReach, []).*label#line:threads:reach_lookup2&

lookup(A, Id, ToReach, Found), new-lookup(A, Id, Val)*hfill// Rule 2: new reachable node found
   -o lookup(A, Id, remove(ToReach, Val), [Val | Found]).

do-search(A, Id, Node, ToReach),*label#line:threads:reach_visit1&*hfill// Rule 3: node has already seen this search
visited(A, Id) // Already visited for this search.
   -o visited(A, Id). *label#line:threads:reach_visit2&

do-search(A, Id, Node, ToReach),*hfill// Rule 4: node found and propagate search
!value(A, Val), Val in ToReach*hfill// New node was found.
   -o visited(A, Id),*label#line:threads:reach_visit_visited1&
      new-lookup(Node, Id, Val),
      {B | !edge(A, B) -o do-search(B, Id, Node, remove(ToReach, Val))}.*label#line:threads:reach_propagate&

do-search(A, Id, Node, ToReach),*hfill// Rule 5: node not found and propagate search
!value(A, Val), ~ Val in ToReach*hfill// Not the node we are looking for.
   -o {B | !edge(A, B) -o do-search(B, ID, Node, ToReach)},*label#line:threads:reach_propagate2&
      visited(A, Id).*label#line:threads:reach_visit_visited2&
\end{Verbatim}

\caption{LM code to perform reachability checking on a graph.}
\label{code:threads:reach_simple}
\end{figure}

Each distinct reachability search is represented by a number (\code{Id}) and a
\code{search} axiom. Associated to each search \code{Id} is a list of nodes to
reach.  The predicate \code{visited} marks nodes that have been already
participated in search, while predicate \code{do-search} is used to propagate a
specific search. The first rule
(lines~\ref{line:threads:reach_lookup1}-\ref{line:threads:reach_lookup2}) starts
a particular search by deriving a \code{do-search} and an \code{lookup} fact.
The \code{lookup} fact is used as an accumulator and is stored in the starting
node. The third rule
(lines~\ref{line:threads:reach_visit1}-\ref{line:threads:reach_visit2}) avoids
visiting the same node twice in the presence of a \code{visited} fact.  This
visited fact is derived in the next two rules
(lines~\ref{line:threads:reach_visit_visited1}
and~\ref{line:threads:reach_visit_visited2}).  If the node where the search is
being performed is in the set of nodes we want to reach (\code{ToReach}) then we
remove the node value from the list and propagate the search to the neighbor
nodes (line~\ref{line:threads:reach_propagate}).  Otherwise, the search is
propagated but no value is removed from \code{ToReach}.

As an example, consider Fig.~\ref{fig:threads:reach_example}, which shows 2
reachability checks on a graph with 10 nodes. For instance, the search with
\code{Id = 0} starts at node \code{@1} and checks if nodes \code{@1}, \code{@2},
and \code{@3} are reachable from \code{@1}. Since \code{@1} is the starting
node, \code{1} is immediately removed from the reachable list, including the
propagated \code{do-search} facts but also the \code{lookup} fact that is stored
at node \code{@1}. Once \code{do-search} reaches node \code{@3}, the value
\code{3} is removed from the list and a new \code{do-search} is propagated to
node \code{@1} (not shown in the figure) and \code{@2}. At the same time, node
\code{@2} receives the list \code{[2,3]}, removes \code{2} and propagates
\code{[3]} to node \code{@3} and \code{@1}. Node \code{@1} receives two
\code{new-lookup} facts, one from \code{@3} and another from \code{@2}, due to
successful searches and the \code{lookup} fact becomes \code{lookup(@1,0,[],
[1,2,3])}.

The attentive reader will notice that node \code{@1} already knows that all the
nodes have been reached and that nodes \code{@7} and \code{@4} will, needlessly,
continue to check if \code{[2,3]} are reachable. This is an issue that arises
because the programmer has valued concurrency by increasing redundancy and
reducing communication between nodes. It would be prohibitly expensive to share
reachability information between nodes. An alternative solution is to store the
results of the search on the thread performing the search and then coordinate
the results with other threads since the number of threads is usually smaller
than the number of nodes. Before showing how the reachability program is solved
using thread-based facts, we first present the changes required in the language.

\begin{figure}[ht]
\begin{center}
   \includegraphics[width=0.9\linewidth]{figures/threads/reach.pdf}
\end{center}

\caption{Performing reachability checks on a graph using nodes \code{@1}
(\code{Id = 0}) and \code{@6} (\code{Id = 1}).  Search with \code{Id = 0} wants
to reach nodes \code{@1}, \code{@2}, and \code{@3} from node \code{@1}. Since
\code{@1} is part of the target nodes, the fact \code{do-search} propagated to
neighbor nodes does not include \code{1}.}

\label{fig:threads:reach_example}
\end{figure}

\subsection{Language Changes}

In the previous chapter, we presented coordination facts such as
\code{thread-id} and \code{set-thread} that already bring some awareness about
the underlying parallel system. Furthermore, such facts also introduce the
\code{thread} type for predicate arguments, which refers to a thread in the
system that is related to a core in a multi core processor. We now introduce the
concept of \emph{thread facts}, which are logical facts stored at the thread
level, meaning that, each thread is now an entity with its own logical facts.
The type \code{thread} is also now the type of the first argument of
\emph{thread predicates}, indicating that the predicate is related and is to be
stored in a specific thread. We also view the available threads as forming a
separate graph from the data graph, a graph of the processing units which are
operating on the data graph.

The introduction of thread facts increases the expressiveness of the system in
the sense that it is now possible to write inference rules that reason about the
state of the threads. This creates optimization opportunies since we can now
write algorithms with global information stored in the thread, while keeping the
LM language fully declarative. Moreover, threads are now allowed to explicitly
communicate with each other, and in conjunction with coordination predicates,
enable the writing of complex scheduling policies.

We discriminate between two new types of inference rules. The first type is the
\emph{thread rule} and has the form \code{a(T), b(T) -o c(T)}, and can be read
as: if thread \code{T} has fact \code{a(T)} and \code{b(T)} then derive fact
\code{c(T)}. The second type is the \emph{mixed rule} and has the form
\code{a(T), d(N) -o e(N)} and can be read as: if thread \code{T} is executing
node \code{N} and has the fact \code{a(T)} and node \code{N} has the fact
\code{d(N)} then derive \code{e(N)} at node \code{N}. Thread rules reason
solely at the thread level, while the mixed rules allow reasoning about both
thread and node facts. Logically, the mixed rule uses an extra fact
\code{running(T, N)}, which indicates that thread \code{T} is currently
executing node \code{N}. The \code{running} fact is implicitly retracted and
asserted every time the thread selects a different node for execution. This
makes our implementation efficient since a thread does not need to look for
nodes that match mixed rules and it is then the scheduling of the program that
drives the matching of such rules.

\subsection{Graph Of Threads}

Figure~\ref{fig:coord:thread_facts} represents a schematic view of the two graph
data structures of a program with three threads: thread $T1$ is executing node
\code{@5}, $T2$ is executing node \code{@4}, and $T3$ is executing node
\code{@3}. Note that every thread has access to its own facts and to the node
facts.

\begin{figure}[ht]
   \includegraphics[width=0.6\linewidth]{figures/threads/threads.pdf}

   \caption{An example program being executed with three threads. Note that each
      threads has a \code{running} fact that stores the node currently being
   executed.}

   \label{fig:coord:thread_facts}
\end{figure}

We added several helpful predicates that allow the programmer to inspect the
graph of threads and reason about the state of computation as it relates to
threads:

\begin{itemize}
   \item \code{\bang thread-list(T, L)}: Fact instantiated in all threads where
      \code{L} is a list of all threads executing in the system.

   \item \code{\bang other-thread(T1, T2)}: Connects thread \code{T1} to all the
      other threads \code{T2} executing in the system. Note that in
      Fig.~\ref{fig:coord:thread_facts}, we use \code{\bang other-thread} fact
      to specify the graph of threads.

   \item \code{\bang leader-thread(T, TLeader)}: Fact instantiated in all
      threads where \code{TLeader} refers to a selected thread (usually the
      first thread in \code{L} of \code{\bang thread-list(T, L)}).

   \item \code{running(T, A)}: Used to retrieve the current node \code{A}
      running on thread \code{T}.
\end{itemize}

With the exception of \code{running}, every other fact is added at the beginning
of the program as a persistent fact.

\subsection{Reachability With Thread Facts}

We know update the graph reachability program presented in
Fig.~\ref{code:threads:reach_simple} to use thread facts in order to avoid
needless searches on the graph. The search process is still done concurrently as
before, but the search state is now stored in each thread, allowing the thread
to store partial results and coordinate with other threads. The code for this new
version is shown in Fig.~\ref{code:threads:reach_threads}.

Lines~\ref{line:threads:reacht_start1}-\ref{line:threads:reacht_start2} start
the search process by assigning a thread \code{Owner} to search \code{Id} using
the persistent fact \code{\bang thread-list} which contains the list of all
available threads in the system. Next, in
line~\ref{line:threads:reacht_threads}, a fact \code{thread-search} is created
for all threads using a comprehension. We use predicate \code{do-search} to
propagate the search through the graph and a predicate \code{visited} to mark
nodes already processed for a specific search.  The two rules in
lines~\ref{line:threads:reacht_check1}-\ref{line:threads:reacht_check2}
propagate the search process to the neighbor nodes and check if the current node
is part of the list of nodes we want to reach.

An interesting property of this version is that each owner thread responsible
for a search keeps track of the remaining nodes that need to be reached. In
line~\ref{line:threads:reacht_remove}, we derive \code{remove-thread-search} in
order to inform owner threads about new reachable nodes. Once an owner thread
detects that all nodes have been reached (lines
\ref{line:threads:reacht_reached1}-\ref{line:threads:reacht_reached2}), all the
other threads will know that and update their search state accordingly
(lines~\ref{line:threads:reacht_knows1}-\ref{line:threads:reacht_knows2}). When
every thread knows that all nodes were reached, they will consume
\code{do-search} facts (lines
\ref{line:threads:reacht_prune1}-\ref{line:threads:reacht_prune2}), effectively
pruning the search space.

\begin{figure}[h]
\begin{Verbatim}[numbers=left,fontsize=\codesize,commandchars=*\#\&]
search(A, Id, ToReach),*label#line:threads:reacht_start1&*hfill// Rule 1: initialize search
*textbf#!thread-list(T, L)&, Owner = nth(L, Id % @threads)*hfill// Allocate search to a thread.
   -o {T2 | *textbf#!other-thread(T, T2)& -o *textbf#thread-search(T2, Id, ToReach, Owner)&},*label#line:threads:reacht_threads&
      do-search(A, Id).*label#line:threads:reacht_start2&

*textbf#thread-search(T, Id, [], Owner)&, *label#line:threads:reacht_prune1&*hfill// Rule 2: search completed
do-search(A, Id)
   -o *textbf#thread-search(T, Id, [], Owner)&. *label#line:threads:reacht_prune2&

do-search(A, Id),*hfill// Rule 3: node already visited
visited(A, Id) // Already visited.
   -o visited(A, Id).

do-search(A, Id),*label#line:threads:reacht_check1&*label#line:bfs_join1&*hfill// Rule 4: node found
*textbf#thread-search(T, Id, ToReach, Owner)&,*label#line:threads:reacht_join2&
!value(A, Val), Val in ToReach
   -o *textbf#thread-search(T, Id, remove(ToReach, Val), Owner)&,
      *textbf#remove-thread-search(Owner, Id, Val)&,*hfill// Tell owner thread about it.*label#line:threads:reacht_remove&
      {B | !edge(A, B) -o do-search(B, Id)},
      visited(A, Id).

do-search(A, Id),*hfill// Rule 5: node not found but propagate search
*textbf#thread-search(T, Id, ToReach, Owner)&,
!value(A, Val), ~ Val in ToReach
   -o *textbf#thread-search(T, Id, ToReach, Owner)&,
      visited(A, Id),
      {B | !edge(A, B) -o do-search(B, Id)}.*label#line:threads:reacht_check2&

*textbf#remove-thread-search(T, Id, Val), thread-search(T, Id, ToReach, Owner)&*hfill// Rule 6: node found
   *textbf#-o thread-search(T, Id, remove(ToReach, Val), Owner),&
      *textbf#check-results(T, Id).&

*textbf#check-results(T, Id),&*label#line:threads:reacht_reached1&*hfill// Rule 7: search is completed
*textbf#thread-search(T, Id, [], Owner)&
   *textbf#-o thread-search(A, Id, [], Owner),&
      *textbf#{B | !other-thread(T, B) -o signal-thread(B, Id)}.&*label#line:threads:reacht_reached2&

 *textbf#check-results(T, Id),&*hfill// Rule 8: search not completed yet
 *textbf#thread-search(T, Id, ToReach, Owner), ToReach <> []&
   *textbf#-o thread-search(T, Id, ToReach, Owner).&

*textbf#signal-thread(T, Id),&*hfill// Rule 9: thread knows search is done*label#line:threads:reacht_knows1&
*textbf#thread-search(T, Id, ToReach, Owner)&
   *textbf#-o thread-search(T, Id, [], Owner).& *label#line:threads:reacht_knows2&
\end{Verbatim}
\caption{Coordinated version of the reachability checking program. Note
that \code{@threads} represent the number of threads in the system.}
\label{code:threads:reach_threads}
\end{figure}

An alternative implementation could force every thread to share its reached
nodes to all the other threads in the system. However, this would generate a lot
of traffic between threads, which would actually make the program perform worse.
Our final solution is a good trade off since it only forces threads to
coordinate when pruning can actually happen.

Figure~\ref{fig:threads:results_search} presents experimental results of the
graph reachability program using 4 different datasets. Except for Random, all
the datasets were already used in the MSSD program and were presented before.
The Random dataset is randomly generated graph with about 50000 nodes and a
million edges.  In the plots, we show the run time of the version without thread
facts (\textbf{Regular}) and the version using thread facts called
\textbf{Threads}. We also show the speedup of the \textbf{Threads} and
\textbf{Regular} versions when using the \textbf{Regular} version with 1 thread.

Our results indicate that using thread facts produces a significant reduction in
run time. This is especially true in the case of datasets with large number of
edges, since less facts are produced and propagated in the graph when the
threads know that the search has been completed. The results also show that, in
the case of the Facebook dataset in which the number of queries is the same as
the number of nodes, the use of thread facts does not produce great improvements
due to the costs of managing the reachability results on each thread's database.
These costs are related to the need to index and lookup \code{thread-search}
facts on the \code{Id} argument every time a node is inspected.

\begin{figure}[]
        \centering
        \begin{subfigure}[b]{\plotsize\textwidth}
           \includegraphics[width=\textwidth]{experiments/threads/cmp-search-facebook.png}
           \caption{Facebook has 2000 nodes and 20000 edges. The dataset makes
           2000 graph queries to 5\% of the graph's nodes.}
           \label{fig:threads:search_facebook}
        \end{subfigure}
        \spacing
        \begin{subfigure}[b]{\plotsize\textwidth}
           \includegraphics[width=\textwidth]{experiments/threads/cmp-search-twitter.png}
           \caption{Twitter has 81306 nodes and 1768149 edges. The dataset makes
           100 graph queries to 1\% of the graph's nodes.}
           \label{fig:threads:search_twitter}
        \end{subfigure} \\
        \begin{subfigure}[b]{\plotsize\textwidth}
           \includegraphics[width=\textwidth]{experiments/threads/cmp-search-random.png}
           \caption{Random is a graph with 50000 nodes and 1052674
              edges. The dataset makes 20 graph queries
              to 5\% of the graph's nodes.}
           \label{fig:threads:search_random}
        \end{subfigure}
        \spacing
        \begin{subfigure}[b]{\plotsize\textwidth}
           \includegraphics[width=\textwidth]{experiments/threads/cmp-search-pokec.png}
           \caption{Pokec has 1632803 nodes and 30622564 edges. The dataset
           makes 3 graph queries to 1\% of the graph's nodes.}
           \label{fig:threads:search_pokec}
        \end{subfigure} \\
        \caption{Measuring the performance of the graph reachability program
        when using thread facts.}
        \label{fig:threads:results_search}
\end{figure}

The graph reachability program shows how to introduce complex coordination
policies between threads by reasoning about the state of each thread. In
addition, the use of linear logic programming makes it easier to prove
properties of the program since computation is done by applying controlled
changes to the state.


\clearpage


\section{Implementation Changes}
To support thread-based facts, both the compilation and runtime system described in
Chapter~\ref{chapter:implementation} require some changes.

\subsection{Compiler}

The compiler needs to recognize rules that use thread facts. For thread rules,
the compiler checks if the rule's body is using facts from the same thread by
checking the first argument of each fact. For mixed rules, the rule's body may
use a thread \code{T} and a node \code{A} and all the node facts have to use
\code{A}, while all threads facts must use \code{T} as the first argument. If
the programmer was to retrieve either the thread or the node for the current
computation, she may use \code{running(T, A)}.

Once rules are type checked, the iteration code for thread-based facts needs to
be adapted. When a rule requires facts from the thread, it must use the data
structures from the thread. The runtime API used for inserting thread facts is
also different since they have to be added to the thread's database.

\subsection{Runtime}

In the runtime system, thread-based facts are implemented just like regular
facts. Each thread has its own database of facts and uses exactly the same data
structures for facts, as presented before for regular nodes.  The major
difference between a regular node and a thread node is that a thread node is
never put into the work queue of its thread. As shown in the updated work loop
presented in Fig.~\ref{alg:threads:work_loop}, the thread node executes
alongside the regular node when $TH.process\_node$ is called. It is also
important to note that, before a thread becomes idle, it may have potential
candidate thread rules that are now derivable because another thread has derived
thread facts in the current thread. In particular, it is entirely possible to have
programs that only deal with thread facts.

\begin{figure}
\begin{algorithm}[H]
   \KwData{Thread TH}
   \While{true}{
      $TH.work\_queue.lock()$\;
      $node \longleftarrow TH.work\_queue.pop\_node()$ \;
      $TH.work\_queue.unlock()$\;
      \uIf{$node$}{
        \underline{$TH.process\_node(node, TH.thread\_node)$}\;
      }
      \Else{
        \tcc{The thread's node may have candidate rules using incoming thread facts}
        \underline{$TH.process\_node(nil, TH.thread\_node)$}\;
        \tcc{Attempt to steal some nodes.}
         \If{$\bang TH.steal\_nodes()$}{
            $TH.become\_idle()$\;
            \While{$len(TH.work\_queue) == 0$}{
               \tcc{Try to terminate}
               \If{$TH.synchronize\_termination()$}{
                  \textbf{terminate}\;
               }
               \If{$TH.steal\_nodes()$}{
                  \tcc{Thread is still in the stealing state}
                  break\;
               }
            }
            \tcc{There's new nodes in the queue.}
            $TH.become\_active()$\;
         }
      }
 }
\end{algorithm}
\caption{Thread work loop updated to take into account thread-based facts.
New or modified code is underlined.}
\label{alg:threads:work_loop}
\end{figure}

Thread-based facts also introduce new synchronization logic in the runtime
system. For instance, when a rule derives a new thread fact on another thread,
it needs to synchronize with that thread (using the appropriate thread node
locks) to add the facts to the thread's database. When a thread is executing its
own node or a regular node, it also must lock the thread node's \emph{DB Lock}
in order to protect its data structures from being manipulated by other threads
concurrently.

Matching rules using thread facts requires special care since they may require
both facts from the regular node and from the thread's node. Before a node is
executed, the rule engine (Section~\ref{sec:implementation:rule_engine}) of the
regular node is updated to take into account the facts of the thread's node so
that mixed rules (rules that use both thread and regular facts) execute. In this
scheme, mixed rules may be unsuccessfully fired repeatedly until a node which
has matching facts gets to execute. Since the LHS of mixed rules use an implicit
\code{running(T, N)} fact, it is enough that a different node is running
to fire mixed rules as long as the running node has the required facts.
Without using an implicit \code{running} fact, the system would need to lookup
for a regular node that would successfully activate a given mixed rule. This
would be prohibitly expensive since some programs might have millions of regular
nodes.


\section{Applications}

This section presents more applications that demonstrate the usefulness and
power of thread-based facts.

\subsection{Binary Search Trees: Caching Results}
In Section~\ref{sec:language:key_value} we have presented an algorithm for
replacing a key's value in a BST dictionary. To make the program more
interesting, we consider a sequence of $n$ lookup or replace operations for
different keys in the BST (which may or may not be repeated). A single lookup or
replace has worst-case time complexity $\mathcal{O}(h)$ where $h$ is the height
of the BST, therefore performing $n$ operations takes $\mathcal{O}(h \times n)$
time.

In order to reduce the execution time of the new program, we can cache the
search and replace operations so that repeated operations become faster. Instead
of traversing the entire height of the BST, we look in the cache and send the
operation immediately to the node where the key is located. Without thread
facts, we might have cached the results at the root node, however, this is not a
scalable approach as it would introduce a serious bottleneck.

Figure~\ref{code:threads:btree_lookup_cache} shows the updated BST code with a thread
cache. We just added two more predicates, \code{cache} and
\code{cache-size}, that are facts placed in the thread and represent cached
keys and the total size of the cache, respectively. We also added three new
rules that handle the following cases:

\begin{enumerate}
      \item A key is found and is also in the cache
         (lines~\ref{line:threads:kv_rule1_start}-\ref{line:threads:kv_rule2_end})

      \item A key is found but is not in the cache
         (lines~\ref{line:threads:kv_rule2_start}-\ref{line:threads:kv_rule2_end});

      \item A key is in the cache, therefore a \code{replace} fact is
         derived in the target node
         (lines~\ref{line:threads:kv_rule3_start}-\ref{line:threads:kv_rule3_end}).

\end{enumerate}

Note that it is quite easy to extend the cache mechanism to use an LRU type
approach in order to limit the size of the cache.

\begin{figure}[ht]
\begin{Verbatim}[numbers=left,fontsize=\codesize,commandchars=*\{\}]
type linear cache(thread, node, int).
type linear cache-size(thread, int).

// (1) Key exists and is also in the cache.*label{line:threads:kv_rule1_start}
replace(A, Key, RValue),
value(A, Key, Value),
*textbf{cache(T, A, Key)}
   -o value(A, Key, RValue).
      *textbf{cache(T, A, Key)}.*label{line:threads:kv_rule1_end}

// (2) Key exists and is not in the cache.*label{line:threads:kv_rule2_start}
replace(A, Key, RValue),
value(A, Key, Value),
*textbf{cache-size(T, Total)}
   -o value(A, Key, RValue),
      *textbf{cache-size(T, Total + 1)},
      *textbf{cache(T, A, Key)}.*label{line:threads:kv_rule2_end}

// (3) Cached by the thread.*label{line:threads:kv_rule3_start}
replace(A, RKey, RValue),
*textbf{cache(T, TargetNode, RKey)}
   -o replace(TargetNode, RKey, RValue),
      *textbf{cache(T, TargetNode, RKey)}.*label{line:threads:kv_rule3_end}

replace(A, RKey, RValue),
value(A, Key, Value),
!left(A, B),
RKey < Key
   -o value(A, Key, Value),
      replace(B, RKey, RValue). // go left

replace(A, RKey, RValue),
value(A, Key, Value),
!right(A, B),
RKey > Key
   -o value(A, Key, Value),
      replace(B, RKey, RValue). // go right
\end{Verbatim}
\caption{LM program for performing lookups in a BST with a thread cache.}
\label{code:threads:btree_lookup_cache}
\end{figure}




\subsection{PowerGrid Problem}
Consider a power grid with $C$ consumers and $G$ generators. We are interested
in connecting each consumer to a single generator, but each generator has a
limited capacity and the consumer draws a certain amount of power from the
generator. A valid power grid is built in such a way that all consumers are
serviced by a generator and that no generator is being overdrawn by too many
consumers. Although consumers and generators may be connected through a complex
network, we analyze the simple case where any consumer is able to attach to any
generator.

\begin{wrapfigure}{r}{0.5\textwidth}
   \begin{center}
      \includegraphics[width=1\linewidth]{figures/threads/powergrid.pdf}
   \end{center}
   \mycap{Configuration of a powergrid with 6 consumers, 4
      generators and 2 threads, with each thread responsible for 2 generators.}
   \label{fig:threads:powergrid}
\end{wrapfigure}

A straightforward distributed implementation for the PowerGrid problem requires
that each consumer is able to connect to any generator. Once a generator
receives a connection request, it may or may not accept it. If the generator has
no power available for the new consumer, it will disconnect from it and the
consumer must select another generator. This randomized algorithm works but may
take a long time to converge, depending on the amount of power available in the
generators. Figure~\ref{code:threads:powergrid} shows the LM code for this
solution. Consumer and generators node types are declared in
lines~\ref{line:threads:pg_decl1}-\ref{line:threads:pg_decl2} using the
\code{node} declaration, allowing us to have different predicates for consumers
and generators. The \code{consumer} and \code{generator} types become a subtype
of \emph{node}, that is, $consumer <: node$ and $generator <: node$.  These
subtypes allow us to declare initial facts that only apply to either the
\code{consumer} or \code{generator} subtype.

\begin{figure}[h!]
\begin{LineCode}[commandchars=*\#\&]
node generator.*label#line:threads:pg_decl1&*hfill// Type declaration
node consumer.*label#line:threads:pg_decl2&
type linear capacity(generator, int Total, int Used).*hfill// Predicate declaration
type linear connected-to(generator, consumer, int).
type linear connected-to-list(generator, list consumer).
type power(consumer, int).
type linear disconnected(consumer).
type linear connected(consumer, generator).
type generators(consumer, list generator).
type linear fails(generator, int).
type linear random-reconnect(generator).
type linear reconnect(consumer).
type linear connect(generator, consumer, int).
type linear disconnect(consumer, generator).

fails(G, Fails), Fails > maxfails*label#line:threads:pg_recon1&*hfill// Rule 1: disconnect one consumer
   -o random-reconnect(G).

capacity(G, Total, Used), random-reconnect(G),*hfill// Rule 2: disconnect one consumer
connected-to-list(G, L), L <> [], C = nth(L, randint(length(L))),
connected-to(G, C, Power)
   -o fails(G, 0), capacity(G, Total, Used - Power),
      connected-to-list(G, remove(L, C)), disconnect(C, G).

capacity(G, Total, Used), random-reconnect(G)*hfill// Rule 3: unable to disconnect one consumer
   -o capacity(G, Total, Used), fails(G, 0).*label#line:threads:pg_recon2&

connect(G, C, Power), capacity(G, Total, Used),*label#line:threads:pg_gen1&*hfill// Rule 4: connect consumer
fails(G, Fails), connected-to-list(G, L), Used + Power <= Total
   -o capacity(G, Total, Used + Power),
      fails(G, max(Fails - 1, 0)), connected-to(G, C, Power),
      connected-to-list(G, [C | L]).*label#line:threads:pg_gen2&

connect(G, C, Power), capacity(G, Total, Used),*label#line:threads:pg_fail1&*hfill// Rule 5: unable to connect consumer
Used + Power > Total, fails(G, Fails)
   -o capacity(G, Total, Used), disconnect(C, G),
      fails(G, Fails + 1).*label#line:threads:pg_fail2&

!generators(C, L), !power(C, Power),*label#line:threads:pg_connect1&*hfill// Rule 6: connect to a generator
reconnect(C), disconnected(C),
G = nth(L, randint(num-generators))
   -o connected(C, G), connect(G, C, Power).*label#line:threads:pg_connect2&

disconnect(C, G), connected(C, G)*hfill// Rule 7: finish disconnection
   -o disconnected(C), reconnect(C).

connected-to-list(G, []). fails(G, 0).*hfill// Initial facts
disconnected(C). reconnect(C). !generators(C, all-generators).
\end{LineCode}
\mycap{LM code for the regular PowerGrid program.}
\label{code:threads:powergrid}
\end{figure}

An example PowerGrid configuration with its initial facts is presented in
Fig.~\ref{code:threads:powergrid_init}. Consumers have a persistent fact
\code{!power(A, P)}, where \code{P} is the amount of power required by the
consumer. Consumers also start with a
\code{reconnect} fact that is used in
lines~\ref{line:threads:pg_connect1}-\ref{line:threads:pg_connect2} in order to
randomly select a generator from list \code{L} in the \code{!generators(A, L)}
fact. The generators have a \code{connect-to-list(A, L)} fact that manages the
list of connected consumers. The generator fact \code{capacity(A, Total, Used)},
stores the \code{Total} capacity of the generator and the amount of power
currently being \code{Used} (\code{Used < Total} at any point in the program).

Consumers and generators complete a connection when the generator receives a
\code{connect} fact which is used in
lines~\ref{line:threads:pg_gen1}-\ref{line:threads:pg_gen2} when the generator
has enough power for the new consumer. When there is not enough power
(\code{Used + Power > Total}), the generator disconnects the consumer in
lines~\ref{line:threads:pg_fail1}-\ref{line:threads:pg_fail2}. Note that each
generator maintains a \code{fail} fact that counts the number of times the
consumers have failed to connected. If there is too many failures, then the
generator decides to disconnect one consumer already connected in
lines~\ref{line:threads:pg_recon1}-\ref{line:threads:pg_recon2}, allowing for
different combinations to happen. In Fig.~\ref{code:threads:powergrid_final} we
present the final database of the example PowerGrid configuration, which shows
that all consumers have been able to find a suitable generator.

\begin{figure}[h!]
\begin{LineCode}[commandchars=*\#\&]
const generators = [@7, @8, @9, @10].

reconnect(@1).   !generators(@1, generators).   !power(@1, 5).
reconnect(@2).   !generators(@2, generators).   !power(@2, 10).
reconnect(@3).   !generators(@3, generators).   !power(@3, 5).
reconnect(@4).   !generators(@4, generators).   !power(@4, 10).
reconnect(@5).   !generators(@5, generators).   !power(@5, 10).
reconnect(@6).   !generators(@6, generators).   !power(@6, 5).

connected-to-list(@7, []).    capacity(@7, 15, 0).   fail(@7, 0).
connected-to-list(@8, []).    capacity(@8, 15, 0).   fail(@8, 0).
connected-to-list(@9, []).    capacity(@9, 10, 0).   fail(@9, 0).
connected-to-list(@10, []).   capacity(@10, 5, 0).   fail(@10, 0).
\end{LineCode}
\mycap{Initial facts for a PowerGrid configuration of 6 consumers and 4 generators.}
\label{code:threads:powergrid_init}
\end{figure}

\begin{figure}[h!]
\begin{LineCode}[commandchars=*\#\&]
connected(@1, @7).    !power(@1, 5).
connected(@2, @7).    !power(@2, 10).
connected(@3, @8).    !power(@3, 5).
connected(@4, @8).    !power(@4, 10).
connected(@5, @9).    !power(@5, 10).
connected(@6, @10).   !power(@6, 5).

connected-to-list(@7, [@1, @2]).   connected-to(@7, @1, 5).   connected-to(@7, @2, 10).
connected-to-list(@8, [@3, @4]).   connected-to(@8, @3, 5).   connected-to(@8, @4, 10).
connected-to-list(@9, [@5]).       connected-to(@9, @5, 10).
connected-to-list(@10, [@6]).      connected-to(@10, @6, 5).
capacity(@7, 15, 15).
capacity(@8, 15, 15).
capacity(@9, 10, 10).
capacity(@10, 5, 5).
\end{LineCode}
\mycap{Final facts for a PowerGrid configuration of 6 consumers and 4 generators.}
\label{code:threads:powergrid_final}
\end{figure}

The issue with this initial implementation presented in
Fig.~\ref{code:threads:powergrid} is that it lacks a global view of the problem,
which introduces inefficiencies and more communication between consumers and
generators. A better algorithm will require a more sophisticated communication
pattern between the nodes. As we have seen before, thread local facts are an
excellent mechanism to introduce a global view of the problem without
complicating the original algorithm written in a declarative style. For our
solution, we will partition the set of generators $G$ among the threads in the
system and make each thread assume the ownership of its generators. Each thread
can then process consumers with a global view over its set of generators,
allowing the immediate assignment of consumers to generators.
Figure~\ref{fig:threads:powergrid} shows how the configuration presented
previously is adjusted to take into account the number of available threads.

The LM code using thread facts shown in Fig.~\ref{code:threads:powergridt}. It
uses four thread predicates: predicates \code{thread-connected-to} and
\code{thread-connected-to-list} assign consumers to the generators owned by the
thread; \code{thread-capacity} stores the capacity of each generator assigned to
the thread; and predicate \code{thread-total-capacity} provides a capacity
overview of all the generators owned by the thread.  The program starts with the
rule in lines~\ref{line:threads:pgt_start1}-\ref{line:threads:pgt_start2} by
moving generators to their corresponding threads using \code{set-thread}. Once
the generator is executing on the proper thread, the rule in
lines~\ref{line:threads:pgt_moved1}-\ref{line:threads:pgt_moved2} is derived
with the \code{just-moved} coordination fact and the state of the thread is
initialized. Consumers connect with the thread's generators with the rule in
lines~\ref{line:threads:pgt_con1}-\ref{line:threads:pgt_con2} by selecting the
first generator with enough power and then by updating the state of the thread.
Otherwise, if the thread does not have a suitable generator, a generator is
randomly selected using the method described before. For such cases, the thread
assigned with the selected generator will derive the rules in
lines~\ref{line:threads:pgt_gen1}-\ref{line:threads:pgt_gen2} and the thread
state is updated accordingly.

\begin{figure}[h!]
\begin{LineCode}[fontsize=\scriptsize,commandchars=*\#\&]
type linear thread-connected-to(thread, generator, consumer, int).*hfill// Predicate declaration
type linear thread-connected-to-list(thread, generator, list consumer).
type linear thread-capacity(thread, generator, int, int).
type linear thread-total-capacity(thread, int, int).
type linear start(generator).

start(G), !generator-id(G, Id)*label#line:threads:pgt_start1&
   -o set-thread(G, Id % @threads).*label#line:threads:pgt_start2&

just-moved(G), capacity(G, Total, Used), thread-total-capacity(T, TotalCapacity, TotalUsed)*label#line:threads:pgt_moved1&
   -o thread-connected-to-list(T, G, []), thread-capacity(T, G, Total, Used),
      thread-total-capacity(T, Total + TotalCapacity, Used + TotalUsed).*label#line:threads:pgt_moved2&

fails(G, Fails), Fails > maxfails
   -o random-reconnect(G).

thread-capacity(T, G, Total, Used), thread-total-capacity(T, TotalCapacity, TotalUsed),
random-reconnect(G), thread-connected-to-list(T, G, L), L <> [], C = nth(L, randint(length(L))),
thread-connected-to(T, G, C, Power), NewUsed = Used - Power
   -o fails(G, 0), thread-capacity(T, G, Total, NewUsed),
      thread-total-capacity(T, TotalCapacity, TotalUsed - Power),
      thread-connected-to-list(T, G, remove(L, C)), disconnect(C, G).

random-reconnect(G) -o fails(G, 0).

connect(G, C, Power), thread-total-capacity(T, TotalCapacity, TotalUsed),*label#line:threads:pgt_gen1&
thread-capacity(T, G, Total, Used), fails(G, Fails), thread-connected-to-list(T, G, L),
NewUsed = Used + Power, NewUsed <= Total
   -o thread-capacity(T, G, Total, NewUsed),
      thread-total-capacity(T, TotalCapacity, TotalUsed + Power),
      fails(G, max(Fails - 1, 0)), thread-connected-to(T, G, C, Power),
      thread-connected-to-list(T, G, [C | L]).*label#line:threads:pgt_gen11&

connect(G, C, Power), thread-capacity(T, G, Total, Used), Used + Power > Total, fails(G, Fails)
   -o thread-capacity(T, G, Total, Used), disconnect(C, G), fails(G, Fails + 1).*label#line:threads:pgt_gen2&

!power(C, Power), reconnect(C), disconnected(C), thread-total-capacity(T, TotalCapacity, TotalUsed),*label#line:threads:pgt_con1&
TotalNewUsed = TotalUsed + Power, NewUsed <= TotalCapacity,
thread-capacity(T, G, Total, Used), thread-connected-to-list(T, G, ConnectList),
NewUsed = Used + Power, NewUsed <= Total
   -o connected(C, G), thread-capacity(T, G, Total, NewUsed),
      thread-connected-to-list(T, G, [C | ConnectList]), thread-connected-to(T, G, C, Power),
      thread-total-capacity(T, TotalCapacity, TotalNewUsed).*label#line:threads:pgt_con2&

!generators(C, L), !power(C, Power), reconnect(C), disconnected(C), G = nth(L, randint(num-generators))*label#line:threads:pgt_conn1&
   -o connected(C, G), connect(G, C, Power).*label#line:threads:pgt_conn2&

disconnect(C, G), connected(C, G)
   -o disconnected(C), reconnect(C).

thread-total-capacity(T, 0, 0).*hfill// Initial facts
fails(G, 0).  disconnected(C).  reconnect(C).  !generators(C, all-generators).  start(G).
\end{LineCode}
\mycap{LM code for the optimized PowerGrid program.}
\label{code:threads:powergridt}
\end{figure}

The experimental results for the PowerGrid program are presented in
Fig.~\ref{fig:threads:results_powergrid}. In the plots, we show the run time of
the version without thread facts, named \textbf{Regular}, and the version using
thread facts, called \textbf{Threads}. We also show the speedup of the
\textbf{Threads} and \textbf{Regular} versions against the \textbf{Regular}
version with 1 thread.  We experimented with four datasets:

\begin{enumerate}
      \item 0.5 M C / 2000 G: Half a million consumers connected to 2000
         generators. This is the baseline dataset.

      \item 0.5 M C / 64 G: Half a million consumers connected to 64
         generators. This is similar to the previous dataset, but the number of
         generators is much smaller.

      \item 1 M C / 4000 G / Low Variability: 1 million consumers connected to
         4000 generators. The capacity of each generator has low variability so
         that each generator has a similar capacity.

      \item 1 M C / 4000 G / High Variability: 1 million consumers connected to
         4000 generators. The capacity of each generator has high variability so
         that some generators have no capacity or three times the average the
         capacity.

\end{enumerate}

All the datasets were generated randomly so that the total capacity of the
generators is about 2\% more than the required consumer capacity. The capacity
of each generator is generated randomly and, except for the last two datasets,
the capacity is between 0 and twice the average capacity (total capacity divided
by the number of generators).

The first important observation relates to the first two datasets. In the 0.5 M
C / 2000 G, the \textbf{Threads} version has more than a 150-fold speedup for 32
threads while the 0.5 M C / 64 G dataset only reaches a 70-fold speedup. Having
only 64 generators instead of 2000 makes it harder to scale the program since
there is only a few generators to distribute among threads. However, it must be
noted that the 0.5 M C / 64 G dataset performs proportionally faster when using
1 thread than the 2000 G dataset since that one thread has a small number of
generators to choose from when processing consumers. The rule that assigns
generators to consumers
(lines~\ref{line:threads:pgt_gen1}-\ref{line:threads:pgt_gen11} in
Fig.~\ref{code:threads:powergridt}) needs to
iterate through all the generators to find one with enough power to handle the
consumer, therefore, a smaller number of generators is a clear advantage over
having many generators in the case of using just 1 thread.

\begin{figure}[]
        \centering
        \begin{subfigure}[b]{\plotsize\textwidth}
           \includegraphics[width=\textwidth]{experiments/threads/cmp-powergrid-500000C2000G.png}
           \mycap{}
           \label{fig:threads:powergrid1}
        \end{subfigure}
        ~
        \begin{subfigure}[b]{\plotsize\textwidth}
           \includegraphics[width=\textwidth]{experiments/threads/cmp-powergrid-500000C64G.png}
           \mycap{}
           \label{fig:threads:powergrid2}
        \end{subfigure} \\
        \begin{subfigure}[b]{\plotsize\textwidth}
           \includegraphics[width=\textwidth]{experiments/threads/cmp-powergrid-1M4000C-low.png}
           \mycap{}
           \label{fig:threads:powergrid3}
        \end{subfigure} ~
        \begin{subfigure}[b]{\plotsize\textwidth}
           \includegraphics[width=\textwidth]{experiments/threads/cmp-powergrid-1M4000C-high.png}
           \mycap{}
           \label{fig:threads:powergrid4}
        \end{subfigure} \\
        \mycap{Measuring the performance of the PowerGrid program
        when using thread facts.}
        \label{fig:threads:results_powergrid}
\end{figure}

The second important observation relates to the last two datasets where we
experimented with a variable capacity for the generators. For the Low
Variability dataset, the consumers have identical capacities, while in the High
Variability dataset, generators have a more variable capacity, which should make
it harder for the algorithm to find a valid generator/consumer assignment.  Our
results show exactly that: the Low Variability shows a small difference between
the \textbf{Regular} and \textbf{Threads} version, while in the High Variability
dataset, the \textbf{Threads} version is much faster than the \textbf{Regular}
version. However, for the High Variability dataset, we were expecting a speedup
that was closer to the 0.5 M C / 2000 G dataset, since the number of generators
is much higher. Furthermore, if we compare the run times of the \textbf{Regular}
and \textbf{Threads} version when using 1 thread, we notice that the
\textbf{Threads} version is actually slower. As noted before, this may be due to
the fact that the LM rule for assigning generators to consumers needs to perform
a linear scan on the available generators to find a suitable generator which
then negatively impacts performance. This is a clear drawback of the logic
programming model that could potentially be solved by maintaining a sorted list
of \texttt{thread-capacity} facts.

In Table~\ref{table:threads:powergrid_stats}, we present several fact statistics
that compare the \textbf{Regular} version with the \textbf{Threads} version when
executing with multiple threads. The \textbf{\# Derived} column indicates the
number of derived facts, \textbf{\# Deleted} indicates the number of retracted
facts, while \textbf{\# Final} is the number of facts in the database after the
program terminates. The table results clearly show that using thread-based facts
results in a decrease in the number of generated facts, which is more
significant in the 0.5M C / 2000 G dataset (10-fold reduction). The table also
explains why this dataset performs much better than the 1M C / 4000 G High
Variability dataset, which only sees a 2-fold reduction in derived facts.

When comparing the number of facts derived when using a different number of
threads, the overall trend indicates that having more threads slightly increases
the number of derived facts. This is especially true for the 0.5 M C / 64 G
datasets, where twice as many facts are generated when using 32 threads when
compared to 1 thread.

\begin{table}[ht]
   \begin{center}
      \begin{tabular}{c | c || c c | c c | c c} \hline
	 \multirow{2}{*}{\textbf{Dataset}} & \multirow{2}{*}{\textbf{Threads}} & \multicolumn{2}{c|}{\textbf{\# Derived}} & \multicolumn{2}{c|}{\textbf{\# Deleted}} & \multicolumn{2}{c}{\textbf{\# Final}}\\
	 & & Regular & Threads & Regular & Threads & Regular & Threads\\ \hline \hline
\multirow{7}{*}{0.5M C / 2000 G}  & 1 &  49.7M & 8\% & 48.2M & 5\% & 2.5M & 2.5M \\
 & 2 &  48.5M & 8\% & 47.7M & 5\% & 2.5M & 2.5M \\
 & 4 &  49.4M & 8\% & 47.9M & 5\% & 2.5M & 2.5M \\
 & 8 &  49.4M & 8\% & 47.9M & 5\% & 2.5M & 2.5M \\
 & 16 &  50.8M & 8\% & 49.3M & 5\% & 2.5M & 2.5M \\
 & 24 &  49.8M & 8\% & 48.3M & 5\% & 2.5M & 2.5M \\
 & 32 &  49.3M & 9\% & 47.8M & 6\% & 2.5M & 2.5M \\
	\hline
\multirow{7}{*}{0.5M C / 64 G}  & 1 &  20.2M & 20\% & 18.5M & 13\% & 2.5M & 2.5M \\
 & 2 &  19.9M & 21\% & 18.4M & 14\% & 2.5M & 2.5M \\
 & 4 &  20.7M & 22\% & 18.5M & 15\% & 2.5M & 2.5M \\
 & 8 &  19.9M & 28\% & 18.4M & 22\% & 2.5M & 2.5M \\
 & 16 &  19.8M & 34\% & 18.3M & 28\% & 2.5M & 2.5M \\
 & 24 &  19.7M & 42\% & 18.2M & 38\% & 2.5M & 2.5M \\
 & 32 &  19.9M & 45\% & 18.4M & 40\% & 2.5M & 2.5M \\
	\hline
\multirow{7}{*}{\makecell{1M C / 4000 G \\Low Variability}}  & 1 &  9.4M & 80\% & 6.4M & 70\% & 5.1M & 5.1M \\
 & 2 &  9.4M & 79\% & 6.4M & 70\% & 5.1M & 5.1M \\
 & 4 &  9.4M & 79\% & 6.4M & 70\% & 5.1M & 5.1M \\
 & 8 &  9.4M & 80\% & 6.4M & 71\% & 5.1M & 5.1M \\
 & 16 &  9.4M & 80\% & 6.4M & 71\% & 5.1M & 5.1M \\
 & 24 &  9.4M & 81\% & 6.3M & 72\% & 5.1M & 5.1M \\
 & 32 &  9.3M & 80\% & 6.3M & 71\% & 5.1M & 5.1M \\
	\hline
\multirow{7}{*}{\makecell{1M C / 4000 G \\High Variability}}  & 1 &  16.4M & 53\% & 13.4M & 43\% & 5.1M & 5.1M \\
 & 2 &  16.5M & 53\% & 13.5M & 43\% & 5.1M & 5.1M \\
 & 4 &  16.5M & 53\% & 13.4M & 43\% & 5.1M & 5.1M \\
 & 8 &  16.5M & 54\% & 13.5M & 43\% & 5.1M & 5.1M \\
 & 16 &  16.5M & 53\% & 13.5M & 43\% & 5.1M & 5.1M \\
 & 24 &  16.5M & 54\% & 13.5M & 44\% & 5.1M & 5.1M \\
 & 32 &  16.5M & 54\% & 13.5M & 44\% & 5.1M & 5.1M \\
	\hline
\end{tabular}

   \end{center}

   \mycap{Measuring the reduction in derived facts when using thread-based
      facts. The first two Threads columns show the ratio of the
      \textbf{Threads} version over the \textbf{Regular} version. Percentages
      less than 100\% means that the \textbf{Threads} version produces fewer
   facts.}

   \label{table:threads:powergrid_stats}
\end{table}

%\clearpage


\subsection{Splash Belief Propagation}\label{sec:coordination:bp}
Approximation algorithms can obtain significant benefits from using customized
scheduling policies since they follow important statistical properties and thus
can trade correctness for faster convergence. An example of such algorithm is
the Loopy Belief Propagation (LBP)~\cite{Murphy99loopybelief}. LBP is an
approximate inference algorithm used in graphical models with cycles which
employs a sum-product message passing algorithm where nodes exchange messages
with their immediate neighbors and apply some computations to the messages
received.

\begin{figure}[h]
   \begin{center}
      \includegraphics[width=0.3\textwidth]{figures/bp/bp.pdf}
   \end{center}

   \mycap{LBP communication patterns. \code{new-neighbor-belief} facts are
   sent to the neighborhood when the node's belief value is updated. If the new
   value sent to the neighbor differs significantly from the value sent before
   the current the round, then an \code{update} fact is also sent (to the node
   above and below in this case).}

\label{fig:coordination:bp}
\end{figure}

LBP is an algorithm that maps well to the graph-based model of LM. The
original algorithm computes the belief of all nodes using several iterations
with synchronization between iterations. However, it is possible to avoid the
synchronization step, if we take advantage of the fact that LBP will converge
even when using an asynchronous approach. So, instead of computing the belief
iteratively, we keep track of all messages sent/received (and overwrite them
when we receive a new one) and recompute the belief asynchronously.
Figure~\ref{fig:coordination:bp} presents the communication patterns of the
program, while Fig.~\ref{code:coordination:bp} presents the LM code for the
asynchronous version of LBP.

\begin{figure}[ht]
\begin{LineCode}[commandchars=\*\#\&]
type list float belief.*hfill// Type declaration.

type potential(node, belief).*hfill// Predicate declaration
type edge(node, node).
type linear neighbor-belief(node, node, belief).
type linear new-neighbor-belief(node, node, belief).
type linear sent-neighbor-belief(node, node, belief).
type linear check-residual(node, float, node).
type linear belief(node, belief).
type linear update-messages(node, belief).
type linear update(node).

neighbor-belief(A, B, Belief),*label#line:coord:bp_first1&*hfill// Rule 1: update neighbor belief value
new-neighbor-belief(A, B, NewBelief)
   -o neighbor-belief(A, B, NewBelief).*label#line:coord:bp_first2&

check-residual(A, Residual, B),*label#line:coord:bp_check1&*hfill// Rule 2: check residual
Residual > bound
   -o update(B).

check-residual(A, _, _) -o 1.*label#line:coord:bp_check2&*hfill// Rule 3: check residual

update-messages(A, NewBelief),*hfill// Rule 4: compute belief to be sent to a neighbor node*label#line:coord:bp_iterate1&
   -o {B, OldIn, OldOut, Cavity, Convolved, OutMessage, Residual |
         !edge(A, B),
         neighbor-belief(A, B, OldIn),
         sent-neighbor-belief(A, B, OldOut),
         Cavity = normalize(divide(NewBelief, OldIn)),
         Convolved = normalize(convolve(global-potential, Cavity)),
         OutMessage = damp(Convolved, OldOut, damping)
         Residual = residual(OutMessage, OldOut)
         -o check-residual(A, Residual, B),
            new-neighbor-belief(B, A, OutMessage),
            neighbor-belief(A, B, OldIn),
            sent-neighbor-belief(A, B, OutMessage)}.*label#line:coord:bp_iterate2&

*label#line:coord:bp_last1&
update(A), update(A) -o update(A).*label#line:coord:bp_update&*hfill// Rule 5: prune redundant update operations

update(A),*hfill// Rule 6: initiate update operation*label#line:coord:bp_update1&
!potential(A, Potential),
belief(A, MyBelief)
   -o [sum Potential => Belief; B, Belief |*label#line:coord:bp_agg1&
         neighbor-belief(A, B, Belief) -o
         neighbor-belief(A, B, Belief) ->
         Normalized = normalizestruct(Belief),
         update-messages(A, Normalized), belief(A, Normalized)].*label#line:coord:bp_last2&*label#line:coord:bp_update2&*label#line:coord:bp_agg2&
\end{LineCode}

\mycap{LM code for the asynchronous version of the Loopy Belief Propagation
problem.}

\label{code:coordination:bp}
\end{figure}

\clearpage

Belief values are arrays of floats and are represented by \code{belief/2} facts.
The first rule (lines~\ref{line:coord:bp_first1}-\ref{line:coord:bp_first2})
updates a given neighbor belief whenever a new belief value is received. This is
the highest priority rule since we want to update the neighbor beliefs before
doing anything else. In order to store the belief values of the neighbor nodes,
we use \code{neighbor-belief/3} facts, where the second argument is the neighbor
address and the third argument is the belief value.

The last two rules (lines~\ref{line:coord:bp_last1}-\ref{line:coord:bp_last2})
update the belief value of a node. An \code{update} fact starts the process.
The first rule (line~\ref{line:coord:bp_update}) simply removes redundant
\code{update} facts and the second rule
(lines~\ref{line:coord:bp_update1}-\ref{line:coord:bp_update2}) performs the
belief update by aggregating all the neighbor belief values. The aggregate in
lines~\ref{line:coord:bp_agg1}-\ref{line:coord:bp_agg2} also derives copies of
the neighbors beliefs that need to be consumed in order to compute the belief
value that is going to be sent to the target neighbor. The aggregate uses a
custom accumulator that takes two arrays and adds the floating point numbers at
each index of the array.

The rule in lines~\ref{line:coord:bp_iterate1}-\ref{line:coord:bp_iterate2}
iterates through the neighbor belief values and sends new belief values by
performing the appropriate computations on the new belief value of the current
node and on the belief value sent previously. For each neighbor update, we also
check in lines~\ref{line:coord:bp_check1}-\ref{line:coord:bp_check2} if the
change in belief values is greater than \code{bound} (a program constant) and
then force the neighbor nodes to update their belief values by deriving
\code{update(B)}. This allows neighbor nodes to use updated neighbor values and
recompute their own belief values using more up-to-date information. The
computation of belief values will then start to converge to their true belief
values, independently of the node scheduling used.

However, if we prioritize nodes that receive new neighbor belief values with a
larger \code{Residual} then we may converge faster.
Figure~\ref{code:coordination:improved_bp} shows the fourth rule modified with a
\code{add-priority} fact, which increases the priority of neighbor nodes when
the source node has large changes in its belief value.

\begin{figure}[h!]
\begin{LineCode}[commandchars=\\\{\}]
update-messages(A, NewBelief),\hfill// Rule 4: compute belief to be sent to a neighbor node
   -o \{B, OldIn, OldOut, Cavity, Convolved, OutMessage, Residual |
         !edge(A, B),
         neighbor-belief(A, B, OldIn),
         sent-neighbor-belief(A, B, OldOut),
         Cavity = normalize(divide(NewBelief, OldIn)),
         Convolved = normalize(convolve(global-potential, Cavity)),
         OutMessage = damp(Convolved, OldOut, damping)
         Residual = residual(OutMessage, OldOut)
         -o check-residual(A, Residual, B),
            new-neighbor-belief(B, A, OutMessage),
            neighbor-belief(A, B, OldIn),
            \underline{add-priority(B, Residual)},
            sent-neighbor-belief(A, B, OutMessage)\}.
\end{LineCode}
\mycap{Extending the LBP program with priorities.}
\label{code:coordination:improved_bp}
\end{figure}


The proposed asynchronous approach has shown to be an improvement over the
synchronous version because it leads to faster convergence time. An improved
evaluation strategy is the Splash Belief
Propagation~(SBP)~\cite{Gonzalez+al:aistats09paraml}, where belief values are
computed asynchronously by first building a tree and then by updating the
beliefs of each node twice, first from the leaves to the root and then from the
root to the leaves. These \emph{splash trees} are built by starting at a node
whose belief changed the most in the last update. The trees must be built
iteratively until convergence is achieved.

In an environment with $T$ threads, it is then possible to build $T$ splash
trees concurrently. First, we partition the nodes into $T$ regions and then
assign each region to a thread. A thread is then responsible for iteratively
building splash trees on that region until convergence is reached.
Fig.~\ref{fig:threads:splash_bp} shows a grid of nodes that has been partitioned
in two regions where splash trees will be built. To build a splash tree, a
thread starts from the highest priority node (the tree's root) from its region
and then performs a breadth-first search from that node to construct the rest of
the tree. The belief values are then computed in order.

\begin{figure}[ht]
   \begin{center}
      \includegraphics[width=0.7\linewidth]{figures/threads/splash_bp}
   \end{center}
   \caption{Creating splash trees using two threads. The graph is
      partitioned into two regions and each thread is able to build separate
   splash trees starting from the highest priority node.}
   \label{fig:threads:splash_bp}
\end{figure}

The LM implementation for SBP is shown in Fig.~\ref{code:threads:sbp}. First,
in lines \ref{line:threads:splash_part1}-\ref{line:threads:splash_part2}, we
partition the nodes into regions using \code{set-thread} and then we start the
creation of the first splash tree (line~\ref{line:threads:splash_first}) by
deriving \code{start-tree(T)}.  The remaining phases of the algorithm are
explained next.

\begin{figure}[!htb]
\begin{Verbatim}[numbers=left,commandchars=*\{\},fontsize=\codesize]
type list node tree.

type linear partitioning(thread, int).*hfill// Number of nodes to receive.
type linear start-tree(thread).
type linear new-tree(thread, tree, tree).
type linear expand-tree(thread, tree).
type linear first-phase(thread, tree, tree).
type linear second-phase(thread, tree).
type linear start(node).

start(A).
partitioning(T, @world / @threads).*hfill// Move @world/@threads nodes.

!coord(A, X, Y), start(A)*hfill// Moving this node.*label{line:threads:splash_part1}
   -o set-thread(A, grid(X, Y)).
just-moved(A), partitioning(T, Left)*hfill// Thread received another node.
   -o partitioning(T, Left - 1).
partitioning(T, 0) -o start-tree(T).*label{line:threads:splash_part2}*label{line:threads:splash_first}

start-tree(T),*label{line:threads:splash_building1} priority(A, P), P > 0.0 *hfill{} // Tree building
   -o priority(A, P), expand-tree(T, [A], []).*label{line:threads:splash_building2}
expand-tree(T, [A | All], Next)
   -o thread-id(A, Id),
      [collect => L ; | !edge(A, L), ~ L in All, ~ L in Next,*label{line:threads:splash_agg1} priority(L, P), P > 0.0,
         thread-id(L, Id2), Id1 = Id2 -o priority(L, P), thread-id(L, Id2) ->
         new-tree(T, [A | All],
            if len(All) + 1 >= maxnodes then [] else Next ++ L end)].*label{line:threads:splash_agg2}*label{line:threads:splash_next}

new-tree(T, [A | All], [])
   -o schedule-next(A), first-phase(T, reverse([A | All]), [A | All]).*label{line:threads:splash_first_phase}
new-tree(T, All, [B | Next])
   -o schedule-next(B), expand-tree(T, [B | All], Next).

first-phase(T, [A], [A]), running(T, A) *hfill{} // First phase
   -o running(T, A), update(A), remove-priority(A), start-tree(T).
first-phase(T, [A, B | Next], [A]), running(T, A)
   -o running(T, A), update(A), schedule-next(B), second-phase(T, [B | Next]).*label{line:threads:splash_first_update1}
first-phase(T, All, [A, B | Next]), running(T, A)
   -o running(T, A), update(A), schedule-next(B), first-phase(T, All, [B | Next]).*label{line:threads:splash_first_update2}

second-phase(T, [A]), running(T, A) *hfill{} // Second phase
   -o running(T, A), update(A), remove-priority(A), start-tree(T).*label{line:threads:splash_second_update1}
second-phase(T, [A, B | Next]), running(T, A)
   -o running(T, A), update(A), schedule-next(B), second-phase(T, [B | Next]).*label{line:threads:splash_second_update2}
\end{Verbatim}

   \caption{LM code for the Splash Belief Propagation program.}
  \label{code:threads:sbp}
\end{figure}

\begin{description}

   \item[Tree building:] Starts after the rule in lines
      \ref{line:threads:splash_building1}-\ref{line:threads:splash_building2} is
      derived. Since the thread always picks the highest priority node, we start
      by adding that node to the list that represents the tree. In lines
      \ref{line:threads:splash_agg1}-\ref{line:threads:splash_agg2}, we use an
      aggregate to gather all the neighbor nodes that have a positive priority
      (due to a new belief update) and are in the same thread. Nodes are
      collected into list \code{L} and appended to list \code{Next}
      (line~\ref{line:threads:splash_next}).

   \item[First phase:] When the number of nodes in the tree reaches a certain
      limit, a \code{first-phase} is generated to update the beliefs of all
      nodes in the tree (line~\ref{line:threads:splash_first_phase}). As the
      nodes are updated, starting from the leaves and ending at the root, an
      \code{update} fact is derived to update the belief values
      (lines~\ref{line:threads:splash_first_update1}
      and~\ref{line:threads:splash_first_update2}).

   \item[Second phase:] Performs the computation of beliefs from the root to the
      leaves and the belief values are updated a second time
      (lines~\ref{line:threads:splash_second_update1}
      and~\ref{line:threads:splash_second_update2}).

\end{description}

SBP is also implemented in GraphLab~\cite{GraphLab2010}, a C++ framework for
writing machine learning algorithms. GraphLab provides different schedulers that
change how machine learning algorithms are computed and one of the available
schedulers is the \textbf{splash} scheduler, which implements the scheduling
described above. For comparison purposes with LM, we also experimented with two
other GraphLab schedulers: \textbf{fifo}, a first-in first-out scheduler and
\textbf{multiqueue}, a first-in first-out scheduler that also allows for
\textit{work stealing} (we used 1 queue per thread).

The default LM scheduling policy of LM is somewhat similar to both the
\textbf{fifo} and the \textbf{multiqueue} schedulers, however, since LM also
implements node stealing by default, the \textbf{multiqueue} scheduler will be
our main focus. We measured the run time of LBP and SBP for both LM and
GraphLab. For SBP, we used splash trees of 100 nodes in both systems.

Fig.~\ref{fig:threads:results_splash} compares the performance of the LBP
program in LM against both the \textbf{fifo} and \textbf{multiqueue} schedulers.
The results show that GraphLab's \textbf{fifo} scheduler performance
deteriorates with more 10 threads, while both LM and \textbf{multiqueue} tend to
scale very well and have a similar behaviour. For 1 thread, LM is about 1.5
times slower than GraphLab but that ratio increases to about to about 2 once the
number of threads increases. We think this is because the LBP program is fairly
sensitive to scheduling policies and the \textbf{multiqueue} scheduler is
better than LM's default scheduling policy.

\begin{figure}[]
        \centering
        \begin{subfigure}[b]{\plotsize\textwidth}
           \includegraphics[width=\textwidth]{experiments/threads/cmp-fifo-belief-propagation-400.png}
           \caption{Comparing the scalability of LM and of GraphLab's
              \textbf{fifo} scheduler}
           \label{fig:threads:splash_fifo}
        \end{subfigure}
        ~
        \begin{subfigure}[b]{\plotsize\textwidth}
           \includegraphics[width=\textwidth]{experiments/threads/cmp-multi-belief-propagation-400.png}
           \caption{Comparing the scalability of LM and of GraphLab's
              \textbf{multiqueue} scheduler}
           \label{fig:threads:splash_multi}
        \end{subfigure} \\
        \caption{Comparing the scalability of LM against GraphLab's (\textbf{fifo}
        and \textbf{multiqueue} schedulers.}
        \label{fig:threads:results_splash}
\end{figure}

The scalability and run time of the SBP program is compared in
Fig.~\ref{fig:threads:results_splash_ratio}(a). When compared to LM, GraphLab is
about twice as fast and that advantage is constant across the number of threads
since both systems have a similar scalability. Finally, in
Fig.~\ref{fig:threads:results_splash_ratio}(b), we compare the performance of
LBP program against the performance of SBP by calculating $LBP(t)/SBP(t)$, where
$LBP(t)$ is the run time of LBP for $t$ threads, while $SBP(t)$ is the run time
of SBP for $t$ threads. For LM, SBP improves noticeably over LBP but that
advantage is reduced as the number of threads increases. The same behavior is
also seen in GraphLab's \textbf{multiqueue} scheduler but since this scheduler
works so well under a shared memory setting, the advantages of the
\textbf{splash} scheduler are reduced. LM is able to improve its performance
with SBP since it is a slower language and reducing the number of facts derived
provides an improved performance.

\begin{figure}[]
        \centering
\end{figure}

\begin{figure}[]
        \centering
        \begin{subfigure}[b]{\plotsize\textwidth}
        \includegraphics[width=\textwidth]{experiments/threads/cmp-splash-bp-400.png}
        \caption{Measuring and comparing the performance of SBP in LM and
           the same program using GraphLab's \textbf{splash} scheduler.}
           \label{fig:threads:results_splash_final}
        \end{subfigure}~ ~
        \begin{subfigure}[b]{\plotsize\textwidth}
           \includegraphics[width=\textwidth]{experiments/threads/cmp-ratio-belief-propagation-400.png}
           \caption{Comparing the GraphLab's \textbf{splash}
           scheduler against the \textbf{fifo} and the \textbf{multiqueue}
        schedulers.}
           \label{fig:threads:splash_ratio_fifo}
        \end{subfigure}\\
        \caption{Evaluating the performance of SBP over LBP in LM and GraphLab.}
        \label{fig:threads:results_splash_ratio}
\end{figure}


\section{Modeling the Operational Semantics in LM}
The introduction of thread-based facts allows for explicit scheduling and even
explicit parallelism in a language that is purely implicit. This introduces
issues when attempting to prove the correctness of programs because the behavior
of threads and the scheduling strategy is now also part of the program logic.
Some of this behavior is hidden from programs because it is part of how
coordination facts and thread scheduling works on the virtual machine.

Consider the SBP program in Fig.~\ref{code:threads:sbp} where in
lines~\ref{line:threads:splash_part1}-\ref{line:threads:splash_part2} the graph
of nodes is partitioned into regions. In order to prove the correct
partitioning, we need to know how the VM initially randomly assigns nodes to
threads and also how coordination facts \code{set-thread} and \code{just-moved}
are used by the VM.  Fortunately, since linear logic is the foundation of LM, it
is possible to model the semantics of LM by using LM rules. In
Chapter~\ref{chapter:implementation}, we have seen that threads and nodes
transition between different states during execution and thus we are going to
model that. We first define the following node facts:

\begin{itemize}

   \item \code{inactive(node A)}: Fact present on nodes that are not currently
      running on a thread. Facts \code{running(T, A)} and \code{inactive(A)} are
      mutually exclusive.

   \item \code{owner(node A, thread T)}: Fact that indicates the current thread
      \code{T} that currently owns node \code{A}.

   \item \code{available-work(node A, bool F)}: Fact that indicates if node
      \code{A} has new facts to be processed.

\end{itemize}

In terms of thread facts we have the following:

\begin{itemize}
   \item \code{active(thread T)}: Fact exists if thread \code{T} is currently
      active.

   \item \code{idle(thread T)}: Fact exists if thread \code{T} is currently
      idle. Facts \code{idle(T)} and \code{active(T)} are mutually exclusive.
\end{itemize}

Figure~\ref{code:threads:modeling} presents how the operational semantics for a
given LM program is modeled using the LM language itself.

First, we define the initial facts: \code{owner(A, T)} on
line~\ref{line:threads:model_owner}, which assigns a node to a thread;
\code{available-work(A, F)} on line~\ref{line:threads:model_available}, where
\code{F = true} if node \code{A} has initial facts, otherwise \code{F = false};
\code{active(T)} on line~\ref{line:threads:model_active} to mark each thread as
\emph{active}; and \code{is-moving(A)} on line~\ref{line:threads:model_moving} so
that all nodes can move between threads.

Each program rule is translated as shown in
lines~\ref{line:threads:model_rule1}-\ref{line:threads:model_rule2}. The
original rule was \code{node-fact(A, Y), other-fact(A, B) -o remote-fact(B),
local-fact(A)}, so we have a local derivation of \code{local-fact(A)} and a
remote-derivation of \code{remote-fact(B)}. In the translation, we update
\code{available-work} of node \code{B} to \code{true} because there is a new
derivation for \code{B}. The fact \code{running(T, A)} is used to ensure that
thread \code{T} is running on node \code{A}. Note that for thread rules we do
not need to use \code{running(T, A)} on the rule's LHS and the thread running
the rule does not even need to have \code{active(T)}. This enforces the
non-deterministic semantics for thread rules.

After the program rules are translated, we have the rule in
lines~\ref{line:threads:model_drop_node1}-\ref{line:threads:model_drop_node2}
which forces thread \code{T} to stop running on node \code{A}. Here, we use the
coordination fact \code{default-priority} to update the priority of node
\code{A}. The thread's state switches to \code{idle(T)}, while the node's state
changes to \code{inactive(A)}. Note that this rule must appear after the
program's rules because the rule priorities are exploited in order to force
thread \code{T} to derive all the candidate rules for \code{A}.

If a thread is idle, then it is able to derive the rule in
lines~\ref{line:threads:model_next_node1}-\ref{line:threads:model_next_node2} in
order to select another node for execution. We use a \code{max} selector to
select the node \code{A} with the highest priority \code{Prio}. If there is such
node, the node changes to \code{running(T, A)} and thread \code{T} changes to
\code{active(T)}.

Finally, the rule in
lines~\ref{line:threads:model_steal1}-\ref{line:threads:model_steal2} allows for
threads to steal nodes owned by other threads. If a node is not currently being
executed (\code{inactive(A)}), can be moved (\code{is-moving(A)}), and is owned by
another thread \code{Other} (\code{owner(A, Other)}), then the thread owner is
updated, potentially allowing the previous rule to execute.

\begin{figure}[h!]
\begin{Verbatim}[numbers=left,fontsize=\codesize,commandchars=\\\#\&]
type linear running(thread, node).
type linear inactive(node).
type linear priority(node, float).
type linear default-priority(node, float).
type linear available-work(node, bool).
type linear active(thread).
type linear idle(thread).
type linear owner(node, thread).
type linear is-moving(node).

owner(A, T). \hfill// Initially node assignment.\label#line:threads:model_owner&
available-work(A, F). \hfill// Some nodes have available work.\label#line:threads:model_available&
moving(A). \hfill// All nodes can be stolen.\label#line:threads:model_moving&
active(T). \hfill// All threads are active.\label#line:threads:model_active&

\underline#node-fact(A, Y)&,\label#line:threads:model_rule1&\hfill // Program rules go here.
\underline#other-fact(A, B)&,
running(T, A), available-work(B, _)
   -o \underline#remote-fact(B)&, \underline#local-fact(A)&,
      running(T, A),
      available-work(B, true).\label#line:threads:model_rule2&

active(T), running(T, A), priority(A, Prio), \hfill// Switching to another node.\label#line:threads:model_drop_node1&
default-priority(A, DefPrio), available-work(A, T)
   -o inactive(A), priority(A, DefPrio),
      default-priority(A, DefPrio),
      available-work(A, false), idle(T).\label#line:threads:model_drop_node2&

[max => Prio | \hfill// Select next node to be processed.\label#line:threads:model_next_node1&
   idle(T), owner(A, T),
   priority(A, Prio), available-work(A, true)]
   -o active(T), owner(A, T),
      running(T, A), available-work(A, false),
      priority(A, Prio).\label#line:threads:model_next_node2&

idle(T), !other-thread(T, Other) \hfill// Attempt to steal a node.\label#line:threads:model_steal1&
owner(A, Other), inactive(A),
available-work(A, true),
moving(A)
   -o idle(T), owner(A, T), moving(A),
      inactive(A), available-work(A, true).\label#line:threads:model_steal2&
\end{Verbatim}
\caption{Modeling the operational semantics as a LM program. The underlined code
represents how an example rule \code{node-fact(A, Y), other-fact(A, B) -o remote-fact(B),
local-fact(A)} needs to be translated for modelling the semantics.}
\label{code:threads:modeling}
\end{figure}

We now model several coordination facts presented in
Chapter~\ref{chapter:coordination} using LM rules. We focus on
\code{set-thread}, \code{set-priority}, \code{just-moved}, and
\code{schedule-next}. The rules are presented in
Fig.~\ref{code:threads:modeling_scheduling} and should be the highest priority
rules in LM programs.

We start with the axiom~\code{priority(A, initial-priority)} and
\code{default-priority(A, initial-priority)}
(lines~\ref{line:threads:model_prio} and~\ref{line:threads:model_defprio}) to
define the initial priorities of nodes. In line~\ref{line:threads:model_snext}
we have the rule for the \code{schedule-next} coordination fact, which simply
re-derives a \code{set-priority} but with an infinite priority. Fact
\code{set-priority} is processed in
lines~\ref{line:threads:model_set1}-\ref{line:threads:model_set2} by updating
the priority values in the \code{priority} facts. As explained in
Chapter~\ref{chapter:coordination}, only higher priorities are taken into
account.

For the \code{set-thread} coordination fact, we have
lines~\ref{line:threads:model_thread1}-\ref{line:threads:model_thread2}. The
first rule applies when the node is currently executing on some thread, forcing
the thread to stop executing the node and to derive \code{just-moved(A)}. In the
second rule, node \code{A} is not being executed and the \code{owner} fact is
simply updated to the new thread.

Note that the rules for updating the coordination sensing facts do not require
the \code{running} predicate in the rule's body, therefore it should not matter
which thread does the update as long as it is done. In the VM, the update is
always done by thread that derives the coordination fact for efficiency reasons.

\begin{figure}[h!]
\begin{Verbatim}[numbers=left,fontsize=\codesize,commandchars=\\\#\&]
type linear is-static(node).
type linear is-moving(node).
type linear set-priority(node, float).
type linear just-moved(node).
type linear move-to-thread(node, thread).

priority(A, initial-priority).\label#line:threads:model_prio&\hfill// Priority facts.
default-priority(A, initial-priority).\label#line:threads:model_defprio&

schedule-next(A) -o set-priority(A, +00).\label#line:threads:model_snext&

set-priority(A, P1), priority(A, P2), P2 < P1\label#line:threads:model_set1&
   -o priority(A, P1).

set-priority(A, P1), priority(A, P2), P2 >= P1
   -o priority(A, P2).\label#line:threads:model_set2&

running(T, A), set-thread(A, T),\label#line:threads:model_thread1&
available-work(A, _), is-moving(A)
   -o available-work(A, true),
      inactive(A), is-static(A),
      just-moved(A).

inactive(A), set-thread(A, T),
owner(A, TOld), is-moving(A),
available-work(A, _)
   -o is-static(A), owner(A, T), just-moved(A),
      available-work(A, true).\label#line:threads:model_thread2&
\end{Verbatim}
\caption{Modeling the operational semantics for coordination facts as a LM
program.}
\label{code:threads:modeling_scheduling}
\end{figure}


\section{Related Work}
As already seen in the previous chapter, there are several programming models
such as Galois~\cite{nguyen11}, Elixir~\cite{Prountzos:2012:ESS:2384616.2384644}
and Halide~\cite{Ragan-Kelley:2013:HLC:2491956.2462176}
which allow the programmer to apply different scheduling policies to programs.
Unfortunately, these models only reason about the data or program being computed
and not about the parallel architecture.

In the logic programming community, there have been some attempts at exposing a
low level programming interface in Prolog programs to permit explicit programmer
control. An example is the proposal by Casas et
al.~\cite{Casas_towardshigh-level} which exposes execution primitives for
AND-parallelism, allowing for different scheduling policies. Compared to LM,
this approach offers a more fine grained control to parallelism but has limited
support for reasoning about thread state.


\section{Chapter Summary}

In this chapter, we have extended the LM language with a declarative mechanism
for reasoning about the underlying parallel architecture. LM programs can be
first written in a data-driven fashion and then optimized by reasoning about the
state of threads, enabling the move from implicit parallelism to explicit
parallelism. We have presented four programs that showcase the potential of the
new mechanism and several experimental results that validate our approach.



\chapter{Multicore Implementation}
To support thread-based facts, both the compilation and runtime system described in
Chapter~\ref{chapter:implementation} require some changes.

\subsection{Compiler}

The compiler needs to recognize rules that use thread facts. For thread rules,
the compiler checks if the rule's body is using facts from the same thread by
checking the first argument of each fact. For mixed rules, the rule's body may
use a thread \code{T} and a node \code{A} and all the node facts have to use
\code{A}, while all threads facts must use \code{T} as the first argument. If
the programmer was to retrieve either the thread or the node for the current
computation, she may use \code{running(T, A)}.

Once rules are type checked, the iteration code for thread-based facts needs to
be adapted. When a rule requires facts from the thread, it must use the data
structures from the thread. The runtime API used for inserting thread facts is
also different since they have to be added to the thread's database.

\subsection{Runtime}

In the runtime system, thread-based facts are implemented just like regular
facts. Each thread has its own database of facts and uses exactly the same data
structures for facts, as presented before for regular nodes.  The major
difference between a regular node and a thread node is that a thread node is
never put into the work queue of its thread. As shown in the updated work loop
presented in Fig.~\ref{alg:threads:work_loop}, the thread node executes
alongside the regular node when $TH.process\_node$ is called. It is also
important to note that, before a thread becomes idle, it may have potential
candidate thread rules that are now derivable because another thread has derived
thread facts in the current thread. In particular, it is entirely possible to have
programs that only deal with thread facts.

\begin{figure}
\begin{algorithm}[H]
   \KwData{Thread TH}
   \While{true}{
      $TH.work\_queue.lock()$\;
      $node \longleftarrow TH.work\_queue.pop\_node()$ \;
      $TH.work\_queue.unlock()$\;
      \uIf{$node$}{
        \underline{$TH.process\_node(node, TH.thread\_node)$}\;
      }
      \Else{
        \tcc{The thread's node may have candidate rules using incoming thread facts}
        \underline{$TH.process\_node(nil, TH.thread\_node)$}\;
        \tcc{Attempt to steal some nodes.}
         \If{$\bang TH.steal\_nodes()$}{
            $TH.become\_idle()$\;
            \While{$len(TH.work\_queue) == 0$}{
               \tcc{Try to terminate}
               \If{$TH.synchronize\_termination()$}{
                  \textbf{terminate}\;
               }
               \If{$TH.steal\_nodes()$}{
                  \tcc{Thread is still in the stealing state}
                  break\;
               }
            }
            \tcc{There's new nodes in the queue.}
            $TH.become\_active()$\;
         }
      }
 }
\end{algorithm}
\caption{Thread work loop updated to take into account thread-based facts.
New or modified code is underlined.}
\label{alg:threads:work_loop}
\end{figure}

Thread-based facts also introduce new synchronization logic in the runtime
system. For instance, when a rule derives a new thread fact on another thread,
it needs to synchronize with that thread (using the appropriate thread node
locks) to add the facts to the thread's database. When a thread is executing its
own node or a regular node, it also must lock the thread node's \emph{DB Lock}
in order to protect its data structures from being manipulated by other threads
concurrently.

Matching rules using thread facts requires special care since they may require
both facts from the regular node and from the thread's node. Before a node is
executed, the rule engine (Section~\ref{sec:implementation:rule_engine}) of the
regular node is updated to take into account the facts of the thread's node so
that mixed rules (rules that use both thread and regular facts) execute. In this
scheme, mixed rules may be unsuccessfully fired repeatedly until a node which
has matching facts gets to execute. Since the LHS of mixed rules use an implicit
\code{running(T, N)} fact, it is enough that a different node is running
to fire mixed rules as long as the running node has the required facts.
Without using an implicit \code{running} fact, the system would need to lookup
for a regular node that would successfully activate a given mixed rule. This
would be prohibitly expensive since some programs might have millions of regular
nodes.


\chapter{Applications and Proofs}
In this section, we present solutions to well-known problems. We start with
straightforward graph-based problems such as bipartitness checking and two
versions of the PageRank program. Next, we present the LM version of the
QuickSort algorithm, which from a first impression may not fit well under the
programming paradigm offered by LM. Informal correctness and termination proofs
are also included to further show that such important properties are relatively
easy to prove for programs written in LM.

\subsection{Bipartiteness Checking}

The problem of checking if a graph is bipartite can be seen as a 2-color graph
coloring problem.  The code for this algorithm is shown in
Fig.~\ref{language:code:bichecking}. All nodes in the graph start as
\texttt{uncolored},
because they do not have a color yet. The axiom \texttt{visit(@1, 1)} is
instantiated at node \texttt{@1} (line 9) in order to color it with color 1.

If a node is \texttt{uncolored} and needs to be marked with a color \texttt{P}
then the rule in lines 11-12 is applied. We consume the \texttt{uncolored} fact
and derive a \texttt{colored(A, P)} to effectively color the node with
\texttt{P}. We also derive \texttt{visit(B, next(P))} in neighbor nodes to color
them with the other color. Line 

The coloring can fail if a node is already colored with a color \texttt{P} and
needs to be colored with a different color (line 15) or if it has already failed
(line 16).

\begin{figure}[h!]
\begin{Verbatim}[numbers=left,fontsize=\scriptsize]
type route edge(node, node).
type linear visit(node, int).
type linear uncolored(node).
type linear colored(node, int).
type linear fail(node).

fun next(int X) : int = if X <> 1 then 1 else 2 end.

visit(@1, 1).

visit(A, P), uncolored(A)
   -o {B | !edge(A, B) | visit(B, next(P))}, colored(A, P).

visit(A, P), colored(A, P) -o colored(A, P).
visit(A, P1), colored(A, P2), P1 <> P2 -o fail(A).
visit(A, P), fail(A) -o fail(A).
\end{Verbatim}
  \caption{Bipartiteness Checking program.}
  \label{language:code:bichecking}
\end{figure}

\subsubsection{Proof Of Correctness}

In order to show that the code in Fig.~\ref{language:code:bichecking} works as
intended, we first setup some invariants that hold throughout the execution of
the program. Assume that the set of nodes in the graph is represented as $N$.

\begin{invariant}[Node state]
Set of nodes $N$ is partitioned into 4 different states that represent the 4
possible states that a node can be in, namely:

\begin{itemize}
   \item $U$ (\texttt{uncolored} nodes)
   \item $F$ (\texttt{fail} nodes)
   \item $C_{true}$ (\texttt{colored(A, 1)} nodes)
   \item $C_{false}$ (\texttt{colored(A, 2)} nodes)
\end{itemize}
\end{invariant}
\begin{proof}
Initially, all nodes start in set $U$. All the 4 rules of the programs either
keep the node in the same set or exchange the node with another set.
\end{proof}

A bipartite graph is one where in every edge $a \leftrightarrow b$, there is a
valid assignment that makes $a$ member of set $C_{true}$ or $C_{false}$ and node
$b$ member of either $C_{false}$ or $C_{true}$ respectively.

\begin{lemma}[Bipartiteness
   Convergence]\label{language:lemma:bipartite_convergence}
   We now reason from the application of the program rules. After each
   application of an inference rule, one of the following will happen:

   \begin{itemize}
      \item Set $U$ will decrease and set $C_{true}$ or $C_{false}$ will
         increase, with a potential increase in the number of \texttt{visit}
         facts.
      \item Set $C_{true}$ or $C_{false}$ will stay the same, while the number
         of \texttt{visit} facts will be reduced.

      \item Set $C_{true}$ or $C_{false}$ will decrease and set $F$ will
         increase, while the number of \texttt{visit} facts will be reduced.

      \item Set $F$ will stay the same, while the number of \texttt{visit} facts
         decreases.
   \end{itemize}

\end{lemma}
\begin{proof}
Directly from the rules.
\end{proof}

From this invariant, it can be inferred that set $U$ never increases in size
and in a node transition from \texttt{uncolored} to \texttt{colored}, the
database may increase in size. For every other rule application, the database of
facts always decreases. This also means that the program will eventually
terminate, since it is limited by the number of \texttt{visit} facts that can be
generated.

\begin{theorem}[Bipartiteness Correctness]
If the graph is connected and bipartite then the nodes will be partitioned into
sets $C_{true}$ and $C_{false}$, while sets $F$ and $U$ are empty.
\end{theorem}
\begin{proof}
   By induction, we prove that uncolored nodes become part of either $C_{true}$
   and $C_{false}$ and, if there is an edge between nodes in the two sets then
   they have different colors.

   In the base case, we start with empty sets but node \texttt{@1} is made
   member of $C_{true}$. Rule 1 sends \texttt{visit} facts to the neighbors of
   \texttt{@1}, forcing them to be members of $C_{false}$.

   In the inductive case, we have sets $C'_{true}$ and $C'_{false}$ with some
   nodes already colored. From Lemma~\ref{language:lemma:bipartite_convergence},
   we know that $U$ always decreases. Since the graph is bipartite, events 3 and
   4 never happen since there is a possible partitioning of nodes. With event 1,
   we have set $C_{true} = C'_{true}, n$, (or $C_{false}$) where $n$ is the
   node and with event 2, the sets remain the same. Since the graph is
   connected, every node will be colored, therefore event 1 will happen for
   every node of the graph.
\end{proof}

\subsection{PageRank}

PageRank~\cite{Page:2001:MNR} is a well known graph algorithm that is used to
compute the relative relevance of web pages.  The code for a synchronous version
of the algorithm is shown in Fig.~\ref{code:pagerank}.  As the name indicates,
the pagerank is computed for a certain number of iterations. The initial
pagerank is the same for every page and is initialized in the first rule (lines
15-16) along with an accumulator.

The second rule of the program (lines 17-19) propagates a newly computed
pagerank value to all neighbors. The fact \texttt{neighbor-pagerank} informs the
neighbor node about the pagerank value of node \texttt{A} for iteration
\texttt{Iter + 1}. For every iteration, each node will accumulate the all the
\texttt{neighbor-pagerank} facts into the \texttt{accumulator} fact (lines
27-28). When all inbound neighbor pagerank values are accumulated, the third
rule (lines 21-25) is fired and a pagerank value is derived for iteration
\texttt{Iter}.

\begin{figure}[h!]
\begin{Verbatim}[numbers=left,fontsize=\scriptsize]
type outbound(node, node, float).
type linear pagerank(node, float, int).
type numInbound(node, int).
type linear accumulator(node, float Acc, int Left, int Iteration).
type linear neighbor-pagerank(node, node Neighbor, float Rank, int Iteration).
type linear start(node).

const damping = 0.85. // probability of user following a link in the current page.
const iterations = str2int(@arg1). // iterations to compute.
const pages = @world. // number of pages in the graph.

start(A).

start(A), !numInbound(A, T)
   -o accumulator(A, 0.0, T, 1), pagerank(A, 1.0 / float(pages), 0).

pagerank(A, V, Iter), // propagate new pagerank value
Iter < iterations
   -o {B, W | !outbound(A, B, W) | neighbor-pagerank(B, A, V * W, Iter + 1)}.

accumulator(A, Acc, 0, Iter), // new pagerank
!numInbound(A, T),
V = damping + (1.0 - damping) * Acc,
Iter <= iterations
   -o pagerank(A, V, Iter), accumulator(A, 0.0, T, Iter + 1).
	
neighbor-pagerank(A, B, V, Iter), accumulator(A, Acc, T, Iter)
   -o accumulator(A, Acc + V, T - 1, Iter).
\end{Verbatim}
\caption{Synchronous PageRank program.}
\label{language:code:pagerank}
\end{figure}

\subsubsection{Asynchronous PageRank}

We also have an asynchronous version of the algorithm that trades correctness
with convergence speed since it does not synchronize between iterations.
Figure~\ref{language:code:async_pagerank} shows the LM code for this particular
version, where two major differences can be observed: (1) there is a linear fact
\texttt{neighbor-pagerank} containing the most up-to-date pagerank value of a
neighbor node; (2) a new \texttt{update} fact that forces the node to re-compute
its pagerank by processing the currently available \texttt{neighbor-pagerank}
facts. Rules in lines 13-21 update the \texttt{neighbor-pagerank} values, while
rule in lines 23-29 asynchronously update the current pagerank value. This last
rule derives multiple \texttt{new-neighbor-rank} that is used to inform the
neighbor about the new pagerank value.

\begin{figure}[h!]
\begin{Verbatim}[numbers=left,fontsize=\scriptsize]
type outbound(node, node, float).
type linear pagerank(node, float, int).
type numInbound(node, int).
type linear neighbor-pagerank(node, node Neighbor, float Rank, int Iteration).
type linear new-neighbor-rank(node, node Neighbor, float Rank, int Iteration).
type linear update(A).
type linear sum-ranks(node, float).

pagerank(A, 1.0 / float(pages), 0).
update(A).
neighbor-pagerank(A, B, 1.0 / float(pages), 0). // pagerank of B is ...

// save incoming pagerank value.
new-neighbor-rank(A, B, New, Iteration),
neighbor-pagerank(A, B, Old, OldIteration),
Iteration > OldIteration
   -o neighbor-pagerank(A, B, New, Iteration).
new-neighbor-rank(A, B, New, Iteration),
neighbor-pagerank(A, B, Old, OldIteration),
Iteration <= OldIteration
   -o neighbor-pagerank(A, B, Old, OldIteration).

sum-ranks(A, Acc),
NewRank = damping/float(pages) + (1.0 - damping) * Acc,
pagerank(A, OldRank, Iteration)
      -o pagerank(A, NewRank, Iteration + 1),
         {B, W, Delta, Iter | !outbound(A, B, W), Delta = fabs(NewRank -
               OldRank) * W | new-neighbor-rank(B, A, NewRank, Iteration + 1),
               if Delta > bound then update(B) end}.

update(A), update(A) -o update(A).

update(A),
!numInbound(A, T)
   -o [sum => V | B, W, Val, Iter | neighbor-pagerank(A, B, Val, Iter)
         V = Val/float(T) | neighbor-pagerank(A, B, Val, Iter) | sum-ranks(A, V)].
\end{Verbatim}
\caption{Asynchronous PageRank program.}
\label{language:code:async_pagerank}
\end{figure}

\subsubsection{Proof Of Correctness}

To build the proof of correctness, we must again prove several program
invariants. This will help us prove that this partifcular program is similar to
a computation on a nonnegative matrix of of unit spectral radius, which has been
proven that it converges~\cite{DBLP:journals/corr/abs-cs-0606047,
Lubachevsky:1986:CAA:4904.4801}.

\begin{invariant}[Page Invariant]
Each page/node has a single \texttt{pagerank(A, Value, Iteration)} and:
\begin{itemize}
   \item For each outbound link, a single \texttt{\bang outbound(A, B, W)}.
   \item For each inbound link, a single \texttt{neighbor-pagerank(A, B, V, Iter)}.
   \item For each \texttt{\bang outbound(A, B, W)}, a \texttt{neighbor-pagerank(A,
      B, V, Iter}.
\end{itemize}
\end{invariant}

\begin{proof}

All axioms validate the 3 conditions of the variant. Note that the third
condition is also validated by the axioms, although not all axioms are shown in
the code.

In relation to rule application:

\begin{itemize}
   \item Rule 1: inbound link re-derived.
   \item Rule 2: inbound link re-derived.
   \item Rule 3: \texttt{pagerank/2} re-derived.
   \item Rule 4: Nothing happens.
   \item Rule 5: inbound links re-derived in the comprehension.
\end{itemize}
\end{proof}

\begin{lemma}[Neighbor rank lemma]
Given a fact \texttt{neighbor-pagerank(A, B, V, Iter)} and a set of facts
\texttt{new-neighbor-rank(A, B, New, Iter2)}, we end up with a single
\texttt{neighbor-pagerank(A, B, V', Iter')}, where \texttt{Iter} is the greater of
\texttt{Iter} or all of \texttt{Iter2'}.
\end{lemma}
\begin{proof}
By induction on the number of \texttt{new-neighbor-rank} facts.

Base case: \texttt{neighbor-pagerank} remains.

Inductive case: given one \texttt{new-neighbor-rank} fact:

\begin{itemize}
   \item Rule 1: the new iteration is older and thus \texttt{neighbor-pagerank}
   is replaced. By applying induction, we know that we will select either the
   new best iteration or a better iteration from the remaining set of
   \texttt{new-neighbor-rank} facts.
   \item Rule 2: the new iteration is not older and we keep the old
   \texttt{neighbor-pagerank} fact. By induction, we select the best from either
   the current iteration or some other (from the set).
\end{itemize}
\end{proof}

\begin{lemma}[Update lemma]
Given at least 1 \texttt{update/1} fact, rule 7 will run.
\end{lemma}
\begin{proof}
By induction on the number of \texttt{update} facts.

Base case: rule 5 will run.

Inductive case: rule 4 will run first because it has a higher priority, reducing
the number of \texttt{update} facts by one. By induction, we know that by
using the remaining \texttt{update} facts, rule 7 will run.
\end{proof}

\begin{lemma}[Pagerank update lemma]
(1) Given at least one \texttt{update} fact, the \texttt{pagerank(A, $V_{I}$,
I)} fact will be updated to become \texttt{pagerank(A, $V_{I + 1}$, I +
1)}, where \texttt{$V_{I + 1} = damping / P + (1.0 - damping)\sum_{B,
I} (W_{B} \times  N_{I,B})$}.

where $W_B = 1.0/T$ ($T$ from \texttt{\bang numInbound(A, $T$)})
and $N_{I,B}$ from \texttt{neighbor-pagerank(A, B, $N_{I, B}$, $I$)}.

(2) For all \texttt{B} outbound nodes (represented using \texttt{\bang outbound(A, B,
W)}, a \texttt{new-neighbor-rank(B, A, $V_{I+1}$, $I + 1$)} is generated.

(3) For all \texttt{B} outbound nodes (represented using \texttt{\bang outbound(A, B,
W)}), a \texttt{update(B)} is generated if 
$fabs(V_{I + 1} - V_{I}) \times W > bound$.
\end{lemma}
\begin{proof}
Using the Update lemma, rule 5 will necessarily run.

It derives \texttt{sum-ranks(A, $\sum_{B, I} (W_B \times N_{I,B})$)} and
fulfills (3).

\texttt{sum-ranks/2} will necessarily fire rule 6,
computing $V_{I+1}$ and updating \texttt{pagerank}. (2) and (3) are fulfilled
through the comprehension of rule 6.
\end{proof}

\begin{invariant}[New neighbor rank equality]
All \texttt{new-neighbor-rank(A, B, V, I)} facts are generated from a corresponding
\texttt{pagerank(B, V, I)} fact, therefore the iteration of any
\texttt{new-neighbor-rank} is at least the same or less than the iteration of
the current pagerank.
\end{invariant}
\begin{proof}
No axioms to prove.

\begin{itemize}
   \item Rule 3: true, new fact is generated.
   \item Rule 6: the fact is kept.
\end{itemize}
\end{proof}

\begin{invariant}[Neighbor rank equality]
All \texttt{neighbor-pagerank(A, B, V, I)} facts have one corresponding
\texttt{pagerank(B, V, I)} fact and the iteration of the \texttt{neighbor-pagerank}
is the same or less than the current iteration of the corresponding
\texttt{pagerank}.
\end{invariant}
\begin{proof}
By analyzing axioms and rules.

Axioms: true.

Rule cases:

\begin{itemize}
   \item Rule 1: uses \texttt{new-neighbor-rank} fact (use new neighbor rank
         equality invariant).
   \item Rule 2: same fact is re-derived.
\end{itemize}
\end{proof}

\begin{theorem}[Pagerank convergence]
The program will compute the pagerank of all nodes that is within \texttt{bound} error
of an asynchronous pagerank computation.
\end{theorem}
\begin{proof}

Using the program axioms, we start with the same pagerank value for all nodes.
The \texttt{\bang outbound(A, B, W)} fact forms a $n \times n$ square matrix (number
of nodes) and is the so-called "Google Matrix".  All the initial pagerank values
can be seen as a vector that adds up to $1$.

The pagerank computation from the "Pagerank update lemma" computes $V_{I + 1} =
damping / P + (1.0 - damping)\sum_{B, I'} (W_{B} \times N_{I',B})$, where $I' <=
I$
(from Neighbor rank equality invariant).

Consider that each node contains a column $G_i$ of the Google matrix. The
pagerank computation can then be represented as: \newline


$V_{I + 1} = G_i fix([N_{I_1, B_1}, ..., N_{I_p, B_p}])$ \hfill (1) \\


Where $p$ is the number of inbound links and $N_{I_j, B_j}$ is the value of
the \texttt{neighbor-pagerank(A, $B_j$, $N_{I_j, B_j}$, $I_j$)}. The $fix()$
function takes the neighbor vector and expands it with zeros corresponding to
nodes that are not inbound links.

From~\cite{DBLP:journals/corr/abs-cs-0606047, Lubachevsky:1986:CAA:4904.4801} we
know that equation (1) will improve (converge) the pagerank value, given that some new
neighbor pagerank values are sent to node $i$ and by the fact that $G_i$ is a
nonnegative matrix of unit spectral radius. Let's use induction by assuming that there
is at least one \texttt{update/1} fact that
schedules a node to improve its pagerank. We want to prove that such fact will
not only improve the node's pagerank but also the pagerank vector.
If the pagerank vector is now close enough (within \texttt{bound}), then the
program will terminate.

\begin{itemize}
   \item Base case: Since we have an \texttt{update} fact as an axiom, rule 7
   will compute a new pagerank (Pagerank update lemma) for all nodes that is
   improved (from equation (1)). From these updates, a new \texttt{update}
   fact is generated that correspond to nodes that have inbound links from the
   source node and need to update their pagerank. These \texttt{update} facts
   may not be generated if the pagerank vector is close enough to its real
   value.

   \item The induction hypothesis tells us that there is at least one node that
   has an \texttt{update} fact. From pagerank update lemma, this
   generates \texttt{new-neighbor-rank} facts if the new value differs
   significantly from the older value. When this happens and using the ``Neighbor
   rank lemma'', the target node will update its \texttt{neighbor-pagerank} fact
   with the newest iteration and then execute a valid pagerank computation that
   brings the pagerank vector close to its solution.

\end{itemize}

\end{proof}


\subsection{Quick-Sort}

The quick-sort algorithm is a divide and conquer sorting algorithm that works by splitting
a list of items into two sublists and then recursively sorting the two sublists.
To split the list, we pick a pivot element and put the items that are smaller than the pivot
into the first sublist and items greater than the pivot into the second list.

The quick-sort algorithm is interesting because it does not map immediately to the graph-based
model of LM. Our approach considers that the program starts with a single node where
the initial list is located. Then we split the list as usual and create two nodes
that will recursively sort the sublists. Interestingly, this will create a tree
that will look similar to a call tree in a functional language.

Fig.~\ref{code:quicksort} presents the code for the quick-sort algorithm in LM.
For each sublist to sort, we start with a \texttt{down} fact that must be (eventually)
transformed into an \texttt{up} fact, where the sublist in the \texttt{up} fact is sorted.
In line 11 we start with the initial list at node \texttt{@0}. Lines 13-16 will immediately
sort the list when the number of items is very small. Otherwise, we apply the rule in line 17.
\texttt{buildpivot} will first split the list using the pivot \texttt{X} using rules in
lines 23-26. When there is nothing more to split, we apply the rule in lines 19-21
that uses an exist construct to create nodes \texttt{B} and \texttt{C}. The sublists
are then sent to these nodes using \texttt{down} facts. Note, however, that we also
derive \texttt{back} facts, that will be used to send the sorted list back using the rule
in line 40.

When the sublists are finally sorted, we get two \texttt{sorted} facts that will match
against \texttt{waitpivot} in the rule located in lines 28-31. The sorted sublists
are appended and then an \texttt{up} fact is finally derived (line 37).

\begin{figure}[h!]
\small\begin{Verbatim}[numbers=left]
type route back(node, node).
type linear down(node, list int).
type linear up(node, list int).
type linear sorted(node, node, list int).
type linear buildpivot(node, list int, int, list int, list int).
type linear waitpivot(node, node, node, int).
type linear append(node, list int, list int).
type linear reverse(node, list int, list int, list int).
type linear reverse2(node, list int, list int).

down(@0, tosort).

down(A, []) -o up(A, []).
down(A, [X]) -o up(A, [X]).
down(A, [X, Y]), X < Y -o up(A, [X, Y]).
down(A, [X, Y]), X >= Y -o up(A, [Y, X]).
down(A, [X | L]) -o buildpivot(A, L, X, [], []).

buildpivot(A, [], X, Smaller, Greater)
   -o exists B, C. (back(B, A), down(B, Smaller),
            back(C, A), down(C, Greater), waitpivot(A, B, C, X)).

buildpivot(A, [Y | L], X, Smaller, Greater), Y <= X
   -o buildpivot(A, L, X, [Y | Smaller], Greater).
buildpivot(A, [Y | L], X, Smaller, Greater), Y > X
   -o buildpivot(A, L, X, Smaller, [Y | Greater]).
   
waitpivot(A, NodeSmaller, NodeGreater, Pivot),
sorted(A, NodeSmaller, Smaller),
sorted(A, NodeGreater, Greater)
   -o append(A, Smaller, [Pivot | Greater]).

append(A, L1, L2) -o reverse(A, L1, L2, []).

reverse(A, [], L2, L3) -o reverse2(A, L3, L2).
reverse(A, [X | L], L2, L3) -o reverse(A, L, L2, [X | L3]).
reverse2(A, [], Result) -o up(A, Result).
reverse2(A, [X | L1], L2) -o reverse2(A, L1, [X | L2]).

up(A, L), back(A, B) -o sorted(B, A, L).
\end{Verbatim}
  \caption{Quick-Sort program.}
  \label{code:quicksort}
\end{figure}
\normalsize




\chapter{Experimental Results}\label{chapter:exp}
\section{Sequential Performance}
\section{Scalability}
\section{Coordinated Programs}
\section{Thread Facts}


\chapter{Conclusions}

In this chapter, we summarize the main contributions of this thesis and suggest
several directions for further research.

\section{Main Contributions}

The goal of our thesis was to show the potential of using a forward-chaining
linear logic programming language to exploit parallelism in a declarative,
efficient and scalable way. For this, we designed and implemented LM, a
programming language suitable for writing concurrent algorithms over graph
data-structures. LM programs are composed of a set of inference rules that apply
over a database of logical facts. Concurrency is achieved by partitioning the
database across a graph data structures and forcing rule derivation to happen at
the node level. Since LM is based on linear logic, facts used in rules may be
retracted, making it possible for the programmer to declaratively manage
structured state.

We introduced coordination facts in LM to show that it is possible for the
programmer to change how programs are scheduled in order to improve program run
time and scalability. This is possible due to the existence of linear facts,
which make it possible for different scheduling decisions to have an effect on
program computation. Without them, a logic program would always compute the same
result.

We also introduced explicit parallelism in LM by adding support for thread
facts. Thread facts are facts stored at the thread level and allow the
programmer to write rules that reason about thread state. This opens new
opportunities for program optimization and improved parallelism because programs
are aware of the existence of threads. Furthermore, the availability of both
thread and coordination facts allows and is convenient for implementing more
sophisticated scheduling parallel algorithms.

We now highlight in more detail the main contributions of this dissertation:

\begin{description}
   \item[Structured State]

Since LM is based on linear logic, LM enables programs to manage state in a
structured manner. Due to the restriction over the inference rules, rules are
derived independently on different nodes of the graph data structure, which
makes it possible to run LM programs in parallel.

   \item[Implementation]

Writing efficient implementations of declarative languages is challenging,
especially when adding support for parallel execution. In this thesis, we have
shown a compilation strategy and memory layout organization for LM programs,
which allows programs to run less than one order of magnitude slower than
hand-written C++ programs. We also described how the LM runtime supports
multicore execution by partitioning the graph among threads. To the best of our
knowledge, LM is the fastest linear logic based programming language available.

\item[Semantics and Abstract Machine]

We showed how LM semantics are specified in order to allow concurrent
programs. We demonstrated how the language is based upon the sequent calculus of
linear logic and we specified a low level abstract machine that closely
resembles the real implementation. We also proved the soundness of the
abstract machine.

\item[Coordination]

We presented new coordination features that improve the expressive
power of the language by making coordination a first class programming construct
that is semantically equivalent to regular computation. In turn, this allows the
programmer to specify how declarative programs are scheduled by the runtime
system, therefore improving overall run time and scalability.

Coordination facts are divided into scheduling and partitioning facts.
Scheduling facts change how nodes are scheduled by the system while partitioning
facts change how nodes are assigned to threads and how they move between
threads. Both these two types of facts are divided into sensing facts (with
information about the state of the runtime system) and action facts (which
perform actions on the runtime system). The interplay between regular facts,
sensing facts and action facts results in faster execution time and improved
parallelism because regular facts affect how action facts are derived and,
conversely, action facts may affect which regular facts are derived.

The coordination facts do not affect the correctness of programs and are well
integrated into proofs since they do not change how rules are scheduled but how
nodes are scheduled during execution.

\item[Explicit Parallelism]

We also introduced the concept of thread facts, which enable LM programs to
exploit the underlying architecture by making it possible to reason about the
state of threads. Thread facts allow the programmer to escape the default
implicit parallelism of LM and allows the implementation of structured
scheduling algorithms which require explicit communication between threads.
To the best of our knowledge, this is the first time that such paradigm
is available in a logic programming language and we are not aware of competing
systems that allow the programmer to reason directly about thread state in a
structured fashion.

\item[Experimentation]

We compared LM sequential and multithreaded execution to hand-written sequential
C++ programs and against frameworks such as GraphLab and Ligra. We showed how
well the LM runtime is able to scale using different programs and datasets. We
measured the run time, scalability and memory usage effects of using
coordination facts and their overheads. For thread facts, we analyzed different
applications and measured the performance improvements of using explicit
parallelism.

In our experiments, we noted that the memory layout of applications, especially
the memory allocator, tends to have a significant effect on the overall
performance. In modern architectures, good memory locality is as important as
having efficient algorithms and in LM this is no different.
We experimented with two allocators to analyze how
performance and scalability may be affected by using different strategies. It is
our belief that it is important to focus on faster sequential execution at the
expense of scalability in order to make declarative parallel languages more
competitive with sequential programs written in languages such as C++.

\end{description}

\section{Future Work}

While much progress has been achieved with this thesis, many new research
avenues have been opened with this work. We now enumerate further research goals
that should be interesting to pursue.

\begin{description}
   \item[Faster Implementation]

LM is still not competitive enough to replace efficient parallel frameworks such
as Ligra. LM is a programming language on its own right and thus requires more
engineering effort to be competitive with frameworks implemented in languages
such as C or C++. Better compilation and runtime systems will be required in
order to reduce the overhead even further, especially as it relates to memory
usage.

Aggressive code analysis should be employed to prove invariants about predicates
and introduce more specialized data structures for storing logical facts. The
goal should be to recover more of the \emph{imperative flavor} that is present
in linear logic programs in order to make them more efficient. The restrictions
of LM rules that make concurrency possible makes this task harder since there is
an inherent tension between concurrency and execution speed since concurrency
implies communication. However, local node computation has still some room for
improvement.

\item[Provability]

We need automated tools for reasoning about correctness and termination of
programs. While we have shown that writing informal proofs is relatively easy
because programs tend to be small, automated proof systems will increase the
faith that programs will work correctly. Ideally, the programmer should be able
to write invariants about the program and the compiler should be able to prove
if such invariants are or are not being met with the given inference rules.

\item[Expressiveness]

Although LM programs are expressive, some work must be done in order to reduce
the restrictions on LM rules and allow for more programmer freedom.  LM
currently only allows rules where the LHS refers to the same node, however, it
should be possible to allow rules that use facts from different nodes. The use
of linear logic facts makes this hard because we need to ensure that a linear
fact is used only once, therefore the compiler should generate code to enforce
this, probably through the use of transactions. In the CHR community, Lam et
al.~\cite{Lam:2013:DEC:2505879.2505892} have developed an encoding for
distributed rules using at most one immediate neighbor into rules that run
locally. It should be relatively straightforward to provide a similar encoding
for LM and then assess how performance is affected by such encoding.

\item[Implicit and Explicit Parallelism] We need more applications that take
   advantage of the mixed parallelism that is available with thread facts. We
   feel that this paradigm needs to be further explored in order to make it
   possible to write new scheduling and parallel algorithms that could be
   compiled to efficient code in other application areas. Furthermore,
   mechanized proofs about such algorithms could then be automatic, improving
   the correctness and reliability of parallel algorithms.

\end{description}

\section{Final Remark}

We argue that our work makes LM the ideal framework for prototyping new
(correct) graph algorithms since LM programs tend to be relatively short and the
programmer only needs to reason about the state of the graph, without the need
to understand how the framework must be used to express the intended algorithms.
Furthermore, the addition of coordination facts and thread facts help the
programmer exploit the underlying parallel architecture in order to create
better programs that take advantage of those architectures without radically
changing the underlying parallel algorithm.  Finally, the good performance of
the LM system allows programs to run reasonably fast when executed on multicore
systems.



\appendix
\chapter{Sequent Calculus}\label{sec:fragment}

\[
\infer[\otimes R]
{\Psi ; \seqx{\Gamma}{\Delta, \Delta'}{A \otimes B}}
{\Psi ; \seqx{\Gamma}{\Delta}{A} & \Psi ; \seqx{\Gamma}{\Delta}{B}}
\tab
\infer[\otimes L]
{\Psi ; \seqx{\Gamma}{\Delta, A \otimes B}{C}}
{\Psi ; \seqx{\Gamma}{\Delta, A, B}{C}}
\]

\[
\infer[\lolli R]
{\Psi ; \seqx{\Gamma}{\Delta}{A \lolli B}}
{\Psi ; \seqx{\Gamma}{\Delta, A}{B}}
\tab
\infer[\lolli L]
{\seqx{\Gamma}{\Delta, \Delta', A \lolli B}{C}}
{\Psi ; \seqx{\Gamma}{\Delta}{A} &
   \Psi ; \seqx{\Gamma}{\Delta', B}{C}}
\]


\[
\infer[\bang R]
{\Psi ; \seqx{\Gamma}{\cdot}{\bang A}}
{\Psi ; \seqx{\Gamma}{\cdot}{A}}
\tab
\infer[\bang L]
{\Psi ; \seqx{\Gamma}{\Delta, \bang A}{C}}
{\Psi ; \seqx{\Gamma, A}{\Delta}{C}}
\tab
\infer[\m{copy}]
{\Psi ; \seqx{\Gamma, A}{\Delta}{C}}
{\Psi ; \seqx{\Gamma, A}{\Delta, A}{C}}
\]

\[
\infer[\one R]
{\Psi ; \seqx{\Gamma}{\cdot}{\one}}
{}
\tab
\infer[\one L]
{\Psi ; \seqx{\Gamma}{\Delta, \one}{C}}
{\Psi ; \seqx{\Gamma}{\Delta}{C}}
\]

\[
\infer[\forall R]
{\Psi ; \seqx{\Gamma}{\Delta}{\forall_{n:\tau}. A}}
{\Psi, m:\tau ; \seqx{\Gamma}{\Delta}{A\{m/n\}}}
\tab
\infer[\forall L]
{\Psi ; \seqx{\Gamma}{\Delta, \forall_{n:\tau}. A}{C}}
{\Psi \vdash M : \tau & \Psi ; \seqx{\Gamma}{\Delta, A\{M/n\}}{C}}
\]

\[
\infer[\exists R]
{\Psi ; \seqx{\Gamma}{\Delta}{\exists_{n: \tau}. A}}
{\Psi \vdash M : \tau &
   \Psi ; \seqx{\Gamma}{\Delta}{A \{M/n\}}}
\tab
\infer[\exists L]
{\Psi ; \seqx{\Gamma}{\Delta, \exists_{n:\tau}. A}{C}}
{\Psi, m:\tau ; \seqx{\Gamma}{\Delta, A\{m/n\}}{C}}
\]

\[
\infer[\itersname^* R]
{\Psi ; \seqx{\Gamma}{\Delta}{\iters{name}{\widehat{V}}}}
{\Psi ; \seqx{\Gamma}{\Delta}{\one}}
\tab
\infer[\itersname^* L]
{\Psi ; \seqx{\Gamma}{\Delta, \iters{name}{\widehat{V}}}{C}}
{\Psi ; \seqx{\Gamma}{\Delta, \iter{name}{N}{\widehat{V}}}{C}}
\]

\[
\infer[\itersname^N L_1]
{\Psi ; \seqx{\Gamma}{\Delta, \iter{name}{N}{\widehat{V}}}{C}}
{\Psi ; \seqx{\Gamma}{\Delta, \forall_{\widehat{x}}. (A \widehat{x}
      \otimes \iter{name}{N-1}{(\iterop{x}{V})})}{C}}
\tab
\infer[\itersname^N L_2]
{\Psi ; \seqx{\Gamma}{\Delta, \iter{name}{N}{\widehat{V}}}{C}}
{\Psi ; \seqx{\Gamma}{\Delta, \one}{C}}
\]

\[
\infer[\itersname^N R]
{\Psi ; \seqx{\Gamma}{\Delta}{\iter{name}{N}{\widehat{V}}}}
{\Psi ; \seqx{\Gamma}{\Delta}{\one}}
\tab
\infer[\itersname^0 L]
{\Psi ; \seqx{\Gamma}{\Delta, \iter{name}{0}{\widehat{V}}}{C}}
{\Psi ; \seqx{\Gamma}{\Delta, (\lambda_{\widehat{x}}. C)\widehat{V}}{C}}
\]


\chapter{High Level Dynamic Semantics}\label{sec:hld}

\section{Step}

{\footnotesize
\[
\infer[\stepz]
{\stepz [\Gamma_1 .. \Gamma_i .. \Gamma_n]; [\Delta_1 .. \Delta_i ..
   \Delta_n]; \Phi \Longrightarrow [\Gamma_1, \Gamma'_1; .. \Gamma_i,
   \Gamma'_i; .. \Gamma_n, \Gamma'_n]; [\Delta_1, \Delta'_1; .. (\Delta_i -
         \Xi'), \Delta'_i; .. \Delta_n, \Delta'_n]}
{\doz \Gamma_i; \Delta_i; \Phi \rightarrow \Xi'; \Delta'_1 .. \Delta'_n;
   \Gamma'_1 .. \Gamma'_n}
\]
}


\section{Application}

\[
\infer[\az \m{rule}]
{\az \Gamma ; \Delta_1, \Delta_2 ; A \lolli B \rightarrow \Xi' ; \Delta' ; \Gamma'}
{\mz \Gamma ; \Delta_1 \rightarrow A & \dz \Gamma ; \Delta_2; \Delta_1; \cdot ; \cdot ; B \rightarrow \Xi' ; \Delta' ; \Gamma'}
\]

\[
\infer[\doz \m{rule}]
{\doz \Gamma ; \Delta ; R, \Phi \rightarrow \Xi' ; \Delta' ; \Gamma'}
{\az \Gamma ; \Delta ; R \rightarrow \Xi' ; \Delta' ; \Gamma'}
\]


\section{Match}

\[
\infer[\mzname \one]
{\mz{\Gamma}{\cdot}{\one}}
{}
\]
\[
\infer[\mzname p]
{\mz{\Gamma}{p}{p}}
{}
\tab
\infer[\mzname \bang p]
{\mz{\Gamma, p}{\cdot}{\bang p}}
{}
\]

\[
\infer[\mzname \otimes]
{\mz{\Gamma}{\Delta_1, \Delta_2}{A \otimes B}}
{\mz{\Gamma}{\Delta_1}{A} & \mz{\Gamma}{\Delta_2}{B}}
\]


\section{Derivation}

\[
\infer[\dz p]
{\dz \Gamma ; \Delta ; \Xi ; \Gamma_1 ; \Delta_1 ; p, \Omega \rightarrow \Xi' ; \Delta' ; \Gamma'}
{\dz \Gamma ; \Delta ; \Xi ; \Gamma_1 ; p, \Delta_1 ; \Omega \rightarrow \Xi' ; \Delta' ; \Gamma'}
\]

\[
\infer[\dz \bang p]
{\dz \Gamma ; \Delta ; \Xi ; \Gamma_1 ; \Delta_1 ; \bang p, \Omega \rightarrow \Xi' ; \Delta' ; \Gamma'}
{\dz \Gamma ; \Delta ; \Xi ; \Gamma_1, p ; \Delta_1 ; \Omega \rightarrow \Xi' ; \Delta' ; \Gamma'}
\]

\[
\infer[\dz \otimes]
{\dz \Gamma ; \Delta ; \Xi ; \Gamma_1 ; \Delta_1 ; A \otimes B, \Omega \rightarrow \Xi' ; \Delta' ; \Gamma'}
{\dz \Gamma ; \Delta ; \Xi ; \Gamma_1 ; \Delta_1 ; A, B, \Omega \rightarrow \Xi' ; \Delta' ; \Gamma'}
\]

\[
\infer[\dz \one]
{\dz \Gamma ; \Delta ; \Xi ; \Gamma_1; \Delta_1 ; 1, \Omega \rightarrow \Xi' ; \Delta' ; \Gamma'}
{\dz \Gamma ; \Delta ; \Xi ; \Gamma_1; \Delta_1 ; \Omega \rightarrow \Xi' ; \Delta' ; \Gamma'}
\]

\[
\infer[\dz end]
{\dz \Gamma ; \Delta ; \Xi' ; \Gamma' ; \Delta' ; \cdot \rightarrow \Xi' ; \Delta' ; \Gamma'}
{}
\]

\[
\infer[\dz \m{comp}^*]
{\dz \Gamma ; \Delta ; \Xi ; \Gamma_1 ; \Delta_1 ; \compsz{A}{B}, \Omega \rightarrow \Xi' ; \Delta' ; \Gamma'}
{\dz \Gamma ; \Delta ; \Xi ; \Gamma_1 ; \Delta_1 ; \compz{N}{A}{B}, \Omega \rightarrow \Xi' ; \Delta' ; \Gamma'}
\]

\[
\infer[\dz \m{comp}^N]
{\dz \Gamma ; \Delta ; \Xi ; \Gamma_1 ; \Delta_1 ; \compz{N}{A}{B}, \Omega \rightarrow \Xi' ; \Delta' ; \Gamma'}
{\dz \Gamma ; \Delta ; \Xi ; \Gamma_1 ; \Delta_1 ; \compunfold{N-1}{A}{B}, \Omega \rightarrow \Xi' ; \Delta' ; \Gamma'}
\]

\[
\infer[\dz \m{comp}^0]
{\dz \Gamma ; \Delta ; \Xi ; \Gamma_1 ; \Delta_1 ; \compz{0}{A}{B}, \Omega \rightarrow \Xi' ; \Delta' ; \Gamma'}
{\dz \Gamma ; \Delta ; \Xi ; \Gamma_1 ; \Delta_1 ; \compunfoldz, \Omega \rightarrow \Xi' ; \Delta' ; \Gamma'}
\]

\[
\infer[\dz \m{agg}^*]
{\dz \Gamma ; \Delta ; \Xi ; \Gamma_1 ; \Delta_1 ; \aggsz{A}{B}{C}, \Omega \rightarrow \Xi' ; \Delta' ; \Gamma'}
{\dz \Gamma ; \Delta ; \Xi ; \Gamma_1 ; \Delta_1 ; \aggz{N}{A}{B}{C}{0}, \Omega \rightarrow \Xi' ; \Delta' ; \Gamma'}
\]

{\small
\[
\infer[\dz \m{agg}^N]
{\dz \Gamma ; \Delta ; \Xi ; \Gamma_1 ; \Delta_1 ; \aggz{N}{A}{B}{C}{V}, \Omega \rightarrow \Xi' ; \Delta' ; \Gamma'}
{\dz \Gamma ; \Delta ; \Xi ; \Gamma_1 ; \Delta_1 ; \aggunfold{N-1}{A}{B}{C}{V}, \Omega \rightarrow \Xi' ; \Delta' ; \Gamma'}
\]
}

\[
\infer[\dz \m{agg}^0]
{\dz \Gamma ; \Delta ; \Xi ; \Gamma_1 ; \Delta_1 ; \aggz{0}{A}{B}{C}{V}, \Omega \rightarrow \Xi' ; \Delta' ; \Gamma'}
{\dz \Gamma ; \Delta ; \Xi ; \Gamma_1 ; \Delta_1 ; \aggunfoldz{C}{V}, \Omega \rightarrow \Xi' ; \Delta' ; \Gamma'}
\]

\[
\infer[\dzname \lolli]
{\dz{\Gamma}{\Pi}{\Delta_a, \Delta_b}{\Xi}{\Gamma_1}{\Delta_1}{A \lolli B,
   \Omega}{\outsem}}
{\mz{\Gamma}{\Delta_a}{A} & \dz{\Gamma}{\Pi}{\Delta_b}{\Xi, \Delta_a}
   {\Gamma_1}{\Delta_1}{B, \Omega}{\outsem}}
\]

\[
\infer[\dzname \forall]
{\dz{\Gamma}{\Pi}{\Delta}{\Xi}{\Gamma_1}{\Delta_1}{\forall_x. A,
   \Omega}{\outsem}}
{\dz{\Gamma}{\Pi}{\Delta}{\Xi}{\Gamma_1}{\Delta_1}{A\{V/x\},
   \Omega}{\outsem}}
\]



\chapter{Low Level Dynamic Semantics}\label{sec:lld}

%\section{Step}

%\[
\infer[\stepo]
{\begin{split}
\stepo [\Gamma_1, \dotsc, \Gamma_i, \dotsc, \Gamma_n] &; [\Delta_1, \dotsc,
   \Delta_i, \dotsc, \Delta_n];
   \Phi \\ \Longrightarrow& \\ [\Gamma_1, \Gamma'_1, \dotsc, \Gamma_i, \Gamma'_i,
   \dotsc,
   \Gamma_n, \Gamma'_n]; & [\Delta_1, \Delta'_1, \dotsc, (\Delta_i - \Xi'),
   \Delta'_i, \dotsc, \Delta_n, \Delta'_n]
\end{split}
}
{
   \doo \Gamma_i; \Delta_i; \Phi \rightarrow \Xi'; \Delta'_1, \dotsc, \Delta'_n;
   \Gamma'_1, \dotsc, \Gamma'_n
}
\]


\section{Application}

\input{lld/init}
\input{lld/fail}
\input{lld/open_rule}
\input{lld/init_rule}


\section{Match}


\begin{multline}
\transx{\matstateb{A \lolli B}{\rulestk}{\lstack{C}}{\Gamma}{\Delta, p_1,
\Delta''}{p(\widehat{x}),
   \Omega}{\Delta' \rightarrow \Omega'}{\Psi}}
{\matstateb{A \lolli B}{\rulestk}{\lframe{\Delta,
p_1}{\Delta''}{p(\widehat{x})}{\Omega; \m{extend}(\Psi, \theta)}{\Delta'}{\Omega'}, \lstack{C}}{\Gamma}{\Delta,
   \Delta''}{\Omega}{\Delta', p_1 \rightarrow \Omega' \otimes
      p(\widehat{x}\theta)}{\m{extend}(\Psi, \theta)}} \\
   \;\;\; (p_1,
   \Delta'' \prec p(\widehat{x}) \;\;\; \Delta \npreceq p(\widehat{x}))
   \tag{match p ok}
\end{multline}

\begin{align}
   \trans{\matstate{A \lolli
   B}{\rulestk}{\lstack{C}}{\Gamma}{\Delta}{p(\widehat{x}),
   \Omega}{\Delta' \rightarrow \Omega'}}
{\contstate{A \lolli B}{\rulestk}{\lstack{C}}{\Gamma}} \;\;\; (\Delta \npreceq
p(\widehat{x})) \tag{match p fail}
\end{align}


\[
\infer[\mo \bang p~\m{first}]
{\mo \Gamma, p_1, \Gamma'' ; \Delta; \Xi; \bang p, \Omega; H; \cdot; \lstack{R}
   \rightarrow \outsem}
{
   \begin{gathered}
      p_1, \Gamma'' \prec \bang p \\
      \mo \Gamma, p_1, \Gamma'' ; \Delta; \Xi; \Omega;
      H; [\Gamma''; \Delta; \bang p ; \Omega; \Xi; \cdot; \cdot]; \lstack{R} \rightarrow \outsem
   \end{gathered}
}
\]

\[
\infer[\mo \bang p~\m{on}~q]
{\mo \Gamma, p_1, \Gamma'' ; \Delta; \Xi; \bang p, \Omega; H; f, \lstack{C};
   \lstack{R}
   \rightarrow \outsem}
{
   \begin{gathered}
      p_1, \Gamma'' \prec \bang p \\
      f = (\Delta_{old}; \Delta'_{old};
         q; \Omega_{old}; \Xi_{old}; \Lambda; \Upsilon) \\
      \mo \Gamma, p_1,
         \Gamma'' ; \Delta; \Xi; \Omega; H; [\Gamma''; \Delta; \bang p ; \Omega; \Xi; q,
      \Lambda; \Upsilon], f, \lstack{C}; \lstack{R} \rightarrow \outsem
   \end{gathered}
}
\]


\[
\infer[\mo \bang p~\m{on}~\bang q]
{\mo \Gamma, p_1, \Gamma'' ; \Delta; \Xi; \bang p, \Omega; H; f, \lstack{C};
   \lstack{R}
   \rightarrow \outsem}
{
   \begin{gathered}
      p_1, \Gamma'' \prec \bang p \\
      f = [\Gamma_{old}; \Delta_{old}; \bang q; \Omega_{old}; \Xi_{old}; \Lambda; \Upsilon] \\
      \mo \Gamma, p_1, \Gamma'' ; \Delta; \Xi; \Omega; H; [\Gamma''; \Delta;
      \bang p ; \Omega; \Xi; \Lambda; q, \Upsilon], f, \lstack{C}; \lstack{R} \rightarrow \outsem
   \end{gathered}
}
\]

\[
\infer[\mo \bang p~\m{fail}]
{\mo \Gamma ; \Delta; \Xi; \bang p, \Omega; H; \lstack{C}; \lstack{R} \rightarrow \outsem}
{\Gamma \npreceq \bang p & \cont \lstack{C}; H; \lstack{R}; \Gamma \rightarrow \outsem}
\]


\begin{align}
\trans{\matstate{A \lolli B}{\rulestk}{\lstack{C}}{\Gamma}{\Delta}{\one,
   \Omega}{\Delta' \rightarrow \Omega'}}
{\matstate{A \lolli B}{\rulestk}{\lstack{C}}{\Gamma}{\Delta}{\Omega}{\Delta'
   \rightarrow \Omega'}} \tag{match $\one$}
\end{align}

\begin{align}
\trans{\matstate{A \lolli B}{\rulestk}{\lstack{C}}{\Gamma}{\Delta}{X \otimes Y,
   \Omega}{\Delta' \rightarrow \Omega'}}
{\matstate{A \lolli B}{\rulestk}{\lstack{C}}{\Gamma}{\Delta}{X, Y,
   \Omega}{\Delta' \rightarrow \Omega;}} \tag{match $\otimes$}
\end{align}

\begin{align}
\trans{\matstate{A \lolli
   B}{\rulestk}{\lstack{C}}{\Gamma}{\Delta}{\cdot}{\Delta' \rightarrow \Omega'}}
{
   \derstatex{\Gamma}{\Delta}{\Delta'}{\cdot}{\cdot}{B}
} \tag{match end}
\end{align}


\section{Continuation}
\begin{align}
\trans{\contstate{A \lolli B}{\rulestk}{\lframe{\Delta}{p_2,
   \Delta''}{p}{\Omega}{\Delta'}{\Omega'}, \lstack{C}}{\Gamma}}
{
   \matstate{A \lolli B}{\rulestk}{\lframe{\Delta,
      p_2}{\Delta''}{p}{\Omega}{\Delta'}{\Omega'},
   \lstack{C}}{\Gamma}{\Delta}{\Omega}{\Delta', p_2 \rightarrow \Omega' \otimes p}}
   \tag{next p}
\end{align}

\begin{align}
\trans{\contstate{A \lolli
   B}{\rulestk}{\lframe{\Delta}{\cdot}{p}{\Omega}{\Delta'}{\Omega'},
      \lstack{C}}{\Gamma}}
{
   \contstate{A \lolli B}{\rulestk}{\lstack{C}}{\Gamma}} \tag{next frame}
\end{align}

\[
\infer[\cont \bang p~\m{next}]
{\cont [p_1, \Gamma'; \Delta; \bang p, \Omega; \Xi; \Lambda; \Upsilon],
   \lstack{C}; H; \lstack{R};
   \Gamma \rightarrow \outsem}
{\mo \Gamma; \Delta; \Xi; \Omega; H; [\Gamma'; \Delta; \bang p, \Omega; \Xi;
   \Lambda; \Upsilon], \lstack{C}; \lstack{R} \rightarrow \outsem}
\]

\[
\infer[\cont \bang p~\m{no~more}]
{\cont [\cdot; \Delta; \bang p, \Omega; \Xi; \Lambda; \Upsilon], \lstack{C}; H;
   \lstack{R}; \Gamma
   \rightarrow \outsem}
{\cont \lstack{C}; H; \lstack{R}; \Gamma \rightarrow \outsem}
\]

\[
\infer[\cont \m{next~rule}]
{\cont \cdot; H; (\Phi, \Delta); \Gamma \rightarrow \Xi'; \Delta'; \Gamma'}
{\doo \Gamma; \Delta; \Phi \rightarrow \Xi'; \Delta'; \Gamma'}
\]


\section{Derivation}
{\footnotesize
\[
\infer[\done p]
{\done \Gamma ; \Delta; \Xi; \Gamma_1 ; \Delta_1; p, \Omega \rightarrow \Xi'; \Delta'; \Gamma'}
{\done \Gamma ; \Delta; \Xi; \Gamma_1 ; p, \Delta_1; \Omega \rightarrow \Xi'; \Delta'; \Gamma'}
\tab
\infer[\done \bang p]
{\done \Gamma ; \Delta ; \Xi; \Gamma_1 ; \Delta_1; \bang p, \Omega \rightarrow \Xi'; \Delta'; \Gamma'}
{\done \Gamma ; \Delta ; \Xi; \Gamma_1, p; \Delta_1; \Omega \rightarrow \Xi'; \Delta'; \Gamma'}
\]
}

\[
\infer[\done 1]
{\done \Gamma; \Delta; \Xi; \Gamma_1 ; \Delta_1; 1, \Omega \rightarrow \outsem}
{\done \Gamma; \Delta; \Xi; \Gamma_1 ; \Delta_1; \Omega \rightarrow \outsem}
\tab
\infer[\done \otimes]
{\done \Gamma ; \Delta; \Xi; \Gamma_1; \Delta_1; A \otimes B, \Omega \rightarrow
   \outsem}
{\done \Gamma ; \Delta; \Xi; \Gamma_1; \Delta_1; A, B, \Omega \rightarrow
   \outsem}
\]

\[
\infer[\done \m{agg}]
{\done \Gamma; \Delta ; \Xi; \Gamma_1; \Delta_1; \aggsz{A}{B}{C}, \Omega
   \rightarrow \outsem}
{\ma \Gamma; \Delta; \Xi; \Gamma_1; \Delta_1; \cdot; A ; \cdot; \cdot;
   \aggsz{A}{B}{C}; \Omega; \Delta; \cdot \rightarrow \outsem}
\]

\[
\infer[\done \m{end}]
{\done \Gamma; \Delta; \Xi; \Gamma_1; \Delta_1; \cdot \rightarrow \Xi; \Delta_1; \Gamma_1}
{}
\]


\section{Aggregates}
\subsection{Match}
\[
\infer[\ma{AG} p~\m{first}]
{\ma{AG} \Gamma; \Delta, p_1, \Delta''; \Xi_N; \Gamma_{N1}; \Delta_{N1}; \cdot; p,
   \Omega; \cdot; \cdot; \Omega_N; \Delta_N; \Sigma \rightarrow \outsem}
{
   \begin{gathered}
      p_1, \Delta'' \prec p \\
      f = (\Delta, p_1; \Delta''; \cdot; p; \Omega;
            \cdot; \cdot) \\
      \ma{AG} \Gamma; \Delta, \Delta''; \Xi_N; \Gamma_{N1};
         \Delta_{N1}; \Xi, p_1; \Omega; f; \cdot; \Omega_N; \Delta_N; \Sigma \rightarrow \outsem
   \end{gathered}
}
\]

\[
\infer[\ma{AG} p~\m{on}~q]
{\ma{AG} \Gamma; \Delta, p_1, \Delta''; \Xi_N; \Gamma_{N1}; \Delta_{N1}; \Xi; p,
   \Omega; C_1, \lstack{C}; \lstack{P}; \Omega_N; \Delta_N; \Sigma \rightarrow \outsem}
{
   \begin{gathered}
      p_1, \Delta'' \prec p \\
      f = (\Delta_{old}; \Delta'_{old}; \Xi_{old}; q; \Omega_{old}; \Lambda; \Upsilon) \\
      f' =  (\Delta, p_1; \Delta''; \Xi; p; \Omega; q, \Lambda; \Upsilon) \\
      \ma{AG} \Gamma; \Delta, \Delta''; \Xi_N; \Gamma_{N1};
         \Delta_{N1}; \Xi, p_1; \Omega; f', f, \lstack{C}; \lstack{P}; \Omega_N;
         \Delta_N; \Sigma \rightarrow \outsem
   \end{gathered}
}
\]

\[
\infer[\ma{AG} p~\m{on}~\bang q~\lstack{C}]
{\ma{AG} \Gamma; \Delta, p_1, \Delta''; \Xi_N; \Gamma_{N1}; \Delta_{N1}; \Xi; p,
   \Omega; C_1, \lstack{C}; \lstack{P}; \Omega_N; \Delta_N; \Sigma \rightarrow \outsem}
{
   \begin{gathered}
      p_1, \Delta'' \prec p \\
      f = [\Gamma_{old}; \Delta_{old}; \Xi_{old}; q;
         \Omega_{old}; \Lambda; \Upsilon]\\
      f' = (\Delta, p_1; \Delta''; \Xi; p; \Omega; \Lambda; q, \Upsilon) \\
      \ma{AG} \Gamma; \Delta, \Delta''; \Xi_N; \Gamma_{N1};
         \Delta_{N1}; \Xi, p_1; \Omega;
         f', f, \lstack{C}; \lstack{P}; \Omega_N; \Delta_N;
         \Sigma \rightarrow \outsem
   \end{gathered}
}
\]
\[
\infer[\ma{AG} p~\m{on}~\bang q~\lstack{P}]
{\ma{AG} \Gamma; \Delta, p_1, \Delta''; \Xi_N; \Gamma_{N1}; \Delta_{N1}; \cdot; p,
   \Omega; \cdot; f, \lstack{P}; \Omega_N; \Delta_N; \Sigma \rightarrow \outsem}
{
   \begin{gathered}
      p_1, \Delta'' \prec p \\
      f = [\Gamma_{old}; \Delta_N; \cdot; q; \Omega_{old}; \cdot; \Upsilon]\\
      f' = (\Delta, p_1; \Delta''; \cdot; p; \Omega; \cdot; q, \Upsilon) \\
      \ma{AG} \Gamma; \Delta, \Delta''; \Xi_N;
            \Gamma_{N1}; \Delta_{N1}; p_1; \Omega; f'; f, \lstack{P}; \Omega_N;
            \Delta_N; \Sigma \rightarrow \outsem
   \end{gathered}
}
\]

\[
\infer[\ma{AG} p~\m{fail}]
{\ma{AG} \Gamma; \Delta; \Xi_N; \Gamma_{N1}; \Delta_{N1}; \Xi; p, \Omega;
   \lstack{C}; \lstack{P}; \Omega_N; \Delta_N; \Sigma \rightarrow \outsem}
{\conta{AG} \Gamma; \Delta_N; \Xi_N; \Gamma_{N1}; \Delta_{N1}; \lstack{C};
   \lstack{P}; \Omega_N;
   \Sigma \rightarrow \outsem}
\]



\begin{multline}
\transx{
   \matstatea{\Delta_N}{\cdot;
      \lstack{P}}{\Gamma, p_1, \Gamma''}{\Delta}{\bang p, \Omega}{\Delta' \rightarrow
         \Omega'}{\Sigma}
}
{
   \matstatea{\Delta_N}{\cdot; \pframe{\Gamma''}{\Delta}{\bang
   p}{\Omega}{\Delta'}{\Omega'}, \lstack{P}}{\Gamma, p_1, \Gamma''}{\Delta}{\Omega}
   {\Delta' \rightarrow \Omega' \otimes \bang p}{\Sigma}
} \tag{agg match \bang p ok $\lstack{P}$}
\end{multline}

\begin{multline}
\transx{
   \matstatea{\Delta_N}{\lstack{C};
      \lstack{P}}{\Gamma, p_1, \Gamma''}{\Delta}{\bang p, \Omega}{\Delta' \rightarrow
         \Omega'}{\Sigma}
}
{
   \matstatea{\Delta_N}{\pframe{\Gamma''}{\Delta}{\bang
   p}{\Omega}{\Delta'}{\Omega'}, \lstack{C} ; \lstack{P}}{\Gamma, p_1, \Gamma''}{\Delta}{\Omega}
   {\Delta' \rightarrow \Omega' \otimes \bang p}{\Sigma}
} \tag{agg match \bang p ok $\lstack{C}$}
\end{multline}

\[
\trans{
   \matstatea{\Delta_N}{\lstack{C}; \lstack{P}}{\Gamma}{\Delta}{\bang p,
      \Omega}{\Delta' \rightarrow \Omega'}{\Sigma}
}
{
   \contstatea{\Delta_N}{\lstack{C} ; \lstack{P}}{\Gamma}{\Sigma}
} \tag{agg match \bang p fail}
\]



\begin{multline}
\transx{
   \matstatea{\Delta_N}{\lstack{C};
      \lstack{P}}{\Gamma}{\Delta}{X \otimes Y, \Omega}{\Delta' \rightarrow
         \Omega'}{\Sigma}
}
{
   \matstatea{\Delta_N}{\lstack{C};
      \lstack{P}}{\Gamma}{\Delta}{X, Y, \Omega}{\Delta' \rightarrow
         \Omega'}{\Sigma}
} \tag{agg match $\otimes$}
\end{multline}

\begin{multline}
\transx{
   \matstatea{\Delta_N}{\lstack{C};
      \lstack{P}}{\Gamma}{\Delta}{\one, \Omega}{\Delta' \rightarrow
         \Omega'}{\Sigma}
}
{
   \matstatea{\Delta_N}{\lstack{C};
      \lstack{P}}{\Gamma}{\Delta}{\Omega}{\Delta' \rightarrow
         \Omega'}{\Sigma}
      } \tag{agg match $\one$}
\end{multline}

\subsection{Stack Transformation}
\[
\trans{
   \fixstatea{\Delta}{\Xi; \Delta'}{\_, f, \lstack{C}; \lstack{P}}{\Gamma}{\Sigma}
}
{
   \fixstatea{\Delta}{\Xi; \Delta'}{f, \lstack{C}; \lstack{P}}{\Gamma}{\Sigma}
} \tag{agg fix rec}
\]

\[
\underset{
   \begin{gathered}
   \Pi(\m{agg}) = \forall_{\widehat{v}, \Sigma'}.
   (\defstwo{agg}{\widehat{v}}{\Sigma'} \lolli ((\lambda x. C x)\Sigma' \with (\forall_{\widehat{x}, \sigma}.
                                                (A \lolli B \otimes
                                                 \defstwo{agg}{\widehat{v}}{\sigma
                                                 ::\Sigma'})))) \\
                                                 f' = \texttt{remove}(f, \Delta') \\
                                                 \lstack{P'} = \texttt{remove}(\lstack{P}, \Delta') \\
   V = \Psi(\sigma)
   \end{gathered}
}
{
   \transx{
      \fixstatea{\Delta}{\Xi; \Delta'}{f; \lstack{P}}{\Gamma}{\Sigma}
   }
   {
      \derstatea{\Delta}{\Xi; \Delta'}{\gammanew}{\deltanew}{V :: \Sigma}{f';
         \lstack{P'}}{B\{\Psi(\widehat{x}), V / \widehat{x}, \sigma \}}
   }
}
   \tag{agg fix end1}
\]

\[
\underset{
   \begin{gathered}
   \Pi(\m{agg}) = \forall_{\widehat{v}, \Sigma'}.
   (\defstwo{agg}{\widehat{v}}{\Sigma'} \lolli ((\lambda x. C x)\Sigma' \with (\forall_{\widehat{x}, \sigma}.
                                                (A \lolli B \otimes
                                                 \defstwo{agg}{\widehat{v}}{\sigma
                                                 ::\Sigma'})))) \\
                                                 \lstack{P'} = \texttt{remove}(\lstack{P}, \Delta') \\
   V = \Psi(\sigma)
   \end{gathered}
}
{
   \trans{
      \fixstatea{\Delta}{\Xi; \Delta'}{\cdot; \lstack{P}}{\Gamma}{\Sigma}
   }
   {
      \derstatea{\Delta}{\Xi, \Delta'}{\Gamma_{N1}}{\Delta_{N1}}{V :: \Sigma}{\cdot;
         \lstack{P}'}{B\{\Psi(\widehat{x}), V / \widehat{x}, \sigma \}}
   }
} \tag{agg fix end2}
\]

\subsection{Backtracking}
{\footnotesize
\[
\infer[\conta \m{next}~C~p]
{\conta \Gamma; \Delta_N; \Xi_N; \Gamma_{N1}; \Delta_{N1}; (\Delta; p_1, \Delta''; \Xi; p; \Omega; \Lambda; \Upsilon), C; P; AG; \Omega_N; T \rightarrow \Xi'; \Delta'; \Gamma'}
{\ma \Gamma; \Delta; \Xi_N; \Gamma_{N1}; \Delta_{N1}; \Xi; \Omega; (\Delta, p_1; \Delta''; \Xi; p; \Omega; \Lambda; \Upsilon), C; P; AG; \Omega_N; \Delta_N; T \rightarrow \Xi'; \Delta'; \Gamma'}
\]

\[
\infer[\conta \m{next}~C~\bang p]
{\conta \Gamma; \Delta_N; \Xi_N; \Gamma_{N1}; \Delta_{N1}; [p_1, \Gamma'; \Delta; \Xi; \bang p; \Omega; \Lambda; \Upsilon], C; P; AG; \Omega_N; T \rightarrow \Xi'; \Delta'; \Gamma'}
{\ma \Gamma; \Delta; \Xi_N; \Gamma_{N1}; \Delta_{N1}; \Xi; \Omega; [\Gamma'; \Delta; \Xi; \bang p; \Omega; \Lambda; \Upsilon], C; P; AG; \Omega_N; \Delta_N; T \rightarrow \Xi'; \Delta'; \Gamma'}
\]

\[
\infer[\conta \m{next}~C~\m{empty}~p]
{\conta \Gamma; \Delta_N; \Xi_N; \Gamma_{N1}; \Delta_{N1}; (\Delta; \cdot; \Xi; p; \Omega; \Lambda; \Upsilon), C; P; AG; \Omega_N; T \rightarrow \Xi'; \Delta'; \Gamma'}
{\conta \Gamma; \Delta_N; \Xi_N; \Gamma_{N1}; \Delta_{N1}; C; P; AG; \Omega_N; T \rightarrow \Xi'; \Delta'; \Gamma'}
\]

\[
\infer[\conta \m{next}~C~\m{empty}~\bang p]
{\conta \Gamma; \Delta_N; \Xi_N; \Gamma_{N1}; \Delta_{N1}; [\cdot; \Delta; \Xi; \bang p; \Omega; \Lambda; \Upsilon], C; P; AG; \Omega_N; T \rightarrow \Xi'; \Delta'; \Gamma'}
{\conta \Gamma; \Delta_N; \Xi_N; \Gamma_{N1}; \Delta_{N1}; C; P; AG; \Omega_N; T \rightarrow \Xi'; \Delta'; \Gamma'}
\]
}



\begin{multline}
\transx{
   \contstatea{\deltan}{\cdot ; \pframe{p_1, \Gamma''}{\Delta}{\bang
   p}{\Omega}{\Delta'}{\Omega'}, \lstack{P}}{\Gamma}{\Sigma}
}
{
   \matstatea{\deltan}{\cdot; \pframe{\Gamma''}{\Delta}{\bang p}
      {\Omega}{\Delta'}{\Omega'}, \lstack{P}}{\Gamma}{\Delta}{p,
      \Omega}{\Delta' \rightarrow \Omega' \otimes \bang p}{\Sigma}
} \tag{agg next \bang p $\lstack{P}$}
\end{multline}

\[
\trans{
   \contstatea{\deltan}{\cdot; \pframe{\cdot}{\Delta}{\bang
   p}{\Omega}{\Delta'}{\Omega'}, \lstack{P}}{\Gamma}{\Sigma}
}
{
   \contstatea{\deltan}{\cdot ; \lstack{P}}{\Gamma}{\Sigma}
} \tag{agg next \bang frame $\lstack{P}$}
\]


\[
\underset{
   \Pi(\m{agg}) = \forall_{\widehat{v}, \Sigma'}.
   (\defstwo{agg}{\widehat{v}}{\Sigma'} \lolli ((\lambda x. C x)\Sigma' \with (\forall_{\widehat{x}, \sigma}.
                                                (A \lolli B \otimes
                                                 \defstwo{agg}{\widehat{v}}{\sigma
                                                 ::\Sigma'}))))
}
{
\trans{
   \contstatea{\Delta_N}{\cdot ; \cdot}{\Gamma}{\Sigma}
}
{
   \derstatex{\Gamma}{\Delta_N}{\Xi}{\Gamma_{N1}}{\Delta_{N1}}{(\lambda x.
         C\{\Psi(\widehat{v})/\widehat{v}\} x) \Sigma,
      \Omega_N}
}
}
\]

\subsection{Derivation}

\[
\trans{
   \derstatea{\Delta}{\Xi}{\gammanew}{\deltanew}{\Sigma}{\lstack{C};
      \lstack{P}}{p, \Omega}
}
{
   \derstatea{\Delta}{\Xi}{\gammanew}{\deltanew, p}{\Sigma}{\lstack{C};
      \lstack{P}}{\Omega}
} \tag{agg new p}
\]

\[
\trans{
   \derstatea{\Delta}{\Xi}{\gammanew}{\deltanew}{\Sigma}{\lstack{C};
      \lstack{P}}{\bang p, \Omega}
}
{
   \derstatea{\Delta}{\Xi}{\gammanew, p}{\deltanew}{\Sigma}{\lstack{C};
      \lstack{P}}{\Omega}
} \tag{agg new \bang p}
\]

\[
\trans{
   \derstatea{\Delta}{\Xi}{\gammanew}{\deltanew}{\Sigma}{\lstack{C};
      \lstack{P}}{X \otimes Y, \Omega}
}
{
   \derstatea{\Delta}{\Xi}{\gammanew, p}{\deltanew}{\Sigma}{\lstack{C};
      \lstack{P}}{X, Y, \Omega}
} \tag{agg new $\otimes$}
\]

\[
\trans{
   \derstatea{\Delta}{\Xi}{\gammanew}{\deltanew}{\Sigma}{\lstack{C};
      \lstack{P}}{\one, \Omega}
}
{
   \derstatea{\Delta}{\Xi}{\gammanew, p}{\deltanew}{\Sigma}{\lstack{C};
      \lstack{P}}{\Omega}
} \tag{agg new $\one$}
\]

\[
\trans{
   \derstatea{\Delta}{\Xi}{\gammanew}{\deltanew}{\Sigma}{\lstack{C};
      \lstack{P}}{\cdot}
}
{
   \contstatea{\Delta}{\lstack{C} ; \lstack{P}}{\Gamma}{\Sigma}
} \tag{agg next}
\]





\backmatter

%\renewcommand{\baselinestretch}{1.0}\normalsize

% By default \bibsection is \chapter*, but we really want this to show
% up in the table of contents and pdf bookmarks.
\renewcommand{\bibsection}{\chapter{\bibname}}
%\newcommand{\bibpreamble}{This text goes between the ``Bibliography''
%  header and the actual list of references}
\bibliographystyle{plainnat}
\bibliography{refs} %your bib file

\end{document}
