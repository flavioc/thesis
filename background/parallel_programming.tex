Most software applications are developed using a \emph{sequential imperative
model} where there is a sequence of steps that the processor must do and a
\emph{memory area} where the processor stores and retrieves data during the
course of execution.  The increase of clock frequency on single-core processors
during the last decades has allowed applications using the sequential imperative model to
run faster without any changes in the source code. Today, due to the stalling of
clock frequencies, processor manufacturers are focusing on increasing the number
of cores per processor, but this is useless to the imperative model since it
does not take advantage of multiple processing cores. This model is also
unsuitable for distributed applications that need to be run in a cluster of
computers because the processors in the cluster do not share the same memory
area and require communication to coordinate computation.

Different programming models have been suggested to solve the limitations of the
sequential imperative model and allow programmers to exploit parallelism. We
classify these models into three main classes: \emph{automatic parallelism},
\emph{imperative parallel programming} and
\emph{declarative parallel programming}.

In automatic parallelism, we have a compiler that transforms programs into
parallel code. A clear example of automatic parallelism is \emph{instruction
level parallelism}, which is available on computer architectures to allow
processors to execute multiple instructions at the same time. On the software
side, the compiler may re-order instructions to allow the hardware to take
advantage of parallelism. Unfortunately, such approach has limited applicability
because important parallelization opportunities are available at a different,
higher, levels of abstraction. These high level approaches are discussed in
Section~\ref{section:background:declarative}.

In \emph{imperative parallel programming}, imperative applications are modified
using new concurrency or communication constructs that allow the programmer to
explicitly exploit parallelism. It is required that the programmer writes code
to efficiently split computation among processing units and allow sharing of
data between processing units. In contrast to declarative parallel programming,
imperative parallel programming is more low-level because the programmer needs
to deal with the finer details of parallelism.

\section{Imperative Parallel Programming Models}

Exploiting parallelism using imperative programming is known to be hard since it
is difficult to coordinate processing units effectively and without bugs. The
very nature of parallel execution means increased non-determinism during
execution which leads to execution interweavings that the programmer needs to be
aware of. Furthermore, many well-known imperative algorithms are not trivially
parallelizable and require complete new approaches to run in a scalable fashion.
Moreover, non-determinism makes it hard to prove properties of the program
because the simpler assumptions of the imperative model no longer hold under the
new programming model.

Parallelism has been traditionally classified into two classes: \emph{data
parallelism} and \emph{task parallelism}. In data parallelism, the data is
partitioned among the processing units and each unit performs the same
computation on their piece of data. In task parallelism, the program is split
into different tasks that are then assigned to processing units. If the data or
tasks are well-defined, relatively independent and regular (i.e., they take the
same time to be completed) then parallelization is trivial. However, issues
arise when it is hard to partition the tasks or the tasks that need to be
completed are not static but are dynamically generated during execution. To
complicate matters even further, some tasks may depend on other tasks being
completed in order to be started. In such situations, the programmer is required
to implement a \emph{scheduler} that efficiently assigns tasks to processing
units and is able to \emph{balance} the load among those units. A scheduler may
use a \emph{centralized strategy} where there is a \emph{master} processing unit
that makes work distribution decisions or the scheduler uses a \emph{distributed
strategy} where each processing unit is able to perform \emph{work stealing} or
\emph{work sharing}~\cite{Blumofe:1999} on other units to improve load balance.

In terms of communication and synchronization between the available processing
units, there are two main parallel programming models available for writing
parallel programss: shared memory~\cite{Mellor-Crummey:1991} and message passing.

As we mentioned before, the imperative model uses a memory area to store and
retrieve data. The \emph{shared memory model} extends this area to allow
communication between \emph{workers}, processes or threads, which are processing
units that have their own execution flow but share the same memory area. The
existence of a shared memory area makes it easier to share data between workers,
however, the access to data shared by multiple workers needs to be protected,
otherwise the data may become inconsistent. Many constructs are available to
ensure \emph{mutual exclusion} such as \emph{locks}~\cite{Silberschatz:2008},
\emph{semaphores}~\cite{Dijkstra:2002}, \emph{mutexes}~\cite{Silberschatz:2008},
and \emph{condition variables}~\cite{Hoare:1974}.

In the \emph{message passing} model, processing units do not share the same
memory area. Instead, processing units send messages to each other to coordinate
parallel execution. Message passing is well suited for programming clusters of
computers, where it is not possible to have a shared memory area, however
message processing is more costly than shared memory area due to the extra work
required to send and serialize messages.  The most well known framework for
message passing is the Message Passing Interface~(MPI~)~\cite{Forum:1994}.

Message passing is also used as a foundation to implement higher level parallel
programming models such as \emph{Remote Procedure
Call}~(RPC)~\cite{Birrell:1984}, which makes it possible to seamlessly call a
procedure which is executed remotely, removing the need for explicit message
passing.

