Although declarative languages give limited opportunities to coordinate
programs, there are other programming languages that directly support different
kinds of coordination~\cite{Papadopoulos98coordinationmodels}. Most of the
following languages cannot be easily classified into either imperative or
declarative languages, but tend to be a mix of both.

Coordination programming languages tend to divide execution in two parts:
\emph{computation}, where the actual computation is performed, and
\emph{coordination}, which deals with communication and cooperation between
processing units. This paradigm attempts to clearly distinguish between these
two parts by providing abstractions for coordination as a way to provide
architecture and system-independent forms of communication.

We can identify two main types of coordination models:

\begin{description}
   \item[Data-Driven:]
   
   In a data-driven model, the state of the computation depends on both the data
   being received or transmitted by the available processes and the current configuration
   of the coordinated processes. The coordinated process is not only responsible
   for reading and manipulating the data but is also responsible for
   coordinating itself and/or other processes. Each process must intermix the
   coordination directives provided by the coordination model with the
   computation code. While these directives have a clear interface, it is
   the programmer's responsibility to use them correctly.

   \item[Task-Driven:]
   
   In this model, the coordination code is more cleanly separated from the
   computation code. While in data-driven models, the content of the data
   exchanged by the processes will affect how the processes coordinate with each
   other, in a task-driven model, the process behavior depends only on the
   coordination patterns that are setup before hand. This means that the
   computation component is defined as a black box and there are clearly defined
   interfaces for input/output. These interfaces are usually defined as a
   full-fledged coordination language and not as simple directives present in
   the data-driven models.  \end{description}

Linda~\cite{linda} is probably the most well-known coordination model. Linda
implements a data-driven coordination model and features a \emph{tuple space}
that can be manipulated using the following coordination directives:
\texttt{out(t)} writes a tuple \texttt{t} into the tuple space; \texttt{in(t)}
reads a tuple using the template \texttt{t}; \texttt{rd(t)} retrieves a copy of
the tuple \texttt{t} from the tuple space; and \texttt{eval(p)} puts a process
\texttt{p} in the tuple space and executes it in parallel.  Linda processes do
not need to know the identity of other processes because processes only
communicate through the tuple space.  Linda can be implemented on top of many
popular languages by simply creating a communication and storage mechanism for
the tuple space and then adding the directives as a language library.

Another early coordination language is Delirium~\cite{Delirium}. Unlike Linda,
which is embedded into another language, Delirium actually embeds operators
written in other languages inside the Delirium language. The advantages of
Delirium are improved abstraction and easier debugging because sequential
operators are isolated from the coordination language.

The original Meld~\cite{ashley-rollman-iclp09} can also be seen as a kind of
data-driven coordination language. The important distinction is that in Meld
there is no explicit coordination directives. When Meld rules are activated they
may derive facts that need to be sent to a neighboring robot. In turn, this will
activate computation on the neighbor. Robot communication is implemented by
\emph{localizing} the program rules and then by creating \emph{communication
rules}. The LM language also implements communication rules, however it goes
further because some facts, e.g., coordination facts, can change how the
processing units schedule computation, which may in turn change the program's
final results or execution time. This results in a more complete inter-play
between coordination code and data.

There are also several important programming models, which are not programming
languages on their own right, but provide ways for a programmer to specify how
computation should be coordinated. On example is the
Galois~\cite{Pingali:2011:TPA:1993316.1993501} system, which is a graph-based
programming model that applies operators over the nodes of a graph. Operator
activities may be scheduled through \emph{runtime coordination}, where
activities are ordered in a specific fashion. In the context of Galois, Nguyen
et al.~\cite{nguyen11} expanded the concept of runtime coordination with the
introduction of an approach to specify scheduling algorithms. The scheduling
language specifies three base schedulers that can be composed and synthesized
without requiring users to write complex concurrent code.  Another language that
builds on top of Galois is Elixir~\cite{Prountzos:2012:ESS:2384616.2384644},
which allows the user to specify how operator application should be scheduled
from a high level specification, that is then compiled into low level code. In
Section~\ref{sec:coordination:related}, we delve deeply into these and other
systems.
