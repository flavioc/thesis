Many techniques and formal systems have been devised to help reason about
parallel programs.  One such example is the
Owicki-Gries~\cite{Owicki:1976:VPP:360051.360224} deductive system for proving
properties about imperative parallel programs (deadlock detection, termination,
etc). It extends Hoare logic~\cite{Hoare:1969} with a stronger set axioms such
as parallel execution, critical section and auxiliary variables. The formal
system can be successfully used in small imperative programs, although using it
on languages such as C is difficult since these do not restrict the use of
shared variables.

Some formal systems do not build on top of a known programming paradigm, but
instead create an entirely new formal system for describing concurrent systems.
Process calculus such as $\pi$-calculus~\cite{Milner:1999:CMS:329902} is a good
example of this. The $\pi$-calculus describes the interactions between
processes through the use of channels for communication. Interestingly, channels
can also be transmitted as messages, allowing for changes in the network of
processes. The $\pi$-calculus is powerful enough to prove that two processes
behave the same through the use of bi-simulation equivalence.

Mobile UNITY~\cite{Roman97anintroduction} is a proof logic designed for
applications with \emph{mobility} of processes. The basic UNITY
model~\cite{UNITY} assumes that statements could be executed
non-deterministically in order to create parallelism. Mobile UNITY extends UNITY
by adding locations to processes and removing the nondeterministic aspect from
local processes.  Processes can then communicate or move between locations.

The Meld language, as a logic programming language, has been used to produce
proofs of correctness. A Meld program is amenable to mechanized analysis via
theorem checkers such as Twelf~\cite{twelf}, a logic system designed for
analyzing program logics and logic program implementations. For instance, a
meta-module based shape planner program was proven to be correct under the
assumption that actions derived by the program are always successfully applied
in the outside world~\cite{dewey-iros08,ashley-rollman-iclp09}.  While the fault
tolerance aspect is lax, the planner will always reach the target shape in
finite time.

In this thesis, we limit ourselves to informal correctness proofs for LM
programs. Unfortunately, there are no available meta reasoning tools for linear
logic based programs and proof mechanization is beyond the scope of this work.
