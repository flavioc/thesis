This chapter describes the multicore implementation of the LM language,
including its compiler, supporting runtime and parallelism. We first start with
an overview of the implementation in order to understand how all pieces of the
implementation fit together. Secondly, we describe how parallelism is achieved,
including its data structures and thread scheduling. Thirdly, we present the
implementation details of coordination and how it relates with parallelism.  We
then describe the runtime and data structure used to implement nodes and the
database of facts, followed by the compilation algorithm used by our compiler to
turn logical rules into efficient C++ code.

\section{Overview}
The implementation of LM is composed of a compiler and a virtual machine~(VM).
Figure~\ref{fig:implementation:overview} presents an overview of the compilation
process for LM programs. The two main boxes represent the two major components
of the system, namely, the compiler and virtual machine.

\begin{figure}[ht]
  \centering
  \includegraphics[width=.75\linewidth]{figures/implementation/overview.pdf}
  \caption{Compilation of a LM program into an executable. The compiler
     transforms a LM program into \code{file.cpp}, C++ file with compiled
     rules, and a \code{data} file with the graph structure and axioms. The virtual
     machine which includes code for managing multithreaded execution and the
     database of facts is then linked with \code{file.cpp} to create a
     standalone executable that can be run in parallel.}
  \label{fig:implementation:overview}
\end{figure}

The virtual machine contains supporting data structures for managing the
database of facts and to schedule the execution of rules. The parallel engine is
also a major part of the virtual machine and is responsible for managing
multithreaded execution by launching threads, managing communication and
and scheduling parallel execution, including coordination.

The compiler transforms LM files into C++ code that use the virtual machine
facilities to implement the program logic.  The compiled code implements the
inference rules of the program and uses the API of the virtual machine to derive
and retract facts and to schedule the execution of rules.  The compiler also
creates a separate file, named \code{file.data}, with the program's axioms and
graph structure. The graph structure is reconstructed in parallel once the
program starts.

To complete the compilation process, we use a C++ compiler to compile the
virtual machine files and \code{file.cpp} into object files that are then
linked along with \code{file.data}. At the end, we have a standalone
executable that allows the user to input the number of threads to use,
scheduling strategies (i.e., disable coordination), run time measurement
facilities and also database printing facilities.

Alternatively, the programmer may also decide to compile a more general version
of the virtual machine that is able to run byte-code files generated by the
compiler. This allows faster development since the programmer only needs to
recompile the LM program and not the whole stack. However, LM programs will run
slower since the byte-code must be interpreted by the virtual machine. This
severely affects programs with many mathematical operations, especially floating
point computations.

\iffalse
\subsection{Graph Clustering}

The graph structure stored in file \code{file.data} is constructed by the
compiler by analyzing the program's axioms and then ordering the nodes of the
graph.  In order to distribute computation across threads it is important to
increase locality of communication, so that a node makes most of its
communication to neighbor nodes that are being handled by the same thread. The
graph structure in \code{file.data} is written in order to improve locality
and reduce communication between threads.

The compiler analyzes the node address constants (that are prepended by the
symbol @) and the axioms of the program. After parsing and type-checking the
program code, the compiler then optimizes the topology by building an internal
representation of the graph.  In this phase, each node address $a$ is mapped
using a function $M(x)$ to a normalized node address $n$. Function $M(x)$ is
bijective and the domain is the set of all nodes described in the source code.
The co-domain of $M$ is the discrete interval $[0, N[$, where $N$ is the number
of nodes in the graph. The byte-code of a LM program includes all the pairs $(x,
M(x))$ so that the runtime system can put this information to use.

We have three methods for defining the function $M(x)$:

\begin{itemize}
   \item \emph{Static}: the compiler uses the node addresses presented in the
      code as long as they fill the discrete interval $[0, N[$.
   \item \emph{Randomized}: the mapping is done randomly.
   \item \emph{Breadth-First}: the mapping is built by picking an arbitrary node, $n_{zero}$
   and setting $M(n_{zero}) = 0$, then we select all neighbors of $n_{zero}$ and start defining
   their mappings in increasing order, $1, \dotsc, N-1$, and adding its neighbors for later processing
   in a breadth first fashion.
\end{itemize}

The breadth-first method is used with the intent of clustering closer nodes in
an ordered fashion.  While not optimal, using a breadth-first approach is very
efficient and has good results for irregular graphs. If we use a static division
of work between $N$ threads, where each thread is responsible to process a
pre-defined set of nodes, we can efficiently slice the codomain of function
$M(x)$ and divide it between the $N$ threads.

For an example, consider the graph in Fig.~\ref{fig:compiler:topology1}. The
node addresses represented are the ones included in the source code. Using a
breadth-first method starting by node 1, we get the following order: 1, 2, 3, 7,
5, 6 and 4. If we had to do a static division with 2 threads, thread 1 would get
1, 2, 3, 7 and thread 2 would get 5, 6 and 4. Note from
Fig.~\ref{fig:compiler:topology1} that only 3 edges exist between the nodes of
thread 1 and thread 2. This greatly reduces communication between threads and
improves parallel efficiency.

\begin{figure}[ht]
  \centering
  \includegraphics[width=0.6\textwidth]{figures/compiler/topology1.pdf}
  \caption{Topology using a breadth-first method.}
  \label{fig:compiler:topology1}
\end{figure}
\fi


\section{Parallelism}\label{sec:implementation:parallelism}
A key goal of our parallel design is to keep the threads as busy as possible and
to reduce inter-thread communication. Initially, the VM partitions the
application graph of $N$ nodes into $T$ subgraphs (the number of threads) and
then each thread works on their own subgraph. During execution, threads can
steal nodes of other threads to keep themselves busy. The load balancing aspect
of the system is performed by our work scheduler that is based on a simple work
stealing algorithm.

While our VM uses shared memory for thread communication, each
thread has a logical space that contains its subgraph, i.e., the nodes owned by
the thread, and a \emph{Work Queue}, which contains \textbf{active} nodes, i.e.,
nodes that have new facts to process.  The work queue is implemented as a linked
list. Initially, the work queue is filled with the nodes in the thread's
subgraph in order to derive the initial facts. A thread may go through different
states during its lifetime. The state is kept track in the \emph{State} flag
which may have one of the following values:

\begin{itemize}

   \item \textbf{active}: The thread has a non-empty \emph{Work Queue} or is
      currently executing rules for a node (which is not in the \emph{Work
      Queue}).

   \item \textbf{stealing}: The thread has no active nodes in the \emph{Work
      Queue} and is attempting to steal nodes from other
      threads.

   \item \textbf{idle}: The thread is trying to synchronize with other threads
      to terminate the program.
   
   \item \textbf{terminated}: The thread (and all the other threads) have
      terminated and the program is finished.

\end{itemize}

Figure~\ref{fig:implementation:thread_states} presents the valid transitions for
the thread state flag. The dashed line from \textbf{idle} to \textbf{stealing}
indicates that the transition is made in a non-deterministic fashion.

\begin{figure}[ht]
   \centering
   \includegraphics[width=0.65\textwidth]{figures/implementation/thread_states.pdf}
   \mycap{The thread state machine as represented by the \emph{State} flag. During
      the lifetime of a program, each thread goes through different states as
      specified by the state machine.}
   \label{fig:implementation:thread_states}
\end{figure}

The pseudo-code for the main thread loop is shown in
Fig.~\ref{alg:thread_work_loop}. In each round, a thread inspects its \emph{Work
Queue} and while there are active nodes, procedure \code{process\_node()}
(complete pseudo-code in Fig.~\ref{alg:multicore:process_node}) will perform
local computation on active nodes. When a thread's \emph{Work Queue} is empty,
it attempts to steal half of the nodes from another thread. Starting from a
random thread, it cycles through all the threads to find one active thread from
whom it will try to steal half of its nodes. If the thread fails to steal work,
it will go \textbf{idle} and periodically attempt to steal work from another
active thread. Eventually, all threads will fail to steal work since there is no
more work to do and they will go idle.  There is a global atomic counter that is
used to detect termination. Once a thread goes idle, it decrements the global
counter and changes its flag to idle.  Since every thread will be busy-waiting
and checking the global counter, they will detect a zero value and stop
executing, transitioning to the \textbf{terminated} state.

\begin{figure}
\begin{algorithm}[H]
   \KwData{Thread TH}
   \While{true}{
      $TH.work\_queue.lock()$\;
      $node \longleftarrow TH.work\_queue.pop\_node()$ \;
      $TH.work\_queue.unlock()$\;
      \uIf{$node$}{
         $process\_node(node)$\;
      }
      \Else{
         \tcp{Attempt to steal some nodes.}
         \If{$\bang TH.steal\_nodes()$}{
            $TH.become\_idle()$\;
            \While{$len(TH.work\_queue) == 0$}{
               \tcp{Try to terminate}
               \If{$TH.synchronize\_termination()$}{
                  \textbf{terminate}\;
               }
               \If{$TH.steal\_nodes()$}{
                  \tcp{Thread is still in the stealing state}
                  break\;
               }
            }
            \tcp{There's new nodes in the queue.}
            $TH.become\_active()$\;
         }
      }
 }
\end{algorithm}
\mycap{Thread work loop: threads process active nodes from the work queue
   until no more active nodes are available. Node stealing using a \emph{steal
      half} strategy is employed when the thread has no more active nodes.}
 \label{alg:thread_work_loop}
\end{figure}

Figure~\ref{fig:implementation:vm_overview} ties everything together and
presents the layout of our virtual machine for a program with six nodes and two
running threads. In the figure, we represent the thread's \emph{Work Queue} and
the thread logical space where the thread's subgraph is located. We also show
the internals of node \code{@1} and the thread operations that force threads to
interact with the node data structures.

\begin{figure*}[t]
\centering
\includegraphics[width=\textwidth]{figures/implementation/vm_overview.pdf}
\mycap{Layout of the virtual machine. Each thread has a work queue that
   contains active nodes (nodes with facts to process) that are processed one
   by one by the thread. Communication between threads happens when nodes
   send facts to nodes located in other threads.}
\label{fig:implementation:vm_overview}
\end{figure*}

In order to understand how threads interact between each other, we now review
the node data structure that was presented in Section~\ref{sec:data_structures}.
The node lock \emph{DB Lock} protects the data structures of the database,
including the array, trie, linked list and hash table data structures and the
\emph{Rule Engine}. The \emph{State Lock} structure protects everything else,
especially the \emph{State} flag and the temporary set of facts represented by
the \emph{Fact Buffer}. The \emph{Incoming Fact Buffer} is used to hold logical facts
that can not be added immediately to the database data structures. The
\emph{Owner} field points to the thread responsible for processing the node and
the \emph{Rule Engine} schedules local computation.

Whenever a new fact is derived through rule derivation, we need to update the
data structures for the corresponding node. If the node is currently being
executed by the thread (local send), then the fact is added to the node data
structures since the \emph{DB Lock} is being held while the node is being
processed. If that is not the case, then we have to synchronize since multiple
threads might be updating the same node's data structures. For example, in
Fig.~\ref{fig:implementation:vm_overview}, when thread 2 derives a fact to node
\code{@1} (owned by thread 1), it first locks node \code{@1} using the
\emph{State Lock} and then it attempts to lock \emph{DB Lock}, which gives thread
2 full access to the node. In this case, thread 2 adds the new fact to the
database (\emph{New fact (2)} in Fig.~\ref{fig:implementation:vm_overview}) and
to the \emph{Rule Engine}. However, if the \emph{DB Lock} could not be acquired
because the node \code{@1} is currently being processed, then the new fact is
added to \emph{Incoming Fact Buffer} (\emph{New fact (1)} in
Fig.~\ref{fig:implementation:vm_overview}). The facts stored in \emph{Fact
Buffer} will then be processed whenever the corresponding node is processed
again.

When a thread interacts with another thread to send a fact, it also needs to
make sure that the target node is made \textbf{active} (see
Fig.~\ref{fig:local:node_states}) and that it is also placed in the target
thread's \emph{Work Queue} (\emph{Activate node} in
Fig.~\ref{fig:implementation:vm_overview}). To handle concurrency issues, we
have a per \emph{Work Queue} lock called the \emph{Queue Lock} that is held when
the \emph{Work Queue} is being operated on.  As an example, consider again the
situation in which thread 2 sends a new fact to node \code{@1}. If node
\code{@1} is not active, then thread 2 also needs to activate node \code{@1} by
pushing it to the \emph{Work Queue} of thread 1.  After this synchronization
point, the target thread is ensured to be active and with a new node to process.

\begin{figure}
\begin{algorithm}[H]
   \KwData{Node N}
   $N.state\_lock.lock()$\;
   $N.db\_lock.lock()$\;
   \tcp{Add facts from the Incoming Fact Buffer into the database}
   $N.DB.merge(N.incoming\_fact\_buffer)$\;
   $N.state\_lock.unlock()$\;

   $N.rule\_engine.run\_rules()$\;
   $N.db\_lock.unlock()$\;

   \tcp{Check if node N is done for now}

   $N.state\_lock.lock()$\;
   \If{$N.rule\_engine.has\_candidate\_rules()$ or \\
      \hspace{2cm} $N.incoming\_fact\_buffer.has\_facts()$}{
      $N.state\_lock.unlock()$\;
      goto beginning\;
   }
   $N.state \longleftarrow inactive$\;
   $N.state\_lock.unlock()$\;
\end{algorithm}
\mycap{Pseudo-code for the \code{process\_node} procedure.}
 \label{alg:multicore:process_node}
\end{figure}


\section{Node Data Structure}\label{sec:data_structures}
The main characteristic of LM rules is that they are constrained by the first
argument\footnote{In the implementation, the first argument of each fact is not
stored.}. Rule derivation uses only facts from the same node, therefore there is
no need to synchronize with other nodes to derive rules. However, when nodes
derive non-local facts (owned by other nodes), then the implementation must
synchronize and \emph{send} the facts to the target node. From the point of view
of the receiving node, these are called \emph{incoming facts}. Note that this is related
to the parallel aspects of the virtual machine and more details are given in the
next chapter.

As shown in Fig.~\ref{fig:local:node_overview}, each node contains the
following attributes:

\begin{itemize}
   \item \emph{State}: the node state flag which can be one of the following:
      \textbf{running}, the node is deriving rules; \textbf{inactive}, the node
      has no new facts to be considered and all candidate rules have been tried;
      \textbf{active}, the node has new facts to be considered but is waiting to
      be executed by a thread. Figure~\ref{fig:local:node_states} presents the
      state machine with the valid state transitions.
   \item \emph{Main Lock}: a lock that protects the node attributes enumerated
      in this list except for \emph{Linear DB} and \emph{Persistent DB}.
   \item \emph{DB Lock}: a lock that protects \emph{Linear DB},
      \emph{Persistent DB} and \emph{Rule Engine}. The lock is held when the node is deriving rules or
      when incoming facts from other nodes need to be added to the database data
      structures.
   \item \emph{Rule Engine}: a data structure that detects which rules are
      candidates and should be derived. The data structure is fully explained
      in Section~\ref{section:local:rule_engine}.
   \item \emph{Owner}: a pointer to the thread responsible for executing this
      node.
   \item \emph{Fact Buffer}: a data structure that holds incoming facts that
      could not be added to the database data structures since the node is
      currently deriving rules.
   \item \emph{Linear DB}: the database of linear facts as an array of data
      structures for storing linear facts for each linear predicate.
   \item \emph{Persistent DB}: the database of persistent facts.
\end{itemize}

\begin{figure}[ht]
   \centering
   \includegraphics[width=0.55\textwidth]{figures/local/node_state.pdf}
   \caption{The node state machine as represented by the state variable. During
      the lifetime of a program, each node goes through different states as
      specified by the state machine.}
   \label{fig:local:node_states}
\end{figure}

\begin{figure*}[t]
\centering
\includegraphics[width=0.5\textwidth]{figures/local/node.pdf}
\caption{Layout of the node data structure.}
\label{fig:local:node_overview}
\end{figure*}

\paragraph{Database data structures}

The database of facts must be implemented efficiently because during matching
of rules we need to restrict the facts using \emph{join constraints}, which fix
arguments of predicates to instantiated values. A database fact is made up of 2
pointers (\code{prev} and \code{next}) and a variable number of arguments.
Each argument is large enough to contain a value of any of the available LM types.
Facts are stored in one of the four data structures:

\begin{itemize}

\item \emph{Trie Data Structures} are used to store persistent facts. Tries are
   trees where facts are indexed by common prefix arguments. The \code{prev}
   and \code{next} pointers are used to store all the facts stored in the trie.

\item \emph{Array Data Structures} are used to store persistent facts that are
   only derived as initial facts and used in rules LHS. Facts stored in this
   data structure do not have the \code{prev} and \code{next} pointers
   because they are already chained by being part of a contiguous memory area.

\item \emph{Doubly Linked List Data Structures} are used to store linear facts.
   We use a double linked list because it is a very efficient way to add and
   remove facts. The \code{prev} and \code{next} pointers are used to chain
   the facts of the linked list.

\item \emph{Hash Table Data Structures} are used to improve lookup when linked
   lists are too long and when we need to do search filtered by a fixed
   argument. The virtual machine decides which arguments are best to be indexed
   (see Section~\ref{sec:implementation:indexing}) and then uses a hash table
   indexed by the appropriate argument. If we need to go through all the facts,
   we just iterate through all the facts in the table. For collisions, we use
   the doubly linked list data structure mentioned above.

\end{itemize}

Figure~\ref{fig:implementation:hash_table} shows an example for a hash table
data structure for a \code{a(int,int)} predicate with 3 linear facts indexed
by the second argument and stored as a doubly linked list in bucket \code{2}.

\begin{figure}[ht]
\centering
\includegraphics[width=0.6\textwidth]{figures/implementation/hash_table.pdf}
\caption{Hash table and doubly linked data structures for 
   a \texttt{a(int,int)} predicate containing the following facts: \code{a(1,
   2)}, \code{a(2, 12)}, \code{a(2, 42)}.}
\label{fig:implementation:hash_table}
\end{figure}

%%%%%%%%%%%%%%%%%%%%%%%%%%%%%%%%%%%%%%%%%%%%%%%%%%%%%%%%%%%%%%%%%%%%%%

\subsection{Indexing}\label{sec:implementation:indexing}

To improve fact lookup, the VM employs a fully dynamic mechanism to
decide which argument may be optimal to index.  The algorithm is
performed in the beginning of execution and empirically tries to
assess the argument of each predicate that more equally spreads the
database across the values of the argument. 

The indexing algorithm is performed in three main steps. First, it
gathers lookup statistics by keeping a counter for each
predicate's argument.  Every time a fact search is performed where
arguments are fixed to a value, the counter of such arguments is
incremented. This phase is performed during rule execution for a small
fraction of the nodes in the program.

The second step of the algorithm selects the candidate arguments of each
predicate.  If a predicate was not searched with any fixed arguments, then it
will be not indexed and there are no candidates.  If only one argument was
fixed, then such argument is the only available candidate argument and thus
immediatelly becomes the indexing argument. Otherwise, the top 2 arguments are
selected for the third phase, where \emph{entropy statistics} are collected
dynamically.

During the third phase, each candidate argument has an entropy score.
Before a node is executed, the facts of the target predicate
are used in the following formula applied for the two arguments:

\[
Entropy(A, F) = - \sum_{v \in values(F, A)} \frac{count(F, A = v)}{total(F)} \log_2 \frac{count(F, A = v)}{total(F)}
\]

\noindent where $A$ is the target argument, $F$ is the set of linear facts for
the target predicate, $values(F, A)$ is set of values of the argument $A$,
$count(F, A = v)$ counts the number of linear facts where argument $A$ is equal
to $v$ and $total(F)$ counts the number of linear facts in $F$.  The entropy
value is a good metric because it tells us how much information is needed to
describe an argument. If more information is needed, then that must be the best
argument to index.

For one of the arguments to score, $Entropy(A, F)$ multiplied by the number of
times it has been used for lookup, must be larger than the other argument. The
argument with the best score is selected and then a global variable called
\texttt{indexing\_epoch} is updated. In order to convert the node's linked lists
into hash tables, each node also has a local variable called
\texttt{indexing\_epoch} that is compared to the global variable in order to
rebuild the database according to the new indexing information.

The VM also dynamically resizes the hash table if necessary. When the hash table
becomes too dense, it is doubled in size. When it becomes too sparse, it is
reduced in half or simply transformed back into a doubly linked list. This is
done once in a while, before a node executes.

We have seen very good results with this scheme. The overhead of dynamic
indexing is negligible since programs run almost as fast as if the indices have
been added from the start. However, the programmer can still index predicates
statically, if needed, using the directive \code{index pred/arg}, where
\code{pred} is the argument name and \code{arg} is the argument number to index.


\subsection{Communication}

At this point, we can now summarize the main thread synchronization hotspots in the
virtual machine. Threads synchronize with each other using mutual exclusion. We
use a spin-lock in the work queue to protect queue operations.  Given threads $T_1$
and $T_2$, we enumerate the most important synchronization hotspots:

\begin{itemize}

   \item \textbf{New facts}: When a node executes in $T_1$ and derives facts to
   a node in $T_2$, $T_1$ first buffers the facts and then sends them to the
   target node. Here, it checks if the node is currently \textbf{idle} and then
   synchronizes with $T_2$ to add the node to the $T_2$'s queue.  If $T_2$ is
   inactive  then $T_1$ also synchronizes with $T_2$ to change $T_2$'s state to
   \emph{active}.

   \item \textbf{Node stealing}: $T_1$ synchronizes with $T_2$ when it attempts
   to steal nodes from $T_2$ by removing half of the nodes from one of $T_2$'s
   queues.

\end{itemize}

In order to protect these important synchronization points, we use ticket spin-locks.
We now summarize the locks used in the different data structures of the VM:

\begin{description}
   \item[Queue Locks] A lock per queue data structure that protects access to
      the work queue.
   \item[Node Locks] Lock refered in Section~\ref{sec:node_state_machine} and used to protect the state of the node. We have one node lock per node.
   \item[Database Locks] Locks that protect the database data structures of each node. The lock is activated when the node is executing and can be grabbed by other threads when a fact is sent to another thread. We have one database lock per node.
\end{description}

In order to manipulate the state flag of each thread (see Section~\ref{sec:implementation:parallelism}) we do not use locks but instead manipulate the state flag using lock-free \emph{compare-and-swap} operations to implement a state machine.

Finally, we now improve the description of the synchronization hotspots by describing how the locks are used in order to implement those hotspots.

\begin{description}
\item \textbf{New Facts}: We use the node lock and then attempt to lock the database lock. If the database lock cannot be used, then the new facts are added to the fact buffer, otherwise the node data structure is updated with new facts. If the target node is not currently in any queue, we lock the destination queue (either the standard queue or the priority queue) and then add the node and change the state of the node to \textbf{queue}. Finally, if the target thread that owns the target node is idle, we activate it by updating its state flag.
\item \textbf{Node Stealing}: For node stealing, we simply have to lock the queue spin-lock of the target thread's queue and then copy the nodes we need to a temporary buffer. Once the buffer is filled up, we unlock the queue's spin-lock. For each node, we use the node lock to update its \textbf{owner} field and then add it to thread's queue.
\end{description}



\section{Memory Allocation}

Memory allocation in the VM is extremely important because there is a need to
repeatedly allocate and deallocate logical facts. Logical facts tend to be small
memory objects, requiring, on average, 24 to 30 bytes of memory, which may lead
to fragmentation if allocation is not careful.  Moreover, since VM is
multithreaded, allocation also needs to be scalable, therefore using the
standard \code{malloc} facility provided by the POSIX standard may not be the
best idea since each operating system uses a different implementation that may
or may scale well in multithreaded environments.

\begin{figure}[ht]
   \begin{center}
      \includegraphics[width=0.7\linewidth]{figures/implementation/pool.pdf}
   \end{center}
   \caption{Thread allocator: each thread has a pool of slabs for allocating
      data structures. Each object size has: (1) several slabs that store contiguous chunks of
      memory of the same size; (2) a free list \code{free} of chunks that were freed; and (3) a \code{next} pointer that points to the next available free chunk.}
   \label{fig:implementation:pool}
\end{figure}

In order to solve these two issues, we decided to implement an allocator based
on the SLAB allocator~\cite{Bonwick-94}. SLAB allocation is a memory management
technique created in the Solaris 5.4 kernel used for efficiently allocate kernel
objects. Its advantages include reduced fragmentation and improved reuse of
deallocated data structures since these are reused in newer allocations.

Our particular implementation is presented in Fig.~\ref{fig:implementation:pool}
We pre-allocate multiple pools of memory chunks. We create one pool per object
size or data structure and each pool allocates large contiguous chunks of memory
that contain multiple objects of the same size. When allocating a particular
data structure, we first lookup the pool that
handles objects of the fact size and then check if there is an available object
in the chunks of memory. If all the chunks are empty, we use the \code{next}
pointer to allocate a new object inside a chunk. If there is no available space
in the chunk pointed by \code{next}, then we allocate a bigger chunk. We also
have a pointer \code{free} that points to deallocate objects and which creates a
list of free objects inside the available chunks. If the list has free objects,
we use the object pointed by \code{free} and update the \code{free} pointer
accodingly.

In order to reduce thread contention in the allocator, each thread uses a
different instance of the threaded allocator. When a thread wants to allocate an
object, it asks its own allocator for a new object. When deallocating, a thread
may deallocate an object that is part of another thread's allocator. However,
this is not an issue since the thread allocator does not need to synchronize
because chunks are never garbage collected from the system.

\subsection{Fact Allocation}

Although threads only allocate from their own memory chunks, they may reuse
objects from chunks created by another thread. Consider the following sequence
of events: (1) thread \code{T1} allocates multiple facts for node \code{A} in
the memory chunk \code{C1}; (2) thread \code{T2} executes node \code{A} and
allocates several facts in its own memory chunk \code{C2}; (3) thread \code{T1}
executes \code{A} again and needs to iterate through \code{A}'s facts which are
located in different multiple chunks, resulting in poor memory locality and
increased cache line misses. It is obvious that, in a perfect scenario,
\code{A}'s facts should be placed in a contiguous memory area in order to reduce
cache misses.

To tackle these issues, we have implemented a per-node fact allocator that is
implemented on top of the threaded allocator. The fact allocator allocates
chunks of memory from the main allocator which are then used to allocate facts
for that particular node. When a thread needs to allocate or deallocate a fact,
it acquires the allocator lock of the target node's allocator and performs the
allocation operation. Each fact allocator can allocate multiple chunks of
memory, depending on the amount of required facts. There is a doubly-linked list
of memory chunks and each chunk contains facts of different sizes (predicates).

Figure~\ref{fig:implementation:fact_allocator} presents an example state of a
fact allocator. The node has 3 memory chunks, all connected using the
\code{next} and \code{prev} pointers. Each chunk also has a reference count
(\code{refcount}) of the facts allocated in the chunk. If the reference count
ever drops to zero, then the memory chunk is deallocated. Deallocated facts are
kept on an array of linked lists named \code{free\_list} (contiguous in the
chunk, not as separate data structure), where each position of the array is a
linked list for facts with the same size. When allocating a new fact, we use the
first element of the linked list for facts with the same size.

\begin{figure}[ht]
   \begin{center}
      \includegraphics[width=0.7\linewidth]{figures/implementation/fact_allocator.pdf}
   \end{center}
   \caption{Fact allocator: each node has a pool of memory chunks for allocating
      logical facts. Each chunk contains: (1) several linked lists of free facts
      of the same size (\code{free\_list}); (2) a reference count of used facts
      (\code{refcount}); (3) a \code{ptr} pointer that points to unallocated
      space in the chunk. In this figure, predicates \code{f} and \code{g} have
   several deallocated facts that are ready to be used when a new fact needs to
be acquired.}
   \label{fig:implementation:fact_allocator}
\end{figure}


\section{Compilation}
As an intermediate step, our compiler first transforms rules into high level
instructions that are then transformed into C++. In Appendix~\ref{appendix:vm}
we present an overview of the high level instructions that can be used as a
reference to the operations that are required by the compiler. In this
section, we present the main algorithm of the compiler and its key
optimizations. However, all the examples are shown in C++ since it makes it
easier to understand how the final code looks like.

\subsection{Ordinary Rules}\label{sec:compile}

After an inference rule is compiled, it must respect the \emph{fact constraints}
(facts must exist in the database) and the \emph{join constraints} that can be
represented by variable constraints and/or boolean expressions. For instance,
consider gain the second rule of the SSSP program presented in
Fig.~\ref{code:shortest_path_program}:

\begin{Verbatim}[label=example_rule,fontsize=\codesize]
shortest(A, D1, P1), D1 <= D2, relax(A, D2, P2)
   -o shortest(A, D1, P1).
\end{Verbatim}

The fact constraints include the facts required to trigger the rule, namely
\texttt{shortest(A, D1, P1)} and \texttt{relax(A, D2, P2)}, and the join
constraints include the expression \texttt{D1 <= D2}.

However, rules may also have other less obvious join constraints such as:

\begin{Verbatim}[fontsize=\codesize]
new-neighbor-pagerank(A, B, New),
neighbor-pagerank(A, B, Old)
   -o neighbor-pagerank(A, B, New).
\end{Verbatim}

\noindent
where variable \texttt{B} must have the same value in both facts\footnote{Rule taken
from an asynchronous PageRank program.}.

\subsection{Iterators}

The data structures for facts presented in Section~\ref{sec:data_structures}
support the \emph{iterator} pattern. For linked lists, the iterator goes
through every fact in the list while the hash table iterator can either iterate
through the whole table or iterate through a single bucket. A bucket iterator is
in fact a linked list iterator that starts from a given argument.
For tries, while the default iterator goes through every fact in
the trie, it can be customized with a matching specification in
order to reduce search. A matching specification includes argument
assignments (e.g., argument $i = V$, where $V$ is a concrete value).

Iterators are heavily used in the compiled code. For instance, the second rule in
Fig.~\ref{code:shortest_path_program} is compiled as follows:

\begin{Verbatim}[numbers=left,fontsize=\codesize]
for(auto it1(linked_list("shortest").begin()); it1 != linked_list("shortest").end(); )
{
   fact *f1(*it1);
   for(auto it2(linked_list("relax").begin()); it2 != linked_list("relax").end(); )
   {
      fact *f2(*it2);

      if(f1->get_int(1) <= f2->get_int(1)) {
         fact *new_shortest(new fact("shortest"));
         new_shortest->set_int(1, f1->get_int(1));
         new_shortest->set_list(2, f1->get_list(2));

         // new fact was derived
         linked_list("shortest").push_back(new_shortest);

         // deleting facts
         it1 = linked_list("shortest").erase(it1); // remove from list
         it2 = linked_list("relax").erase(it2);
         return;
      }
      ++it2;
   }
   ++it1;
}
\end{Verbatim}


The compilation algorithm iterates through the fact expressions in the body of
the rule and creates nested loops to try all the possible combinations of facts.
For this rule, all the pairs of facts \texttt{shortest} and \texttt{relax} must
be matched until the constraint \texttt{D1 <= D2} is true. First, an iterator
for \texttt{shortest} is created that will loop through all \texttt{shortest}
facts in the list. Inside the loop, a nested iterator is created for predicate
\texttt{relax}. This inner loop includes a check for the \texttt{D1 <= D2}
constraint. If the constraint expression is true then the rule matches and a new
\texttt{shortest} fact is derived and two used linear facts are retracted by
erasing the iterators from the linked lists. Note that after the rule is
derived, the code must return since there is a higher priority rule that may be
triggered with the new \texttt{shortest} fact (see
Fig.~\ref{code:shortest_path_program}). This enforces the priority semantics
of the language.
    
If the constraint had failed, another \texttt{relax} fact would have been tried
by incrementing \texttt{it2}. Likewise, if the current \texttt{f1} fact fails
for all \texttt{f2} facts, then the next one in the list is tried
by incrementing \texttt{it1}.

\begin{figure}
\begin{algorithm}[H]
 \KwData{Rule R1, Rules}
 \KwResult{Compiled Code}
 $FactExprs \longleftarrow FactExprsFromRule(R1)$\;
 $Constraints \longleftarrow ConstraintsFromRule(R1)$\;
 $Code \longleftarrow CreateFunctionForRule()$\;
 $Iterators \longleftarrow []$\;
 $CompiledFacts = []$\;
 \While{$FactExprs$ not empty}{
  $Fact \longleftarrow RemoveBestFactExpr(FactExprs)$\;
  $CompiledFacts.push(Fact)$\;
  $Iterator \longleftarrow Code.InsertIterator(Fact)$\;
  $Iterators.push(Iterator)$\;
  \tcc{Select constraints that are covered by CompiledFacts.}
  $NextConstraints \longleftarrow RemoveConstraints(Constraints, CompiledFacts)$\;
  $Code.InsertConstraints(NextConstraints)$\;
 }
 $HeadFacts = HeadTemplatesFromRule(R1)$\;
 \While{$HeadFacts$ not empty}{
    $Fact \longleftarrow RemoveFact(HeadFacts)$\;
    $Code.InsertDerivation(Fact)$\;
 }
 \For{$Iterator \in Iterators$}{
    \If{$IsLinear(Iterator)$}{
       $Code.InsertRemove(Iterator)$\;
    }
 }
 \tcc{Enforce rule priorities.}
 \uIf{$FactsDerivedUsedBefore(Head, Program, R1)$}{
    $Code.InsertReturn()$\;
 }
 \Else{
    $Code.InsertGoto(FirstLinear(Iterators))$\;
 }
 \Return{$Code$}
\end{algorithm}
 \caption{Compiling LM rules into C++ code.}
 \label{alg:compile_rule}
\end{figure}

Figure~\ref{alg:compile_rule} presents the algorithm for compiling rules into
C++ code.  First we split the body of the rule into fact expressions and
constraints. Fact expressions map directly to iterators while fact constraints
map to \emph{if} expressions. A possible compilation strategy is to first
compile all the fact expressions and then compile the constraints. However, this
may require unneeded database lookups since some constraints may fail early.
Therefore, our compiler introduces constraints as soon as all the variables in
the constraint are all included in the already compiled fact expressions. The
order in which fact expressions are selected for compilation does not interfer
with the correctness of the compiled code, thus our compiler selects the fact
expressions ($RemoveBestFactExpr$) by their potential to activate constraints,
therefore avoiding undesirable database lookups. If two fact
expressions have the same number of new constraints, then the
compiler always picks the persistent fact expression since
persistent facts are not deleted.

Derivation of new facts belonging to the local node implies adding the new fact
to the local node data structure. Facts that belong to other nodes are sent
using an appropriate runtime API.

\subsection{Persistence Checking}

Not all linear facts need to be deleted. For instance, in the compiled rule
above, the fact \texttt{shortest(A, D1, P1)} is re-derived in the head. Our
compiler is able to turn linear loops into persistent loops for linear facts
that are retracted and then asserted.  The rule is then compiled as follows:

\begin{Verbatim}[numbers=left,fontsize=\codesize,commandchars=\$\#\&]
for(auto it1(linked_list("shortest").begin()); it1 != linked_list("shortest").end(); )
{
   fact *f1(*it1);
   for(auto it2(linked_list("relax").begin()); it2 != linked_list("relax").end(); )
   {
      fact *f2(*it2);
      if(f1->get_int(1) <= f2->get_int(1)) {
         it2 = linked_list("relax").erase(it2);
         goto next;$label#line:implementation:goto&
      }
      ++it2;
next: continue;
   }
   ++it1;
}
\end{Verbatim}

In this new version of the code, only the \texttt{relax} facts are deleted,
while the \texttt{shortest} facts remain untouched. In the SSSP program, each
node has one \texttt{shortest} fact and this compiled code simply filters out
the \texttt{relax} facts with the distances that are equal or greater than the
current best distance. Note that now have a \emph{goto statement}
(line~\ref{line:implementation:goto}) that is executed when the rule is fired.
In this case, since no new \texttt{shortest} fact was derived, we can avoid
returning to enforce rule priorities and we can continue to try to fire the rule
as many times as possible.

All the rule combinations are attempted in cases where a rule does not derive
any facts or the facts derived do not appear before the rule, that is, the new
facts are only used in lower priority rules. This is specified in the final
\emph{if statement} in Fig.~\ref{alg:compile_rule}. If the rule does not return,
then we always jump to the first loop that uses linear facts. We must jump to
the first linear loop because we cannot use
the next fact from the deepest loop since we may have constraints between the
first linear loop and the deepest loop that were validated using deleted facts.

\subsection{Updating Facts}

Many inference rules retract and then derive the same predicate but with
different arguments. The compiler recognizes those cases and instead of
retracting the fact from its linked list or hash table, it updates the fact
in-place. As an example, consider the following rule:

\begin{Verbatim}[fontsize=\codesize]
new-neighbor-pagerank(A, B, New),
neighbor-pagerank(A, B, Old)
   -o neighbor-pagerank(A, B, New).
\end{Verbatim}

Assuming that \texttt{neighbor-pagerank} is stored in a hash table and indexed by the
second argument, the code for the rule above is as follows:

\begin{Verbatim}[numbers=left,fontsize=\codesize]
for(auto it1(linked_list("new-neighbor-pagerank").begin()); it1 !=
      linked_list("new-neighbor-pagerank").end(); )
{
   fact *f1(*it1);
   // hash table for neighbor-pagerank is indexed by the second argument
   // therefore we search for the bucket using the second argument of new-neighbor-pagerank
   hash_bucket bucket(hash_table("neighbor-pagerank").find(f1->get_node(1));
   for(auto it2(bucket.begin()); it2 != bucket.end(); )
   {
      fact *f2(*it2);
      if(f1->get_node(1) == f2->get_node(1)) {
         f2->set_float(2, f1->get_float(2)); // update neighbor-pagerank
         it1 = linked_list("new-neighbor-rank").erase(it1);
         goto next;
      }
      ++it2;
   }
   ++it1;
next: continue;
}
\end{Verbatim}

Note that \texttt{neighbor-pagerank} is updated using \texttt{set\_float}. The
rule also does not return since this is the highest priority rule. If there
was a higher priority rule using \texttt{neighbor-pagerank}, then the code
would have to return since an update fact represents a new fact.

\subsection{Enforcing Linearity}

We have already introduced the \emph{goto} statement as a way to avoid reusing
retracted linear facts. However, this is not enough in order to enforce
linearity of facts. Consider the following inference rule:

\begin{Verbatim}[fontsize=\codesize]
add(A, N1), add(A, N2) -o add(A, N1 + N2).
\end{Verbatim}

Using the standard compilation algorithm, two nested loops are created, one for
each \texttt{add} fact. However, notice that there is an implicit constraint
when creating the iterator for \texttt{add(A, N2)} since this fact cannot be the
same as the first one. That would invalidate linearity since a single linear fact would
be used to prove two linear facts. This is easily solved by adding a constraint
for the inner loop by checking if the fact pointer is the same as the first one.

\begin{Verbatim}[numbers=left,fontsize=\codesize]
for(auto it1(linked_list("add").begin()); it1 != linked_list("add").end(); )
{
   fact *f1(*it1);
   for(auto it2(linked_list("add").begin()); it2 != linked_list("add").end(); )
   {
      fact *f2(*it2);
      if(f1 != f2) {
         f1->set_int(1, f1->get_int(1) + f2->get_int(1));
         it2 = linked_list("add").erase(it2);
         goto next;
      }
      ++it2;
   }
   ++it1;
next: continue;
}
\end{Verbatim}

Figure~\ref{fig:update_add} presents the steps for executing this rule when the
database contains three facts. The iterators never point to the same fact.

\begin{figure}
\centering
\begin{minipage}{.5\textwidth}
  \centering
  \includegraphics[width=.8\linewidth]{figures/compiler/update}
\end{minipage}%
\begin{minipage}{.5\textwidth}
  \centering
  \includegraphics[width=0.8\linewidth]{figures/compiler/update2}
\end{minipage}
\begin{minipage}{.5\textwidth}
   \centering
   \includegraphics[width=0.8\linewidth]{figures/compiler/update3}
\end{minipage}
\caption{Executing the add rule. First, the two iterators point to
   the first and second facts and the former is updated while the latter is
   retracted. The second iterator then moves to the next fact and the first fact is
   updated again, now to the value \texttt{6}, the expected result.}
\label{fig:update_add}
\end{figure}

\subsection{Comprehensions}

Comprehensions were initially presented in the first rule of the SSSP program.

\begin{Verbatim}[fontsize=\codesize]
shortest(A, D1, P1), D1 > D2, relax(A, D2, P2)
   -o shortest(A, D2, P2),
   {B, W | !edge(A, B, W) | relax(B, D2 + W, P2 ++ [B])}.
\end{Verbatim}

The attentive reader will remember that comprehensions are sub-rules, therefore
they should be compiled like normal rules. However, they do not need to return
due to rule priorities since all the combinations of the comprehension must be
derived. However, the rule itself must return if any of its comprehensions
has derived a fact that is used by a higher priority rule.
The example rule does not need to return since it has the highest priority and the
\texttt{relax} facts derived in the comprehension are all sent to other nodes.
The code for the rule is shown below:

\begin{Verbatim}[numbers=left,fontsize=\codesize]
for(auto it1(linked_list("shortest").begin()); it1 != linked_list("shortest").end(); )
{
   fact *f1(*it1);
   for(auto it2(linked_list("relax").begin()); it2 != linked_list("relax").end(); )
   {
      fact *f2(*it2);
      if(f1->get_int(1) > f2->get_int(1)) {
         // comprehension code
         for(auto it3(trie("edge").begin()); it3 != trie("edge").end(); ) {
            fact *f3(*it3);
            fact *new_relax(new fact("relax"));
            new_relax->set_int(1, f2->get_int(1) + f3->get_int(2));
            new_relax->set_list(append(f2->get_list(2), list(f3->get_node(1))));
            send_fact(new_relax, f3->get_node(1));
            ++it3;
         }
         f1->set_int(1, f2->get_int(1));
         f1->set_list(2, f2->get_list(2));
         it2 = linked_list("relax").erase(it2);
         goto next;
      }
      ++it2;
   }
   ++it1;
next: continue;
}
\end{Verbatim}

Special care must be taken when the comprehension's sub-rule uses the same
predicates that are derived by the main rule.
Rule inference must be atomic in the sense that after a rule matches, the
comprehensions in the head of the rule can use the facts that were present
before the body of the rule was matched.
Consider a rule with $n$ comprehensions or aggregates, where $CB_i$ and $CH_i$
is the body and head of the comprehension/aggregate, respectively, and $H$
represents the fact templates found in the head of the rule.
The formula used by the compiler to detect conflicts between predicates is the
following:

\[
\bigcup^{n}_i[CB_i \cap H] \cup \bigcup^{n}_i [CB_i \cap \bigcup^{n}_j[CH_j]]
\]

If the result of the formula is not empty, then the compiler disables
optimizations for the conflicting predicates and derives the corresponding facts
into the fact buffer that are then added back into the database.
Fortunately, most rules in LM programs do not show conflicts and thus
can be fully optimized.

\subsection{Aggregates}

Aggregates are similar to comprehensions. They are also sub-rules but a value is
accumulated for each combination of the sub-rule. After all the combinations are
inferred, a final head term is derived with the accumulated term. Consider the following
PageRank rule:

\begin{Verbatim}[fontsize=\codesize]
update(A), pagerank(A, OldRank)
      -o [sum => V | B | neighbor-pagerank(A, B, V) | neighbor-pagerank(A, B, V) |
            pagerank(A, damp/P + (1.0 - damp) * V)].
\end{Verbatim}

The variable \texttt{V} is initialized to \texttt{0.0} and sums all
the PageRank values of the neighbors as seen in the code below. The aggregate
value is then used to update the second argument of the initial
\texttt{pagerank} fact.

\begin{Verbatim}[numbers=left,fontsize=\codesize]
for(auto it1(linked_list("pagerank").begin()); it1 != linked_list("pagerank").end(); )
{
   fact *f1(*it1);
   for(auto it2(linked_list("update").begin()); it2 != linked_list("update").end(); )
   {
      fact *f2(*it2);
      double acc(0.0); // aggregate accumulator.
      for(auto it3(linked_list("neighbor-pagerank").begin()); it3 !=
            linked_list("neighbor-pagerank").end(); ) {
         fact *f3(*it3);
         acc += f3->get_float(2);
         ++it3; // the sub-rule has no head since neighbor-pagerank is re-derived
      }
      // head of the aggregate
      f1->set_float(1, damp / P + (1.0 - damp) * V);
      goto next;
   }
   ++it1;
next: continue;
}
\end{Verbatim}

\subsection{Coordination Directives}

Coordination directives are compiled in two different ways, depending on whether
they appear in the body or in the head of the rule. Coordination facts in the
body are compiled into code that inspects the state of the virtual machine. For
example, the fact \texttt{priority} will inspect the target node, retrieve the
current priority and assign the priority to a variable.  Coordination facts in
the head of the rule are also implemented as API commands of the virtual
machine instead of being added to the database as facts. Semantically, action
facts are like any other. However, since they are immediately used by the
machine, there is no need to store them in the database, therefore avoiding
unnecessary allocations and deallocations.



\section{Related Work}
\paragraph{Virtual Machines}

Virtual machines are a popular technique for implementing interpreters for high
level programming languages. Due to increased availability of parallel machines
and distributed architectures, several machines have been developed with
parallelism in mind~\cite{Kara:1997:AMM:265274}. One example of such machine is
the Parallel Virtual Machine~(PVM)~\cite{Sunderam90pvm:a}, which servers as an
abstraction to program heterogeneous computers as a single machine. Another
important machine is the Threaded Abstract
Machine~(TAM)~\cite{CullerGSvE93,goldstein-tr94}, which defines a self-scheduled
machine language of parallel threads where a program is represented using
conventional control flow.

Prolog, the most prominent logic programming language, has a rich history of
virtual machine research centered around the Warren Abstract
Machine~(WAM)~\cite{AICPub641:1983}. The WAM offers special purpose
instructions, including unification instructions for different kinds of data and
control flow instructions to implement backtracking. The WAM is fully sequential
and uses four memory areas for building and matching terms: heap, stack, trail
and the push down list. Because Prolog programs tend to be naturally parallel,
much research has been done to either parallelize the WAM or create new parallel
machines. Two types of parallelism are possible in Prolog:
\emph{OR-parallelism}, where several clauses for the same goal can be executed,
and \emph{AND-parallelism}, where goals in the same clause are tried in
parallel. For OR-parallelism, we have several models such as: the SRI
model~\cite{Warren:1987:OEM:67683.67699}, the Argonne
model~\cite{ButlerDLOOS88}, the MUSE model~\cite{Ali:1990fk} and the BC
machine~\cite{Ali88}. For AND-parallelism, different implementations were built
on top of the WAM~\cite{Hermenegildo:1986:AMB:913061,Lin:1988:AEL:900478},
however more general models such as the Andorra
model~\cite{Haridi:1990:KAP:87961.87964} were developed which allows both AND
and OR-parallelism.

\paragraph{Datalog Execution} The Datalog language is a forwards-chaining logic
programming and therefore requires different evaluation strategies than those
used by Prolog, which is a backwards-chaining logic programming language.
Arguably, the most well-known strategy for Datalog programs with recursive rules
is the \emph{Semi-Na\"{\i}ve Fixpoint}
algorithm~\cite{Balbin:1987:GDA:34657.34661}, where the computation is split
into iterations and the facts generated in the previous generation are used as
inputs to derive the facts in the next iteration. The advantage of this
mechanism is that no redundant computations are performed, which could happen in
the case of recursive rules but is avoided when the next iteration only uses new
facts derived by the previous iteration.
   
In the context of the P2 system~\cite{Loo-condie-garofalakis-p2}, which was
created for developing declarative networking programs, the previous strategy is
not suitable since it is centralized, therefore a new strategy called
\emph{Pipelined Semi-Na\"{\i}ve}~(PSN) evaluation was developed for distributed
computation. PSN evaluation evaluates one fact at the time by firing any rule
that is derivable using the new fact and older facts. If the new fact is already
in the database, then the fact is not used for derivation since it would derive
redudant facts.

 LM's evaluation strategy is similar to PSN, however, it considers the whole
 database of facts when firing rules due to the existence of rule priorities.
 For instance, when a node fires a rule, new facts added for the local node are
 considered as a whole in order to select the next inference rule. In the case
 of PSN, each new fact would be considered separately. Still, PSN and the LM
 strategy are both asynchronous strategies which take advantage of the nature of
 distributed and parallel architectures, respectively. Finally, an important
 distinction that should be made between PSN and LM is that PSN is for logic
 programs with only persistent facts, which results in deterministic results,
 however because LM uses linear facts, it follows a \emph{don't care} or
 \emph{committed choice} non-determinism, which may result in different results
 depending on the order of computations.

\paragraph{CHR implementations} Many basic optimizations used in the LM compiler
such as join optimizations and the use of different data structures for indexing
facts were inspired in work done on CHR~\cite{DBLP:journals/corr/cs-PL-0408025}.
Wuille et al.~\cite{42866} have described a CHR to C compiler that follows some
of the ideas presented in this chapter and De Koninck et al.~\cite{chrp} showed
how to compile CHR programs with dynamic priorities into Prolog. The novelty of
our work focuses on supporting a combination of comprehensions, aggregates and
rule priorities.



\section{Experimental Evaluation}
In this section, we evaluate the performance and scalability of the VM. The main
goals of this evaluation are as follows:

\begin{itemize}
   \item Compare the performance of LM programs against hand-written
      imperative C++ programs;
   \item Evaluate the scalability of said LM programs when using up to 20 cores
      concurrently;
   \item Understand the impact and effectiveness of our dynamic indexing
      algorithm and its indexing data structures used for logical facts (namely,
      hash tables);
   \item Understand the impact of the thread allocator and fact allocator on scalability and
      multicore performance;
\end{itemize}

For our experimental setup, we used a computer with a 24 (4x6) Core AMD
Opteron(tm) Processor 8425 HE $@$ 800 MHz with 64 GBytes of RAM memory running
the Linux Kernel 3.15.10-201.fc20.x86\_64. The C++ compiler used is the GCC
4.8.3 (g++) with the following \code{CXXFLAGS} flags: \code{-O3 -std=c++11
-march=x86-64}.  All experiments were executed 3 times and the running times
were averaged.

\subsection{Performance}\label{section:implementation:performance}
To understand the absolute performance of LM programs, we compared their
execution time using a single thread against hand-written sequential C++
programs. All C++ programs were compiled with the same compilation flags used
for LM for fairness. Arguably, compiled C/C++ programs are a good standard for
understanding the baseline performance of new language, since compiled C/C++
programs tend to come up on top on several popular programming language
benchmarks~\cite{language_benchmarks}. Furthermore, a sequential C++ program is
also a better baseline since it does not include any synchronization or
communication code. In order to make our comparison more interesting, we also
provide comparisons against the Python language and other relevant systems.
Python is a \emph{scripting} programming language that is much slower than
compiled C/C++ programs and thus can be seen as a good upper-bound in terms of
performance.

The goal of the evaluation is to understand the effectiveness of our compilation
strategy and the effectiveness of our dynamic indexing algorithms, including the
data structures (hash tables) used to index logical facts. We used the following
programs in our experiments\footnote{All programs available on
   \url{http://github.com/flavioc/misc}.}:

\begin{itemize}
   \item Belief Propagation: a machine learning algorithm to denoise images. Program is
      presented in Section~\ref{sec:coordination:bp}.

   \item Heat Transfer: an asynchronous program that performs transfer of heat
      between nodes. Program is presented in Section~\ref{section:coord:ht}.

   \item Multiple Single Shortest Distance (MSSD): a program that computes the
      shortest distance from a subset of nodes of the graph to all the nodes in
      the graph. A modified version is later presented in
      Section~\ref{section:coord:rationale}.

    \item MiniMax: the AI algorithm for selecting the best player move in a
       Tic-Tac-Toe game. The initial board was augmented in order to provide a
       longer running benchmark. Program is presented in
       Section~\ref{section:coord:minimax}.


   \item N-Queens: the classic puzzle for placing queens on a chess board so
      that no two queens threaten each other. Program is presented in
      Section~\ref{section:coord:nqueens}.

\end{itemize}

Table~\ref{table:implementation:absolute} presents the comparison between LM and
sequential C++ programs. Comparisons to other systems are shown under the
\textbf{Other} column. Since we also want to assess the VM's scalability, we use
different dataset sizes for each program.

\begin{table}[ht]
   \begin{center}
      \begin{tabular}{c | c || c | c | c} \hline
	\textbf{Program} & \textbf{Size} & \textbf{C++ Time} (s) & \textbf{LM} & \textbf{Other} \\ \hline \hline
	\multirow{4}{*}{Belief Propagation}  & 50x50 &  3.16  &  1.27  &  1.08 (GraphLab) \\
		 & 200x200 &  49.36  &  1.36  &  1.25 (GraphLab) \\
		 & 300x300 &  135.56  &  1.35  &  1.25 (GraphLab) \\
		 & 400x400 &  169.99  &  1.35  &  1.27 (GraphLab) \\
	\hline
	\multirow{2}{*}{Heat Transfer}  & 80x80 &  4.62  &  7.28  &  - \\
		 & 120x120 &  20.29  &  7.07  &  - \\
	\hline
	\multirow{6}{*}{MSSD}  & US 500 Airports &  0.69  &  2.76  &  13.44 (Python) 0.25 (Ligra) \\
		 & OCLinks &  7.00  &  7.35  &  16.10 (Python) 0.34 (Ligra) \\
		 & EU Email &  13.47  &  2.08  &  9.80 (Python) 0.32 (Ligra) \\
		 & Twitter &  27.22  &  8.58  &  8.28 (Python) 0.27 (Ligra) \\
		 & US Power Grid &  55.33  &  5.64  &  10.99 (Python) 0.30 (Ligra) \\
		 & Orkut &  281.59  &  1.66  &  4.23 (Python) 0.23 (Ligra) \\
	\hline
	\multirow{2}{*}{MiniMax}  & Small &  2.89  &  7.58  &  27.43 (Python) \\
		 & Big &  21.47  &  8.81  &  - \\
	\hline
	\multirow{4}{*}{N-Queens}  & 11 &  0.28  &  1.81  &  20.36 (Python) \\
		 & 12 &  1.42  &  3.09  &  24.30 (Python) \\
		 & 13 &  7.90  &  4.99  &  27.85 (Python) \\
		 & 14 &  47.90  &  6.42  &  31.52 (Python) \\
	\hline
\end{tabular}

   \end{center}

   \mycap{Experimental results comparing different programs against
      hand-written versions in C++. For the C++ programs, we show the execution
      time in seconds (\textbf{C++ Time (s)}). For the other approaches, we show
      the overhead ratio compared with the corresponding C++ program. The
      overhead numbers (\textbf{lower is better}) are computed by dividing the
   execution time of the approach on that column by the execution time of the
similar hand-written C++ program.}

   \label{table:implementation:absolute}
\end{table}

The Belief Propagation experiment is the program where LM performs the best when
compared to the C++ version. We found out that the mathematical operations
required to update the nodes belief values are expensive and make up a huge part
of the total computation time. This is clearly shown by the low overhead
numbers. We also compared our performance against the GraphLab and LM is only
slightly slower, which is also a point in LM's favor.

The Heat Transfer program behaves somewhat like Belief Propagation but LM is
almost an order of magnitude slower than the C++ version. We think this is
because the heat transfer computation is very small which tends to exacerbate
slower and repeated fact derivations from and to different nodes in the graph.

For the MSSD program, we used seven different datasets:

\begin{itemize}
   \item US 500 Airports~\cite{usairports,tnet}, a graph of the 500 top airports in the US with around
      5000 connections. The shortest distance is calculated for all nodes;
      
   \item OCLinks~\cite{tnet,oclinks}, a Facebook-like social network with around 2000 nodes and 20000 edges. The shortest
      distance is calculated for all nodes;

   \item US Power Grid~\cite{tnet,uspowergrid}, the power grid for western US with around 5000
      nodes and 13000 edges. The shortest distance is calculated for all nodes;

   \item Twitter~\cite{snapnets,NIPS2012_4532}, a graph with 81306 nodes and 1768149 edges.
      The shortest distance is calculated for 40 nodes; 

   \item EU Email~\cite{snapnets,Leskovec:2007:GED:1217299.1217301} a graph with
      265000 nodes and 420000 edges. The shortest distance is calculated for 100
      nodes;

   \item Orkut~\cite{snapnets,Yang:2012:DEN:2350190.2350193}, a large graph
      representing data from the Orkut social network. The graph contains
      3072441 nodes and 117185083 edges.

   \item Live Journal~\cite{snapnets,Backstrom06groupformation}, a large graph representing data from the
      Live Journal social network. The graph contains 4847571 nodes and 68993773
      edges.

\end{itemize}

The C++'s MSSD version applies the Dijkstra algorithm for each node we want to
compute the shortest distance from. While the Dijkstra algorithm has a better
complexity than the algorithm used in LM, LM's algorithm is able to
process distances from multiple sources at the same time. Our experiments show
that the C++ program effectively beats LM's version by a large margin, but that
gap is reduced when using more larger graphs such as EU Email. The Python
version also uses the Dijkstra algorithm and is one order of magnitude slower
than the C++ version and usually slower than LM. We also wrote the MSSD program
in the Ligra system~\cite{Shun:2013:LLG:2517327.2442530}\footnote{Available at
   \url{http://github.com/flavioc/ligra}.}, an efficient and scalable framework
   for writing graph-based programs. Our experiments show that Ligra is, on
   average, three times as fast as our C++ program, which means that LM is
   around 15 times slower than Ligra. Note that we compiled Ligra without
   parallel support.

For the MiniMax program, we used two different starting states, Small and Big,
where the tree generated by Big is ten times larger than the one generated by
Small. The C++ version of the MiniMax program uses a single recursive function
that updates a single state as it is recursively called to generate the best
score and corresponding move. The LM version is seven to eight times slower due
to the memory requirements and costs of creating new nodes using the exists
construct. In Chapter~\ref{chapter:coordination}, we will show how to improve
the space complexity of the MiniMax program and the corresponding run time.

The LM's N-Queens programs shows some scalability issues since the overhead
ratio increases as the size of the problem increases. However, the same behavior
is seen in the Python program. The C++ version uses a backtracking strategy to
try out all the possibilities and uses a vector to store board states.  Since
there is only at most $N$ (size of the board) vectors at the same time, it shows
better behavior than all the other programs. However, we should note that a 3 to
6-fold slowdown is a fairly good trade-off for a higher-level program that will
execute faster when using more threads as we are going to observe next.

From these results, it is possible to conclude that LM's virtual machine offers
a decent performance when compared to hand-written C++ programs. We think these
results originate from four main aspects of our system: efficient indexing, good
locality with array data structures for persistent facts, an efficient memory
allocator, and a good compilation scheme. As noted in~\cite{cost}, scalability
should not be the sole focus of a parallel/distributed system such as LM. This
is even more important in declarative languages which are known to suffer from
performance issues when compared to programming languages such as C++.

\subsection{Memory usage}

To complete our comparison against the C++ programs, we have measured and
compared the average memory used by both systems.
Table~\ref{table:implementation:mem} presents the memory statistics for LM
programs while Table~\ref{table:implementation:cmem} presents statistics for the
C++ programs.

In the Belief Propagation program, LM has a much higher resource usage than the
corresponding C++ version. This is because the nodes in the LM's program keep a
copy of the belief values of neighbor nodes, while the C++ program uses shared
memory and the stack to read and compute the neighbors belief values.
Interestingly, this issue does not happen in the Heat Transfer program, where
heat values are integers and are shared between nodes, reducing the need to keep
facts with unique lists around. Notably, the C++ memory usage for Heat Transfer
is not much smaller than LM's.

\begin{table}[ht]
   \begin{center}
      \begin{tabular}{c | c || c | c | c || c c} \hline
	\textbf{Program} & \textbf{Size} & \textbf{Average} & \textbf{Final} & \textbf{Malloc} & \textbf{\# Facts} & \textbf{Each} \\ \hline \hline
	\multirow{4}{*}{MSSD}  & US 500 Airports & 35MB & 13MB & 151 & 255K & 0.05KB \\
		 & OCLinks & 137MB & 58MB & 229 & 875K & 0.07KB \\
		 & US Power Grid & 218MB & 273MB & 150 & 4M & 0.06KB \\
		 & EU Email & 401MB & 334MB & 520 & 3M & 0.09KB \\
	\hline
	MiniMax  & - & 1778MB & 31KB & 129 & 2 & 15.50KB \\
	\hline
	\multirow{4}{*}{Belief Propagation}  & 50x50 & 131MB & 264MB & 76 & 34K & 7.87KB \\
		 & 200x200 & 2GB & 4GB & 101 & 557K & 8.66KB \\
		 & 300x300 & 6GB & 12GB & 108 & 1256K & 10.43KB \\
		 & 400x400 & 7GB & 15GB & 111 & 2M & 7.47KB \\
	\hline
	\multirow{2}{*}{Heat Transfer}  & 80x80 & 8MB & 8MB & 89 & 63K & 0.14KB \\
		 & 120x120 & 18MB & 19MB & 98 & 143K & 0.14KB \\
	\hline
	\multirow{4}{*}{NQueens}  & 11 & 1280KB & 671KB & 38 & 3K & 0.20KB \\
		 & 12 & 4MB & 2MB & 40 & 14K & 0.20KB \\
		 & 13 & 20MB & 14MB & 43 & 74K & 0.20KB \\
		 & 14 & 95MB & 73MB & 46 & 366K & 0.20KB \\
	\hline
\end{tabular}

   \end{center}

   \mycap{Memory statistics for LM programs. The meaning of each column is as
      follows: column \textbf{Average} represents the average memory use of the
      program; \textbf{Final} represents the memory usage after the program
      completes; \textbf{Malloc} represents the number of \code{malloc}
      operations requested to the operating system by the VM's memory allocator;
      \textbf{\# Facts} represents the number of facts in the database after the
      program completes; \textbf{Each} is the result of dividing \textbf{Final}
      by \textbf{\#~Facts} and represents the average memory required per fact.}

   \label{table:implementation:mem}
\end{table}

\begin{table}[ht]
   \begin{center}
      \begin{tabular}{c | c | c } \hline
	\textbf{Program} & \textbf{Size} & \textbf{Average} \\ \hline \hline
	\multirow{4}{*}{MSSD}  & US 500 Airports & 7MB\\
		 & OCLinks & 26MB\\
		 & US Power Grid & 150MB\\
		 & EU Email & 170MB\\
	\hline
	MiniMax  & - & 1KB\\
	\hline
	\multirow{4}{*}{Belief Propagation}  & 50x50 & 2MB\\
		 & 200x200 & 45MB\\
		 & 300x300 & 99MB\\
		 & 400x400 & 181MB\\
	\hline
	\multirow{2}{*}{Heat Transfer}  & 80x80 & 2MB\\
		 & 120x120 & 5MB\\
	\hline
	\multirow{4}{*}{NQueens}  & 11 & 1KB\\
		 & 12 & 1KB\\
		 & 13 & 1KB\\
		 & 14 & 1KB\\
	\hline
\end{tabular}

   \end{center}
   \mycap{Average and final memory usage of each C++ program.}
   \label{table:implementation:cmem}
\end{table}

When the MiniMax program completes, there are only two facts on the database
that indicate the best player move. The MiniMax program is also the only program
in this experiment that dynamically generates a (tree) graph, which is destroyed
once the best move is found. The VM's garbage collector that collects empty
nodes is able to delete the tree nodes (except the root) and the \textbf{Final}
memory usage reflects that since it is much smaller than the \textbf{Average}
statistic. However, because the garbage collector retains a small number of
freed nodes that may be reused later, the average size per fact is 15KB, which
also includes those freed nodes. Note that the memory usage of the C++ program
is much smaller because it uses function calls to represent the tree structure
of the MiniMax algorithm.

The MSSD program shows that LM's VM requires at most one order of magnitude more
memory than the corresponding C++ program. The ratio is larger when the graph
and computed distances are smaller and this is due to the extra data structures
required by the VM (i.e., the node data structure). In terms of average memory
per fact, we see that the MSSD requires on average 100B, where a big part of it
are the hash table data structures used for indexing. For the
Orkut dataset, where 100 million facts are derived, the average memory per fact
is exactly 30B, which is about the 32B required to store a particular linear
fact for representing one shortest distance.

Finally, for the N-Queens program, the results show why there are scalability
issues when using a larger $N$ since the memory usage increases significantly.
On the positive side, the average memory usage per fact remains the same for all
data sets. In respect to the C++ program, it is expected that it should consume
almost no memory because it uses the stack to compute the solutions.

\subsection{Dynamic Indexing}

We now evaluate the impact of our dynamic indexing algorithm and related data
structures. For this, we have rerun the previous experiments without indexing
and compared the results against the version with indexing enabled.
Table~\ref{table:implementation:compare_absolute} shows the comparison between
the versions with and without indexing.
Column \textbf{Run Time} shows that the MSSD program benefits from
indexing because the version without indexing is around 2 to 100 times slower than
the version with indexing enabled. Since MSSD computes the shortest distance to
multiple nodes, its rules require searching for the shortest distance facts of
arbitrary nodes. All the other programs do not require significant indexing but
also do not display any significant slowdown from using dynamic indexing. In
terms of memory usage, the version with indexing uses slightly more memory,
especially for the MSSD program, requiring, on average, 50\% more memory due to
the existence of hash tables used for supporting indexing.

\begin{table}[ht]
   \begin{center}
      \begin{tabular}{c | c || c | c} \hline
	\textbf{Program} & \textbf{Size} & \textbf{Run Time} & \textbf{Average Memory}\\ \hline \hline
	\multirow{4}{*}{MSSD}  & US 500 Airports &  25.49  &  1.01
  \\
		 & OCLinks &  24.25  &  1.02
  \\
		 & US Power Grid &  79.88  &  1.44
  \\
		 & EU Email &  4.11  &  1.12
  \\
	\hline
	MiniMax  & - &  1.01  &  1.00
  \\
	\hline
	\multirow{4}{*}{Belief Propagation}  & 50x50 &  1.04  &  1.00
  \\
		 & 200x200 &  1.02  &  1.00
  \\
		 & 300x300 &  1.00  &  1.00
  \\
		 & 400x400 &  0.99  &  1.00
  \\
	\hline
	\multirow{2}{*}{Heat Transfer}  & 80x80 &  1.01  &  1.00
  \\
		 & 120x120 &  1.05  &  1.00
  \\
	\hline
	\multirow{4}{*}{NQueens}  & 11 &  0.99  &  1.00
  \\
		 & 12 &  0.99  &  1.00
  \\
		 & 13 &  1.00  &  1.00
  \\
		 & 14 &  1.00  &  1.00
  \\
	\hline
\end{tabular}

   \end{center}

   \mycap{Measuring the impact of dynamic indexing and related data
      structures. Column \textbf{Run Time} shows the slow down ratio of the
      unoptimized version (numbers greater than 1 show indexing improvements).
      Column \textbf{Average Memory} is the result of dividing the average
   memory of the optimized version by the unoptimized version (large numbers
indicate that more memory is needed when using indexing mechanisms).}

   \label{table:implementation:compare_absolute}
\end{table}

\iffalse
\subsubsection{Array data structures}

Another important implementation detail is the use of array data structures for
storing persistent facts that are not derived by rule derivation but only exist
as initial facts. These cases are detected by the compiler and allow the VM to
improve memory locality and reduce memory usage by packing persistent facts in a
contiguous memory area.

\begin{table}[ht]
   \begin{center}
      \begin{tabular}{c | c || c | c} \hline
	\textbf{Program} & \textbf{Size} & \textbf{Run Time} & \textbf{Average Memory}\\ \hline \hline
	\multirow{4}{*}{Belief Propagation}  & 50x50 &  0.94  &  0.03
  \\
		 & 200x200 &  1.03  &  0.02
  \\
		 & 300x300 &  1.02  &  0.02
  \\
		 & 400x400 &  1.05  &  0.03
  \\
	\hline
	\multirow{2}{*}{Heat Transfer}  & 80x80 &  1.17  &  0.65
  \\
		 & 120x120 &  1.19  &  0.65
  \\
	\hline
	\multirow{4}{*}{MSSD}  & US 500 Airports &  1.64  &  0.76
  \\
		 & OCLinks &  0.31  &  3.55
  \\
		 & EU Email &  0.68  &  1.02
  \\
		 & US Power Grid &  0.08  &  9.60
  \\
	\hline
	MiniMax  & Small &  0.98  &  0.94
  \\
	\hline
	\multirow{4}{*}{N-Queens}  & 11 &  0.85  &  2.10
  \\
		 & 12 &  0.71  &  2.73
  \\
		 & 13 &  0.58  &  3.40
  \\
		 & 14 &  0.55  &  4.24
  \\
	\hline
\end{tabular}

   \end{center}

   \mycap{Measuring the impact of using array data structures for persistent
      facts. Column \textbf{Run Time} shows the speedups obtained by using
      arrays. Column \textbf{Average Memory} is the result of dividing the
      average memory of the programs using arrays by the version without them
      (e.g., numbers show memory usage reduction when using array data
   structures).}

   \label{table:implementation:compare_arrays}
\end{table}

I think we should remove this stuff XXX
\fi



\subsection{Scalability}
\begin{figure}[h]
        \centering
        \begin{subfigure}[b]{0.5\textwidth}
                \includegraphics[width=\textwidth]{experiments/scalability/scale-8queens-14.png}
                \caption{Initial database. Replace axiom instantiated at the
                   \code{@3} root node.}
                   \label{fig:implementation:scale_queens}
        \end{subfigure}%
        ~
        \begin{subfigure}[b]{0.5\textwidth}
                \includegraphics[width=\textwidth]{experiments/scalability/scale-min-max-tictactoe.png}

                \caption{After applying rule 3 at node \code{@3}. The
                \code{replace} fact is \emph{sent} (or derived at) to node
                \code{@5}.}

                \label{fig:implementation:scale_minmax}
        \end{subfigure}\\
        \begin{subfigure}[b]{0.5\textwidth}
                \includegraphics[width=\textwidth]{experiments/scalability/scale-new-heat-transfer-120.png}
                \caption{Initial database. Replace axiom instantiated at the
                   \code{@3} root node.}
                   \label{fig:implementation:scale_heat}
        \end{subfigure}%
        ~
        \begin{subfigure}[b]{0.5\textwidth}
                \includegraphics[width=\textwidth]{experiments/scalability/scale-belief-propagation-400.png}

                \caption{After applying rule 3 at node \code{@3}. The
                \code{replace} fact is \emph{sent} (or derived at) to node
                \code{@5}.}

                \label{fig:implementation:scale_bp}
        \end{subfigure}\\
        \caption{An execution trace for the binary tree dictionary
           algorithm. The first argument of each fact was dropped and the
           address of the node was placed beside it.}
        \label{fig:implementation:scale1}
\end{figure}


\clearpage

\subsection{Thread allocator evaluation: comparing against \code{malloc}}
In order to understand the role which our threaded allocator plays in the
performance of programs, we evaluate and compare it against the performance of
the LM system by using the \code{malloc} function provided by the operating
system.

Figure~\ref{fig:implementation:malloc_results} presents several programs
comparing \code{malloc} with the threaded allocator. The programs Belief
Propagation, MiniMax, and N-Queens have very poor scalability when using the
\code{malloc} operator due to high thread contention when calling \code{malloc}
since those programs need to allocate many small objects. The PageRank programs
shows good scalability with \code{malloc} but it is only because the
configuration with 1 thread runs very slowly when compared to the other
configurations. For all the remaining programs, there is less overall
performance and scalability when compared to the threaded allocator.

As these results show, the performance of the memory allocator is crucial for
good performance and scalability. Linear logic programs spend a significant
amount of time asserting and retracting linear facts which require that memory
allocation must be done efficiently and without starving other threads. Reuse of
deallocated linear facts also makes more likely that threads will hit hot cache
lines and thus improve speed significantly.

\newcommand{\smallplotsize}[0]{0.3}

\begin{figure}[h]
        \begin{subfigure}[b]{\smallplotsize\textwidth}
                \includegraphics[width=\textwidth]{experiments/scalability/malloc-allocator-belief-propagation-400.png}
                \label{fig:implementation:malloc_bp}
        \end{subfigure}
        ~
        \begin{subfigure}[b]{\smallplotsize\textwidth}
                \includegraphics[width=\textwidth]{experiments/scalability/malloc-allocator-greedy-graph-coloring-gplus.png}
                \label{fig:implementation:malloc_ggc}
        \end{subfigure}
        ~
        \begin{subfigure}[b]{\smallplotsize\textwidth}
                \includegraphics[width=\textwidth]{experiments/scalability/malloc-allocator-new-heat-transfer-120.png}
                \label{fig:implementation:malloc_ht}
        \end{subfigure}
        ~
        \begin{subfigure}[b]{\smallplotsize\textwidth}
                \includegraphics[width=\textwidth]{experiments/scalability/malloc-allocator-min-max-tictactoe-big.png}
                \label{fig:implementation:malloc_minimax}
        \end{subfigure}
        ~
        \begin{subfigure}[b]{\smallplotsize\textwidth}
                \includegraphics[width=\textwidth]{experiments/scalability/malloc-allocator-shortest-uspowergrid.png}
                \label{fig:implementation:malloc_sssp}
        \end{subfigure}
        ~
        \begin{subfigure}[b]{\smallplotsize\textwidth}
                \includegraphics[width=\textwidth]{experiments/scalability/malloc-allocator-shortest-twitter.png}
                \label{fig:implementation:malloc_sssp}
        \end{subfigure}\\
        \begin{subfigure}[b]{\smallplotsize\textwidth}
                \includegraphics[width=\textwidth]{experiments/scalability/malloc-allocator-pagerank-gplus.png}
                \label{fig:implementation:malloc_pagerank}
        \end{subfigure}
        ~
        \begin{subfigure}[b]{\smallplotsize\textwidth}
                \includegraphics[width=\textwidth]{experiments/scalability/malloc-allocator-8queens-14.png}
                \label{fig:implementation:malloc_queens}
        \end{subfigure} \\

        \mycap{Comparing the threaded allocator described in
           Section~\ref{section:implementation:allocation} against the
           \texttt{malloc} allocator provided by the C library. The threaded
           allocator is represented by plus markers, while \texttt{malloc} is
        represented by circular markers. Note that the two speedup lines (right
     axis) use dashed lines, while the two lines representing the run time (left
  axis) are contiguous.}

        \label{fig:implementation:malloc_results}
\end{figure}


\subsection{Thread allocator evaluation: alternative allocator}\label{section:implementation:alternative_allocator}
While our default threaded allocator performs and scales well, we also explored
some alternative designs. In this section, we explore a design where we move the
allocation decisions from the thread to the node itself, in order to store the
facts of the same node close together and thus increase locality. However, this
design requires locking since multiple threads may allocate facts from the same
node. Our expectation is that such costs will be offset by the increased
locality and reduced cache misses when deriving rules.

The fact allocator allocates pages of memory from the threaded allocator which
are then used to allocate facts for that particular node. When a thread needs to
allocate or deallocate a fact, it acquires the allocator lock of the target
node's allocator and performs the allocation operation. There is a doubly-linked
list of memory pages and each page contains facts of different sizes
(predicates).

\begin{figure}[ht]
   \begin{center}
      \includegraphics[width=0.7\linewidth]{figures/implementation/fact_allocator.pdf}
   \end{center}

   \mycap{Fact allocator: each node has a pool of memory pages for allocating
      logical facts. Each page contains: (i) several linked lists of free facts
      of the same size (\code{free\_array}); (ii) a reference count of used
      facts (\code{refcount}); (iii) a \code{ptr} pointer that points to
      unallocated space in the page. In this figure, predicates \code{f} and
      \code{g} have several deallocated facts that are ready to be used when a
   new fact needs to be acquired.}

   \label{fig:implementation:fact_allocator}
\end{figure}

Figure~\ref{fig:implementation:fact_allocator} presents an example state of a
fact allocator. The node has 3 memory pages, all connected using the \code{next}
and \code{prev} pointers. Each page also has a reference count (\code{refcount})
of the objects allocated in the page. If the reference count ever drops to zero,
then the memory page is deallocated. Deallocated facts are kept on an ordered
array for different sizes using the mechanism we implemented for the threaded
allocator. We have decided to reference count objects in the fact allocator
because it is more difficult to share objects between nodes and maintaining a
reference count helps reduce memory usage.

Figure~\ref{fig:implementation:node_results} presents the comparison between the
threaded and fact allocator for the most relevant datasets (the performance was
similar for the other datasets).

\begin{figure}[h]
        \begin{subfigure}[b]{\smallplotsize\textwidth}
                \includegraphics[width=\textwidth]{experiments/scalability/node-allocator-belief-propagation-400.png}
                \label{fig:implementation:node_bp}
        \end{subfigure}
        ~
        \begin{subfigure}[b]{\smallplotsize\textwidth}
                \includegraphics[width=\textwidth]{experiments/scalability/node-allocator-greedy-graph-coloring-twitter.png}
                \label{fig:implementation:node_ggc}
        \end{subfigure}
        ~
        \begin{subfigure}[b]{\smallplotsize\textwidth}
                \includegraphics[width=\textwidth]{experiments/scalability/node-allocator-new-heat-transfer-120.png}
                \label{fig:implementation:node_ht}
        \end{subfigure}
        ~
        \begin{subfigure}[b]{\smallplotsize\textwidth}
                \includegraphics[width=\textwidth]{experiments/scalability/node-allocator-min-max-tictactoe-big.png}
                \label{fig:implementation:node_minimax}
        \end{subfigure}
        ~
        \begin{subfigure}[b]{\smallplotsize\textwidth}
                \includegraphics[width=\textwidth]{experiments/scalability/node-allocator-shortest-uspowergrid.png}
                \label{fig:implementation:node_sssp}
        \end{subfigure}
        ~
        \begin{subfigure}[b]{\smallplotsize\textwidth}
                \includegraphics[width=\textwidth]{experiments/scalability/node-allocator-shortest-twitter.png}
                \label{fig:implementation:node_sssp2}
        \end{subfigure}\\
        \begin{subfigure}[b]{\smallplotsize\textwidth}
                \includegraphics[width=\textwidth]{experiments/scalability/node-allocator-pagerank-pokec.png}
                \label{fig:implementation:node_pagerank}
        \end{subfigure}
        ~
        \begin{subfigure}[b]{\smallplotsize\textwidth}
                \includegraphics[width=\textwidth]{experiments/scalability/node-allocator-8queens-14.png}
                \label{fig:implementation:node_queens}
        \end{subfigure} \\
        \mycap{Comparing the threaded allocator described in
        Section~\ref{section:implementation:allocation} against the fact
     allocator.}

        \label{fig:implementation:node_results}
\end{figure}

The first major observation from the comparison is that the threaded allocator
has better sequential performance than the fact allocator for most programs.
This is probably the result of better locality for the threaded allocator
because it does not need to create pages for each node and instead uses the
thread pages to maintain all facts, reducing cache line misses. Overall, the
performance of the threaded allocator is better or similar than the fact
allocator, for both single and multithreaded threaded execution (see GGC for a
clear example).

The second observation is in the N-Queens program, where the threaded allocator
beats the fact allocator by a high margin in terms of scalability and
performance. Note that the N-Queens program allocates many of lists and those
lists are not allocated on a per node basis but on a thread basis, therefore the
extra work required to maintain each fact allocator is not offset by the
increased locality because the threads need to traverse lists to find valid
board states.

To make our comparison more interesting, we also deactivated the reference
counting mechanisms of the fact allocator and compared it against the threaded
allocator. The goal is to understand how reference counting affects the
performance of the alternative allocator. The complete comparison are shown in
Fig.~\ref{fig:implementation:no_refs_results}. In a nutshell, it performs almost
the same as the full featured fact allocator, except for programs which require
more memory such as MiniMax and Heat Transfer. In the case of MiniMax, we were
unable to completely execute the Big dataset since the VM required more memory
than the available memory, resulting in long execution times. This is the result
of not collecting unused memory pages, which results in high memory usage.


\begin{figure}[h]
        \begin{subfigure}[b]{\smallplotsize\textwidth}
                \includegraphics[width=\textwidth]{experiments/scalability/no-refs-allocator-belief-propagation-400.png}
                \label{fig:implementation:no_refs_bp}
        \end{subfigure}
        ~
        \begin{subfigure}[b]{\smallplotsize\textwidth}
                \includegraphics[width=\textwidth]{experiments/scalability/no-refs-allocator-greedy-graph-coloring-twitter.png}
                \label{fig:implementation:no_refs_ggc}
        \end{subfigure}
        ~
        \begin{subfigure}[b]{\smallplotsize\textwidth}
                \includegraphics[width=\textwidth]{experiments/scalability/no-refs-allocator-new-heat-transfer-120.png}
                \label{fig:implementation:no_refs_ht}
        \end{subfigure}
        ~
        \begin{subfigure}[b]{\smallplotsize\textwidth}
                \includegraphics[width=\textwidth]{experiments/scalability/no-refs-allocator-min-max-tictactoe-small.png}
                \label{fig:implementation:no_refs_minimax}
        \end{subfigure}
        ~
        \begin{subfigure}[b]{\smallplotsize\textwidth}
                \includegraphics[width=\textwidth]{experiments/scalability/no-refs-allocator-shortest-uspowergrid.png}
                \label{fig:implementation:no_refs_sssp}
        \end{subfigure}
        ~
        \begin{subfigure}[b]{\smallplotsize\textwidth}
                \includegraphics[width=\textwidth]{experiments/scalability/no-refs-allocator-shortest-twitter.png}
                \label{fig:implementation:no_refs_sssp}
        \end{subfigure}\\
        \begin{subfigure}[b]{\smallplotsize\textwidth}
                \includegraphics[width=\textwidth]{experiments/scalability/no-refs-allocator-pagerank-pokec.png}
                \label{fig:implementation:no_refs_pagerank}
        \end{subfigure}
        ~
        \begin{subfigure}[b]{\smallplotsize\textwidth}
                \includegraphics[width=\textwidth]{experiments/scalability/no-refs-allocator-8queens-14.png}
                \label{fig:implementation:no_refs_queens}
        \end{subfigure} \\

        \mycap{Comparing the threaded allocator described in
           Section~\ref{section:implementation:allocation} against the fact
           allocator without reference counting. The threaded allocator is
           represented by plus markers, while the fact allocator is represented
           by circular markers. Note that the two speedup lines (right axis) use
        dashed lines, while the two lines representing the run time (left axis)
     are contiguous.}

        \label{fig:implementation:no_refs_results}
\end{figure}


Overall, the performance of both allocators is similar, therefore we have
decided to use the threaded allocator by default since it is simpler and offers
good performance across the board. Furthermore, the threaded allocator requires
less allocated memory because it enables more sharing between nodes.




\section{Chapter Summary}

This chapter provided a full description of the implementation of LM, including
its compiler and virtual machine. We explained how the virtual machine is
organized to provide scalable multi threaded execution and fast fact assertion
and retraction using efficient data structures. We also gave a detailed
description of the compilation algorithm used to transform LM rules into
efficient C++ code.
