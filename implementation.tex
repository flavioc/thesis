This chapter describes the multicore implementation of the LM language,
including its compiler, supporting runtime and parallelism. We first start with
an overview of the implementation in order to understand how all pieces of the
implementation fit together. Secondly, we describe how parallelism is achieved,
including its data structures and thread scheduling. Thirdly, we present the
implementation details of coordination and how it relates with parallelism.  We
then describe the runtime and data structure used to implement nodes and the
database of facts, followed by the compilation algorithm used by our compiler to
turn logical rules into efficient C++ code.

\section{Overview}
The implementation of LM is composed of a compiler and a virtual machine~(VM).
Figure~\ref{fig:implementation:overview} presents an overview of the compilation
process for LM programs. The two main boxes represent the two major components
of the system, namely, the compiler and virtual machine.

\begin{figure}[ht]
  \centering
  \includegraphics[width=.75\linewidth]{figures/implementation/overview.pdf}
  \caption{Compilation of a LM program into an executable. The compiler
     transforms a LM program into \code{file.cpp}, C++ file with compiled
     rules, and a \code{data} file with the graph structure and axioms. The virtual
     machine which includes code for managing multithreaded execution and the
     database of facts is then linked with \code{file.cpp} to create a
     standalone executable that can be run in parallel.}
  \label{fig:implementation:overview}
\end{figure}

The virtual machine contains supporting data structures for managing the
database of facts and to schedule the execution of rules. The parallel engine is
also a major part of the virtual machine and is responsible for managing
multithreaded execution by launching threads, managing communication and
and scheduling parallel execution, including coordination.

The compiler transforms LM files into C++ code that use the virtual machine
facilities to implement the program logic.  The compiled code implements the
inference rules of the program and uses the API of the virtual machine to derive
and retract facts and to schedule the execution of rules.  The compiler also
creates a separate file, named \code{file.data}, with the program's axioms and
graph structure. The graph structure is reconstructed in parallel once the
program starts.

To complete the compilation process, we use a C++ compiler to compile the
virtual machine files and \code{file.cpp} into object files that are then
linked along with \code{file.data}. At the end, we have a standalone
executable that allows the user to input the number of threads to use,
scheduling strategies (i.e., disable coordination), run time measurement
facilities and also database printing facilities.

Alternatively, the programmer may also decide to compile a more general version
of the virtual machine that is able to run byte-code files generated by the
compiler. This allows faster development since the programmer only needs to
recompile the LM program and not the whole stack. However, LM programs will run
slower since the byte-code must be interpreted by the virtual machine. This
severely affects programs with many mathematical operations, especially floating
point computations.

\iffalse
\subsection{Graph Clustering}

The graph structure stored in file \code{file.data} is constructed by the
compiler by analyzing the program's axioms and then ordering the nodes of the
graph.  In order to distribute computation across threads it is important to
increase locality of communication, so that a node makes most of its
communication to neighbor nodes that are being handled by the same thread. The
graph structure in \code{file.data} is written in order to improve locality
and reduce communication between threads.

The compiler analyzes the node address constants (that are prepended by the
symbol @) and the axioms of the program. After parsing and type-checking the
program code, the compiler then optimizes the topology by building an internal
representation of the graph.  In this phase, each node address $a$ is mapped
using a function $M(x)$ to a normalized node address $n$. Function $M(x)$ is
bijective and the domain is the set of all nodes described in the source code.
The co-domain of $M$ is the discrete interval $[0, N[$, where $N$ is the number
of nodes in the graph. The byte-code of a LM program includes all the pairs $(x,
M(x))$ so that the runtime system can put this information to use.

We have three methods for defining the function $M(x)$:

\begin{itemize}
   \item \emph{Static}: the compiler uses the node addresses presented in the
      code as long as they fill the discrete interval $[0, N[$.
   \item \emph{Randomized}: the mapping is done randomly.
   \item \emph{Breadth-First}: the mapping is built by picking an arbitrary node, $n_{zero}$
   and setting $M(n_{zero}) = 0$, then we select all neighbors of $n_{zero}$ and start defining
   their mappings in increasing order, $1, \dotsc, N-1$, and adding its neighbors for later processing
   in a breadth first fashion.
\end{itemize}

The breadth-first method is used with the intent of clustering closer nodes in
an ordered fashion.  While not optimal, using a breadth-first approach is very
efficient and has good results for irregular graphs. If we use a static division
of work between $N$ threads, where each thread is responsible to process a
pre-defined set of nodes, we can efficiently slice the codomain of function
$M(x)$ and divide it between the $N$ threads.

For an example, consider the graph in Fig.~\ref{fig:compiler:topology1}. The
node addresses represented are the ones included in the source code. Using a
breadth-first method starting by node 1, we get the following order: 1, 2, 3, 7,
5, 6 and 4. If we had to do a static division with 2 threads, thread 1 would get
1, 2, 3, 7 and thread 2 would get 5, 6 and 4. Note from
Fig.~\ref{fig:compiler:topology1} that only 3 edges exist between the nodes of
thread 1 and thread 2. This greatly reduces communication between threads and
improves parallel efficiency.

\begin{figure}[ht]
  \centering
  \includegraphics[width=0.6\textwidth]{figures/compiler/topology1.pdf}
  \caption{Topology using a breadth-first method.}
  \label{fig:compiler:topology1}
\end{figure}
\fi


\section{Parallelism}
A key goal of our parallel design is to keep the threads as busy as possible and
to reduce inter-thread communication. Initially, the VM partitions the
application graph of $N$ nodes into $T$ subgraphs (the number of threads) and
then each thread works on their own subgraph. During execution, threads can
steal nodes of other threads to keep themselves busy. The load balancing aspect
of the system is performed by our work scheduler that is based on a simple work
stealing algorithm.

While our VM uses shared memory for thread communication, each
thread has a logical space that contains its subgraph, i.e., the nodes owned by
the thread, and a \emph{Work Queue}, which contains \textbf{active} nodes, i.e.,
nodes that have new facts to process.  The work queue is implemented as a linked
list. Initially, the work queue is filled with the nodes in the thread's
subgraph in order to derive the initial facts. A thread may go through different
states during its lifetime. The state is kept track in the \emph{State} flag
which may have one of the following values:

\begin{itemize}

   \item \textbf{active}: The thread has a non-empty \emph{Work Queue} or is
      currently executing rules for a node (which is not in the \emph{Work
      Queue}).

   \item \textbf{stealing}: The thread has no active nodes in the \emph{Work
      Queue} and is attempting to steal nodes from other
      threads.

   \item \textbf{idle}: The thread is trying to synchronize with other threads
      to terminate the program.
   
   \item \textbf{terminated}: The thread (and all the other threads) have
      terminated and the program is finished.

\end{itemize}

Figure~\ref{fig:implementation:thread_states} presents the valid transitions for
the thread state flag. The dashed line from \textbf{idle} to \textbf{stealing}
indicates that the transition is made in a non-deterministic fashion.

\begin{figure}[ht]
   \centering
   \includegraphics[width=0.65\textwidth]{figures/implementation/thread_states.pdf}
   \mycap{The thread state machine as represented by the \emph{State} flag. During
      the lifetime of a program, each thread goes through different states as
      specified by the state machine.}
   \label{fig:implementation:thread_states}
\end{figure}

The pseudo-code for the main thread loop is shown in
Fig.~\ref{alg:thread_work_loop}. In each round, a thread inspects its \emph{Work
Queue} and while there are active nodes, procedure \code{process\_node()}
(complete pseudo-code in Fig.~\ref{alg:multicore:process_node}) will perform
local computation on active nodes. When a thread's \emph{Work Queue} is empty,
it attempts to steal half of the nodes from another thread. Starting from a
random thread, it cycles through all the threads to find one active thread from
whom it will try to steal half of its nodes. If the thread fails to steal work,
it will go \textbf{idle} and periodically attempt to steal work from another
active thread. Eventually, all threads will fail to steal work since there is no
more work to do and they will go idle.  There is a global atomic counter that is
used to detect termination. Once a thread goes idle, it decrements the global
counter and changes its flag to idle.  Since every thread will be busy-waiting
and checking the global counter, they will detect a zero value and stop
executing, transitioning to the \textbf{terminated} state.

\begin{figure}
\begin{algorithm}[H]
   \KwData{Thread TH}
   \While{true}{
      $TH.work\_queue.lock()$\;
      $node \longleftarrow TH.work\_queue.pop\_node()$ \;
      $TH.work\_queue.unlock()$\;
      \uIf{$node$}{
         $process\_node(node)$\;
      }
      \Else{
         \tcp{Attempt to steal some nodes.}
         \If{$\bang TH.steal\_nodes()$}{
            $TH.become\_idle()$\;
            \While{$len(TH.work\_queue) == 0$}{
               \tcp{Try to terminate}
               \If{$TH.synchronize\_termination()$}{
                  \textbf{terminate}\;
               }
               \If{$TH.steal\_nodes()$}{
                  \tcp{Thread is still in the stealing state}
                  break\;
               }
            }
            \tcp{There's new nodes in the queue.}
            $TH.become\_active()$\;
         }
      }
 }
\end{algorithm}
\mycap{Thread work loop: threads process active nodes from the work queue
   until no more active nodes are available. Node stealing using a \emph{steal
      half} strategy is employed when the thread has no more active nodes.}
 \label{alg:thread_work_loop}
\end{figure}

Figure~\ref{fig:implementation:vm_overview} ties everything together and
presents the layout of our virtual machine for a program with six nodes and two
running threads. In the figure, we represent the thread's \emph{Work Queue} and
the thread logical space where the thread's subgraph is located. We also show
the internals of node \code{@1} and the thread operations that force threads to
interact with the node data structures.

\begin{figure*}[t]
\centering
\includegraphics[width=\textwidth]{figures/implementation/vm_overview.pdf}
\mycap{Layout of the virtual machine. Each thread has a work queue that
   contains active nodes (nodes with facts to process) that are processed one
   by one by the thread. Communication between threads happens when nodes
   send facts to nodes located in other threads.}
\label{fig:implementation:vm_overview}
\end{figure*}

In order to understand how threads interact between each other, we now review
the node data structure that was presented in Section~\ref{sec:data_structures}.
The node lock \emph{DB Lock} protects the data structures of the database,
including the array, trie, linked list and hash table data structures and the
\emph{Rule Engine}. The \emph{State Lock} structure protects everything else,
especially the \emph{State} flag and the temporary set of facts represented by
the \emph{Fact Buffer}. The \emph{Incoming Fact Buffer} is used to hold logical facts
that can not be added immediately to the database data structures. The
\emph{Owner} field points to the thread responsible for processing the node and
the \emph{Rule Engine} schedules local computation.

Whenever a new fact is derived through rule derivation, we need to update the
data structures for the corresponding node. If the node is currently being
executed by the thread (local send), then the fact is added to the node data
structures since the \emph{DB Lock} is being held while the node is being
processed. If that is not the case, then we have to synchronize since multiple
threads might be updating the same node's data structures. For example, in
Fig.~\ref{fig:implementation:vm_overview}, when thread 2 derives a fact to node
\code{@1} (owned by thread 1), it first locks node \code{@1} using the
\emph{State Lock} and then it attempts to lock \emph{DB Lock}, which gives thread
2 full access to the node. In this case, thread 2 adds the new fact to the
database (\emph{New fact (2)} in Fig.~\ref{fig:implementation:vm_overview}) and
to the \emph{Rule Engine}. However, if the \emph{DB Lock} could not be acquired
because the node \code{@1} is currently being processed, then the new fact is
added to \emph{Incoming Fact Buffer} (\emph{New fact (1)} in
Fig.~\ref{fig:implementation:vm_overview}). The facts stored in \emph{Fact
Buffer} will then be processed whenever the corresponding node is processed
again.

When a thread interacts with another thread to send a fact, it also needs to
make sure that the target node is made \textbf{active} (see
Fig.~\ref{fig:local:node_states}) and that it is also placed in the target
thread's \emph{Work Queue} (\emph{Activate node} in
Fig.~\ref{fig:implementation:vm_overview}). To handle concurrency issues, we
have a per \emph{Work Queue} lock called the \emph{Queue Lock} that is held when
the \emph{Work Queue} is being operated on.  As an example, consider again the
situation in which thread 2 sends a new fact to node \code{@1}. If node
\code{@1} is not active, then thread 2 also needs to activate node \code{@1} by
pushing it to the \emph{Work Queue} of thread 1.  After this synchronization
point, the target thread is ensured to be active and with a new node to process.

\begin{figure}
\begin{algorithm}[H]
   \KwData{Node N}
   $N.state\_lock.lock()$\;
   $N.db\_lock.lock()$\;
   \tcp{Add facts from the Incoming Fact Buffer into the database}
   $N.DB.merge(N.incoming\_fact\_buffer)$\;
   $N.state\_lock.unlock()$\;

   $N.rule\_engine.run\_rules()$\;
   $N.db\_lock.unlock()$\;

   \tcp{Check if node N is done for now}

   $N.state\_lock.lock()$\;
   \If{$N.rule\_engine.has\_candidate\_rules()$ or \\
      \hspace{2cm} $N.incoming\_fact\_buffer.has\_facts()$}{
      $N.state\_lock.unlock()$\;
      goto beginning\;
   }
   $N.state \longleftarrow inactive$\;
   $N.state\_lock.unlock()$\;
\end{algorithm}
\mycap{Pseudo-code for the \code{process\_node} procedure.}
 \label{alg:multicore:process_node}
\end{figure}


\section{Coordination}
In order to support priorities, the work queue is implemented as two pairs of
queues: a pair of doubly linked lists known as the \emph{standard queue} and a
pair of \emph{min/max} heaps known as the \emph{priority queue}.  The standard
queue contains nodes without priorities and supports push into tail, remove node
from the head, remove arbitrary node, and remove first half of nodes.  The
priority queue contains nodes with priorities and is implemented as a binary
heap array. It supports the following operations: push into the heap, remove the
\emph{min} node, remove an arbitrary node, remove half of the nodes (horizontal
split), and priority update.  Operations for removing half of the queue are
implemented in order to support node stealing, while operations to remove
arbitrary nodes or update priority allows threads to change the priority of
nodes. Table~\ref{fig:implementation:table_queue} show the complexity of queue
operations and compares the standard queue against the priority queue.

\begin{table}[h]
   \begin{tabular}{| c | l | l |}
      \hline
      \textbf{Operation} & \textbf{Standard queue} & \textbf{Priority Queue} \\
      \hline
      Push & $\mathcal{O}(1)$ & $\mathcal{O}(\log{N})$ \\ \hline
      Pop & $\mathcal{O}(1)$ & $\mathcal{O}(\log{N})$ \\ \hline
      Remove & $\mathcal{O}(1)$ & $\mathcal{O}(\log{N})$ \\ \hline
      Remove Half & $\mathcal{O}(N)$ & $\mathcal{O}(\log{N})$ \\ \hline
      Priority Update & - & $\mathcal{O}(\log{N})$ \\ \hline
   \end{tabular}
   \caption{Comparing the complexity of queue operations for both standard
      queue and priority queue. Except for the remove half operation, priority
      queue operations are more expensive.}
   \label{fig:implementation:table_queue}
\end{table}

\begin{figure*}[t]
\centering
\includegraphics[width=0.95\textwidth]{figures/implementation/work_queue.pdf}
\caption{Thread's work queue and its interaction with other threads: the dynamic queue contains nodes that can be
   stolen while the static queue contains nodes that cannot be stolen. Both
   queues are implemented with one standard queue and one priority queue.}
\label{fig:implementation:work_queue}
\end{figure*}

The \texttt{next} and \texttt{prev} pointers of the regular queue are part of
the node structure in order to save space. These pointers are also used as the
index in the priority queue and current priority, respectively. Both the regular
and priority queue are implemented as a pair of queues.  This first queue is the
\emph{static queue} which contains nodes that cannot be stolen.  The other queue
is the \emph{dynamic queue} which contains nodes which can be stolen by other
threads.

To minimize inter-thread communication, node priorities are implemented at the
thread level. Thus, when a thread picks the highest priority node from the
priority queue, it is only the highest priority with respect to the set of nodes
owned by the thread and not the highest priority node in the whole program.  

\subsection{Communication}

At this point, we can summarize the main thread synchronization hotspots in the
virtual machine. Threads synchronize with each other using mutual exclusion. We
use a spin-lock in each queue to protect queue operations.  Given threads $T_1$
and $T_2$, we enumerate the most important synchronization hotspots:

\begin{itemize}

   \item \textbf{New facts}: When a node executes in $T_1$ and derives facts to
      a node in $T_2$, $T_1$ first buffers the facts and then sends them to the
      target node. Here, it checks if the node is currently \textbf{idle} and
      then synchronizes with $T_2$ to add the node to the $T_2$'s queue.

   \item \textbf{Thread activation}: If $T_2$ is inactive when adding facts to a
      node in $T_2$, $T_1$ also synchronizes with $T_2$ to change $T_2$'s state
      to \emph{active}.

   \item \textbf{Node stealing}: $T_1$ synchronizes with $T_2$ when it attempts
      to steal nodes from $T_2$ by removing half of the nodes from one of
      $T_2$'s queues.

   \item \textbf{Coordination}: If $T_1$ needs to perform coordination
      operations to a node in $T_2$, it may need to synchronize with $T_2$
      during priority updates in order to move the node in $T_2$'s queues.

\end{itemize}

\subsection{Node State Machine}

In order to facilitate the implementation of coordination, we added a state
variable for each node. The state machine in
Fig.~\ref{fig:implementation:node_states} represents the valid state transitions
of a node:

\begin{itemize}
   \item \textbf{working}: the node is executing.
   \item \textbf{inactive}: the node is inactive, i.e., it has no new facts and is not in any
   queue for processing.
   \item \textbf{queue}: the node is active with new facts and is in some queue waiting
   to be processed.
   \item \textbf{stealing}: the node has just been stolen and is in the process of being
   moved to another thread.
   \item \textbf{coordinating}: the node is moving from one queue to another.
\end{itemize}

\begin{figure}[ht]
   \centering
   \includegraphics[width=0.55\textwidth]{figures/implementation/node_states.pdf}
   \caption{The node state machine as represented by the state variable. During
      the lifetime of a program, each node goes through different states as
      specified by the state machine.}
   \label{fig:implementation:node_states}
\end{figure}

Each node is protected by a main spin-lock that allows threads to change node
attributes: the fact buffer, owner thread, state variable and locality
information. There is also a database spin-lock that protects the internal
database of the node and is locked whenever the node is in the \textbf{working}
state.  

To avoid unnecessary copying, when a node sends facts to a node located in
another thread, the current thread first attempts to lock the database lock of
the target node in order to directly update its database and indexing
structures, otherwise it adds the facts to the list of incoming facts that are
later processed by the owner thread of the target node.


\section{Node Data Structure}\label{sec:data_structures}
The main characteristic of LM rules is that they are constrained by the first
argument\footnote{In the implementation, the first argument of each fact is not
stored.}. Rule derivation uses only facts from the same node, therefore there is
no need to synchronize with other nodes to derive rules. However, when nodes
derive non-local facts (owned by other nodes), then the implementation must
synchronize and \emph{send} the facts to the target node. From the point of view
of the receiving node, these are called \emph{incoming facts}. Note that this is related
to the parallel aspects of the virtual machine and more details are given in the
next chapter.

As shown in Fig.~\ref{fig:local:node_overview}, each node contains the
following attributes:

\begin{itemize}
   \item \emph{State}: the node state flag which can be one of the following:
      \textbf{running}, the node is deriving rules; \textbf{inactive}, the node
      has no new facts to be considered and all candidate rules have been tried;
      \textbf{active}, the node has new facts to be considered but is waiting to
      be executed by a thread. Figure~\ref{fig:local:node_states} presents the
      state machine with the valid state transitions.
   \item \emph{Main Lock}: a lock that protects the node attributes enumerated
      in this list except for \emph{Linear DB} and \emph{Persistent DB}.
   \item \emph{DB Lock}: a lock that protects \emph{Linear DB},
      \emph{Persistent DB} and \emph{Rule Engine}. The lock is held when the node is deriving rules or
      when incoming facts from other nodes need to be added to the database data
      structures.
   \item \emph{Rule Engine}: a data structure that detects which rules are
      candidates and should be derived. The data structure is fully explained
      in Section~\ref{section:local:rule_engine}.
   \item \emph{Owner}: a pointer to the thread responsible for executing this
      node.
   \item \emph{Fact Buffer}: a data structure that holds incoming facts that
      could not be added to the database data structures since the node is
      currently deriving rules.
   \item \emph{Linear DB}: the database of linear facts as an array of data
      structures for storing linear facts for each linear predicate.
   \item \emph{Persistent DB}: the database of persistent facts.
\end{itemize}

\begin{figure}[ht]
   \centering
   \includegraphics[width=0.55\textwidth]{figures/local/node_state.pdf}
   \caption{The node state machine as represented by the state variable. During
      the lifetime of a program, each node goes through different states as
      specified by the state machine.}
   \label{fig:local:node_states}
\end{figure}

\begin{figure*}[t]
\centering
\includegraphics[width=0.5\textwidth]{figures/local/node.pdf}
\caption{Layout of the node data structure.}
\label{fig:local:node_overview}
\end{figure*}

\paragraph{Database data structures}

The database of facts must be implemented efficiently because during matching
of rules we need to restrict the facts using \emph{join constraints}, which fix
arguments of predicates to instantiated values. A database fact is made up of 2
pointers (\code{prev} and \code{next}) and a variable number of arguments.
Each argument is large enough to contain a value of any of the available LM types.
Facts are stored in one of the four data structures:

\begin{itemize}

\item \emph{Trie Data Structures} are used to store persistent facts. Tries are
   trees where facts are indexed by common prefix arguments. The \code{prev}
   and \code{next} pointers are used to store all the facts stored in the trie.

\item \emph{Array Data Structures} are used to store persistent facts that are
   only derived as initial facts and used in rules LHS. Facts stored in this
   data structure do not have the \code{prev} and \code{next} pointers
   because they are already chained by being part of a contiguous memory area.

\item \emph{Doubly Linked List Data Structures} are used to store linear facts.
   We use a double linked list because it is a very efficient way to add and
   remove facts. The \code{prev} and \code{next} pointers are used to chain
   the facts of the linked list.

\item \emph{Hash Table Data Structures} are used to improve lookup when linked
   lists are too long and when we need to do search filtered by a fixed
   argument. The virtual machine decides which arguments are best to be indexed
   (see Section~\ref{sec:implementation:indexing}) and then uses a hash table
   indexed by the appropriate argument. If we need to go through all the facts,
   we just iterate through all the facts in the table. For collisions, we use
   the doubly linked list data structure mentioned above.

\end{itemize}

Figure~\ref{fig:implementation:hash_table} shows an example for a hash table
data structure for a \code{a(int,int)} predicate with 3 linear facts indexed
by the second argument and stored as a doubly linked list in bucket \code{2}.

\begin{figure}[ht]
\centering
\includegraphics[width=0.6\textwidth]{figures/implementation/hash_table.pdf}
\caption{Hash table and doubly linked data structures for 
   a \texttt{a(int,int)} predicate containing the following facts: \code{a(1,
   2)}, \code{a(2, 12)}, \code{a(2, 42)}.}
\label{fig:implementation:hash_table}
\end{figure}

%%%%%%%%%%%%%%%%%%%%%%%%%%%%%%%%%%%%%%%%%%%%%%%%%%%%%%%%%%%%%%%%%%%%%%

\subsection{Indexing}\label{sec:implementation:indexing}

To improve fact lookup, the VM employs a fully dynamic mechanism to
decide which argument may be optimal to index.  The algorithm is
performed in the beginning of execution and empirically tries to
assess the argument of each predicate that more equally spreads the
database across the values of the argument. 

The indexing algorithm is performed in three main steps. First, it
gathers lookup statistics by keeping a counter for each
predicate's argument.  Every time a fact search is performed where
arguments are fixed to a value, the counter of such arguments is
incremented. This phase is performed during rule execution for a small
fraction of the nodes in the program.

The second step of the algorithm selects the candidate arguments of each
predicate.  If a predicate was not searched with any fixed arguments, then it
will be not indexed and there are no candidates.  If only one argument was
fixed, then such argument is the only available candidate argument and thus
immediatelly becomes the indexing argument. Otherwise, the top 2 arguments are
selected for the third phase, where \emph{entropy statistics} are collected
dynamically.

During the third phase, each candidate argument has an entropy score.
Before a node is executed, the facts of the target predicate
are used in the following formula applied for the two arguments:

\[
Entropy(A, F) = - \sum_{v \in values(F, A)} \frac{count(F, A = v)}{total(F)} \log_2 \frac{count(F, A = v)}{total(F)}
\]

\noindent where $A$ is the target argument, $F$ is the set of linear facts for
the target predicate, $values(F, A)$ is set of values of the argument $A$,
$count(F, A = v)$ counts the number of linear facts where argument $A$ is equal
to $v$ and $total(F)$ counts the number of linear facts in $F$.  The entropy
value is a good metric because it tells us how much information is needed to
describe an argument. If more information is needed, then that must be the best
argument to index.

For one of the arguments to score, $Entropy(A, F)$ multiplied by the number of
times it has been used for lookup, must be larger than the other argument. The
argument with the best score is selected and then a global variable called
\texttt{indexing\_epoch} is updated. In order to convert the node's linked lists
into hash tables, each node also has a local variable called
\texttt{indexing\_epoch} that is compared to the global variable in order to
rebuild the database according to the new indexing information.

The VM also dynamically resizes the hash table if necessary. When the hash table
becomes too dense, it is doubled in size. When it becomes too sparse, it is
reduced in half or simply transformed back into a doubly linked list. This is
done once in a while, before a node executes.

We have seen very good results with this scheme. The overhead of dynamic
indexing is negligible since programs run almost as fast as if the indices have
been added from the start. However, the programmer can still index predicates
statically, if needed, using the directive \code{index pred/arg}, where
\code{pred} is the argument name and \code{arg} is the argument number to index.


\section{Compilation}
As an intermediate step, our compiler first transforms rules into high level
instructions that are then transformed into C++. In Appendix~\ref{appendix:vm}
we present an overview of the high level instructions that can be used as a
reference to the operations that are required by the compiler. In this
section, we present the main algorithm of the compiler and its key
optimizations. However, all the examples are shown in C++ since it makes it
easier to understand how the final code looks like.

\subsection{Ordinary Rules}\label{sec:compile}

After an inference rule is compiled, it must respect the \emph{fact constraints}
(facts must exist in the database) and the \emph{join constraints} that can be
represented by variable constraints and/or boolean expressions. For instance,
consider gain the second rule of the SSSP program presented in
Fig.~\ref{code:shortest_path_program}:

\begin{Verbatim}[label=example_rule,fontsize=\codesize]
shortest(A, D1, P1), D1 <= D2, relax(A, D2, P2)
   -o shortest(A, D1, P1).
\end{Verbatim}

The fact constraints include the facts required to trigger the rule, namely
\texttt{shortest(A, D1, P1)} and \texttt{relax(A, D2, P2)}, and the join
constraints include the expression \texttt{D1 <= D2}.

However, rules may also have other less obvious join constraints such as:

\begin{Verbatim}[fontsize=\codesize]
new-neighbor-pagerank(A, B, New),
neighbor-pagerank(A, B, Old)
   -o neighbor-pagerank(A, B, New).
\end{Verbatim}

\noindent
where variable \texttt{B} must have the same value in both facts\footnote{Rule taken
from an asynchronous PageRank program.}.

\subsection{Iterators}

The data structures for facts presented in Section~\ref{sec:data_structures}
support the \emph{iterator} pattern. For linked lists, the iterator goes
through every fact in the list while the hash table iterator can either iterate
through the whole table or iterate through a single bucket. A bucket iterator is
in fact a linked list iterator that starts from a given argument.
For tries, while the default iterator goes through every fact in
the trie, it can be customized with a matching specification in
order to reduce search. A matching specification includes argument
assignments (e.g., argument $i = V$, where $V$ is a concrete value).

Iterators are heavily used in the compiled code. For instance, the second rule in
Fig.~\ref{code:shortest_path_program} is compiled as follows:

\begin{Verbatim}[numbers=left,fontsize=\codesize]
for(auto it1(linked_list("shortest").begin()); it1 != linked_list("shortest").end(); )
{
   fact *f1(*it1);
   for(auto it2(linked_list("relax").begin()); it2 != linked_list("relax").end(); )
   {
      fact *f2(*it2);

      if(f1->get_int(1) <= f2->get_int(1)) {
         fact *new_shortest(new fact("shortest"));
         new_shortest->set_int(1, f1->get_int(1));
         new_shortest->set_list(2, f1->get_list(2));

         // new fact was derived
         linked_list("shortest").push_back(new_shortest);

         // deleting facts
         it1 = linked_list("shortest").erase(it1); // remove from list
         it2 = linked_list("relax").erase(it2);
         return;
      }
      ++it2;
   }
   ++it1;
}
\end{Verbatim}


The compilation algorithm iterates through the fact expressions in the body of
the rule and creates nested loops to try all the possible combinations of facts.
For this rule, all the pairs of facts \texttt{shortest} and \texttt{relax} must
be matched until the constraint \texttt{D1 <= D2} is true. First, an iterator
for \texttt{shortest} is created that will loop through all \texttt{shortest}
facts in the list. Inside the loop, a nested iterator is created for predicate
\texttt{relax}. This inner loop includes a check for the \texttt{D1 <= D2}
constraint. If the constraint expression is true then the rule matches and a new
\texttt{shortest} fact is derived and two used linear facts are retracted by
erasing the iterators from the linked lists. Note that after the rule is
derived, the code must return since there is a higher priority rule that may be
triggered with the new \texttt{shortest} fact (see
Fig.~\ref{code:shortest_path_program}). This enforces the priority semantics
of the language.
    
If the constraint had failed, another \texttt{relax} fact would have been tried
by incrementing \texttt{it2}. Likewise, if the current \texttt{f1} fact fails
for all \texttt{f2} facts, then the next one in the list is tried
by incrementing \texttt{it1}.

\begin{figure}
\begin{algorithm}[H]
 \KwData{Rule R1, Rules}
 \KwResult{Compiled Code}
 $FactExprs \longleftarrow FactExprsFromRule(R1)$\;
 $Constraints \longleftarrow ConstraintsFromRule(R1)$\;
 $Code \longleftarrow CreateFunctionForRule()$\;
 $Iterators \longleftarrow []$\;
 $CompiledFacts = []$\;
 \While{$FactExprs$ not empty}{
  $Fact \longleftarrow RemoveBestFactExpr(FactExprs)$\;
  $CompiledFacts.push(Fact)$\;
  $Iterator \longleftarrow Code.InsertIterator(Fact)$\;
  $Iterators.push(Iterator)$\;
  \tcc{Select constraints that are covered by CompiledFacts.}
  $NextConstraints \longleftarrow RemoveConstraints(Constraints, CompiledFacts)$\;
  $Code.InsertConstraints(NextConstraints)$\;
 }
 $HeadFacts = HeadTemplatesFromRule(R1)$\;
 \While{$HeadFacts$ not empty}{
    $Fact \longleftarrow RemoveFact(HeadFacts)$\;
    $Code.InsertDerivation(Fact)$\;
 }
 \For{$Iterator \in Iterators$}{
    \If{$IsLinear(Iterator)$}{
       $Code.InsertRemove(Iterator)$\;
    }
 }
 \tcc{Enforce rule priorities.}
 \uIf{$FactsDerivedUsedBefore(Head, Program, R1)$}{
    $Code.InsertReturn()$\;
 }
 \Else{
    $Code.InsertGoto(FirstLinear(Iterators))$\;
 }
 \Return{$Code$}
\end{algorithm}
 \caption{Compiling LM rules into C++ code.}
 \label{alg:compile_rule}
\end{figure}

Figure~\ref{alg:compile_rule} presents the algorithm for compiling rules into
C++ code.  First we split the body of the rule into fact expressions and
constraints. Fact expressions map directly to iterators while fact constraints
map to \emph{if} expressions. A possible compilation strategy is to first
compile all the fact expressions and then compile the constraints. However, this
may require unneeded database lookups since some constraints may fail early.
Therefore, our compiler introduces constraints as soon as all the variables in
the constraint are all included in the already compiled fact expressions. The
order in which fact expressions are selected for compilation does not interfer
with the correctness of the compiled code, thus our compiler selects the fact
expressions ($RemoveBestFactExpr$) by their potential to activate constraints,
therefore avoiding undesirable database lookups. If two fact
expressions have the same number of new constraints, then the
compiler always picks the persistent fact expression since
persistent facts are not deleted.

Derivation of new facts belonging to the local node implies adding the new fact
to the local node data structure. Facts that belong to other nodes are sent
using an appropriate runtime API.

\subsection{Persistence Checking}

Not all linear facts need to be deleted. For instance, in the compiled rule
above, the fact \texttt{shortest(A, D1, P1)} is re-derived in the head. Our
compiler is able to turn linear loops into persistent loops for linear facts
that are retracted and then asserted.  The rule is then compiled as follows:

\begin{Verbatim}[numbers=left,fontsize=\codesize,commandchars=\$\#\&]
for(auto it1(linked_list("shortest").begin()); it1 != linked_list("shortest").end(); )
{
   fact *f1(*it1);
   for(auto it2(linked_list("relax").begin()); it2 != linked_list("relax").end(); )
   {
      fact *f2(*it2);
      if(f1->get_int(1) <= f2->get_int(1)) {
         it2 = linked_list("relax").erase(it2);
         goto next;$label#line:implementation:goto&
      }
      ++it2;
next: continue;
   }
   ++it1;
}
\end{Verbatim}

In this new version of the code, only the \texttt{relax} facts are deleted,
while the \texttt{shortest} facts remain untouched. In the SSSP program, each
node has one \texttt{shortest} fact and this compiled code simply filters out
the \texttt{relax} facts with the distances that are equal or greater than the
current best distance. Note that now have a \emph{goto statement}
(line~\ref{line:implementation:goto}) that is executed when the rule is fired.
In this case, since no new \texttt{shortest} fact was derived, we can avoid
returning to enforce rule priorities and we can continue to try to fire the rule
as many times as possible.

All the rule combinations are attempted in cases where a rule does not derive
any facts or the facts derived do not appear before the rule, that is, the new
facts are only used in lower priority rules. This is specified in the final
\emph{if statement} in Fig.~\ref{alg:compile_rule}. If the rule does not return,
then we always jump to the first loop that uses linear facts. We must jump to
the first linear loop because we cannot use
the next fact from the deepest loop since we may have constraints between the
first linear loop and the deepest loop that were validated using deleted facts.

\subsection{Updating Facts}

Many inference rules retract and then derive the same predicate but with
different arguments. The compiler recognizes those cases and instead of
retracting the fact from its linked list or hash table, it updates the fact
in-place. As an example, consider the following rule:

\begin{Verbatim}[fontsize=\codesize]
new-neighbor-pagerank(A, B, New),
neighbor-pagerank(A, B, Old)
   -o neighbor-pagerank(A, B, New).
\end{Verbatim}

Assuming that \texttt{neighbor-pagerank} is stored in a hash table and indexed by the
second argument, the code for the rule above is as follows:

\begin{Verbatim}[numbers=left,fontsize=\codesize]
for(auto it1(linked_list("new-neighbor-pagerank").begin()); it1 !=
      linked_list("new-neighbor-pagerank").end(); )
{
   fact *f1(*it1);
   // hash table for neighbor-pagerank is indexed by the second argument
   // therefore we search for the bucket using the second argument of new-neighbor-pagerank
   hash_bucket bucket(hash_table("neighbor-pagerank").find(f1->get_node(1));
   for(auto it2(bucket.begin()); it2 != bucket.end(); )
   {
      fact *f2(*it2);
      if(f1->get_node(1) == f2->get_node(1)) {
         f2->set_float(2, f1->get_float(2)); // update neighbor-pagerank
         it1 = linked_list("new-neighbor-rank").erase(it1);
         goto next;
      }
      ++it2;
   }
   ++it1;
next: continue;
}
\end{Verbatim}

Note that \texttt{neighbor-pagerank} is updated using \texttt{set\_float}. The
rule also does not return since this is the highest priority rule. If there
was a higher priority rule using \texttt{neighbor-pagerank}, then the code
would have to return since an update fact represents a new fact.

\subsection{Enforcing Linearity}

We have already introduced the \emph{goto} statement as a way to avoid reusing
retracted linear facts. However, this is not enough in order to enforce
linearity of facts. Consider the following inference rule:

\begin{Verbatim}[fontsize=\codesize]
add(A, N1), add(A, N2) -o add(A, N1 + N2).
\end{Verbatim}

Using the standard compilation algorithm, two nested loops are created, one for
each \texttt{add} fact. However, notice that there is an implicit constraint
when creating the iterator for \texttt{add(A, N2)} since this fact cannot be the
same as the first one. That would invalidate linearity since a single linear fact would
be used to prove two linear facts. This is easily solved by adding a constraint
for the inner loop by checking if the fact pointer is the same as the first one.

\begin{Verbatim}[numbers=left,fontsize=\codesize]
for(auto it1(linked_list("add").begin()); it1 != linked_list("add").end(); )
{
   fact *f1(*it1);
   for(auto it2(linked_list("add").begin()); it2 != linked_list("add").end(); )
   {
      fact *f2(*it2);
      if(f1 != f2) {
         f1->set_int(1, f1->get_int(1) + f2->get_int(1));
         it2 = linked_list("add").erase(it2);
         goto next;
      }
      ++it2;
   }
   ++it1;
next: continue;
}
\end{Verbatim}

Figure~\ref{fig:update_add} presents the steps for executing this rule when the
database contains three facts. The iterators never point to the same fact.

\begin{figure}
\centering
\begin{minipage}{.5\textwidth}
  \centering
  \includegraphics[width=.8\linewidth]{figures/compiler/update}
\end{minipage}%
\begin{minipage}{.5\textwidth}
  \centering
  \includegraphics[width=0.8\linewidth]{figures/compiler/update2}
\end{minipage}
\begin{minipage}{.5\textwidth}
   \centering
   \includegraphics[width=0.8\linewidth]{figures/compiler/update3}
\end{minipage}
\caption{Executing the add rule. First, the two iterators point to
   the first and second facts and the former is updated while the latter is
   retracted. The second iterator then moves to the next fact and the first fact is
   updated again, now to the value \texttt{6}, the expected result.}
\label{fig:update_add}
\end{figure}

\subsection{Comprehensions}

Comprehensions were initially presented in the first rule of the SSSP program.

\begin{Verbatim}[fontsize=\codesize]
shortest(A, D1, P1), D1 > D2, relax(A, D2, P2)
   -o shortest(A, D2, P2),
   {B, W | !edge(A, B, W) | relax(B, D2 + W, P2 ++ [B])}.
\end{Verbatim}

The attentive reader will remember that comprehensions are sub-rules, therefore
they should be compiled like normal rules. However, they do not need to return
due to rule priorities since all the combinations of the comprehension must be
derived. However, the rule itself must return if any of its comprehensions
has derived a fact that is used by a higher priority rule.
The example rule does not need to return since it has the highest priority and the
\texttt{relax} facts derived in the comprehension are all sent to other nodes.
The code for the rule is shown below:

\begin{Verbatim}[numbers=left,fontsize=\codesize]
for(auto it1(linked_list("shortest").begin()); it1 != linked_list("shortest").end(); )
{
   fact *f1(*it1);
   for(auto it2(linked_list("relax").begin()); it2 != linked_list("relax").end(); )
   {
      fact *f2(*it2);
      if(f1->get_int(1) > f2->get_int(1)) {
         // comprehension code
         for(auto it3(trie("edge").begin()); it3 != trie("edge").end(); ) {
            fact *f3(*it3);
            fact *new_relax(new fact("relax"));
            new_relax->set_int(1, f2->get_int(1) + f3->get_int(2));
            new_relax->set_list(append(f2->get_list(2), list(f3->get_node(1))));
            send_fact(new_relax, f3->get_node(1));
            ++it3;
         }
         f1->set_int(1, f2->get_int(1));
         f1->set_list(2, f2->get_list(2));
         it2 = linked_list("relax").erase(it2);
         goto next;
      }
      ++it2;
   }
   ++it1;
next: continue;
}
\end{Verbatim}

Special care must be taken when the comprehension's sub-rule uses the same
predicates that are derived by the main rule.
Rule inference must be atomic in the sense that after a rule matches, the
comprehensions in the head of the rule can use the facts that were present
before the body of the rule was matched.
Consider a rule with $n$ comprehensions or aggregates, where $CB_i$ and $CH_i$
is the body and head of the comprehension/aggregate, respectively, and $H$
represents the fact templates found in the head of the rule.
The formula used by the compiler to detect conflicts between predicates is the
following:

\[
\bigcup^{n}_i[CB_i \cap H] \cup \bigcup^{n}_i [CB_i \cap \bigcup^{n}_j[CH_j]]
\]

If the result of the formula is not empty, then the compiler disables
optimizations for the conflicting predicates and derives the corresponding facts
into the fact buffer that are then added back into the database.
Fortunately, most rules in LM programs do not show conflicts and thus
can be fully optimized.

\subsection{Aggregates}

Aggregates are similar to comprehensions. They are also sub-rules but a value is
accumulated for each combination of the sub-rule. After all the combinations are
inferred, a final head term is derived with the accumulated term. Consider the following
PageRank rule:

\begin{Verbatim}[fontsize=\codesize]
update(A), pagerank(A, OldRank)
      -o [sum => V | B | neighbor-pagerank(A, B, V) | neighbor-pagerank(A, B, V) |
            pagerank(A, damp/P + (1.0 - damp) * V)].
\end{Verbatim}

The variable \texttt{V} is initialized to \texttt{0.0} and sums all
the PageRank values of the neighbors as seen in the code below. The aggregate
value is then used to update the second argument of the initial
\texttt{pagerank} fact.

\begin{Verbatim}[numbers=left,fontsize=\codesize]
for(auto it1(linked_list("pagerank").begin()); it1 != linked_list("pagerank").end(); )
{
   fact *f1(*it1);
   for(auto it2(linked_list("update").begin()); it2 != linked_list("update").end(); )
   {
      fact *f2(*it2);
      double acc(0.0); // aggregate accumulator.
      for(auto it3(linked_list("neighbor-pagerank").begin()); it3 !=
            linked_list("neighbor-pagerank").end(); ) {
         fact *f3(*it3);
         acc += f3->get_float(2);
         ++it3; // the sub-rule has no head since neighbor-pagerank is re-derived
      }
      // head of the aggregate
      f1->set_float(1, damp / P + (1.0 - damp) * V);
      goto next;
   }
   ++it1;
next: continue;
}
\end{Verbatim}

\subsection{Coordination Directives}

Coordination directives are compiled in two different ways, depending on whether
they appear in the body or in the head of the rule. Coordination facts in the
body are compiled into code that inspects the state of the virtual machine. For
example, the fact \texttt{priority} will inspect the target node, retrieve the
current priority and assign the priority to a variable.  Coordination facts in
the head of the rule are also implemented as API commands of the virtual
machine instead of being added to the database as facts. Semantically, action
facts are like any other. However, since they are immediately used by the
machine, there is no need to store them in the database, therefore avoiding
unnecessary allocations and deallocations.



\section{Summary}

This section provided a full description of the implementation of LM. Although the parallel and load balancing facilities of the
runtime system are mostly completed, there is still some implementation work to be done, specially at the coordination level.
We also explained how \lang can be implemented in different distributed systems.
