To support thread-based facts, both the compilation and runtime system described in
Chapter~\ref{chapter:implementation} require some changes.

\subsection{Compiler}

The compiler needs to recognize rules that use thread facts. For thread rules,
the compiler checks if the rule's body is using facts from the same thread by
checking the first argument of each fact. For mixed rules, the rule's body may
use a thread \code{T} and a node \code{A} and all the node facts have to use
\code{A}, while all threads facts must use \code{T} as the first argument. If
the programmer was to retrieve either the thread or the node for the current
computation, she may use \code{running(T, A)}.

Once rules are type checked, the iteration code for thread-based facts needs to
be adapted. When a rule requires facts from the thread, it must use the data
structures from the thread. The runtime API used for inserting thread facts is
also different since they have to be added to the thread's database.

\subsection{Runtime}

In the runtime system, thread-based facts are implemented just like regular
facts. Each thread has its own database of facts and uses exactly the same data
structures for facts, as presented before for regular nodes.  The major
difference between a regular node and a thread node is that a thread node is
never put into the work queue of its thread. As shown in the updated work loop
presented in Fig.~\ref{alg:threads:work_loop}, the thread node executes
alongside the regular node when $TH.process\_node$ is called. It is also
important to note that, before a thread becomes idle, it may have potential
candidate thread rules that are now derivable because another thread has derived
thread facts in the current thread. In particular, it is entirely possible to have
programs that only deal with thread facts.

\begin{figure}
\begin{algorithm}[H]
   \KwData{Thread TH}
   \While{true}{
      $TH.work\_queue.lock()$\;
      $node \longleftarrow TH.work\_queue.pop\_node()$ \;
      $TH.work\_queue.unlock()$\;
      \uIf{$node$}{
        \underline{$TH.process\_node(node, TH.thread\_node)$}\;
      }
      \Else{
        \tcp{The thread's node may have candidate rules using incoming thread facts}
        \underline{$TH.process\_node(nil, TH.thread\_node)$}\;
        \tcp{Attempt to steal some nodes.}
         \If{$\bang TH.steal\_nodes()$}{
            $TH.become\_idle()$\;
            \While{$len(TH.work\_queue) == 0$}{
               \tcp{Try to terminate}
               \If{$TH.synchronize\_termination()$}{
                  \textbf{terminate}\;
               }
               \If{$TH.steal\_nodes()$}{
                  \tcp{Thread is still in the stealing state}
                  break\;
               }
            }
            \tcp{There's new nodes in the queue.}
            $TH.become\_active()$\;
         }
      }
 }
\end{algorithm}
\mycap{Thread work loop updated to take into account thread-based facts.
New or modified code is underlined.}
\label{alg:threads:work_loop}
\end{figure}

Thread-based facts also introduce new synchronization logic in the runtime
system. For instance, when a rule derives a new thread fact on another thread,
it needs to synchronize with that thread (using the appropriate thread node
locks) to add the facts to the thread's database. When a thread is executing its
own node or a regular node, it also must lock the thread node's \emph{DB Lock}
in order to protect its data structures from being manipulated by other threads
concurrently.

Matching rules using thread facts requires special care since they may require
both facts from the regular node and from the thread's node. Before a node is
executed, the rule engine (Section~\ref{sec:implementation:rule_engine}) of the
regular node is updated to take into account the facts of the thread's node so
that mixed rules (rules that use both thread and regular facts) execute. In this
scheme, mixed rules may be unsuccessfully fired repeatedly until a node which
has matching facts gets to execute. Since the LHS of mixed rules use an implicit
\code{running(T, N)} fact, it is enough that a different node is running
to fire mixed rules as long as the running node has the required facts.
Without using an implicit \code{running} fact, the system would need to lookup
for a regular node that would successfully activate a given mixed rule. This
would be expensive since some programs might have millions of regular nodes.
