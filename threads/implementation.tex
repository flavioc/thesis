The compilation and runtime system described in
Chapter~\ref{chapter:implementation} requires ysome changes in both the compiler
and in the runtime system to support thread-based facts.

\subsection{Compiler}

The compiler needs to recognize rules that use thread facts. For thread rules,
the compiler checks if the rule's body is using facts from the same thread by
checking the first argument of each fact. For mixed rules, the rule's body may
use a thread \code{T} and a node \code{A} and all the node facts have to use
\code{A}, while all threads facts must use \code{T} as the first argument. If
the programmer was to retrieve either the thread or the node for the current
computation, she may use \code{running(T, A)}.

Once rules are type checked, the iteration code for thread-based facts
needs to change. The database iterator will refer to the current thread in order to
fetch candidate facts since using the standard node database will return an
empty iterator. The runtime API used for inserting thread facts is also
different since they have to be added to the thread's database.

\subsection{Runtime}

Each thread has its own database of facts that is identical to a regular node
but only contains thread predicates. The major difference between a regular node
and a thread node is that a thread node is never put into the work queue of its
thread. As shown in the updated work loop presented in
Fig.~\ref{alg:threads:work_loop}, the thread node executes alongside the regular
node when $TH.process\_node$ is called. It is also important to note that,
before a thread becomes idle, it may have potential candidate thread rules that
are now derivable because another thread has derived thread facts in the current
thread. Note that it is entirely possible to have programs that only deal with
thread facts. These kinds of programs use only explicit parallelism.

\begin{figure}
\begin{algorithm}[H]
   \KwData{Thread TH}
   \While{true}{
      $TH.work\_queue.lock()$\;
      $node \longleftarrow TH.work\_queue.pop\_node()$ \;
      $TH.work\_queue.unlock()$\;
      \uIf{$node$}{
        \underline{$TH.process\_node(node, TH.my\_node)$}\;
      }
      \Else{
        \tcc{The thread's node may have candidate rules using incoming thread facts}
        \underline{$TH.process\_node(nil, TH.my\_node)$}\;
        \tcc{Attempt to steal some nodes.}
         \If{$\bang TH.steal\_nodes()$}{
            $TH.become\_idle()$\;
            \While{$len(TH.work\_queue) == 0$}{
               \tcc{Try to terminate}
               \If{$TH.synchronize\_termination()$}{
                  \textbf{terminate}\;
               }
               \If{$TH.steal\_nodes()$}{
                  \tcc{Thread is still in the stealing state}
                  break\;
               }
            }
            \tcc{There's new nodes in the queue.}
            $TH.become\_active()$\;
         }
      }
 }
\end{algorithm}
\caption{Thread work loop updated to take into account thread-based facts.
New and modified code is underlined.}
\label{alg:threads:work_loop}
\end{figure}

Thread facts also add points of synchronization the runtime system. For
instance, when a rule derives a new thread fact on another thread, it needs to
synchronize with that thread (using locks) to add the facts to the thread's
database.

\subsubsection{Matching rules}

Matching rules using thread facts requires special care since some rules may
require both facts from a node and from a thread. Before a node is executed, the
matching engine of the regular node is updated to take into account the facts in
the thread. If the mixed rules are activated, they need to be executed, even if
they have failed under a different node. The reason is simple: the rule
constraints may now hold under the current node's database and therefore the
rule needs to be executed again.

\subsection{Graph Of Threads}

We added several helpful predicates that allow the programmer to inspect the
graph of threads and reason about the state of computation as it relates to
threads.

\begin{itemize}
   \item \code{\bang thread-list(T, L)}: Fact instantiated in all threads where
      \code{L} is a list of all threads executing in the system.

   \item \code{\bang other-thread(T1, T2)}: Connects thread \code{T1} to all the
      other threads \code{T2} executing in the system.

   \item \code{\bang leader-thread(T, TLeader)}: Fact instantiated in all
      threads where \code{TLeader} refers to a selected thread (usually the
      first thread in \code{L} of \code{\bang thread-list(T, L)}).

   \item \code{running(T, A)}: Used to retrieve the current node \code{A}
      running on thread \code{T}.
\end{itemize}

With the exception of \code{running}, every other fact is added at the beginning
of the program as a persistent axiom.
