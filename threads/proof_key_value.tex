In this section, we prove several properties of the Binary Search Tree~(BST) as
presented in Fig.~\ref{code:threads:btree_lookup_cache}. First, we assume that
the initial facts of the program represent a valid binary search tree,
therefore, for every node with a given key $k$, the left branch contains keys
that are lesser or equal than $k$, and the right branch contains keys that are
greater than $k$.

\begin{invariant}[Immutable keys]
Every node has a \code{value}$(A, Key, Value)$ fact and the $Key$ never changes.
\end{invariant}
\begin{proof}
By induction. In the base case, every node has a \code{value} initial fact. In the
inductive case, each rule that uses a \code{value} fact also re-derives it and
the $Key$ argument never changes.
\end{proof}

\begin{invariant}[Thread state]\label{appendix:proof:key_value_invariant}
Every thread $T$ in the program has a \code{cache-size}$(T, Total)$ fact that
represents the number of valid cache items in the form of \code{cache}$(T, Node, Key)$
facts. Furthermore, there are no two \code{cache}$(T, Node_1, Key_1)$ and
\code{cache}$(T, Node_2, Key_2)$ facts where $Node_1 = Node_2$.
\end{invariant}
\begin{proof}
By induction on each rule application.

In the base case, all threads start with a \code{cache-size}$(T, 0)$ fact and no
\code{cache}$(T, Node, Key)$ facts.
 
In the inductive case, we must analyze each rule separately.

\begin{itemize}

  \item Rule 1: the \code{cache}$(T, A, Key)$ fact is kept in the database.

  \item Rule 2: this rule adds a new valid \code{cache}$(T, A, Key)$ fact and
     correctly increments the \code{cache-size} counter. However, we must ensure
     that the cache item is unique. For that, we know that this rule uses the
     same \code{replace} and a \code{value} facts of rule 1, therefore, if there
     was already a cache item in the database, rule 1 would run before rule 2
     since rule 1 has higher priority. Rule 1 deals with cases where there is
     already a \code{cache} item, while rule 2 adds a new item if it does not
     exist.

   \item Rule 3: the \code{cache}$(T, TargetNode, RKey)$ is kept in the
      database.

   \item Rules 4 and 5: they do not derive or consume any cache related facts.

\end{itemize}
\end{proof}

\begin{theorem}[Valid replace]
Assume a valid BST and a set of threads with a valid cache. Given a
\code{replace}$(A, RKey, RValue)$ fact at node $A$, the node $N$ with key $RKey$
will replace its \code{value}$(N, RKey, Value)$ fact to \code{value}$(N, RKey,
RValue)$.
\end{theorem}
\begin{proof}

First consider that node $A$ makes up a BST and then use induction on the size
of that smaller BST.

In the base case, where $A = N$, rule 1 and 2 will apply. Rule 1 is applied if
the executing thread has node $N$ already in the cache. Rule 2 applies when the
executing thread does not have the node $N$ in the cache.

In the inductive case, there are two sub-cases:

\begin{itemize}
   \item The executing thread has a valid \code{cache} item for key $RKey$ (from
      Variant~\ref{appendix:proof:key_value_invariant}),
      therefore rule 3 is applied and a \code{replace} fact is derived at node
      $N$.

   \item Without a valid \code{cache} item, rules 4 or 5 are applied. Rule 4
      applies if the key is in the left branch and rule 5 applies if the key is
      in the right branch. Use the induction hypothesis on the selected branch.

\end{itemize}
\end{proof}
