\subsection{Scheduling}

\begin{figure}[h!]
\begin{Verbatim}[numbers=left,fontsize=\codesize,commandchars=*\#\&]
type linear running(thread, node). type linear inactive(node).
type linear priority(node, float). type linear default-priority(node, float).
type linear available-work(node, bool).
type linear active(thread).  type linear idle(thread).
type linear owner(node, thread).

owner(A, T). // Initially node assignment.
available-work(A, true). // All nodes have available work.

active(T). // All threads are active.

node-fact(A, X, Y),
other-fact(A, X, B),
priority(A, PA), priority(B, PB),
owner(B, TB), running(T, A),
available-work(B, _)
   -o remote-fact(B), local-fact(A),
      owner(B, TB), running(T, A),
      priority(A, PA), priority(B, PB),
      available-work(B, true).

active(T), running(T, A), priority(A, Prio),
default-priority(A, DefPrio), available-work(A, T),
   -o inactive(A), priority(A, DefPrio),
      default-priority(A, DefPrio),
      available-work(A, false), idle(T).

[max => Prio |
   idle(T), owner(A, T),
   priority(A, Prio), available-work(A, true)]
   -o active(T), owner(A, T),
      running(T, A), available-work(A, true),
      priority(A, Prio).

idle(T), !other-thread(T, Other)
owner(A, Other), inactive(A),
available-work(A, true)
   -o idle(T), owner(A, T),
      inactive(A), available-work(A, true).
\end{Verbatim}
\caption{Modeling the operational semantics as a LM program.}
\label{code:threads:modeling}
\end{figure}

\begin{figure}[h!]
\begin{Verbatim}[numbers=left,fontsize=\codesize,commandchars=*\#\&]
type linear static(node).  type linear moving(node).
type linear set-priority(node, float).
type linear set-default-priority(node, float).
type linear just-moved(node). type linear move-to-thread(node, thread).

moving(A). // All nodes can be stolen.
inactive(A). // Nodes are not being processed initially.

// Priority facts.
priority(A, initial-priority).
default-priority(A, initial-priority).

schedule-next(A), running(T, A)
   -o running(T, A), set-priority(A, +00).

running(T, A), set-priority(A, P1),
priority(A, P2), P2 < P1
   -o running(T, A), priority(A, P1).

running(T, A), set-priority(A, P1),
priority(A, P2), P2 >= P1
   -o running(T, A), priority(A, P2).

running(T, A), set-default-priority(A, P1),
default-priority(A, P2), P2 < P1
   -o running(T, A), default-priority(A, P1).

running(T, A), set-default-priority(A, P1),
default-priority(A, P2), P2 >= P1
   -o running(T, A), default-priority(A, P2).

running(T, A), set-thread(A, T),
available-work(A, _)
   -o available-work(A, true),
      move-to-thread(A, T).

move-to-thread(A, T2),
running(T, A), owner(A, T),
moving(A), active(T)
   -o idle(T), inactive(A),
      owner(A, T2), static(A),
      just-moved(A).
\end{Verbatim}
\caption{Modeling the operational semantics as a LM program.}
\label{code:threads:modeling}
\end{figure}
