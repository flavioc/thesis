The introduction of thread-based facts allows for explicit parallelism in a
language that is purely implicit. This introduces issues when attempting to
prove the correctness of programs because the behavior of threads and the
scheduling strategy is now also part of the program logic. Some of this behavior
is hidden from programs because it is part of how coordination facts and thread
scheduling works on the virtual machine.

Consider the SBP program in Fig.~\ref{code:threads:sbp} where in
lines~\ref{line:threads:splash_part1}-\ref{line:threads:splash_part2} the graph
of nodes is partitioned into regions. In order to prove the correct
partitioning, we need to know how the VM initially randomly assigns nodes to
threads and also how coordination facts \code{set-thread} and \code{just-moved}
are used by the VM.  Fortunately, since linear logic is the foundation of LM, it
is possible to model the semantics of LM by using LM rules. In
Chapter~\ref{chapter:implementation}, we have seen that threads and nodes
transition between different states during execution and thus we are going to
model that. We first define the following node facts:

\begin{itemize}

   \item \code{inactive(node A)}: Fact present on nodes that are not currently
      running on a thread. Facts \code{running(T, A)} and \code{inactive(A)} are
      mutually exclusive.

   \item \code{owner(node A, thread T)}: Fact that indicates the current thread
      \code{T} that currently owns node \code{A}.

   \item \code{available-work(node A, bool F)}: Fact that indicates if node
      \code{A} has new facts to be processed.

\end{itemize}

In terms of thread facts we have the following:

\begin{itemize}
   \item \code{active(thread T)}: Fact exists if thread \code{T} is currently
      active.

   \item \code{idle(thread T)}: Fact exists if thread \code{T} is currently
      idle. Facts \code{idle(T)} and \code{active(T)} are mutually exclusive.
\end{itemize}

Figure~\ref{code:threads:modeling} presents how the operational semantics for a
given LM program is modeled using the LM language itself.

First, we define the initial facts: \code{owner(A, T)} on
line~\ref{line:threads:model_owner}, which assigns a node to a thread;
\code{available-work(A, F)} on line~\ref{line:threads:model_available}, where
\code{F = true} if node \code{A} has initial facts, otherwise \code{F = false};
\code{active(T)} on line~\ref{line:threads:model_active} to mark each thread as
\emph{active}; and \code{moving(A)} on line~\ref{line:threads:model_moving} so
that all nodes can move between threads.

Each program rule is translated as shown in
lines~\ref{line:threads:model_rule1}-\ref{line:threads:model_rule2}. The
original rule was \code{node-fact(A, Y), other-fact(A, B) -o remote-fact(B),
local-fact(A)}, so we have a local derivation of \code{local-fact(A)} and a
remote-derivation of \code{remote-fact(B)}. In the translation, we update
\code{available-work} of node \code{B} to \code{true} because there is a new
derivation for \code{B}. The fact \code{running(T, A)} is used to ensure that
thread \code{T} is running on node \code{A}. Note that for thread rules we do
not need to use \code{running(T, A)} on the rule's LHS and the thread running
the rule does not even need to have \code{active(T)}. This enforces the
non-deterministic semantics for thread rules.

After the program rules are translated, we have the rule in
lines~\ref{line:threads:model_drop_node1}-\ref{line:threads:model_drop_node2}
which forces thread \code{T} to stop running on node \code{A}. Here, we use the
coordination fact \code{default-priority} to update the priority of node
\code{A}. The thread's state switches to \code{idle(T)}, while the node's state
changes to \code{inactive(A)}. Note that this rule must appear after the
program's rules because the rule priorities are exploited in order to force
thread \code{T} to derive all the candidate rules for \code{A}.

If a thread is idle, then it is able to derive the rule in
lines~\ref{line:threads:model_next_node1}-\ref{line:threads:model_next_node2} in
order to select another node for execution. We use a \code{max} selector to
select the node \code{A} with the highest priority \code{Prio}. If there is such
node, the node changes to \code{running(T, A)} and thread \code{T} changes to
\code{active(T)}.

Finally, the rule in
lines~\ref{line:threads:model_steal1}-\ref{line:threads:model_steal2} allows for
threads to steal nodes owned by other threads. If a node is not currently being
executed (\code{inactive(A)}), can be moved (\code{moving(A)}), and is owned by
another thread \code{Other} (\code{owner(A, Other)}), then the thread owner is
updated, potentially allowing the previous rule to execute.

\begin{figure}[h!]
\begin{Verbatim}[numbers=left,fontsize=\codesize,commandchars=\\\#\&]
type linear running(thread, node). type linear inactive(node).
type linear priority(node, float). type linear default-priority(node, float).
type linear available-work(node, bool).
type linear active(thread).  type linear idle(thread).
type linear owner(node, thread).
type linear moving(node).

owner(A, T). // Initially node assignment.\label#line:threads:model_owner&
available-work(A, F). // Some nodes have available work.\label#line:threads:model_available&
moving(A). // All nodes can be stolen.\label#line:threads:model_moving&
active(T). // All threads are active.\label#line:threads:model_active&

// Program rules go here.\label#line:threads:model_rule1&
\underline#node-fact(A, Y)&,
\underline#other-fact(A, B)&,
running(T, A), available-work(B, _)
   -o \underline#remote-fact(B)&, \underline#local-fact(A)&,
      running(T, A),
      available-work(B, true).\label#line:threads:model_rule2&

// Switching to another node.\label#line:threads:model_drop_node1&
active(T), running(T, A), priority(A, Prio),
default-priority(A, DefPrio), available-work(A, T)
   -o inactive(A), priority(A, DefPrio),
      default-priority(A, DefPrio),
      available-work(A, false), idle(T).\label#line:threads:model_drop_node2&

// Select next node to be processed.\label#line:threads:model_next_node1&
[max => Prio |
   idle(T), owner(A, T),
   priority(A, Prio), available-work(A, true)]
   -o active(T), owner(A, T),
      running(T, A), available-work(A, false),
      priority(A, Prio).\label#line:threads:model_next_node2&

// Attempt to steal a node.\label#line:threads:model_steal1&
idle(T), !other-thread(T, Other)
owner(A, Other), inactive(A),
available-work(A, true),
moving(A)
   -o idle(T), owner(A, T), moving(A),
      inactive(A), available-work(A, true).\label#line:threads:model_steal2&
\end{Verbatim}
\caption{Modeling the operational semantics as a LM program.}
\label{code:threads:modeling}
\end{figure}

\subsection{Scheduling}

We now model several coordination facts presented in
Chapter~\ref{chapter:coordination} using LM rules. We focus on
\code{set-thread}, \code{set-priority}, \code{just-moved}, and
\code{schedule-next}. The rules are presented in
Fig.~\ref{code:threads:modeling_scheduling} and should be the highest priority
rules in LM programs.

We start with the axiom~\code{priority(A, initial-priority)} and
\code{default-priority(A, initial-priority)}
(lines~\ref{line:threads:model_prio} and~\ref{line:threads:model_defprio}) to
define the initial priorities of nodes. In line~\ref{line:threads:model_snext}
we have the rule for the \code{schedule-next} coordination fact, which simply
re-derives a \code{set-priority} but with an infinite priority. Fact
\code{set-priority} is processed in
lines~\ref{line:threads:model_set1}-\ref{line:threads:model_set2} by updating
the priority values in the \code{priority} facts. As explained in
Chapter~\ref{chapter:coordination}, only higher priorities are taken into
account.

For the \code{set-thread} coordination fact, we have
lines~\ref{line:threads:model_thread1}-\ref{line:threads:model_thread2}. The
first rule applies when the node is currently executing on some thread, forcing
the thread to stop executing the node and to derive \code{just-moved(A)}. In the
second rule, node \code{A} is not being executed and the \code{owner} fact is
simply updated to the new thread.

Note that the rules for updating the coordination sensing facts do not require
the \code{running} predicate in the rule's body, therefore it should not matter
which thread does the update as long as it is done. In the VM, the update is
always done by thread that derives the coordination fact for efficiency reasons.

\begin{figure}[h!]
\begin{Verbatim}[numbers=left,fontsize=\codesize,commandchars=\\\#\&]
type linear static(node).  type linear moving(node).
type linear set-priority(node, float).
type linear just-moved(node). type linear move-to-thread(node, thread).

// Priority facts.
priority(A, initial-priority).\label#line:threads:model_prio&
default-priority(A, initial-priority).\label#line:threads:model_defprio&

schedule-next(A) -o set-priority(A, +00).\label#line:threads:model_snext&

set-priority(A, P1), priority(A, P2), P2 < P1\label#line:threads:model_set1&
   -o priority(A, P1).

set-priority(A, P1), priority(A, P2), P2 >= P1
   -o priority(A, P2).\label#line:threads:model_set2&

running(T, A), set-thread(A, T),\label#line:threads:model_thread1&
available-work(A, _), moving(A)
   -o available-work(A, true),
      inactive(A), static(A),
      just-moved(A).

inactive(A), set-thread(A, T),
owner(A, TOld), moving(A),
available-work(A, _)
   -o static(A), owner(A, T), just-moved(A),
      available-work(A, true).\label#line:threads:model_thread2&
\end{Verbatim}
\caption{Modeling the operational semantics for coordination facts as a LM program.}
\label{code:threads:modeling_scheduling}
\end{figure}
