The asynchronous version of the PageRank algorithm avoids synchronization
between iterations, thus trading precision for convergence speed.  The
formulation is similar to the formulation presented in
Equation~\ref{eq:language:pagerank}:

\begin{align}
x_{i}(t + 1) = G_{i} [x_{0}(t^{i}_{0}(t)), \cdots, x_{n}(t^{i}_{n}(t)]^{T}\label{eq:appendix:pagerank_async}
\end{align}

The major difference is that now we multiply $G_{i}$ (the row of inbound links
for page $i$) with a transposed vector with PageRank values that are necessarily
not from iteration $t$. The expression $t^{i}_{j}(t)$ refers to the iteration
before $t$ when the page $i$ received the PageRank value of page $j$. The value
$x_{j}(t^{i}_{j}(t))$ then refers to the most up-to-date PageRank value of $j$
received at page $i$ that is going to be used to compute the PageRank value of
$i$ at $t + 1$.

Figure~\ref{language:code:async_pagerank} shows the LM code for this particular
version. The program uses seven predicates which are described as follows:
\code{outbound} represents an outbound page link; \code{numInbound} is the
number of inbound page links; \code{pagerank} represents the current PageRank
value of the node; \code{neighbor-pagerank} is the PageRank value of an inbound
page; \code{new-neighbor-pagerank} represents a new PageRank value of an inbound
page; \code{sum-ranks} is a temporary predicate used for computing new
PageRanks; and \code{update} re-computes the PageRank value from the neighbor's
PageRank values. Rules in
lines~\ref{line:language:apagerank_update1}-\ref{line:language:apagerank_update2}
update the \code{neighbor-pagerank} values, while rule in
lines~\ref{line:language:apagerank_new1}-\ref{line:language:apagerank_new2}
asynchronously updates the current PageRank value. Finally, the third rule in
lines~\ref{line:language:apagerank_new1}-\ref{line:language:apagerank_new2}
informs the neighbor nodes about the newly computed PageRank value by deriving
multiple \code{new-neighbor-pagerank} facts. Note that in this rule, we use the
function \code{fabs} that computes the absolute value of a floating point
number.

\begin{figure}[h!]
\begin{Verbatim}[numbers=left,fontsize=\codesize,commandchars=\\\#\&]
type outbound(node, node, float).\hfill// Predicate declaration.
type numInbound(node, int).
type linear pagerank(node, float, int).
type linear neighbor-pagerank(node, node Neighbor, float Rank, int Iteration).
type linear new-neighbor-pagerank(node, node Neighbor, float Rank, int Iteration).
type linear sum-ranks(node, float).
type linear update(A).

new-neighbor-pagerank(A, B, New, Iteration),\label#line:language:apagerank_update1&\hfill// Rule 1: update neighbor value.
neighbor-pagerank(A, B, Old, OldIteration),
Iteration > OldIteration
   -o neighbor-pagerank(A, B, New, Iteration).

new-neighbor-pagerank(A, B, New, Iteration),\hfill// Rule 2: update neighbor value.
neighbor-pagerank(A, B, Old, OldIteration),
Iteration <= OldIteration
   -o neighbor-pagerank(A, B, Old, OldIteration).\label#line:language:apagerank_update2&

sum-ranks(A, Acc),\label#line:language:apagerank_new1&\hfill// Rule 3: propagate new pagerank.
NewRank = damping/float(pages) + (1.0 - damping) * Acc,
pagerank(A, OldRank, Iteration)
      -o pagerank(A, NewRank, Iteration + 1),
         {B, W, Delta | !outbound(A, B, W), Delta = fabs(NewRank -
               OldRank) * W -o new-neighbor-pagerank(B, A, NewRank, Iteration + 1),
               if Delta > bound then update(B) end}.\label#line:language:apagerank_new2&

update(A), update(A) -o update(A).\hfill// Rule 4: prune update facts.

update(A),\label#line:language:apagerank_new1&\hfill// Rule 5: start update process.
!numInbound(A, T)
   -o [sum => V; B, Val, Iter | neighbor-pagerank(A, B, Val, Iter),
         V = Val/float(T) -o neighbor-pagerank(A, B, Val, Iter) -> sum-ranks(A, V)].\label#line:language:apagerank_new2&

pagerank(A, 1.0 / float(pages), 0).\hfill// Initial facts.
update(A).
neighbor-pagerank(A, B, 1.0 / float(pages), 0). // pagerank of B is ...
\end{Verbatim}
\caption{Asynchronous PageRank program.}
\label{language:code:async_pagerank}
\end{figure}

To build the proof of correctness, we must again prove several program
invariants. In what follows, we will prove that this particular program
corresponds to the computation on a nonnegative matrix of unit spectral
radius, which has been proven to
converge~\cite{DBLP:journals/corr/abs-cs-0606047,
Lubachevsky:1986:CAA:4904.4801}.

\begin{invariant}[Page invariant]
Each page/node has a single \code{pagerank(A, Value, Iteration)} and:
\begin{itemize}
   \item for each outbound link, a single \code{\bang outbound(A, B, W)} fact.
   \item for each inbound link, a single \code{neighbor-pagerank(A, B, V, Iter)}
      fact.
   \item for each \code{\bang outbound(A, B, W)}, a single
      \code{neighbor-pagerank(B, A, V, Iter)}.
\end{itemize}
\end{invariant}

\begin{proof}

All initial facts validate the 3 conditions of the variant. Note that the third
condition is also validated by the initial facts, although not all facts are shown in
the code.

In relation to rule application:

\begin{itemize}
   \item Rule 1: inbound link re-derived.
   \item Rule 2: inbound link re-derived.
   \item Rule 3: \code{pagerank} re-derived.
   \item Rule 4: Nothing happens.
   \item Rule 5: inbound links re-derived in the comprehension.
\end{itemize}
\end{proof}

\begin{lemma}[Neighbor rank lemma]

Given a fact \code{neighbor-pagerank(A, B, V, Iter)} and a set of facts
\code{new-neighbor-pagerank(A, B, New, Iter2)}, we end up with a single
\code{neighbor-pagerank(A, B, V', Iter')}, where \code{Iter} is the greater of
\code{Iter} and \code{Iter2'}.

\end{lemma}
\begin{proof}
By induction on the number of \code{new-neighbor-pagerank} facts.

Base case: \code{neighbor-pagerank} remains.

Inductive case: given one \code{new-neighbor-pagerank} fact:

\begin{itemize}
   \item Rule 1: the new iteration is older and thus \code{neighbor-pagerank}
   is replaced. By applying induction, we know that we will select either the
   new best iteration or a better iteration from the remaining set of
   \code{new-neighbor-pagerank} facts.
   \item Rule 2: the new iteration is not older and we keep the old
   \code{neighbor-pagerank} fact. By induction, we select the best from either
   the current iteration or some other (from the set).
\end{itemize}
\end{proof}

\begin{lemma}[Update lemma]
Given a new \code{update} fact, rule 5 will run.
\end{lemma}
\begin{proof}
By induction on the number of \code{update} facts.

Base case: rule 5 will run for the first update fact.

Inductive case: rule 4 runs first because it has a higher priority, reducing
the number of \code{update} facts by one. By induction, we know that by
using the remaining \code{update} facts, rule 5 will run.
\end{proof}

\begin{lemma}[Pagerank update lemma]
(1) Given at least one \code{update} fact, the \code{pagerank(A, $V_{I}$,
I)} fact will be updated to become \code{pagerank(A, $V_{I + 1}$, I +
1)}, where \code{$V_{I + 1} = damping / P + (1.0 - damping)\sum_{B,
I} Val_{I,B} W_{B}$}.

with $W_{B} = 1.0/T$ (where $T$ is the number of outbound links of \code{B}) and
$Val_{I,B}$ from \code{neighbor-pagerank(A, B, $Val_{I, B}$, $I$)}.

(2) For all \code{B} outbound nodes (from \code{\bang outbound(A, B,
W)}, a \code{new-neighbor-pagerank(B, A, $V_{I+1}$, $I + 1$)} is generated.

(3) For all \code{B} outbound nodes (from \code{\bang outbound(A, B,
W)}), a \code{update(B)} is generated if 
$fabs(V_{I + 1} - V_{I}) W > bound$.
\end{lemma}
\begin{proof}
Using the Update lemma, rule 5 will necessarily run, which will derive
\code{sum-ranks(A, $\sum_{B, I} (Val_{I,B} W_B)$)} and
fulfills (3).

Fact \code{sum-ranks} will necessarily activate rule 4, computing $V_{I+1}$ and
updating \code{pagerank}. (2) and (3) are fulfilled through the comprehension of
rule 4.

\end{proof}

\begin{invariant}[New neighbor rank equality]
All \code{new-neighbor-pagerank(A, B, V, I)} facts are generated from a corresponding
\code{pagerank(B, V, I)} fact, therefore the iteration of any
\code{new-neighbor-pagerank} is at least the same or less than the iteration of
the current PageRank.
\end{invariant}
\begin{proof}
No initial facts to prove.

\begin{itemize}
   \item Rule 3: true, new fact is generated.
   \item Rule 4: the fact is kept.
\end{itemize}
\end{proof}

\begin{invariant}[Neighbor rank equality]
All \code{neighbor-pagerank(A, B, V, I)} facts have one corresponding
\code{pagerank(B, V, I)} fact and the iteration of the \code{neighbor-pagerank}
is the same or less than the current iteration of the corresponding
\code{pagerank}.
\end{invariant}
\begin{proof}
By analyzing initial facts and rules.

Axioms: true.

Rule cases:

\begin{itemize}
   \item Rule 1: uses \code{new-neighbor-pagerank} fact (use new neighbor rank
         equality invariant).
   \item Rule 2: same fact is re-derived.
\end{itemize}
\end{proof}

\begin{theorem}[Pagerank convergence]
The program will compute the PageRank of all nodes that is within \code{bound} error
of an asynchronous PageRank computation.
\end{theorem}
\begin{proof}

Using the program initial facts, we start with the same PageRank value for all
nodes.  The \code{\bang outbound(A, B, W)} fact forms the $n \times n$ square
matrix (number of nodes) and is the Google Matrix.  All the initial PageRank
values can be seen as a vector that adds up to $1$.

The PageRank computation from the "Pagerank update lemma" computes $V_{I + 1} =
damping / P + (1.0 - damping)\sum_{B, I'} (W_{B} Val_{I',B})$, where $I'
\leq I$ (from Neighbor rank equality invariant).

Consider that each node contains a column $G_i$ of the Google matrix. The
PageRank computation can then be represented as: \newline


$V_{I + 1} = G_i fix([Val_{I_1, B_1}, ..., Val_{I_p, B_p}])$ \hfill (1) \\


Where $p$ is the number of inbound links and $Val_{I_j, B_j}$ is the value of the
\code{neighbor-pagerank(A, $B_j$, $Val_{I_j, B_j}$, $I_j$)}. The $fix()$ function
takes the neighbor vector and expands it with zeros corresponding to nodes that
are not inbound links. This is the expected formulation for the asynchronous
PageRank computation~\cite{DBLP:journals/corr/abs-cs-0606047} as shown
in~\ref{eq:appendix:pagerank_async}.

From~\cite{DBLP:journals/corr/abs-cs-0606047, Lubachevsky:1986:CAA:4904.4801} we
know that equation (1) will improve (converge) the PageRank value, given that
some new neighbor PageRank values are sent to node $i$ and by the fact that
$G_i$ is a nonnegative matrix of unit spectral radius. Let's use induction by
assuming that there is at least one \code{update} fact that schedules a node to
improve its PageRank. We want to prove that such fact will not only improve the
node's PageRank but also the PageRank vector.  If the PageRank vector is now
close enough (within \code{bound}), then the program will terminate.

\begin{itemize}

   \item Base case: since we have an \code{update} fact as an axiom, rule 5 will
      compute a new PageRank (Pagerank update lemma) for all nodes that is
      improved (from equation (1)). From these updates, a new \code{update} fact
      is generated that correspond to nodes that have inbound links from the
      source node and need to update their PageRank. These \code{update} facts
      may not be generated if the PageRank vector is close enough to its real
      value.

   \item The induction hypothesis tells us that there is at least one node that
      has an \code{update} fact. From PagePank update lemma, this generates
      \code{new-neighbor-pagerank} facts if the new value differs significantly from
      the older value. When this happens and using the ``Neighbor rank lemma'',
      the target node will update its \code{neighbor-pagerank} fact with the
      newest iteration and then execute a valid PageRank computation that brings
      the PageRank vector close to its solution.

\end{itemize}

\end{proof}
