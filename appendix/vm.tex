In this appendix, we present the instructions implemented by our virtual
machine. We consider that the machine has an arbitrary number of registers (or
variables) that can be assigned to values or facts and a database of facts
indexed by program predicates.

\paragraph{Values}\label{appendix:vm:values} Values can be assigned to registers or to fact arguments
which are stored in the database. The following values are allowed:

\begin{itemize}
      \item Boolean values;
      \item Integer numbers;
      \item Floating point numbers;
      \item Node pointers;
      \item Thread pointers;
      \item String pointers;
      \item List pointers;
      \item Array pointers;
\end{itemize}

\section{Iteration Instructions}

The iteration instructions perform searches on the database of facts. When
inferring a rule, each iteration instruction iterates over a predicate in the
body of the rule in order to derive all combinations of facts that match the
rule.

\paragraph{Match Specification} Every instruction points to a \code{match}
specification that
filters facts from the database. The specification maps argument indexes to value
templates. Value templates include the values described
in~\ref{appendix:vm:values} and the following variable templates:

\begin{itemize}
      \item Register: fact argument must equal to the value stored in the
         register;
      \item Fact argument: fact argument must be equal to the value stored in
         another fact argument;
      \item List template: describes the structure of a list, namely, the head
         and tail value template;
\end{itemize}

\paragraph{Iteration} Each iteration instruction contains the \code{body}, a list of instructions
that must be executed for each fact found in the database. The \code{body} also
returns a status indicating whether rule inference was successful or not.

\begin{itemize}
   \item \code{persistent iterate [pred : predicate, dest : register, m : match, body : instructions]}

   Iterate over persistent facts with the predicate \code{pred} that match
   \code{m}. For each iteration, we assign \code{dest} the next candidate fact
   and execute instructions \code{body}.

   \item \code{linear iterate [pred: predicate, dest : register, m : match, body : instructions]}

   Same as before but iterate over linear facts.

   \item \code{reuse linear iterate [pred: predicate, dest : register, m :
      match, body : instructions]}

   Iterates over linear facts but, if the rule has been successful, it does not
   remove the linear fact stored in \code{dest} since the same fact has been
   re-derived in the head of the rule.

\end{itemize}

\subsection{Return Instructions}

Return instructions terminate the current execution flow and force the machine
to backtrack to a previous state. In most cases, they are the return values used by iterate
instructions and force the iterate instruction to use the next candidate fact.

\begin{itemize}
   \item \code{return}

      Return without return value. Used at the end of rules to complete rule
      inference or in rules where a higher priority rule is available.

   \item \code{return derived}

      The rule was successfully derived.

   \item \code{return next}

      The matching has failed and the iterate instruction must use the next
      candidate fact.

\end{itemize}

\section{Conditional Instructions} Conditional instructions are used to
conditionally execute certain instructions.

\begin{itemize}
   \item \code{if [reg : register, body : instructions]}

   If the boolean value stored in \code{reg} is true then execute \code{body}.

   \item \code{if [reg : register, body1 : instructions, body2 : instructions]}

   If the boolean value stored in \code{reg} is true then execute \code{body1},
   otherwise execute \code{body2}.

   \item \code{jump [offset : integer]}

   Used as a goto at the end of \code{body1} in the instruction above. Jumps a
   certain number of instructions.

\end{itemize}

\section{Arithmetic and Boolean Instructions}

We briefly describe the most important arithmetic and boolean instructions.

\paragraph{Operations}

Available operations are as follows:

\begin{itemize}
   \item Addition \code{+};
   \item Subtraction \code{-};
   \item Multiplication \code{*};
   \item Division \code{/};
   \item Integer remainder \code{\%};
   \item Equality \code{=};
   \item Inequality \code{<>};
   \item Greater \code{>};
   \item Greater or equal \code{>=};
   \item Lesser \code{<};
   \item Lesser or equal \code{<=};
   \item Logical or \code{or};
   \item Logical and \code{and};
\end{itemize}

\paragraph{Instructions}

These operations are used as follows:

\begin{itemize}
   \item \code{test nil [reg : register, dest : register]}

   If the list stored in \code{reg} is null then set \code{dest} as \code{true}.

   \item \code{not [reg : register, dest : register]}

   Perform a boolean \code{not} operation.

   \item \code{int op [op1 : register, op2 : register, dest : register]}

   Perform an binary arithmetic operation on two integer numbers \code{op1} and
   \code{op2} and store result in \code{dest}. Operations include: \code{+},
   \code{-}, \code{*}, \code{/}, \code{\%}, \code{=}, \code{<>}, \code{<}, \code{<=},
   \code{>=}, and \code{>}.

   \item \code{float op [op1 : register, op2 : register, dest : register]}

   Perform an binary arithmetic operation on two floating point numbers \code{op1} and
   \code{op2} and store result in \code{dest}. Operations include: \code{+},
   \code{-}, \code{*}, \code{/}, \code{=}, \code{<>}, \code{<}, \code{<=},
   \code{>=}, and \code{>}.

   \item \code{node op [op1 : register, op2 : register, dest : register]}

   Perform an operation on two node addresses \code{op1} and
   \code{op2} and store result in \code{dest}. Operations include: \code{=} and
   \code{<>}.

   \item \code{thread op [op1 : register, op2 : register, dest : register]}

   Perform an operation on two thread addresses \code{op1} and
   \code{op2} and store result in \code{dest}. Operations include: \code{=} and
   \code{<>}.

   \item \code{[bool op [op1 : register, op2 : register, dest : register]}

   Perform an operation on two boolean values \code{op1} and
   \code{op2} and store result in \code{dest}. Operations include: \code{=} and
   \code{<>}, \code{and} and \code{or}.

\end{itemize}

\section{Data Instructions}

Data instructions include instructions which move values between registers and fact arguments and
instructions which call external functions written in C++ code.

\begin{itemize}
   \item \code{move-value-to-reg [val : value, reg : register]}

      Moves any value presented in~\ref{appendix:vm:values} to register \code{reg}.

   \item \code{move-arg-to-reg [fact : register, arg : integer, reg : register]}

      Moves the $\mathtt{arg}^{th}$ argument of fact stored in \code{fact} to
      register \code{reg}.

   \item \code{move-reg-to-arg [reg : register, fact : register, arg : integer]}

      Moves the value stored in \code{reg} to the $\mathtt{arg}^{th}$ argument of
      fact stored in \code{fact}.

   \item \code{call [id : integer, gc : boolean, dest : register, n : integer, arg1 :
      register, arg2 : register, ...]}

      Call external function with number \code{id} with \code{n} arguments
      \code{arg1} up to \code{argn} and store the return value in \code{dest}.
      If \code{gc} is \code{true} then garbage collect the return value at the
      end of the rule.

\end{itemize}

\section{List Instructions}

List instructions manipulate and access list addresses.

\begin{itemize}
   \item \code{cons [head : register, tail : register, gc : boolean, dest : register]}

      Create a new list and store its address in \code{dest}. The value of
      \code{tail} must be a list address or a null list. If \code{gc} is
      \code{true} then the list must be garbage collected after the rule
      executes.

   \item \code{head [ls : register, dest : register]}

   Retrieve the head value from the list address \code{ls} and store it in
   \code{dest}.

   \item \code{tail [ls : register, dest : register]}

   Retrieve the tail value from the list address \code{ls} and store it in
   \code{dest}.

\end{itemize}

\section{Derivation and Removal Instructions}

Derivation instructions create facts and add them to the corresponding node
databases. We also include instructions that remove facts from the database.

\begin{itemize}
   \item \code{alloc [pred : predicate, dest : register]}

      Allocate new fact for predicate \code{pred} and assign it to \code{dest}.

   \item \code{add linear [fact : register]}

      Add linear fact stored in \code{fact} to the current node's database.

   \item \code{add persistent [fact : register]}

      Add persistent fact stored in \code{fact} to the current node's database.

   \item \code{send [fact : register, dest : register]}

      Send fact stored in \code{fact} to the node address stored in \code{dest}.

   \item \code{run action [fact : register]}

   Execute action fact stored in \code{fact} and then deallocate the fact. This
   instruction is used for coordination instructions.

   \item \code{remove [fact : register]}

      Removes linear fact stored in \code{fact} from the database. The fact has
      been retrieved from one of the iterate instructions.
\end{itemize}
