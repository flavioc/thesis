In Chapter~\ref{chapter:implementation}, we saw that the set of nodes of an LM
program is represented as a graph data structure $G = (V, E)$ with nodes $V$ and
edges $E$ where $T$ threads perform work. Our implementation also allows threads
to steal nodes from other threads in order to improve load balancing.  However,
there are many scheduling details that are left undefined. How should a thread
schedule the computation of its sub-graph? Is node stealing beneficial to all
programs? What is the best sub-graph partitioning for a given LM program?  To
answer these questions, we introduce \emph{coordination facts}, a coordination
mechanism that is indistinguishable from regular computation and that allows the
programmer to specify custom scheduling and node partitioning policies. This is
an important first step in allowing the programmer to coordinate and improve
declarative programs.

\section{Motivation}\label{section:coord:rationale}

In order to justify the introduction of coordination, we present the Single
Source Shortest Path~(SSSP), a concise program that can take advantage of custom
scheduling policies to improve its performance. The SSSP program starts (lines
1-3) with the declaration of the predicates. The first predicate, \texttt{edge},
is a persistent predicate that describes the relationship between the nodes of
the graph, where the third argument represents the weight of the edge.  The
program computes the shortest distance from node \texttt{@1} to all other nodes
in the graph. Every node has a \texttt{shortest} fact that is improved with new
\texttt{relax} facts.  Lines 5-9 declare the axioms of the program:
\texttt{edge} facts describe the graph; \texttt{shortest(A, +00, [])} is the
initial shortest distance (infinity) for all nodes; and \texttt{relax(@1, 0,
   [@1])} starts the algorithm by setting the distance from \texttt{@1} to
\texttt{@1} to be 0.

\begin{figure}[ht]
\begin{Verbatim}[numbers=left]
type route edge(node, node, int).
type linear shortest(node, int, list int).
type linear relax(node, int, list int).

!edge(@1, @2, 3). !edge(@1, @3, 1).
!edge(@3, @2, 1). !edge(@3, @4, 5).
!edge(@2, @4, 1).
shortest(A, +00, []).
relax(@1, 0, [@1]).

shortest(A, D1, P1), D1 > D2, relax(A, D2, P2)
   -o shortest(A, D2, P2),
      {B, W | !edge(A, B, W) | relax(B, D2 + W, P2 ++ [B])}.

shortest(A, D1, P1), D1 <= D2, relax(A, D2, P2)
   -o shortest(A, D1, P1).
\end{Verbatim}
\caption{Single Source Shortest Path program code.}
\label{code:shortest_path_program}
\end{figure}

\begin{figure}
\begin{center}
   \begin{subfigure}[b]{0.4\textwidth}
      \includegraphics[width=\textwidth]{figures/sssp/shortest2}
      \caption{}
   \end{subfigure}
   \begin{subfigure}[b]{0.4\textwidth}
      \includegraphics[width=\textwidth]{figures/sssp/shortest3}
      \caption{}
   \end{subfigure}
   \begin{subfigure}[b]{0.4\textwidth}
      \includegraphics[width=\textwidth]{figures/sssp/shortest8}
      \caption{}
   \end{subfigure}
\end{center}
\caption{Graphical representation of the SSSP program. (a) represents the
   program after propagating initial distance at node \texttt{@1}, followed by
   (b) where the first rule is applied in node \texttt{@2}. (c)
   represents the state of the final program, where all the shortest paths
   have been computed.}
\label{fig:shortest_path_program}
\end{figure}

The first rule of the program (lines 11-14) reads as following: if the current
\texttt{shortest} path \texttt{P1} with distance \texttt{D1} is larger than a
new path \texttt{relax} with distance \texttt{D2}, then replace the current
shortest path with \texttt{D2}, delete the new \texttt{relax} path and propagate
new paths to the neighbors (lines 13-14).  The comprehension iterates over the
edges of node \texttt{A} and derives a new \texttt{relax} fact for each node
\texttt{B} with the distance \texttt{D2 + W}, where \texttt{W} is the weight of
the edge. For example, in Fig.~\ref{fig:shortest_path_program}~(a) we apply rule
1 in node \texttt{@1} where two new \texttt{relax} facts are derived at node
\texttt{@2} and \texttt{@3}. Fig.~\ref{fig:shortest_path_program}~(b) is the
result after applying the same rule but at node \texttt{2}.

The second rule of the program (lines 16-17) retracts a \texttt{relax} fact
that has a longer distance than the current shortest distance stored in
\texttt{shortest}.

There are many opportunities for custom scheduling in the SSSP program. For
instance, after applying rule 1 in Fig.~\ref{fig:shortest_path_program}~(a), it
is possible to either apply rules in either node \texttt{@2} or node
\texttt{@3}. This decision depends largely on implementation factors such as
node partitioning and number of threads in the system.  Still, it is easy to
prove that no matter the scheduling used, the final result, as presented in
Fig.~\ref{fig:shortest_path_program}~(c), is achieved.

The SSSP program is concise and declarative but its performance depends on the
order in which nodes are executed. If nodes with greater distances are
prioritized over other nodes, the program will generate more \texttt{relax}
facts since it will take longer to reach the shortest distances. From
Fig.~\ref{fig:shortest_path_program}, it is clear that the best scheduling is
the following: \texttt{@1}, \texttt{@3}, \texttt{@2} and then \texttt{@4}, where
only 4 \texttt{relax} facts are generated. If we had decided to process nodes in
order \texttt{@1}, \texttt{@2}, \texttt{@4}, \texttt{@3}, \texttt{@4},
\texttt{@2}, then 6 \texttt{relax} facts would have been generated.  The optimal
solution for SSSP is to schedule the node with the shortest distance, which is
essentially the Dijkstra shortest path algorithm~\cite{Dijkstra}. Note how it is
possible to change the nature of the algorithm by simply changing the order of
node computation, but still retain the declarative nature of the program.



\section{Types of Facts}

LM introduces the concept of coordination that allows the programmer to write
code that changes how the runtime system schedules and partitions node across
threads of execution. Beyond the distinction between linear and persistent
facts, LM further classifies facts into 3 categories: \emph{computation} facts,
\emph{structural} facts and \emph{coordination} facts.
Predicates are also classified accordingly.

Computation facts are regular facts used to represent the program state. In
Fig.~\ref{code:shortest_path_program}, \texttt{relax} and \texttt{shortest} are
all computation facts.

Structural facts describe information about the connections between the nodes in
the graph. In the example of Fig.~\ref{code:shortest_path_program},
\texttt{edge} facts are structural since the corresponding \texttt{edge} is used
for communication between nodes.  Note that structural facts can also be seen as
computation facts since they are heavily used in the program logic.

\emph{Coordination facts} allow the programmer to change how the run time
schedules nodes and how it partitions the nodes among threads of execution.
Coordination facts can be used in either in the LHS, RHS or both.  This allows
scheduling and partition decisions to be made based on the state of the program
and on the state of the underlying machine.  In this fashion, we keep the
language declarative because we reason logically about the state of execution,
without the need to introduce extra-logical operators into the language that
would introduce significant issues when proving properties about programs.

Coordination facts are further classified into two kinds of facts:
\emph{sensing} and \emph{action} facts. Sensing facts are used to sense
information about the underlying runtime system, including node placement and
node scheduling.  Action facts are used to apply a coordination operations on
the runtime system.

%%%%%%%%%%%%%%%%%%%%%%%%%%%%%%%%%%%%%%%%%%%%%%%%%%%%%%%%%%%%%%%%%%%%%%%%%%%

Sensing facts are facts about the current state of the runtime system, such as
the placement of nodes in the CPU and scheduling information. In the original
Meld, sensing facts were used to get information about the outside world, like
temperature, touch data, neighborhood status, etc.

Action facts are linear facts which are consumed when the corresponding action
is performed. In the original Meld, they were used to make the robots perform
actions in the outside world.  For LM we use them to change the order in which
nodes are evaluated in the runtime system and to make partitioning decisions
(assign nodes to threads). It is possible to give hints to the virtual machine
in order to prioritize the computation of some nodes.

With sensing facts and action facts, we can write \emph{meta-rules} that will
sense the state of the runtime system and then apply decisions in order to
improve execution speed or change partitioning information. In some situations,
this set of rules can be added to the program without any modifications to the
original rules.



\section{Scheduling Facts}\label{sec:coord:fifo}


In order to allow different scheduling strategies, we introduce the concept of
\emph{node priority} by assigning a priority value to every node in the program
and by introducing coordination facts that manipulate such priority values.  By
default, nodes have no priority and can be picked in any order. In our
implementation, we use a FIFO approach because older nodes tend to have a higher
number of unexamined facts, from which to derive subsequent new facts.

We have two kinds of priorities: a \emph{temporary priority} and a \emph{default
   priority}. A temporary priority momentarily changes the default priority $D$
of a node, so that once the node is done, the priority will default back to $D$.
Initially, all nodes have a default priority of $-\infty$.

The following list presents the action facts available to manipulate the
scheduling decisions of the system:

\begin{itemize}

   \item \code{set-priority(node A, float F)}: Sets the temporary priority of
      node \code{A} to \code{F}.  The programmer can decide if priorities
      are to be ordered in ascending or descending order, thus if node \code{A}
      has priority \code{G}, we only change it to \code{F} is \code{F > G}
      (ascending order) or \code{F < G} (descending order).

   \item \code{add-priority(node A, float F)}: Increases the temporary priority
      of node \code{A} by \code{F}.

   \item \code{remove-priority(node A)}: Removes the temporary priority from node
   \code{A}.

   \item \code{schedule-next(node A)}: Changes the temporary priority of node
   \code{A} to be $+\infty$.

   \item \code{set-default-priority(node A, float F)}: Sets the default
      priority of node \code{A} to \code{F}.

   \item \code{stop-program()}: Immediately stops the execution of the whole program.

\end{itemize}

LM also provides two sensing facts \code{priority(node A, float P)} and
\code{default-priority(node A, float P)}, which consult, respectively, the
temporary priority or default priority \code{P} of node \code{A}.  Sensing facts
can only be used in the body of rules and are exempt from the constraint that
forces every fact used in the body to have the same first argument. Also note
that when sensing facts are used to prove new facts, they are re-derived
automatically. All coordination facts are linear and thus consumed when used in
the body of a rule.  The system creates the necessary code to re-derive them
without programmer interaction. Likewise, \code{set-priority} and
\code{set-default-priority} update the value of \code{priority} facts by
retracting and re-asserting them but this is done implicitely by the runtime
system.

The priorities assigned to nodes are followed on a per-thread basis, therefore a
thread will always pick the highest priority node on its sub-graph but not the
highest priority node of the whole graph.
Figure~\ref{fig:coordination:priorities} shows an example of a graph being
processed by two threads, \code{T0} and \code{T1}. The order for \code{T0} will
be \code{@0}, \code{@1}, \code{@3}, \code{@2} and for thread \code{T1} it will
be \code{@4}, \code{@6}, \code{@5}. Priorities can also be seen as hints because
they do not provide a global ordering but only a per-thread ordering.

Note that priorities of nodes can be set from any node in the graph, even if those nodes
live on different threads. Of course, this implies communication between
threads.

\begin{figure}
\begin{center}
   \includegraphics[width=0.6\textwidth]{figures/coordination/priorities.pdf}
\end{center}
\caption{Priorities with sub-graph partitioning. Priorities are used on a
   per-thread basis therefore thread \code{T0} schedules \code{@0} to
   execute, while \code{T1} schedules node \code{@4}.}
\label{fig:coordination:priorities}
\end{figure}


\section{Partitioning Facts}
We provide several coordination facts for dealing with node partitioning among
the running threads. Since each node is placed in some thread, the partitioning
facts revolve around thread placement.  In terms of action facts, we have the
following:

\begin{itemize}
   \item \code{set-thread(node A, thread T)}: Places node \code{A} in thread \code{T} until the program terminates or a \code{set-moving(A)} fact is derived.
      
   \item \code{set-affinity(node A, node B)}: Places node \code{B} in the thread
      of node \code{A} until the program terminates or a \code{set-moving(B)} fact is derived.

   \item \code{set-moving(node A)}: Allows node \code{A} to move freely
   between threads.

\end{itemize}

As an example of \code{set-thread}, consider again the graph in
Fig.~\ref{fig:coordination:priorities}. If a coordination fact
\code{set-thread(@2, T1)} is derived, then node \code{@2} will become part
of the sub-graph of thread \code{T1}. The result is shown in
Fig.~\ref{fig:coordination:partitioning}.

\begin{figure}
\begin{center}
   \includegraphics[width=0.6\textwidth]{figures/coordination/partitioning.pdf}
\end{center}
\mycap{Moving node \code{@2} to thread \code{T1} using
   \code{set-thread(@2, T1)}.}
\label{fig:coordination:partitioning}
\end{figure}

Sensing facts retrieve information about node placement and are specified as
follows:

\begin{itemize}

   \item \code{thread-id(node A, thread T)}: Linear fact that maps node \code{A}
      to thread \code{T} which \code{A} belongs to. Fact \code{set-thread}
      implicitly updates fact \code{thread-id}.

   \item \code{is-static(node A)}: Fact available at node \code{A} if
      \code{A} is not allowed to move between threads.

   \item \code{is-moving(node A)}: Fact available at node \code{A} if
      \code{A} is allowed to move between threads.

   \item \code{just-moved(node A)}: Linear fact derived by the
      \code{set-thread} action if, at that moment, the node \code{A} is running
      on the target thread.

\end{itemize}



\section{Global Directives}\label{sec:coordination:global}

We also provide a few global coordination statements that cannot be specified
as sensing or action facts but are still important:


\begin{itemize}

   \item \texttt{priority @order ORDER} (\texttt{ORDER} can be either \code{asc}
      or \code{desc}): Defines if priorities are to be selected by the smallest
      or the greatest value, respectively.

   \item \texttt{priority @default P}: Informs the runtime system that all nodes
      must start with default priority \code{P}. Alternatively, the programmer can define a
      \texttt{set-default-priority(A, P)} fact.

   \item \texttt{priority @base P}: Informs the runtime system that the
      \emph{base priority} should be \code{P}. The base priority is, by default,
      0 when the priority ordering is \code{desc} and $+\infty$ when the ordering
      is \code{asc} and represents the smallest priority value possible.

   \item \texttt{priority @initial P}: Informs the runtime system that all nodes
   must start with temporary priority \code{P}. Alternatively, the programmer can define an
      \texttt{set-priority(A, P)} fact. The default is value is the one defined
      as the \emph{base priority}.

\end{itemize}


\section{Implementation}
In order to support priorities, the work queue is implemented as two pairs of
queues: a pair of doubly linked lists known as the \emph{standard queue} and a
pair of \emph{min/max} heaps known as the \emph{priority queue}.  The standard
queue contains nodes without priorities and supports push into tail, remove node
from the head, remove arbitrary node, and remove first half of nodes.  The
priority queue contains nodes with priorities and is implemented as a binary
heap array. It supports the following operations: push into the heap, remove the
\emph{min} node, remove an arbitrary node, remove half of the nodes (horizontal
split), and priority update.  Operations for removing half of the queue are
implemented in order to support node stealing, while operations to remove
arbitrary nodes or update priority allows threads to change the priority of
nodes. Table~\ref{fig:implementation:table_queue} show the complexity of queue
operations and compares the standard queue against the priority queue.

\begin{table}[h]
   \begin{tabular}{| c | l | l |}
      \hline
      \textbf{Operation} & \textbf{Standard queue} & \textbf{Priority Queue} \\
      \hline
      Push & $\mathcal{O}(1)$ & $\mathcal{O}(\log{N})$ \\ \hline
      Pop & $\mathcal{O}(1)$ & $\mathcal{O}(\log{N})$ \\ \hline
      Remove & $\mathcal{O}(1)$ & $\mathcal{O}(\log{N})$ \\ \hline
      Remove Half & $\mathcal{O}(N)$ & $\mathcal{O}(\log{N})$ \\ \hline
      Priority Update & - & $\mathcal{O}(\log{N})$ \\ \hline
   \end{tabular}
   \caption{Comparing the complexity of queue operations for both standard
      queue and priority queue. Except for the remove half operation, priority
      queue operations are more expensive.}
   \label{fig:implementation:table_queue}
\end{table}

\begin{figure*}[t]
\centering
\includegraphics[width=0.95\textwidth]{figures/implementation/work_queue.pdf}
\caption{Thread's work queue and its interaction with other threads: the dynamic queue contains nodes that can be
   stolen while the static queue contains nodes that cannot be stolen. Both
   queues are implemented with one standard queue and one priority queue.}
\label{fig:implementation:work_queue}
\end{figure*}

The \texttt{next} and \texttt{prev} pointers of the regular queue are part of
the node structure in order to save space. These pointers are also used as the
index in the priority queue and current priority, respectively. Both the regular
and priority queue are implemented as a pair of queues.  This first queue is the
\emph{static queue} which contains nodes that cannot be stolen.  The other queue
is the \emph{dynamic queue} which contains nodes which can be stolen by other
threads.

To minimize inter-thread communication, node priorities are implemented at the
thread level. Thus, when a thread picks the highest priority node from the
priority queue, it is only the highest priority with respect to the set of nodes
owned by the thread and not the highest priority node in the whole program.  

\subsection{Communication}

At this point, we can summarize the main thread synchronization hotspots in the
virtual machine. Threads synchronize with each other using mutual exclusion. We
use a spin-lock in each queue to protect queue operations.  Given threads $T_1$
and $T_2$, we enumerate the most important synchronization hotspots:

\begin{itemize}

   \item \textbf{New facts}: When a node executes in $T_1$ and derives facts to
      a node in $T_2$, $T_1$ first buffers the facts and then sends them to the
      target node. Here, it checks if the node is currently \textbf{idle} and
      then synchronizes with $T_2$ to add the node to the $T_2$'s queue.

   \item \textbf{Thread activation}: If $T_2$ is inactive when adding facts to a
      node in $T_2$, $T_1$ also synchronizes with $T_2$ to change $T_2$'s state
      to \emph{active}.

   \item \textbf{Node stealing}: $T_1$ synchronizes with $T_2$ when it attempts
      to steal nodes from $T_2$ by removing half of the nodes from one of
      $T_2$'s queues.

   \item \textbf{Coordination}: If $T_1$ needs to perform coordination
      operations to a node in $T_2$, it may need to synchronize with $T_2$
      during priority updates in order to move the node in $T_2$'s queues.

\end{itemize}

\subsection{Node State Machine}

In order to facilitate the implementation of coordination, we added a state
variable for each node. The state machine in
Fig.~\ref{fig:implementation:node_states} represents the valid state transitions
of a node:

\begin{itemize}
   \item \textbf{working}: the node is executing.
   \item \textbf{inactive}: the node is inactive, i.e., it has no new facts and is not in any
   queue for processing.
   \item \textbf{queue}: the node is active with new facts and is in some queue waiting
   to be processed.
   \item \textbf{stealing}: the node has just been stolen and is in the process of being
   moved to another thread.
   \item \textbf{coordinating}: the node is moving from one queue to another.
\end{itemize}

\begin{figure}[ht]
   \centering
   \includegraphics[width=0.55\textwidth]{figures/implementation/node_states.pdf}
   \caption{The node state machine as represented by the state variable. During
      the lifetime of a program, each node goes through different states as
      specified by the state machine.}
   \label{fig:implementation:node_states}
\end{figure}

Each node is protected by a main spin-lock that allows threads to change node
attributes: the fact buffer, owner thread, state variable and locality
information. There is also a database spin-lock that protects the internal
database of the node and is locked whenever the node is in the \textbf{working}
state.  

To avoid unnecessary copying, when a node sends facts to a node located in
another thread, the current thread first attempts to lock the database lock of
the target node in order to directly update its database and indexing
structures, otherwise it adds the facts to the list of incoming facts that are
later processed by the owner thread of the target node.


\section{Coordinating SSSP}
Now that we have presented the coordination facts for LM, we are now in a
position to use them in the SSSP program presented before.  The coordinated
version of the SSSP~(Fig.~\ref{code:shortest_path_program_coord}) uses the
coordination fact \texttt{set-priority} (line~\ref{line:coord:sssp_set}) and a global program directive
to order priorities in ascending order (line~\ref{line:coord:sssp_asc}).

When run with one thread, the algorithm behaves like Dijkstra's shortest path
algorithm~\cite{Dijkstra}. When using multiple threads, each thread will pick
the shortest distance from their subset of nodes.  While this does not yield the
optimal program with relation to 1 thread, it allows for parallel execution and
locally avoids unnecessary work. The result scales well and it is close to
Dijkstra's algorithm.

\begin{figure}[ht]
\begin{Verbatim}[numbers=left,commandchars=\\\{\},fontsize=\scriptsize]
type route edge(node, node, int).
type linear shortest(node, int, list int).
type linear relax(node, int, list int).

\underline{priority @order asc}.\label{line:coord:sssp_asc}

shortest(A, +00, []).
relax(@1, 0, [@1]).

shortest(A, D1, P1), D1 > D2, relax(A, D2, P2)
   -o shortest(A, D2, P2),
      \{B, W | !edge(A, B, W) |
         relax(B, D2 + W, P2 ++ [B]),
         \underline{set-priority(B, float(D2 + W))}\}.\label{line:coord:sssp_set}

shortest(A, D1, P1), D1 <= D2, relax(A, D2, P2)
   -o shortest(A, D1, P1).
\end{Verbatim}
   \caption{Shortest Path Program coordinated with \texttt{set-priority}.}
   \label{code:shortest_path_program_coord}
\end{figure}

%%% show new steps
Figure~\ref{fig:coordination:new_sssp} presents the 4 steps of computation for
the new SSSP program when executing with 1 thread. In every step a new shortest
path is computed at a different node, starting from the shorter paths up to
the longer paths. This is exactly the same behaviour as the Dijkstra's
algorithm.

\begin{figure}
\begin{center}
   \begin{subfigure}[b]{0.4\textwidth}
      \includegraphics[width=\textwidth]{figures/sssp/coord1}
      \caption{}
   \end{subfigure}
   \begin{subfigure}[b]{0.4\textwidth}
      \includegraphics[width=\textwidth]{figures/sssp/coord2}
      \caption{}
   \end{subfigure}
   \begin{subfigure}[b]{0.4\textwidth}
      \includegraphics[width=\textwidth]{figures/sssp/coord3}
      \caption{}
   \end{subfigure}
   \begin{subfigure}[b]{0.4\textwidth}
      \includegraphics[width=\textwidth]{figures/sssp/coord4}
      \caption{}
   \end{subfigure}
\end{center}
\caption{Graphical representation of the new SSSP program. (a) represents the
   program after propagating initial distance at node \texttt{@1}, followed by
   (b) where the first rule is applied in node \texttt{@3}. (c)
   represents the result after retracting all the \texttt{relax} facts at node
   \texttt{@2} and (d) is the final state of the program where all the shortest paths
   have been computed.}
\label{fig:coordination:new_sssp}
\end{figure}

\subsection{Proof Of Correctness}

The most interesting property of the SSSP program presented in
Fig.~\ref{code:shortest_path_program_coord} is that it remains provably correct,
although it applies rules using a smarter ordering and the code remains
declarative. We now show the complete proof of correctness.

\begin{invariant}[Distance]

\texttt{relax(A, D, P)} represents a valid distance \texttt{D} and a valid path
\texttt{P} from node \texttt{@1} to node \texttt{A}. If the shortest distance to
\texttt{@1} is $D'$, then $D >= D'$.

\texttt{shortest(A, D, P)} represents a valid distance \texttt{D} and a valid
path \texttt{P} from node \texttt{@1} to node \texttt{A}. If the shortest
distance to \texttt{@1} is $D'$, then $D >= D'$. The \texttt{shortest} fact may
also represent an invalid distance if \texttt{D = +00}, where \texttt{P = []}.

\end{invariant}

\begin{proof}
By mutual induction. All the axioms are valid and rule 1 and 2 validate the
invariant using the inductive hypothesis.
\end{proof}

\begin{lemma}[Relaxation]
Every new improved distance will be propagated to the neighbor nodes exactly once.
\end{lemma}
\begin{proof}
By rule 1, we know that when the distance is relaxed, we keep the new shorter
distance and propagate the distances. Every new distance that is longer will be
ignored by rule 2.
\end{proof}

\begin{theorem}[Correctness]

   Assume a graph $G = (V, E)$ where $W_{e \in E) >= 0}$ (positive weights).
   Consider that there is a set of nodes $S \in V$ where the shortest distances
   has been computed and a set $U \in V$ where it has not yet been reached.
   Sets are $S$ and $U$ are disjunct. At any given point, $\Sigma$ is the sum of
   all current shortest distances. For the distance \texttt{+00} we assign the
   value $\Sigma' + 1$, where $\Sigma'$ is the sum of the true shortest
   distances of nodes reachable from \texttt{@1}.  Every rule inference will
   either:

   \begin{itemize}
      \item Maintain the size of $S$ and reduce the total number of facts in
         the database.
      \item Increase the size of $S$, reduce $\Sigma$ and potentially increase the number of
         facts in the database.
      \item Maintain the size of $S$, decrease $\Sigma$
         and potentially increase the number of facts in the database.
   \end{itemize}

   Eventually, set $S = V$ and every \texttt{shortest(A, D, P)} will represent
   the shortest distance from \texttt{A} to \texttt{@1} and \texttt{P} is its
   corresponding path.
\end{theorem}

\begin{proof}
   By nested induction on $\Sigma$ and on the number of facts in the database.

   In the base case, we have \texttt{relax(@1, 0, [@1])} that will give us
   the shortest distance for node \texttt{@1}, therefore $S = \{@1\}$ and
   $\Sigma$ is reduced.

   In the inductive case, we have a set $S'$ where the shortest distance was
   reached and \texttt{relax} distances may have been propagated (Relaxation
   Lemma).

   Now consider the two rules of the program:

   \begin{itemize}

      \item Rule 1 will only apply at nodes in $U$. If the shortest
         \texttt{relax} is selected, then the node is added to $S$, otherwise it
         stays in $U$ but improves the shortest path, reduces $\Sigma$ and
         \texttt{relax} facts are generated (Relaxation Lemma).

      \item Rule 2 is applied in either nodes of $S$ or $U$. For both sets, the rule
         retracts the \texttt{relax} fact.

   \end{itemize}

   The case where rule 1 derives the true shortest distance happens when the
   node that minimizes $\argmin_t d(s \in S) + w(s, t)$ is selected, where
   $d(x)$ is the distance to the node \texttt{@1} and $w(a, b)$ the weight of
   the edge between $a$ and $b$. Using \texttt{set-priority} increases the
   probability of that node being selected, but it does not matter since
   the program always makes progress and the shortest distances will be
   eventually computed.
\end{proof}


\section{Applications}

To further understand how coordination facts work, we present other programs that
take advantage of them.

\subsection{Belief Propagation}

Randomized and approximation algorithms can obtain significant benefits from
coordination directives because although the final program results will not be
exact, they follow important statistical properties and can be computed faster.
An examples of such programs is PageRank~\cite{Lubachevsky:1986:CAA:4904.4801}
and Loopy Belief Propagation~\cite{Gonzalez+al:aistats09paraml}, which is the
focus of this section.

Loopy Belief Propagation (LBP) is an approximate inference algorithm used in
graphical models with cycles~\cite{Murphy99loopybelief}. In its essence, LBP is
a sum-product message passing algorithm where nodes exchange messages with their
immediate neighbors and apply some computations to the messages received.

LBP is an algorithm that maps very well to the graph-based model of LM. The
original algorithm computes the belief of all nodes for several iterations with
synchronization between iterations. However, it is possible to avoid the
synchronization step, if we take advantage of the fact that LBP will converge
even when using an asynchronous approach. So, instead of computing the belief
iteratively, we keep track of all messages sent/received (and overwrite them
when we receive a new one) and recompute the belief asynchronously.
Figure~\ref{fig:coordination:bp} shows the communication patterns for our
application and Fig.~\ref{code:coordination:bp} presents the LM code for the
implementation.

\begin{figure}[h]
   \begin{center}
      \includegraphics[width=0.3\textwidth]{figures/bp/bp.pdf}
   \end{center}
\caption{LBP: communication patterns}
\label{fig:coordination:bp}
\end{figure}

\begin{figure}[h!]
   \begin{Verbatim}[numbers=left, fontsize=\scriptsize]
neighbor-belief(A, B, Belief),
new-neighbor-belief(A, B, NewBelief)
   -o neighbor-belief(A, B, NewBelief).

check-residual(A, Residual, B),
Residual > bound
   -o update(B).
check-residual(A, _, _) -o 1.

// update belief to be sent to one neighbor
update-messages(A, NewBelief),
!edge(A, B),
neighbor-belief-old(A, B, OldIn),
sent-neighbor-belief(A, B, OldOut),
Cavity = normalize(divide(NewBelief, OldIn)),
Convolved = normalize(convolve(global-potential, Cavity)),
OutMessage = damp(Convolved, OldOut, damping)
   -o Residual = residual(OutMessage, OldOut),
      check-residual(A, Residual, B),
      update-messages(A, NewBelief),
      new-neighbor-belief(B, A, OutMessage),
      sent-neighbor-belief(A, B, OutMessage).

update-messages(A, NewBelief) -o 1.

// if we have two update functions, just run one of them
update(A), update(A) -o update(A).

// make a copy of neighbors beliefs in order to add them up
update(A),
!potential(A, Potential),
belief(A, MyBelief)
   -o [custom addfloats Potential => Belief | B, Belief |
         neighbor-belief(A, B, Belief) |
         neighbor-belief-old(A, B, Belief), neighbor-belief(A, B, Belief) |
         Normalized = normalizestruct(Belief),
         update-messages(A, Normalized), belief(A, Normalized)].
\end{Verbatim}
\caption{LM code for the Loopy Belief Propagation problem.}
\label{code:coordination:bp}
\end{figure}

Belief values are arrays of floats and are represented by \texttt{belief/2}
facts. The first rule (lines 1-3) updates a given neighbor belief whenever a new
belief value is received. This is the highest priority rule since we want to
update the neighbor beliefs before doing anything else. In order to store the
belief values of the neighbor nodes, we use \texttt{neighbor-belief/3} facts,
where the second argument is the neighbor address and the third argument is the
belief value.

The last two rules (lines 26-37) update the belief value of a node. An
\texttt{update/1} fact starts the process. The first rule (lines 27) simply
consumes redundant \texttt{update/1} facts and the second rule (lines 30-37)
performs the belief update by aggregating all the neighbor belief values. The
aggregate in lines 33-37 also derives copies of the neighbors beliefs that need
to be consumed in order to compute the belief value that is going to be sent to
the target neighbor. The aggregate uses a custom accumulator that takes two
arrays and adds the floating point numbers at each index of the array.  The rule
in lines 10-22 iterates through the neighbor belief values and sends new belief
values by performing the appropriate computations on the new belief value of the
current node and on the belief value sent previously.  Once the facts
\texttt{neighbor-belief-old} are fully consumed, the rule in line 24 is fired in
order to consume \texttt{update-messages}.

For each neighbor update, we also check in lines 5-8 if the change in belief
values is greater than \texttt{bound} (a program constant) and then force the
neighbor nodes to update their belief values by deriving \texttt{update(B)}.
This allows neighbor nodes to use updated neighbor values and recompute their
own belief values using better information. The computation of belief values
will then start to converge to their true belief values, independently of the
node scheduling used. However, if we prioritize nodes that receive new neighbor
belief values with a larger \texttt{Residual} then we will converge faster.
Figure~\ref{code:coordination:improved_bp} shows the fourth rule modified with
\texttt{add-priority} in order to increase to priority of neighbor nodes when
the source node has large changes in its belief value.

\begin{figure}[h!]
\begin{Verbatim}[numbers=left,commandchars=\\\{\},fontsize=\scriptsize]
// update belief to be sent to one neighbor
update-messages(A, NewBelief),
!edge(A, B),
neighbor-belief-old(A, B, OldIn),
sent-neighbor-belief(A, B, OldOut),
Cavity = normalize(divide(NewBelief, OldIn)),
Convolved = normalize(convolve(global-potential, Cavity)),
OutMessage = damp(Convolved, OldOut, damping)
   -o Residual = residual(OutMessage, OldOut),
      \underline{add-priority(B, Residual)},
      check-residual(A, Residual, B),
      update-messages(A, NewBelief),
      new-neighbor-belief(B, A, OutMessage),
      sent-neighbor-belief(A, B, OutMessage).
\end{Verbatim}
\caption{Updating the BP program to use priorities.}
\label{code:coordination:improved_bp}
\end{figure}

\subsection{N Queens}

The N-Queens puzzle is the problem of placing N chess queens on an NxN
chessboard so that no pair of two queens attack each
other~\cite{8queens}. The specific challenge of finding all the
distinct solutions to this problem is a good benchmark in designing
parallel algorithms.  Our solution is presented next in
Fig.~\ref{coordination:code:nqueens}.

In our implementation, nodes are the squares of the chessboard. Each
square can communicate with other 4 squares: the adjacent right and
the adjacent left on the same row, and the first non-diagonal square
to the right and to the left on the row below. To represent a
partial/valid board state, we use a list of integers, where each pair
of integers represents a coordinate in which a queen is placed. For
example $[1, 2, 0, 0]$ means that a queen is placed in square $(0, 0)$
and another in square $(1, 2)$. At any given time, many partial states
can be using the same squares. Each square can also have many states
at the same time.

\begin{figure}[ht]
\includegraphics[width=0.4\textwidth]{figures/coordination/nqueens.pdf}
\caption{Concurrent propagation of N-Queens states.}
\label{coordination:fig:nqueens}
\end{figure}

An empty state is instantiated in the top-left square (line 17) and is
then propagated to all squares in the same row (rule in lines
22-24). Once a square $S$ receives a new state $L$, it checks if $S$ can be
incorporated into $L$. For that, it checks if there is no queen on $S$'s
column (rules in lines 29-35), if there is no queen on $S$'s left
diagonal (rules in lines 37-43) and if there is no queen on $S$'s right
diagonal (rules in lines 45-51).

If there is any conflict, we do not derive anything and for that we use the
language expression \texttt{1} (lines 33, 41 and 49), which corresponds to an
empty rule head. If there are no conflicts, this means that it is possible to
add a queen to the current state (line 31). The fact \texttt{send-down/2} is
used to either complete the computation of a valid state (lines 53-54) or to
propagate the state to the row below (lines 55-57) as shown in
Fig.~\ref{coordination:fig:nqueens}.

Most popular parallel implementations of the N-Queens problem
distribute the search space of the problem by assigning incomplete
boards as tasks to threads. Our approach is unusual because the tasks
are the squares of the board.

\begin{figure}[h!]
\begin{Verbatim}[numbers=left,fontsize=\scriptsize]
type list int state.

type left(node, node).
type right(node, node).
type down-left(node, node).
type down-right(node, node).
type coord(node, int, int).
type linear propagate-left(node, state).
type linear propagate-right(node, state).
type linear test-y(node, int, state, state).
type linear test-diag-left(node, int, int, state, state).
type linear test-diag-right(node, int, int, state, state).
type linear send-down(node, state).
type linear new-state(node, state).
type linear final-state(node, state).

propagate-right(@0, []).

propagate-left(A, State)
  -o {L | !left(A, L), L <> A | propagate-left(L, State)},
     new-state(A, State).
propagate-right(A, State)
  -o {R | !right(A, R), R <> A | propagate-right(R, State)},
     new-state(A, State).

new-state(A, State), !coord(A, X, Y)
  -o test-y(A, Y, State, State).

// check if there is no queen on the same column
test-y(A, Y, [], State), !coord(A, OX, OY)
  -o test-diag-left(A, OX - 1, OY - 1, State, State).
test-y(A, Y, [X, Y1 | RestState], State), Y = Y1
  -o 1. // fail
test-y(A, Y, [X, Y1 | RestState], State), Y <> Y1
  -o test-y(A, Y, RestState, State).

// check if there is no queen on the left diagonal
test-diag-left(A, X, Y, _, State), X < 0 || Y < 0, !coord(A, OX, OY)
  -o test-diag-right(A, OX - 1, OY + 1, State, State).
test-diag-left(A, X, Y, [X1, Y1 | RestState], State), X = X1, Y = Y1
  -o 1. // fail
test-diag-left(A, X, Y, [X1, Y1 | RestState], State), X <> X1 || Y <> Y1
  -o test-diag-left(A, X - 1, Y - 1, RestState, State).

// check if there is no queen on the right diagonal
test-diag-right(A, X, Y, [], State), X < 0 || Y >= size, !coord(A, OX, OY)
  -o send-down(A, [OX, OY | State]). // add new queen
test-diag-right(A, X, Y, [X1, Y1 | RestState], State), X = X1, Y = Y1
  -o 1. // fail
test-diag-right(A, X, Y, [X1, Y1 | RestState], State), X <> X1 || Y <> Y1
  -o test-diag-right(A, X - 1, Y + 1, RestState, State).

send-down(A, State), !coord(A, size - 1, _)
  -o final-state(A, State).
send-down(A, State), !coord(A, CX, _), CX <> size - 1
  -o {B | !down-right(A, B), B <> A | propagate-right(B, State)},
     {B | !down-left(A, B), B <> A | propagate-left(B, State)}.
\end{Verbatim}
  \caption{N-Queens problem solved in LM.}
  \label{coordination:code:nqueens}
\end{figure}


\subsection{Heat Transfer}




\section{Related Work}\label{sec:coordination:related}

In this section, we explore programming languages and programming models that
allow coordination and/or scheduling of computation and/or processing units.

\subsection{Programming Languages}

Many programming languages follow what is called the coordination
paradigm~\cite{Papadopoulos98coordinationmodels}. This form of distributed
programming divides execution in two parts: \emph{computation}, where the actual
computation is performed, and \emph{coordination}, which deals with
communication and cooperation between processing units. This paradigm attempts
to clearly distinguish between these two parts by providing abstractions for
coordination in an attempt to provide architecture and system-independent forms
of communication.

We can identify two main types of coordination models:

\begin{description}
   \item[Data-Driven:]
   
   In a data-driven model, the state of the computation depends on both the data
   being received or transmitted by the processes and the current configuration
   of the coordinated processes. The coordinated process is not only responsible
   for reading and manipulating the data but is also responsible for
   coordinating itself and/or other processes. Each process must intermix the
   coordination directives provided by the coordination model with the
   computation code. While these directives have a very clear interface, it is
   in the programmer's responsibility to use them correctly.

   \item[Task-Driven:]
   
   In this model, the coordination code is more cleanly separated from the
   computation code. While in data-driven models, the content of the data
   exchanged by the processes will affect how the processes coordinate with each
   other, in a task-driven model, the process behavior depends only on the
   coordination patterns that are setup before hand. This means that the
   computation component is defined as a black box and there are clearly defined
   interfaces for input/output. These interfaces are usually defined as a
   full-fledged coordination language and not as simple directives present in
   the data-driven models.  \end{description}

Linda~\cite{linda} is probably the most famous coordination model. Linda
implements a data-driven coordination model and features a \emph{tuple space}
that can be manipulated using the following coordination directives:
\texttt{out(t)} writes a tuple \texttt{t} into the tuple space; \texttt{in(t)}
reads a tuple using the template \texttt{t}; \texttt{rd(t)} retrieves a copy of
the tuple \texttt{t} from the tuple space; and \texttt{eval(p)} puts a process
\texttt{p} in the tuple space and executes it in parallel.  Linda processes do
not need to know the identity of other processes because processes only
communicate through the tuple space.  Linda can be implemented on top of many
popular languages by simply creating a communication and storage mechanism for
the tuple space and then adding the directives as a language library.

Another early coordination language is Delirium~\cite{Delirium}. Unlike Linda,
which is embedded into another language, Delirium actually embeds operators
written in other languages inside the Delirium language. The advantages of
Delirium are improved abstraction and easier debugging because sequential
operators are isolated from the coordination language.

Linda and Delirium are limited in the sense that the programmer can only
coordinate the scheduling of processing units, while placement of data is left
to the implementation. LM differs from those languages because coordination acts
on data instead of processing units. The abstraction is then raised by
considering data and algorithmic aspects of the program instead of focusing on
how processing units are used. Furthermore, LM is both a coordination language
and a computation language and there is no distinction between the two
components.

The original Meld~\cite{ashley-rollman-iclp09} can also be seen as a kind of
data-driven coordination language. The important distinction is that in Meld
there's no explicit coordination directives. When Meld rules are activated they
may derive facts that need to be sent to a neighboring robot. In turn, this will
activate computation on the neighbor. Robot communication is implemented by
\emph{localizing} the program rules and then by creating \emph{communication
rules}.

The LM language also implements communication rules, however it goes further
because some facts, action facts, can change how the processing units schedule
nodes to be executed, namely, which node is to be computed next, which may in
turn change the program's final result. This result in a more complete
inter-play between coordination code and data.

\subsection{Programming Models}

The Galois~\cite{Pingali:2011:TPA:1993316.1993501} programming model implements
autonomous scheduling by default, where activities may be rolled back in case of
conflict. However, it is possible to employ a concrete scheduling strategy for
coordinating parallel execution in order to improve execution and avoid
conflicts.  First, there is \emph{compile-time coordination}, where the
scheduling ordered is computed during compilation and is pre-defined before the
program is executed. Secondly, there is \emph{runtime coordination}, where the
order of activities is computed during execution. The execution of the algorithm
proceeds in rounds: first, a set of non-conflicting activities is computed and
then executed by applying the operator; conflicting activities are postponed to
the next round. The third and last scheduling strategy is \emph{just-in-time
coordination} where the order of activities is defined by the underlying data
structure where the operator is applied (for instance, computing on a graph
may depend on its topology).

In the context of the Galois model, Nguyen et al.~\cite{nguyen11} expanded the
concept of runtime coordination with the introduction of a flexible approach to specify
scheduling policies for Galois programs. This approach was motivated by the fact
that some algorithms can be executed faster if computations use better
scheduling strategies. The scheduling language specifies 3 main scheduler types:
\texttt{FIFO} (First-In First-Out), \texttt{LIFO} (Last-In First-Out) and
\texttt{OrderedByMetric} (order activities by some metric). These schedulers can
be composed and synthesized without requiring users to write complex concurrent
code.

Elixir~\cite{Prountzos:2012:ESS:2384616.2384644} is a domain specific language
that builds on top of the Galois and allows easy specification of scheduling
strategies.  The main idea behind Elixir is that the user should be able to
specify how operator application is scheduled and the framework will compile
this high level specification to low level code using the provided scheduling
specification. One of the motivating examples is the Single Source Shortest Path
program that can be specified using multiple scheduling specifications,
generating different well-known shortest path algorithms such as the
Dijkstra or Bellman-Ford algorithm. Unlike the work of Nguyen et
all.~\cite{nguyen11}, Elixir does not allow graph mutations.

Halide~\cite{Ragan-Kelley:2013:HLC:2491956.2462176} is a language and compiler
for image processing pipelines with the goal of optimizing parallelism, locality
and re-computation. Halide decouples the algorithm definition from its execution
strategy, allowing the compiler to find which execution strategy may be the best
for optimizing for locality and parallelism. The language allows the programmer
to specify the scheduling strategy, allowing the programmer to decide the order
of computations, what intermediate results need to be stored, how to split the
data among processing units and how to use vectorization and the well-known
sliding window mechanism. However, the compiler is able to use stochastic search
to find good schedules for Halide pipelines. Notably, experimental results
indicate that automatic search sometimes leads to better execution than
hand-written code.

In contrast to the previous systems, LM stands alone in making coordination
(both scheduling and partitioning) a first-class programming construct and
semantically equivalent to computation. Furthermore, LM distinguishes itself by
supporting data-driven dynamic coordination, particularly for irregular data
structures. Elixir and Galois do not support coordination for data partitioning,
and, in Elixir, the coordination specification is separated from computation,
limiting the programmability of coordination. Compared to LM, Halide is
targeted for regular applications and therefore only supports compile time
coordination.

\section{Chapter Summary}

In this chapter, we presented the set of coordination facts, a new declarative
mechanism for coordinating declarative parallel programs. Coordination facts are
implemented as sensing and action facts and allow the programmer to write
derivation rules that change how the runtime system schedules computation and
partitions the data in the parallel system, thus improving the executing time.
In terms of programming language design, our coordination mechanisms are unique
in the sense that they are treated like regular computation, which allows for
complex run-time coordination policies that are declarative and can be made part
of the main program's logic.
