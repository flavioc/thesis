In the last chapter, we introduced new language facilities that can be used by
the programmer to coordinate execution. This new approach retains the implicit
parallelism of the standard LM language but it does not allow the programmer to
fully reason about the underlying parallel architecture. The only architecture
reasoning allowed relates to node partitioning and movement between threads. In
principle, it should be advantageous to reason about thread state, that is, to
perform rule inference about facts stored on each thread and allow threads to
communicate and coordinate between them depending on their current state. This
would introduce a kind of explicit parallelism into an implicitly parallel
language such as LM. However, this explicit parallelism remains declarative
since it should remain easy to prove properties about the thread's state.

\section{Rationale: Graph Searching}
Consider the problem of checking if a set of nodes $S$ in a graph $G$ is
reachable from an arbitrary node $N$. An obvious solution to this problem is to
start at $N$, gather all the neighbor nodes into a list and then recursively
visit all those reachable nodes, until $S$ is covered. This reduces to a problem
of performing a breadth or depth-first search on graph $G$. However, this
solution is sequential and does not have much concurrency.  An alternative
solution to the problem is to recursively propagate the search to all neighbors
and aggregate the results in the node where the search started.  The code for
this later solution is shown in Fig.~\ref{code:threads:reach_simple}.

\begin{figure}[h]
\begin{Verbatim}[numbers=left,fontsize=\codesize,commandchars=*\#\&]
type int id.*hfill// Type declaration
type list int reach-list.

type edge(node, node).*hfill// Predicate declaration
type value(node, int).
type linear search(node, id, reach-list).
type linear do-search(node, id, node, reach-list).
type linear lookup(node, id, reach-list, int Val).
type linear new-lookup(node, id, int Val).
type linear visited(node, id).

search(A, Id, ToReach)*label#line:threads:reach_lookup1&*hfill// Rule 1: initialize search
   -o do-search(A, Id, A, ToReach),
      lookup(A, Id, ToReach, []).*label#line:threads:reach_lookup2&

lookup(A, Id, ToReach, Found), new-lookup(A, Id, Val)*hfill// Rule 2: new reachable node found
   -o lookup(A, Id, remove(ToReach, Val), [Val | Found]).

do-search(A, Id, Node, ToReach),*label#line:threads:reach_visit1&*hfill// Rule 3: node has already seen this search
visited(A, Id) // Already visited for this search.
   -o visited(A, Id). *label#line:threads:reach_visit2&

do-search(A, Id, Node, ToReach),*hfill// Rule 4: node found and propagate search
!value(A, Val), Val in ToReach*hfill// New node was found.
   -o visited(A, Id),*label#line:threads:reach_visit_visited1&
      new-lookup(Node, Id, Val),
      {B | !edge(A, B) -o do-search(B, Id, Node, remove(ToReach, Val))}.*label#line:threads:reach_propagate&

do-search(A, Id, Node, ToReach),*hfill// Rule 5: node not found and propagate search
!value(A, Val), ~ Val in ToReach*hfill// Not the node we are looking for.
   -o {B | !edge(A, B) -o do-search(B, ID, Node, ToReach)},*label#line:threads:reach_propagate2&
      visited(A, Id).*label#line:threads:reach_visit_visited2&
\end{Verbatim}

\caption{LM code to perform reachability checking on a graph.}
\label{code:threads:reach_simple}
\end{figure}

Each distinct reachability search is represented by a number (\code{Id}) and a
\code{search} axiom. Associated to each search \code{Id} is a list of nodes to
reach.  The predicate \code{visited} marks nodes that have been already
participated in search, while predicate \code{do-search} is used to propagate a
specific search. The first rule
(lines~\ref{line:threads:reach_lookup1}-\ref{line:threads:reach_lookup2}) starts
a particular search by deriving a \code{do-search} and an \code{lookup} fact.
The \code{lookup} fact is used as an accumulator and is stored in the starting
node. The third rule
(lines~\ref{line:threads:reach_visit1}-\ref{line:threads:reach_visit2}) avoids
visiting the same node twice in the presence of a \code{visited} fact.  This
visited fact is derived in the next two rules
(lines~\ref{line:threads:reach_visit_visited1}
and~\ref{line:threads:reach_visit_visited2}).  If the node where the search is
being performed is in the set of nodes we want to reach (\code{ToReach}) then we
remove the node value from the list and propagate the search to the neighbor
nodes (line~\ref{line:threads:reach_propagate}).  Otherwise, the search is
propagated but no value is removed from \code{ToReach}.

As an example, consider Fig.~\ref{fig:threads:reach_example}, which shows 2
reachability checks on a graph with 10 nodes. For instance, the search with
\code{Id = 0} starts at node \code{@1} and checks if nodes \code{@1}, \code{@2},
and \code{@3} are reachable from \code{@1}. Since \code{@1} is the starting
node, \code{1} is immediately removed from the reachable list, including the
propagated \code{do-search} facts but also the \code{lookup} fact that is stored
at node \code{@1}. Once \code{do-search} reaches node \code{@3}, the value
\code{3} is removed from the list and a new \code{do-search} is propagated to
node \code{@1} (not shown in the figure) and \code{@2}. At the same time, node
\code{@2} receives the list \code{[2,3]}, removes \code{2} and propagates
\code{[3]} to node \code{@3} and \code{@1}. Node \code{@1} receives two
\code{new-lookup} facts, one from \code{@3} and another from \code{@2}, due to
successful searches and the \code{lookup} fact becomes \code{lookup(@1,0,[],
[1,2,3])}.

The attentive reader will notice that node \code{@1} already knows that all the
nodes have been reached and that nodes \code{@7} and \code{@4} will, needlessly,
continue to check if \code{[2,3]} are reachable. This is an issue that arises
because the programmer has valued concurrency by increasing redundancy and
reducing communication between nodes. It would be prohibitly expensive to share
reachability information between nodes. An alternative solution is to store the
results of the search on the thread performing the search and then coordinate
the results with other threads since the number of threads is usually smaller
than the number of nodes. Before showing how the reachability program is solved
using thread-based facts, we first present the changes required in the language.

\begin{figure}[ht]
\begin{center}
   \includegraphics[width=0.9\linewidth]{figures/threads/reach.pdf}
\end{center}

\caption{Performing reachability checks on a graph using nodes \code{@1}
(\code{Id = 0}) and \code{@6} (\code{Id = 1}).  Search with \code{Id = 0} wants
to reach nodes \code{@1}, \code{@2}, and \code{@3} from node \code{@1}. Since
\code{@1} is part of the target nodes, the fact \code{do-search} propagated to
neighbor nodes does not include \code{1}.}

\label{fig:threads:reach_example}
\end{figure}

\subsection{Language Changes}

In the previous chapter, we presented coordination facts such as
\code{thread-id} and \code{set-thread} that already bring some awareness about
the underlying parallel system. Furthermore, such facts also introduce the
\code{thread} type for predicate arguments, which refers to a thread in the
system that is related to a core in a multi core processor. We now introduce the
concept of \emph{thread facts}, which are logical facts stored at the thread
level, meaning that, each thread is now an entity with its own logical facts.
The type \code{thread} is also now the type of the first argument of
\emph{thread predicates}, indicating that the predicate is related and is to be
stored in a specific thread. We also view the available threads as forming a
separate graph from the data graph, a graph of the processing units which are
operating on the data graph.

The introduction of thread facts increases the expressiveness of the system in
the sense that it is now possible to write inference rules that reason about the
state of the threads. This creates optimization opportunies since we can now
write algorithms with global information stored in the thread, while keeping the
LM language fully declarative. Moreover, threads are now allowed to explicitly
communicate with each other, and in conjunction with coordination predicates,
enable the writing of complex scheduling policies.

We discriminate between two new types of inference rules. The first type is the
\emph{thread rule} and has the form \code{a(T), b(T) -o c(T)}, and can be read
as: if thread \code{T} has fact \code{a(T)} and \code{b(T)} then derive fact
\code{c(T)}. The second type is the \emph{mixed rule} and has the form
\code{a(T), d(N) -o e(N)} and can be read as: if thread \code{T} is executing
node \code{N} and has the fact \code{a(T)} and node \code{N} has the fact
\code{d(N)} then derive \code{e(N)} at node \code{N}. Thread rules reason
solely at the thread level, while the mixed rules allow reasoning about both
thread and node facts. Logically, the mixed rule uses an extra fact
\code{running(T, N)}, which indicates that thread \code{T} is currently
executing node \code{N}. The \code{running} fact is implicitly retracted and
asserted every time the thread selects a different node for execution. This
makes our implementation efficient since a thread does not need to look for
nodes that match mixed rules and it is then the scheduling of the program that
drives the matching of such rules.

\subsection{Graph Of Threads}

Figure~\ref{fig:coord:thread_facts} represents a schematic view of the two graph
data structures of a program with three threads: thread $T1$ is executing node
\code{@5}, $T2$ is executing node \code{@4}, and $T3$ is executing node
\code{@3}. Note that every thread has access to its own facts and to the node
facts.

\begin{figure}[ht]
   \includegraphics[width=0.6\linewidth]{figures/threads/threads.pdf}

   \caption{An example program being executed with three threads. Note that each
      threads has a \code{running} fact that stores the node currently being
   executed.}

   \label{fig:coord:thread_facts}
\end{figure}

We added several helpful predicates that allow the programmer to inspect the
graph of threads and reason about the state of computation as it relates to
threads:

\begin{itemize}
   \item \code{\bang thread-list(T, L)}: Fact instantiated in all threads where
      \code{L} is a list of all threads executing in the system.

   \item \code{\bang other-thread(T1, T2)}: Connects thread \code{T1} to all the
      other threads \code{T2} executing in the system. Note that in
      Fig.~\ref{fig:coord:thread_facts}, we use \code{\bang other-thread} fact
      to specify the graph of threads.

   \item \code{\bang leader-thread(T, TLeader)}: Fact instantiated in all
      threads where \code{TLeader} refers to a selected thread (usually the
      first thread in \code{L} of \code{\bang thread-list(T, L)}).

   \item \code{running(T, A)}: Used to retrieve the current node \code{A}
      running on thread \code{T}.
\end{itemize}

With the exception of \code{running}, every other fact is added at the beginning
of the program as a persistent fact.

\subsection{Reachability With Thread Facts}

We know update the graph reachability program presented in
Fig.~\ref{code:threads:reach_simple} to use thread facts in order to avoid
needless searches on the graph. The search process is still done concurrently as
before, but the search state is now stored in each thread, allowing the thread
to store partial results and coordinate with other threads. The code for this new
version is shown in Fig.~\ref{code:threads:reach_threads}.

Lines~\ref{line:threads:reacht_start1}-\ref{line:threads:reacht_start2} start
the search process by assigning a thread \code{Owner} to search \code{Id} using
the persistent fact \code{\bang thread-list} which contains the list of all
available threads in the system. Next, in
line~\ref{line:threads:reacht_threads}, a fact \code{thread-search} is created
for all threads using a comprehension. We use predicate \code{do-search} to
propagate the search through the graph and a predicate \code{visited} to mark
nodes already processed for a specific search.  The two rules in
lines~\ref{line:threads:reacht_check1}-\ref{line:threads:reacht_check2}
propagate the search process to the neighbor nodes and check if the current node
is part of the list of nodes we want to reach.

An interesting property of this version is that each owner thread responsible
for a search keeps track of the remaining nodes that need to be reached. In
line~\ref{line:threads:reacht_remove}, we derive \code{remove-thread-search} in
order to inform owner threads about new reachable nodes. Once an owner thread
detects that all nodes have been reached (lines
\ref{line:threads:reacht_reached1}-\ref{line:threads:reacht_reached2}), all the
other threads will know that and update their search state accordingly
(lines~\ref{line:threads:reacht_knows1}-\ref{line:threads:reacht_knows2}). When
every thread knows that all nodes were reached, they will consume
\code{do-search} facts (lines
\ref{line:threads:reacht_prune1}-\ref{line:threads:reacht_prune2}), effectively
pruning the search space.

\begin{figure}[h]
\begin{Verbatim}[numbers=left,fontsize=\codesize,commandchars=*\#\&]
search(A, Id, ToReach),*label#line:threads:reacht_start1&*hfill// Rule 1: initialize search
*textbf#!thread-list(T, L)&, Owner = nth(L, Id % @threads)*hfill// Allocate search to a thread.
   -o {T2 | *textbf#!other-thread(T, T2)& -o *textbf#thread-search(T2, Id, ToReach, Owner)&},*label#line:threads:reacht_threads&
      do-search(A, Id).*label#line:threads:reacht_start2&

*textbf#thread-search(T, Id, [], Owner)&, *label#line:threads:reacht_prune1&*hfill// Rule 2: search completed
do-search(A, Id)
   -o *textbf#thread-search(T, Id, [], Owner)&. *label#line:threads:reacht_prune2&

do-search(A, Id),*hfill// Rule 3: node already visited
visited(A, Id) // Already visited.
   -o visited(A, Id).

do-search(A, Id),*label#line:threads:reacht_check1&*label#line:bfs_join1&*hfill// Rule 4: node found
*textbf#thread-search(T, Id, ToReach, Owner)&,*label#line:threads:reacht_join2&
!value(A, Val), Val in ToReach
   -o *textbf#thread-search(T, Id, remove(ToReach, Val), Owner)&,
      *textbf#remove-thread-search(Owner, Id, Val)&,*hfill// Tell owner thread about it.*label#line:threads:reacht_remove&
      {B | !edge(A, B) -o do-search(B, Id)},
      visited(A, Id).

do-search(A, Id),*hfill// Rule 5: node not found but propagate search
*textbf#thread-search(T, Id, ToReach, Owner)&,
!value(A, Val), ~ Val in ToReach
   -o *textbf#thread-search(T, Id, ToReach, Owner)&,
      visited(A, Id),
      {B | !edge(A, B) -o do-search(B, Id)}.*label#line:threads:reacht_check2&

*textbf#remove-thread-search(T, Id, Val), thread-search(T, Id, ToReach, Owner)&*hfill// Rule 6: node found
   *textbf#-o thread-search(T, Id, remove(ToReach, Val), Owner),&
      *textbf#check-results(T, Id).&

*textbf#check-results(T, Id),&*label#line:threads:reacht_reached1&*hfill// Rule 7: search is completed
*textbf#thread-search(T, Id, [], Owner)&
   *textbf#-o thread-search(A, Id, [], Owner),&
      *textbf#{B | !other-thread(T, B) -o signal-thread(B, Id)}.&*label#line:threads:reacht_reached2&

 *textbf#check-results(T, Id),&*hfill// Rule 8: search not completed yet
 *textbf#thread-search(T, Id, ToReach, Owner), ToReach <> []&
   *textbf#-o thread-search(T, Id, ToReach, Owner).&

*textbf#signal-thread(T, Id),&*hfill// Rule 9: thread knows search is done*label#line:threads:reacht_knows1&
*textbf#thread-search(T, Id, ToReach, Owner)&
   *textbf#-o thread-search(T, Id, [], Owner).& *label#line:threads:reacht_knows2&
\end{Verbatim}
\caption{Coordinated version of the reachability checking program. Note
that \code{@threads} represent the number of threads in the system.}
\label{code:threads:reach_threads}
\end{figure}

An alternative implementation could force every thread to share its reached
nodes to all the other threads in the system. However, this would generate a lot
of traffic between threads, which would actually make the program perform worse.
Our final solution is a good trade off since it only forces threads to
coordinate when pruning can actually happen.

Figure~\ref{fig:threads:results_search} presents experimental results of the
graph reachability program using 4 different datasets. Except for Random, all
the datasets were already used in the MSSD program and were presented before.
The Random dataset is randomly generated graph with about 50000 nodes and a
million edges.  In the plots, we show the run time of the version without thread
facts (\textbf{Regular}) and the version using thread facts called
\textbf{Threads}. We also show the speedup of the \textbf{Threads} and
\textbf{Regular} versions when using the \textbf{Regular} version with 1 thread.

Our results indicate that using thread facts produces a significant reduction in
run time. This is especially true in the case of datasets with large number of
edges, since less facts are produced and propagated in the graph when the
threads know that the search has been completed. The results also show that, in
the case of the Facebook dataset in which the number of queries is the same as
the number of nodes, the use of thread facts does not produce great improvements
due to the costs of managing the reachability results on each thread's database.
These costs are related to the need to index and lookup \code{thread-search}
facts on the \code{Id} argument every time a node is inspected.

\begin{figure}[]
        \centering
        \begin{subfigure}[b]{\plotsize\textwidth}
           \includegraphics[width=\textwidth]{experiments/threads/cmp-search-facebook.png}
           \caption{Facebook has 2000 nodes and 20000 edges. The dataset makes
           2000 graph queries to 5\% of the graph's nodes.}
           \label{fig:threads:search_facebook}
        \end{subfigure}
        \spacing
        \begin{subfigure}[b]{\plotsize\textwidth}
           \includegraphics[width=\textwidth]{experiments/threads/cmp-search-twitter.png}
           \caption{Twitter has 81306 nodes and 1768149 edges. The dataset makes
           100 graph queries to 1\% of the graph's nodes.}
           \label{fig:threads:search_twitter}
        \end{subfigure} \\
        \begin{subfigure}[b]{\plotsize\textwidth}
           \includegraphics[width=\textwidth]{experiments/threads/cmp-search-random.png}
           \caption{Random is a graph with 50000 nodes and 1052674
              edges. The dataset makes 20 graph queries
              to 5\% of the graph's nodes.}
           \label{fig:threads:search_random}
        \end{subfigure}
        \spacing
        \begin{subfigure}[b]{\plotsize\textwidth}
           \includegraphics[width=\textwidth]{experiments/threads/cmp-search-pokec.png}
           \caption{Pokec has 1632803 nodes and 30622564 edges. The dataset
           makes 3 graph queries to 1\% of the graph's nodes.}
           \label{fig:threads:search_pokec}
        \end{subfigure} \\
        \caption{Measuring the performance of the graph reachability program
        when using thread facts.}
        \label{fig:threads:results_search}
\end{figure}

The graph reachability program shows how to introduce complex coordination
policies between threads by reasoning about the state of each thread. In
addition, the use of linear logic programming makes it easier to prove
properties of the program since computation is done by applying controlled
changes to the state.


\clearpage


\section{Implementation Changes}
To support thread-based facts, both the compilation and runtime system described in
Chapter~\ref{chapter:implementation} require some changes.

\subsection{Compiler}

The compiler needs to recognize rules that use thread facts. For thread rules,
the compiler checks if the rule's body is using facts from the same thread by
checking the first argument of each fact. For mixed rules, the rule's body may
use a thread \code{T} and a node \code{A} and all the node facts have to use
\code{A}, while all threads facts must use \code{T} as the first argument. If
the programmer was to retrieve either the thread or the node for the current
computation, she may use \code{running(T, A)}.

Once rules are type checked, the iteration code for thread-based facts needs to
be adapted. When a rule requires facts from the thread, it must use the data
structures from the thread. The runtime API used for inserting thread facts is
also different since they have to be added to the thread's database.

\subsection{Runtime}

In the runtime system, thread-based facts are implemented just like regular
facts. Each thread has its own database of facts and uses exactly the same data
structures for facts, as presented before for regular nodes.  The major
difference between a regular node and a thread node is that a thread node is
never put into the work queue of its thread. As shown in the updated work loop
presented in Fig.~\ref{alg:threads:work_loop}, the thread node executes
alongside the regular node when $TH.process\_node$ is called. It is also
important to note that, before a thread becomes idle, it may have potential
candidate thread rules that are now derivable because another thread has derived
thread facts in the current thread. In particular, it is entirely possible to have
programs that only deal with thread facts.

\begin{figure}
\begin{algorithm}[H]
   \KwData{Thread TH}
   \While{true}{
      $TH.work\_queue.lock()$\;
      $node \longleftarrow TH.work\_queue.pop\_node()$ \;
      $TH.work\_queue.unlock()$\;
      \uIf{$node$}{
        \underline{$TH.process\_node(node, TH.thread\_node)$}\;
      }
      \Else{
        \tcc{The thread's node may have candidate rules using incoming thread facts}
        \underline{$TH.process\_node(nil, TH.thread\_node)$}\;
        \tcc{Attempt to steal some nodes.}
         \If{$\bang TH.steal\_nodes()$}{
            $TH.become\_idle()$\;
            \While{$len(TH.work\_queue) == 0$}{
               \tcc{Try to terminate}
               \If{$TH.synchronize\_termination()$}{
                  \textbf{terminate}\;
               }
               \If{$TH.steal\_nodes()$}{
                  \tcc{Thread is still in the stealing state}
                  break\;
               }
            }
            \tcc{There's new nodes in the queue.}
            $TH.become\_active()$\;
         }
      }
 }
\end{algorithm}
\caption{Thread work loop updated to take into account thread-based facts.
New or modified code is underlined.}
\label{alg:threads:work_loop}
\end{figure}

Thread-based facts also introduce new synchronization logic in the runtime
system. For instance, when a rule derives a new thread fact on another thread,
it needs to synchronize with that thread (using the appropriate thread node
locks) to add the facts to the thread's database. When a thread is executing its
own node or a regular node, it also must lock the thread node's \emph{DB Lock}
in order to protect its data structures from being manipulated by other threads
concurrently.

Matching rules using thread facts requires special care since they may require
both facts from the regular node and from the thread's node. Before a node is
executed, the rule engine (Section~\ref{sec:implementation:rule_engine}) of the
regular node is updated to take into account the facts of the thread's node so
that mixed rules (rules that use both thread and regular facts) execute. In this
scheme, mixed rules may be unsuccessfully fired repeatedly until a node which
has matching facts gets to execute. Since the LHS of mixed rules use an implicit
\code{running(T, N)} fact, it is enough that a different node is running
to fire mixed rules as long as the running node has the required facts.
Without using an implicit \code{running} fact, the system would need to lookup
for a regular node that would successfully activate a given mixed rule. This
would be prohibitly expensive since some programs might have millions of regular
nodes.


\section{Applications}

We present more applications that demonstrate the usefulness and power of thread-based facts.

\subsection{Binary Search Trees: Caching Results}
In Section~\ref{sec:language:key_value} we have presented an algorithm for
replacing a key's value in a BST dictionary. To make the program more
interesting, we consider a sequence of $n$ lookup or replace operations for
different keys in the BST (which may or may not be repeated). A single lookup or
replace has worst-case time complexity $\mathcal{O}(h)$ where $h$ is the height
of the BST, therefore performing $n$ operations takes $\mathcal{O}(h \times n)$
time.

In order to reduce the execution time of the new program, we can cache the
search and replace operations so that repeated operations become faster. Instead
of traversing the entire height of the BST, we look in the cache and send the
operation immediately to the node where the key is located. Without thread
facts, we might have cached the results at the root node, however, this is not a
scalable approach as it would introduce a serious bottleneck.

Figure~\ref{code:threads:btree_lookup_cache} shows the updated BST code with a thread
cache. We just added two more predicates, \code{cache} and
\code{cache-size}, that are facts placed in the thread and represent cached
keys and the total size of the cache, respectively. We also added three new
rules that handle the following cases:

\begin{enumerate}
      \item A key is found and is also in the cache
         (lines~\ref{line:threads:kv_rule1_start}-\ref{line:threads:kv_rule2_end})

      \item A key is found but is not in the cache
         (lines~\ref{line:threads:kv_rule2_start}-\ref{line:threads:kv_rule2_end});

      \item A key is in the cache, therefore a \code{replace} fact is
         derived in the target node
         (lines~\ref{line:threads:kv_rule3_start}-\ref{line:threads:kv_rule3_end}).

\end{enumerate}

Note that it is quite easy to extend the cache mechanism to use an LRU type
approach in order to limit the size of the cache.

\begin{figure}[ht]
\begin{Verbatim}[numbers=left,fontsize=\codesize,commandchars=*\{\}]
type linear cache(thread, node, int).
type linear cache-size(thread, int).

// (1) Key exists and is also in the cache.*label{line:threads:kv_rule1_start}
replace(A, Key, RValue),
value(A, Key, Value),
*textbf{cache(T, A, Key)}
   -o value(A, Key, RValue).
      *textbf{cache(T, A, Key)}.*label{line:threads:kv_rule1_end}

// (2) Key exists and is not in the cache.*label{line:threads:kv_rule2_start}
replace(A, Key, RValue),
value(A, Key, Value),
*textbf{cache-size(T, Total)}
   -o value(A, Key, RValue),
      *textbf{cache-size(T, Total + 1)},
      *textbf{cache(T, A, Key)}.*label{line:threads:kv_rule2_end}

// (3) Cached by the thread.*label{line:threads:kv_rule3_start}
replace(A, RKey, RValue),
*textbf{cache(T, TargetNode, RKey)}
   -o replace(TargetNode, RKey, RValue),
      *textbf{cache(T, TargetNode, RKey)}.*label{line:threads:kv_rule3_end}

replace(A, RKey, RValue),
value(A, Key, Value),
!left(A, B),
RKey < Key
   -o value(A, Key, Value),
      replace(B, RKey, RValue). // go left

replace(A, RKey, RValue),
value(A, Key, Value),
!right(A, B),
RKey > Key
   -o value(A, Key, Value),
      replace(B, RKey, RValue). // go right
\end{Verbatim}
\caption{LM program for performing lookups in a BST with a thread cache.}
\label{code:threads:btree_lookup_cache}
\end{figure}




\subsection{PowerGrid Problem}

\subsection{Splash Belief Propagation}

Loopy Belief Propagation~\cite{Murphy99loopybelief} (LBP) is an approximate inference algorithm
used in graphical models with cycles. In its essence, LBP is a sum-product message passing algorithm
where nodes exchange messages with their immediate neighbors and apply some computations to the messages
received.

LBP is an algorithm that maps very well to the graph based model of LM. In its original form, we need to compute
the belief of all nodes for several iterations and also synchronize after each iteration.
However, it is still possible to apply
some optimizations in order to take advantage of the fact that LBP will converge even when using
an asynchronous approach. So, instead of computing the belief iteratively,
we first keep track of all messages sent/received (and overwrite them when we receive a new one)
and recompute the belief when we want, instead of synchronizing between nodes.
This way, we can prioritize the computation of beliefs when
a node's belief value changes significantly. When that happens, we set the priority of its
neighbors so that they can recompute their beliefs.

The asynchronous approach proves to be a nice improvement over the synchronous version. Still, it
is possible to do even better. Gonzalez et al~\cite{Gonzalez+al:aistats09paraml} developed an optimal
algorithm to compute this algorithm by first building a tree and then updating the beliefs of each node twice, first from the leaves to the root and then from the root to the leaves. The root of this tree
is the node with the highest priority (based on belief) while the other nodes in the tree
must have a non-zero priority. Note that the priorities are updated whenever a neighbor updates
their belief. These splash trees are built iteratively until we reach convergence.

The code in Fig.~\ref{code:sbp} presents the coordination code of the Belief Propagation problem.
Please note that we just appended the code in Fig.~\ref{code:sbp} to a working but
unoptimized version of the algorithm.
In the coordination code we have three sections:
\begin{description}
   \item[Tree building]: Each node has a \texttt{waiting} fact that is used to start the tree building process. When the highest priority is picked we create a token that will navigate through the tree. Note that in the rule located in lines 11-18 we check if the priority of the new node to add to the tree is positive and that both nodes are in the same worker. We want the tree to be kept in the same worker.
   \item[First phase]: In the third rule (lines 8-9), when we reach a certain number of nodes in the tree, we generate \texttt{first-phase} in order to update the beliefs of all nodes in tree starting from the leaves and ending at the root. As we update the nodes, we generate \texttt{update} to update the belief values (line 29).
   \item[Second phase]: In the second phase we go from the root to the leaves and update the beliefs a second time (line 39).
\end{description}

When we have several workers, every worker will generate their own trees by taking into account the highest priority node in their own queues.

\begin{figure}[h!]
\small\begin{Verbatim}[numbers=left,commandchars=*\{\}]
// TREE BUILDING
// continue tree
waiting(A), token(A, All, Next) -o token(A, All, Next).
// start tree
waiting(A), *underline{@priority(A, A, P)}, P > 0.0
   -o token(A, [A], [A]).
// end tree building
token(A, All, Next), length(All) > maxnodes
   -o first-phase(A, All, reverse(All)).
// expand tree
token(A, All, [A | Next])
   -o [collect => L | Side | !edge(A, L, Side),
         0 = count(All, L),
         0 = count(Next, L),
         *underline{priority(A, L, P)}, P > 0.0,
         *underline{thread-id(L, Id1)},
         *underline{thread-id(A, Id2)}, Id1 = Id2 |
         send-token(A, All, Next ++ L)].

send-token(A, All, [])
   -o first-phase(A, All, reverse(All)).
send-token(A, All, [B | Next])
   -o *underline{schedule-next(B)},
      token(B, All ++ [B], [B | Next]).

// FIRST PHASE
first-phase(A, [A], [A]) -o second-phase(A, [], A).
first-phase(A, [A, B | Next], [A])
   -o update(A), *underline{schedule-next(B)},
      second-phase(B, [B | Next], A).
first-phase(A, All, [A, B | Next])
   -o update(A), *underline{schedule-next(B)},
      first-phase(B, All, [B | Next]).

// SECOND PHASE
second-phase(A, [], _)
   -o *underline{set-priority(A, 0.0)}, waiting(A), update(A).
second-phase(A, [A], Back)
   -o update(A), waiting(Back),
      waiting(A), *underline{set-priority(A, 0.0)}.
second-phase(A, [A, B | Next], Back)
   -o update(A), waiting(Back), *underline{schedule-next(B)},
      second-phase(B, [B | Next], A).
\end{Verbatim}
  \caption{Coordination code for the Splash Belief Propagation program.}
  \label{code:sbp}
\end{figure}
\normalsize



\section{Modeling the Operational Semantics in Linear Logic}
\subsection{Scheduling}

\section{Experimental Evaluation}

\section{Chapter Summary}

In this chapter, we have extended the LM language with a declarative mechanism
for reasoning about the underlying parallel architecture. LM programs can be
first written in a data-driven fashion and then optimized by reasoning about the
state of threads, enabling the move from implicit parallelism to explicit
parallelism. We have presented four programs that showcase the potential of the
new mechanism and several experimental results that validate our approach.

