To further understand how coordination facts work, we present other programs that
take advantage of them.

\subsection{Belief Propagation}

Randomized and approximation algorithms can obtain significant benefits from
coordination directives because although the final program results will not be
exact, they follow important statistical properties and can be computed faster.
An examples of such programs is PageRank~\cite{Lubachevsky:1986:CAA:4904.4801}
and Loopy Belief Propagation~\cite{Gonzalez+al:aistats09paraml}, which is the
focus of this section.

Loopy Belief Propagation (LBP) is an approximate inference algorithm used in
graphical models with cycles~\cite{Murphy99loopybelief}. In its essence, LBP is
a sum-product message passing algorithm where nodes exchange messages with their
immediate neighbors and apply some computations to the messages received.

LBP is an algorithm that maps very well to the graph-based model of LM. The
original algorithm computes the belief of all nodes for several iterations with
synchronization between iterations. However, it is possible to avoid the
synchronization step, if we take advantage of the fact that LBP will converge
even when using an asynchronous approach. So, instead of computing the belief
iteratively, we keep track of all messages sent/received (and overwrite them
when we receive a new one) and recompute the belief asynchronously.
Figure~\ref{fig:coordination:bp} shows the communication patterns for our
application and Fig.~\ref{code:coordination:bp} presents the LM code for the
implementation.

\begin{figure}[h]
   \begin{center}
      \includegraphics[width=0.3\textwidth]{figures/bp/bp.pdf}
   \end{center}
\caption{LBP: communication patterns}
\label{fig:coordination:bp}
\end{figure}

\begin{figure}[h!]
   \begin{Verbatim}[numbers=left, fontsize=\scriptsize]
neighbor-belief(A, B, Belief),
new-neighbor-belief(A, B, NewBelief)
   -o neighbor-belief(A, B, NewBelief).

check-residual(A, Residual, B),
Residual > bound
   -o update(B).
check-residual(A, _, _) -o 1.

// update belief to be sent to one neighbor
update-messages(A, NewBelief),
!edge(A, B),
neighbor-belief-old(A, B, OldIn),
sent-neighbor-belief(A, B, OldOut),
Cavity = normalize(divide(NewBelief, OldIn)),
Convolved = normalize(convolve(global-potential, Cavity)),
OutMessage = damp(Convolved, OldOut, damping)
   -o Residual = residual(OutMessage, OldOut),
      check-residual(A, Residual, B),
      update-messages(A, NewBelief),
      new-neighbor-belief(B, A, OutMessage),
      sent-neighbor-belief(A, B, OutMessage).

update-messages(A, NewBelief) -o 1.

// if we have two update functions, just run one of them
update(A), update(A) -o update(A).

// make a copy of neighbors beliefs in order to add them up
update(A),
!potential(A, Potential),
belief(A, MyBelief)
   -o [custom addfloats Potential => Belief | B, Belief |
         neighbor-belief(A, B, Belief) |
         neighbor-belief-old(A, B, Belief), neighbor-belief(A, B, Belief) |
         Normalized = normalizestruct(Belief),
         update-messages(A, Normalized), belief(A, Normalized)].
\end{Verbatim}
\caption{LM code for the Loopy Belief Propagation problem.}
\label{code:coordination:bp}
\end{figure}

Belief values are arrays of floats and are represented by \texttt{belief/2}
facts. The first rule (lines 1-3) updates a given neighbor belief whenever a new
belief value is received. This is the highest priority rule since we want to
update the neighbor beliefs before doing anything else. In order to store the
belief values of the neighbor nodes, we use \texttt{neighbor-belief/3} facts,
where the second argument is the neighbor address and the third argument is the
belief value.

The last two rules (lines 26-37) update the belief value of a node. An
\texttt{update/1} fact starts the process. The first rule (lines 27) simply
consumes redundant \texttt{update/1} facts and the second rule (lines 30-37)
performs the belief update by aggregating all the neighbor belief values. The
aggregate in lines 33-37 also derives copies of the neighbors beliefs that need
to be consumed in order to compute the belief value that is going to be sent to
the target neighbor. The aggregate uses a custom accumulator that takes two
arrays and adds the floating point numbers at each index of the array.  The rule
in lines 10-22 iterates through the neighbor belief values and sends new belief
values by performing the appropriate computations on the new belief value of the
current node and on the belief value sent previously.  Once the facts
\texttt{neighbor-belief-old} are fully consumed, the rule in line 24 is fired in
order to consume \texttt{update-messages}.

For each neighbor update, we also check in lines 5-8 if the change in belief
values is greater than \texttt{bound} (a program constant) and then force the
neighbor nodes to update their belief values by deriving \texttt{update(B)}.
This allows neighbor nodes to use updated neighbor values and recompute their
own belief values using better information. The computation of belief values
will then start to converge to their true belief values, independently of the
node scheduling used. However, if we prioritize nodes that receive new neighbor
belief values with a larger \texttt{Residual} then we will converge faster.
Figure~\ref{code:coordination:improved_bp} shows the fourth rule modified with
\texttt{add-priority} in order to increase to priority of neighbor nodes when
the source node has large changes in its belief value.

\begin{figure}[h!]
\begin{Verbatim}[numbers=left,commandchars=\\\{\},fontsize=\scriptsize]
// update belief to be sent to one neighbor
update-messages(A, NewBelief),
!edge(A, B),
neighbor-belief-old(A, B, OldIn),
sent-neighbor-belief(A, B, OldOut),
Cavity = normalize(divide(NewBelief, OldIn)),
Convolved = normalize(convolve(global-potential, Cavity)),
OutMessage = damp(Convolved, OldOut, damping)
   -o Residual = residual(OutMessage, OldOut),
      \underline{add-priority(B, Residual)},
      check-residual(A, Residual, B),
      update-messages(A, NewBelief),
      new-neighbor-belief(B, A, OutMessage),
      sent-neighbor-belief(A, B, OutMessage).
\end{Verbatim}
\caption{Updating the BP program to use priorities.}
\label{code:coordination:improved_bp}
\end{figure}

\subsection{N Queens}

The N queens~\cite{8queens} puzzle is the problem of placing N chess queens on an NxN chessboard so
that no pair of two queens attack each other. The specific challenge of finding all the distinct
solutions to this problem is a good benchmark in designing parallel algorithms.

First, we consider each square of the chessboard as a node
that can communicate with the adjacent left, right and bottom squares, but not top square.
The states are represented as a list of integers, where each integer is the column number where
the queen was placed. For example $[2, 0]$ means that a queen is placed in square $(0, 0)$ and another in square $(1, 2)$.

An empty state is instantiated in the top-left node and is then propagated to all nodes in the same row.
Every node will then check if a queen can be placed on such square. If true, each node will send at most
two new states to the row below, one to the first non-diagonal column to the left and another to the column
in the right.
Recursively, when a node receives a new state, it will (i) send the state to the left
or to the right and (ii) try to place the queen in its square. With this method,
all states will be computed since we have facts for each valid state
at that point. When a suare cannot place a queen, that state is deleted.
When the program ends, all valid states will be placed in the bottom row.

We find our solution very elegant, since it can be easily executed in parallel and is an uncommon
approach to this problem.

\begin{comment}
\begin{figure}[h!]
\small\begin{Verbatim}[numbers=left]
type left(node, node).
type right(node, node).
type down(node, node).
type coord(node, int, int).
type linear propagate-left(node, list node, list int).
type linear propagate-right(node, list node, list int).
type linear receive-down(node, list node, list int).
type linear test-and-send-down(node, list node, list int).
type linear test-y(node, int, list int, list node, list int).
type linear test-diag-left(node, int, int, list int, list node, list int).
type linear test-diag-right(node, int, int, list int, list node, list int).
type linear send-down(node, list node, list int).
type linear final-state(node, list node, list int).

const size = 11.

receive-down(@0, [], []).

receive-down(A, Nodes, Coords)
   -o {R | !right(A, R), R <> A | propagate-right(R, Nodes, Coords)},
      {L | !left(A, L), L <> A | propagate-left(L, Nodes, Coords)},
      test-and-send-down(A, Nodes, Coords).

propagate-left(A, Nodes, Coords)
   -o {L | !left(A, L), L <> A | propagate-left(L, Nodes, Coords)},
      test-and-send-down(A, Nodes, Coords).

propagate-right(A, Nodes, Coords)
   -o {R | !right(A, R), R <> A | propagate-right(R, Nodes, Coords)},
      test-and-send-down(A, Nodes, Coords).

test-and-send-down(A, Nodes, Coords),
!coord(A, X, Y)
   -o test-y(A, Y, Coords, Nodes, Coords).

test-y(A, Y, [], Nodes, Coords), !coord(A, OX, OY) -o test-diag-left(A, OX - 1, OY - 1, Coords, Nodes, Coords).
test-y(A, Y, [X, Y1 | RestCoords], Nodes, Coords), Y = Y1 -o 1. // fail
test-y(A, Y, [X, Y1 | RestCoords], Nodes, Coords), Y <> Y1 -o test-y(A, Y, RestCoords, Nodes, Coords).

test-diag-left(A, X, Y, _, Nodes, Coords),
X < 0 || Y < 0,
!coord(A, OX, OY)
   -o test-diag-right(A, OX - 1, OY + 1, Coords, Nodes, Coords).

test-diag-left(A, X, Y, [X1, Y1 | RestCoords], Nodes, Coords),
X = X1, Y = Y1
   -o 1. // fail

test-diag-left(A, X, Y, [X1, Y1 | RestCoords], Nodes, Coords),
X <> X1 || Y <> Y1
   -o test-diag-left(A, X - 1, Y - 1, RestCoords, Nodes, Coords).

test-diag-right(A, X, Y, [], Nodes, Coords),
X < 0 || Y >= size,
!coord(A, OX, OY)
   -o send-down(A, [A | Nodes], [OX, OY | Coords]).

test-diag-right(A, X, Y, [X1, Y1 | RestCoords], Nodes, Coords),
X = X1, Y = Y1
   -o 1. // fail

test-diag-right(A, X, Y, [X1, Y1 | RestCoords], Nodes, Coords),
X <> X1 || Y <> Y1
   -o test-diag-right(A, X - 1, Y + 1, RestCoords, Nodes, Coords).

send-down(A, Nodes, Coords),
!down(A, A)
   -o final-state(A, Nodes, Coords).
   
send-down(A, Nodes, Coords),
!down(A, B),
A <> B
   -o receive-down(B, Nodes, Coords).
\end{Verbatim}
  \caption{Visit program.}
  \label{code:visit}
\end{figure}
\normalsize
\end{comment}


\subsection{Heat Transfer}


