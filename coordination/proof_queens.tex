We prove that the N-Queens program finds all the distinct solutions for the
puzzle. In the following lemmas, we assume that $(X_0, Y_0)$ is the coordinate
of square/node $A$.

\begin{lemma}[test-y lemma]

From a fact \code{test-y}$(A, Y, State, OrigState)$, there are two possible
scenarios:

\begin{itemize}
   \item The \code{test-y} fact and everything derived from it is retracted and
      there is a coordinate $(X', Y) \in State$.
   \item The \code{test-y} fact is retracted and is used to derive a
      \code{test-diag-left}$(A, X_0 - 1, Y_0 - 1,
      OrigState, OrigState)$ fact since 
      there is no such coordinate $(X', Y) \in State$.
\end{itemize}

\end{lemma}
\begin{proof}
Induction on the size of \code{State}.

First rule: immediately the second scenario.

Second rule: immediately the first scenario.

Third rule: by induction.
\end{proof}

\begin{lemma}[test-diag-left lemma]
   From a fact \code{test-diag-left}$(A, X, Y, State, OrigState)$, there are
  two possible scenarios:
   
  \begin{itemize}
     \item The \code{test-diag-left} fact and everything derived from it is
        retracted and there is a coordinate $(x', y') \in State$, where $x' = X
        - a$ and $y' = Y - a$, where $a$ is non-negative.
     \item The \code{test-diag-left} is retracted and is used to derive a
        \code{test-diag-right}$(A, X_0 - 1, Y_0 + 1, OrigState, OrigState)$
        fact since there is no $(x', y')$ coordinate as specified above.
  \end{itemize}
\end{lemma}
\begin{proof}
Induction on the size of \code{State}.

First rule: immediately the second scenario.

Second rule: immediately the first scenario.

Third rule: by induction.
\end{proof}

\begin{lemma}[test-diag-right lemma]
From a fact \code{test-diag-right}$(A, X, Y, State, OrigState)$, there are
two possible scenarios:

\begin{itemize}
   \item The \code{test-diag-right} fact and everything derived from it is
      retracted and there is a coordinate $(x', y') \in State$, where $x' = X -
      a$ and $y' = Y + a$, where $a$ is non-negative.
   \item The \code{test-diag-right} fact is retracted and the fact
      \code{send-down}$(A, [(X_0, Y_0) | OrigState])$ is derived.
\end{itemize}
\end{lemma}

\begin{proof}
Induction on the size of \code{State}.

First rule: immediately the second scenario.

Second rule: immediately the first scenario.

Third rule: by induction.
\end{proof}

\begin{theorem}[State validation]
From a fact \code{test-y}$(A, Y_0, State, State)$, there are two possible
scenarios:
   
   \begin{itemize}
      \item The \code{test-y} fact and anything derived from it is retracted.
      \item The \code{test-y} fact is retracted and a \code{send-down}$(A,
         [(X_0, Y_0) | State])$ fact is derived, where $[(X_0, Y_0) | State]$ is
         a valid board state.
   \end{itemize}
\end{theorem}
\begin{proof}
Use the previous three lemmas.
\end{proof}

\begin{lemma}[Propagate left lemma]
If a \code{propagate-left}$(A, State)$ fact exists then every cell to the left,
including \code{A} will derive \code{new-state(A, State)}. The original
\code{propagate-left} fact and any other fact derived from it (except
\code{new-state}) are retracted.
\end{lemma}
\begin{proof}

By induction on the number of cells to the left of $A$ and using the only rule
that uses \code{propagate-left}.

\end{proof}

\begin{lemma}[Propagate right lemma]
If a \code{propagate-right}$(A, State)$ fact exists then every cell to the left,
including \code{A} will derive \code{new-state(A, State)}. The original
\code{propagate-right} fact and any other fact derived from it (except
\code{new-state}) are retracted.
\end{lemma}
\begin{proof}
By induction on the number of cells to the left of $A$ and using the only rule
that uses \code{propagate-right}.
\end{proof}

\begin{theorem}[States theorem]
For a given row, we compute several \code{send-down}$(A, State)$ facts that
represent valid states that include that row and the rows above.
\end{theorem}
\begin{proof}
By induction on the number of rows.

For row 0, we use the initial fact \code{propagate-right}$(\mathtt{@0}, [])$,
that will be propagated to all nodes in row 0 (propagate right lemma). By using
the state validation theorem, we know that every node will derive
\code{send-down(A, [(X, Y)])}, which are all the valid states.

By induction, we know that row $X'$ has derived every \code{send-down} fact
possible. Such facts will be sent downwards to row $X = X' + 1$ using the last
rule in the program, deriving \code{propagate-right} or \code{propagate-left}
that, in turn, will derive a \code{new-state} fact at each right or left square.
Nothing is derived at the square below or the diagonal cells since they are not
valid.  From the \code{new-state} fact, a \code{test-y} fact is derived, which will be
checked using the state validation theorem, filtering all new valid
states and then finally deriving a new \code{send-down} fact.

\end{proof}

