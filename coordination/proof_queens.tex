We now prove that the N-Queens program finds all the distinct solutions for
the puzzle.

\begin{lemma}[test-y lemma]

If \code{test-y(A, Y, State, OriginalState)} then either $\exists_{x'}. {(x',
y) \in \mathtt{State}}$ and \code{test-y} is consumed or
\code{test-diag-left(A, OX - 1, OY - 1, OriginalState, OriginalState)}, where
\code{OX} and \code{OY} are the coordinates of the square.

\end{lemma}
\begin{proof}
Induction on the size of \code{State}.

First rule: immediately the second conclusion.

Second rule: immediately the third conclusion.

Third rule: by induction.
\end{proof}

\begin{lemma}[test-diag-left lemma]
If \code{test-diag-left(A, X, Y, State, OriginalState)} then either $\exists_{x', y'}. {(x', y') \in \mathtt{State}}$, where $x = x' - a$ and $y' = y - a$, where $a$ is positive or $0$ and \code{test-diag-left} is consumed or \code{test-diag-right(A, OX - 1, OY + 1, OriginalState, OriginalState)}, where \code{OX} and \code{OY} are the coordinates of the square.
\end{lemma}
\begin{proof}
Induction on the size of \code{State}.

First rule: immediately the second conclusion.

Second rule: immediately the first conclusion.

Third rule: by induction.
\end{proof}

\begin{lemma}[test-diag-right lemma]
If \code{test-diag-right(A, X, Y, State, OriginalState)} then either $\exists_{x', y'}. {(x', y') \in \mathtt{State}}$, where $x = x' - a$ and $y' = y + a$, where $a$ is positive or $0$ and \code{test-diag-right} is consumed or \code{send-down(A, [(OX, OY) | OriginalState])}, where \code{OX} and \code{OY} are the coordinates of the square.
\end{lemma}
\begin{proof}
Induction on the size of \code{State}.

First rule: immediately the second conclusion.

Second rule: immediately the first conclusion.

Third rule: by induction.
\end{proof}

\begin{theorem}[State validation]
If \code{test-y(A, OY, State, State)} then either everything is consumed or \code{send-down(A, [(OX, OY) | State])} is derived, where \code{OX} and \code{OY} are the coordinates of the square and are a valid addition to the \code{State}.
\end{theorem}
\begin{proof}
Use the previous three lemmas.
\end{proof}

\begin{lemma}[Propagate left lemma]
If \code{propagate-left(A, State)} then every cell to the left, including \code{A} will derive \code{new-state(A, State)}.
\end{lemma}
\begin{proof}
By induction on the number of cells to the left of \code{A}. The only rule that uses \code{propagate-left/2} will prove the lemma.
\end{proof}

\begin{lemma}[Propagate right lemma]
If \code{propagate-right(A, State)} then every cell to the right, including \code{A} will derive \code{new-state(A, State)}.
\end{lemma}
\begin{proof}
By induction on the number of cells to the right of \code{A}. The only rule that uses \code{propagate-right/2} will prove the lemma.
\end{proof}

\begin{theorem}[States theorem]
For a given row, we will compute several \code{send-down(A, State)} facts that represent valid configurations that include that row and the rows above.
\end{theorem}
\begin{proof}
By induction on the number of rows.

For row 0, we use the axiom \code{propagate-right(@0, [])}, that will be propagated to all nodes in row 0. By using the state validation theorem, we know that every node will derive \code{send-down(A, [(X, Y)])}, all valid configurations.


By induction, we know that row $X'$ has derived every \code{send-down/2} possible. Such facts will be sent downwards to row $X = X' + 1$ using the last rule in the program, deriving \code{propagate-right} or \code{propagate-left} that will derive \code{new-state} at each right or left cell. We do not derive anything at the cell below or the ones to the sides since they would not be valid. Using the \code{new-state} fact, we get a \code{test-y} fact that will be checked using the state validation theorem, filtering all new valid configurations and deriving \code{send-down/2}.
\end{proof}

