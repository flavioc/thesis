Randomized and approximation algorithms can obtain significant benefits from
coordination directives because although the final program results will not be
exact, they follow important statistical properties and can be computed faster.
An examples of such programs is PageRank~\cite{Lubachevsky:1986:CAA:4904.4801}
and Loopy Belief Propagation~\cite{Gonzalez+al:aistats09paraml}, which is the
focus of this section.

Loopy Belief Propagation (LBP) is an approximate inference algorithm used in
graphical models with cycles~\cite{Murphy99loopybelief}. In its essence, LBP is
a sum-product message passing algorithm where nodes exchange messages with their
immediate neighbors and apply some computations to the messages received.

LBP is an algorithm that maps very well to the graph-based model of LM. The
original algorithm computes the belief of all nodes for several iterations with
synchronization between iterations. However, it is possible to avoid the
synchronization step, if we take advantage of the fact that LBP will converge
even when using an asynchronous approach. So, instead of computing the belief
iteratively, we keep track of all messages sent/received (and overwrite them
when we receive a new one) and recompute the belief asynchronously.
Figure~\ref{fig:coordination:bp} shows the communication patterns for our
application and Fig.~\ref{code:coordination:bp} presents the LM code for the
implementation.

\begin{figure}[h]
   \begin{center}
      \includegraphics[width=0.3\textwidth]{figures/bp/bp.pdf}
   \end{center}
\caption{LBP: communication patterns}
\label{fig:coordination:bp}
\end{figure}

\begin{figure}[h!]
\begin{Verbatim}[numbers=left, fontsize=\codesize, commandchars=\*\{\}]
neighbor-belief(A, B, Belief),*label{line:coord:bp_first1}
new-neighbor-belief(A, B, NewBelief)
   -o neighbor-belief(A, B, NewBelief).*label{line:coord:bp_first2}

check-residual(A, Residual, B),*label{line:coord:bp_check1}
Residual > bound
   -o update(B).
check-residual(A, _, _) -o 1.*label{line:coord:bp_check2}

// update belief to be sent to one neighbor*label{line:coord:bp_iterate1}
update-messages(A, NewBelief),
!edge(A, B),
neighbor-belief-old(A, B, OldIn),
sent-neighbor-belief(A, B, OldOut),
Cavity = normalize(divide(NewBelief, OldIn)),
Convolved = normalize(convolve(global-potential, Cavity)),
OutMessage = damp(Convolved, OldOut, damping)
   -o Residual = residual(OutMessage, OldOut),
      check-residual(A, Residual, B),
      update-messages(A, NewBelief),
      new-neighbor-belief(B, A, OutMessage),
      sent-neighbor-belief(A, B, OutMessage).*label{line:coord:bp_iterate2}

update-messages(A, NewBelief) -o 1.*label{line:coord:bp_iterate_final}

// if we have two update functions, just run one of them*label{line:coord:bp_last1}
update(A), update(A) -o update(A).*label{line:coord:bp_update}

// make a copy of neighbors beliefs in order to add them up*label{line:coord:bp_update1}
update(A),
!potential(A, Potential),
belief(A, MyBelief)
   -o [custom addfloats Potential => Belief | B, Belief |*label{line:coord:bp_agg1}
         neighbor-belief(A, B, Belief) |
         neighbor-belief-old(A, B, Belief), neighbor-belief(A, B, Belief) |
         Normalized = normalizestruct(Belief),
         update-messages(A, Normalized), belief(A, Normalized)].*label{line:coord:bp_last2}*label{line:coord:bp_update2}*label{line:coord:bp_agg2}
\end{Verbatim}
\caption{LM code for the Loopy Belief Propagation problem.}
\label{code:coordination:bp}
\end{figure}

Belief values are arrays of floats and are represented by \code{belief/2} facts.
The first rule (lines~\ref{line:coord:bp_first1}-\ref{line:coord:bp_first2})
updates a given neighbor belief whenever a new belief value is received. This is
the highest priority rule since we want to update the neighbor beliefs before
doing anything else. In order to store the belief values of the neighbor nodes,
we use \code{neighbor-belief/3} facts, where the second argument is the neighbor
address and the third argument is the belief value.

The last two rules (lines~\ref{line:coord:bp_last1}-\ref{line:coord:bp_last2})
update the belief value of a node. An \code{update/1} fact starts the process.
The first rule (lines~\ref{line:coord:bp_update}) simply consumes redundant
\code{update/1} facts and the second rule
(lines~\ref{line:coord:bp_update1}-\ref{line:coord:bp_update2}) performs the
belief update by aggregating all the neighbor belief values. The aggregate in
lines~\ref{line:coord:bp_agg1}-\ref{line:coord:bp_agg2} also derives copies of
the neighbors beliefs that need to be consumed in order to compute the belief
value that is going to be sent to the target neighbor. The aggregate uses a
custom accumulator that takes two arrays and adds the floating point numbers at
each index of the array. The rule in
lines~\ref{line:coord:bp_iterate1}-\ref{line:coord:bp_iterate2} iterates through
the neighbor belief values and sends new belief values by performing the
appropriate computations on the new belief value of the current node and on the
belief value sent previously.  Once the facts \code{neighbor-belief-old} are
fully consumed, the rule in line~\ref{line:coord:bp_iterate_final} is fired in
order to consume \code{update-messages}.

For each neighbor update, we also check in
lines~\ref{line:coord:bp_check1}-\ref{line:coord:bp_check2} if the change in
belief values is greater than \code{bound} (a program constant) and then force
the neighbor nodes to update their belief values by deriving \code{update(B)}.
This allows neighbor nodes to use updated neighbor values and recompute their
own belief values using better information. The computation of belief values
will then start to converge to their true belief values, independently of the
node scheduling used. However, if we prioritize nodes that receive new neighbor
belief values with a larger \code{Residual} then we will converge faster.
Figure~\ref{code:coordination:improved_bp} shows the fourth rule modified with
\code{add-priority} in order to increase to priority of neighbor nodes when the
source node has large changes in its belief value.

\begin{figure}[h!]
\begin{Verbatim}[numbers=left,commandchars=\\\{\},fontsize=\codesize]
// update belief to be sent to one neighbor
update-messages(A, NewBelief),
!edge(A, B),
neighbor-belief-old(A, B, OldIn),
sent-neighbor-belief(A, B, OldOut),
Cavity = normalize(divide(NewBelief, OldIn)),
Convolved = normalize(convolve(global-potential, Cavity)),
OutMessage = damp(Convolved, OldOut, damping)
   -o Residual = residual(OutMessage, OldOut),
      \underline{add-priority(B, Residual)},
      check-residual(A, Residual, B),
      update-messages(A, NewBelief),
      new-neighbor-belief(B, A, OutMessage),
      sent-neighbor-belief(A, B, OutMessage).
\end{Verbatim}
\caption{Updating the BP program to use priorities.}
\label{code:coordination:improved_bp}
\end{figure}

