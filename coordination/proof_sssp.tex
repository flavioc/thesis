The most interesting property of the SSSP program presented in
Fig.~\ref{code:shortest_path_program_coord} is that it remains provably correct,
although it applies rules using a different ordering. We now show the complete
proof of correctness.

\begin{invariant}[Distance]

   $\mathtt{relax}(A, D, P)$ represents a valid distance $D$ and a valid path
   $P$ from node $\mathtt{@1}$ to node $A$. If the shortest distance to
\code{@1} is $D'$, then $D >= D'$.

$\mathtt{shortest}(A, D, P)$ represents a valid distance $D$ and a valid
path $P$ from node \code{@1} to node $A$. If the shortest
distance to \code{@1} is $D'$, then $D >= D'$. The \code{shortest} fact may
also represent an invalid distance if $D = \infty$, where $P = []$.

\end{invariant}

\begin{proof}

By mutual induction. All the initial facts are valid and the first
(lines~\ref{line:coord:ssspfirst1}-\ref{line:coord:ssspfirst2}) and second rules
(lines~\ref{line:coord:ssspsecond1}-\ref{line:coord:ssspsecond2}) of the program
validate the invariant using the inductive hypothesis.

\end{proof}

\begin{lemma}[Relaxation]
Shorter distances are propagated to the neighbor nodes exactly once.
\end{lemma}

\begin{proof}
By the first rule, we know that for a new shorter distance, we keep the shorter
distance and propagate it. By the second rule, longer distances are ignored.
\end{proof}

\begin{theorem}[Correctness]

   Assume a graph $G = (V, E)$ where $w(a, b) >= 0$ and $(a, b) \in E$ (positive
   weights). Consider that there is a set of nodes $S \in V$ where the shortest
   distance has been computed and a set $U \in V$ where the shortest distance
   has not been computed yet. Sets are $S$ and $U$ are disjunct. At any given
   point, $\Sigma$ is the sum of all current distances. For the
   distance $\infty$ we assign the value $\Sigma' + 1$, where $\Sigma'$ is the
   largest shortest distance of any node reachable from \code{@1}. Every rule
   inference will either:

   \begin{itemize}
      \item Maintain the size of $S$ and reduce the total number of facts in
         the database.
      \item Increase the size of $S$, reduce $\Sigma$ and potentially increase the number of
         facts in the database.
      \item Maintain the size of $S$, reduce $\Sigma$
         and potentially increase the number of facts in the database.
   \end{itemize}

   Eventually, set $S = V$ and every $\mathtt{shortest}(A, D, P)$ will represent
   the shortest distance from $A$ to \code{@1} and $P$ is its corresponding
   path.

\end{theorem}

\begin{proof}
   By nested induction on $\Sigma$ and on the number of facts in the database.

   In the base case, we have \code{relax(@1, 0, [@1])} that will give us
   the shortest distance for node \code{@1}, therefore $S = \{@1\}$ and
   $\Sigma$ is reduced.

   In the inductive case, we have a set $S'$ where the shortest distance was
   reached and \code{relax} distances may have been propagated (Relaxation
   Lemma).

   Now consider the two rules of the program:

   \begin{itemize}

      \item The first rule will only apply at nodes in $U$. If the shortest
         \code{relax} is selected, then the node is added to $S$, otherwise it
         stays in $U$ but improves the shortest path, reduces $\Sigma$ and
         \code{relax} facts are generated (Relaxation Lemma).

      \item The second rule is applied in either nodes of $S$ or $U$. For both
         sets, the rule retracts the \code{relax} fact.

   \end{itemize}

   The case where the first rule derives the shortest \code{relax} distance to
   node $t \in U$ happens when some node $s \in S$ which minimizes $\argmin_t
   d(s) + w(s, t)$ is selected, where $d(s)$ is the shortest distance to node
   \code{@1} and $w(s, t)$ the weight of the edge between $(s, t)$.
   Using \code{set-priority} increases the probability of that node being
   selected, but it does not matter since the program always makes progress and
   the shortest distances will be eventually computed.

\end{proof}
