In order to justify the introduction of coordination, we present the Single
Source Shortest Path~(SSSP), a concise program that can take advantage of custom
scheduling policies to improve its performance. The SSSP program starts (lines
1-3) with the declaration of the predicates. The first predicate, \texttt{edge},
is a persistent predicate that describes the relationship between the nodes of
the graph, where the third argument represents the weight of the edge.  The
program computes the shortest distance from node \texttt{@1} to all other nodes
in the graph. Every node has a \texttt{shortest} fact that is improved with new
\texttt{relax} facts.  Lines 5-9 declare the axioms of the program:
\texttt{edge} facts describe the graph; \texttt{shortest(A, +00, [])} is the
initial shortest distance (infinity) for all nodes; and \texttt{relax(@1, 0,
   [@1])} starts the algorithm by setting the distance from \texttt{@1} to
\texttt{@1} to be 0.

\begin{figure}[ht]
\begin{Verbatim}[numbers=left]
type route edge(node, node, int).
type linear shortest(node, int, list int).
type linear relax(node, int, list int).

!edge(@1, @2, 3). !edge(@1, @3, 1).
!edge(@3, @2, 1). !edge(@3, @4, 5).
!edge(@2, @4, 1).
shortest(A, +00, []).
relax(@1, 0, [@1]).

shortest(A, D1, P1), D1 > D2, relax(A, D2, P2)
   -o shortest(A, D2, P2),
      {B, W | !edge(A, B, W) | relax(B, D2 + W, P2 ++ [B])}.

shortest(A, D1, P1), D1 <= D2, relax(A, D2, P2)
   -o shortest(A, D1, P1).
\end{Verbatim}
\caption{Single Source Shortest Path program code.}
\label{code:shortest_path_program}
\end{figure}

\begin{figure}
\begin{center}
   \begin{subfigure}[b]{0.4\textwidth}
      \includegraphics[width=\textwidth]{figures/sssp/shortest2}
      \caption{}
   \end{subfigure}
   \begin{subfigure}[b]{0.4\textwidth}
      \includegraphics[width=\textwidth]{figures/sssp/shortest3}
      \caption{}
   \end{subfigure}
   \begin{subfigure}[b]{0.4\textwidth}
      \includegraphics[width=\textwidth]{figures/sssp/shortest8}
      \caption{}
   \end{subfigure}
\end{center}
\caption{Graphical representation of the SSSP program. (a) represents the
   program after propagating initial distance at node \texttt{@1}, followed by
   (b) where the first rule is applied in node \texttt{@2}. (c)
   represents the state of the final program, where all the shortest paths
   have been computed.}
\label{fig:shortest_path_program}
\end{figure}

The first rule of the program (lines 11-14) reads as following: if the current
\texttt{shortest} path \texttt{P1} with distance \texttt{D1} is larger than a
new path \texttt{relax} with distance \texttt{D2}, then replace the current
shortest path with \texttt{D2}, delete the new \texttt{relax} path and propagate
new paths to the neighbors (lines 13-14).  The comprehension iterates over the
edges of node \texttt{A} and derives a new \texttt{relax} fact for each node
\texttt{B} with the distance \texttt{D2 + W}, where \texttt{W} is the weight of
the edge. For example, in Fig.~\ref{fig:shortest_path_program}~(a) we apply rule
1 in node \texttt{@1} where two new \texttt{relax} facts are derived at node
\texttt{@2} and \texttt{@3}. Fig.~\ref{fig:shortest_path_program}~(b) is the
result after applying the same rule but at node \texttt{2}.

The second rule of the program (lines 16-17) retracts a \texttt{relax} fact
that has a longer distance than the current shortest distance stored in
\texttt{shortest}.

There are many opportunities for custom scheduling in the SSSP program. For
instance, after applying rule 1 in Fig.~\ref{fig:shortest_path_program}~(a), it
is possible to either apply rules in either node \texttt{@2} or node
\texttt{@3}. This decision depends largely on implementation factors such as
node partitioning and number of threads in the system.  Still, it is easy to
prove that no matter the scheduling used, the final result, as presented in
Fig.~\ref{fig:shortest_path_program}~(c), is achieved.

The SSSP program is concise and declarative but its performance depends on the
order in which nodes are executed. If nodes with greater distances are
prioritized over other nodes, the program will generate more \texttt{relax}
facts since it will take longer to reach the shortest distances. From
Fig.~\ref{fig:shortest_path_program}, it is clear that the best scheduling is
the following: \texttt{@1}, \texttt{@3}, \texttt{@2} and then \texttt{@4}, where
only 4 \texttt{relax} facts are generated. If we had decided to process nodes in
order \texttt{@1}, \texttt{@2}, \texttt{@4}, \texttt{@3}, \texttt{@4},
\texttt{@2}, then 6 \texttt{relax} facts would have been generated.  The optimal
solution for SSSP is to schedule the node with the shortest distance, which is
essentially the Dijkstra shortest path algorithm~\cite{Dijkstra}. Note how it is
possible to change the nature of the algorithm by simply changing the order of
node computation, but still retain the declarative nature of the program.

