Now that we have presented the coordination facts for LM, we are now in a
position to use them in the SSSP program presented before.  The coordinated
version of the SSSP~(Fig.~\ref{code:shortest_path_program_coord}) uses the
coordination fact \texttt{set-priority} (line 14) and a global program directive
to order priorities in ascending order (line 5).

When run with one thread, the algorithm behaves like Dijkstra's shortest path
algorithm~\cite{Dijkstra}. When using multiple threads, each thread will pick
the shortest distance from their subset of nodes.  While this does not yield the
optimal program with relation to 1 thread, it allows for parallel execution and
locally avoids unnecessary work. The result scales well and it is close to
Dijkstra's algorithm.

\begin{figure}[ht]
\begin{Verbatim}[numbers=left,xleftmargin=7mm,commandchars=\\\{\}]
type route edge(node, node, int).
type linear shortest(node, int, list int).
type linear relax(node, int, list int).

\underline{priority @order asc}.

shortest(A, +00, []).
relax(@1, 0, [@1]).

shortest(A, D1, P1), D1 > D2, relax(A, D2, P2)
   -o shortest(A, D2, P2),
      \{B, W | !edge(A, B, W) |
         relax(B, D2 + W, P2 ++ [B]),
         \underline{set-priority(B, float(D2 + W))}\}.

shortest(A, D1, P1), D1 <= D2, relax(A, D2, P2)
   -o shortest(A, D1, P1).
\end{Verbatim}
   \caption{Shortest Path Program coordinated with \texttt{set-priority}.}
   \label{code:shortest_path_program_coord}
\end{figure}

The most interesting property of the SSSP program presented in
Fig.~\ref{code:shortest_path_program_coord} is that it remains provably correct,
although it applies rules using smarter ordering and the code remains
declarative. Since the proof of correctness considers that, eventually, the shortest
path is computed at all nodes of the graph, the use of \texttt{set-priority}
does not change this at all.
