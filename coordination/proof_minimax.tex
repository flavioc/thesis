To prove that the MiniMax code shown in Fig.~\ref{code:coord:minimax} computes
the best score along with the best possible move that the \code{root-player} can
make, we have to generalize the proof and inductively prove that each node
selects the best score depending if its a minimizing or a maximizing node. We
start with several useful lemmas.

\begin{lemma}[Play Lemma]

If \code{expands(A, FirstGame, RestGame, NumberDescendants, NextPlayer, Play,
Depth)} then $\exists_n$ where $n$ is the number of available plays that
\code{NextPlayer} can make on the remaining game state \code{RestGame}.  We also
create $n$ children $B$ where \code{play(B, Game', OtherPlayer, Play', Depth +
1)} and \code{parent(B, A)} is true, where \code{OtherPlayer =
next-player(NextPlayer)}, \code{Play'} is the index of an empty position in
\code{RestGame} plus the length of \code{FirstGame}, and \code{Game'} represents
the concatenation of \code{FirstGame} and \code{RestGame} where an empty
position has been modified. Eventually, \code{expands} disappear and if
\code{NextPlayer = root-player}, \code{maximize(A, NumberDescendants + Empty,
mininf, 0)} or \code{minimize(A, NumberDescendants + Empty, maxinf, 0)},
respectively. The value \code{Empty} indicates the number of empty positions in
the state \code{RestGame}.

\end{lemma}

\begin{proof}
By induction on the size of the list \code{RestGame}. There are 5 possible rules.

Rule 1 (lines~\ref{line:coord:minimax_rule11}-\ref{line:coord:minimax_rule12}:
\code{RestGame = []} thus a \code{maximize} fact is derived as expected.

Rule 2 (lines~\ref{line:coord:minimax_rule21}-\ref{line:coord:minimax_rule22}:
\code{RestGame = []} thus a \code{minimize} fact is derived as expected.

Rule 3 (lines~\ref{line:coord:minimax_rule31}-\ref{line:coord:minimax_rule32}:
in this case we have a non-null \code{RestGame} and an empty position on index
0. As expected, we create a new \code{B} node with the right facts and a new
\code{expand} fact with a smaller \code{RestGame}. Apply the induction
hypothesis to get the remaining $n-1$ potential plays for \code{NextPlayer}.

Rule 4 (lines~\ref{line:coord:minimax_rule41}-\ref{line:coord:minimax_rule42}:
same reasoning as rule 3. Note that the presence of coordination facts do not
change the proof because they are not used in the rule's LHS.

Rule 5 (lines~\ref{line:coord:minimax_rule51}-\ref{line:coord:minimax_rule52}:
no free space on the first position of \code{RestGame}. We derive another
\code{expand} fact with a reduced \code{RestGame} and use the induction
hypothesis.

\end{proof}

\begin{lemma}[Children Lemma]
If \code{play(A, Game, NextPlayer, LastPlay, Depth)} then either \code{score(A, Score,
      LastPlay)} or $\exists_n$ where $n$ is the number of available plays for
\code{NextPlayer} in \code{Game}, creating $n$ new children $B$ where \code{play(B,
      Game', OtherPlayer, Play, Depth + 1)} and a \code{maximize(A, N,
         mininf, 0)} or \code{minimize(A, N, maxinf, 0)} is created if
         $NextPlayer = root-player$ or not, respectively. \code{Game'} is an
         updated \code{Game} state where an empty position is played by
         \code{NextPlayer}.
\end{lemma}
\begin{proof}

A linear fact \code{play(A, Game, NextPlayer, LastPlay, Depth)} can only be used
in either in the first or second rules.
If the rule in
lines~\ref{line:coord:minimax_play11}-\ref{line:coord:minimax_play12} executes,
(\code{minimax\_score} returns a score) it means that \code{Game} is a final
game state and there's no available plays for \code{NextPlayer}.
Otherwise, the second rule in
lines~\ref{line:coord:minimax_play21}-\ref{line:coord:minimax_play22} will
apply and the Play lemma is used to prove this lemma.

\end{proof}

We now prove that the minimization and maximization rules work given the right
amount of scores available.

\begin{lemma}[Maximize Lemma]
Given a fact \code{maximize(A, N, BestScore, BestPlay)} and \code{N} facts
\code{new-score(A, OtherScore, OtherPlay)}, then we end up with a single
\code{score(A, BestScore', BestPlay')} where \code{BestScore'} is the
highest score from \code{BestScore} or \code{OtherScore'}s and
\code{BestPlay'} is the corresponding play.
\end{lemma}
\begin{proof}
By induction on the number of \code{new-score} facts.

Case 1 (\code{N = 0}): trivial by applying the rule in
line~\ref{line:coord:minimax_maximize2}.

Case 2 (\code{N > 0}): by picking an arbitrary \code{new-score(A, OtherScore,
OtherPlay} fact.

\begin{itemize}
   \item Sub case 2.1
      (lines~\ref{line:coord:minimax_maximize_rule11}-\ref{line:coord:minimax_maximize_rule12})

      If \code{BestScore < OtherScore} then we derive \code{maximize(A, N-1,
      OtherScore, OtherPlay)}. Use induction to get the final \code{score} fact.

   \item Sub case 2.2
      (lines~\ref{line:coord:minimax_maximize_rule21}-\ref{line:coord:minimax_maximize_rule22})

      If \code{BestScore >= OtherScore} then we derive \code{maximize(A, N-1,
      BestScore, BestPlay)}. Use induction to get the final \code{score} fact.

\end{itemize}

\end{proof}

\begin{lemma}[Minimize Lemma]
Given a fact \code{minimize(A, N, BestScore, BestPlay)} and \code{N} facts
\code{new-score(A, OtherScore, OtherPlay)}, then we end up with a single
\code{score(A, BestScore', BestPlay')} where \code{BestScore'} is the
lowest score from either \code{BestScore} or \code{OtherScore'}s and
\code{BestPlay'} is the corresponding play.
\end{lemma}
\begin{proof}
Use the same process used in Maximize Lemma.
\end{proof}

Finally, we are in a position to prove that a \code{play} fact eventually
produces a \code{score} fact that indicates the best score and best play for the
player playing at that node.

\begin{theorem}[Score Theorem]
For every \code{play(A, Game, NextPlayer, LastPlay, Depth)}, we either
directly get a \code{score} fact (leaf case) or a recursive alternate maximization or
minimization (depending if \code{NextPlayer = root-player}, respectively) of
the children nodes. This max/minization also results in a \code{score(A, Score,
BestPlay)} fact where \code{Score} is the max/minimum and \code{BestPlay} is the
corresponding play for that score.
\end{theorem}
\begin{proof}
By applying the Children Lemma and using induction on the number of free
positions on the state \code{Game}.

Case 1 (no free positions or game is final): direct score.

Case 2 (available positions): $n$ children nodes $B$ with \code{play(B, Game',
next-player(NextPlayer), X, Play, Depth + 1)}, where \code{Game'} has position
\code{X} filled up. We apply induction on the \code{play} fact of each child
\code{B} to get a \code{score(B, Score, Play)}. Since we also derived a
\code{parent(B, A)} fact, rule in line~\ref{line:coord:minimax_new} eventually
executes, deriving \code{new-score(A, Score, Play)}.

Since we also derived \code{maximize(A, N, mininf, 0)} (or \code{minimize}), we
have this maximize fact and $n$ \code{new-score} facts from the $n$ children
nodes. Applying the Maximize Lemma, we get \code{score(A, BestScore, BestPlay)}.

\end{proof}

\begin{corollary}[MiniMax]
Starting from a \code{play(@0, initial-game, root-player, 0, 1)}, the fact
\code{score(A, BestScore, BestPlay)} is eventually derived, where \code{BestPlay} represents the
best play which player \code{root-player} is able to make that minimizes the possible loss for
a worst case scenario giving \code{initial-game}.
\end{corollary}
