To prove that the MiniMax code shown in Fig.~\ref{code:coord:minimax} computes
the best score along with the best possible move that the \code{root-player} can
make, we have to generalize the proof and inductively prove that each node
selects the best score depending if its a minimizing or a maximizing node. We
start with several useful lemmas.

\begin{lemma}[Play Lemma]

If a fact $\mathtt{expand}(A, FirstGame, RestGame, ND,
NextPlayer, Play, Depth)$ exists then $\exists_n$ where $n$ is the number of
available plays that $NextPlayer$ can make on the remaining game state
$RestGame$. The initial $\mathtt{expand}$ fact is retracted and generates $n$
children $B$ with facts $\mathtt{play}(B, Game', OPlayer, Play', Depth + 1)$
and $\mathtt{parent}(B, A)$, where $OPlayer =
\mathtt{next-player}(NextPlayer)$, $Play'$ is the index of an empty position in
$RestGame$ plus the length of $FirstGame$, and $Game'$ represents the
concatenation of $FirstGame$ and $RestGame$ where an empty position has been
modified. The initial $\mathtt{expand}$ fact eventually derives either the
$\mathtt{maximize}(A, ND + Empty, -\infty, 0)$ fact (if
$NextPlayer = root-player$) or the $\mathtt{minimize}(A, ND +
Empty, \infty, 0)$ (otherwise).  The value $Empty$ indicates the number of empty
positions in the state $RestGame$.

\end{lemma}

\begin{proof}
By induction on the size of the list $RestGame$. There are 5 possible rules.

Rule 1 (lines~\ref{line:coord:minimax_rule11}-\ref{line:coord:minimax_rule12}:
$RestGame = []$ thus a \code{maximize} fact is derived as expected.

Rule 2 (lines~\ref{line:coord:minimax_rule21}-\ref{line:coord:minimax_rule22}:
$RestGame = []$ thus a \code{minimize} fact is derived as expected.

Rule 3 (lines~\ref{line:coord:minimax_rule31}-\ref{line:coord:minimax_rule32}:
in this case we have a non-null $RestGame$ and an empty position on index 0. As
expected, we create a new $B$ node with the expected facts and a new \code{expand}
fact with a smaller $RestGame$. Apply the induction hypothesis to get the
remaining $n-1$ potential plays for $NextPlayer$.

Rule 4 (lines~\ref{line:coord:minimax_rule41}-\ref{line:coord:minimax_rule42}:
same reasoning as rule 3. Note that the presence of coordination facts do not
change the proof because they are not used in the rule's LHS.

Rule 5 (lines~\ref{line:coord:minimax_rule51}-\ref{line:coord:minimax_rule52}:
no free space on the first position of $RestGame$. We derive another
\code{expand} fact with a reduced $RestGame$ and use the induction
hypothesis.

\end{proof}

\begin{lemma}[Children Lemma]

The fact $\mathtt{play}(A, Game, NextPlayer, LastPlay, Depth)$ is consumed by
the program's rule to realize one of the following scenarios:

\begin{itemize}
   \item A new fact $\mathtt{score}(A, Score, LastPlay)$ is derived.

   \item There exists a number $n$ of available plays for $NextPlayer$ in
      $Game$, which generate $n$ new children $B$ where each $B$ will have facts
      $\mathtt{play}(B, Game', OPlayer, Play,$$ Depth + 1)$ and
      $\mathtt{parent}(B, A)$. Furthermore, from the use of the initial
      $play$ fact, one of the following facts is derived:

\begin{itemize}
   \item $\mathtt{maximize}(A, N, -\infty, 0)$ is derived (if $NextPlayer =
      root-player$);
   \item $\mathtt{minimize}(A, N, \infty, 0)$ (otherwise).
\end{itemize}

Note that $Game'$ is an updated $Game$ state where an empty position is played
by $NextPlayer$.

\end{itemize}

\end{lemma}
\begin{proof}

A linear fact $\mathtt{play}(A, Game, NextPlayer, LastPlay, Depth)$ can only be
used in either the first or second rules.  If the rule in
lines~\ref{line:coord:minimax_play11}-\ref{line:coord:minimax_play12} executes,
(\code{minimax\_score} returns a score) it means that $Game$ is a final game
state and there's no available plays for $NextPlayer$.  Otherwise, the second
rule in lines~\ref{line:coord:minimax_play21}-\ref{line:coord:minimax_play22}
will apply and the Play lemma is used to complete the proof.

\end{proof}

We now prove that the minimization and maximization rules work given the right
amount of scores available.

\begin{lemma}[Maximize Lemma]
Given a fact $\mathtt{maximize}(A, N, BScore, BPlay)$ and $N$ facts
\code{new-score}$(A, OScore, OPlay)$, the program's rule will retract
them all and generate a single 
$\mathtt{score}(A, BScore', BPlay')$ fact where $BScore'$ is the
highest score from $BScore$ or $OScore'$ and
$BPlay'$ is the corresponding play.
\end{lemma}
\begin{proof}
By induction on the number of \code{new-score} facts.

Case 1 ($N = 0$): trivial by applying the rule in
line~\ref{line:coord:minimax_maximize2}.

Case 2 ($N > 0$): by picking an arbitrary \code{new-score}$(A, OScore,
OPlay)$ fact.

\begin{itemize}
   \item Sub case 2.1
      (lines~\ref{line:coord:minimax_maximize_rule11}-\ref{line:coord:minimax_maximize_rule12})

      If $BScore < OScore$ then we derive $\mathtt{maximize}(A, N-1,
      OScore, OPlay)$. Use induction to get the final \code{score} fact.

   \item Sub case 2.2
      (lines~\ref{line:coord:minimax_maximize_rule21}-\ref{line:coord:minimax_maximize_rule22})

      If $BScore >= OScore$ then we derive $\mathtt{maximize}(A, N-1,
      BScore, BPlay)$. Use induction to get the final \code{score} fact.

\end{itemize}

\end{proof}

\begin{lemma}[Minimize Lemma]

Given a fact $\mathtt{minimize}(A, N, BScore, BPlay)$ and $N$
\code{new-score}$(A, OScore, OPlay)$ facts, the program's rules will retract all
those facts and derive a single $\mathtt{score}(A, BScore', BPlay')$ fact
where $BScore'$ is the lowest score from either $BScore$ or the $N$
$OScore$ and $BPlay'$ is the corresponding play.

\end{lemma}

Finally, we are in a position to prove that a \code{play} fact is retracted and
then eventually produces a \code{score} fact that indicates the best score and
best play for the player playing at the given node.

\begin{theorem}[Score Theorem]

For every $\mathtt{play}(A, Game, NextPlayer, LastPlay, Depth)$ fact, we either
get a \code{score} fact (leaf case) or a recursive alternate maximization or
minimization (depending if \code{NextPlayer = root-player} or otherwise) of the
children nodes. This max/minimization also results in a $\mathtt{score}(A,
Score, BPlay)$ fact where $Score$ is the max/minimum and $BPlay$ is the
corresponding play for that score.

\end{theorem}
\begin{proof}
By applying the Children Lemma and using induction on the number of free
positions on the state $Game$.

Case 1 (no free positions or game is final): direct score.

Case 2 (available positions): $n$ children nodes $B$ are created with two facts:
$\mathtt{parent}(B, A)$ and $\mathtt{play}(B, Game',$
\code{next-player}$(NextPlayer), X, Play, Depth + 1)$, where $Game'$ has
position $X$ filled up. We apply induction on the \code{play} fact of each child
$B$ to get a $\mathtt{score}(B, Score, Play)$ fact. Since we also derived a
$\mathtt{parent}(B, A)$ fact, rule in line~\ref{line:coord:minimax_new}
eventually executes, deriving $\mathtt{new-score}(A, Score, Play)$.

We also derived $\mathtt{maximize}(A, N, -\infty, 0)$ fact (or \code{minimize})
and $n$ \code{new-score} facts from the $n$ children nodes, from which we apply
the Maximize Lemma to get $\mathtt{score}(A, BScore, BPlay)$.

\end{proof}

\begin{corollary}[MiniMax]

Starting from a $\mathtt{play}(\mathtt{@0}, initial-game, root-player, 0, 1)$
fact, the fact $\mathtt{score}(A, BScore, BPlay)$ is eventually derived,
where $BPlay$ represents the best play which player $root-player$ is able to
make that minimizes the possible loss for a worst case scenario given
$initial-game$.

\end{corollary}
