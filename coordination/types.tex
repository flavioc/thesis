LM introduces the concept of coordination that allows the programmer to write
code that changes how the runtime system schedules and partitions node across
threads of execution. Beyond the distinction between linear and persistent
facts, LM further classifies facts into 3 categories: \emph{computation} facts,
\emph{structural} facts and \emph{coordination} facts.
Predicates are also classified accordingly.

Computation facts are regular facts used to represent the program state. In
Fig.~\ref{code:shortest_path_program}, \texttt{relax} and \texttt{shortest} are
all computation facts.

Structural facts describe information about the connections between the nodes in
the graph. In the example of Fig.~\ref{code:shortest_path_program},
\texttt{edge} facts are structural since the corresponding \texttt{edge} is used
for communication between nodes.  Note that structural facts can also be seen as
computation facts since they are heavily used in the program logic.

\emph{Coordination facts} allow the programmer to change how the run time
schedules nodes and how it partitions the nodes among threads of execution.
Coordination facts can be used in either in the LHS, RHS or both.  This allows
scheduling and partition decisions to be made based on the state of the program
and on the state of the underlying machine.  In this fashion, we keep the
language declarative because we reason logically about the state of execution,
without the need to introduce extra-logical operators into the language that
would introduce significant issues when proving properties about programs.

Coordination facts are further classified into two kinds of facts:
\emph{sensing} and \emph{action} facts. Sensing facts are used to sense
information about the underlying runtime system, including node placement and
node scheduling.  Action facts are used to apply a coordination operations on
the runtime system.

%%%%%%%%%%%%%%%%%%%%%%%%%%%%%%%%%%%%%%%%%%%%%%%%%%%%%%%%%%%%%%%%%%%%%%%%%%%

Sensing facts are facts about the current state of the runtime system, such as
the placement of nodes in the CPU and scheduling information. In the original
Meld, sensing facts were used to get information about the outside world, like
temperature, touch data, neighborhood status, etc.

Action facts are linear facts which are consumed when the corresponding action
is performed. In the original Meld, they were used to make the robots perform
actions in the outside world.  For LM we use them to change the order in which
nodes are evaluated in the runtime system and to make partitioning decisions
(assign nodes to threads). It is possible to give hints to the virtual machine
in order to prioritize the computation of some nodes.

With sensing facts and action facts, we can write \emph{meta-rules} that will
sense the state of the runtime system and then apply decisions in order to
improve execution speed or change partitioning information. In some situations,
this set of rules can be added to the program without any modifications to the
original rules.

