LM introduces the concept of coordination that allows the programmer to write
code that changes how the runtime system schedules and partitions the set of
available nodes across the threads of execution. Beyond the distinction between
linear and persistent facts, LM further classifies facts into 3 categories:
\emph{computation facts}, \emph{structural facts} and \emph{coordination facts}.
Predicates are also classified accordingly.

Computation facts are regular facts used to represent the program state. In
Fig.~\ref{code:shortest_path_program}, \texttt{relax} and \texttt{shortest} are
both computation facts.

Structural facts describe information about the connections between the nodes in
the graph. In Fig.~\ref{code:shortest_path_program}, \texttt{edge} facts are
structural since the corresponding \texttt{edge} is used for communication
between nodes.  Note that structural facts can also be seen as computation facts
since they are heavily used in the program's logic.

\emph{Coordination facts} are classified into \emph{scheduling facts} and
\emph{partitioning facts} and allow the programmer to change how the run time
schedules nodes and how it partitions the nodes among threads of execution,
respectively. Coordination facts can be used either in the LHS, RHS or both.
This allows scheduling and partition decisions to be made based on the state of
the program and on the state of the underlying machine.  In this fashion, we
keep the language declarative because we reason logically about the state of
execution, without the need to introduce extra-logical operators into the
language that would introduce significant issues when proving properties about
the programs.

Both scheduling and partitioning facts can be further classified into two kinds
of facts: \emph{sensing facts} and \emph{action facts}. Sensing facts are used
to sense information about the underlying runtime system, such as the placement
of nodes in the CPU and scheduling information.\footnote{In the original
   Meld~\cite{ashley-rollman-iclp09}, sensing facts were used to get information
about the outside world, like temperature, touch data, neighborhood status,
etc.}

Action facts are used to apply coordination operations on the runtime system.
Action facts are linear facts which are consumed when the corresponding action
is performed.\footnote{Like sensing facts, action facts were also introduced in the
original Meld and were used to make the robots perform actions in the outside
world (e.g., moving, changing speed).} We use them to change the order in
which nodes are evaluated in the runtime system and to make partitioning
decisions (assign nodes to threads). It is possible to give hints to the virtual
machine in order to prioritize the computation of some nodes.

With sensing facts and action facts, we can write \emph{meta-rules} that will
sense the state of the runtime system and then apply decisions in order to
improve execution speed or change partitioning information. In most situations,
this set of rules can be added to the program without modifying the meaning of
the original rules.
