
Proving that deriving an aggregate in LLD is sound in relation to HLD is
built from 4 results: (1) proving that matching the body of the aggregate is
sound in relation to HLD; (2) proving that updating the continuation stacks
makes them suitable for use in the next aggregate applications; (3) proving
that deriving the head of the aggregate is sound in relation to HLD; (4)
proving that we can apply as many aggregates as the database allows.

\iffalse
\begin{lemma}[Aggregate body match]\label{thm:aggregate_body_match}
Given an aggregate $AG = \aggsz{A}{B}{C}$, consider a triplet $T = A; \Gamma;
\Delta_{N}$ and a context $\Delta_{N} = \Delta_1, \Delta_2, \Xi$.

If $\ma{AG} \Gamma; \Delta_1, \Delta_2; \Xi_N; \Gamma_{N1}; \Delta_{N1}; \Xi;
\Omega; \lstack{C}; \lstack{P}; \Omega_N; \Delta_N; \Sigma \rightarrow \outsem$ is well-formed in relation to $T$ then either:

\begin{itemize}[leftmargin=*]
   \item Match fails:
   \begin{itemize}[leftmargin=\secondm]
      \item $\done \Gamma; \Delta_N; \Xi_N; \Gamma_{N1}; \Delta_{N1}; \Omega_N \rightarrow \outsem$
   \end{itemize}
   
   \item Match succeeds with no backtracking to frames of stack $\lstack{C}$
   or $\lstack{P}$ ($\lstack{C} \neq \cdot$):

   \begin{itemize}[leftmargin=\secondm]
      \item $\mz \Gamma; \Delta_2 \rightarrow A$
      \item $\ma{AG} \Gamma; \Delta_1; \Xi_N; \Gamma_{N1}; \Delta_{N1}; \Xi,
         \Delta_2; \cdot; \lstack{C'}, \lstack{C}; \lstack{P'}, \lstack{P};
   \Omega_N; \Delta_N; \Sigma \rightarrow \outsem$ (well-formed in relation to $T$)
   \end{itemize}

   \item Match succeeds with no backtracking to frames of stack $\lstack{P}$
($\lstack{C} = \cdot$):
\begin{itemize}[leftmargin=\secondm]
   \item $\mz \Gamma; \Delta_2 \rightarrow A$
   \item $\ma{AG} \Gamma; \Delta_1; \Xi_N; \Gamma_{N1}; \Delta_{N1}; \Xi,
      \Delta_2; \cdot; \lstack{C'}; \lstack{P'}, \lstack{P}; \Omega_N; \Delta_N;
      \Sigma \rightarrow \outsem$ (well-formed in relation to $T$)
\end{itemize}

\item Match succeeds with backtracking to a linear continuation frame in
stack $\lstack{C}$ ($\lstack{C} \neq \cdot$):

\begin{itemize}[leftmargin=\secondm]
   \item $\mz \Gamma; \Xi_1, \dotsc, \Xi_m, p_2, \Xi_c$
   \item $\exists_{f \in \lstack{C}}. f = (\Delta_a; \Delta_{b_1}, p_2,
         \Delta_{b_2}; p; \Xi_1, \dotsc, \Xi_m; \Omega_1, \dotsc,
         \Omega_k; \Lambda_1, \dotsc, \Lambda_m; \Upsilon_1, \dotsc,
         \Upsilon_n)$
   \item $\lstack{C} = \lstack{C'}, f, \lstack{C''}$
   \item $f$ turns into $f' = (\Delta_a, \Delta_{b_1}, p_2;
         \Delta_{b_2}; p; \Xi_1, \dotsc, \Xi_m;
         \Omega_1, \dotsc, \Omega_k; \Lambda_1, \dotsc, \Lambda_m;
         \Upsilon_1, \dotsc, \Upsilon_n)$
   \item $\ma{AG} \Gamma; \Delta_c; \Xi_N; \Gamma_{N1}; \Delta_{N1}; \Xi_1,
      \dotsc, \Xi_m, p_2, \Xi_c; \cdot; \lstack{C'''}, f', \lstack{C''}; \lstack{P};
      \Omega_N; \Delta_N; \Sigma \rightarrow \outsem$ (well-formed in relation to $T$)
   \item $\Delta_c = (\Delta_1, \Delta_2, \Xi) - (\Xi_1, \dotsc, \Xi_m,
         p_2, \Xi_c)$
\end{itemize}

\item Match succeeds with backtracking to a persistent continuation frame
in stack $\lstack{C}$ ($\lstack{C} \neq \cdot$):
\begin{itemize}[leftmargin=\secondm]
   \item $\mz \Gamma; \Delta_{c_2}, \Xi_1, \dotsc, \Xi_m \rightarrow A$
   \item $\exists_{f \in \lstack{C}}. f = [\Gamma_1, p_2, \Gamma_2; \Delta_{c_1},
      \Delta_{c_2}; \Xi_1, \dotsc, \Xi_m; \bang p; \Omega_1, \dotsc, \Omega_k;
      \Lambda_1, \dotsc, \Lambda_m; \Upsilon_1, \dotsc, \Upsilon_n]$
   \item $\lstack{C} = \lstack{C'}, f, \lstack{C''}$
   \item $f$ turns into $f' = [\Gamma_2; \Delta_{c_1}, \Delta_{c_2};
      \Xi_1, \dotsc, \Xi_m; \bang p; \Omega_1, \dotsc, \Omega_k; \Lambda_1,
      \dotsc, \Lambda_m; \Upsilon_1, \dotsc, \Upsilon_n]$
   \item $\ma{AG} \Gamma; \Delta_{c_1}; \Xi_N; \Gamma_{N1}; \Delta_{N1};
      \Delta_{c_2}, \Xi_1, \dotsc, \Xi_m; \cdot; \lstack{C'''}, f', \lstack{C''}; \lstack{P};
      \Omega_N; \Delta_N; \Sigma \rightarrow \outsem$ (well-formed in relation to $T$)
   \item $\Delta_{c_1}, \Delta_{c_2} = (\Delta_1, \Delta_2, \Xi) - (\Xi_1, \dotsc, \Xi_m)$
\end{itemize}

\item Match succeeds with backtracking to a persistent continuation frame
in stack $\lstack{P}$ ($\lstack{C} = \cdot$):
   \begin{itemize}[leftmargin=\secondm]
      \item $\mz \Gamma; \Delta_{c_2}, \Xi_1, \dotsc, \Xi_m \rightarrow A$
      \item $\exists_{f \in \lstack{P}}. f = [\Gamma_1, p_2, \Gamma_2; \Delta_{c_1}, \Delta_{c_2};
   \Xi_1, \dotsc, \Xi_m; \bang p; \Omega_1, \dotsc, \Omega_k; \Lambda_1,
   \dotsc, \Lambda_m; \Upsilon_1, \dotsc, \Upsilon_n]$
      \item $\lstack{P} = \lstack{P'}, f, \lstack{P''}$
      \item $f$ turns into $f' = [\Gamma_2; \Delta_{c_1},
         \Delta_{c_2}; \Xi_1, \dotsc, \Xi_m; \bang p; \Omega_1, \dotsc,
         \Omega_k; \Lambda_1, \dotsc, \Lambda_m; \Upsilon_1, \dotsc, \Upsilon_n]$
      \item $\ma{AG} \Gamma; \Delta_{c_1}; \Xi_N; \Gamma_{N1}; \Delta_{N1};
         \Delta_{c_2}, \Xi_1, \dotsc, \Xi_m; \cdot; \lstack{C'}; \lstack{P'''}, f', \lstack{P''};
         \Omega_N; \Delta_N; \Sigma \rightarrow \outsem$
         (well-formed in relation to $T$)
      \item $\Delta_{c_1}, \Delta_{c_2} = (\Delta_1, \Delta_2, \Xi) - (\Xi_1, \dotsc,
            \Xi_m)$
   \end{itemize}


\end{itemize}

If $\conta{AG} \Gamma; \Delta_{N}; \Xi_{N}; \Gamma_{N1}; \Delta_{N1};
\lstack{C}; \lstack{P}; \Omega_N; \Sigma \rightarrow \outsem$ and $\lstack{C}$ and
$\lstack{P}$ are well-formed in relation to $T$ then either:

\begin{itemize}[leftmargin=*]
   \item Match fails:
   \begin{itemize}[leftmargin=\secondm]
      \item $\done \Gamma; \Delta_N; \Xi_N; \Gamma_{N1}; \Delta_{N1}; \Omega_N \rightarrow \outsem$
   \end{itemize}

   \item Match succeeds with no backtracking to frames of stack $\lstack{P}$
($\lstack{C} = \cdot$):
\begin{itemize}[leftmargin=\secondm]
   \item $\mz \Gamma; \Delta_2 \rightarrow A$
   \item $\ma{AG} \Gamma; \Delta_1; \Xi_N; \Gamma_{N1}; \Delta_{N1}; \Xi,
      \Delta_2; \cdot; \lstack{C'}; \lstack{P'}, \lstack{P}; \Omega_N; \Delta_N;
      \Sigma \rightarrow \outsem$ (well-formed in relation to $T$)
\end{itemize}

\item Match succeeds with backtracking to a linear continuation frame in
stack $\lstack{C}$ ($\lstack{C} \neq \cdot$):

\begin{itemize}[leftmargin=\secondm]
   \item $\mz \Gamma; \Xi_1, \dotsc, \Xi_m, p_2, \Xi_c$
   \item $\exists_{f \in \lstack{C}}. f = (\Delta_a; \Delta_{b_1}, p_2,
         \Delta_{b_2}; p; \Xi_1, \dotsc, \Xi_m; \Omega_1, \dotsc,
         \Omega_k; \Lambda_1, \dotsc, \Lambda_m; \Upsilon_1, \dotsc,
         \Upsilon_n)$
   \item $\lstack{C} = \lstack{C'}, f, \lstack{C''}$
   \item $f$ turns into $f' = (\Delta_a, \Delta_{b_1}, p_2;
         \Delta_{b_2}; p; \Xi_1, \dotsc, \Xi_m;
         \Omega_1, \dotsc, \Omega_k; \Lambda_1, \dotsc, \Lambda_m;
         \Upsilon_1, \dotsc, \Upsilon_n)$
   \item $\ma{AG} \Gamma; \Delta_c; \Xi_N; \Gamma_{N1}; \Delta_{N1}; \Xi_1,
      \dotsc, \Xi_m, p_2, \Xi_c; \cdot; \lstack{C'''}, f', \lstack{C''}; \lstack{P};
      \Omega_N; \Delta_N; \Sigma \rightarrow \outsem$ (well-formed in relation to $T$)
   \item $\Delta_c = (\Delta_1, \Delta_2, \Xi) - (\Xi_1, \dotsc, \Xi_m,
         p_2, \Xi_c)$
\end{itemize}

\item Match succeeds with backtracking to a persistent continuation frame
in stack $\lstack{C}$ ($\lstack{C} \neq \cdot$):
\begin{itemize}[leftmargin=\secondm]
   \item $\mz \Gamma; \Delta_{c_2}, \Xi_1, \dotsc, \Xi_m \rightarrow A$
   \item $\exists_{f \in \lstack{C}}. f = [\Gamma_1, p_2, \Gamma_2; \Delta_{c_1},
      \Delta_{c_2}; \Xi_1, \dotsc, \Xi_m; \bang p; \Omega_1, \dotsc, \Omega_k;
      \Lambda_1, \dotsc, \Lambda_m; \Upsilon_1, \dotsc, \Upsilon_n]$
   \item $\lstack{C} = \lstack{C'}, f, \lstack{C''}$
   \item $f$ turns into $f' = [\Gamma_2; \Delta_{c_1}, \Delta_{c_2};
      \Xi_1, \dotsc, \Xi_m; \bang p; \Omega_1, \dotsc, \Omega_k; \Lambda_1,
      \dotsc, \Lambda_m; \Upsilon_1, \dotsc, \Upsilon_n]$
   \item $\ma{AG} \Gamma; \Delta_{c_1}; \Xi_N; \Gamma_{N1}; \Delta_{N1};
      \Delta_{c_2}, \Xi_1, \dotsc, \Xi_m; \cdot; \lstack{C'''}, f', \lstack{C''}; \lstack{P};
      \Omega_N; \Delta_N; \Sigma \rightarrow \outsem$ (well-formed in relation to $T$)
   \item $\Delta_{c_1}, \Delta_{c_2} = (\Delta_1, \Delta_2, \Xi) - (\Xi_1, \dotsc, \Xi_m)$
\end{itemize}

\item Match succeeds with backtracking to a persistent continuation frame
in stack $\lstack{P}$ ($\lstack{C} = \cdot$):
   \begin{itemize}[leftmargin=\secondm]
      \item $\mz \Gamma; \Delta_{c_2}, \Xi_1, \dotsc, \Xi_m \rightarrow A$
      \item $\exists_{f \in \lstack{P}}. f = [\Gamma_1, p_2, \Gamma_2; \Delta_{c_1}, \Delta_{c_2};
   \Xi_1, \dotsc, \Xi_m; \bang p; \Omega_1, \dotsc, \Omega_k; \Lambda_1,
   \dotsc, \Lambda_m; \Upsilon_1, \dotsc, \Upsilon_n]$
      \item $\lstack{P} = \lstack{P'}, f, \lstack{P''}$
      \item $f$ turns into $f' = [\Gamma_2; \Delta_{c_1},
         \Delta_{c_2}; \Xi_1, \dotsc, \Xi_m; \bang p; \Omega_1, \dotsc,
         \Omega_k; \Lambda_1, \dotsc, \Lambda_m; \Upsilon_1, \dotsc, \Upsilon_n]$
      \item $\ma{AG} \Gamma; \Delta_{c_1}; \Xi_N; \Gamma_{N1}; \Delta_{N1};
         \Delta_{c_2}, \Xi_1, \dotsc, \Xi_m; \cdot; \lstack{C'}; \lstack{P'''}, f', \lstack{P''};
         \Omega_N; \Delta_N; \Sigma \rightarrow \outsem$
         (well-formed in relation to $T$)
      \item $\Delta_{c_1}, \Delta_{c_2} = (\Delta_1, \Delta_2, \Xi) - (\Xi_1, \dotsc,
            \Xi_m)$
   \end{itemize}

   
\end{itemize}
\end{lemma}

Proving that matching the body of a aggregate is sound in relation to HLD
follows the structure of the Lemma~\ref{thm:comprehension_body_match}. Next, we
need to prove the theorem that corrects the continuation stack for the next
application of the aggregate. Note that the aggregate value is appended to
$\Sigma$ after matching finishes. We do not show that step here, but in
Corollary~\ref{thm:agg_match_to_derivation}.

\begin{theorem}[From update to derivation]\label{thm:agg_from_update_to_derivation}
Assume an aggregate $AG = \aggsz{A}{B}{C}$. \\ A stack update
$\fixa{AG} \Gamma; \Delta; \Xi_N; \Gamma_{N1}; \Delta_{N1}; \Xi; \lstack{C};
\lstack{P}; \Omega_N; \Delta_N; \Sigma \rightarrow \outsem$ implies a derivation
$\da{AG} \Gamma; \Delta_N - \Xi; \Xi_N, \Xi; \Gamma_{N1}; \Delta_{N1}; B;
\lstack{C'} ; \lstack{P'}; \Omega_N; \Sigma \rightarrow
\outsem$, where:

\begin{itemize}[leftmargin=*]
   \item If $\lstack{C} = \cdot$ then $\lstack{C'} = \cdot$

   \item If $\lstack{C} = \lstack{C_{1}}, (\Delta_a; \Delta_b; \cdot; p; \Omega; \cdot; \Upsilon)$
   then $\lstack{C'} = (\Delta_a - \Xi; \Delta_b - \Xi; \cdot; p; \Omega; \cdot;
         \Upsilon)$

   \item $\lstack{P'}$ is the transformation of stack $\lstack{P}$, where for every frame $f \in
   \lstack{P}$ of the form $[\Gamma'; \Delta_N; \cdot; \bang p; \Omega; \cdot; \Upsilon]$
   will turn into $f' = [\Gamma';\Delta_N-\Xi;\cdot;\bang p;\Omega;\cdot;\Upsilon]$

\end{itemize}
\end{theorem}
\begin{proof}
Use induction on the size of the stack $\lstack{C}$ to get the result of
$\lstack{C'}$ and then apply Theorem~\ref{thm:stack_update} to get $\lstack{P'}$.
\end{proof}

\begin{corollary}[Match to derivation]\label{thm:agg_match_to_derivation}
If $\ma{AG} \Gamma; \Delta; \Xi_N; \Gamma_{N1}; \Delta_{N1}; \Xi; \cdot; B;
\lstack{C}; \lstack{P};
\Omega_N; \Delta_N; \Sigma \rightarrow \outsem$ then\\
$\da{AG} \Gamma; \Delta_N - \Xi; \Xi_N, \Xi; \Gamma_{N1}; \Delta_{N1}; B; \lstack{C'};
\lstack{P'}; \Omega_N; V :: \Sigma \rightarrow \outsem$ where:
   
\begin{itemize}[leftmargin=*]
   \item If $\lstack{C} = \cdot$ then $\lstack{C'} = \cdot$
   \item If $\lstack{C} = \lstack{C_1}, (\Delta_a; \Delta_b; \cdot; p; \Omega;
         \cdot; \Upsilon)$ then $\lstack{C'} = (\Delta_a - \Xi; \Delta_b - \Xi; \cdot; p;
            \Omega; \cdot; \Upsilon)$
   \item $\lstack{P'}$ is the transformation of stack $\lstack{P}$, where for every frame $f \in
   \lstack{P}$ of the form $[\Gamma'; \Delta_N; \cdot; \bang p; \Omega; \cdot; \Upsilon]$
   will turn into $f' = [\Gamma';\Delta_N-\Xi;\cdot;\bang p;\Omega;\cdot;\Upsilon]$
\end{itemize}
\end{corollary}

\begin{proof}
Invert the assumption and then apply Theorem~\ref{thm:agg_from_update_to_derivation}.
\end{proof}


\paragraph{Aggregate Derivation}

We have just seen that after a single aggregate application, we add a value $V$
to the $\Sigma$ context and that the continuation stacks are now valid.
Now, we need to prove that deriving the head of the
aggregate is sound in relation to HLD by using the new stacks.

\begin{theorem}[Aggregate derivation soundness]\label{thm:aggregate_derivation}
If $\da{AG} \Gamma; \Delta_N; \Xi_N; \Gamma_{N1}; \Delta_{N1}; \Omega_1, \dotsc,
   \Omega_n; \lstack{C}; \lstack{P}; \Omega_N; \Sigma \rightarrow \outsem$ then:

\begin{itemize}[leftmargin=*]
   \item $\da{AG} \Gamma; \Delta_N; \Xi_N; \Gamma_{N1}, \Gamma_1, \dotsc, \Gamma_n; \Delta_{N1},
   \Delta_1, \dotsc, \Delta_n; \cdot; \lstack{C}; \lstack{P}; \Omega_N \rightarrow
   \outsem$;

   \item $\forall_{\Omega_x}.($ if $\dz \Gamma; \Delta_N; \Xi_N;
   \Gamma_{N1}, \Gamma_1, \dotsc, \Gamma_n; \Delta_{N1}, \Delta_1, \dotsc,
   \Delta_n; \Omega_x \rightarrow \outsem$ then

   $\dz \Gamma; \Delta_N; \Xi_N; \Gamma_{N1}; \Delta_{N1}; \Omega_1, \dotsc,
   \Omega_n, \Omega_x \rightarrow \outsem)$.

\end{itemize}
\end{theorem}

\begin{proof}
Straightforward induction on $\Omega_1, \dotsc, \Omega_n$.
\end{proof}


\paragraph{Multiple Aggregate Derivation} We now prove that it is possible
to apply an aggregate several times in order to a get multiple values (one per
application). In turn, we also conclude important results for the
soundness of the aggregate computation mechanism.

\begin{theorem}[Multiple aggregate derivation]\label{thm:multiple_aggregate_derivation}
Consider a triplet $T = A; \Gamma; \Delta_{N}$ and an aggregate $AG =
\aggsz{A}{B}{C}$. Assume that there exists $n \geq 0$ applications of $AG$
where the $i_{th}$ application is related to the following information:
\begin{description}
   \item[$\Delta_i$]: context of derived linear facts;
   \item[$\Gamma_i$]: context of derived persistent facts;
   \item[$\Xi_i$]: context of consumed linear facts;
   \item[$V_i$]: a value representing the aggregate application.
\end{description}

Since each application consumes $\Xi_i$ then the initial context $\Delta_N =
\Delta, \Xi_1, \dotsc, \Xi_n$. We now define the two main implications of the
theorem.

\begin{itemize}[leftmargin=*]
   \item Assume that $\Delta_N = \Delta_a, \Delta_b$, $\Delta_b =
   \Delta'_b, p_1$ and there is a frame $f = (\Delta_a, p_1; \Delta'_b; \cdot;
         p; \Omega; \cdot; \Upsilon)$.

   If $\ma{AG} \Gamma; \Delta_a, \Delta'_b; \Xi_N; \Gamma_{N1}; \Delta_{N1};
      p_1; \Omega; f; \lstack{P}; \Omega_N; \Delta, \Xi_1, \dotsc, \Xi_n;
      \Sigma \rightarrow \outsem$ (well-formed in relation to $T$) then:

   \begin{itemize}[leftmargin=\secondm]
      \item $n$ values $V_i$ for each implication ($\Sigma' = V_n :: \dots :: V_1 :: \Sigma$)
      \item $n$ aggregate applications are derived:\\
      $\done \Gamma; \Delta_N; \Xi_N, \Xi_1, \dotsc, \Xi_n; \Gamma_{N1},
      \Gamma_1, \dotsc, \Gamma_n; \Delta_{N1}, \Delta_1, \dotsc, \Delta_n;
(\lambda x. C)\Sigma', \Omega_N \rightarrow \outsem$
      \item $n$ $\mz$propositions for the $n$ aggregate matches:
      \begin{itemize}[leftmargin=\thirdm]
         \item $\mz \Gamma; \Xi_1 \rightarrow A$
         \item $\dots$
         \item $\mz \Gamma; \Xi_n \rightarrow A$
      \end{itemize}
      \item $n$ derivation implications for HLD: \\
      $\forall_{\Omega_x}.($ if $\dz \Gamma; \Delta, \Xi_{i+1}, \dotsc, \Xi_{n}; \Xi_N, \Xi_1,
            \dotsc, \Xi_i; \Gamma_{N1}, \Gamma_1, \dotsc, \Gamma_i; \Delta_{N1},
            \Delta_1, \dotsc, \Delta_i; \Omega_x \rightarrow \outsem$ then $\dz \Gamma; \Delta, \Xi_{i+1}, \dotsc, \Xi_{n}; \Xi_N, \Xi_1,
            \dotsc,
            \Xi_i; \Gamma_{N1}, \Gamma_1, \dotsc, \Gamma_{i-1}; \Delta_{N1},
            \Delta_1, \dotsc, \Delta_{i-1}; B, \Omega_x \rightarrow \outsem)$
   \end{itemize}

   \item If $\ma{AG} \Gamma; \Delta_N; \Xi_N; \Gamma_{N1}; \Delta_{N1}; \cdot; \Omega;
      \cdot; \lstack{P}; \Omega_N; \Delta, \Xi_1, \dotsc, \Xi_n; \Sigma \rightarrow \outsem$ (well-formed in relation to $T$) then:

   \begin{itemize}[leftmargin=\secondm]
      \item $n$ values $V_i$ for each implication ($\Sigma' = V_n :: \dotsb :: V_1 :: \Sigma$)
      \item $n$ aggregate applications are derived:\\
      $\done \Gamma; \Delta_N; \Xi_N, \Xi_1, \dotsc, \Xi_n; \Gamma_{N1},
      \Gamma_1, \dotsc, \Gamma_n; \Delta_{N1}, \Delta_1, \dotsc, \Delta_n;
(\lambda x. C)\Sigma', \Omega_N \rightarrow \outsem$

      \item $n$ $\mz$propositions for the $n$ aggregate matches:
      \begin{itemize}[leftmargin=\thirdm]
         \item $\mz \Gamma; \Xi_1 \rightarrow A$
         \item \dots
         \item $\mz \Gamma; \Xi_n \rightarrow A$
      \end{itemize}

      \item $n$ derivation implications for HLD: \\
      $\forall_{\Omega_x}.($ if $\dz \Gamma; \Delta, \Xi_{i+1}, \dotsc, \Xi_{n}; \Xi_N, \Xi_1,
            \dotsc, \Xi_i; \Gamma_{N1}, \Gamma_1, \dotsc, \Gamma_i; \Delta_{N1},
            \Delta_1, \dotsc, \Delta_i; \Omega_x \rightarrow \outsem$ then $\dz \Gamma; \Delta, \Xi_{i+1}, \dotsc, \Xi_{n}; \Xi_N, \Xi_1,
            \dotsc,
            \Xi_i; \Gamma_{N1}, \Gamma_1, \dotsc, \Gamma_{i-1}; \Delta_{N1},
            \Delta_1, \dotsc, \Delta_{i-1}; B, \Omega_x \rightarrow \outsem)$

   \end{itemize}

\end{itemize}
   
\end{theorem}
\begin{proof}
By mutual induction, first on either the size of $\Delta'_b$ (second argument of
the linear continuation frame) or $\Gamma'$ (second argument of the
persistent frame in $\lstack{P}$) and then on the size of $\lstack{C},
\lstack{P}$.  We only show how to prove the first implication since the
second implication is proven in a similar way.

$\ma{AG} \Gamma; \Delta_a, \Delta'_b; \Xi_N; \Gamma_{N1}; \Delta_{N1}; p_1;
\Omega; f; \lstack{P}; \Omega_N; \Delta, \Xi_1, \dotsc, \Xi_n; \Sigma \rightarrow \outsem$ \hfill (1) assumption\\

By applying Lemma~\ref{thm:comprehension_body_match} to (1), we get:

\begin{itemize}[leftmargin=*]
   \item Failure:
   
   $\done \Gamma; \Delta_N; \Xi_N; \Gamma_{N1}; \Delta_{N1}; (\lambda x.
         C)\Sigma, \Omega_N
      \rightarrow \outsem$ \hfill (2) from lemma, thus $n = 0$\\
   
   \item Success with no backtracking to frames of stack $\lstack{C}$ or
   $\lstack{P}$:
   
      $\mz \Gamma; \Xi_1 \rightarrow A$ \hfill (2) from lemma \\

      $\Xi_1 = \Xi'_1, p_1$ \hfill (3) from the well-formedness of (1) \\
      $f = (\Delta_a, p_1; \Delta'_b; \cdot; p; \Omega; \cdot; \Upsilon)$ \\

      $\ma{AG} \Gamma; \Delta, \Xi_2, \dotsc, \Xi_n; \Xi_N; \Gamma_{N1};
            \Delta_{N1}; p_1, \Xi'_1; \cdot; \lstack{C'}, f; \lstack{P};
            \Omega_N; \Delta_N; \Sigma \rightarrow \outsem$ \\
      \dots \hfill (4) from lemma (1) \\

      $f' = (\Delta_a, p_1 - \Xi_1; \Delta_b - \Xi_1; \cdot; p; \Omega; \cdot;
            \Upsilon)$ \\

      $\da{AG} \Gamma; \Delta, \Xi_2, \dotsc, \Xi_n; \Xi_N, \Xi_1; \Gamma_{N1}; \Delta_{N1}; B; f';
      \lstack{P'}; \Omega_N; V_1 :: \Sigma \rightarrow \outsem$ \\
      \dots \hfill (5) using Corollary~\ref{thm:agg_match_to_derivation} on (4) \\

      $\da{AG} \Gamma; \Delta, \Xi_2, \dotsc, \Xi_n; \Xi_N, \Xi_1; \Gamma_{N1}, \Gamma_1; \Delta_{N1}, \Delta_1;
            \cdot; f'; \lstack{P'}; \Omega_N; V_1 :: \Sigma \rightarrow \outsem$
      \\ \dots \hfill (6) applying Theorem~\ref{thm:aggregate_derivation} on (5)

      $\forall_{\Omega_x}. ($ if $\dz \Gamma; \Delta, \Xi_2, \dotsc, \Xi_n; \Xi_N, \Xi_1;
            \Gamma_{N1}, \Gamma_1; \Delta_{N1}, \Delta_1; \Omega_x \rightarrow
            \outsem$ then \\ \hspace*{0.5cm} $\dz \Gamma;
            \Delta, \Xi_2, \dotsc, \Xi_n; \Xi_N, \Xi_1; \Gamma_{N1}; \Delta_{N1}; B, \Omega_x
            \rightarrow \outsem)$ \hfill (7) from
      Theorem~\ref{thm:aggregate_derivation} on (5) \\

      $\conta{AG} \Gamma; \Delta, \Xi_2, \dotsc, \Xi_n; \Xi_N, \Xi_1; \Gamma_{N1},
         \Gamma_1; \Delta_{N1}, \Delta_1; f'; \lstack{P'}; \Omega_N; V_1 ::
         \Sigma \rightarrow \outsem$\\ \dots \hfill (8) inversion of (6) \\
        
        By inverting (8) we either fail (thus $n = 1$) or we get a new match.
        For the latter case, we apply mutual induction to get the remaining $n -
        1$ aggregate values.
      
   \item Success with backtracking to the linear continuation frame of stack $\lstack{C}$:
      
      $\mz \Gamma; \Xi_1 \rightarrow A$ \hfill (2) from lemma \\

      $f = (\Delta_a, p_1; \Delta'_b; \cdot; p; \Omega; \cdot; \Upsilon)$ \hfill (3) frame to backtrack to \\
      turns into $f' = (\Delta_a, p_1, \Delta'''_b, p_2; \Delta''_b; \cdot; p; \Omega; \cdot; \Upsilon)$ \hfill (4) resulting frame \\

      $\ma{AG} \Gamma; \Delta, \Xi_2, \dotsc, \Xi_n; \Xi_N; \Gamma_{N1};
         \Delta_{N1}; p_2, \Xi'_1; \cdot; \lstack{C'}, f'; \lstack{P}; \Omega_N;
         \Delta_N; \Sigma \rightarrow \outsem$\\ \dots \hfill (5) from lemma (1) \\
      
      Use the same approach as the case with no backtracking.
      
   \item Success with backtracking to a persistent continuation frame of stack
   $\lstack{P}$:

      $\mz \Gamma; \Xi_1 \rightarrow A$ \hfill (2) from lemma \\

      $f = [\Gamma''_1, p_2, \Gamma''_2; \Delta_N; \cdot; \bang p; \Omega; \cdot; \Upsilon]$ \hfill (4) from theorem \\
      turns into $f' = [\Gamma''_2; \Delta_N; \cdot; \bang p; \Omega; \cdot;
      \Upsilon]$ \hfill (5) from theorem \\

      $\ma{AG} \Gamma; \Delta, \Xi_2, \dotsc, \Xi_n; \Xi_N; \Gamma_{N1};
         \Delta_{N1}; \Xi_1; \cdot; \lstack{C'}; \lstack{P'}, f', \lstack{P''};
         \Omega_N; \Delta_N; \Sigma \rightarrow
         \outsem$ \\ \dots \hfill (6) from theorem \\
         
      Use the same approach as the case with no backtracking.
      
\end{itemize}
\end{proof}




This last theorem proves that from a certain initial continuation stack, we are
able to apply the aggregate multiple times (until the stack is exhausted). The
results of the theorem allows us to rebuild the proof tree in HLD since we get
the HLD matching and derivation propositions. What remains to be done is to
prove that we do the same for an empty continuation stack.

\begin{lemma}[All aggregates]\label{thm:aggregates}
Consider a triplet $T = A; \Gamma; \Delta_{N}$ and an aggregate $AG =
\aggz{A}{B}{C}$. Assume that there exists $n \geq 0$ applications of $AG$
where the $i_{th}$ application is related to the following information:
\begin{description}
   \item[$\Delta_i$]: context of derived linear facts;
   \item[$\Gamma_i$]: context of derived persistent facts;
   \item[$\Xi_i$]: context of consumed linear facts;
   \item[$V_i$]: a value representing the aggregate application.
\end{description}

Since each application consumes $\Xi_i$ then the initial context $\Delta_N =
\Delta, \Xi_1, \dotsc, \Xi_n$.

If $\ma{AG} \Gamma; \Delta, \Xi_1, \dotsc, \Xi_n;
\Xi_N; \Gamma_{N1}; \Delta_{N1}; \cdot; A; \cdot; \cdot; \Omega_N;
\Delta, \Xi_1, \dotsc, \Xi_n; \cdot \rightarrow \outsem$ (well-formed in
relation to $T$) then:

\begin{itemize}[leftmargin=*]
   \item $n$ values $V_i$ for each implication ($\Sigma = V_n :: \dotsb :: V_1
         :: \cdot$)
   \item $n$ aggregates are derived:\\
   $\done \Gamma; \Delta_N; \Xi_N, \Xi_1, \dotsc, \Xi_n; \Gamma_{N1},
   \Gamma_1, \dotsc, \Gamma_n; \Delta_{N1}, \Delta_1, \dotsc, \Delta_n; (\lambda
         x. C)\Sigma, \Omega_N \rightarrow \outsem$
   \item $n$ $\mz$propositions for the $n$ aggregate matches:
   \begin{itemize}[leftmargin=\secondm]
      \item $\mz \Gamma; \Xi_1 \rightarrow A$
      \item $\dots$
      \item $\mz \Gamma; \Xi_n \rightarrow A$
   \end{itemize}
   \item $n$ derivation implications for HLD: \\
   $\forall_{\Omega_x}.($ if $\dz \Gamma; \Delta, \Xi_{i+1}, \dotsc, \Xi_n; \Xi_N, \Xi_1,
         \dotsc, \Xi_i; \Gamma_{N1}, \Gamma_1, \dotsc, \Gamma_i; \Delta_{N1},
         \Delta_1, \dotsc, \Delta_i; \Omega_x \rightarrow \outsem$ then $\dz \Gamma; \Delta, \Xi_{i+1}, \dotsc, \Xi_n; \Xi_N, \Xi_1,
         \dotsc,
         \Xi_i; \Gamma_{N1}, \Gamma_1, \dotsc, \Gamma_{i-1}; \Delta_{N1},
         \Delta_1, \dotsc, \Delta_{i-1}; B, \Omega_x \rightarrow \outsem)$
\end{itemize}
\end{lemma}

\begin{proof}
Apply Lemma~\ref{thm:aggregate_body_match} to get two sub-cases:
   
\begin{itemize}[leftmargin=*]
   \item Match fails:
   
   $\done \Gamma; \Delta_N; \Xi_N; \Gamma_{N1}; \Delta_{N1}; (\lambda x. C)\cdot \Omega_N
   \rightarrow \outsem$\\
   \dots \hfill (1) no aggregate application was possible ($n = 0$)\\
   
   \item Match succeeds:
   
   $\ma{AG} \Gamma; \Xi_2, \dotsc, \Xi_n; \Xi_N; \Gamma_{N1}; \Delta_{N1};
\Xi_1; \cdot; \lstack{C}; \lstack{P}; \Omega_N; \Delta_N; \cdot \rightarrow \outsem$
   
   \dots \hfill (1) result from Lemma~\ref{thm:aggregate_body_match}
   
   $\mz \Gamma; \Xi_1 \rightarrow A$
   \hfill (2) also from Lemma~\ref{thm:aggregate_body_match}
   
   $\da{AG} \Gamma; \Delta, \Xi_2, \dotsc, \Xi_n; \Xi_N, \Xi_1; \Gamma_{N1}; \Delta_{N1}; B; \lstack{C'};
\lstack{P'}; \Omega_N; V_1 :: \cdot \rightarrow \outsem$
   
   \dots \hfill (3) applying Corollary~\ref{thm:agg_match_to_derivation} on (1)
   
   $\da{AG} \Gamma; \Delta, \Xi_2, \dotsc, \Xi_n; \Xi_N, \Xi_1; \Gamma_{N1}, \Gamma_1; \Delta_{N1}, \Delta_1;
   \cdot; \lstack{C'}; \lstack{P'}; \Omega_N; V_1 :: \cdot \rightarrow \outsem$
   
   \dots \hfill (4) using Theorem~\ref{thm:aggregate_derivation} on (3)\\
   
   $\forall_{\Omega_x}. ($ if $\dz \Gamma; \Delta, \Xi_2, \dotsc, \Xi_n; \Xi_N,
         \Xi_1; \Gamma_{N1}, \Gamma_1; \Delta_{N1}, \Delta_1; \Omega_x
         \rightarrow \outsem$ then
   
    \hspace*{0.5cm} $\dz \Gamma; \Delta, \Xi_2, \dotsc, \Xi_n; \Xi_N, \Xi_1; \Gamma_{N1};
    \Delta_{N1}; B, \Omega_x \rightarrow \outsem)$ \\ \dots \hfill (5)
   from the theorem applied in (4)\\
   
   $\conta{AG} \Gamma; \Delta, \Xi_2, \dotsc, \Xi_n; \Xi_N, \Xi_1; \Gamma_{N1},
   \Gamma_1; \Delta_{N1}, \Delta_1; \lstack{C'}; \lstack{P'}; \Omega_N; V_1 ::
   \cdot \rightarrow \outsem$
   
   \dots \hfill (6) inversion of (5)\\
   
   Invert (6) to get either $n = 1$ application of the aggregate or apply
   Theorem~\ref{thm:multiple_aggregate_derivation} to the inversion to get the remaining $n-1$. 
\end{itemize}
\end{proof}
\fi
