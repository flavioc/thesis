The first definition defines the equality between two multi-sets of terms.  Two
multi-sets $A$ and $B$ are equal, $\feq{A}{B}$, when there a substitution
$\theta$ that allows $A\theta$ and $B\theta$ to have the same
constituent atoms. Each continuation frame builds a substitution $\Psi$ which
can be used to determine if two terms are equal.

\[
\infer[\equiv p]
{\feq{p, A}{q, B}}{\feq{A}{B} & p\theta \triangleq q\theta}
\tab
\infer[\equiv \bang p]
{\feq{\bang p, A}{\bang p, B}}{\feq{A}{B} & \bang p \theta \triangleq
\bang q\theta}
\tab
\infer[\equiv \one~L]
{\feq{\one, A}{B}}{\feq{A}{B}}
\]

\[
\infer[\equiv \one~R]
{\feq{A}{\one, B}}{\feq{A}{B}}
\tab
\infer[\equiv \cdot]
{\feq{\cdot}{\cdot}} {}
\tab
\infer[\equiv \otimes~L]
{\feq{A \otimes B, C}{D}}{\feq{A, B, C}{D}}
\tab
\infer[\equiv \otimes~R]
{\feq{A}{B \otimes C, D}}{\feq{A}{B, C, D}}
\]

\begin{theorem}[Match equivalence]

If two multi-sets are equivalent, $\feq{A_1, \dotsc, A_n}{B_1, \dotsc, B_m}$,
and we can match $A_1 \otimes \dotsb \otimes A_n$ in HLD such that
$\mz{\Gamma}{\Delta}{(A_1 \otimes \dotsb \otimes A_n)\theta}$ then
$\mz{\Gamma}{\Delta}{(B_1 \otimes \dotsb \otimes B_m)\theta}$ is also true.

\end{theorem}
\begin{proof}
By straightforward induction on the first assumption.
\end{proof}

\begin{definition}[Split contexts]
$split(\Omega)$ is defined as $split(\Omega) = times(flatten(\Omega))$, where:

\begin{align}
flatten(\cdot) = \cdot \\
flatten(\one, \Omega) = flatten(\Omega) \\
flatten(A \otimes B, \Omega) = flatten(A), flatten(B), flatten(\Omega) \\
flatten(p, \Omega) = p, flatten(\Omega) \\
flatten(\bang p, \Omega) = \bang p, flatten(\Omega)
\end{align}

And $times(A_1, \dotsc, A_n) = A_1 \otimes \dotsb \otimes A_n$.
\end{definition}

\begin{theorem}[Split equivalence]
$\feqa{split(\Omega)}{\Omega}{\theta}{\theta}$.
\end{theorem}
\begin{proof}
Induction on the structure of $\Omega$.
\end{proof}

