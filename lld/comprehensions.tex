The most intricate part of the derivation process is processing comprehensions
and aggregates. For both of them, we need to perform as many derivations as the
database allows, therefore we need to deterministically check the contents of
the database until no more derivations are possible.  The matching process is
then similar to the process used for matching the body of the rule presented in
Section~\ref{sec:lld_body_match}, however we use two continuation stacks,
$\lstack{C}$ and $\lstack{P}$. In $\lstack{P}$, we put all the initial
persistent frames and in $\lstack{C}$ we put the first linear frame and then
everything else.

In order to reuse the stacks $\lstack{C}$ and $\lstack{P}$, we need to update them by removing all
the frames in $\lstack{C}$ pushed after the first linear continuation frame.  If we tried
to use those frames, we would assumed that the linear facts used by the other
frames were still in the database, but that is not true because they have been
consumed during the first application of the comprehension.  For example, if the
body is $\bang \mathtt{a(X)} \otimes \mathtt{b(X)} \otimes \mathtt{c(X)}$ and
the continuation stack has three frames (one per fact), we cannot backtrack to
the frame of $\mathtt{c(X)}$ because, at that point, the matching process was
assuming that the previous \texttt{b(X)} linear fact was still available.
Moreover, we also need to remove the consumed linear facts from the frames of
\texttt{b(X)} and $\bang$\texttt{a(X)} in order to make the stack fully
consistent with the new database. We will see later on how to do that.

The match judgment $\mc{AB} \Gamma; \Delta; \Xi_N; \Gamma_{N1}; \Delta_{N1};
\Xi; \Omega; \lstack{C}; \lstack{P}; \Omega_N; \Delta_N \rightarrow \outsem$ is
as follows:

\begin{enumerate}

   \item[$\Delta$] multi-set of linear facts remaining up to this point in the
   matching process;

   \item[$\Xi_N$] multi-set of linear facts used during the matching process of
   the body of the rule and all the previous comprehensions;

   \item[$\Gamma_{N1}$] set of persistent facts derived up to this point in the
   head of the rule;

   \item[$\Delta_{N1}$] multi-set of linear facts derived up to this point in
   the head of the rule;

   \item[$\Xi$] multi-set of linear facts consumed up to this point;

   \item[$\Omega$] ordered list of terms that need to be matched for the
   comprehension to be applied;

   \item[$\lstack{C}$] continuation stack that contains both linear and
   persistent frames. The first frame must be linear;

   \item[$\lstack{P}$] initial part of the continuation stack with only persistent
   frames;

   \item[$\Omega_N$] ordered list of remaining terms of the head of the rule to
   be derived;

   \item[$\Delta_N$] multi-set of linear facts that were still available after
   matching the body of the rule and all the previous comprehensions. Note that
   $\Delta, \Xi = \Delta_N$.

\end{enumerate}

\subsubsection{Linear fact expressions}

The following five inference rules deal with the case when there is a linear
fact expression in the body of the comprehension.


\[
\infer[\mc{AB} p~\m{first}]
{\mc{AB} \Gamma; \Delta, p_1, \Delta''; \Xi_N; \Gamma_{N1}; \Delta_{N1}; \cdot; p,
   \Omega; \cdot; \cdot; \Omega_N; \Delta_N \rightarrow \outsem}
{
   \begin{gathered}
      p_1, \Delta'' \prec p \\
      \mc{AB} \Gamma; \Delta, \Delta''; \Xi_N; \Gamma_{N1};
      \Delta_{N1}; \Xi, p_1; \Omega; (\Delta, p_1; \Delta''; \cdot; p; \Omega;
            \cdot; \cdot); \cdot; \Omega_N; \Delta_N \rightarrow \outsem
   \end{gathered}}
\]

\[
\infer[\mc{AB} p~\m{on}~q]
{\mc{AB} \Gamma; \Delta, p_1, \Delta''; \Xi_N; \Gamma_{N1}; \Delta_{N1}; \Xi; p,
   \Omega; f, \lstack{C}; \lstack{P}; \Omega_N; \Delta_N \rightarrow \outsem}
{
   \begin{gathered}
      p_1, \Delta'' \prec p \\
      f = (\Delta_{old}; \Delta'_{old}; \Xi_{old}; q; \Omega_{old}; \Lambda; \Upsilon) \\
      f' = (\Delta, p_1; \Delta''; \Xi; p; \Omega; q, \Lambda; \Upsilon) \\
      \mc{AB} \Gamma; \Delta, \Delta''; \Xi_N; \Gamma_{N1}; \Delta_{N1}; \Xi, p_1; \Omega;
      f', f, \lstack{C}; \lstack{P}; \Omega_N; \Delta_N \rightarrow \outsem
   \end{gathered}
}
\]


\[
\infer[\mc{AB} p~\m{on}~\bang q~\lstack{C}]
{\mc{AB} \Gamma; \Delta, p_1, \Delta''; \Xi_N; \Gamma_{N1}; \Delta_{N1}; \Xi; p,
   \Omega; f, \lstack{C}; \lstack{P}; \Omega_N; \Delta_N \rightarrow \outsem}
{
   \begin{gathered}
      p_1, \Delta'' \prec p \\
      f = [\Gamma_{old}; \Delta_{old}; \Xi_{old}; q; \Omega_{old}; \Lambda; \Upsilon] \\
      f' = (\Delta, p_1; \Delta''; \Xi; p; \Omega; \Lambda; q, \Upsilon) \\
      \mc{AB} \Gamma; \Delta, \Delta''; \Xi_N; \Gamma_{N1}; \Delta_{N1}; \Xi,
      p_1; \Omega; f', f, \lstack{C}; \lstack{P}; \Omega_N; \Delta_N \rightarrow \outsem
   \end{gathered}
}
\]


\[
\infer[\mc{AB} p~\m{on}~\bang q~\lstack{P}]
{\mc{AB} \Gamma; \Delta, p_1, \Delta''; \Xi_N; \Gamma_{N1}; \Delta_{N1}; \cdot; p,
   \Omega; \cdot; f, \lstack{P}; \Omega_N; \Delta_N \rightarrow \outsem}
{
   \begin{gathered}
      p_1, \Delta'' \prec p \\
      f = [\Gamma_{old}; \Delta_N; \cdot; q; \Omega_{old}; \cdot; \Upsilon]\\
      \Delta_N = \Delta, p_1, \Delta'' \\
      f' = (\Delta, p_1; \Delta''; \cdot; p; \Omega; \cdot; q, \Upsilon) \\
      \mc{AB} \Gamma; \Delta, \Delta''; \Xi_N; \Gamma_{N1}; \Delta_{N1}; p_1; \Omega;
         f'; f, \lstack{P};
         \Omega_N; \Delta_N \rightarrow \outsem
   \end{gathered}
}
\]


\[
\infer[\mc{AB} p~\m{fail}]
{\mc{AB} \Gamma; \Delta; \Xi_N; \Gamma_{N1}; \Delta_{N1}; \Xi; p, \Omega;
   \lstack{C}; \lstack{P}; \Omega_N; \Delta_N \rightarrow \outsem}
{\Delta \npreceq p & \contc \Gamma; \Delta_N; \Xi_N; \Gamma_{N1}; \Delta_{N1};
   \lstack{C}; \lstack{P}; \Omega_N \rightarrow \outsem}
\]


\subsubsection{Persistent fact expressions}

The inference rules for matching persistent fact expressions are similar to the
previous ones.


\[
\infer[\mc{AB} \bang p~\m{first}]
{\mc{AB} \Gamma, p_1, \Gamma''; \Delta_N; \Xi_N; \Gamma_{N1}; \Delta_{N1};
   \cdot; \bang p, \Omega; \cdot; \cdot; \Omega_N; \Delta_N \rightarrow \outsem}
{\begin{gathered}
   p_1, \Gamma'' \prec \bang p \\
   f = [\Gamma''; \Delta_N; \cdot; \bang p; \cdot; \Omega; \cdot; \cdot] \\
   \mc{AB} \Gamma, p_1, \Gamma''; \Delta_N; \Xi_N; \Gamma_{N1}; \Delta_{N1};
   \cdot; \Omega; \cdot; f;
   \Omega_N; \Delta_N \rightarrow \outsem
\end{gathered}
}
\]

\[
\infer[\mc{AB} \bang p~\m{on}~\bang q~\lstack{P}]
{\mc{AB} \Gamma, p_1, \Gamma''; \Delta_N; \Xi_N; \Gamma_{N1}; \Delta_{N1};
   \cdot; \bang p, \Omega; \cdot; f, \lstack{P}; \Omega_N; \Delta_N \rightarrow \outsem}
{
   \begin{gathered}
      p_1, \Gamma'' \prec \bang p \\
      f = [\Gamma_{old}; \Delta_N; \cdot; \bang q; \Omega_{old}; \cdot; \Upsilon] \\
      f' = [\Gamma''; \Delta_N; \cdot; \bang p; \cdot; \Omega; \cdot; q,
      \Upsilon] \\
      \mc{AB} \Gamma, p_1, \Gamma''; \Delta_N; \Xi_N; \Gamma_{N1}; \Delta_{N1};
      \cdot; \Omega; f', f, \lstack{P}; \Omega_N; \Delta_N \rightarrow \outsem
   \end{gathered}}
\]


\[
\infer[\mc{AB} \bang p~\m{on}~\bang q~\lstack{C}]
{\mc{AB} \Gamma, p_1, \Gamma''; \Delta; \Xi_N; \Gamma_{N1}; \Delta_{N1}; \Xi;
   \bang p, \Omega; f, \lstack{C}; \lstack{P}; \Omega_N; \Delta_N \rightarrow \outsem}
{
   \begin{gathered}
      p_1, \Gamma'' \prec \bang p \\
      f = [\Gamma_{old}; \Delta_{old}; \Xi_{old}; \bang q; \Omega_{old};
      \Lambda; \Upsilon] \\
      f' = [\Gamma''; \Delta; \Xi; \bang p; \cdot; \Omega; \Lambda; q, \Upsilon] \\
      \mc{AB} \Gamma,
      p_1, \Gamma''; \Delta; \Xi_N; \Gamma_{N1}; \Delta_{N1}; \Xi; \Omega; f', f,
      \lstack{C}; \lstack{P}; \Omega_N; \Delta_N \rightarrow \outsem
   \end{gathered}
}
\]


\[
\infer[\mc{AB} \bang p~\m{on}~q~\lstack{C}]
{\mc{AB} \Gamma, p_1, \Gamma''; \Delta; \Xi_N; \Gamma_{N1}; \Delta_{N1}; \Xi;
   \bang p, \Omega; f, \lstack{C}; \lstack{P}; \Omega_N; \Delta_N \rightarrow \outsem}
{
   \begin{gathered}
      p_1, \Gamma'' \prec \bang p \\
      f = (\Delta_{old}; \Delta'_{old}; \Xi_{old}; q; \Omega_{old}; \Lambda; \Upsilon) \\
      f' = [\Gamma''; \Delta; \Xi; \bang p; \cdot; \Omega; \Lambda, q; \Upsilon] \\
      \mc{AB} \Gamma, p_1, \Gamma''; \Delta; \Xi_N; \Gamma_{N1}; \Delta_{N1}; \Xi; \Omega;
      f', f, \lstack{C}; \lstack{P}; \Omega_N; \Delta_N \rightarrow \outsem
   \end{gathered}
}
\]

\[
\infer[\mc{AB} \bang p~\m{fail}]
{\mc \Gamma; \Delta; \Xi_N; \Gamma_{N1}; \Delta_{N1}; \Xi; \bang p, \Omega;
   \lstack{C}; \lstack{P}; \Omega_N; \Delta_N \rightarrow \outsem}
{\Gamma \npreceq \bang p & \contc{AB} \Gamma; \Delta_N; \Xi_N; \Gamma_{N1};
   \Delta_{N1}; \lstack{C}; \lstack{P}; \Omega_N \rightarrow \outsem}
\]


\subsubsection{Deconstruct body}

The inference rules for deconstructing the body of the comprehension, including
the connectives $\otimes$ and $\one$, are shown next.


\[
\infer[\mc \otimes]
{\mc \Delta; \Xi_N; \Delta_{N1}; \Xi; X \otimes Y, \Omega; C; P; AB; \Omega_N; \Delta_N \rightarrow \Xi'; \Delta'}
{\mc \Delta; \Xi_N; \Delta_{N1}; \Xi; X, Y, \Omega; C; P; AB; \Omega_N; \Delta_N \rightarrow \Xi'; \Delta'}
\]


\subsubsection{Successful match}

Finally, when there is no more terms to match, we initiate the process of the
updating the continuation stack with $\fix{AB}$ followed by deriving the head of the
comprehension.

\[
\infer[\mc{AB} \m{end}]
{\mc{AB} \Gamma; \Delta; \Xi_N; \Gamma_{N1}; \Delta_{N1}; \Xi; \cdot; \lstack{C};
   \lstack{P}; \Omega_N; \Delta_N \rightarrow \outsem}
{\fix{AB} \Gamma; \Delta; \Xi_N; \Gamma_{N1}; \Delta_{N1}; \Xi; \lstack{C}; \lstack{P}; \Omega_N; \Delta_N
   \rightarrow \outsem}
\]


\subsubsection{Continuation stack update}

As we said before, to update the continuation stacks, we need remove to all the
frames except the first linear frame and remove the consumed linear facts from
the remaining frames so that they are still valid for the next application of
the comprehension.  The judgment that updates the stack has the form $\fix{AB}
\Gamma; \Delta; \Xi_N; \Gamma_{N1}; \Delta_{N1}; \Xi; \lstack{C}; \lstack{P}; \Omega_N; \Delta_N
\rightarrow \outsem$, where:

\begin{enumerate}
   \item[$\Delta$] multi-set of linear facts remaining after matching the last
   comprehension;
   \item[$\Xi_N$] multi-set of linear facts used during the matching process of
   the body of the rule and all the previous comprehensions;
   \item[$\Gamma_{N1}$] set of persistent facts derived by the head of the rule
   and all the previous comprehensions;
   \item[$\Delta_{N1}$] multi-set of linear facts derived by the head of the
   rule and all the previous comprehensions;
   \item[$\Xi$] multi-set of linear facts consumed by the previous application
   of the comprehension;
   \item[$\lstack{C}, \lstack{P}$] continuation stacks for the comprehension;
   \item[$\Omega_N$] ordered list of remaining terms of the head of the rule to
   be derived;
   \item[$\Delta_N$] multi-set of linear facts that were still available after
   matching the body of the rule and all the previous comprehensions.
\end{enumerate}

\subsubsection{Remove linear continuation frames}

To remove all linear continuation frames except the first one, we simply go
through all the frames in the stack $\lstack{C}$ with $\fix{AB} \m{more}$
until only one frame remains. This frame is then updated using $\fix{AB} \m{end~linear}$ by removing the linear facts $\Xi$ consumed
during the last application of the comprehension.

\[
\infer[\fix{\compsz{A}{B}} \m{end~linear}]
{\fix{\compsz{A}{B}} \Gamma; \Delta; \Xi_N; \Gamma_{N1}; \Delta_{N1}; \Xi; f; \lstack{P};  \Omega_N; \Delta_N \rightarrow \outsem}
{
   \begin{gathered}
      \strans \Xi; \lstack{P}; \lstack{P'} \\
      f = (\Delta_x; \Delta''; \cdot; p; \Omega; \cdot; \Upsilon) \\
      f' = (\Delta_x - \Xi; \Delta'' - \Xi; \cdot; p; \Omega; \cdot;
            \Upsilon) \\
      \dc{\compsz{A}{B}} \Gamma; \Delta; \Xi_N, \Xi; \Gamma_{N1};
      \Delta_{N1}; B; f' ; \lstack{P'} ; \Omega_N \rightarrow \outsem
   \end{gathered}
}
\]

\[
\infer[\fix{AB} \m{more}]
{\fix{AB} \Gamma; \Delta; \Xi_N; \Gamma_{N1}; \Delta_{N1}; \Xi; \_, f, \lstack{C};
   \lstack{P}; \Omega_N;
   \Delta_N \rightarrow \outsem}
{\fix{AB} \Gamma; \Delta; \Xi_N; \Gamma_{N1}; \Delta_{N1}; \Xi; f, \lstack{C};
   \lstack{P}; \Omega_N;
   \Delta_N \rightarrow \outsem}
\]

\[
\infer[\fix{\compsz{A}{B}} \m{end~empty}]
{\fix{\compsz{A}{B}} \Gamma; \Delta; \Xi_N; \Gamma_{N1}; \Delta_{N1}; \Xi; \cdot;
   \lstack{P}; \Omega_N; \Delta_N \rightarrow \outsem}
{\begin{gathered}
   \strans \Xi; \lstack{P}; \lstack{P'} \\
   \dc{\compsz{A}{B}} \Gamma; \Delta ; \Xi_N, \Xi; \Gamma_{N1};
      \Delta_{N1}; B; \cdot ; \lstack{P'} ; \Omega_N \rightarrow \outsem
 \end{gathered}
}
\]


\subsubsection{Transform persistent continuation frames}

Finally, we also need to subtract the consumed facts in $\Xi$ from the second
argument of every persistent continuation frame.  We have an auxiliary judgment
$\strans \Xi; \lstack{P}; \lstack{P'}$ where $\Xi$ is the multi-set of consumed
linear facts, $\lstack{P}$ is the persistent continuation stack and
$\lstack{P'}$ is the stack $\lstack{P}$ after the update.

\[
\infer[\strans]
{\strans \Xi; [\Gamma'; \Delta_N; \cdot; \bang p; \Omega; \cdot; \Upsilon], P; [\Gamma'; \Delta_N - \Xi; \cdot; \bang p, \Omega; \cdot; \Upsilon], P'}
{\strans \Xi; P; P'}
\]

\[
\infer[\strans \m{end}]
{\strans \Xi; \cdot; \cdot}
{\strans \Xi; \cdot; \cdot}
\]


\subsubsection{Comprehension continuation}

If the comprehension match fails, we need to backtrack to the previous frame and
restore the matching process at that point. The judgment used to backtrack is
similar to the one presented in Section~\ref{sec:lld_match_cont} and has the
form $\contc{AB} \Gamma; \Delta_N; \Xi_N; \Delta_{N1}; \lstack{C}; \lstack{P}; \Omega_N
\rightarrow \outsem$, where:

\begin{enumerate}
   \item[$\Delta_N$] multi-set of linear facts still available after matching
   the body of the rule;
   \item[$\Xi_N$] multi-set of linear facts used to match the body of the rule
   and all the previous comprehensions;
   \item[$\Gamma_{N1}$] set of persistent facts derived by the head of the rule
   and all the previous comprehensions;
   \item[$\Delta_{N1}$] multi-set of linear facts derived by the head of the
   rule and all the previous comprehensions;
   \item[$\lstack{C}, \lstack{P}$] continuation stacks;
   \item[$\Omega_N$] ordered list of remaining terms of the head of the rule to
   be derived.
\end{enumerate}

\paragraph{Using the $\lstack{C}$ stack}

The following 4 inference rules use the $\lstack{C}$ stack, the stack where the
first continuation frame is linear, to perform backtracking.

{\footnotesize
\[
\infer[\contc \m{next}~C~p]
{\contc \Gamma; \Delta_N; \Xi_N; \Gamma_{N1}; \Delta_{N1}; (\Delta; p_1, \Delta''; \Xi; p; \Omega; \Lambda; \Upsilon), C; P; AB; \Omega_N \rightarrow \Xi'; \Delta'; \Gamma'}
{\mc \Gamma; \Delta; \Xi_N; \Gamma_{N1}; \Delta_{N1}; \Xi; \Omega; (\Delta, p_1; \Delta''; \Xi; p; \Omega; \Lambda; \Upsilon), C; P; AB; \Omega_N; \Delta_N \rightarrow \Xi'; \Delta'; \Gamma'}
\]

\[
\infer[\contc \m{next}~C~\bang p]
{\contc \Gamma; \Delta_N; \Xi_N; \Gamma_{N1}; \Delta_{N1}; [p_1, \Gamma'; \Delta; \Xi; \bang p; \Omega; \Lambda; \Upsilon], C; P; AB; \Omega_N \rightarrow \Xi'; \Delta'; \Gamma'}
{\mc \Gamma; \Delta; \Xi_N; \Gamma_{N1}; \Delta_{N1}; \Xi; \Omega; [\Gamma'; \Delta; \Xi; \bang p; \Omega; \Lambda; \Upsilon], C; P; AB; \Omega_N; \Delta_N \rightarrow \Xi'; \Delta'; \Gamma'}
\]

\[
\infer[\contc \m{next}~C~\m{empty}~p]
{\contc \Gamma; \Delta_N; \Xi_N; \Gamma_{N1}; \Delta_{N1}; (\Delta; \cdot; \Xi; p; \Omega; \Lambda; \Upsilon), C; P; AB; \Omega_N \rightarrow \Xi'; \Delta'; \Gamma'}
{\contc \Gamma; \Delta_N; \Xi_N; \Gamma_{N1}; \Delta_{N1}; C; P; AB; \Omega_N \rightarrow \Xi'; \Delta'; \Gamma'}
\]

\[
\infer[\contc \m{next}~C~\m{empty}~\bang p]
{\contc \Gamma; \Delta_N; \Xi_N; \Gamma_{N1}; \Delta_{N1}; [\cdot; \Delta; \Xi; \bang p; \Omega; \Lambda; \Upsilon], C; P; AB; \Omega_N \rightarrow \Xi'; \Delta'; \Gamma'}
{\contc \Gamma; \Delta_N; \Xi_N; \Gamma_{N1}; \Delta_{N1}; C; P; AB; \Omega_N \rightarrow \Xi'; \Delta'; \Gamma'}
\]
}


\paragraph{Using the $\lstack{P}$ stack}

The following 2 inference rules use the $\lstack{P}$ stack instead, the stack where all
continuation frames are persistent.

{\footnotesize
\[
\infer[\contc \m{next}~P~\bang p]
{\contc \Gamma; \Delta_N; \Xi_N; \Gamma_{N1}; \Delta_{N1}; \cdot; [p_1, \Gamma'; \Delta_N; \cdot; \bang p; \Omega; \cdot; \Upsilon], P; AB; \Omega_N \rightarrow \Xi'; \Delta'; \Gamma'}
{\mc \Gamma; \Delta_N; \Xi_N; \Gamma_{N1}; \Delta_{N1}; \cdot; \Omega; \cdot; [\Gamma'; \Delta_N; \cdot; \bang p; \Omega; \cdot; \Upsilon], P; AB; \Omega_N; \Delta_N \rightarrow \Xi'; \Delta'; \Gamma'}
\]

\[
\infer[\contc \m{next}~P~\m{empty}~\bang p]
{\contc \Gamma; \Delta_N; \Xi_N; \Gamma_{N1}; \Delta_{N1}; \cdot; [\cdot; \Delta_N; \cdot; \bang p; \Omega; \cdot; \Upsilon], P; AB; \Omega_N \rightarrow \Xi'; \Delta'; \Gamma'}
{\contc \Gamma; \Delta_N; \Xi_N; \Gamma_{N1}; \Delta_{N1}; \cdot; P; AB; \Omega_N \rightarrow \Xi'; \Delta'; \Gamma'}
\]
}


\paragraph{Comprehension done}

If both the $\lstack{C}$ and $\lstack{P}$ stacks are empty, backtracking is impossible and the
comprehension is done. The remaining head terms in $\Omega$ are then derived

\[
\infer[\contc{AB} \m{end}]
{\contc{AB} \Gamma; \Delta_N; \Xi_N; \Gamma_{N1}; \Delta_{N1}; \cdot; \cdot;
   \Omega \rightarrow \outsem}
{\done \Gamma; \Delta_N; \Xi_N; \Gamma_{N1}; \Delta_{N1}; \Omega \rightarrow
   \outsem}
\]


\subsubsection{Comprehension Derivation}

After updating the continuation stacks, the head of the comprehension is
derived. The judgment has the form $\dc{AB} \Gamma; \Gamma_N; \Xi_N; \Gamma_1; \Delta_1;
\Omega; \lstack{C}; \lstack{P}; \Omega_N; \Delta_N \rightarrow \outsem$, where:

\begin{enumerate}
   \item[$\Delta_N$] multi-set of remaining linear facts that can be used for
   the next comprehension applications.
   \item[$\Xi_N$] multi-set of linear facts consumed both by the body of the rule
   and previous comprehension applications;
   \item[$\Gamma_1$] set of persistent facts derived by the head of the rule,
   previous comprehensions and current derivation;
   \item[$\Delta_1$] multi-set of linear facts derived by the head of the rule,
   previous comprehensions and current derivation;
   \item[$\Omega$] ordered list of terms to derive;
   \item[$\lstack{C}, \lstack{P}$] new continuation stacks;
   \item[$\Omega_N$] ordered list of remaining terms of the head of the rule to
   be derived;
\end{enumerate}

\[
\infer[\dc p]
{\dc \Gamma; \Xi; \Gamma_1; \Delta_1; p, \Omega; C; P; AB; \Omega_N; \Delta_N \rightarrow \Xi'; \Delta'; \Gamma'}
{\dc \Gamma; \Xi; \Gamma_1; \Delta_1, p; \Omega; C; P; AB; \Omega_N; \Delta_N \rightarrow \Xi'; \Delta'; \Gamma'}
\]

\[
\infer[\dc \bang p]
{\dc \Gamma; \Xi; \Gamma_1; \Delta_1; \bang p, \Omega; C; P; AB; \Omega_N; \Delta_N \rightarrow \Xi'; \Delta'; \Gamma'}
{\dc \Gamma; \Xi; \Gamma_1, p; \Delta_1; \Omega; C; P; AB; \Omega_N; \Delta_N \rightarrow \Xi'; \Delta'; \Gamma'}
\]

\[
\infer[\dc \otimes]
{\dc \Gamma; \Xi; \Gamma_1; \Delta_1; A \otimes B, \Omega; C; P; AB; \Omega_N; \Delta_N \rightarrow \Xi'; \Delta';\Gamma'}
{\dc \Gamma; \Xi; \Gamma_1; \Delta_1; A, B, \Omega; C; P; AB; \Omega_N; \Delta_N \rightarrow \Xi'; \Delta';\Gamma'}
\]

\[
\infer[\dc 1]
{\dc \Gamma; \Xi; \Gamma_1; \Delta_1; 1, \Omega; C; P; AB; \Omega_N; \Delta_N \rightarrow \Xi'; \Delta';\Gamma'}
{\dc \Gamma; \Xi; \Gamma_1; \Delta_1; \Omega; C; P; AB; \Omega_N; \Delta_N \rightarrow \Xi'; \Delta';\Gamma'}
\]


The last derivation rule is activated whenever $\Omega$ is empty. We use the
$\contc{AB}$judgment to reuse the continuation stack
and try the next combination for the comprehension.

\[
\infer[\dc \m{end}]
{\dc \Gamma; \Xi; \Gamma_1; \Delta_1; \cdot; C; P; AB; \Omega_N; \Delta_N \rightarrow \Xi'; \Delta'; \Gamma'}
{\contc \Gamma; \Delta_N; \Xi; \Gamma_1; \Delta_1; C; P; AB; \Omega_N \rightarrow \Xi'; \Delta'; \Gamma'}
\]


