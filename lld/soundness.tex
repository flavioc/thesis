
Now that we have presented both the HLD and LLD semantics, we are in position to
start building our soundness theorem.  The soundness theorem proves that if a
rule was successfully derived in the LLD semantics then it can also be derived
in the HLD semantics. Since the HLD semantics are so close to linear logic, we
prove that our language has a determined, correct, proof search behavior when
executing programs. However, the completeness theorem cannot be proven since LLD
lacks the non-determinism inherent in HLD.


First and foremost, we need to prove some auxiliary theorems and definitions
that will be used during the soundness theorem.

\subsection{Term equivalence}

The first definition defines the equality between two multi-sets of terms.  Two
multi-sets $A$ and $B$ are equal, $\feq{A}{B}$, when they have the same
constituent atoms.

\[
\infer[\equiv p]
{\feq{p, A}{p, B}}
{\feq{A}{B}}
\tab
\infer[\equiv \bang p]
{\feq{\bang p, A}{\bang p, B}}
{\feq{A}{B}}
\tab
\infer[\equiv 1~L]
{\feq{1, A}{B}}
{\feq{A}{B}}
\tab
\infer[\equiv 1~R]
{\feq{A}{1, B}}
{\feq{A}{B}}
\]

\[
\infer[\equiv \cdot]
{\feq{\cdot}{\cdot}}
{}
\tab
\infer[\equiv \otimes~L]
{\feq{A \otimes B, C}{D}}
{\feq{A, B, C}{D}}
\tab
\infer[\equiv \otimes~R]
{\feq{A}{B \otimes C, D}}
{\feq{A}{B, C, D}}
\]

\begin{theorem}[Match equivalence]
If two multi-sets are equivalent, $\feq{A_1, \dotsc, A_n}{B_1, \dotsc, B_m}$,
   and we can match $A_1 \otimes \dotsb \otimes A_n$ in HLD such that $\mz
   \Gamma ; \Delta \rightarrow A_1 \otimes \dotsb \otimes A_n$ then $\mz \Gamma;
   \Delta \rightarrow B_1 \otimes \dotsb \otimes B_m$ is also true.
\end{theorem}
\begin{proof}
By straightforward induction on the first assumption.
\end{proof}

\subsection{Well-formed continuation frames}

We now define the concept of a well-formed frame given initial linear and
persistent contexts and a term $A$ that needs to be matched.

\begin{definition}[Well-formed frame]

Consider a triplet $A; \Gamma; \Delta_{N}$ where $A$ is a term, $\Gamma$ is a
multi-set of persistent resources and $\Delta_{N}$ a multi-set of linear
resources. A frame $f$ is well-formed iff:

\begin{enumerate}
   \item Linear frame $f = (\Delta, p_1; \Delta'; \Xi_1, \dotsc, \Xi_m; p;
         \Omega_1, \dotsc, \Omega_n; \Lambda_1, \dotsc, \Lambda_m; \Upsilon_1,
         \dotsc, \Upsilon_k)$

   \begin{enumerate}
      \item $\feq{p, \Omega_1, \dotsc, \Omega_n, \Lambda_1, \dotsc, \Lambda_m,
         \Upsilon_1, \dotsc, \Upsilon_k}{A}$ (the remaining terms and already
               matched terms are equivalent to the initial body $A$);
      \item $\mz \Xi_1, \dotsc, \Xi_m \rightarrow \Lambda_1 \otimes \dotsb \otimes
      \Lambda_m$ and $\mz \Xi_i \rightarrow \Lambda_i$ for every $i$;

      \item $\Delta, \Delta', \Xi, p_1 = \Delta_{N}$ (available facts, candidate
            facts for $p$, consumed facts and the linear fact used for $p$,
            respectively, are the same as the initial $\Delta_{N}$);

      \item $\mz \Gamma; \cdot \rightarrow \Upsilon_1 \otimes \dotsb \otimes
      \Upsilon_k$ (past persistent facts can be matched with $\Gamma$).

   \end{enumerate}
   \item Persistent frame $f = [\Gamma'; \Delta; \Xi_1, \dotsc, \Xi_m; \bang p;
         \Omega_1, \dotsc, \Omega_n; \Lambda_1, \dotsc, \Lambda_m; \Upsilon_1,
         \dotsc, \Upsilon_k]$

      \begin{enumerate}
         \item $\feq{\bang p, \Omega_1, \dotsc, \Omega_n, \Lambda_1, \dotsc,
                     \Lambda_m, \Upsilon_1, \dotsc, \Upsilon_k}{A}$;
         \item $\mz \Xi_1, \dots, \Xi_m \rightarrow \Lambda_1 \otimes \dotsb \otimes
                     \Lambda_m$ and $\mz \Xi_i \rightarrow \Lambda_i$ for every $i$;
         \item $\Delta, \Xi = \Delta_{N}$;
         \item $\mz \Gamma; \cdot \rightarrow \bang p \otimes \Upsilon_1 \otimes
                     \dotsb \otimes \Upsilon_k$;
         \item $\Gamma' \subset \Gamma$ (remaining candidates are a subset of
                     $\Gamma$).
      \end{enumerate}
\end{enumerate}
\end{definition}


\begin{definition}[Well-formed stack]
A continuation stack $C$ is well-formed iff every frame is well-formed.
\end{definition}

Given the previous definitions, we can now define what it means for a matching
judgment to be well-formed.

\begin{definition}[Well-formed body match]

$\mo \Gamma; \Delta; \Xi; \Omega; H; C; R \rightarrow \Xi'; \Delta'; \Gamma'$ is well-formed in relation to a triplet $A; \Gamma; \Delta_{N}$ iff:

\begin{enumerate}
   \item $\Delta, \Xi = \Delta_{N}$
   \item $C$ is well-formed in relation to $A; \Gamma; \Delta_{N}$ and:
   \begin{enumerate}
      \item If $C = \cdot$
   
      $\feq{\Omega}{A}$.
   
      \item If $C = (\Delta_a, p_1; \Delta_b; \Xi''; p; \Omega'; \Lambda_1,
            \dotsc, \Lambda_m; \Upsilon_1, \dotsc, \Upsilon_k), C'$
   
      \begin{enumerate}
         \item $\feq{\Omega'}{\Omega}$;
         \item $p_1 \in \Xi$ and $\mz \Gamma; p_1 \rightarrow p$;
         \item $\Xi = \Xi'', p_1$;
         \item $\Delta = \Delta_a, \Delta_b$.
      \end{enumerate}
      \item If $C = [\Gamma'; \Delta''; \Xi''; \bang p; \Omega'; \Lambda_1,
      \dotsc, \Lambda_m; \Upsilon_1, \dotsc, \Upsilon_k], C'$
      \begin{enumerate}
         \item $\feq{\Omega}{\Omega'}$;
         \item $\Xi = \Xi''$;
         \item $\Delta = \Delta''$.
      \end{enumerate}
   \end{enumerate}
\end{enumerate}

\end{definition}

\begin{definition}[Well-formed comprehension match]
\leavevmode
\newline
$\mc \Gamma; \Delta; \Xi_N; \Gamma_{N1}; \Delta_{N1}; \Xi; \Omega; C; P;
\compsz{A}{B}; \Omega_N; \Delta_N \rightarrow \Xi'; \Delta'; \Gamma'$ is
well-formed in relation to a triplet $A; \Gamma; \Delta_{N}$ iff:

\begin{enumerate}
   \item $P$ is composed solely of persistent frames.
   \item $C$ is composed of either linear or persistent frames, but the first
   frame is linear.
   \item $\Delta, \Xi = \Delta_{N}$
   \item $C$ and $P$ are well-formed in relation to $A; \Gamma; \Delta_{N}$ and
   follow the same rules presented before in "Well-formed body match" as a stack
   $C, P$.
\end{enumerate}
\end{definition}

\begin{definition}[Well-formed aggregate match]
\leavevmode
\newline
$\ma \Gamma; \Delta; \Xi_N; \Gamma_{N1}; \Delta_{N1}; \Xi; \Omega; C; P;
\aggsz{A}{B}{C}; \Omega_N; \Delta_N; T \rightarrow \Xi'; \Delta'; \Gamma'$ is
well-formed in relation to a triplet $A; \Gamma; \Delta_{N}$ iff the rules in
"Well-formed comprehension match" also apply.

\end{definition}


\subsection{Soundness of matching}

The soundness theorem will be proven into two main steps. First, we prove that
performing a rule match at LLD is sound in relation to HLD and then we prove
that the derivation of head terms in LLD is also sound.

In order to prove the soundness of matching, we want to reconstitute a valid
$\mz$in HLD from a valid $\mo$in LLD. However, LLD may fail during matching,
therefore our body match lemma needs to handle unsuccessful matches. In order to
be able to use induction, we must assume a matching proposition $\mo$that
already contains some continuation frames in stack $C$ that is well-formed in
relation to the rule's body $A$ and initial database.

Our lemma also needs to apply to our continuation judgment $\contlld$, because when inverting some of
the matching assumptions, we get a continuation proposition. Apart from an unsuccessful match, we deal
with two situations during a successful match: (1) we succeed without needing to backtrack to a frame
in stack $C$ or (2) we need to backtrack to a frame in $C$. The complete lemma is stated and proven below.

{\footnotesize
\begin{lemma}[Body match soundness]\label{thm:body_match}
   
Given a rule $A \lolli H$, consider a triplet $T = A; \Gamma; \Delta_{N}$ and a context $\Delta_{N} = \Delta_1, \Delta_2, \Xi$.

If $\mo \Gamma; \Delta_1, \Delta_2; \Xi; \Omega; H; C; R \rightarrow \Xi';
\Delta'; \Gamma'$ is well-formed in relation to $T$ then either:

\begin{enumerate}
   \item Match fails: $\cont \cdot; H; R; \Gamma \rightarrow \Xi'; \Delta'; \Gamma'$
   \item Match succeeds: $\mz \Gamma; \Delta_x \rightarrow A$ (where $\Delta_x
         \subseteq \Delta_N$) and either:
      \begin{enumerate}
         \item No backtracking to frames of stack $C$:
         
         \begin{itemize}
            \item $\mo \Gamma; \Delta_1; \Xi, \Delta_2; \cdot; H; C'', C; R
               \rightarrow \Xi'; \Delta'; \Gamma'$ (well-formed in relation to $T$)
            \item $\mo \Gamma; \Delta_1; \Xi, \Delta_2; \Omega; H; C; (\cdot, \Delta_N) \rightarrow \Xi'; \Delta'; \Gamma'$ (well-formed in relation to $T$)
            \item $\Delta_x = \Xi, \Delta_2$
         \end{itemize}

         \item With backtracking to a linear continuation frame:
         
         \begin{itemize}

            \item $\exists_{f \in C}. f = (\Delta_a; \Delta_{b_1}, p_2, \Delta_{b_2}; p;
                  \Omega_1, \dotsc, \Omega_k; \Xi_1 .. \Xi_m; \Lambda_1, \dotsc,
                  \Lambda_m; \Upsilon_1, \dotsc, \Upsilon_n)$

            \item $C = C', f, C''$

            \item $f$ turns into $f' = (\Delta_a, \Delta_{b_1}, p_2;
                  \Delta_{b_2}; p; \Omega_1, \dotsc, \Omega_k; \Xi_1, \dotsc, \Xi_m;
                  \Lambda_1, \dotsc, \Lambda_m; \Upsilon_1, \dotsc, \Upsilon_n)$

            \item $\mo \Gamma; \Delta_c; \Xi_1, \dotsc, \Xi_m, p_2, \Xi_c; \cdot; H;
                  C''', f', C''; R \rightarrow \Xi'; \Delta'; \Gamma'$ (well-formed in
                        relation to $T$)
            \item $\Delta_c = (\Delta_1, \Delta_2, \Xi) - (\Xi_1, \dotsc, \Xi_m, p_2, \Xi_c)$
         \end{itemize}

         \item With backtracking to a persistent continuation frame:
         
         \begin{itemize}
            \item $\exists_{f \in C}. = f = [\Gamma_1, p_2, \Gamma_2; \Delta_{c_1}, \Delta_{c_2}; \Xi_c; \bang
               p; \Omega_1, \dotsc, \Omega_k; \Lambda_1, \dotsc, \Lambda_m;
               \Upsilon_1, \dotsc, \Upsilon_n]$
            \item $C = C', f, C''$
            \item $f$ turns into $f' = [\Gamma_2; \Delta_{c_1}, \Delta_{c_2}; \Xi_1, \dotsc,
               \Xi_m; \bang p; \Omega_1, \dotsc, \Omega_k; \Lambda_1, \dotsc,
               \Lambda_m; \Upsilon_1, \dotsc, \Upsilon_n]$
            \item $\mo \Gamma; \Delta_{c_1}; \Xi_1, \dotsc, \Xi_m, \Delta_{c_2};
               \cdot; H; C'', f', C''; R \rightarrow \Xi'; \Delta'; \Gamma'$ (well-formed in
                  relation to $T$)
            \item $\Delta_{c_1}, \Delta_{c_2} = (\Delta_1, \Delta_2,
                  \Xi) - (\Xi_1, \dotsc, \Xi_m)$
         \end{itemize}
      \end{enumerate}
\end{enumerate}

If $\cont C; H; R; \Gamma \rightarrow \Xi'; \Delta'; \Gamma'$ and $C$ is well-formed in relation to $T$ then either:

\begin{enumerate}
   \item Match fails: $\cont \cdot; H; R; \Gamma \rightarrow \Xi'; \Delta'; \Gamma'$

   \item Match succeeds: $\mz \Delta_x \rightarrow A$ (where $\Delta_x \subseteq
         \Delta_N$) and either:

   \begin{enumerate}
      \item With backtracking to a linear continuation frame:
      \begin{itemize}
         \item $\exists_{f \in C}. f = (\Delta_a; \Delta_{b_1}, p_2, \Delta_{b_2}; \Xi_1,
               \dotsc, \Xi_m; p; \Omega_1, \dotsc, \Omega_k; \Lambda_1, \dotsc,
               \Lambda_m; \Upsilon_1, \dotsc, \Upsilon_n)$
         \item $C = C', f, C''$
         \item $f$ turns into $f' = (\Delta_a,
               \Delta_{b_1}, p_2; \Delta_{b_2}; p; \Omega_1, \dotsc, \Omega_k;
               \Xi_1, \dotsc, \Xi_m; \Lambda_1, \dotsc, \Lambda_m; \Upsilon_1,
               \dotsc, \Upsilon_n)$
         \item $\mo \Gamma; \Delta_c; \Xi_1, \dotsc, \Xi_m, p_2, \Xi_c; \cdot; H;
         C''', f', C''; R \rightarrow \Xi'; \Delta'; \Gamma'$ (well-formed in
               relation to $T$)
         \item $\Delta_c = (\Delta_1, \Delta_2, \Xi) - (\Xi_1, \dotsc, \Xi_m, p_2, \Xi_c)$
         \item $\Delta_x = \Xi_1, \dotsc, \Xi_m, p_2, \Xi_c$
      \end{itemize}

      \item With backtracking to a persistent continuation frame:
      \begin{itemize}
         \item $\exists_{f \in C}. f = [\Gamma_1, p_2, \Gamma_2; \Delta_{c_1},
            \Delta_{c_2}; \Xi_c; \bang p; \Omega_1, \dotsc, \Omega_k; \Lambda_1,
            \dotsc, \Lambda_m; \Upsilon_1, \dotsc, \Upsilon_n]$
         \item $C = C', f, C''$
         \item $f$ turns into $f' = [\Gamma_2; \Delta_{c_1}, \Delta_{c_2};
            \Xi_1, \dotsc, \Xi_m; \bang p; \Omega_1, \dotsc, \Omega_k;
            \Lambda_1, \dotsc, \Lambda_m;
            \Upsilon_1, \dotsc, \Upsilon_n]$

         \item $\mo \Gamma; \Delta_{c_1}; \Xi_1, \dotsc, \Xi_m, \Delta_{c_2};
\cdot; H; C'', f', C''; R \rightarrow \Xi'; \Delta'; \Gamma'$ (well-formed in
      relation to $T$)
         \item $\Delta_{c_1}, \Delta_{c_2} = \Delta_1, \Delta_2, \Xi - (\Xi_1, \dotsc, \Xi_m)$
         \item $\Delta_x = \Xi_1, \dotsc, \Xi_m, \Delta_{c2}$
      \end{itemize}
   \end{enumerate}
\end{enumerate}
\end{lemma}

\begin{proof}
   Proof by mutual induction. In $\mo$on the size of $\Omega$ and on $\contlld$, first on the size of the second argument of the frame ($\Delta''$ and $\Gamma''$) and then on the size of the stack $C$. Sub-cases:
   
\begin{itemize}
   \item $\mo p~\m{first}$, $\mo p~\m{on}~q$, $\mo p~\m{on}~\bang q$, $\mo \bang p~\m{first}$ $\mo \bang p~\m{on}~q$, $\mo \bang p~\m{on}~\bang q$, $\mo \otimes$
   
   When inverting the assumption, the well-formedness of the stack and match are
   proven straightforwardly using the well-formedness of the assumption and the
   match equivalence theorem. The induction hypothesis is then applied
   straightforwardly.
   
   \item $\mo \m{end}$
   
   Trivial.
   
   \item $\mo p~\m{fail}$, $\mo \bang p~\m{fail}$
   
   Invert the assumption and apply induction hypothesis on the $\cont$assumption.
   
   \item $\cont \m{next}~\m{rule}$
   
   Match fails.
   
   \item $\cont p~\m{next}, \cont \bang p~\m{next}$
   
   When inverting the assumption, we get a $\mo$proposition that is trivially
   proven to be well-formed in relation to $T$. Using induction hypothesis on
   this assumption, we have 3 sub-cases:
   
   \begin{itemize}
      \item Match fails: trivial.
      \item Match succeeds with no backtracking: the frame that we updated is the successful frame to backtrack to.
      \item Match succeeds with backtracking: $f \in C$ from the new assumption is the frame we need.
   \end{itemize}
   
   \item $\cont p~\m{no}~\m{more}$, $\cont \bang p~\m{no}~\m{more}$
   
   Invert the assumption to apply the induction hypothesis.
\end{itemize}
   
\end{proof}
}

For the induction hypothesis to be applicable in in Lemma~\ref{thm:body_match} there must be
a relation between the judgments $\mo$and $\contlld$.
We can define a lexicographic ordering $A \prec B$, meaning that proposition $A$ has a smaller proof than proposition $B$ (potentially $A$ is sub-proof of $B$),
or alternatively, $A$ is "executed later" than $B$.
The specific ordering is as follows:

\begin{enumerate}
   \item $\cont C; H; R; \Gamma \rightarrow \Xi'; \Delta'; \Gamma' \prec \cont C', C; H; R; \Gamma \rightarrow \Xi'; \Delta'; \Gamma'$\\
   The continuation must use the top of the stack $C'$ before using $C$;
   \item $\cont C', (\Delta, \Delta_1; \Delta_2; \Xi; p; \Omega; \Lambda; \Upsilon), C; H; R; \Gamma \rightarrow \Xi'; \Delta'; \Gamma'$\\
   \hspace*{1cm}$\prec \cont C'', (\Delta; \Delta_1, \Delta_2; \Xi; p; \Omega; \Lambda; \Upsilon), C; H; R; \Gamma \rightarrow \Xi'; \Delta'; \Gamma'$\\
   A continuation frame with more candidates has more steps to do than a frame with less candidates;
   \item $\cont C', [\Gamma_1; \Delta; \Xi; \bang p; \Omega; \Lambda; \Upsilon], C; H; R; \Gamma \rightarrow \Xi'; \Delta'; \Gamma'$\\
   \hspace*{1cm} $\prec \cont C'', [\Gamma_2, \Gamma_3; \Delta; \Xi; \bang p; \Omega; \Lambda; \Upsilon], C; H; R; \Gamma \rightarrow \Xi'; \Delta'; \Gamma'$\\
      Same as the previous one;
   \item $\cont C; H; R; \Gamma \rightarrow \Xi'; \Delta'; \Gamma' \prec \mo \Gamma; \Delta; \Xi; \Omega; H; C', C; R \rightarrow \Xi'; \Delta'; \Gamma'$\\
   Same as (1);
   \item $\mo \Gamma; \Delta; \Xi; \Omega; H; C; R \rightarrow \Xi'; \Delta'; \Gamma' \prec \cont C', C; H; R; \Gamma \rightarrow \Xi'; \Delta'; \Gamma'$\\
   Same as the previous one;
   \item $\mo \Gamma; \Delta''; \Xi''; \Omega'; H; C', C; R \rightarrow \Xi'; \Delta'; \Gamma' \prec \mo \Gamma; \Delta; \Xi; \Omega; H; C; R \rightarrow \Xi'; \Delta'; \Gamma'$ as long as $\Omega' \prec \Omega$\\
   Adding continuation frames to the stack makes the proof smaller as long as $\Omega$ is also smaller; 
   \item $\mo \Gamma; \Delta; \Xi; \Omega; H; C', (\Delta, \Delta_1; \Delta_2; \Xi; p; \Omega; \Lambda; \Upsilon), C; R \rightarrow \Xi'; \Delta'$\\
   \hspace*{1cm} $\prec \mo \Gamma; \Delta''; \Xi''; \Omega'; C'', (\Delta; \Delta_1, \Delta_2; \Xi; p; \Omega; \Lambda; \Upsilon), C; R \rightarrow \Xi'; \Delta'; \Gamma'$\\
   Same as (2);
   \item $\mo \Gamma; \Delta; \Xi; \Omega; H; C', [\Gamma_1; \Delta; \Xi; \bang p; \Omega; \Lambda; \Upsilon], C; R \rightarrow \Xi'; \Delta'; \Gamma'$\\
   \hspace*{1cm} $\prec \mo \Gamma; \Delta''; \Xi''; \Omega'; C'', [\Gamma_2, \Gamma_3; \Delta; \Xi; \bang p; \Omega; \Lambda; \Upsilon], C; R \rightarrow \Xi'; \Delta'; \Gamma'$\\
   Same as (3).
\end{enumerate}

\subsection{Soundness of derivation}

Proving that the derivation of the head of the rule is sound is trivial except
for comprehensions and aggregates. LLD deterministically computes the number of
available comprehensions to apply while HLD "guesses" the number and then
performs the derivation. In the next two sections, we show how to prove the
soundness of comprehensions and aggregates. The strategy for proving for proving
both is identical due to their inherient similarities.

\subsection{Comprehension soundness}

Proving that deriving a comprehension in LLD is sound in relation to HLD is
built from 4 results: (1) proving that matching the body of a comprehension is
sound in relation to HLD; (2) proving that updating the continuation stacks
makes them suitable for use in the next comprehension applications; (3) proving
that deriving the head of the comprehension is sound in relation to HLD; (4)
proving that we can apply as many comprehensions as the database allows.

{\footnotesize
\begin{lemma}[Comprehension body match]\label{thm:comprehension_body_match}
Given a comprehension $AB = \compsz{A}{B}$, consider a triplet $T = A; \Gamma; \Delta_{N}$ and a context $\Delta_{N} = \Delta_1, \Delta_2, \Xi$.

If $\mc \Gamma; \Delta_1, \Delta_2; \Xi_N; \Gamma_{N1}; \Delta_{N1}; \Xi;
\Omega; C; P; \compsz{A}{B}; \Omega_N; \Delta_N \rightarrow \Xi'; \Delta'; \Gamma'$ is well-formed in relation to $T$ then either:

   \begin{enumerate}
      \item Match fails: $\done \Gamma; \Delta_N; \Xi_N; \Gamma_{N1}; \Delta_{N1}; \Omega_N \rightarrow \Xi'; \Delta'; \Gamma'$
      
      \item Match succeeds: $\mz \Gamma; \Delta_x \rightarrow A$ (where
            $\Delta_x \subseteq \Delta_N$) and either:

      \begin{enumerate}
         \item No backtracking to frames of stack $C$ or $P$ ($C \neq \cdot$):

         \begin{itemize}
            \item $\mc \Gamma; \Delta_1; \Xi_N; \Gamma_{N1}; \Delta_{N1}; \Xi,
               \Delta_2; \cdot; C', C; P; AB; \Omega_N; \Delta_N
               \rightarrow \Xi'; \Delta'; \Gamma'$ (well-formed in relation to
                     $T$)
            \item $\Delta_x = \Delta_2$
         \end{itemize}

         \item No backtracking to frames of stack $P$ ($C = \cdot$):

         \begin{itemize}
            \item $\mc \Gamma; \Delta_1; \Xi_N; \Gamma_{N1}; \Delta_{N1}; \Xi,
               \Delta_2; \cdot; C'; P', P; AB; \Omega_N; \Delta_N
               \rightarrow \Xi'; \Delta'; \Gamma'$ (well-formed in relation to
                     $T$)
            \item $\Delta_x = \Delta_2$
         \end{itemize}
         
         \item With backtracking to a linear continuation frame in stack $C$ ($C \neq \cdot$):
         \begin{itemize}
            \item $\exists_{f \in C}. f = (\Delta_a; \Delta_{b_1}, p_2,
                  \Delta_{b_2}; p; \Xi_1, \dotsc, \Xi_m; \Omega_1, \dotsc,
                  \Omega_k; \Lambda_1, \dotsc, \Lambda_m; \Upsilon_1, \dotsc,
                  \Upsilon_n)$
            \item $C = C', f, C''$
            \item $f$ turns into $f' = (\Delta_a, \Delta_{b_1}, p_2;
                  \Delta_{b_2}; p; \Xi_1, \dotsc, \Xi_m;
                  \Omega_1, \dotsc, \Omega_k; \Lambda_1, \dotsc, \Lambda_m;
                  \Upsilon_1, \dotsc, \Upsilon_n)$
            \item $\mc \Gamma; \Delta_c; \Xi_N; \Gamma_{N1}; \Delta_{N1}; \Xi_1,
               \dotsc, \Xi_m, p_2, \Xi_c; \cdot; C''', f', C''; P;
               AB; \Omega_N; \Delta_N \rightarrow \Xi'; \Delta';
               \Gamma'$ (well-formed in relation to $T$)
            \item $\Delta_c = (\Delta_1, \Delta_2, \Xi) - (\Xi_1, \dotsc, \Xi_m,
                  p_2, \Xi_c)$
            \item $\Delta_x = \Xi_1, \dotsc, \Xi_m, p_2, \Xi_c$
         \end{itemize}
            
         \item With backtracking to a persistent continuation frame in stack $C$ ($C \neq \cdot$):

         \begin{itemize}
            \item $\exists_{f \in C}. f = [\Gamma_1, p_2, \Gamma_2; \Delta_{c_1},
               \Delta_{c_2}; \Xi_1, \dotsc, \Xi_m; \bang p; \Omega_1, \dotsc, \Omega_k;
               \Lambda_1, \dotsc, \Lambda_m; \Upsilon_1, \dotsc, \Upsilon_n]$
            \item $C = C', f, C''$
            \item $f$ turns into $f' = [\Gamma_2; \Delta_{c_1}, \Delta_{c_2};
               \Xi_1, \dotsc, \Xi_m; \bang p; \Omega_1, \dotsc, \Omega_k; \Lambda_1,
               \dotsc, \Lambda_m; \Upsilon_1, \dotsc, \Upsilon_n]$
            \item $\mc \Gamma; \Delta_{c_1}; \Xi_N; \Gamma_{N1}; \Delta_{N1};
               \Delta_{c_2}, \Xi_1, \dotsc, \Xi_m; \cdot; C''', f', C''; P;
               AB; \Omega_N; \Delta_N \rightarrow \Xi'; \Delta'; \Gamma'$ (well-formed in relation to $T$)
            \item $\Delta_{c_1}, \Delta_{c_2} = (\Delta_1, \Delta_2, \Xi) - (\Xi_1, \dotsc, \Xi_m)$
            \item $\Delta_x = \Delta_{c_2}, \Xi_1, \dotsc, \Xi_m$
         \end{itemize}
            
         \item With backtracking to a persistent continuation frame in stack $P$ ($C = \cdot$):

         \begin{itemize}
            \item $\exists_{f \in P}. f = [\Gamma_1, p_2, \Gamma_2; \Delta_{c_1}, \Delta_{c_2};
         \Xi_1, \dotsc, \Xi_m; \bang p; \Omega_1, \dotsc, \Omega_k; \Lambda_1,
         \dotsc, \Lambda_m; \Upsilon_1, \dotsc, \Upsilon_n]$
            \item $P = P', f, P''$
            \item $f$ turns into $f' = [\Gamma_2; \Delta_{c_1},
               \Delta_{c_2}; \Xi_1, \dotsc, \Xi_m; \bang p; \Omega_1, \dotsc,
               \Omega_k; \Lambda_1, \dotsc, \Lambda_m; \Upsilon_1, \dotsc, \Upsilon_n]$
            \item $\mc \Gamma; \Delta_{c_1}; \Xi_N; \Gamma_{N1}; \Delta_{N1};
            \Delta_{c_2}, \Xi_1, \dotsc, \Xi_m; \cdot; C'; P''', f', P'';
            AB; \Omega_N; \Delta_N \rightarrow \Xi'; \Delta'; \Gamma'$
            (well-formed in relation to $T$)
            \item $\Delta_{c_1}, \Delta_{c_2} = (\Delta_1, \Delta_2, \Xi) - (\Xi_1, \dotsc,
                  \Xi_m)$
            \item $\Delta_x = \Delta_{c_2}, \Xi_1, \dotsc, \Xi_m$
         \end{itemize}
      \end{enumerate}
   \end{enumerate}
   
If $\contc \Gamma; \Delta_{N}; \Xi_{N}; \Gamma_{N1}; \Delta_{N1}; C; P;
AB; \Omega_N \rightarrow \Xi'; \Delta'; \Gamma'$ and $C$ and $P$ are
well-formed in relation to $T$ then either:

\begin{enumerate}
   \item Match fails: $\done \Gamma; \Delta_N; \Xi_N; \Gamma_{N1}; \Delta_{N1}; \Omega_N \rightarrow \Xi'; \Delta'; \Gamma'$
   
   \item Match succeeds: $\mz \Delta_x \rightarrow A$ (where $\Delta_x \subseteq \Delta_N$) where one sub-case is true:
   
   \begin{enumerate}
      \item With backtracking to a linear continuation frame in stack $C$ ($C \neq \cdot$):

       \begin{itemize}
         \item $\exists_{f \in C}. f = (\Delta_a; \Delta_{b_1}, p_2,
               \Delta_{b_2}; p; \Xi_1, \dotsc, \Xi_m; \Omega_1, \dotsc,
               \Omega_k; \Lambda_1, \dotsc, \Lambda_m; \Upsilon_1, \dotsc, \Upsilon_n)$
         \item $C = C', f, C''$
         \item $f$ turns into $f' = (\Delta_a, \Delta_{b_1}, p_2; \Delta_{b_2};
               p; \Xi_1, \dotsc, \Xi_m; \Omega_1, \dotsc, \Omega_k; \Lambda_1,
               \dotsc, \Lambda_m; \Upsilon_1, \dotsc, \Upsilon_n)$
         \item $\mc \Gamma; \Delta_c; \Xi_N; \Gamma_{N1}; \Delta_{N1}; \Xi_1,
            \dotsc, \Xi_m, p_2, \Xi_c; \cdot; C''', f', C''; P; AB;
         \Omega_N; \Delta_N \rightarrow \Xi'; \Delta'; \Gamma'$ (well-formed in
               relation to $T$)
         \item $\Delta_c = (\Delta_1, \Delta_2, \Xi) - (\Xi_1, \dotsc, \Xi_m, p_2, \Xi_c)$
      \end{itemize}
      
      \item With backtracking to a persistent continuation frame in stack $C$
      ($C \neq \cdot$):

      \begin{itemize}
         \item $\exists_{f \in C}. f = [\Gamma_1, p_2, \Gamma_2; \Delta_{c_1},
         \Delta_{c_2}; \Xi_1, \dotsc, \Xi_m; \bang p; \Omega_1, \dotsc,
         \Omega_k; \Lambda_1, \dotsc, \Lambda_m; \Upsilon_1, \dotsc, \Upsilon_n]$
         \item $C = C', f, C''$
         \item $f$ turns into $f' = [\Gamma_2; \Delta_{c_1}, \Delta_{c_2};
         \Xi_1, \dotsc, \Xi_m; \bang p; \Omega_1, \dotsc, \Omega_k; \Lambda_1,
         \dotsc, \Lambda_m; \Upsilon_1, \dotsc, \Upsilon_n]$
         \item $\mc \Gamma; \Delta_{c_1}; \Xi_N; \Gamma_{N1}; \Delta_{N1};
            \Delta_{c_2}, \Xi_1, \dotsc, \Xi_m; \cdot; C''', f', C''; P;
            AB; \Omega_N; \Delta_N \rightarrow \Xi'; \Delta'; \Gamma'$ (well-formed in relation to $T$)
         \item $\Delta_{c_1}, \Delta_{c_2} = (\Delta_1, \Delta_2, \Xi) - (\Xi_1,
               \dotsc, \Xi_m)$
         \item $\Delta_x = \Delta_{c_2}, \Xi_1, \dotsc, \Xi_m$
      \end{itemize}
      
      \item With backtracking to a persistent continuation frame in stack $P$
      ($C = \cdot$):

      \begin{itemize}
         \item $\exists_{f \in P}. f = [\Gamma_1, p_2, \Gamma_2; \Delta_{c_1},
         \Delta_{c_2}; \Xi_1, \dotsc, \Xi_m; \bang p; \Omega_1, \dotsc, \Omega_k;
         \Lambda_1, \dotsc, \Lambda_m; \Upsilon_1, \dotsc, \Upsilon_n]$
         \item $P = P', f, P''$
         \item $f$ turns into $f' = [\Gamma_2; \Delta_{c_1}, \Delta_{c_2};
         \Xi_1, \dotsc, \Xi_m; \bang p; \Omega_1, \dotsc, \Omega_k; \Lambda_1,
         \dotsc, \Lambda_m; \Upsilon_1, \dotsc, \Upsilon_n]$
         \item $\mc \Gamma; \Delta_{c_1}; \Xi_N; \Gamma_{N1}; \Delta_{N1};
            \Delta_{c_2}, \Xi_1, \dotsc, \Xi_m; \cdot; C'; P''', f', P'';
            AB; \Omega_N; \Delta_N \rightarrow \Xi'; \Delta'; \Gamma'$ (well-formed in relation to $T$)
         \item $\Delta_{c_1}, \Delta_{c_2} = (\Delta_1, \Delta_2, \Xi) - (\Xi_1,
               \dotsc, \Xi_m)$
         \item $\Delta_x = \Delta_{c_2}, \Xi_1, \dotsc, \Xi_m$
      \end{itemize}

   \end{enumerate}
\end{enumerate}
\end{lemma}
}

Proving that matching the body of a comprehension is sound in relation to HLD
follows the structure of the Lemma~\ref{thm:body_match}. The lemma uses mutual
induction on the recursive judgments $\mc$and $\contc$and considers the three
possible results of matching: failure, success with no backtracking and success
with backtracking.

In order to apply a comprehension again, we need to reuse the continuation
stacks. However, in order to use $C$ and $P$ safely, we need to prove that $C$
will have at most one updated linear continuation frame and $P$ will have all
its frames updated to account the consumption of the facts from the previous
application of the comprehension.

We first prove some auxiliary theorems.

\begin{theorem}[Full stack update]\label{thm:stack_update}
If $\strans \Xi; P; P'$ then $P'$ will be the transformation of stack $P$ where
every frame $f \in P$, where $f = [\Gamma'; \Delta_N; \cdot; \bang p; \Omega; \cdot;
      \Upsilon])$, will turn into $f' = [\Gamma'; \Delta_N - \Xi; \cdot;
      \bang p; \Omega; \cdot; \Upsilon]$, where $f' \in P'$.
\end{theorem}
\begin{proof}
Straightforward induction on the size of $P$.
\end{proof}

\begin{theorem}[From update to derivation]\label{thm:from_update_to_derivation}
~\newline
If $\fix \Gamma; \Xi_N; \Gamma_{N1}; \Delta_{N1}; \Xi; C; P; AB; \Omega_N;
\Delta_N \rightarrow \Xi'; \Delta'; \Gamma'$ then\\
\texttab$\dc \Gamma; \Xi_N, \Xi;
\Gamma_{N1}; \Delta_{N1}; B; C' ; P'; AB; \Omega_N; (\Delta_N - \Xi) \rightarrow
\Xi'; \Delta'; \Gamma'$, where:

\begin{enumerate}
   \item If $C = \cdot$ then $C' = \cdot$

   \item If $C = C_1, (\Delta_a; \Delta_b; \cdot; p; \Omega; \cdot; \Upsilon)$
   then $C' = (\Delta_a - \Xi; \Delta_b - \Xi; \cdot; p; \Omega; \cdot;
         \Upsilon)$

   \item $P'$ is the transformation of stack $P$, where for every frame $f \in
   P$ of the form $[\Gamma'; \Delta_N; \cdot; \bang p; \Omega; \cdot; \Upsilon]$
   will turn into $f' = [\Gamma';\Delta_N-\Xi;\cdot;\bang p;\Omega;\cdot;\Upsilon]$

\end{enumerate}
\end{theorem}
\begin{proof}
Use induction on the size of the stack $C$ to get the result of $C'$ and then
apply Theorem~\ref{thm:stack_update} to get $P'$.
\end{proof}


Now we prove that a match of a comprehension's body implies the start of a
derivation of the comprehension's head with correct continuation stacks. Note
that $\Omega = \cdot$ in $\matchlldc$, so there is nothing left to match.

\begin{corollary}[Match to derivation]\label{thm:match_to_derivation}
~\newline
If $\mc \Gamma; \Delta; \Xi_N; \Gamma_{N1}; \Delta_{N1}; \Xi; \cdot; B; C; P;
AB;\Omega_N; \Delta_N \rightarrow \Xi'; \Delta'; \Gamma'$ then\\
\texttab$\dc \Gamma; \Xi_N, \Xi; \Gamma_{N1}; \Delta_{N1}; B; C'; P'; AB; \Omega_N; (\Delta_N - \Xi) \rightarrow \Xi'; \Delta'; \Gamma'$ where:
   
\begin{enumerate}
   \item If $C = \cdot$ then $C' = \cdot$
   \item If $C = C_1, (\Delta_a; \Delta_b; \cdot; p; \Omega; \cdot; \Upsilon)$ then $C' = (\Delta_a - \Xi; \Delta_b - \Xi; \cdot; p; \Omega; \cdot; \Upsilon)$ then \linebreak $C' = (\Delta_a - \Xi; \Delta_b - \Xi; \cdot; p; \Omega; \cdot; \Upsilon)$
   \item $P'$ is the transformation of stack $P$, where for every frame $f \in
   P$ of the form $[\Gamma'; \Delta_N; \cdot; \bang p; \Omega; \cdot; \Upsilon]$
   will turn into $f' = [\Gamma';\Delta_N-\Xi;\cdot;\bang p;\Omega;\cdot;\Upsilon]$
\end{enumerate}
\end{corollary}

\begin{proof}
Invert the assumption and then apply Theorem~\ref{thm:from_update_to_derivation}.
\end{proof}


\paragraph{Comprehension Derivation}

We also need to prove that deriving the head of a comprehension is sound in
relation to HLD.  With the results of the next theorem we can reuse the
continuation stacks to start the comprehension process all over again, but now
with a non-empty continuation stack.

\begin{theorem}[Comprehension derivation soundness]\label{thm:comprehension_derivation}
~\newline
If $\dc \Gamma; \Xi_N; \Gamma_{N1}; \Delta_{N1}; \Omega_1, \dotsc, \Omega_n; C; P;
AB; \Omega_N; \Delta_N \rightarrow \Xi'; \Delta'; \Gamma'$ then:

\begin{enumerate}
   \item $\dc \Gamma; \Xi_N; \Gamma_{N1}, \Gamma_1, \dotsc, \Gamma_n; \Delta_{N1},
   \Delta_1, \dotsc, \Delta_n; \cdot; C; P; AB; \Omega_N; \Delta_N \rightarrow
   \Xi'; \Delta'; \Gamma'$;

   \item $\forall_{\Omega_x, \Delta_x}.$ If $\dz \Gamma; \Delta_x; \Xi_N;
   \Gamma_{N1}, \Gamma_1, \dotsc, \Gamma_n; \Delta_{N1}, \Delta_1, \dotsc,
   \Delta_n; \Omega_x \rightarrow \Xi'; \Delta'; \Gamma'$ then \linebreak $\dz
   \Gamma; \Delta_x; \Xi_N; \Gamma_{N1}; \Delta_{N1}; \Omega_1, \dotsc, \Omega_n, \Omega_x \rightarrow \Xi'; \Delta'; \Gamma'$.

\end{enumerate}
\end{theorem}

\begin{proof}
Straightforward induction on $\Omega_1, \dotsc, \Omega_n$.
\end{proof}

The second result of this theorem is the soundness result we need because it will allow us to reconstruct the derivation tree in HLD.


\paragraph{Multiple Comprehension Derivation} We are interested in proving that
if we start with a given comprehension match $\matchlldc$ then we can apply the
comprehension several times.

\begin{theorem}[Multiple comprehension derivation]\label{thm:multiple_comprehension_derivation}
Assume that there exists $n \geq 0$ applications of a comprehension
$AB = \compsz{A}{B}$, where the $i_{th}$ application is related to the following contexts:
\begin{itemize}
   \item $\Delta_i$: context of derived linear facts;
   \item $\Gamma_i$: context of derived persistent facts;
   \item $\Xi_i$: context of consumed linear facts.
\end{itemize}

Consider a triplet $T = A; \Gamma; \Delta_{N}$, and $\Delta_N = \Delta, \Xi_1,
         ..., \Xi_n$.  Assume that $\Delta_N = \Delta_a, \Delta'_b, p_1$ and
         $\Delta_b = \Delta'_b, p_1$.

Given a frame $f = (\Delta_a, p_1; \Delta'_b; \cdot; p; \Omega; \cdot;
      \Upsilon)$, if $\mc \Gamma; \Delta_a, \Delta'_b; \Xi_N; \Gamma_{N1}; \Delta_{N1}; p_1;
\Omega; f; P; AB; \Omega_N; \Delta, \Xi_1, ..., \Xi_n \rightarrow \Xi'; \Delta'; \Gamma'$ (well-formed in relation to $T$) then:
   
\end{theorem}
\iffalse
   \begin{enumerate}
      \item $n$ comprehensions are derived:\\
      $\done \Gamma; \Delta_N; \Xi_N, \Xi_1, ..., \Xi_n; \Gamma_{N1}, \Gamma_1, ..., \Gamma_n; \Delta_{N1}, \Delta_1, ..., \Delta_n; \Omega_N \rightarrow \Xi'; \Delta'; \Gamma'$
      \item $n$ $\mz$propositions for the $n$ comprehension matches:\\
      $\mz \Gamma; \Xi_1 \rightarrow A$ ... $\mz \Gamma; \Xi_n \rightarrow A$
      \item $n$ implications from $1...i...n$ such that: \\
      $\forall_{\Omega_x, \Delta_x}.$ if $\done \Gamma; \Delta_x; \Xi_N, \Xi_1, ..., \Xi_i; \Gamma_{N1}, \Gamma_1, ..., \Gamma_i; \Delta_{N1}, \Delta_1, ..., \Delta_i; \Omega_x \rightarrow \Xi'; \Delta'; \Gamma'$ then \linebreak $\dz \Gamma; \Delta_x; \Xi_N, \Xi_1, ..., \Xi_i; \Gamma_{N1}, \Gamma_1, ..., \Gamma_{i-1}; \Delta_{N1}, \Delta_1, ..., \Delta_{i-1}; B, \Omega_x \rightarrow \Xi'; \Delta'; \Gamma'$
   \end{enumerate}
   
   If $\mc \Gamma; \Delta_N; \Xi_N; \Gamma_{N1}; \Delta_{N1}; \cdot; \Omega; \cdot; P; A \com B; \Omega_N; \Delta, \Xi_1, ..., \Xi_n \rightarrow \Xi'; \Delta'; \Gamma'$ (well-formed in relation to $T$) then:
   
   \begin{enumerate}
      \item $n$ comprehensions are derived:\\
      $\done \Gamma; \Delta_N; \Xi_N, \Xi_1, ..., \Xi_n; \Gamma_{N1}, \Gamma_1, ..., \Gamma_n; \Delta_{N1}, \Delta_1, ..., \Delta_n; \Omega_N \rightarrow \Xi'; \Delta'; \Gamma'$
      \item $n$ $\mz$propositions for the $n$ comprehension matches:\\
      $\mz \Gamma; \Xi_1 \rightarrow A$ ... $\mz \Gamma; \Xi_n \rightarrow A$
      \item $n$ implications from $1...i...n$ such that: \\
      $\forall_{\Omega_x, \Delta_x}.$ if $\dz \Gamma; \Delta_x; \Xi_N, \Xi_1, ..., \Xi_i; \Gamma_{N1}, \Gamma_1, ..., \Gamma_i; \Delta_{N1}, \Delta_1, ..., \Delta_i; \Omega_x \rightarrow \Xi'; \Delta'; \Gamma'$ then \\$\dz \Gamma; \Delta_x; \Xi_N, \Xi_1, ..., \Xi_i; ; \Gamma_{N1}, \Gamma_1, ..., \Gamma_{i-1}; \Delta_{N1}, \Delta_1, ..., \Delta_{i-1}; B, \Omega_x \rightarrow \Xi'; \Delta'; \Gamma'$
   \end{enumerate}
   \fi
