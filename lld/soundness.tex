The soundness theorem proves that if a rule was successfully derived in the low level semantics
then it can also be successfully derived in the high level semantics. The completeness theorem cannot
be proven correct because the low level semantics lack the non-determinism of the high level semantics.

The first main lemma of the soundness proof proves that if we can match the body
of a rule at the low level then we can also match the rule in the high level system using the same database.

\begin{lemma}[Body Match]
   Given a match $\mo \Gamma; \Delta_1, \Delta_2; \cdot; A; B; \cdot; R \rightarrow \Xi'; \Delta'; \Gamma'$ that is related to $A$, $\Delta_1, \Delta_2$ and $\Gamma$, we get either:
   
   \begin{enumerate}
      \item $\cont \cdot; B; R; \Gamma; \Xi'; \Delta'; \Gamma'$;
      \item $\mz \Delta_2 \rightarrow A$ and $\mo \Gamma; \Delta_1; \Delta_2; \cdot; B; C'; R \rightarrow \Xi'; \Delta'; \Gamma'$ (related)
   \end{enumerate}
\end{lemma}
\begin{proof}
   Use the body match soundness theorem.
\end{proof}

When we say that a match is related to a term $A$ and a database $\Delta_1, \Delta_2, \Gamma$ we mean that
the matching judgment is related to the body $A$ of a rule and the initial database is $\Delta_1, \Delta_2, \Gamma$. Moreover, the continuation stack is related to $A$ and to the database.

The body match lemma tells us that if we start a match of a body $A$ we will either fail (1) and need to try another rule in $R$ or we succeed by building the high level matching judgment $\mz \Delta_2 \rightarrow A$ and reaching the end of the matching process $\mo \Gamma; \Delta_1; \Delta_2; \cdot; B; C'; R \rightarrow \Xi'; \Delta'; \Gamma'$.

This lemma uses a more complicated theorem that is recursively defined through judgments $\m{match}_1$ and $\m{cont}$ that use mutual induction on the size of the continuation stack, the size of the remaining terms
 to match and also the size of alternatives at each continuation frame.
 
The second stepping stone in the soundness proof is the derivation lemma. After we successfully match the
body of a rule, we need to prove that the derivation process (through judgments $\m{derive}_1$) is also
sound. This lemma is as follows:

\begin{lemma}[Derivation]
   If the low level derivation $\done \Gamma; \Delta; \Xi; \Gamma_1; \Delta_1; \Omega \rightarrow \Xi'; \Delta'; \Gamma'$ is true then the high level derivation $\dz \Gamma; \Delta; \Gamma_1; \Delta_1; \Omega \rightarrow \Xi'; \Delta'; \Gamma'$ is also true.
\end{lemma}
\begin{proof}
   Straightforward use of induction on $\Omega$ except for the sub-case of comprehensions and aggregates, where we need to use the comprehension and aggregate theorems to construct the derivation tree using $n$ applications of the corresponding construct.
\end{proof}

In the case of proving the soundness of comprehensions, we use a very identical theorem to the one used
to prove the body match soundness. However, in this case we need to reuse the continuation stack several
times (as many as many comprehensions can be applied). Using induction on the continuation stack, we get
$n$ (where $n \ge 0$) applications of the comprehension and $n \; \m{match}$ and $n \; \m{derive}$ judgments
that can be used to rebuild the derivation tree at the low level by using the $\dz \with L$, $\dz \with R$
and $\dz \lolli$ rules to fold and unfold the comprehension term. The theorem for aggregates works similarly.

