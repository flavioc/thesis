
Now that we have presented both the HLD and LLD semantics, we are in position to
start building our soundness theorem.  The soundness theorem proves that if a
rule was successfully derived in the LLD semantics then it can also be derived
in the HLD semantics. Since the HLD semantics are so close to linear logic, we
prove that our language has a determined, correct, proof search behavior when
executing programs. However, the completeness theorem cannot be proven since LLD
lacks the non-determinism inherent in HLD.

First and foremost, we need to prove some auxiliary theorems and definitions
that will be used during the soundness theorem.

\subsection{Term equivalence}

The first definition defines the equality between two multi-sets of terms.  Two
multi-sets $A$ and $B$ are equal, $\feq{A}{B}$, when they have the same
constituent atoms.

\[
\infer[\equiv p]
{\feq{p, A}{p, B}}
{\feq{A}{B}}
\tab
\infer[\equiv \bang p]
{\feq{\bang p, A}{\bang p, B}}
{\feq{A}{B}}
\tab
\infer[\equiv \one~L]
{\feq{\one, A}{B}}
{\feq{A}{B}}
\tab
\infer[\equiv \one~R]
{\feq{A}{\one, B}}
{\feq{A}{B}}
\]

\[
\infer[\equiv \cdot]
{\feq{\cdot}{\cdot}}
{}
\tab
\infer[\equiv \otimes~L]
{\feq{A \otimes B, C}{D}}
{\feq{A, B, C}{D}}
\tab
\infer[\equiv \otimes~R]
{\feq{A}{B \otimes C, D}}
{\feq{A}{B, C, D}}
\]

\begin{theorem}[Match equivalence]
If two multi-sets are equivalent, $\feq{A_1, \dotsc, A_n}{B_1, \dotsc, B_m}$,
   and we can match $A_1 \otimes \dotsb \otimes A_n$ in HLD such that
   $\mz{\Gamma}{\Delta}{A_1 \otimes \dotsb \otimes A_n}$ then
      $\mz{\Gamma}{\Delta}{B_1 \otimes \dotsb \otimes B_m}$ is also true.
\end{theorem}
\begin{proof}
By straightforward induction on the first assumption.
\end{proof}

\begin{definition}[Split contexts]
$split(\Omega)$ is defined as $split(\Omega) = times(flatten(\Omega))$, where:

\begin{align}
flatten(\cdot) = \cdot \\
flatten(\one, \Omega) = flatten(\Omega) \\
flatten(A \otimes B, \Omega) = flatten(A), flatten(B), flatten(\Omega) \\
flatten(p, \Omega) = p, flatten(\Omega) \\
flatten(\bang p, \Omega) = \bang p, flatten(\Omega)
\end{align}

And $times(A_1, \dotsc, A_n) = A_1 \otimes \dotsb \otimes A_n$.
\end{definition}

\begin{theorem}[Split equivalence]
$\feq{split(\Omega)}{\Omega}$.
\end{theorem}
\begin{proof}
Induction on the structure of $\Omega$.
\end{proof}

\subsection{Well-formed continuation frames}

We now define the concept of a well-formed frame given initial linear and
persistent contexts and a term $A$ that needs to be matched.

\begin{definition}[Well-formed frame]

Consider a triplet $A; \Gamma; \Delta_{N}$ where $A$ is a term, $\Gamma$ is a
set of persistent resources and $\Delta_{N}$ a multi-set of linear
resources. A frame $f$ is well-formed iff:

\begin{enumerate}[leftmargin=*]
   \item Linear frame $f = \lframe{\Delta,
      p_1}{\Delta''}{p}{\Omega}{\Delta'}{\Omega'}$

   \begin{enumerate}
      \item $\feq{p, \Omega, \Omega'}{A}$ (the remaining terms and already
               matched terms are equivalent to the initial body $A$);
      \item $\Delta' = \Delta'_1, \dotsc, \Delta'_n$ and $\Omega' =
      \Omega'_1 \otimes \dotsb \otimes \Omega'_n$
      \item $\Delta, \Delta'', \Delta, p_1 = \Delta_{N}$ (available facts, candidate
            facts for $p$, consumed facts and the linear fact used for $p$,
            respectively, are the same as the initial $\Delta_{N}$);
      \item $\mz{\Gamma}{\Delta'}{\Omega}$ is valid.

   \end{enumerate}
   \item Persistent frame $f = \pframe{\Gamma''}{\Delta}{\bang
   p}{\Omega}{\Delta'}{\Omega'}$
      \begin{enumerate}
         \item $\feq{\bang p, \Omega, \Omega'}{A}$;
         \item $\Delta' = \Delta'_1, \dotsc, \Delta'_n$ and $\Omega' =
         \Omega'_1 \otimes \dotsb \otimes \Omega'_n$
         \item $\Delta, \Delta' = \Delta_{N}$;
         \item $\Gamma'' \subset \Gamma$ (remaining candidates are a subset of
                     $\Gamma$);
         \item $\mz{\Gamma}{\Delta'}{\Omega}$ is valid.
      \end{enumerate}
\end{enumerate}
\end{definition}


\begin{definition}[Well-formed stack]
A continuation stack $\lstack{C}$ is well-formed iff every frame is well-formed.
\end{definition}

Given the previous definitions, we can now define what it means for a given
machine state to be well-formed.

\begin{definition}[Well-formed body match]

$\matstate{A \lolli B}{\rulestk}{\lstack{C}}{\Gamma}{\Delta}{\Omega}{\Delta'
\rightarrow \Omega'}$ is well-formed in relation to a triplet $A; \Gamma;
\Delta_{N}$ iff:

\begin{itemize}[leftmargin=*]
   \item $\Delta, \Delta' = \Delta_{N}$
   \item $\feq{\Omega', \Omega}{A}$;
   \item $\mz{\Gamma}{\Delta'}{\Omega}$ is valid;
   \item $\lstack{C}$ is well-formed in relation to $A; \Gamma; \Delta_{N}$ and:
   \begin{itemize}[leftmargin=\secondm]
      \item If $\lstack{C} = \cdot$
   
      $\feq{\Omega}{A}$.
   
      \item If $\lstack{C} = \lframe{\Delta_a,
   p_1}{\Delta_b}{p}{\Omega_a}{\Delta_c}{\Omega_b}, \lstack{C'}$:
      \begin{itemize}[leftmargin=\thirdm]
         \item $\feq{\Omega'}{\Omega}$;
         \item $\Delta = \Delta_a, \Delta_b$;
         \item $\Delta' = \Delta_c, p_1$.
      \end{itemize}

      \item If $\lstack{C} = \pframe{\Gamma''}{\Delta_a}{\bang
         p}{\Omega_a}{\Delta_a}{\Omega_b}, \lstack{C'}$:
      \begin{itemize}[leftmargin=\thirdm]
         \item $\feq{\Omega}{\Omega_a}$;
         \item $\Delta' = \Delta_a$;
         \item $\Delta = \Delta_a$.
      \end{itemize}
   \end{itemize}
\end{itemize}

\end{definition}

\begin{definition}[Well-formed backtracking]
$\contstate{A \lolli B}{\rulestk}{\lstack{C}}{\Gamma}$ is well-formed in
relation to a triplet $A; \Gamma; \Delta_{N}$ iff $\lstack{C}$ is well-formed in
relation to $A; \Gamma; \Delta_{N}$.
\end{definition}

\begin{definition}[Well-formed aggregate match]
The matching state $\matstatea{\Delta_N}{\lstack{C};
      \lstack{P}}{\Gamma}{\Delta}{\Omega}{\Delta' \rightarrow
         \Omega'}{\Sigma}$
is well-formed in relation to a triplet $A; \Gamma; \Delta_{N}$ iff:

\begin{itemize}[leftmargin=*]
   \item $\lstack{P}$ is composed solely of persistent frames;
   \item $\lstack{C}$ is composed of either linear or persistent frames, but the first
   frame is linear;
   \item $\mz{\Gamma}{\Delta'}{\Omega}$ is valid;
   \item $\Delta, \Delta' = \Delta_{N}$;
   \item $\feq{\Omega, \Omega'}{A}$;
   \item $\lstack{C}$ and $\lstack{P}$ are well-formed in relation to $A; \Gamma; \Delta_{N}$ and
   follow the same rules presented before in "Well-formed body match" as a stack
   $\lstack{C}, \lstack{P}$.
\end{itemize}
\end{definition}


\begin{definition}[Well-formed aggregate backtracking]
$\contstatea{\Delta_N}{\lstack{C} ; \lstack{P}}{\Gamma}{\Sigma}$ is well-formed
in relation to a triplet $A; \Gamma; \Delta_{N}$ iff:

\begin{itemize}[leftmargin=*]
   \item $\lstack{P}$ is composed solely of persistent frames.
   \item $\lstack{C}$ is composed of either linear or persistent frames, but the first
   frame is linear.
   \item $\lstack{C}$ and $\lstack{P}$ are well-formed in relation to $A; \Gamma; \Delta_{N}$.
\end{itemize}
\end{definition}

\begin{definition}[Well-formed stack update]
$\fixstatea{\Delta}{\Xi; \Delta'}{\lstack{C}; \lstack{P}}{\Gamma}{\Sigma}$ is
well-formed in relation to a triplet $A; \Gamma; \Delta_{N}$ iff:

\begin{itemize}[leftmargin=*]
   \item $\lstack{P}$ is composed solely of persistent frames.
   \item $\lstack{C}$ is composed of either linear or persistent frames, but the first
   frame is linear.
   \item $\lstack{C}$ and $\lstack{P}$ are well-formed in relation to $A; \Gamma; \Delta_{N}$.
   \item $\Delta, \Delta' = \Delta_{N}$
\end{itemize}
\end{definition}

\begin{definition}[Well-formed aggregate derivation]
$\derstatea{\Delta}{\Xi}{\Gamma_1}{\Delta_1}{\Sigma}{\lstack{C};
      \lstack{P}}{\Omega}$
is well-formed in relation to a triplet $A; \Gamma; \Delta_{N}$ iff:

\begin{itemize}[leftmargin=*]
   \item $\lstack{P}$ is composed solely of persistent frames.
   \item $\lstack{C}$ is empty or has a single linear frame;
   \item $\lstack{C}$ and $\lstack{P}$ are well-formed in relation to $A; \Gamma; \Delta_{N}$.
\end{itemize}

\end{definition}

For the theorems that follow, we always assume that the propositions are
well-formed in relation to their main contexts and matching body, be it either a
rule body or an aggregate body.

\subsection{Soundness of matching}

The soundness theorem will be proven into two main steps. First, we prove that
performing a rule match at LLD is sound in relation to HLD and then we prove
that the derivation of head terms in LLD is also sound.

In order to prove the soundness of matching, we want to reconstitute a valid
match $\mz{\Gamma}{\Delta}{A}$ in HLD from machine steps in LLD. Our machine
specification already includes a built-in $\mz{\Gamma}{\Delta}{A}$ judgment that
can be used to prove soundness immediatelly. However, we need to prove that
every state transition preserves the well-formedness of the machine from the
previous definitions.

\begin{theorem}[Rule transitions preserve well-formedness]
Given a rule $A \lolli B$, consider a triplet $T = A; \Gamma; \Delta_{N}$.
If a state $s_1$ is well-formed in relation to $T$ and $\trans{s_1}{s_2}$ then
$s_2$ is also well-formed.
\end{theorem}
\begin{proof}
Case by case analysis.
\begin{itemize}
\item match p ok: simple manipulation of multi-set equality and use of
equivalence rules.
\item match p fail: trivial.
\item match \bang p ok: multi-set manipulation and use of
equivalence rules.
\item match \bang p fail: trivial.
\item match $\one$: use of term equivalence rules.
\item match $\bang$: use of term equivalence rules.
\item next p: simple multi-set manipulation.
\item next frame: trivial.
\item next \bang frame: trivial.
\end{itemize}
\end{proof}

Given this result, we now need to prove that from an initial matching state, we
end up with a final matching state. The final matching state will include the
soundness result since all state transitions preserve well-formedness.
However, LLD may fail during matching, therefore the match lemma needs to handle
unsuccessful matches. In order to be able to use induction, we must assume a
general matching state that  already contains some continuation frames in
stack $\lstack{C}$. The lemma also needs to relate the matching state with the
backtracking state since there is a need to backtrack during the matching
process. Apart from an unsuccessful match, we deal with two situations
during a successful match: (1) we succeed without needing to backtrack to a
frame in stack $\lstack{C}$ or (2) we need to backtrack to a frame in
$\lstack{C}$. The complete lemma is stated and proven below.


\begin{lemma}[Body match result]\label{thm:body_match}
   
Given a rule $A \lolli H$, consider a triplet $T = A; \Gamma; \Delta_{N}$ and a
context $\Delta_{N} = \Delta_1, \Delta_2, \Xi$.

If $s_1 = \matstate{A \lolli B}{\rulestk}{\lstack{C}}{\Gamma}{\Delta_1,
   \Delta_2}{\Omega}{\Delta' \rightarrow \Omega'}$
is well-formed in relation to $T$ and $\transs{s_1}{s_2}$ then:

\begin{itemize}[leftmargin=*]
   \item Match succeeds with no backtracking to frames of stack $\lstack{C}$:
   \begin{itemize}[leftmargin=\secondm]
      \item $s_2 = \matstate{A \lolli B}{\rulestk}{\lstack{C''},
         \lstack{C}}{\Gamma}{\Delta_1}{\cdot}{\Delta', \Delta_2 \rightarrow
            \Omega' \otimes split(\Omega)}$
      %\item $\mo \Gamma; \Delta_1; \Xi, \Delta_2; \cdot; H; \lstack{C''}, \lstack{C}; \lstack{R} \rightarrow \outsem$ (well-formed in relation to $T$)
      %\item $\mo \Gamma; \Delta_1; \Xi, \Delta_2; \Omega; H; \lstack{C}; (\cdot, \Delta_N) \rightarrow \outsem$ (well-formed in relation to $T$)
   \end{itemize}

   
   \item Match fails:
   \begin{itemize}[leftmargin=\secondm]
      \item $s_2 = \contstate{A \lolli B}{\rulestk}{\cdot}{\Gamma}$
   \end{itemize}

   \item Match succeeds with backtracking to a linear frame:
   \begin{itemize}[leftmargin=\secondm]
      \item $s_2 = \matstate{A \lolli B}{\rulestk}{\lstack{C'''}, f',
         \lstack{C''}}{\Gamma}{\Delta_c}{\cdot}{\Delta'_f, p_2, \Delta''_f \rightarrow
         \Omega_f \otimes p \otimes split(\Omega'_f)}$

      \item $\lstack{C} = \lstack{C'}, f, \lstack{C''}$

      \item $f = \lframe{\Delta_a}{\Delta_{b_1}, p_2, \Delta_{b_2}}{p}{\Omega_f}{\Delta_f'}{\Omega_f'}$
      turns into $f' = \lframe{\Delta_a, \Delta_{b_1}, p_2}
{\Delta_{b_2}}{p}{\Omega_f}{\Delta_f'}{\Omega_f'}$

      \item $\Delta_c = (\Delta_1, \Delta_2, \Delta') - (\Delta'_f, \Delta''_f, p_2)$
   \end{itemize}

   \item Match succeeds with backtracking to a persistent frame:
   \begin{itemize}[leftmargin=\secondm]
      \item $s_2 = \matstate{A \lolli B}{\rulestk}{\lstack{C'''}, f',
         \lstack{C''}}{\Gamma}{\Delta_{c_1}}{\cdot}{\Delta'_f, \Delta_{c_2}
            \rightarrow \Omega_f \otimes \bang p \otimes split(\Omega'_f)}$
      \item $\lstack{C} = \lstack{C'}, f, \lstack{C''}$

      \item $f = \pframe{\Gamma_1, p_2,
         \Gamma_2}{\Delta_{c_1}, \Delta_{c_2}}{\bang p}{\Omega_f}{\Delta'_f}{\Omega'_f}$
      turns into $f' = \pframe{\Gamma_2}{\Delta_{c_1},
         \Delta_{c_2}}{\bang p}{\Omega_f}{\Delta'_f}{\Omega'_f}$

      \item $\Delta_{c_1}, \Delta_{c_2} = (\Delta_1, \Delta_2,
            \Delta') - \Delta'_f$
   \end{itemize}


\end{itemize}

If $s_1 = \contstate{A \lolli B}{\rulestk}{\lstack{C}}{\Gamma}$ 
is well-formed in relation to $T$ and $\transs{s_1}{s_2}$ then either:

\begin{itemize}[leftmargin=*]
   
   \item Match fails:
   \begin{itemize}[leftmargin=\secondm]
      \item $s_2 = \contstate{A \lolli B}{\rulestk}{\cdot}{\Gamma}$
   \end{itemize}

   \item Match succeeds with backtracking to a linear frame:
   \begin{itemize}[leftmargin=\secondm]
      \item $s_2 = \matstate{A \lolli B}{\rulestk}{\lstack{C'''}, f',
         \lstack{C''}}{\Gamma}{\Delta_c}{\cdot}{\Delta'_f, p_2, \Delta''_f \rightarrow
         \Omega_f \otimes p \otimes split(\Omega'_f)}$

      \item $\lstack{C} = \lstack{C'}, f, \lstack{C''}$

      \item $f = \lframe{\Delta_a}{\Delta_{b_1}, p_2, \Delta_{b_2}}{p}{\Omega_f}{\Delta_f'}{\Omega_f'}$
      turns into $f' = \lframe{\Delta_a, \Delta_{b_1}, p_2}
{\Delta_{b_2}}{p}{\Omega_f}{\Delta_f'}{\Omega_f'}$

      \item $\Delta_c = (\Delta_1, \Delta_2, \Delta') - (\Delta'_f, \Delta''_f, p_2)$
   \end{itemize}

   \item Match succeeds with backtracking to a persistent frame:
   \begin{itemize}[leftmargin=\secondm]
      \item $s_2 = \matstate{A \lolli B}{\rulestk}{\lstack{C'''}, f',
         \lstack{C''}}{\Gamma}{\Delta_{c_1}}{\cdot}{\Delta'_f, \Delta_{c_2}
            \rightarrow \Omega_f \otimes \bang p \otimes split(\Omega'_f)}$
      \item $\lstack{C} = \lstack{C'}, f, \lstack{C''}$

      \item $f = \pframe{\Gamma_1, p_2,
         \Gamma_2}{\Delta_{c_1}, \Delta_{c_2}}{\bang p}{\Omega_f}{\Delta'_f}{\Omega'_f}$
      turns into $f' = \pframe{\Gamma_2}{\Delta_{c_1},
         \Delta_{c_2}}{\bang p}{\Omega_f}{\Delta'_f}{\Omega'_f}$

      \item $\Delta_{c_1}, \Delta_{c_2} = (\Delta_1, \Delta_2,
            \Delta') - \Delta'_f$
   \end{itemize}

\end{itemize}
\end{lemma}

\begin{proof}
   Proof by lexicographic induction on the state transitions. First on the size of
   $\Omega$ and then on the size of the second argument of the linear frame or
   on the first argument of the linear frame and then on the size of the stack
   $\lstack{C}$.
   Sub-cases:

\begin{itemize}[leftmargin=*]
\item match p ok

Induction on the state with a new frame ($\Omega$ is smaller). Trivial if match
fails, otherwise it succeeds by adding new frames (including the new frame) or by backtracking.

\item match p fail

State gets smaller (see next).

\item match \bang p ok

Use the strategy used for match p ok.

\item match \bang p fail: state gets smaller.

\item match $\one$: trivial because $split$ removes $\one$.

\item match $\bang$: same.
\item next p

Frame gets smaller so we can use the induction hypothesis:
\begin{itemize}
   \item Match fails: trivial.
   \item Match succeeds with no backtracking: the frame that was updated is the
   successful frame to backtrack to.
   \item Match succeeds with backtracking: the frame $f \in \lstack{C}$ is the
   frame we need.
\end{itemize}

\item next frame: stack gets smaller.
\item next \bang frame: stack gets smaller (see above).
\end{itemize}

\end{proof}


For the induction hypothesis to be applicable in Lemma~\ref{thm:body_match}
there must be a relation between the machine states.  We can define a
lexicographic ordering $s_1 \prec s_2$, meaning that $s_1$ has a smaller number
of remaining steps than state $s_2$.  The specific ordering is as follows:

\begin{enumerate}[leftmargin=*]
   \item $\contstate{A \lolli B}{\rulestk}{\lstack{C}}{\Gamma} \prec
   \contstate{A \lolli B}{\rulestk}{\lstack{C'}, \lstack{C}}{\Gamma}$

   The continuation must use the top of the stack $\lstack{C'}$ before using
   $\lstack{C}$;

   \item $\contstate{A \lolli B}{\rulestk}{\lframe{\Delta,
      \Delta_1}{\Delta_2}{p}{\Omega}{\Delta_f}{\Omega_f}, \lstack{C}}{\Gamma}
      \prec \contstate{A \lolli B}{\rulestk}{\lframe{\Delta}{\Delta_1,
         \Delta_2}{p}{\Omega}{\Delta_f}{\Omega_f}, \lstack{C}}{\Gamma}$

   A continuation frame with more candidates has more steps to do than a frame with less candidates;

   \item $\contstate{A \lolli B}{\rulestk}{\pframe{\Gamma_1}{\Delta}{\bang
      p}{\Omega}{\Delta_f}{\Omega_f}}{\Gamma} \prec \contstate{A \lolli
         B}{\rulestk}{\pframe{\Gamma_1, \Gamma_2}{\Delta}{\bang
            p}{\Omega}{\Delta_f}{\Omega_f}}{\Gamma}$

      Same as the previous one;

   \item $\contstate{A \lolli B}{\rulestk}{\lstack{C}}{\Gamma} \prec
      \matstate{A \lolli B}{\rulestk}{\lstack{C'},
         \lstack{C}}{\Gamma}{\Delta}{\Omega}{\Delta' \rightarrow \Omega'}$

   \item $\matstate{A \lolli
      B}{\rulestk}{\lstack{C}}{\Gamma}{\Delta}{\Omega}{\Delta' \rightarrow
         \Omega'} \prec \contstate{A \lolli B}{\rulestk}{\lstack{C'},
            \lstack{C}}{\Gamma}$
   \item $\matstate{A \lolli B}{\rulestk}{\lstack{C'},
      \lstack{C}}{\Gamma}{\Delta_1}{\Omega_1}{\Delta'_1 \rightarrow \Omega'_1} \prec
      \matstate{A \lolli
         B}{\rulestk}{\lstack{C}}{\Gamma}{\Delta_2}{\Omega_2}{\Delta'_2 \rightarrow
            \Omega'_2}$

   Adding continuation frames to the stack makes the proof smaller as long as
   $\Omega$ is also smaller; 

   \item $\matstate{A \lolli B}{\rulestk}{\lstack{C'}, \lframe{\Delta_f,
      \Delta_{f_1}}{\Delta_{f_2}}{p}{\Omega_f}{\Delta'_f}{\Omega'_f},
   \lstack{C}}{\Gamma}{\Delta_1}{\Omega_1}{\Delta'_1 \rightarrow \Omega'_1}$\\
   \hspace*{1cm} $\prec \matstate{A \lolli B}{\rulestk}{\lstack{C''},
      \lframe{\Delta_f}{\Delta_{f_1},
         \Delta_{f_2}}{p}{\Omega_f}{\Delta'_f}{\Omega'_f},
      \lstack{C}}{\Gamma}{\Delta_2}{\Omega_2}{\Delta'_2 \rightarrow \Omega'_2}$

   \item $\matstate{A \lolli B}{\rulestk}{\lstack{C'},
      \pframe{\Gamma_1}{\Delta_f}{\bang p}{\Omega_f}{\Delta_f}{\Omega_f},
   \lstack{C}}{\Gamma}{\Delta_1}{\Omega_1}{\Delta'_1 \rightarrow \Omega'_1}$\\
   \hspace*{1cm} $\prec \matstate{A \lolli B}{\rulestk}{\lstack{C''},
      \pframe{\Gamma_1, \Gamma_2}{\Delta_f}{\bang p}{\Omega_f}{\Delta'_f}{\Omega'_f},
      \lstack{C}}{\Gamma}{\Delta_2}{\Omega_2}{\Delta'_2 \rightarrow \Omega'_2}$

\end{enumerate}

\subsection{Soundness of derivation}

Proving that the derivation of the head of the rule is sound is trivial except
for comprehensions and aggregates. LLD deterministically computes the number of
available aggregates to apply while HLD "guesses" the number of required
derivations.  In the next two sections, we show how to prove the soundness of
aggregates. The strategy for proving for proving both is identical due to their
inherient similarities.

\subsection{Aggregate soundness}


Proving that deriving an aggregate in LLD is sound in relation to HLD is built
from 4 results: (1) proving that matching the aggregate's LHS is sound in
relation to HLD; (2) proving that updating the continuation stacks makes them
suitable for use in the next aggregate applications; (3) proving that deriving
the aggregate's RHS is sound in relation to HLD; (4) proving that we can
apply as many aggregates as the database allows.

\begin{lemma}[Aggregate LHS match]\label{thm:aggregate_body_match}
Consider an aggregate $\m{agg}$, where $\Pi(\m{agg}) = \forall_{\widehat{v}, \Sigma'}.
   (\defstwo{agg}{\widehat{v}}{\Sigma'} \lolli ((\lambda x. C x)\Sigma' \with (\forall_{\widehat{x}, \sigma}.
                                                (A \lolli B \otimes
                                                 \defstwo{agg}{\widehat{v}}{\sigma
                                                 ::\Sigma'}))))$,
a triplet $T = A; \Gamma; \Delta_{N}$ and a context $\Delta_{N} = \Delta_1,
  \Delta_2, \Delta'$.

  If $s_1 = \matstatea{\Delta_N}{\lstack{C};
      \lstack{P}}{\Gamma}{\Delta_1, \Delta_2}{\Omega}{\Delta' \rightarrow
         \Omega'}{\Sigma}$ is well-formed in relation to $T$ and
         $\transs{s_1}{s_2}$ then either:

\begin{itemize}[leftmargin=*]
   \item Match succeeds with no backtracking to frames of stack $\lstack{C}$
   or $\lstack{P}$ ($\lstack{C} \neq \cdot$):

   \begin{itemize}[leftmargin=\secondm]
      \item $s_2 = \matstatea{\Delta_N}{\lstack{C'}, \lstack{C}; \lstack{P'},
         \lstack{P}}{\Gamma}{\Delta_1}{\cdot}{\Delta', \Delta_2 \rightarrow \Omega' \otimes split(\Omega)}{\Sigma}$
   \end{itemize}

   \item Match succeeds with no backtracking to frames of stack $\lstack{P}$
($\lstack{C} = \cdot$):
\begin{itemize}[leftmargin=\secondm]
   \item $\mz \Gamma; \Delta_2 \rightarrow A$
   \item $\ma{AG} \Gamma; \Delta_1; \Xi_N; \Gamma_{N1}; \Delta_{N1}; \Xi,
      \Delta_2; \cdot; \lstack{C'}; \lstack{P'}, \lstack{P}; \Omega_N; \Delta_N;
      \Sigma \rightarrow \outsem$ (well-formed in relation to $T$)
\end{itemize}

\item Match succeeds with backtracking to a linear continuation frame in
stack $\lstack{C}$ ($\lstack{C} \neq \cdot$):

\begin{itemize}[leftmargin=\secondm]
   \item $\mz \Gamma; \Xi_1, \dotsc, \Xi_m, p_2, \Xi_c$
   \item $\exists_{f \in \lstack{C}}. f = (\Delta_a; \Delta_{b_1}, p_2,
         \Delta_{b_2}; p; \Xi_1, \dotsc, \Xi_m; \Omega_1, \dotsc,
         \Omega_k; \Lambda_1, \dotsc, \Lambda_m; \Upsilon_1, \dotsc,
         \Upsilon_n)$
   \item $\lstack{C} = \lstack{C'}, f, \lstack{C''}$
   \item $f$ turns into $f' = (\Delta_a, \Delta_{b_1}, p_2;
         \Delta_{b_2}; p; \Xi_1, \dotsc, \Xi_m;
         \Omega_1, \dotsc, \Omega_k; \Lambda_1, \dotsc, \Lambda_m;
         \Upsilon_1, \dotsc, \Upsilon_n)$
   \item $\ma{AG} \Gamma; \Delta_c; \Xi_N; \Gamma_{N1}; \Delta_{N1}; \Xi_1,
      \dotsc, \Xi_m, p_2, \Xi_c; \cdot; \lstack{C'''}, f', \lstack{C''}; \lstack{P};
      \Omega_N; \Delta_N; \Sigma \rightarrow \outsem$ (well-formed in relation to $T$)
   \item $\Delta_c = (\Delta_1, \Delta_2, \Xi) - (\Xi_1, \dotsc, \Xi_m,
         p_2, \Xi_c)$
\end{itemize}

\item Match succeeds with backtracking to a persistent continuation frame
in stack $\lstack{C}$ ($\lstack{C} \neq \cdot$):
\begin{itemize}[leftmargin=\secondm]
   \item $\mz \Gamma; \Delta_{c_2}, \Xi_1, \dotsc, \Xi_m \rightarrow A$
   \item $\exists_{f \in \lstack{C}}. f = [\Gamma_1, p_2, \Gamma_2; \Delta_{c_1},
      \Delta_{c_2}; \Xi_1, \dotsc, \Xi_m; \bang p; \Omega_1, \dotsc, \Omega_k;
      \Lambda_1, \dotsc, \Lambda_m; \Upsilon_1, \dotsc, \Upsilon_n]$
   \item $\lstack{C} = \lstack{C'}, f, \lstack{C''}$
   \item $f$ turns into $f' = [\Gamma_2; \Delta_{c_1}, \Delta_{c_2};
      \Xi_1, \dotsc, \Xi_m; \bang p; \Omega_1, \dotsc, \Omega_k; \Lambda_1,
      \dotsc, \Lambda_m; \Upsilon_1, \dotsc, \Upsilon_n]$
   \item $\ma{AG} \Gamma; \Delta_{c_1}; \Xi_N; \Gamma_{N1}; \Delta_{N1};
      \Delta_{c_2}, \Xi_1, \dotsc, \Xi_m; \cdot; \lstack{C'''}, f', \lstack{C''}; \lstack{P};
      \Omega_N; \Delta_N; \Sigma \rightarrow \outsem$ (well-formed in relation to $T$)
   \item $\Delta_{c_1}, \Delta_{c_2} = (\Delta_1, \Delta_2, \Xi) - (\Xi_1, \dotsc, \Xi_m)$
\end{itemize}

\item Match succeeds with backtracking to a persistent continuation frame
in stack $\lstack{P}$ ($\lstack{C} = \cdot$):
   \begin{itemize}[leftmargin=\secondm]
      \item $\mz \Gamma; \Delta_{c_2}, \Xi_1, \dotsc, \Xi_m \rightarrow A$
      \item $\exists_{f \in \lstack{P}}. f = [\Gamma_1, p_2, \Gamma_2; \Delta_{c_1}, \Delta_{c_2};
   \Xi_1, \dotsc, \Xi_m; \bang p; \Omega_1, \dotsc, \Omega_k; \Lambda_1,
   \dotsc, \Lambda_m; \Upsilon_1, \dotsc, \Upsilon_n]$
      \item $\lstack{P} = \lstack{P'}, f, \lstack{P''}$
      \item $f$ turns into $f' = [\Gamma_2; \Delta_{c_1},
         \Delta_{c_2}; \Xi_1, \dotsc, \Xi_m; \bang p; \Omega_1, \dotsc,
         \Omega_k; \Lambda_1, \dotsc, \Lambda_m; \Upsilon_1, \dotsc, \Upsilon_n]$
      \item $\ma{AG} \Gamma; \Delta_{c_1}; \Xi_N; \Gamma_{N1}; \Delta_{N1};
         \Delta_{c_2}, \Xi_1, \dotsc, \Xi_m; \cdot; \lstack{C'}; \lstack{P'''}, f', \lstack{P''};
         \Omega_N; \Delta_N; \Sigma \rightarrow \outsem$
         (well-formed in relation to $T$)
      \item $\Delta_{c_1}, \Delta_{c_2} = (\Delta_1, \Delta_2, \Xi) - (\Xi_1, \dotsc,
            \Xi_m)$
   \end{itemize}


\end{itemize}

If $\contstatea{\Delta_N}{\lstack{C}; \lstack{P}}{\Gamma}{\Sigma}$ is
well-formed in relation to $T$ then either:

\begin{itemize}[leftmargin=*]
   \item Match succeeds with no backtracking to frames of stack $\lstack{P}$
($\lstack{C} = \cdot$):
\begin{itemize}[leftmargin=\secondm]
   \item $\mz \Gamma; \Delta_2 \rightarrow A$
   \item $\ma{AG} \Gamma; \Delta_1; \Xi_N; \Gamma_{N1}; \Delta_{N1}; \Xi,
      \Delta_2; \cdot; \lstack{C'}; \lstack{P'}, \lstack{P}; \Omega_N; \Delta_N;
      \Sigma \rightarrow \outsem$ (well-formed in relation to $T$)
\end{itemize}

\item Match succeeds with backtracking to a linear continuation frame in
stack $\lstack{C}$ ($\lstack{C} \neq \cdot$):

\begin{itemize}[leftmargin=\secondm]
   \item $\mz \Gamma; \Xi_1, \dotsc, \Xi_m, p_2, \Xi_c$
   \item $\exists_{f \in \lstack{C}}. f = (\Delta_a; \Delta_{b_1}, p_2,
         \Delta_{b_2}; p; \Xi_1, \dotsc, \Xi_m; \Omega_1, \dotsc,
         \Omega_k; \Lambda_1, \dotsc, \Lambda_m; \Upsilon_1, \dotsc,
         \Upsilon_n)$
   \item $\lstack{C} = \lstack{C'}, f, \lstack{C''}$
   \item $f$ turns into $f' = (\Delta_a, \Delta_{b_1}, p_2;
         \Delta_{b_2}; p; \Xi_1, \dotsc, \Xi_m;
         \Omega_1, \dotsc, \Omega_k; \Lambda_1, \dotsc, \Lambda_m;
         \Upsilon_1, \dotsc, \Upsilon_n)$
   \item $\ma{AG} \Gamma; \Delta_c; \Xi_N; \Gamma_{N1}; \Delta_{N1}; \Xi_1,
      \dotsc, \Xi_m, p_2, \Xi_c; \cdot; \lstack{C'''}, f', \lstack{C''}; \lstack{P};
      \Omega_N; \Delta_N; \Sigma \rightarrow \outsem$ (well-formed in relation to $T$)
   \item $\Delta_c = (\Delta_1, \Delta_2, \Xi) - (\Xi_1, \dotsc, \Xi_m,
         p_2, \Xi_c)$
\end{itemize}

\item Match succeeds with backtracking to a persistent continuation frame
in stack $\lstack{C}$ ($\lstack{C} \neq \cdot$):
\begin{itemize}[leftmargin=\secondm]
   \item $\mz \Gamma; \Delta_{c_2}, \Xi_1, \dotsc, \Xi_m \rightarrow A$
   \item $\exists_{f \in \lstack{C}}. f = [\Gamma_1, p_2, \Gamma_2; \Delta_{c_1},
      \Delta_{c_2}; \Xi_1, \dotsc, \Xi_m; \bang p; \Omega_1, \dotsc, \Omega_k;
      \Lambda_1, \dotsc, \Lambda_m; \Upsilon_1, \dotsc, \Upsilon_n]$
   \item $\lstack{C} = \lstack{C'}, f, \lstack{C''}$
   \item $f$ turns into $f' = [\Gamma_2; \Delta_{c_1}, \Delta_{c_2};
      \Xi_1, \dotsc, \Xi_m; \bang p; \Omega_1, \dotsc, \Omega_k; \Lambda_1,
      \dotsc, \Lambda_m; \Upsilon_1, \dotsc, \Upsilon_n]$
   \item $\ma{AG} \Gamma; \Delta_{c_1}; \Xi_N; \Gamma_{N1}; \Delta_{N1};
      \Delta_{c_2}, \Xi_1, \dotsc, \Xi_m; \cdot; \lstack{C'''}, f', \lstack{C''}; \lstack{P};
      \Omega_N; \Delta_N; \Sigma \rightarrow \outsem$ (well-formed in relation to $T$)
   \item $\Delta_{c_1}, \Delta_{c_2} = (\Delta_1, \Delta_2, \Xi) - (\Xi_1, \dotsc, \Xi_m)$
\end{itemize}

\item Match succeeds with backtracking to a persistent continuation frame
in stack $\lstack{P}$ ($\lstack{C} = \cdot$):
   \begin{itemize}[leftmargin=\secondm]
      \item $\mz \Gamma; \Delta_{c_2}, \Xi_1, \dotsc, \Xi_m \rightarrow A$
      \item $\exists_{f \in \lstack{P}}. f = [\Gamma_1, p_2, \Gamma_2; \Delta_{c_1}, \Delta_{c_2};
   \Xi_1, \dotsc, \Xi_m; \bang p; \Omega_1, \dotsc, \Omega_k; \Lambda_1,
   \dotsc, \Lambda_m; \Upsilon_1, \dotsc, \Upsilon_n]$
      \item $\lstack{P} = \lstack{P'}, f, \lstack{P''}$
      \item $f$ turns into $f' = [\Gamma_2; \Delta_{c_1},
         \Delta_{c_2}; \Xi_1, \dotsc, \Xi_m; \bang p; \Omega_1, \dotsc,
         \Omega_k; \Lambda_1, \dotsc, \Lambda_m; \Upsilon_1, \dotsc, \Upsilon_n]$
      \item $\ma{AG} \Gamma; \Delta_{c_1}; \Xi_N; \Gamma_{N1}; \Delta_{N1};
         \Delta_{c_2}, \Xi_1, \dotsc, \Xi_m; \cdot; \lstack{C'}; \lstack{P'''}, f', \lstack{P''};
         \Omega_N; \Delta_N; \Sigma \rightarrow \outsem$
         (well-formed in relation to $T$)
      \item $\Delta_{c_1}, \Delta_{c_2} = (\Delta_1, \Delta_2, \Xi) - (\Xi_1, \dotsc,
            \Xi_m)$
   \end{itemize}

\end{itemize}
\end{lemma}

Proving this particular lemma follows the same strategy used for
Lemma~\ref{thm:body_match}. Next, we need to prove that, when matching succeeds,
the continuation stack is corrected for the next application of the aggregate.
Note that the aggregate value is appended to $\Sigma$ after the stack is
corrected.

\begin{theorem}[From update to derivation]\label{thm:agg_from_update_to_derivation}
Consider an aggregate $\m{agg}$, where $\Pi(\m{agg}) = \forall_{\widehat{v}, \Sigma'}.
   (\defstwo{agg}{\widehat{v}}{\Sigma'} \lolli ((\lambda x. C x)\Sigma' \with (\forall_{\widehat{x}, \sigma}.
                                                (A \lolli B \otimes
                                                 \defstwo{agg}{\widehat{v}}{\sigma
                                                 ::\Sigma'}))))$,
a triplet $T = A; \Gamma; \Delta_{N}$ and that $\Delta_{N} = \Delta,
  \Delta'$.
A well-formed stack update
$\fixstatea{\Delta}{\Xi; \Delta'}{\lstack{C}; \lstack{P}}{\Gamma}{\Sigma}$
implies
$\derstatea{\Delta}{\Xi, \Delta'}{\Gamma_{N1}}{\Delta_{N1}}{V :: \Sigma}{f';
   \lstack{P'}}{B\{\Psi(\widehat{x}), V / \widehat{x}, \sigma\}}$,
where:

\begin{itemize}[leftmargin=*]
   \item If $\lstack{C} = \cdot$ then $\lstack{C'} = \cdot$

   \item If $\lstack{C} = \lstack{C''},
   \lframe{\Delta_a}{\Delta_b}{p}{\Omega}{\cdot}{\Omega'}$
   then $\lstack{C'} = \lframe{\Delta_a - \Delta'}{\Delta_b -
      \Delta'}{p}{\Omega}{\cdot}{\Omega'}$

   \item $\lstack{P'}$ is the transformation of stack $\lstack{P}$, where for every frame $f \in
   \lstack{P}$ of the form $\pframe{\Gamma'}{\Delta_N}{\bang
      p}{\Omega}{\cdot}{\Omega'}$
   will turn into $f' = \pframe{\Gamma'}{\Delta_N-\Delta'}{\bang
      p}{\Omega}{\cdot}{\Omega'}$

\end{itemize}
\end{theorem}
\begin{proof}
Use induction on the size of the stack $\lstack{C}$.
\end{proof}


\begin{corollary}[Match to derivation]\label{thm:agg_match_to_derivation}
Consider an aggregate $\m{agg}$, where $\Pi(\m{agg}) = \forall_{\widehat{v}, \Sigma'}.
   (\defstwo{agg}{\widehat{v}}{\Sigma'} \lolli ((\lambda x. C x)\Sigma' \with (\forall_{\widehat{x}, \sigma}.
                                                (A \lolli B \otimes
                                                 \defstwo{agg}{\widehat{v}}{\sigma
                                                 ::\Sigma'}))))$,
a triplet $T = A; \Gamma; \Delta_{N}$ and that $\Delta_{N} = \Delta,
  \Delta'$.

A well-formed $\matstatea{\Delta_N}{\lstack{C}; \lstack{P}}{\Gamma}{\Delta}{\cdot}{\Delta' \rightarrow \Omega'}{\Sigma}$
implies
$\derstatea{\Delta}{\Xi, \Delta'}{\Gamma_{N1}}{\Delta_{N1}}{V :: \Sigma}{f';
   \lstack{P'}}{B\{\Psi(\widehat{x}), V / \widehat{x}, \sigma\}}$,
where:
   
\begin{itemize}[leftmargin=*]
   \item If $\lstack{C} = \cdot$ then $\lstack{C'} = \cdot$

   \item If $\lstack{C} = \lstack{C''},
   \lframe{\Delta_a}{\Delta_b}{p}{\Omega}{\cdot}{\Omega'}$
   then $\lstack{C'} = \lframe{\Delta_a - \Delta'}{\Delta_b -
      \Delta'}{p}{\Omega}{\cdot}{\Omega'}$

   \item $\lstack{P'}$ is the transformation of stack $\lstack{P}$, where for every frame $f \in
   \lstack{P}$ of the form $\pframe{\Gamma'}{\Delta_N}{\bang
      p}{\Omega}{\cdot}{\Omega'}$
   will turn into $f' = \pframe{\Gamma'}{\Delta_N-\Delta'}{\bang
      p}{\Omega}{\cdot}{\Omega'}$

\end{itemize}
\end{corollary}

\begin{proof}
Invert the assumption and then apply Theorem~\ref{thm:agg_from_update_to_derivation}.
\end{proof}


\paragraph{Aggregate Derivation}

We have just seen that after a single aggregate application, we add a value $V$
to the $\Sigma$ context and that the continuation stacks are now valid.
Now, we need to prove that deriving the aggregate's RHS
is sound in relation to HLD by using the new stacks.

\begin{theorem}[Aggregate derivation soundness]\label{thm:aggregate_derivation}
If $\derstatea{\Delta_N}{\Xi_N}{\Gamma_{N1}}{\Delta_{N1}}{\Sigma}{\lstack{C}; \lstack{P}}{\Omega_1,
\dotsc, \Omega_n}$ then:

\begin{itemize}[leftmargin=*]
   \item $\derstatea{\Delta_N}{\Xi_N}{\Gamma_{N1}, \Gamma_1, \dotsc,
      \Gamma_n}{\Delta_{N1}, \Delta_1, \dotsc, \Delta_n}{\Sigma}{\lstack{C};
         \lstack{P}}{\cdot}$

   \item If $\dz{\Gamma}{\Pi}{\Delta_N}{\Xi_N}{\Gamma_{N1}, \Gamma_1, \dotsc,
         \Gamma_n}{\Delta_{N1}, \Delta_1, \dotsc, \Delta_n}{\Omega_x}{\outsem}$ then\\
         \hspace{2cm}
   $\dz{\Gamma}{\Pi}{\Delta_N}{\Xi_N}{\Gamma_{N1}}{\Delta_{N1}}{\Omega_1, \dotsc,
   \Omega_n, \Omega_x}{\outsem}$

\end{itemize}
\end{theorem}

\begin{proof}
Straightforward induction on $\Omega_1, \dotsc, \Omega_n$.
\end{proof}


\paragraph{Multiple Aggregate Derivation} We now prove that it is possible
to apply an aggregate several times in order to a get multiple values (one per
application). In turn, we also conclude important results for the
soundness of the aggregate computation mechanism.

\begin{theorem}[Multiple aggregate derivation]\label{thm:multiple_aggregate_derivation}
Consider an aggregate $\m{agg}$, where $\Pi(\m{agg}) = \forall_{\widehat{v}, \Sigma'}.
   (\defstwo{agg}{\widehat{v}}{\Sigma'} \lolli ((\lambda x. C x)\Sigma' \with (\forall_{\widehat{x}, \sigma}.
                                                (A \lolli B \otimes
                                                 \defstwo{agg}{\widehat{v}}{\sigma
                                                 ::\Sigma'}))))$,
and a triplet $T = A; \Gamma; \Delta_{N}$.
Assume that there exists $n \geq 0$ applications of $\m{agg}$
where the $i_{th}$ application is related to the following information:
\begin{description}
   \item[$\Delta_i$]: context of derived linear facts;
   \item[$\Gamma_i$]: context of derived persistent facts;
   \item[$\Xi_i$]: context of consumed linear facts;
   \item[$V_i$]: a value representing the aggregate application;
   \item[$\Psi_i$]: context representing new variable bindings for the aggregate.
\end{description}

Since each application consumes $\Xi_i$ then the initial context $\Delta_N =
\Delta, \Xi_1, \dotsc, \Xi_n$. We now define the two main implications of the
theorem.

\begin{itemize}[leftmargin=*]
   \item Assume that $\Delta_N = \Delta_a, \Delta_b$, $\Delta_b =
   \Delta'_b, p_1$ and there is a frame $f = \lframe{\Delta_a,
      p_1}{\Delta'_b}{p}{\Omega}{\cdot}{\Omega'_f}$

   If $s_1 = \matstatea{\Delta, \Xi_1, \dotsc, \Xi_n}{f; \lstack{P}}{\Gamma}{\Delta_a,
      \Delta'_b}{\Omega}{p_1 \rightarrow \Omega'_f \otimes p}{\Sigma}$
      (well-formed in relation to $T$) and $\transs{s_1}{s_2}$ then:

      \begin{itemize}[leftmargin=\secondm]
      \item $n$ values $V_i$ ($\Sigma' = V_n :: \dots :: V_1 :: \Sigma$)
      \item $n$ aggregate applications are derived:\\
      $s_2 = \derstatex{\Gamma}{\Delta}{\Xi, \Xi_1, \dotsc, \Xi_n}{\Gamma_{N1},
      \Gamma_1, \dotsc, \Gamma_n}{\Delta_{N1}, \Delta_1, \dotsc,
         \Delta_n}{(\lambda x.  C\{\Psi(\widehat{v})/\widehat{v}\} x) \Sigma,
            \Omega_N}$
      \item $n$ soundness proofs for the $n$ aggregate matches:
      \begin{itemize}[leftmargin=\thirdm]
         \item $\mz{\Gamma}{\Xi_1}{A}$
         \item $\dots$
         \item $\mz{\Gamma}{\Xi_n}{A}$
      \end{itemize}

      \item $n$ derivation implications for HLD: \\
      If $\dz{\Gamma}{\Pi}{\Delta, \Xi_{i+1}, \dotsc, \Xi_{n}}{\Xi, \Xi_1,
         \dotsc, \Xi_i}{\Gamma_{N1}, \Gamma_1, \dotsc, \Gamma_i}{\Delta_{N1},
            \Delta_1, \dotsc, \Delta_i}{\Omega_x}{\outsem}$ then\\
            \hspace{5cm} $\dz{\Gamma}{\Pi}{\Delta, \Xi_{i+1}, \dotsc, \Xi_{n}}{\Xi, \Xi_1,
            \dotsc, \Xi_i}{\Gamma_{N1}, \Gamma_1, \dotsc, \Gamma_{i-1}}{\Delta_{N1},
            \Delta_1, \dotsc, \Delta_{i-1}}{B, \Omega_x}{\outsem}$
   \end{itemize}


   \item Assume that there is a frame $f = \pframe{\Gamma'}{\Delta_N}{\bang
      p}{\Omega}{\cdot}{\Omega_f}$.
      
   If $s_1 = \matstatea{\Delta_N}{\cdot;
      f, \lstack{P}}{\Gamma}{\Delta_N}{\Omega}{\cdot \rightarrow \bang p \otimes \Omega_f}{\Sigma}$
      (well-formed in relation to $T$) and $\transs{s_1}{s_2}$ then:

      \begin{itemize}[leftmargin=\secondm]
      \item $n$ values $V_i$ ($\Sigma' = V_n :: \dots :: V_1 :: \Sigma$)
      \item $n$ aggregate applications are derived:\\
      $s_2 = \derstatex{\Gamma}{\Delta}{\Xi, \Xi_1, \dotsc, \Xi_n}{\Gamma_{N1},
      \Gamma_1, \dotsc, \Gamma_n}{\Delta_{N1}, \Delta_1, \dotsc,
         \Delta_n}{(\lambda x.  C\{\Psi(\widehat{v})/\widehat{v}\} x) \Sigma,
            \Omega_N}$
      \item $n$ soundness proofs for the $n$ aggregate matches:
      \begin{itemize}[leftmargin=\thirdm]
         \item $\mz{\Gamma}{\Xi_1}{A}$
         \item $\dots$
         \item $\mz{\Gamma}{\Xi_n}{A}$
      \end{itemize}

      \item $n$ derivation implications for HLD: \\
      If $\dz{\Gamma}{\Pi}{\Delta, \Xi_{i+1}, \dotsc, \Xi_{n}}{\Xi, \Xi_1,
         \dotsc, \Xi_i}{\Gamma_{N1}, \Gamma_1, \dotsc, \Gamma_i}{\Delta_{N1},
            \Delta_1, \dotsc, \Delta_i}{\Omega_x}{\outsem}$ then\\
            \hspace{5cm} $\dz{\Gamma}{\Pi}{\Delta, \Xi_{i+1}, \dotsc, \Xi_{n}}{\Xi, \Xi_1,
            \dotsc, \Xi_i}{\Gamma_{N1}, \Gamma_1, \dotsc, \Gamma_{i-1}}{\Delta_{N1},
            \Delta_1, \dotsc, \Delta_{i-1}}{B, \Omega_x}{\outsem}$
   \end{itemize}

\end{itemize}
   
\end{theorem}
\begin{proof}
By mutual induction, first on either the size of $\Delta'_b$ (second argument of
the linear continuation frame) or $\Gamma'$ (second argument of the
persistent frame in $\lstack{P}$) and then on the size of $\lstack{C},
\lstack{P}$.  We only show how to prove the first implication since the
second implication is proven in a similar way.

$\transs{\matstatea{\Delta, \Xi_1, \dotsc, \Xi_n}{f; \lstack{P}}{\Gamma}{\Delta_a,
      \Delta'_b}{\Omega}{p_1 \rightarrow \Omega'_f \otimes p}{\Sigma}}{s_2}$
         \hfill (1) assumption\\

By applying Lemma~\ref{thm:aggregate_body_match} to (1), we get:

\begin{itemize}[leftmargin=*]
   \item Failure:
   
   $s_2 = \derstatex{\Gamma}{\Delta_N}{\Xi}{\Gamma_{N1}}{\Delta_{N1}}
{(\lambda x.  C\{\Psi(\widehat{v})/\widehat{v}\} x) \Sigma,
   \Omega_N}$ \hfill (2) from lemma, thus $n = 0$\\
   
   \item Success with no backtracking to frames of stack $\lstack{C}$ or
   $\lstack{P}$:
   
      $s_2 = \matstatea{\Delta_N}{\lstack{C'}, f; \lstack{P}}{\Gamma}
         {\Delta, \Xi_2, \dotsc, \Xi_n}{\cdot}{p_1, \Xi'_1 \rightarrow \Omega'_f \otimes p
                                \otimes split(\Omega)}{\Sigma}$ \hfill (2) from
                                lemma \\
      $\Xi_1 = \Xi'_1, p_1$ \hfill (3) by definition \\

      $\feq{A}{\Omega'_f \otimes p \otimes split(\Omega)}$ \hfill (4) by
      well-formedness \\
      $\mz{\Gamma}{\Xi_1}{A}$ \hfill (5) from match equivalence
      theorem and split equivalence on (4) \\

      $f = \lframe{\Delta_a, p_1}{\Delta'_b}{p}{\Omega}{\cdot}{\Omega'_f}$
      \hfill (6) definition \\

      Now, we apply Corollary~\ref{thm:agg_match_to_derivation} on (2) \\

      $f' = \lframe{\Delta_a, p_1 - \Xi_1}{\Delta_b -
         \Xi_1}{p}{\Omega}{\cdot}{\Omega'_f}$ \hfill (7) from the Corollary \\


      $\derstatea{\Delta, \Xi_2, \dotsc, \Xi_n}{\Xi,
         \Xi_1}{\Gamma_{N1}}{\Delta_{N1}}{V_1 :: \Sigma}{f';
            \lstack{P'}}{B\{\Psi(\widehat{x}), V / \widehat{x}, \sigma\}}$
      \dots \hfill (8) from the Corollary \\

      $\derstatea{\Delta, \Xi_2, \dotsc, \Xi_n}{\Xi, \Xi_1}{\Gamma_{N1},
         \Gamma_1}{\Delta_{N1}, \Delta_1}{\Sigma}{f'; \lstack{P'}}{\cdot}$
      \\ \dots \hfill (9) applying Theorem~\ref{thm:aggregate_derivation} on (8)

      If $\dz{\Gamma}{\Pi}{\Delta, \Xi_2, \dotsc, \Xi_n}{\Xi, \Xi_1}{\Gamma_{N1},
         \Gamma_1}{\Delta_{N1}, \Delta_1}{\Omega_x}{\outsem}$ then \\
         \hspace{2cm}
         $\dz{\Gamma}{\Pi}{\Delta, \Xi_2, \dotsc,
            \Xi_n}{\Xi, \Xi_1}{\Gamma_{N1}}{\Delta_{N1}}{B\{\Psi(\widehat{x}), V
               / \widehat{x}, \sigma\}, \Omega_x}{\outsem}$
            \hfill (10) from
      Theorem~\ref{thm:aggregate_derivation} on (8) \\

      $\contstatea{\Delta, \Xi_2, \dotsc, \Xi_n}{f' ; \lstack{P'}}{\Gamma}{V_1
         :: \Sigma}$ \hfill (11) next state of (9) \\

      By executing the next transition on (11) we either fail because there are
      no more candidates or no more frames and thus $n = 1$ or we have a new
      match from which we can apply the inductive hypothesis (smaller number of
      candidates and/or frames) to get the remaining $n - 1$ aggregate
      values.
      
   \item Success with backtracking to the linear continuation frame of stack $\lstack{C}$:
      
      $s_2 = \matstatea{\Delta_N}{\lstack{C'},
         f';\lstack{P}}{\Gamma}{\Delta, \Xi_2, \dotsc, \Xi_n}{\cdot}{\Delta'_f, p_2, \Xi'_1 \rightarrow \Omega'_f \otimes p
                  \otimes split(\Omega)}{\Sigma}$ \hfill (2) from lemma \\

      $\Xi_1 = \Delta'_f, p_2, \Xi'_1$ \hfill (3) by definition \\

      $\feq{A}{\Omega'_f \otimes p \otimes split(\Omega)}$ \hfill (4) by
      well-formedness \\
      $\mz{\Gamma}{\Xi_1}{A}$ \hfill (5) from match equivalence
      theorem and split equivalence on (4) \\

      $f = \lframe{\Delta_a, p_1}{\Delta'''_b, p_2, \Delta''_b}{p}{\Omega}{\cdot}{\Omega'_f}$
      (6) frame to backtrack to \\
      turns into $f' = \lframe{\Delta_a, p_1, \Delta'''_b, p_2}{\Delta''_b}{p}{\Omega}{\cdot}{\Omega'_f}$
      \hfill (4) resulting frame \\

      Use the same approach as the case with no backtracking.
      
   \item Success with backtracking to a persistent continuation frame of stack
   $\lstack{P}$:

      Use the same approach as before.
      
\end{itemize}
\end{proof}

This last theorem proves that from a certain initial continuation stack, we are
able to apply the aggregate multiple times (until the stack is exhausted). The
results of the theorem allows us to rebuild the proof tree in HLD since we get
the HLD matching and derivation propositions. What remains to be done is to
prove that we do the same for an empty continuation stack.

\begin{lemma}[All aggregates]\label{thm:aggregates}
Consider an aggregate $\m{agg}$, where $\Pi(\m{agg}) = \forall_{\widehat{v}, \Sigma'}.
   (\defstwo{agg}{\widehat{v}}{\Sigma'} \lolli ((\lambda x. C x)\Sigma' \with (\forall_{\widehat{x}, \sigma}.
                                                (A \lolli B \otimes
                                                 \defstwo{agg}{\widehat{v}}{\sigma
                                                 ::\Sigma'}))))$,
and a triplet $T = A; \Gamma; \Delta_{N}$.
Assume that there exists $n \geq 0$ applications of $\m{agg}$
where the $i_{th}$ application is related to the following information:
\begin{description}
   \item[$\Delta_i$]: context of derived linear facts;
   \item[$\Gamma_i$]: context of derived persistent facts;
   \item[$\Xi_i$]: context of consumed linear facts;
   \item[$V_i$]: a value representing the aggregate application;
   \item[$\Psi_i$]: context representing new variable bindings for the aggregate.
\end{description}

Since each application consumes $\Xi_i$ then the initial context $\Delta_N =
\Delta, \Xi_1, \dotsc, \Xi_n$.

If $s_1 = \matstatea{\Delta, \Xi_1, \dotsc, \Xi_n}{\cdot;
   \cdot}{\Gamma}{\Delta, \Xi_1, \dotsc, \Xi_n}{A}{\cdot \rightarrow \one}{\cdot}$
   (well-formed in relation to $T$) and $\transs{s_1}{s_2}$ then:

   \begin{itemize}[leftmargin=\secondm]
      \item $n$ values $V_i$ ($\Sigma' = V_n :: \dots :: V_1 :: \Sigma$)
      \item $n$ aggregate applications are derived:\\
      $s_2 = \derstatex{\Gamma}{\Delta}{\Xi, \Xi_1, \dotsc, \Xi_n}{\Gamma_{N1},
      \Gamma_1, \dotsc, \Gamma_n}{\Delta_{N1}, \Delta_1, \dotsc,
         \Delta_n}{(\lambda x.  C\{\Psi(\widehat{v})/\widehat{v}\} x) \Sigma,
            \Omega_N}$
      \item $n$ soundness proofs for the $n$ aggregate matches:
      \begin{itemize}[leftmargin=\thirdm]
         \item $\mz{\Gamma}{\Xi_1}{A}$
         \item $\dots$
         \item $\mz{\Gamma}{\Xi_n}{A}$
      \end{itemize}

      \item $n$ derivation implications for HLD: \\
      If $\dz{\Gamma}{\Pi}{\Delta, \Xi_{i+1}, \dotsc, \Xi_{n}}{\Xi, \Xi_1,
         \dotsc, \Xi_i}{\Gamma_{N1}, \Gamma_1, \dotsc, \Gamma_i}{\Delta_{N1},
            \Delta_1, \dotsc, \Delta_i}{\Omega_x}{\outsem}$ then\\
            \hspace{5cm} $\dz{\Gamma}{\Pi}{\Delta, \Xi_{i+1}, \dotsc, \Xi_{n}}{\Xi, \Xi_1,
            \dotsc, \Xi_i}{\Gamma_{N1}, \Gamma_1, \dotsc, \Gamma_{i-1}}{\Delta_{N1},
            \Delta_1, \dotsc, \Delta_{i-1}}{B, \Omega_x}{\outsem}$
   \end{itemize}


\end{lemma}

\begin{proof}
Apply Lemma~\ref{thm:aggregate_body_match} to get two sub-cases:
   
\begin{itemize}[leftmargin=*]
   \item Match fails:
   
   $s_2 = \derstatex{\Gamma}{\Delta_N}{\Xi}{\Gamma_{N1}}{\Delta_{N1}}
{(\lambda x.  C\{\Psi(\widehat{v})/\widehat{v}\} x) \Sigma,
   \Omega_N}$ \hfill (2) from lemma, thus $n = 0$\\
   
   \item Match succeeds:


      $s_2 = \matstatea{\Delta_N}{\lstack{C}; \lstack{P}}{\Gamma}
         {\Delta, \Xi_2, \dotsc, \Xi_n}{\cdot}{\Xi_1 \rightarrow \one \otimes
                                                 split(A)}{\cdot}$ \hfill (2) from
                                lemma \\
      $\feq{A}{\one \otimes split(\Omega)}$ \hfill (3) by well-formedness \\
      $\mz{\Gamma}{\Xi_1}{A}$ \hfill (4) from match equivalence
      theorem and split equivalence on (4) \\

      Now, we apply Corollary~\ref{thm:agg_match_to_derivation} on (2) \\

      $\derstatea{\Delta, \Xi_2, \dotsc, \Xi_n}{\Xi,
         \Xi_1}{\Gamma_{N1}}{\Delta_{N1}}{V_1 :: \cdot}{f';
            \lstack{P'}}{B\{\Psi(\widehat{x}), V / \widehat{x}, \sigma\}}$
      \dots \hfill (5) \\

      $\derstatea{\Delta, \Xi_2, \dotsc, \Xi_n}{\Xi, \Xi_1}{\Gamma_{N1},
         \Gamma_1}{\Delta_{N1}, \Delta_1}{V_1 :: \cdot}{f'; \lstack{P'}}{\cdot}$
      \hfill (6) applying Theorem~\ref{thm:aggregate_derivation} on (5)

      If $\dz{\Gamma}{\Pi}{\Delta, \Xi_2, \dotsc, \Xi_n}{\Xi, \Xi_1}{\Gamma_{N1},
         \Gamma_1}{\Delta_{N1}, \Delta_1}{\Omega_x}{\outsem}$ then \\
         \hspace{2cm}
         $\dz{\Gamma}{\Pi}{\Delta, \Xi_2, \dotsc,
            \Xi_n}{\Xi, \Xi_1}{\Gamma_{N1}}{\Delta_{N1}}{B\{\Psi(\widehat{x}), V
               / \widehat{x}, \sigma\}, \Omega_x}{\outsem}$
            \hfill (7) from
      Theorem~\ref{thm:aggregate_derivation} on (5) \\


      $\contstatea{\Delta, \Xi_2, \dotsc, \Xi_n}{f' ; \lstack{P'}}{\Gamma}{V_1
         :: \cdot}$ \hfill (8) next state of (6) \\

   
   When executing the next transition of state (6) we either get $n = 1$
   application of the aggregate or we apply
   Theorem~\ref{thm:multiple_aggregate_derivation} to get the remaining $n-1$
   applications.
\end{itemize}
\end{proof}


\subsection{Soundness of derivation}

We are finally ready to prove that the derivation of terms of the head of a rule
is sound in relation to HLD.

\begin{lemma}[Head derivation soundness]\label{thm:head_derivation_soundness}
If $s_1 = \derstatex{\Gamma}{\Delta_N}{\Xi}{\Gamma_{N1}}{\Delta_{N1}}{\Omega}$
then $\transs{s_1}{\finalstate{\Xi, \Xi^{*}}{\Gamma_{N1},
      \Gamma^{*}}{\Delta_{N1}, \Delta^{*}}}$ and
$\dz{\Gamma}{\Pi}{\Delta_N}{\Xi}{\Gamma_{N1}}{\Delta_{N1}}{A \lolli B,
   \Omega}{\Xi, \Xi^{*}; \Gamma_{N1}, \Gamma^{*}; \Delta_{N1}, \Delta^{*}}$
\end{lemma}

\begin{proof}\label{sec:derivation_theorem} Induction on $\Omega$. Most of the
sub-cases can be proven using the induction hypothesis or by straightforward
rule inference. The sub-case for aggregates is more
complicated and is proved beflow.

\paragraph{Aggregates} Apply Lemma~\ref{thm:aggregates} on the assumption to get
$n$ applications of the aggregate. Assume that $\Delta_N = \Delta, \Xi_1,
\dotsc, \Xi_n$, where $\Xi_i$ are the facts consumed and $\Gamma_i, \Delta_i$
the facts derived by the $i^{th}$ application.  The lemma proves the following:

\begin{itemize}[leftmargin=*]
   \item $n$ values $\Sigma = V_n :: \dotsb :: V_1 :: \cdot$
   \item $n$ applications are derived:\\
   $\derstatex{\Gamma}{\Delta}{\Xi, \Xi_1, \dotsc, \Xi_n}{\Gamma_{N1},
      \Gamma_1, \dotsc, \Gamma_n}{\Delta_{N1}, \Delta_1, \dotsc,
         \Delta_n}{(\lambda x.  C\{\Psi(\widehat{v})/\widehat{v}\} x) \Sigma,
            \Omega_N}$ (final state) \hfill (1)
   \item $n$ propositions $\mz{\Gamma}{\Xi_i}{A}$ \hfill (2)
   \item $n$ implications\\
      If $\dz{\Gamma}{\Pi}{\Delta, \Xi_{i+1}, \dotsc, \Xi_{n}}{\Xi, \Xi_1,
         \dotsc, \Xi_i}{\Gamma_{N1}, \Gamma_1, \dotsc, \Gamma_i}{\Delta_{N1},
            \Delta_1, \dotsc, \Delta_i}{\Omega_x}{\outsem}$ then\\
            \hspace*{2cm} $\dz{\Gamma}{\Pi}{\Delta, \Xi_{i+1}, \dotsc, \Xi_{n}}{\Xi, \Xi_1,
            \dotsc, \Xi_i}{\Gamma_{N1}, \Gamma_1, \dotsc, \Gamma_{i-1}}{\Delta_{N1},
            \Delta_1, \dotsc, \Delta_{i-1}}{B, \Omega_x}{\outsem}$ \hfill (3)
   \item $n$ contexts $\Psi_1, \dotsc, \Psi_n$ for variable bindings \hfill (4)
\end{itemize}

\newcommand{\outsemm}[3]{#1, \Xi^{*}; #2, \Gamma^{*}; #3, \Delta^{*}}
\newcommand{\aggoutsem}[0]{\outsemm{\Xi, \Xi_1, \dotsc, \Xi_n}{\Gamma_{N1}, \Gamma_1, \dotsc,
      \Gamma_n}{\Delta_{N1}, \Delta_1, \dotsc, \Delta_n}}

\noindent From (1) we apply the inductive hypothesis since $C$ is smaller than
the original aggregate:\\
$\dz{\Gamma}{\Pi}{\Delta}{\Xi, \Xi_1, \dotsc, \Xi_n}{\Gamma_{N1}, \Gamma_1,
\dotsc, \Gamma_n}{\Delta_{N1}, \Delta_1, \dotsc, \Delta_n}{
   (\lambda x.  C\{\Psi(\widehat{v})/\widehat{v}\} x) \Sigma,
   \Omega\\ \hspace*{2cm}}{\aggoutsem}$

\noindent Since we are building the proof tree backwards, starting from the final
derivation result, we first need to derive $\defstwo{agg}{\widehat{V}}{\Sigma}$ by applying rules
$\dzname \m{agg}_2$:\\
$\dz{\Gamma}{\Pi}{\Delta}{\Xi, \Xi_1, \dotsc, \Xi_n}{\Gamma_{N1}, \Gamma_1,
\dotsc, \Gamma_n}{\Delta_{N1}, \Delta_1, \dotsc, \Delta_n}
{\defstwo{agg}{\Phi(\widehat{v})}{\Sigma}, \Omega \\ \hspace*{2cm}}{\aggoutsem}$ \\

\noindent If $n = 0$ then this is all we need, otherwise
we need to rebuild the matching and derivation tree of the
$n^{th}$ aggregate.  Using the $n^{th}$ implication (3) on (5):

\noindent $\dz{\Gamma}{\Pi}{\Delta, \Xi_n}{\Xi, \Xi_1, \dotsc, \Xi_{n-1}}{\Gamma_{N1}, \Gamma_1,
\dotsc, \Gamma_{n-1}}{\Delta_{N1}, \Delta_1, \dotsc, \Delta_{n-1}}{B, \defstwo{agg}{\Phi(\widehat{v})}{\Sigma},
   \Omega \\ \hspace*{2cm}}{\aggoutsem}$ \\

\noindent Using $\dzname \lolli$ and the matching proposition (2) on (6), the $A \lolli B$
implication is reconstructed:

\noindent $\dz{\Gamma}{\Pi}{\Delta, \Xi_n}{\Xi, \Xi_1, \dotsc, \Xi_{n-1}}{\Gamma_{N1},
   \Gamma_1, \dotsc, \Gamma_{n-1}}{\Delta_{N1}, \Delta_1, \dotsc, \Delta_{n-1}}
{A \lolli B, \defstwo{agg}{\Phi(\widehat{v})}{\Sigma}, \Omega \\ \hspace*{2cm}}{\aggoutsem}$ \\

\noindent Next, we package the implication and the aggregate using $\dzname \otimes$:

\noindent $\dz{\Gamma}{\Pi}{\Delta, \Xi_n}{\Xi, \Xi_1, \dotsc, \Xi_{n-1}}{\Gamma_{N1},
\Gamma_1, \dotsc, \Gamma_{n-1}}{\Delta_{N1}, \Delta_1, \dotsc, \Delta_{n-1}}{
(A \lolli B) \otimes \defstwo{agg}{\Phi(\widehat{v})}{\Sigma}, \Omega \\
\hspace*{2cm}}{\aggoutsem}$ \\

\noindent Now, we apply $\dzname \forall$ to include the whole term and deconstruct $\Sigma$
into $\sigma :: V_{n-1} :: \dotsb :: V_1 :: \cdot$ since $V_n$ is the $\sigma$
variable. The $\widehat{x}$ terms included in the body of the aggregate are
replaced using information constructed in $\Psi_n$:

\noindent $\dz{\Gamma}{\Pi}{\Delta, \Xi_n}{\Xi, \Xi_1, \dotsc, \Xi_{n-1}}
{\Gamma_{N1}, \Gamma_1, \dotsc, \Gamma_{n-1}}{\Delta_{N1}, \Delta_1, \dotsc,
\Delta_{n-1}}{\\ \hspace*{2cm} \forall_{\widehat{x}, \sigma}. ((A' \lolli B') \otimes
      \defstwo{agg}{\Psi(\widehat{v})}{\sigma :: V_{n-1} ::
   \dotsb :: V_1 :: \cdot}), \Omega \\ \hspace*{3cm}}{\aggoutsem}$ \\


\noindent This last expression can be folded into
$\defstwo{agg}{\Psi(\widehat{v})}{V_{n-1} ::
   \dotsb :: V_1 :: \cdot}$:

\noindent $\dz{\Gamma}{\Pi}{\Delta, \Xi_n}{\Xi, \Xi_1, \dotsc, \Xi_{n-1}}
{\Gamma_{N1}, \Gamma_1, \dotsc, \Gamma_{n-1}}{\Delta_{N1}, \Delta_1, \dotsc,
\Delta_{n-1}}{\defstwo{agg}{\Psi(\widehat{v})}{V_{n-1} ::
   \dotsb :: V_1 :: \cdot}, \Omega \\ \hspace*{2cm}}{\aggoutsem}$ \\

\noindent The last 5 steps are then applied $n-1$ times to get:

\noindent $\dz{\Gamma}{\Pi}{\Delta, \Xi_1, \dotsc,
   \Xi_n}{\Xi}{\Gamma_{N1}}{\Delta_{N1}}{\defstwo{agg}{\Psi(\widehat{v})}{\cdot},
      \Omega \\ \hspace*{2cm}}{\aggoutsem}$ \\

\noindent This completes the sub-case for aggregates.


\end{proof}

\subsection{Wrapping-up}

In order to bring everything together, we need to use the Head derivation
soundness lemma (Lemma~\ref{thm:head_derivation_soundness}) and the Body match
result lemma (Lemma~\ref{thm:body_match}).  We first prove that if the LLD
machine is able to reach the final state, then there exists one rule where
matching was successful. Then, we prove that the application of such rule is
sound in relation to HLD.

\iffalse
\begin{theorem}[One rule]\label{thm:one_rule}
If $\transs{\dostate{\Delta}{\Phi}{\Gamma}{\Pi}}{\finalstate{\Xi}{\Gamma_{N1}}{\Delta_{N1}}}$
then $\exists_{R \in \Phi}$ such that
$\transs{\dostate{\Delta}{R}{\Gamma}{\Pi}}{\finalstate{\Xi}{\Gamma_{N1}}{\Delta_{N1}}}$
\end{theorem}
\begin{proof}
Induction on the size of $\Phi$.

Assume $\Phi = A \lolli B, \Phi'$ and $\rulestk = (\Delta; \Phi')$.
Apply two steps to the first state of the assumption to get
$\matstate{A \lolli B}{\rulestk}{\cdot}{\Gamma}{\Delta}{A}{\cdot
   \rightarrow \one}$.

Applying Lemma~\ref{thm:body_match} (body match result) leads to two
subcases:

\begin{itemize}[leftmargin=*]
   \item Match fails: \\
   $\contstate{A \lolli B}{\rulestk}{\cdot}{\Gamma}$ \hfill (1) \\

   $\dostate{\Delta}{\Phi'}{\Gamma}{\Pi}$ \hfill (2) state after (1) \\
   $\transs{\dostate{\Delta}{R'}{\Gamma}{\Pi}}{\finalstate{\Xi}{\Gamma_{N1}}{\Delta_{N1}}}$
   \hfill (3) i.h. on (2) where $R' \in \Phi'$ \\

   \item Match succeeds: \\

   $\matstate{A \lolli B}{\rulestk}{\lstack{C}}{\Gamma}{\Delta_1}{\cdot}{\Delta'
      \rightarrow split(A)}$ \\ \hfill (1) \\
   $\mo \Gamma; \Delta; \cdot; A; B; \cdot; (\cdot, \Delta) \rightarrow \outsem$ \hfill (1)\\
   $\ao \Gamma; \Delta; A \lolli B; (\cdot, \Delta) \rightarrow \outsem$ \hfill (2) rule $\ao \m{rule}$ on (1) \\
   $\doo \Gamma; \Delta; A \lolli B \rightarrow \outsem$ \hfill (3) rule $\doo \m{rule}$ on (2) \\
\end{itemize}
\end{proof}
\fi


\begin{theorem}[Soundness]\label{thm:soundness}
If $s_1 = \dostate{\Delta_N}{\Phi}{\Gamma}{\Pi}$
then either $\transs{s_1}{\finalstate{\Xi}{\Gamma_{N1}}{\Delta_{N1}}}$
and $\exists_{R \in \Phi}. \az{\Gamma}{\Delta_N}{\Pi}{R}{\Xi; \Gamma_{N1};
      \Delta_{N1}}$
or $\transs{s_1}{\failstate{\Gamma}{\Delta_N}}$.
\end{theorem}

\begin{proof}
If $\Phi = \cdot$ then the second conclusion applies immediatelly, otherwise use
induction on the size of $\Phi$.

Assume $\Phi = A \lolli B, \Phi'$ and $\rulestk = (\Delta_N; \Phi')$.

$\dostate{\Delta_N}{\Phi}{\Gamma}{\Pi}$ \hfill (1) first state of the assumption \\

Applying Lemma~\ref{thm:body_match} (body match result) to the state after two
transitions of (1) leads to two subcases:

\begin{itemize}[leftmargin=*]
   \item Match fails: \\
   $\contstate{A \lolli B}{\rulestk}{\cdot}{\Gamma}$ \hfill (2) \\
   $\dostate{\Delta_N}{\Phi'}{\Gamma}{\Pi}$ \hfill (3) state after (2) \\

   $\exists_{R' \in \Phi'}. \az{\Gamma}{\Delta_N}{\Pi}{R'}{\Xi; \Gamma_{N1};
   \Delta_{N1}}$. \hfill (4) i.h. on (3) where $R' \in \Phi'$ if $\Phi'$ is
   not empty \\

   $\failstate{\Gamma}{\Delta_N}$ \hfill (5) if $\Phi' = \cdot$ \\

   \item Match succeeds: \\

   $\matstate{A \lolli B}{\rulestk}{\lstack{C}}{\Gamma}{\Delta}{\cdot}{\Delta'
      \rightarrow split(A)}$ \hfill (1) $\Delta_N = \Delta', \Delta$ \\

   $\feq{split(A)}{A}$ \hfill (2) well-formedness of state (1) \\

   $\mz{\Gamma}{\Delta}{A}$ \hfill (3) match equivalence for (2) and (1) \\

   $\derstatex{\Gamma}{\Delta}{\Delta'}{\cdot}{\cdot}{B}$ \hfill (4) state
   after (1) \\

   $\transs{\derstatex{\Gamma}{\Delta}{\Delta'}{\cdot}{\cdot}{B}}{\finalstate{\Xi}{\Gamma_{N1}}{\Delta_{N1}}}$
   \hfill (5) applying Lemma~\ref{thm:head_derivation_soundness} to (4) \\

   $\dz{\Gamma}{\Pi}{\Delta}{\Xi}{\Gamma_{N1}}{\Delta_{N1}}{B}{\Delta', \Xi^{*}; \Gamma_{N1}; \Delta_{N1}}$ \hfill (5) from the
      same Lemma \\

   $\az{\Gamma}{\Delta, \Delta'}{\Pi}{A \lolli B}{\Delta', \Xi^{*}; \Gamma_{N1};
      \Delta_{N1}}$ \hfill (6) using $\azname \m{rule}$ on (5) and (3) \\

\end{itemize}

\end{proof}
