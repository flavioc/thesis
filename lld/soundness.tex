
Now that we have presented both the HLD and LLD semantics, we are in position to
start building our soundness theorem.  The soundness theorem proves that if a
rule was successfully derived in the LLD semantics then it can also be derived
in the HLD semantics. Since the HLD semantics are so close to linear logic, we
prove that our language has a determined, correct, proof search behavior when
executing programs. However, the completeness theorem cannot be proven since LLD
lacks the non-determinism inherent in HLD.

First and foremost, we need to prove some auxiliary theorems and definitions
that will be used during the soundness theorem.

\subsection{Term equivalence}

The first definition defines the equality between two multi-sets of terms.  Two
multi-sets $A$ and $B$ are equal, $\feq{A}{B}$, when they have the same
constituent atoms.

\[
\infer[\equiv p]
{\feq{p, A}{p, B}}
{\feq{A}{B}}
\tab
\infer[\equiv \bang p]
{\feq{\bang p, A}{\bang p, B}}
{\feq{A}{B}}
\tab
\infer[\equiv 1~L]
{\feq{1, A}{B}}
{\feq{A}{B}}
\tab
\infer[\equiv 1~R]
{\feq{A}{1, B}}
{\feq{A}{B}}
\]

\[
\infer[\equiv \cdot]
{\feq{\cdot}{\cdot}}
{}
\tab
\infer[\equiv \otimes~L]
{\feq{A \otimes B, C}{D}}
{\feq{A, B, C}{D}}
\tab
\infer[\equiv \otimes~R]
{\feq{A}{B \otimes C, D}}
{\feq{A}{B, C, D}}
\]

\begin{theorem}[Match equivalence]
If two multi-sets are equivalent, $\feq{A_1, \dotsc, A_n}{B_1, \dotsc, B_m}$,
   and we can match $A_1 \otimes \dotsb \otimes A_n$ in HLD such that $\mz
   \Gamma ; \Delta \rightarrow A_1 \otimes \dotsb \otimes A_n$ then $\mz \Gamma;
   \Delta \rightarrow B_1 \otimes \dotsb \otimes B_m$ is also true.
\end{theorem}
\begin{proof}
By straightforward induction on the first assumption.
\end{proof}

\subsection{Well-formed continuation frames}

We now define the concept of a well-formed frame given initial linear and
persistent contexts and a term $A$ that needs to be matched.

\begin{definition}[Well-formed frame]

Consider a triplet $A; \Gamma; \Delta_{N}$ where $A$ is a term, $\Gamma$ is a
multi-set of persistent resources and $\Delta_{N}$ a multi-set of linear
resources. A frame $f$ is well-formed iff:

\begin{enumerate}[leftmargin=*]
   \item Linear frame $f = (\Delta, p_1; \Delta'; \Xi_1, \dotsc, \Xi_m; p;
         \Omega_1, \dotsc, \Omega_n; \Lambda_1, \dotsc, \Lambda_m; \Upsilon_1,
         \dotsc, \Upsilon_k)$

   \begin{enumerate}
      \item $\feq{p, \Omega_1, \dotsc, \Omega_n, \Lambda_1, \dotsc, \Lambda_m,
         \Upsilon_1, \dotsc, \Upsilon_k}{A}$ (the remaining terms and already
               matched terms are equivalent to the initial body $A$);
      \item $\mz \Xi_1, \dotsc, \Xi_m \rightarrow \Lambda_1 \otimes \dotsb \otimes
      \Lambda_m$ and $\mz \Xi_i \rightarrow \Lambda_i$ for every $i$;

      \item $\Delta, \Delta', \Xi, p_1 = \Delta_{N}$ (available facts, candidate
            facts for $p$, consumed facts and the linear fact used for $p$,
            respectively, are the same as the initial $\Delta_{N}$);

      \item $\mz \Gamma; \cdot \rightarrow \Upsilon_1 \otimes \dotsb \otimes
      \Upsilon_k$ (past persistent facts can be matched with $\Gamma$).

   \end{enumerate}
   \item Persistent frame $f = [\Gamma'; \Delta; \Xi_1, \dotsc, \Xi_m; \bang p;
         \Omega_1, \dotsc, \Omega_n; \Lambda_1, \dotsc, \Lambda_m; \Upsilon_1,
         \dotsc, \Upsilon_k]$

      \begin{enumerate}
         \item $\feq{\bang p, \Omega_1, \dotsc, \Omega_n, \Lambda_1, \dotsc,
                     \Lambda_m, \Upsilon_1, \dotsc, \Upsilon_k}{A}$;
         \item $\mz \Xi_1, \dots, \Xi_m \rightarrow \Lambda_1 \otimes \dotsb \otimes
                     \Lambda_m$ and $\mz \Xi_i \rightarrow \Lambda_i$ for every $i$;
         \item $\Delta, \Xi = \Delta_{N}$;
         \item $\mz \Gamma; \cdot \rightarrow \bang p \otimes \Upsilon_1 \otimes
                     \dotsb \otimes \Upsilon_k$;
         \item $\Gamma' \subset \Gamma$ (remaining candidates are a subset of
                     $\Gamma$).
      \end{enumerate}
\end{enumerate}
\end{definition}


\begin{definition}[Well-formed stack]
A continuation stack $C$ is well-formed iff every frame is well-formed.
\end{definition}

Given the previous definitions, we can now define what it means for a judgment
to be well-formed.

\begin{definition}[Well-formed body match]

$\mo \Gamma; \Delta; \Xi; \Omega; H; \lstack{C}; \lstack{R} \rightarrow \outsem$
is well-formed in relation to a triplet $A; \Gamma; \Delta_{N}$ iff:

\begin{itemize}[leftmargin=*]
   \item $\Delta, \Xi = \Delta_{N}$
   \item $\lstack{C}$ is well-formed in relation to $A; \Gamma; \Delta_{N}$ and:
   \begin{itemize}[leftmargin=\secondm]
      \item If $\lstack{C} = \cdot$
   
      $\feq{\Omega}{A}$.
   
      \item If $\lstack{C} = (\Delta_a, p_1; \Delta_b; \Xi''; p; \Omega'; \Lambda_1,
            \dotsc, \Lambda_m; \Upsilon_1, \dotsc, \Upsilon_k), C'$
   
      \begin{itemize}[leftmargin=\thirdm]
         \item $\feq{\Omega'}{\Omega}$;
         \item $p_1 \in \Xi$ and $\mz \Gamma; p_1 \rightarrow p$;
         \item $\Xi = \Xi'', p_1$;
         \item $\Delta = \Delta_a, \Delta_b$.
      \end{itemize}
      \item If $\lstack{C} = [\Gamma'; \Delta''; \Xi''; \bang p; \Omega'; \Lambda_1,
      \dotsc, \Lambda_m; \Upsilon_1, \dotsc, \Upsilon_k], \lstack{C'}$
      \begin{itemize}[leftmargin=\thirdm]
         \item $\feq{\Omega}{\Omega'}$;
         \item $\Xi = \Xi''$;
         \item $\Delta = \Delta''$.
      \end{itemize}
   \end{itemize}
\end{itemize}

\end{definition}

\begin{definition}[Well-formed matches]
The following match propositions:
\begin{itemize}
   \item $\mc{\compsz{A}{B}} \Gamma; \Delta; \Xi_N; \Gamma_{N1}; \Delta_{N1}; \Xi;
         \Omega; \lstack{C}; \lstack{P}; \Omega_N; \Delta_N \rightarrow \outsem$
   \item $\ma{\aggsz{A}{B}{C}} \Gamma; \Delta; \Xi_N; \Gamma_{N1}; \Delta_{N1}; \Xi;
      \Omega; \lstack{C}; \lstack{P}; \Omega_N; \Delta_N; \Sigma \rightarrow \outsem$ is
\end{itemize}
are well-formed in relation to a triplet $A; \Gamma; \Delta_{N}$ iff:

\begin{itemize}[leftmargin=*]
   \item $\lstack{P}$ is composed solely of persistent frames.
   \item $\lstack{C}$ is composed of either linear or persistent frames, but the first
   frame is linear.
   \item $\Delta, \Xi = \Delta_{N}$
   \item $\lstack{C}$ and $\lstack{P}$ are well-formed in relation to $A; \Gamma; \Delta_{N}$ and
   follow the same rules presented before in "Well-formed body match" as a stack
   $\lstack{C}, \lstack{P}$.
\end{itemize}
\end{definition}

\begin{definition}[Well-formed continuations]
The following continuation propositions:
\begin{itemize}[leftmargin=*]
   \item $\contc{\compsz{A}{B}} \Gamma; \Delta_N; \Xi_N; \Gamma_{N1}; \Delta_{N1}; \lstack{C}; \lstack{P}; \Omega_N \rightarrow \outsem$
   \item $\conta{\aggsz{A}{B}{C}} \Gamma; \Delta_N; \Xi_N; \Gamma_{N1}; \Delta_{N1}; \lstack{C}; \lstack{P}; \Omega_N; \Sigma \rightarrow \outsem$
\end{itemize}
are well-formed in relation to a triplet $A; \Gamma; \Delta_{N}$ iff:

\begin{itemize}[leftmargin=*]
   \item $\lstack{P}$ is composed solely of persistent frames.
   \item $\lstack{C}$ is composed of either linear or persistent frames, but the first
   frame is linear.
   \item $\lstack{C}$ and $\lstack{P}$ are well-formed in relation to $A; \Gamma; \Delta_{N}$.
\end{itemize}
\end{definition}

\begin{definition}[Well-formed stack updates]
The following continuation stack updates propositions:
\begin{itemize}[leftmargin=*]
   \item $\fix{\compsz{A}{B}} \Gamma; \Delta; \Xi_N; \Gamma_{N1}; \Delta_{N1}; \Xi; \lstack{C}; \lstack{P}; \Omega_N;
      \Delta_N \rightarrow \outsem$
   \item $\fixa{\aggsz{A}{B}{C}} \Gamma; \Delta; \Xi_N; \Gamma_{N1}; \Delta_{N1}; \Xi; \lstack{C}; \lstack{P}; \Omega_N;
      \Delta_N; \Sigma \rightarrow \outsem$
\end{itemize}
are well-formed in relation to a triplet $A; \Gamma; \Delta_{N}$ iff:

\begin{itemize}[leftmargin=*]
   \item $\lstack{P}$ is composed solely of persistent frames.
   \item $\lstack{C}$ is composed of either linear or persistent frames, but the first
   frame is linear.
   \item $\lstack{C}$ and $\lstack{P}$ are well-formed in relation to $A; \Gamma; \Delta_{N}$.
   \item $\Delta, \Xi = \Delta_{N}$
\end{itemize}
\end{definition}

\begin{definition}[Well-formed derivations]
The following derivation propositions:
\begin{itemize}[leftmargin=*]
   \item $\dc{\compsz{A}{B}} \Gamma; \Delta_N; \Xi_N; \Gamma_1; \Delta_1;
      \Omega; \lstack{C}; \lstack{P}; \Omega_N \rightarrow \outsem$
   \item $\da{\aggsz{A}{B}{C}} \Gamma; \Delta_N; \Xi_N; \Gamma_1; \Delta_1;
      \Omega; \lstack{C}; \lstack{P}; \Omega_N \rightarrow \outsem$
\end{itemize}
are well-formed in relation to a triplet $A; \Gamma; \Delta_{N}$ iff:

\begin{itemize}[leftmargin=*]
   \item $\lstack{P}$ is composed solely of persistent frames.
   \item $\lstack{C}$ is composed of either linear or persistent frames, but the first
   frame is linear.
   \item $\lstack{C}$ and $\lstack{P}$ are well-formed in relation to $A; \Gamma; \Delta_{N}$.
\end{itemize}

\end{definition}

For the theorems that follow, we always assume that the propositions are
well-formed in relation to their main contexts and matching body, be it either a
rule body, a comprehension body or aggregate body. All the conclusions in the
proofs also must prove that the resulting propositions are well-formed.

\subsection{Soundness of matching}

The soundness theorem will be proven into two main steps. First, we prove that
performing a rule match at LLD is sound in relation to HLD and then we prove
that the derivation of head terms in LLD is also sound.

In order to prove the soundness of matching, we want to reconstitute a valid
$\mz$in HLD from a valid $\mo$in LLD. However, LLD may fail during matching,
therefore our body match lemma needs to handle unsuccessful matches. In order to
be able to use induction, we must assume a matching proposition $\mo$that
already contains some continuation frames in stack $\lstack{C}$ that is
well-formed in relation to the rule's body $A$ and initial database.

Our lemma also needs to apply to our continuation judgment $\contlld$, because
when inverting some of the matching assumptions, we get a continuation
proposition. Apart from an unsuccessful match, we deal with two situations
during a successful match: (1) we succeed without needing to backtrack to a
frame in stack $\lstack{C}$ or (2) we need to backtrack to a frame in
$\lstack{C}$. The complete lemma is stated and proven below.

\begin{lemma}[Body match soundness]\label{thm:body_match}
   
Given a rule $A \lolli H$, consider a triplet $T = A; \Gamma; \Delta_{N}$ and a
context $\Delta_{N} = \Delta_1, \Delta_2, \Xi$.

If $\mo \Gamma; \Delta_1, \Delta_2; \Xi; \Omega; H; \lstack{C}; \lstack{R}
\rightarrow \outsem$ is well-formed in relation to $T$ then either:

\begin{itemize}[leftmargin=*]
   \item Match fails:
   \begin{itemize}[leftmargin=\secondm]
      \item $\cont \cdot; H; \lstack{R}; \Gamma \rightarrow \outsem$
   \end{itemize}

   \item Match succeeds with no backtracking to frames of stack $\lstack{C}$:
   \begin{itemize}[leftmargin=\secondm]
      \item $\mz \Gamma; \Xi, \Delta_2 \rightarrow A$
      \item $\mo \Gamma; \Delta_1; \Xi, \Delta_2; \cdot; H; \lstack{C''},
   \lstack{C}; \lstack{R}
         \rightarrow \outsem$ (well-formed in relation to $T$)
      \item $\mo \Gamma; \Delta_1; \Xi, \Delta_2; \Omega; H; \lstack{C}; (\cdot, \Delta_N) \rightarrow \outsem$ (well-formed in relation to $T$)
   \end{itemize}

   \item Match succeeds with backtracking to a linear frame:
   \begin{itemize}[leftmargin=\secondm]
      \item $\mz \Gamma; \Xi_1, \dotsc, \Xi_m, p_2, \Xi_c \rightarrow A$
      \item $\exists_{f \in \lstack{C}}. f = (\Delta_a; \Delta_{b_1}, p_2, \Delta_{b_2}; p;
            \Omega_1, \dotsc, \Omega_k; \Xi_1 .. \Xi_m; \Lambda_1, \dotsc,
            \Lambda_m; \Upsilon_1, \dotsc, \Upsilon_n)$

      \item $\lstack{C} = \lstack{C'}, f, \lstack{C''}$

      \item $f$ turns into $f' = (\Delta_a, \Delta_{b_1}, p_2;
            \Delta_{b_2}; p; \Omega_1, \dotsc, \Omega_k; \Xi_1, \dotsc, \Xi_m;
            \Lambda_1, \dotsc, \Lambda_m; \Upsilon_1, \dotsc, \Upsilon_n)$

      \item $\mo \Gamma; \Delta_c; \Xi_1, \dotsc, \Xi_m, p_2, \Xi_c; \cdot; H;
            \lstack{C'''}, f', \lstack{C''}; \lstack{R} \rightarrow \outsem$ (well-formed in
                  relation to $T$)
      \item $\Delta_c = (\Delta_1, \Delta_2, \Xi) - (\Xi_1, \dotsc, \Xi_m, p_2, \Xi_c)$
   \end{itemize}

   \item Match succeeds with backtracking to a persistent frame:
   \begin{itemize}[leftmargin=\secondm]
      \item $\mz \Gamma; \Xi_1, \dotsc, \Xi_m, \Delta_{c_2} \rightarrow A$
      \item $\exists_{f \in \lstack{C}}. = f = [\Gamma_1, p_2, \Gamma_2; \Delta_{c_1}, \Delta_{c_2}; \Xi_c; \bang
         p; \Omega_1, \dotsc, \Omega_k; \Lambda_1, \dotsc, \Lambda_m;
         \Upsilon_1, \dotsc, \Upsilon_n]$
      \item $\lstack{C} = \lstack{C'}, f, \lstack{C''}$
      \item $f$ turns into $f' = [\Gamma_2; \Delta_{c_1}, \Delta_{c_2}; \Xi_1, \dotsc,
         \Xi_m; \bang p; \Omega_1, \dotsc, \Omega_k; \Lambda_1, \dotsc,
         \Lambda_m; \Upsilon_1, \dotsc, \Upsilon_n]$
      \item $\mo \Gamma; \Delta_{c_1}; \Xi_1, \dotsc, \Xi_m, \Delta_{c_2};
         \cdot; H; \lstack{C''}, f', \lstack{C''}; \lstack{R} \rightarrow \outsem$ (well-formed in
            relation to $T$)
      \item $\Delta_{c_1}, \Delta_{c_2} = (\Delta_1, \Delta_2,
            \Xi) - (\Xi_1, \dotsc, \Xi_m)$
   \end{itemize}
\end{itemize}

If $\cont \lstack{C}; H; \lstack{R}; \Gamma \rightarrow \outsem$ and $\lstack{C}$ is
well-formed in relation to $T$ then either:

\begin{itemize}[leftmargin=*]
   \item Match fails:
   \begin{itemize}[leftmargin=\secondm]
      \item $\cont \cdot; H; \lstack{R}; \Gamma \rightarrow \outsem$
   \end{itemize}

   \item Match succeeds with backtracking to a linear frame:
   \begin{itemize}[leftmargin=\secondm]
      \item $\mz \Gamma; \Xi_1, \dotsc, \Xi_m, p_2, \Xi_c \rightarrow A$
      \item $\exists_{f \in \lstack{C}}. f = (\Delta_a; \Delta_{b_1}, p_2, \Delta_{b_2}; p;
            \Omega_1, \dotsc, \Omega_k; \Xi_1 .. \Xi_m; \Lambda_1, \dotsc,
            \Lambda_m; \Upsilon_1, \dotsc, \Upsilon_n)$

      \item $\lstack{C} = \lstack{C'}, f, \lstack{C''}$

      \item $f$ turns into $f' = (\Delta_a, \Delta_{b_1}, p_2;
            \Delta_{b_2}; p; \Omega_1, \dotsc, \Omega_k; \Xi_1, \dotsc, \Xi_m;
            \Lambda_1, \dotsc, \Lambda_m; \Upsilon_1, \dotsc, \Upsilon_n)$

      \item $\mo \Gamma; \Delta_c; \Xi_1, \dotsc, \Xi_m, p_2, \Xi_c; \cdot; H;
            \lstack{C'''}, f', \lstack{C''}; \lstack{R} \rightarrow \outsem$ (well-formed in
                  relation to $T$)
      \item $\Delta_c = (\Delta_1, \Delta_2, \Xi) - (\Xi_1, \dotsc, \Xi_m, p_2, \Xi_c)$
   \end{itemize}

   \item Match succeeds with backtracking to a persistent frame:
   \begin{itemize}[leftmargin=\secondm]
      \item $\mz \Gamma; \Xi_1, \dotsc, \Xi_m, \Delta_{c_2} \rightarrow A$
      \item $\exists_{f \in \lstack{C}}. = f = [\Gamma_1, p_2, \Gamma_2; \Delta_{c_1}, \Delta_{c_2}; \Xi_c; \bang
         p; \Omega_1, \dotsc, \Omega_k; \Lambda_1, \dotsc, \Lambda_m;
         \Upsilon_1, \dotsc, \Upsilon_n]$
      \item $\lstack{C} = \lstack{C'}, f, \lstack{C''}$
      \item $f$ turns into $f' = [\Gamma_2; \Delta_{c_1}, \Delta_{c_2}; \Xi_1, \dotsc,
         \Xi_m; \bang p; \Omega_1, \dotsc, \Omega_k; \Lambda_1, \dotsc,
         \Lambda_m; \Upsilon_1, \dotsc, \Upsilon_n]$
      \item $\mo \Gamma; \Delta_{c_1}; \Xi_1, \dotsc, \Xi_m, \Delta_{c_2};
         \cdot; H; \lstack{C''}, f', \lstack{C''}; \lstack{R} \rightarrow \outsem$ (well-formed in
            relation to $T$)
      \item $\Delta_{c_1}, \Delta_{c_2} = (\Delta_1, \Delta_2,
            \Xi) - (\Xi_1, \dotsc, \Xi_m)$
   \end{itemize}
\end{itemize}
\end{lemma}

\begin{proof}
   Proof by mutual induction. In $\mo$on the size of $\Omega$ and on $\contlld$,
   first on the size of the second argument of the frame ($\Delta''$ and
   $\Gamma''$) and then on the size of the stack $C$. Sub-cases:
   
\begin{itemize}[leftmargin=*]
   \item $\mo p~\m{first}$, $\mo p~\m{on}~q$, $\mo p~\m{on}~\bang q$, $\mo \bang p~\m{first}$ $\mo \bang p~\m{on}~q$, $\mo \bang p~\m{on}~\bang q$, $\mo \otimes$
   
   When inverting the assumption, the well-formedness of the stack and match are
   proven straightforwardly using the well-formedness of the assumption and the
   match equivalence theorem. The induction hypothesis is then applied
   straightforwardly.
   
   \item $\mo \m{end}$
   
   Trivial.
   
   \item $\mo p~\m{fail}$, $\mo \bang p~\m{fail}$
   
   Invert the assumption and apply induction hypothesis on the $\cont$assumption.
   
   \item $\cont \m{next}~\m{rule}$
   
   Match fails.
   
   \item $\cont p~\m{next}, \cont \bang p~\m{next}$
   
   When inverting the assumption, we get a $\mo$proposition that is trivially
   proven to be well-formed in relation to $T$. Using induction hypothesis on
   this assumption, we have 3 sub-cases:
   
   \begin{itemize}[leftmargin=\secondm]
      \item Match fails: trivial.
      \item Match succeeds with no backtracking: the frame that we updated is the successful frame to backtrack to.
      \item Match succeeds with backtracking: $f \in C$ from the new assumption is the frame we need.
   \end{itemize}
   
   \item $\cont p~\m{no}~\m{more}$, $\cont \bang p~\m{no}~\m{more}$
   
   Invert the assumption to apply the induction hypothesis.
\end{itemize}
\end{proof}

For the induction hypothesis to be applicable in in Lemma~\ref{thm:body_match}
there must be a relation between the judgments $\mo$and $\contlld$.  We can
define a lexicographic ordering $A \prec B$, meaning that proposition $A$ has a
smaller proof than proposition $B$ (potentially $A$ is sub-proof of $B$), or
alternatively, $A$ is "executed later" than $B$.  The specific ordering is as
follows:

\begin{enumerate}[leftmargin=*]
   \item $\cont \lstack{C}; H; \lstack{R}; \Gamma \rightarrow \outsem \prec
   \cont \lstack{C'}, \lstack{C}; H; \lstack{R}; \Gamma \rightarrow \outsem$\\

   The continuation must use the top of the stack $\lstack{C'}$ before using
   $\lstack{C}$;

   \item $\cont \lstack{C'}, (\Delta, \Delta_1; \Delta_2; \Xi; p; \Omega;
         \Lambda; \Upsilon), \lstack{C}; H; \lstack{R}; \Gamma \rightarrow \outsem$\\

   \hspace*{1cm}$\prec \cont \lstack{C''}, (\Delta; \Delta_1, \Delta_2; \Xi; p;
         \Omega; \Lambda; \Upsilon), \lstack{C}; H; \lstack{R}; \Gamma \rightarrow \outsem$\\
   A continuation frame with more candidates has more steps to do than a frame with less candidates;

   \item $\cont \lstack{C'}, [\Gamma_1; \Delta; \Xi; \bang p; \Omega; \Lambda;
   \Upsilon], \lstack{C}; H; \lstack{R}; \Gamma \rightarrow \outsem$\\
   \hspace*{1cm} $\prec \cont \lstack{C''}, [\Gamma_2, \Gamma_3; \Delta; \Xi;
\bang p; \Omega; \Lambda; \Upsilon], \lstack{C}; H; \lstack{R}; \Gamma \rightarrow \outsem$\\
      Same as the previous one;

   \item $\cont \lstack{C}; H; \lstack{R}; \Gamma \rightarrow \outsem \prec \mo
   \Gamma; \Delta; \Xi; \Omega; H; \lstack{C'}, \lstack{C}; \lstack{R}
   \rightarrow \outsem$\\ Same as (1);

   \item $\mo \Gamma; \Delta; \Xi; \Omega; H; \lstack{C}; \lstack{R} \rightarrow \outsem \prec
   \cont \lstack{C'}, \lstack{C}; H; \lstack{R}; \Gamma \rightarrow \outsem$\\ Same as the previous one;

   \item $\mo \Gamma; \Delta''; \Xi''; \Omega'; H; \lstack{C'}, \lstack{C};
         \lstack{R} \rightarrow \outsem \prec \mo \Gamma; \Delta; \Xi; \Omega;
         H; \lstack{C}; \lstack{R} \rightarrow \outsem$ as long
   as $\Omega' \prec \Omega$\\ Adding continuation frames to the stack makes the
   proof smaller as long as $\Omega$ is also smaller; 

   \item $\mo \Gamma; \Delta; \Xi; \Omega; H; \lstack{C'}, (\Delta, \Delta_1;
         \Delta_2; \Xi; p; \Omega; \Lambda; \Upsilon), \lstack{C}; \lstack{R}
   \rightarrow \outsem$\\ \hspace*{1cm} $\prec \mo \Gamma; \Delta''; \Xi'';
   \Omega'; \lstack{C''}, (\Delta; \Delta_1, \Delta_2; \Xi; p; \Omega; \Lambda;
         \Upsilon), \lstack{C}; \lstack{R} \rightarrow \outsem$\\ Same as (2);

   \item $\mo \Gamma; \Delta; \Xi; \Omega; H; \lstack{C'}, [\Gamma_1; \Delta;
   \Xi; \bang p; \Omega; \Lambda; \Upsilon], \lstack{C}; \lstack{R} \rightarrow \outsem$\\
   \hspace*{1cm} $\prec \mo \Gamma; \Delta''; \Xi''; \Omega'; \lstack{C''},
   [\Gamma_2, \Gamma_3; \Delta; \Xi; \bang p; \Omega; \Lambda; \Upsilon],
   \lstack{C}; \lstack{R} \rightarrow \outsem$\\
   Same as (3).

\end{enumerate}

\subsection{Soundness of derivation}

Proving that the derivation of the head of the rule is sound is trivial except
for comprehensions and aggregates. LLD deterministically computes the number of
available comprehensions to apply while HLD "guesses" the number and then
performs the derivation. In the next two sections, we show how to prove the
soundness of comprehensions and aggregates. The strategy for proving for proving
both is identical due to their inherient similarities.

\subsection{Comprehension soundness}


Proving that deriving a comprehension in LLD is sound in relation to HLD is
built from 4 results: (1) proving that matching the body of a comprehension is
sound in relation to HLD; (2) proving that updating the continuation stacks
makes them suitable for use in the next comprehension applications; (3) proving
that deriving the head of the comprehension is sound in relation to HLD; (4)
proving that we can apply as many comprehensions as the database allows.

\begin{lemma}[Comprehension body match]\label{thm:comprehension_body_match}
Given a comprehension $AB = \compsz{A}{B}$, consider a triplet $T = A; \Gamma;
\Delta_{N}$ and a context $\Delta_{N} = \Delta_1, \Delta_2, \Xi$.

If $\mc{AB} \Gamma; \Delta_1, \Delta_2; \Xi_N; \Gamma_{N1}; \Delta_{N1}; \Xi;
\Omega; \lstack{C}; \lstack{P}; \Omega_N; \Delta_N \rightarrow \outsem$ is
well-formed in relation to $T$ then either:

\begin{itemize}[leftmargin=*]
   \item Match fails:
   \begin{itemize}[leftmargin=\secondm]
      \item $\done \Gamma; \Delta_N; \Xi_N; \Gamma_{N1}; \Delta_{N1}; \Omega_N \rightarrow \outsem$
   \end{itemize}
   
   \item Match succeeds with no backtracking to frames of stack $\lstack{C}$ or
   $\lstack{P}$ ($\lstack{C} \neq \cdot$):

   \begin{itemize}[leftmargin=\secondm]
      \item $\mz \Gamma; \Delta_2 \rightarrow A$
      \item $\mc{AB} \Gamma; \Delta_1; \Xi_N; \Gamma_{N1}; \Delta_{N1}; \Xi,
         \Delta_2; \cdot; \lstack{C'}, \lstack{C}; \lstack{P}; \Omega_N; \Delta_N
         \rightarrow \outsem$ (well-formed in relation to $T$)
   \end{itemize}

   \item Match succeeds with no backtracking to frames of stack $\lstack{P}$ ($\lstack{C} =
         \cdot$):
   \begin{itemize}[leftmargin=\secondm]
      \item $\mz \Gamma; \Delta_2 \rightarrow A$
      \item $\mc{AB} \Gamma; \Delta_1; \Xi_N; \Gamma_{N1}; \Delta_{N1}; \Xi,
         \Delta_2; \cdot; \lstack{C'}; \lstack{P'}, \lstack{P}; \Omega_N; \Delta_N
         \rightarrow \outsem$ (well-formed in relation to $T$)
   \end{itemize}

      \item Match succeeds with backtracking to a linear continuation frame in
   stack $\lstack{C}$ ($\lstack{C} \neq \cdot$):

   \begin{itemize}[leftmargin=\secondm]
      \item $\mz \Gamma; \Xi_1, \dotsc, \Xi_m, p_2, \Xi_c$
      \item $\exists_{f \in C}. f = (\Delta_a; \Delta_{b_1}, p_2,
            \Delta_{b_2}; p; \Xi_1, \dotsc, \Xi_m; \Omega_1, \dotsc,
            \Omega_k; \Lambda_1, \dotsc, \Lambda_m; \Upsilon_1, \dotsc,
            \Upsilon_n)$
      \item $\lstack{C} = \lstack{C'}, f, \lstack{C''}$
      \item $f$ turns into $f' = (\Delta_a, \Delta_{b_1}, p_2;
            \Delta_{b_2}; p; \Xi_1, \dotsc, \Xi_m;
            \Omega_1, \dotsc, \Omega_k; \Lambda_1, \dotsc, \Lambda_m;
            \Upsilon_1, \dotsc, \Upsilon_n)$
      \item $\mc{AB} \Gamma; \Delta_c; \Xi_N; \Gamma_{N1}; \Delta_{N1}; \Xi_1,
         \dotsc, \Xi_m, p_2, \Xi_c; \cdot; \lstack{C'''}, f', \lstack{C''}; \lstack{P};
         \Omega_N; \Delta_N \rightarrow \outsem$ (well-formed in relation to $T$)
      \item $\Delta_c = (\Delta_1, \Delta_2, \Xi) - (\Xi_1, \dotsc, \Xi_m, p_2, \Xi_c)$
   \end{itemize}

   \item Match succeeds with backtracking to a persistent continuation frame
   in stack $\lstack{C}$ ($\lstack{C} \neq \cdot$):
   \begin{itemize}[leftmargin=\secondm]
      \item $\mz \Gamma; \Delta_{c_2}, \Xi_1, \dotsc, \Xi_m \rightarrow A$
      \item $\exists_{f \in C}. f = [\Gamma_1, p_2, \Gamma_2; \Delta_{c_1},
         \Delta_{c_2}; \Xi_1, \dotsc, \Xi_m; \bang p; \Omega_1, \dotsc, \Omega_k;
         \Lambda_1, \dotsc, \Lambda_m; \Upsilon_1, \dotsc, \Upsilon_n]$
      \item $\lstack{C} = \lstack{C'}, f, \lstack{C''}$
      \item $f$ turns into $f' = [\Gamma_2; \Delta_{c_1}, \Delta_{c_2};
         \Xi_1, \dotsc, \Xi_m; \bang p; \Omega_1, \dotsc, \Omega_k; \Lambda_1,
         \dotsc, \Lambda_m; \Upsilon_1, \dotsc, \Upsilon_n]$
      \item $\mc{AB} \Gamma; \Delta_{c_1}; \Xi_N; \Gamma_{N1}; \Delta_{N1};
         \Delta_{c_2}, \Xi_1, \dotsc, \Xi_m; \cdot; \lstack{C'''}, f',
         \lstack{C''}; \lstack{P};
         \Omega_N; \Delta_N \rightarrow \outsem$ (well-formed in relation to $T$)
      \item $\Delta_{c_1}, \Delta_{c_2} = (\Delta_1, \Delta_2, \Xi) - (\Xi_1, \dotsc, \Xi_m)$
   \end{itemize}

   \item Match succeeds with backtracking to a persistent continuation frame
   in stack $\lstack{P}$ ($\lstack{C} = \cdot$):
   \begin{itemize}[leftmargin=\secondm]
      \item $\mz \Gamma; \Delta_{c_2}, \Xi_1, \dotsc, \Xi_m \rightarrow A$
      \item $\exists_{f \in P}. f = [\Gamma_1, p_2, \Gamma_2; \Delta_{c_1}, \Delta_{c_2};
   \Xi_1, \dotsc, \Xi_m; \bang p; \Omega_1, \dotsc, \Omega_k; \Lambda_1,
   \dotsc, \Lambda_m; \Upsilon_1, \dotsc, \Upsilon_n]$
      \item $\lstack{P} = \lstack{P'}, f, \lstack{P''}$
      \item $f$ turns into $f' = [\Gamma_2; \Delta_{c_1},
         \Delta_{c_2}; \Xi_1, \dotsc, \Xi_m; \bang p; \Omega_1, \dotsc,
         \Omega_k; \Lambda_1, \dotsc, \Lambda_m; \Upsilon_1, \dotsc, \Upsilon_n]$
      \item $\mc{AB} \Gamma; \Delta_{c_1}; \Xi_N; \Gamma_{N1}; \Delta_{N1};
      \Delta_{c_2}, \Xi_1, \dotsc, \Xi_m; \cdot; \lstack{C'}; \lstack{P'''}, f',
      \lstack{P''};
      \Omega_N; \Delta_N \rightarrow \outsem$
      (well-formed in relation to $T$)
      \item $\Delta_{c_1}, \Delta_{c_2} = (\Delta_1, \Delta_2, \Xi) - (\Xi_1, \dotsc,
            \Xi_m)$
   \end{itemize}


\end{itemize}
   
If $\contc{AB} \Gamma; \Delta_{N}; \Xi_{N}; \Gamma_{N1}; \Delta_{N1};
\lstack{C}; \lstack{P}; \Omega_N \rightarrow \outsem$ and $\lstack{C}$ and
$\lstack{P}$ are well-formed in relation to $T$ then either:

\begin{itemize}[leftmargin=*]
   \item Match fails:
   \begin{itemize}[leftmargin=\secondm]
      \item $\done \Gamma; \Delta_N; \Xi_N; \Gamma_{N1}; \Delta_{N1}; \Omega_N \rightarrow \outsem$
   \end{itemize}

      \item Match succeeds with backtracking to a linear continuation frame in
   stack $\lstack{C}$ ($\lstack{C} \neq \cdot$):

   \begin{itemize}[leftmargin=\secondm]
      \item $\mz \Gamma; \Xi_1, \dotsc, \Xi_m, p_2, \Xi_c$
      \item $\exists_{f \in C}. f = (\Delta_a; \Delta_{b_1}, p_2,
            \Delta_{b_2}; p; \Xi_1, \dotsc, \Xi_m; \Omega_1, \dotsc,
            \Omega_k; \Lambda_1, \dotsc, \Lambda_m; \Upsilon_1, \dotsc,
            \Upsilon_n)$
      \item $\lstack{C} = \lstack{C'}, f, \lstack{C''}$
      \item $f$ turns into $f' = (\Delta_a, \Delta_{b_1}, p_2;
            \Delta_{b_2}; p; \Xi_1, \dotsc, \Xi_m;
            \Omega_1, \dotsc, \Omega_k; \Lambda_1, \dotsc, \Lambda_m;
            \Upsilon_1, \dotsc, \Upsilon_n)$
      \item $\mc{AB} \Gamma; \Delta_c; \Xi_N; \Gamma_{N1}; \Delta_{N1}; \Xi_1,
         \dotsc, \Xi_m, p_2, \Xi_c; \cdot; \lstack{C'''}, f', \lstack{C''}; \lstack{P};
         \Omega_N; \Delta_N \rightarrow \outsem$ (well-formed in relation to $T$)
      \item $\Delta_c = (\Delta_1, \Delta_2, \Xi) - (\Xi_1, \dotsc, \Xi_m, p_2, \Xi_c)$
   \end{itemize}

   \item Match succeeds with backtracking to a persistent continuation frame
   in stack $\lstack{C}$ ($\lstack{C} \neq \cdot$):
   \begin{itemize}[leftmargin=\secondm]
      \item $\mz \Gamma; \Delta_{c_2}, \Xi_1, \dotsc, \Xi_m \rightarrow A$
      \item $\exists_{f \in C}. f = [\Gamma_1, p_2, \Gamma_2; \Delta_{c_1},
         \Delta_{c_2}; \Xi_1, \dotsc, \Xi_m; \bang p; \Omega_1, \dotsc, \Omega_k;
         \Lambda_1, \dotsc, \Lambda_m; \Upsilon_1, \dotsc, \Upsilon_n]$
      \item $\lstack{C} = \lstack{C'}, f, \lstack{C''}$
      \item $f$ turns into $f' = [\Gamma_2; \Delta_{c_1}, \Delta_{c_2};
         \Xi_1, \dotsc, \Xi_m; \bang p; \Omega_1, \dotsc, \Omega_k; \Lambda_1,
         \dotsc, \Lambda_m; \Upsilon_1, \dotsc, \Upsilon_n]$
      \item $\mc{AB} \Gamma; \Delta_{c_1}; \Xi_N; \Gamma_{N1}; \Delta_{N1};
         \Delta_{c_2}, \Xi_1, \dotsc, \Xi_m; \cdot; \lstack{C'''}, f',
         \lstack{C''}; \lstack{P};
         \Omega_N; \Delta_N \rightarrow \outsem$ (well-formed in relation to $T$)
      \item $\Delta_{c_1}, \Delta_{c_2} = (\Delta_1, \Delta_2, \Xi) - (\Xi_1, \dotsc, \Xi_m)$
   \end{itemize}

   \item Match succeeds with backtracking to a persistent continuation frame
   in stack $\lstack{P}$ ($\lstack{C} = \cdot$):
   \begin{itemize}[leftmargin=\secondm]
      \item $\mz \Gamma; \Delta_{c_2}, \Xi_1, \dotsc, \Xi_m \rightarrow A$
      \item $\exists_{f \in P}. f = [\Gamma_1, p_2, \Gamma_2; \Delta_{c_1}, \Delta_{c_2};
   \Xi_1, \dotsc, \Xi_m; \bang p; \Omega_1, \dotsc, \Omega_k; \Lambda_1,
   \dotsc, \Lambda_m; \Upsilon_1, \dotsc, \Upsilon_n]$
      \item $\lstack{P} = \lstack{P'}, f, \lstack{P''}$
      \item $f$ turns into $f' = [\Gamma_2; \Delta_{c_1},
         \Delta_{c_2}; \Xi_1, \dotsc, \Xi_m; \bang p; \Omega_1, \dotsc,
         \Omega_k; \Lambda_1, \dotsc, \Lambda_m; \Upsilon_1, \dotsc, \Upsilon_n]$
      \item $\mc{AB} \Gamma; \Delta_{c_1}; \Xi_N; \Gamma_{N1}; \Delta_{N1};
      \Delta_{c_2}, \Xi_1, \dotsc, \Xi_m; \cdot; \lstack{C'}; \lstack{P'''}, f',
      \lstack{P''};
      \Omega_N; \Delta_N \rightarrow \outsem$
      (well-formed in relation to $T$)
      \item $\Delta_{c_1}, \Delta_{c_2} = (\Delta_1, \Delta_2, \Xi) - (\Xi_1, \dotsc,
            \Xi_m)$
   \end{itemize}

\end{itemize}
\end{lemma}

Proving that matching the body of a comprehension is sound in relation to HLD
follows the structure of the Lemma~\ref{thm:body_match}. The lemma uses mutual
induction on the recursive judgments $\mc{AB}$and $\contc{AB}$and considers the three
possible results of matching: failure, success with no backtracking and success
with backtracking.

In order to apply a comprehension again, we need to reuse the continuation
stacks. However, in order to use $\lstack{C}$ and $\lstack{P}$ safely, we need
to prove that $\lstack{C}$ will have at most one updated linear continuation
frame and $\lstack{P}$ will have all its frames updated to account the
consumption of the facts from the previous application of the comprehension.

We first prove some auxiliary theorems.

\begin{theorem}[Full stack update]\label{thm:stack_update}
If $\strans \Xi; \lstack{P}; \lstack{P'}$ then $\lstack{P'}$ will be the
transformation of stack $\lstack{P}$ where
every frame $f \in \lstack{P}$, where $f = [\Gamma'; \Delta_N; \cdot; \bang p; \Omega; \cdot;
      \Upsilon])$, will turn into $f' = [\Gamma'; \Delta_N - \Xi; \cdot;
      \bang p; \Omega; \cdot; \Upsilon]$, where $f' \in \lstack{P'}$.
\end{theorem}
\begin{proof}
Straightforward induction on the size of $\lstack{P}$.
\end{proof}

\begin{theorem}[From update to derivation]\label{thm:from_update_to_derivation}
If $\fix{AB} \Gamma; \Delta; \Xi_N; \Gamma_{N1}; \Delta_{N1}; \Xi; \lstack{C};
\lstack{P}; \Omega_N; \Delta_N \rightarrow \outsem$ then\\
\texttab$\dc{AB} \Gamma; \Xi_N, \Xi;
\Gamma_{N1}; \Delta_{N1}; B; \lstack{C'} ; \lstack{P'}; \Omega_N; (\Delta_N - \Xi) \rightarrow
\outsem$, where:

\begin{itemize}[leftmargin=*]
   \item If $\lstack{C} = \cdot$ then $\lstack{C'} = \cdot$

   \item If $\lstack{C} = \lstack{C_1}, (\Delta_a; \Delta_b; \cdot; p; \Omega; \cdot; \Upsilon)$
   then $\lstack{C'} = (\Delta_a - \Xi; \Delta_b - \Xi; \cdot; p; \Omega; \cdot;
         \Upsilon)$

   \item $\lstack{P'}$ is the transformation of stack $\lstack{P}$, where for every frame $f \in
   \lstack{P}$ of the form $[\Gamma'; \Delta_N; \cdot; \bang p; \Omega; \cdot; \Upsilon]$
   will turn into $f' = [\Gamma';\Delta_N-\Xi;\cdot;\bang p;\Omega;\cdot;\Upsilon]$

\end{itemize}
\end{theorem}
\begin{proof}
Use induction on the size of the stack $\lstack{C}$ to get the result of
$\lstack{C'}$ and then
apply Theorem~\ref{thm:stack_update} to get $\lstack{P'}$.
\end{proof}


Now we prove that a match of a comprehension's body implies the start of a
derivation of the comprehension's head with correct continuation stacks. Note
that $\Omega = \cdot$ in $\matchlldc$, so there is nothing left to match.

\begin{corollary}[Match to derivation]\label{thm:match_to_derivation}
If $\mc{AB} \Gamma; \Delta; \Xi_N; \Gamma_{N1}; \Delta_{N1}; \Xi; \cdot; B;
\lstack{C}; \lstack{P};
\Omega_N; \Delta_N \rightarrow \outsem$ then\\
\texttab$\dc{AB} \Gamma; \Xi_N, \Xi; \Gamma_{N1}; \Delta_{N1}; B; \lstack{C'};
\lstack{P'}; \Omega_N; (\Delta_N - \Xi) \rightarrow \outsem$ where:
   
\begin{itemize}[leftmargin=*]
   \item If $\lstack{C} = \cdot$ then $\lstack{C'} = \cdot$
   \item If $\lstack{C} = \lstack{C_1}, (\Delta_a; \Delta_b; \cdot; p; \Omega;
         \cdot; \Upsilon)$ then $\lstack{C'} = (\Delta_a - \Xi; \Delta_b - \Xi; \cdot; p;
            \Omega; \cdot; \Upsilon)$ then \linebreak $\lstack{C'} = (\Delta_a - \Xi; \Delta_b - \Xi; \cdot; p; \Omega; \cdot; \Upsilon)$
   \item $\lstack{P'}$ is the transformation of stack $\lstack{P}$, where for every frame $f \in
   \lstack{P}$ of the form $[\Gamma'; \Delta_N; \cdot; \bang p; \Omega; \cdot; \Upsilon]$
   will turn into $f' = [\Gamma';\Delta_N-\Xi;\cdot;\bang p;\Omega;\cdot;\Upsilon]$
\end{itemize}
\end{corollary}

\begin{proof}
Invert the assumption and then apply Theorem~\ref{thm:from_update_to_derivation}.
\end{proof}


\paragraph{Comprehension Derivation}

We also need to prove that deriving the head of a comprehension is sound in
relation to HLD.  With the results of the next theorem we can reuse the
continuation stacks to start the comprehension process all over again, but now
with a non-empty continuation stack.

\begin{theorem}[Comprehension derivation soundness]\label{thm:comprehension_derivation}
If $\dc{AB} \Gamma; \Delta; \Xi_N; \Gamma_{N1}; \Delta_{N1}; \Omega_1, \dotsc,
   \Omega_n; \lstack{C}; \lstack{P}; \Omega_N; \Delta_N \rightarrow \outsem$ then:

\begin{itemize}[leftmargin=*]
   \item $\dc{AB} \Gamma; \Delta; \Xi_N; \Gamma_{N1}, \Gamma_1, \dotsc, \Gamma_n; \Delta_{N1},
   \Delta_1, \dotsc, \Delta_n; \cdot; \lstack{C}; \lstack{P}; \Omega_N; \Delta_N \rightarrow
   \outsem$;

   \item $\forall_{\Omega_x}.($ if $\dz \Gamma; \Delta; \Xi_N;
   \Gamma_{N1}, \Gamma_1, \dotsc, \Gamma_n; \Delta_{N1}, \Delta_1, \dotsc,
   \Delta_n; \Omega_x \rightarrow \outsem$ then

   $\dz \Gamma; \Delta; \Xi_N; \Gamma_{N1}; \Delta_{N1}; \Omega_1, \dotsc,
   \Omega_n, \Omega_x \rightarrow \outsem)$.

\end{itemize}
\end{theorem}

\begin{proof}
Straightforward induction on $\Omega_1, \dotsc, \Omega_n$.
\end{proof}

The second result of this theorem is the soundness result we need because it will allow us to reconstruct the derivation tree in HLD.


\paragraph{Multiple Comprehension Derivation} We are interested in proving that
if we start with a given comprehension match $\matchlldc$ then we can apply the
comprehension several times.

\begin{theorem}[Multiple comprehension derivation]\label{thm:multiple_comprehension_derivation}
Consider a triplet $T = A; \Gamma; \Delta_{N}$ and a comprehension $AB =
\compsz{A}{B}$. Assume that there exists $n \geq 0$ applications of $AB$
where the $i_{th}$ application is related to the following contexts:
\begin{description}
   \item[$\Delta_i$]: context of derived linear facts;
   \item[$\Gamma_i$]: context of derived persistent facts;
   \item[$\Xi_i$]: context of consumed linear facts.
\end{description}

Since each application consumes $\Xi_i$ then the initial context $\Delta_N =
\Delta, \Xi_1, \dotsc, \Xi_n$. We now define the two main implications of the
theorem.

\begin{itemize}[leftmargin=*]
   \item Assume that $\Delta_N = \Delta_a, \Delta_b$, $\Delta_b =
   \Delta'_b, p_1$ and there is a frame $f = (\Delta_a, p_1; \Delta'_b; \cdot;
         p; \Omega; \cdot; \Upsilon)$.

   If $\mc{AB} \Gamma; \Delta_a, \Delta'_b; \Xi_N; \Gamma_{N1}; \Delta_{N1};
      p_1; \Omega; f; \lstack{P}; \Omega_N; \Delta, \Xi_1, \dotsc, \Xi_n \rightarrow
      \outsem$ (well-formed in relation to $T$) then:

   \begin{itemize}[leftmargin=\secondm]
      \item $n$ comprehensions are derived:\\
      $\done \Gamma; \Delta_N; \Xi_N, \Xi_1, \dotsc, \Xi_n; \Gamma_{N1},
      \Gamma_1, \dotsc, \Gamma_n; \Delta_{N1}, \Delta_1, \dotsc, \Delta_n; \Omega_N \rightarrow \outsem$
      \item $n$ $\mz$propositions for the $n$ comprehension matches:
      \begin{itemize}[leftmargin=\thirdm]
         \item $\mz \Gamma; \Xi_1 \rightarrow A$
         \item $\dots$
         \item $\mz \Gamma; \Xi_n \rightarrow A$
      \end{itemize}
      \item $n$ derivation implications for HLD: \\
      $\forall_{\Omega_x}.($ if $\dz \Gamma; \Delta, \Xi_{i+1}, \dotsc, \Xi_{n}; \Xi_N, \Xi_1,
            \dotsc, \Xi_i; \Gamma_{N1}, \Gamma_1, \dotsc, \Gamma_i; \Delta_{N1},
            \Delta_1, \dotsc, \Delta_i; \Omega_x \rightarrow \outsem$ then $\dz \Gamma; \Delta, \Xi_{i+1}, \dotsc, \Xi_{n}; \Xi_N, \Xi_1,
            \dotsc,
            \Xi_i; \Gamma_{N1}, \Gamma_1, \dotsc, \Gamma_{i-1}; \Delta_{N1},
            \Delta_1, \dotsc, \Delta_{i-1}; B, \Omega_x \rightarrow \outsem)$
   \end{itemize}

   \item If $\mc{AB} \Gamma; \Delta_N; \Xi_N; \Gamma_{N1}; \Delta_{N1}; \cdot; \Omega;
      \cdot; \lstack{P}; \Omega_N; \Delta, \Xi_1, \dotsc, \Xi_n \rightarrow \outsem$ (well-formed in relation to $T$) then:

   \begin{itemize}[leftmargin=\secondm]
      \item $n$ comprehensions are derived:\\
      $\done \Gamma; \Delta_N; \Xi_N, \Xi_1, \dotsc, \Xi_n; \Gamma_{N1},
      \Gamma_1, \dotsc, \Gamma_n; \Delta_{N1}, \Delta_1, \dotsc, \Delta_n; \Omega_N \rightarrow \outsem$

      \item $n$ $\mz$propositions for the $n$ comprehension matches:
      \begin{itemize}[leftmargin=\thirdm]
         \item $\mz \Gamma; \Xi_1 \rightarrow A$
         \item \dots
         \item $\mz \Gamma; \Xi_n \rightarrow A$
      \end{itemize}

      \item $n$ derivation implications for HLD: \\
      $\forall_{\Omega_x}.($ if $\dz \Gamma; \Delta, \Xi_{i+1}, \dotsc, \Xi_{n}; \Xi_N, \Xi_1,
            \dotsc, \Xi_i; \Gamma_{N1}, \Gamma_1, \dotsc, \Gamma_i; \Delta_{N1},
            \Delta_1, \dotsc, \Delta_i; \Omega_x \rightarrow \outsem$ then $\dz \Gamma; \Delta, \Xi_{i+1}, \dotsc, \Xi_{n}; \Xi_N, \Xi_1,
            \dotsc,
            \Xi_i; \Gamma_{N1}, \Gamma_1, \dotsc, \Gamma_{i-1}; \Delta_{N1},
            \Delta_1, \dotsc, \Delta_{i-1}; B, \Omega_x \rightarrow \outsem)$
   \end{itemize}

\end{itemize}
   
\end{theorem}
\begin{proof}

By mutual induction, first on either the size of $\Delta'_b$ (second argument of
the linear continuation frame) or $\Gamma'$ (second argument of the
persistent frame in $\lstack{P}$) and then on the size of $\lstack{C},
\lstack{P}$.  We only show how to prove the first implication since the
second implication is proven in a similar way.

$\mc{AB} \Gamma; \Delta_a, \Delta'_b; \Xi_N; \Gamma_{N1}; \Delta_{N1}; p_1;
\Omega; f; \lstack{P}; \Omega_N; \Delta, \Xi_1, \dotsc, \Xi_n \rightarrow \outsem$ \hfill (1) assumption\\

By applying Lemma~\ref{thm:comprehension_body_match} to (1), we get:

\begin{itemize}[leftmargin=*]
   \item Failure:
   
   $\done \Gamma; \Delta_N; \Xi_N; \Gamma_{N1}; \Delta_{N1}; \Omega_N
   \rightarrow \outsem$ \hfill (2) from lemma, thus $n = 0$\\
   
   \item Success with no backtracking to frames of stack $\lstack{C}$ or
   $\lstack{P}$:
   
      $\mz \Gamma; \Xi_1 \rightarrow A$ \hfill (2) from lemma \\

      $\Xi_1 = \Xi'_1, p_1$ \hfill (3) from the well-formedness of (1) \\
      $f = (\Delta_a, p_1; \Delta'_b; \cdot; p; \Omega; \cdot; \Upsilon)$ \\

      $\mc{AB} \Gamma; \Delta, \Xi_2, \dotsc, \Xi_n; \Xi_N; \Gamma_{N1};
            \Delta_{N1}; p_1, \Xi'_1; \cdot; \lstack{C'}, f; \lstack{P}; \Omega_N; \Delta_N \rightarrow
            \outsem$ \\
      \dots \hfill (4) from lemma (1) \\

      $f' = (\Delta_a, p_1 - \Xi_1; \Delta_b - \Xi_1; \cdot; p; \Omega; \cdot;
            \Upsilon)$ \\

      $\dc{AB} \Gamma; \Xi_N, \Xi_1; \Gamma_{N1}; \Delta_{N1}; B; f';
      \lstack{P'}; \Omega_N; \Delta, \Xi_2, \dotsc, \Xi_n \rightarrow \outsem$ \\
      \dots \hfill (5) using Corollary~\ref{thm:match_to_derivation} on (4) \\

      $\dc{AB} \Gamma; \Xi_N, \Xi_1; \Gamma_{N1}, \Gamma_1; \Delta_{N1}, \Delta_1;
            \cdot; f'; \lstack{P'}; \Omega_N; \Delta, \Xi_2, \dotsc, \Xi_n \rightarrow \outsem$
      \\ \dots \hfill (6) applying Theorem~\ref{thm:comprehension_derivation} on (5)

      $\forall_{\Omega_x}. ($ if $\dz \Gamma; \Delta, \Xi_2, \dotsc, \Xi_n; \Xi_N, \Xi_1;
            \Gamma_{N1}, \Gamma_1; \Delta_{N1}, \Delta_1; \Omega_x \rightarrow
            \outsem$ then \\ \hspace*{0.5cm} $\dz \Gamma;
            \Delta, \Xi_2, \dotsc, \Xi_n; \Xi_N, \Xi_1; \Gamma_{N1}; \Delta_{N1}; B, \Omega_x
            \rightarrow \outsem)$ \hfill (7) from
      Theorem~\ref{thm:comprehension_derivation} on (5) \\

      $\contc{AB} \Gamma; \Delta, \Xi_2, \dotsc, \Xi_n; \Xi_N, \Xi_1; \Gamma_{N1},
         \Gamma_1; \Delta_{N1}, \Delta_1; f'; \lstack{P'}; \Omega_N
         \rightarrow \outsem$\\ \dots \hfill (8) inversion of (6) \\
        
        By inverting (8) we either fail (thus $n = 1$) or we get a new match.
        For the latter case, we apply mutual induction to get the remaining $n -
        1$ comprehensions.
      
   \item With backtracking to the linear continuation frame of stack $\lstack{C}$:
      
      $\mz \Gamma; \Xi_1 \rightarrow A$ \hfill (2) from lemma \\

      $f = (\Delta_a, p_1; \Delta'_b; \cdot; p; \Omega; \cdot; \Upsilon)$ \hfill (3) frame to backtrack to \\
      turns into $f' = (\Delta_a, p_1, \Delta'''_b, p_2; \Delta''_b; \cdot; p; \Omega; \cdot; \Upsilon)$ \hfill (4) resulting frame \\

      $\mc{AB} \Gamma; \Delta, \Xi_2, \dotsc, \Xi_n; \Xi_N; \Gamma_{N1};
\Delta_{N1}; p_2, \Xi'_1; \cdot; \lstack{C'}, f'; \lstack{P}; \Omega_N; \Delta_N \rightarrow
\outsem$\\ \dots \hfill (5) from lemma (1) \\
      
      Use the same approach as the case with no backtracking.
      
   \item With backtracking to a persistent continuation frame of stack
   $\lstack{P}$:

      $\mz \Gamma; \Xi_1 \rightarrow A$ \hfill (2) from lemma \\

      $f = [\Gamma''_1, p_2, \Gamma''_2; \Delta_N; \cdot; \bang p; \Omega; \cdot; \Upsilon]$ \hfill (4) from theorem \\
      turns into $f' = [\Gamma''_2; \Delta_N; \cdot; \bang p; \Omega; \cdot;
      \Upsilon]$ \hfill (5) from theorem \\

      $\mc{AB} \Gamma; \Delta, \Xi_2, \dotsc, \Xi_n; \Xi_N; \Gamma_{N1};
\Delta_{N1}; \Xi_1; \cdot; \lstack{C'}; \lstack{P'}, f', \lstack{P''}; \Omega_N; \Delta_N \rightarrow
\outsem$ \\ \dots \hfill (6) from theorem \\
         
      Use the same approach as the case with no backtracking.
      
\end{itemize}
\end{proof}

For this theorem, we derive three important propositions for HLD: (1) the final
derivation proposition; (2) the matching propositions for each comprehension
application; (2) derivation implications to get from (1) to a derivation
judgment without any derivations of the comprehension. However, the theorem
starts from an initial stack with frames and the comprehension process starts
with an empty stack. We need another theorem that gives us one application of
the comprehension plus the other $n$ that we get from this theorem.

\begin{lemma}[All comprehensions]\label{thm:comprehension}
Consider a triplet $T = A; \Gamma; \Delta_{N}$ and a comprehension $AB =
\compsz{A}{B}$. Assume that there exists $n \geq 0$ applications of $AB$
where the $i_{th}$ application is related to the following contexts:
\begin{description}
   \item[$\Delta_i$]: context of derived linear facts;
   \item[$\Gamma_i$]: context of derived persistent facts;
   \item[$\Xi_i$]: context of consumed linear facts.
\end{description}

Since each application consumes $\Xi_i$ then the initial context $\Delta_N =
\Delta, \Xi_1, \dotsc, \Xi_n$.

If $\mc{AB} \Gamma; \Delta, \Xi_1, \dotsc, \Xi_n;
\Xi_N; \Gamma_{N1}; \Delta_{N1}; \cdot; A; \cdot; \cdot; \Omega_N;
\Delta, \Xi_1, \dotsc, \Xi_n \rightarrow \outsem$ (well-formed in
relation to $T$) then:

\begin{itemize}[leftmargin=*]
   \item $n$ comprehensions are derived:\\
   $\done \Gamma; \Delta_N; \Xi_N, \Xi_1, \dotsc, \Xi_n; \Gamma_{N1},
   \Gamma_1, \dotsc, \Gamma_n; \Delta_{N1}, \Delta_1, \dotsc, \Delta_n; \Omega_N \rightarrow \outsem$
   \item $n$ $\mz$propositions for the $n$ comprehension matches:
   \begin{itemize}[leftmargin=\secondm]
      \item $\mz \Gamma; \Xi_1 \rightarrow A$
      \item $\dots$
      \item $\mz \Gamma; \Xi_n \rightarrow A$
   \end{itemize}
   \item $n$ derivation implications for HLD: \\
   $\forall_{\Omega_x}.($ if $\dz \Gamma; \Delta, \Xi_{i+1}, \dotsc, \Xi_n; \Xi_N, \Xi_1,
         \dotsc, \Xi_i; \Gamma_{N1}, \Gamma_1, \dotsc, \Gamma_i; \Delta_{N1},
         \Delta_1, \dotsc, \Delta_i; \Omega_x \rightarrow \outsem$ then $\dz \Gamma; \Delta, \Xi_{i+1}, \dotsc, \Xi_n; \Xi_N, \Xi_1,
         \dotsc,
         \Xi_i; \Gamma_{N1}, \Gamma_1, \dotsc, \Gamma_{i-1}; \Delta_{N1},
         \Delta_1, \dotsc, \Delta_{i-1}; B, \Omega_x \rightarrow \outsem)$
\end{itemize}
\end{lemma}

\begin{proof}
Apply Lemma~\ref{thm:comprehension_body_match} to get two sub-cases:
   
\begin{itemize}[leftmargin=*]
   \item Match fails:
   
   
   $\done \Gamma; \Delta_N; \Xi_N; \Gamma_{N1}; \Delta_{N1}; \Omega_N
   \rightarrow \outsem$\\
   \dots \hfill (1) no comprehension application was possible ($n = 0$)\\
   
   \item Match succeeds:
   
   $\mc{AB} \Gamma; \Xi_2, \dotsc, \Xi_n; \Xi_N; \Gamma_{N1}; \Delta_{N1};
\Xi_1; \cdot; \lstack{C}; \lstack{P}; \Omega_N; \Delta_N \rightarrow \outsem$
   
   \dots \hfill (1) result from Lemma~\ref{thm:comprehension_body_match}
   
   $\mz \Gamma; \Xi_1 \rightarrow A$
   \hfill (2) also from Lemma~\ref{thm:comprehension_body_match}
   
   $\dc{AB} \Gamma; \Xi_N, \Xi_1; \Gamma_{N1}; \Delta_{N1}; B; \lstack{C'};
\lstack{P'}; \Omega_N; \Delta, \Xi_2, \dotsc, \Xi_n \rightarrow \outsem$
   
   \dots \hfill (3) applying Corollary~\ref{thm:match_to_derivation} on (1)
   
   $\dc{AB} \Gamma; \Xi_N, \Xi_1; \Gamma_{N1}, \Gamma_1; \Delta_{N1}, \Delta_1;
   \cdot; \lstack{C'}; \lstack{P'}; \Omega_N; \Delta, \Xi_2, \dotsc, \Xi_n \rightarrow \outsem$
   
   \dots \hfill (4) using Theorem~\ref{thm:comprehension_derivation} on (3)\\
   
   $\forall_{\Omega_x}. ($ if $\dz \Gamma; \Delta, \Xi_2, \dotsc, \Xi_n; \Xi_N,
         \Xi_1; \Gamma_{N1}, \Gamma_1; \Delta_{N1}, \Delta_1; \Omega_x
         \rightarrow \outsem$ then
   
    \hspace*{0.5cm} $\dz \Gamma; \Delta, \Xi_2, \dotsc, \Xi_n; \Xi_N, \Xi_1; \Gamma_{N1};
    \Delta_{N1}; B, \Omega_x \rightarrow \outsem)$ \\ \dots \hfill (5)
   from the theorem applied in (4)\\
   
   $\contc{AB} \Gamma; \Delta, \Xi_2, \dotsc, \Xi_n; \Xi_N, \Xi_1; \Gamma_{N1},
   \Gamma_1; \Delta_{N1}, \Delta_1; \lstack{C'}; \lstack{P'}; \Omega_N \rightarrow \outsem$
   
   \dots \hfill (6) inversion of (5)\\
   
   Invert (6) to get either $n = 1$ application of the comprehension or apply Theorem~\ref{thm:multiple_comprehension_derivation} to the inversion to get the remaining $n-1$. 
\end{itemize}
\end{proof}

If the previous lemma, the comprehension is applied for as many times as the
database allows. We now have to map these $n$ applications to HLD by rebuilding
the proof tree for these $n$ matches and derivations and then using
$n$ when "guessing" the number of iterative definitions in HLD.


\subsection{Aggregate soundness}


Proving that deriving an aggregate in LLD is sound in relation to HLD is built
from 4 results: (1) proving that matching the aggregate's LHS is sound in
relation to HLD; (2) proving that updating the continuation stacks makes them
suitable for use in the next aggregate applications; (3) proving that deriving
the aggregate's RHS is sound in relation to HLD; (4) proving that we can
apply as many aggregates as the database allows.

\begin{lemma}[Aggregate LHS match]\label{thm:aggregate_body_match}
Consider an aggregate $\m{agg}$, where $\Pi(\m{agg}) = \forall_{\widehat{v}, \Sigma'}.
   (\defstwo{agg}{\widehat{v}}{\Sigma'} \lolli ((\lambda x. C x)\Sigma' \with (\forall_{\widehat{x}, \sigma}.
                                                (A \lolli B \otimes
                                                 \defstwo{agg}{\widehat{v}}{\sigma
                                                 ::\Sigma'}))))$,
a triplet $T = A; \Gamma; \Delta_{N}$ and a context $\Delta_{N} = \Delta_1,
  \Delta_2, \Delta'$.

  If $s_1 = \matstatea{\Delta_N}{\lstack{C};
      \lstack{P}}{\Gamma}{\Delta_1, \Delta_2}{\Omega}{\Delta' \rightarrow
         \Omega'}{\Sigma}$ is well-formed in relation to $T$ and
         $\transs{s_1}{s_2}$ then either:

\begin{itemize}[leftmargin=*]
   \item Match succeeds with no backtracking to frames of stack $\lstack{C}$
   or $\lstack{P}$ ($\lstack{C} \neq \cdot$):

   \begin{itemize}[leftmargin=\secondm]
      \item $s_2 = \matstatea{\Delta_N}{\lstack{C'}, \lstack{C}; \lstack{P'},
         \lstack{P}}{\Gamma}{\Delta_1}{\cdot}{\Delta', \Delta_2 \rightarrow \Omega' \otimes split(\Omega)}{\Sigma}$
   \end{itemize}

   \item Match succeeds with no backtracking to frames of stack $\lstack{P}$
($\lstack{C} = \cdot$):
\begin{itemize}[leftmargin=\secondm]
   \item $\mz \Gamma; \Delta_2 \rightarrow A$
   \item $\ma{AG} \Gamma; \Delta_1; \Xi_N; \Gamma_{N1}; \Delta_{N1}; \Xi,
      \Delta_2; \cdot; \lstack{C'}; \lstack{P'}, \lstack{P}; \Omega_N; \Delta_N;
      \Sigma \rightarrow \outsem$ (well-formed in relation to $T$)
\end{itemize}

\item Match succeeds with backtracking to a linear continuation frame in
stack $\lstack{C}$ ($\lstack{C} \neq \cdot$):

\begin{itemize}[leftmargin=\secondm]
   \item $\mz \Gamma; \Xi_1, \dotsc, \Xi_m, p_2, \Xi_c$
   \item $\exists_{f \in \lstack{C}}. f = (\Delta_a; \Delta_{b_1}, p_2,
         \Delta_{b_2}; p; \Xi_1, \dotsc, \Xi_m; \Omega_1, \dotsc,
         \Omega_k; \Lambda_1, \dotsc, \Lambda_m; \Upsilon_1, \dotsc,
         \Upsilon_n)$
   \item $\lstack{C} = \lstack{C'}, f, \lstack{C''}$
   \item $f$ turns into $f' = (\Delta_a, \Delta_{b_1}, p_2;
         \Delta_{b_2}; p; \Xi_1, \dotsc, \Xi_m;
         \Omega_1, \dotsc, \Omega_k; \Lambda_1, \dotsc, \Lambda_m;
         \Upsilon_1, \dotsc, \Upsilon_n)$
   \item $\ma{AG} \Gamma; \Delta_c; \Xi_N; \Gamma_{N1}; \Delta_{N1}; \Xi_1,
      \dotsc, \Xi_m, p_2, \Xi_c; \cdot; \lstack{C'''}, f', \lstack{C''}; \lstack{P};
      \Omega_N; \Delta_N; \Sigma \rightarrow \outsem$ (well-formed in relation to $T$)
   \item $\Delta_c = (\Delta_1, \Delta_2, \Xi) - (\Xi_1, \dotsc, \Xi_m,
         p_2, \Xi_c)$
\end{itemize}

\item Match succeeds with backtracking to a persistent continuation frame
in stack $\lstack{C}$ ($\lstack{C} \neq \cdot$):
\begin{itemize}[leftmargin=\secondm]
   \item $\mz \Gamma; \Delta_{c_2}, \Xi_1, \dotsc, \Xi_m \rightarrow A$
   \item $\exists_{f \in \lstack{C}}. f = [\Gamma_1, p_2, \Gamma_2; \Delta_{c_1},
      \Delta_{c_2}; \Xi_1, \dotsc, \Xi_m; \bang p; \Omega_1, \dotsc, \Omega_k;
      \Lambda_1, \dotsc, \Lambda_m; \Upsilon_1, \dotsc, \Upsilon_n]$
   \item $\lstack{C} = \lstack{C'}, f, \lstack{C''}$
   \item $f$ turns into $f' = [\Gamma_2; \Delta_{c_1}, \Delta_{c_2};
      \Xi_1, \dotsc, \Xi_m; \bang p; \Omega_1, \dotsc, \Omega_k; \Lambda_1,
      \dotsc, \Lambda_m; \Upsilon_1, \dotsc, \Upsilon_n]$
   \item $\ma{AG} \Gamma; \Delta_{c_1}; \Xi_N; \Gamma_{N1}; \Delta_{N1};
      \Delta_{c_2}, \Xi_1, \dotsc, \Xi_m; \cdot; \lstack{C'''}, f', \lstack{C''}; \lstack{P};
      \Omega_N; \Delta_N; \Sigma \rightarrow \outsem$ (well-formed in relation to $T$)
   \item $\Delta_{c_1}, \Delta_{c_2} = (\Delta_1, \Delta_2, \Xi) - (\Xi_1, \dotsc, \Xi_m)$
\end{itemize}

\item Match succeeds with backtracking to a persistent continuation frame
in stack $\lstack{P}$ ($\lstack{C} = \cdot$):
   \begin{itemize}[leftmargin=\secondm]
      \item $\mz \Gamma; \Delta_{c_2}, \Xi_1, \dotsc, \Xi_m \rightarrow A$
      \item $\exists_{f \in \lstack{P}}. f = [\Gamma_1, p_2, \Gamma_2; \Delta_{c_1}, \Delta_{c_2};
   \Xi_1, \dotsc, \Xi_m; \bang p; \Omega_1, \dotsc, \Omega_k; \Lambda_1,
   \dotsc, \Lambda_m; \Upsilon_1, \dotsc, \Upsilon_n]$
      \item $\lstack{P} = \lstack{P'}, f, \lstack{P''}$
      \item $f$ turns into $f' = [\Gamma_2; \Delta_{c_1},
         \Delta_{c_2}; \Xi_1, \dotsc, \Xi_m; \bang p; \Omega_1, \dotsc,
         \Omega_k; \Lambda_1, \dotsc, \Lambda_m; \Upsilon_1, \dotsc, \Upsilon_n]$
      \item $\ma{AG} \Gamma; \Delta_{c_1}; \Xi_N; \Gamma_{N1}; \Delta_{N1};
         \Delta_{c_2}, \Xi_1, \dotsc, \Xi_m; \cdot; \lstack{C'}; \lstack{P'''}, f', \lstack{P''};
         \Omega_N; \Delta_N; \Sigma \rightarrow \outsem$
         (well-formed in relation to $T$)
      \item $\Delta_{c_1}, \Delta_{c_2} = (\Delta_1, \Delta_2, \Xi) - (\Xi_1, \dotsc,
            \Xi_m)$
   \end{itemize}


\end{itemize}

If $\contstatea{\Delta_N}{\lstack{C}; \lstack{P}}{\Gamma}{\Sigma}$ is
well-formed in relation to $T$ then either:

\begin{itemize}[leftmargin=*]
   \item Match succeeds with no backtracking to frames of stack $\lstack{P}$
($\lstack{C} = \cdot$):
\begin{itemize}[leftmargin=\secondm]
   \item $\mz \Gamma; \Delta_2 \rightarrow A$
   \item $\ma{AG} \Gamma; \Delta_1; \Xi_N; \Gamma_{N1}; \Delta_{N1}; \Xi,
      \Delta_2; \cdot; \lstack{C'}; \lstack{P'}, \lstack{P}; \Omega_N; \Delta_N;
      \Sigma \rightarrow \outsem$ (well-formed in relation to $T$)
\end{itemize}

\item Match succeeds with backtracking to a linear continuation frame in
stack $\lstack{C}$ ($\lstack{C} \neq \cdot$):

\begin{itemize}[leftmargin=\secondm]
   \item $\mz \Gamma; \Xi_1, \dotsc, \Xi_m, p_2, \Xi_c$
   \item $\exists_{f \in \lstack{C}}. f = (\Delta_a; \Delta_{b_1}, p_2,
         \Delta_{b_2}; p; \Xi_1, \dotsc, \Xi_m; \Omega_1, \dotsc,
         \Omega_k; \Lambda_1, \dotsc, \Lambda_m; \Upsilon_1, \dotsc,
         \Upsilon_n)$
   \item $\lstack{C} = \lstack{C'}, f, \lstack{C''}$
   \item $f$ turns into $f' = (\Delta_a, \Delta_{b_1}, p_2;
         \Delta_{b_2}; p; \Xi_1, \dotsc, \Xi_m;
         \Omega_1, \dotsc, \Omega_k; \Lambda_1, \dotsc, \Lambda_m;
         \Upsilon_1, \dotsc, \Upsilon_n)$
   \item $\ma{AG} \Gamma; \Delta_c; \Xi_N; \Gamma_{N1}; \Delta_{N1}; \Xi_1,
      \dotsc, \Xi_m, p_2, \Xi_c; \cdot; \lstack{C'''}, f', \lstack{C''}; \lstack{P};
      \Omega_N; \Delta_N; \Sigma \rightarrow \outsem$ (well-formed in relation to $T$)
   \item $\Delta_c = (\Delta_1, \Delta_2, \Xi) - (\Xi_1, \dotsc, \Xi_m,
         p_2, \Xi_c)$
\end{itemize}

\item Match succeeds with backtracking to a persistent continuation frame
in stack $\lstack{C}$ ($\lstack{C} \neq \cdot$):
\begin{itemize}[leftmargin=\secondm]
   \item $\mz \Gamma; \Delta_{c_2}, \Xi_1, \dotsc, \Xi_m \rightarrow A$
   \item $\exists_{f \in \lstack{C}}. f = [\Gamma_1, p_2, \Gamma_2; \Delta_{c_1},
      \Delta_{c_2}; \Xi_1, \dotsc, \Xi_m; \bang p; \Omega_1, \dotsc, \Omega_k;
      \Lambda_1, \dotsc, \Lambda_m; \Upsilon_1, \dotsc, \Upsilon_n]$
   \item $\lstack{C} = \lstack{C'}, f, \lstack{C''}$
   \item $f$ turns into $f' = [\Gamma_2; \Delta_{c_1}, \Delta_{c_2};
      \Xi_1, \dotsc, \Xi_m; \bang p; \Omega_1, \dotsc, \Omega_k; \Lambda_1,
      \dotsc, \Lambda_m; \Upsilon_1, \dotsc, \Upsilon_n]$
   \item $\ma{AG} \Gamma; \Delta_{c_1}; \Xi_N; \Gamma_{N1}; \Delta_{N1};
      \Delta_{c_2}, \Xi_1, \dotsc, \Xi_m; \cdot; \lstack{C'''}, f', \lstack{C''}; \lstack{P};
      \Omega_N; \Delta_N; \Sigma \rightarrow \outsem$ (well-formed in relation to $T$)
   \item $\Delta_{c_1}, \Delta_{c_2} = (\Delta_1, \Delta_2, \Xi) - (\Xi_1, \dotsc, \Xi_m)$
\end{itemize}

\item Match succeeds with backtracking to a persistent continuation frame
in stack $\lstack{P}$ ($\lstack{C} = \cdot$):
   \begin{itemize}[leftmargin=\secondm]
      \item $\mz \Gamma; \Delta_{c_2}, \Xi_1, \dotsc, \Xi_m \rightarrow A$
      \item $\exists_{f \in \lstack{P}}. f = [\Gamma_1, p_2, \Gamma_2; \Delta_{c_1}, \Delta_{c_2};
   \Xi_1, \dotsc, \Xi_m; \bang p; \Omega_1, \dotsc, \Omega_k; \Lambda_1,
   \dotsc, \Lambda_m; \Upsilon_1, \dotsc, \Upsilon_n]$
      \item $\lstack{P} = \lstack{P'}, f, \lstack{P''}$
      \item $f$ turns into $f' = [\Gamma_2; \Delta_{c_1},
         \Delta_{c_2}; \Xi_1, \dotsc, \Xi_m; \bang p; \Omega_1, \dotsc,
         \Omega_k; \Lambda_1, \dotsc, \Lambda_m; \Upsilon_1, \dotsc, \Upsilon_n]$
      \item $\ma{AG} \Gamma; \Delta_{c_1}; \Xi_N; \Gamma_{N1}; \Delta_{N1};
         \Delta_{c_2}, \Xi_1, \dotsc, \Xi_m; \cdot; \lstack{C'}; \lstack{P'''}, f', \lstack{P''};
         \Omega_N; \Delta_N; \Sigma \rightarrow \outsem$
         (well-formed in relation to $T$)
      \item $\Delta_{c_1}, \Delta_{c_2} = (\Delta_1, \Delta_2, \Xi) - (\Xi_1, \dotsc,
            \Xi_m)$
   \end{itemize}

\end{itemize}
\end{lemma}

Proving this particular lemma follows the same strategy used for
Lemma~\ref{thm:body_match}. Next, we need to prove that, when matching succeeds,
the continuation stack is corrected for the next application of the aggregate.
Note that the aggregate value is appended to $\Sigma$ after the stack is
corrected.

\begin{theorem}[From update to derivation]\label{thm:agg_from_update_to_derivation}
Consider an aggregate $\m{agg}$, where $\Pi(\m{agg}) = \forall_{\widehat{v}, \Sigma'}.
   (\defstwo{agg}{\widehat{v}}{\Sigma'} \lolli ((\lambda x. C x)\Sigma' \with (\forall_{\widehat{x}, \sigma}.
                                                (A \lolli B \otimes
                                                 \defstwo{agg}{\widehat{v}}{\sigma
                                                 ::\Sigma'}))))$,
a triplet $T = A; \Gamma; \Delta_{N}$ and that $\Delta_{N} = \Delta,
  \Delta'$.
A well-formed stack update
$\fixstatea{\Delta}{\Xi; \Delta'}{\lstack{C}; \lstack{P}}{\Gamma}{\Sigma}$
implies
$\derstatea{\Delta}{\Xi, \Delta'}{\Gamma_{N1}}{\Delta_{N1}}{V :: \Sigma}{f';
   \lstack{P'}}{B\{\Psi(\widehat{x}), V / \widehat{x}, \sigma\}}$,
where:

\begin{itemize}[leftmargin=*]
   \item If $\lstack{C} = \cdot$ then $\lstack{C'} = \cdot$

   \item If $\lstack{C} = \lstack{C''},
   \lframe{\Delta_a}{\Delta_b}{p}{\Omega}{\cdot}{\Omega'}$
   then $\lstack{C'} = \lframe{\Delta_a - \Delta'}{\Delta_b -
      \Delta'}{p}{\Omega}{\cdot}{\Omega'}$

   \item $\lstack{P'}$ is the transformation of stack $\lstack{P}$, where for every frame $f \in
   \lstack{P}$ of the form $\pframe{\Gamma'}{\Delta_N}{\bang
      p}{\Omega}{\cdot}{\Omega'}$
   will turn into $f' = \pframe{\Gamma'}{\Delta_N-\Delta'}{\bang
      p}{\Omega}{\cdot}{\Omega'}$

\end{itemize}
\end{theorem}
\begin{proof}
Use induction on the size of the stack $\lstack{C}$.
\end{proof}


\begin{corollary}[Match to derivation]\label{thm:agg_match_to_derivation}
Consider an aggregate $\m{agg}$, where $\Pi(\m{agg}) = \forall_{\widehat{v}, \Sigma'}.
   (\defstwo{agg}{\widehat{v}}{\Sigma'} \lolli ((\lambda x. C x)\Sigma' \with (\forall_{\widehat{x}, \sigma}.
                                                (A \lolli B \otimes
                                                 \defstwo{agg}{\widehat{v}}{\sigma
                                                 ::\Sigma'}))))$,
a triplet $T = A; \Gamma; \Delta_{N}$ and that $\Delta_{N} = \Delta,
  \Delta'$.

A well-formed $\matstatea{\Delta_N}{\lstack{C}; \lstack{P}}{\Gamma}{\Delta}{\cdot}{\Delta' \rightarrow \Omega'}{\Sigma}$
implies
$\derstatea{\Delta}{\Xi, \Delta'}{\Gamma_{N1}}{\Delta_{N1}}{V :: \Sigma}{f';
   \lstack{P'}}{B\{\Psi(\widehat{x}), V / \widehat{x}, \sigma\}}$,
where:
   
\begin{itemize}[leftmargin=*]
   \item If $\lstack{C} = \cdot$ then $\lstack{C'} = \cdot$

   \item If $\lstack{C} = \lstack{C''},
   \lframe{\Delta_a}{\Delta_b}{p}{\Omega}{\cdot}{\Omega'}$
   then $\lstack{C'} = \lframe{\Delta_a - \Delta'}{\Delta_b -
      \Delta'}{p}{\Omega}{\cdot}{\Omega'}$

   \item $\lstack{P'}$ is the transformation of stack $\lstack{P}$, where for every frame $f \in
   \lstack{P}$ of the form $\pframe{\Gamma'}{\Delta_N}{\bang
      p}{\Omega}{\cdot}{\Omega'}$
   will turn into $f' = \pframe{\Gamma'}{\Delta_N-\Delta'}{\bang
      p}{\Omega}{\cdot}{\Omega'}$

\end{itemize}
\end{corollary}

\begin{proof}
Invert the assumption and then apply Theorem~\ref{thm:agg_from_update_to_derivation}.
\end{proof}


\paragraph{Aggregate Derivation}

We have just seen that after a single aggregate application, we add a value $V$
to the $\Sigma$ context and that the continuation stacks are now valid.
Now, we need to prove that deriving the aggregate's RHS
is sound in relation to HLD by using the new stacks.

\begin{theorem}[Aggregate derivation soundness]\label{thm:aggregate_derivation}
If $\derstatea{\Delta_N}{\Xi_N}{\Gamma_{N1}}{\Delta_{N1}}{\Sigma}{\lstack{C}; \lstack{P}}{\Omega_1,
\dotsc, \Omega_n}$ then:

\begin{itemize}[leftmargin=*]
   \item $\derstatea{\Delta_N}{\Xi_N}{\Gamma_{N1}, \Gamma_1, \dotsc,
      \Gamma_n}{\Delta_{N1}, \Delta_1, \dotsc, \Delta_n}{\Sigma}{\lstack{C};
         \lstack{P}}{\cdot}$

   \item If $\dz{\Gamma}{\Pi}{\Delta_N}{\Xi_N}{\Gamma_{N1}, \Gamma_1, \dotsc,
         \Gamma_n}{\Delta_{N1}, \Delta_1, \dotsc, \Delta_n}{\Omega_x}{\outsem}$ then\\
         \hspace{2cm}
   $\dz{\Gamma}{\Pi}{\Delta_N}{\Xi_N}{\Gamma_{N1}}{\Delta_{N1}}{\Omega_1, \dotsc,
   \Omega_n, \Omega_x}{\outsem}$

\end{itemize}
\end{theorem}

\begin{proof}
Straightforward induction on $\Omega_1, \dotsc, \Omega_n$.
\end{proof}


\paragraph{Multiple Aggregate Derivation} We now prove that it is possible
to apply an aggregate several times in order to a get multiple values (one per
application). In turn, we also conclude important results for the
soundness of the aggregate computation mechanism.

\begin{theorem}[Multiple aggregate derivation]\label{thm:multiple_aggregate_derivation}
Consider an aggregate $\m{agg}$, where $\Pi(\m{agg}) = \forall_{\widehat{v}, \Sigma'}.
   (\defstwo{agg}{\widehat{v}}{\Sigma'} \lolli ((\lambda x. C x)\Sigma' \with (\forall_{\widehat{x}, \sigma}.
                                                (A \lolli B \otimes
                                                 \defstwo{agg}{\widehat{v}}{\sigma
                                                 ::\Sigma'}))))$,
and a triplet $T = A; \Gamma; \Delta_{N}$.
Assume that there exists $n \geq 0$ applications of $\m{agg}$
where the $i_{th}$ application is related to the following information:
\begin{description}
   \item[$\Delta_i$]: context of derived linear facts;
   \item[$\Gamma_i$]: context of derived persistent facts;
   \item[$\Xi_i$]: context of consumed linear facts;
   \item[$V_i$]: a value representing the aggregate application;
   \item[$\Psi_i$]: context representing new variable bindings for the aggregate.
\end{description}

Since each application consumes $\Xi_i$ then the initial context $\Delta_N =
\Delta, \Xi_1, \dotsc, \Xi_n$. We now define the two main implications of the
theorem.

\begin{itemize}[leftmargin=*]
   \item Assume that $\Delta_N = \Delta_a, \Delta_b$, $\Delta_b =
   \Delta'_b, p_1$ and there is a frame $f = \lframe{\Delta_a,
      p_1}{\Delta'_b}{p}{\Omega}{\cdot}{\Omega'_f}$

   If $s_1 = \matstatea{\Delta, \Xi_1, \dotsc, \Xi_n}{f; \lstack{P}}{\Gamma}{\Delta_a,
      \Delta'_b}{\Omega}{p_1 \rightarrow \Omega'_f \otimes p}{\Sigma}$
      (well-formed in relation to $T$) and $\transs{s_1}{s_2}$ then:

      \begin{itemize}[leftmargin=\secondm]
      \item $n$ values $V_i$ ($\Sigma' = V_n :: \dots :: V_1 :: \Sigma$)
      \item $n$ aggregate applications are derived:\\
      $s_2 = \derstatex{\Gamma}{\Delta}{\Xi, \Xi_1, \dotsc, \Xi_n}{\Gamma_{N1},
      \Gamma_1, \dotsc, \Gamma_n}{\Delta_{N1}, \Delta_1, \dotsc,
         \Delta_n}{(\lambda x.  C\{\Psi(\widehat{v})/\widehat{v}\} x) \Sigma,
            \Omega_N}$
      \item $n$ soundness proofs for the $n$ aggregate matches:
      \begin{itemize}[leftmargin=\thirdm]
         \item $\mz{\Gamma}{\Xi_1}{A}$
         \item $\dots$
         \item $\mz{\Gamma}{\Xi_n}{A}$
      \end{itemize}

      \item $n$ derivation implications for HLD: \\
      If $\dz{\Gamma}{\Pi}{\Delta, \Xi_{i+1}, \dotsc, \Xi_{n}}{\Xi, \Xi_1,
         \dotsc, \Xi_i}{\Gamma_{N1}, \Gamma_1, \dotsc, \Gamma_i}{\Delta_{N1},
            \Delta_1, \dotsc, \Delta_i}{\Omega_x}{\outsem}$ then\\
            \hspace{5cm} $\dz{\Gamma}{\Pi}{\Delta, \Xi_{i+1}, \dotsc, \Xi_{n}}{\Xi, \Xi_1,
            \dotsc, \Xi_i}{\Gamma_{N1}, \Gamma_1, \dotsc, \Gamma_{i-1}}{\Delta_{N1},
            \Delta_1, \dotsc, \Delta_{i-1}}{B, \Omega_x}{\outsem}$
   \end{itemize}


   \item Assume that there is a frame $f = \pframe{\Gamma'}{\Delta_N}{\bang
      p}{\Omega}{\cdot}{\Omega_f}$.
      
   If $s_1 = \matstatea{\Delta_N}{\cdot;
      f, \lstack{P}}{\Gamma}{\Delta_N}{\Omega}{\cdot \rightarrow \bang p \otimes \Omega_f}{\Sigma}$
      (well-formed in relation to $T$) and $\transs{s_1}{s_2}$ then:

      \begin{itemize}[leftmargin=\secondm]
      \item $n$ values $V_i$ ($\Sigma' = V_n :: \dots :: V_1 :: \Sigma$)
      \item $n$ aggregate applications are derived:\\
      $s_2 = \derstatex{\Gamma}{\Delta}{\Xi, \Xi_1, \dotsc, \Xi_n}{\Gamma_{N1},
      \Gamma_1, \dotsc, \Gamma_n}{\Delta_{N1}, \Delta_1, \dotsc,
         \Delta_n}{(\lambda x.  C\{\Psi(\widehat{v})/\widehat{v}\} x) \Sigma,
            \Omega_N}$
      \item $n$ soundness proofs for the $n$ aggregate matches:
      \begin{itemize}[leftmargin=\thirdm]
         \item $\mz{\Gamma}{\Xi_1}{A}$
         \item $\dots$
         \item $\mz{\Gamma}{\Xi_n}{A}$
      \end{itemize}

      \item $n$ derivation implications for HLD: \\
      If $\dz{\Gamma}{\Pi}{\Delta, \Xi_{i+1}, \dotsc, \Xi_{n}}{\Xi, \Xi_1,
         \dotsc, \Xi_i}{\Gamma_{N1}, \Gamma_1, \dotsc, \Gamma_i}{\Delta_{N1},
            \Delta_1, \dotsc, \Delta_i}{\Omega_x}{\outsem}$ then\\
            \hspace{5cm} $\dz{\Gamma}{\Pi}{\Delta, \Xi_{i+1}, \dotsc, \Xi_{n}}{\Xi, \Xi_1,
            \dotsc, \Xi_i}{\Gamma_{N1}, \Gamma_1, \dotsc, \Gamma_{i-1}}{\Delta_{N1},
            \Delta_1, \dotsc, \Delta_{i-1}}{B, \Omega_x}{\outsem}$
   \end{itemize}

\end{itemize}
   
\end{theorem}
\begin{proof}
By mutual induction, first on either the size of $\Delta'_b$ (second argument of
the linear continuation frame) or $\Gamma'$ (second argument of the
persistent frame in $\lstack{P}$) and then on the size of $\lstack{C},
\lstack{P}$.  We only show how to prove the first implication since the
second implication is proven in a similar way.

$\transs{\matstatea{\Delta, \Xi_1, \dotsc, \Xi_n}{f; \lstack{P}}{\Gamma}{\Delta_a,
      \Delta'_b}{\Omega}{p_1 \rightarrow \Omega'_f \otimes p}{\Sigma}}{s_2}$
         \hfill (1) assumption\\

By applying Lemma~\ref{thm:aggregate_body_match} to (1), we get:

\begin{itemize}[leftmargin=*]
   \item Failure:
   
   $s_2 = \derstatex{\Gamma}{\Delta_N}{\Xi}{\Gamma_{N1}}{\Delta_{N1}}
{(\lambda x.  C\{\Psi(\widehat{v})/\widehat{v}\} x) \Sigma,
   \Omega_N}$ \hfill (2) from lemma, thus $n = 0$\\
   
   \item Success with no backtracking to frames of stack $\lstack{C}$ or
   $\lstack{P}$:
   
      $s_2 = \matstatea{\Delta_N}{\lstack{C'}, f; \lstack{P}}{\Gamma}
         {\Delta, \Xi_2, \dotsc, \Xi_n}{\cdot}{p_1, \Xi'_1 \rightarrow \Omega'_f \otimes p
                                \otimes split(\Omega)}{\Sigma}$ \hfill (2) from
                                lemma \\
      $\Xi_1 = \Xi'_1, p_1$ \hfill (3) by definition \\

      $\feq{A}{\Omega'_f \otimes p \otimes split(\Omega)}$ \hfill (4) by
      well-formedness \\
      $\mz{\Gamma}{\Xi_1}{A}$ \hfill (5) from match equivalence
      theorem and split equivalence on (4) \\

      $f = \lframe{\Delta_a, p_1}{\Delta'_b}{p}{\Omega}{\cdot}{\Omega'_f}$
      \hfill (6) definition \\

      Now, we apply Corollary~\ref{thm:agg_match_to_derivation} on (2) \\

      $f' = \lframe{\Delta_a, p_1 - \Xi_1}{\Delta_b -
         \Xi_1}{p}{\Omega}{\cdot}{\Omega'_f}$ \hfill (7) from the Corollary \\


      $\derstatea{\Delta, \Xi_2, \dotsc, \Xi_n}{\Xi,
         \Xi_1}{\Gamma_{N1}}{\Delta_{N1}}{V_1 :: \Sigma}{f';
            \lstack{P'}}{B\{\Psi(\widehat{x}), V / \widehat{x}, \sigma\}}$
      \dots \hfill (8) from the Corollary \\

      $\derstatea{\Delta, \Xi_2, \dotsc, \Xi_n}{\Xi, \Xi_1}{\Gamma_{N1},
         \Gamma_1}{\Delta_{N1}, \Delta_1}{\Sigma}{f'; \lstack{P'}}{\cdot}$
      \\ \dots \hfill (9) applying Theorem~\ref{thm:aggregate_derivation} on (8)

      If $\dz{\Gamma}{\Pi}{\Delta, \Xi_2, \dotsc, \Xi_n}{\Xi, \Xi_1}{\Gamma_{N1},
         \Gamma_1}{\Delta_{N1}, \Delta_1}{\Omega_x}{\outsem}$ then \\
         \hspace{2cm}
         $\dz{\Gamma}{\Pi}{\Delta, \Xi_2, \dotsc,
            \Xi_n}{\Xi, \Xi_1}{\Gamma_{N1}}{\Delta_{N1}}{B\{\Psi(\widehat{x}), V
               / \widehat{x}, \sigma\}, \Omega_x}{\outsem}$
            \hfill (10) from
      Theorem~\ref{thm:aggregate_derivation} on (8) \\

      $\contstatea{\Delta, \Xi_2, \dotsc, \Xi_n}{f' ; \lstack{P'}}{\Gamma}{V_1
         :: \Sigma}$ \hfill (11) next state of (9) \\

      By executing the next transition on (11) we either fail because there are
      no more candidates or no more frames and thus $n = 1$ or we have a new
      match from which we can apply the inductive hypothesis (smaller number of
      candidates and/or frames) to get the remaining $n - 1$ aggregate
      values.
      
   \item Success with backtracking to the linear continuation frame of stack $\lstack{C}$:
      
      $s_2 = \matstatea{\Delta_N}{\lstack{C'},
         f';\lstack{P}}{\Gamma}{\Delta, \Xi_2, \dotsc, \Xi_n}{\cdot}{\Delta'_f, p_2, \Xi'_1 \rightarrow \Omega'_f \otimes p
                  \otimes split(\Omega)}{\Sigma}$ \hfill (2) from lemma \\

      $\Xi_1 = \Delta'_f, p_2, \Xi'_1$ \hfill (3) by definition \\

      $\feq{A}{\Omega'_f \otimes p \otimes split(\Omega)}$ \hfill (4) by
      well-formedness \\
      $\mz{\Gamma}{\Xi_1}{A}$ \hfill (5) from match equivalence
      theorem and split equivalence on (4) \\

      $f = \lframe{\Delta_a, p_1}{\Delta'''_b, p_2, \Delta''_b}{p}{\Omega}{\cdot}{\Omega'_f}$
      (6) frame to backtrack to \\
      turns into $f' = \lframe{\Delta_a, p_1, \Delta'''_b, p_2}{\Delta''_b}{p}{\Omega}{\cdot}{\Omega'_f}$
      \hfill (4) resulting frame \\

      Use the same approach as the case with no backtracking.
      
   \item Success with backtracking to a persistent continuation frame of stack
   $\lstack{P}$:

      Use the same approach as before.
      
\end{itemize}
\end{proof}

This last theorem proves that from a certain initial continuation stack, we are
able to apply the aggregate multiple times (until the stack is exhausted). The
results of the theorem allows us to rebuild the proof tree in HLD since we get
the HLD matching and derivation propositions. What remains to be done is to
prove that we do the same for an empty continuation stack.

\begin{lemma}[All aggregates]\label{thm:aggregates}
Consider an aggregate $\m{agg}$, where $\Pi(\m{agg}) = \forall_{\widehat{v}, \Sigma'}.
   (\defstwo{agg}{\widehat{v}}{\Sigma'} \lolli ((\lambda x. C x)\Sigma' \with (\forall_{\widehat{x}, \sigma}.
                                                (A \lolli B \otimes
                                                 \defstwo{agg}{\widehat{v}}{\sigma
                                                 ::\Sigma'}))))$,
and a triplet $T = A; \Gamma; \Delta_{N}$.
Assume that there exists $n \geq 0$ applications of $\m{agg}$
where the $i_{th}$ application is related to the following information:
\begin{description}
   \item[$\Delta_i$]: context of derived linear facts;
   \item[$\Gamma_i$]: context of derived persistent facts;
   \item[$\Xi_i$]: context of consumed linear facts;
   \item[$V_i$]: a value representing the aggregate application;
   \item[$\Psi_i$]: context representing new variable bindings for the aggregate.
\end{description}

Since each application consumes $\Xi_i$ then the initial context $\Delta_N =
\Delta, \Xi_1, \dotsc, \Xi_n$.

If $s_1 = \matstatea{\Delta, \Xi_1, \dotsc, \Xi_n}{\cdot;
   \cdot}{\Gamma}{\Delta, \Xi_1, \dotsc, \Xi_n}{A}{\cdot \rightarrow \one}{\cdot}$
   (well-formed in relation to $T$) and $\transs{s_1}{s_2}$ then:

   \begin{itemize}[leftmargin=\secondm]
      \item $n$ values $V_i$ ($\Sigma' = V_n :: \dots :: V_1 :: \Sigma$)
      \item $n$ aggregate applications are derived:\\
      $s_2 = \derstatex{\Gamma}{\Delta}{\Xi, \Xi_1, \dotsc, \Xi_n}{\Gamma_{N1},
      \Gamma_1, \dotsc, \Gamma_n}{\Delta_{N1}, \Delta_1, \dotsc,
         \Delta_n}{(\lambda x.  C\{\Psi(\widehat{v})/\widehat{v}\} x) \Sigma,
            \Omega_N}$
      \item $n$ soundness proofs for the $n$ aggregate matches:
      \begin{itemize}[leftmargin=\thirdm]
         \item $\mz{\Gamma}{\Xi_1}{A}$
         \item $\dots$
         \item $\mz{\Gamma}{\Xi_n}{A}$
      \end{itemize}

      \item $n$ derivation implications for HLD: \\
      If $\dz{\Gamma}{\Pi}{\Delta, \Xi_{i+1}, \dotsc, \Xi_{n}}{\Xi, \Xi_1,
         \dotsc, \Xi_i}{\Gamma_{N1}, \Gamma_1, \dotsc, \Gamma_i}{\Delta_{N1},
            \Delta_1, \dotsc, \Delta_i}{\Omega_x}{\outsem}$ then\\
            \hspace{5cm} $\dz{\Gamma}{\Pi}{\Delta, \Xi_{i+1}, \dotsc, \Xi_{n}}{\Xi, \Xi_1,
            \dotsc, \Xi_i}{\Gamma_{N1}, \Gamma_1, \dotsc, \Gamma_{i-1}}{\Delta_{N1},
            \Delta_1, \dotsc, \Delta_{i-1}}{B, \Omega_x}{\outsem}$
   \end{itemize}


\end{lemma}

\begin{proof}
Apply Lemma~\ref{thm:aggregate_body_match} to get two sub-cases:
   
\begin{itemize}[leftmargin=*]
   \item Match fails:
   
   $s_2 = \derstatex{\Gamma}{\Delta_N}{\Xi}{\Gamma_{N1}}{\Delta_{N1}}
{(\lambda x.  C\{\Psi(\widehat{v})/\widehat{v}\} x) \Sigma,
   \Omega_N}$ \hfill (2) from lemma, thus $n = 0$\\
   
   \item Match succeeds:


      $s_2 = \matstatea{\Delta_N}{\lstack{C}; \lstack{P}}{\Gamma}
         {\Delta, \Xi_2, \dotsc, \Xi_n}{\cdot}{\Xi_1 \rightarrow \one \otimes
                                                 split(A)}{\cdot}$ \hfill (2) from
                                lemma \\
      $\feq{A}{\one \otimes split(\Omega)}$ \hfill (3) by well-formedness \\
      $\mz{\Gamma}{\Xi_1}{A}$ \hfill (4) from match equivalence
      theorem and split equivalence on (4) \\

      Now, we apply Corollary~\ref{thm:agg_match_to_derivation} on (2) \\

      $\derstatea{\Delta, \Xi_2, \dotsc, \Xi_n}{\Xi,
         \Xi_1}{\Gamma_{N1}}{\Delta_{N1}}{V_1 :: \cdot}{f';
            \lstack{P'}}{B\{\Psi(\widehat{x}), V / \widehat{x}, \sigma\}}$
      \dots \hfill (5) \\

      $\derstatea{\Delta, \Xi_2, \dotsc, \Xi_n}{\Xi, \Xi_1}{\Gamma_{N1},
         \Gamma_1}{\Delta_{N1}, \Delta_1}{V_1 :: \cdot}{f'; \lstack{P'}}{\cdot}$
      \hfill (6) applying Theorem~\ref{thm:aggregate_derivation} on (5)

      If $\dz{\Gamma}{\Pi}{\Delta, \Xi_2, \dotsc, \Xi_n}{\Xi, \Xi_1}{\Gamma_{N1},
         \Gamma_1}{\Delta_{N1}, \Delta_1}{\Omega_x}{\outsem}$ then \\
         \hspace{2cm}
         $\dz{\Gamma}{\Pi}{\Delta, \Xi_2, \dotsc,
            \Xi_n}{\Xi, \Xi_1}{\Gamma_{N1}}{\Delta_{N1}}{B\{\Psi(\widehat{x}), V
               / \widehat{x}, \sigma\}, \Omega_x}{\outsem}$
            \hfill (7) from
      Theorem~\ref{thm:aggregate_derivation} on (5) \\


      $\contstatea{\Delta, \Xi_2, \dotsc, \Xi_n}{f' ; \lstack{P'}}{\Gamma}{V_1
         :: \cdot}$ \hfill (8) next state of (6) \\

   
   When executing the next transition of state (6) we either get $n = 1$
   application of the aggregate or we apply
   Theorem~\ref{thm:multiple_aggregate_derivation} to get the remaining $n-1$
   applications.
\end{itemize}
\end{proof}


\subsection{Soundness of derivation}

We are finally ready to prove that the derivation of terms of the head of a rule
is sound in relation to HLD.

\begin{lemma}[Head derivation soundness]\label{thm:head_derivation_soundness}
If $\done \Gamma; \Delta_N; \Xi_N; \Gamma_{N1}; \Delta_{N1}; \Omega \rightarrow \outsem$ then
$\dz \Gamma; \Delta_N; \Xi_N; \Gamma_{N1}; \Delta_{N1}; \Omega \rightarrow \outsem$.
\end{lemma}

\begin{proof}\label{sec:derivation_theorem} Induction on $\Omega$. Most of the
sub-cases can be proven using the induction hypothesis or by straightforward
rule inference. The sub-cases for the comprehensions and aggregates are
complicated and are proved beflow.

\paragraph{Comprehensions} Apply Lemma~\ref{thm:comprehension} on the assumption
to get $n$ applications of the comprehension. Assume that 
$\Delta_N = \Delta, \Xi_1, \dotsc, \Xi_n$, where $\Xi_i$ are the facts consumed
and $\Gamma_i, \Delta_i$ the facts derived by the $i^{th}$ application.
The lemma proves the following:

\begin{itemize}[leftmargin=*]
   \item $\done \Gamma; \Delta; \Xi_N, \Xi_1, \dotsc, \Xi_n; \Gamma_{N1},
   \Gamma_1, \dotsc, \Gamma_n; \Delta_{N1}, \Delta_1, \dotsc, \Delta_n;
\Omega_N \rightarrow \outsem$ \hfill (1)
   \item $n$ propositions $\mz \Gamma; \Xi_i \rightarrow A$ \hfill (2)
   \item $n$ implications\\
   $\forall_{\Omega_x}.($ if $\dz \Gamma; \Delta, \Xi_{i+1}, \dotsc,
         \Xi_{n}; \Xi_N, \Xi_1,
         \dotsc, \Xi_i; \Gamma_{N1}, \Gamma_1, \dotsc, \Gamma_i; \Delta_{N1},
         \Delta_1, \dotsc, \Delta_i; \Omega_x \rightarrow \outsem$ then $\dz \Gamma; \Delta, \Xi_{i+1}, \dotsc, \Xi_n; \Xi_N, \Xi_1,
         \dotsc,
         \Xi_i; \Gamma_{N1}, \Gamma_1, \dotsc, \Gamma_{i-1}; \Delta_{N1},
         \Delta_1, \dotsc, \Delta_{i-1}; B, \Omega_x \rightarrow \outsem)$ \hfill (3) \\
\end{itemize}

\noindent From (1) we apply the inductive hypothesis since $\Omega$ gets
smaller:\\
$\dz \Gamma; \Delta; \Xi_N, \Xi_1, \dotsc, \Xi_n; \Gamma_{N1}, \Gamma_1,
\dotsc, \Gamma_n; \Delta_{N1}, \Delta_1, \dotsc, \Delta_n; \Omega \rightarrow
\outsem$ \\

\noindent Since we are building the proof tree backwards, starting from the final
derivation result, we first need to derive $\compz{0}{A}{B}$ by applying rules
$\dz \one$ and $\dz \m{comp}^0$:\\
$\dz \Gamma; \Delta; \Xi_N, \Xi_1, \dotsc, \Xi_n; \Gamma_{N1}, \Gamma_1,
\dotsc, \Gamma_n; \Delta_{N1}, \Delta_1, \dotsc, \Delta_n; \compz{0}{A}{B}, \Omega \rightarrow
\outsem$
\\

\noindent From result (4), we first rebuild the matching and derivation process of the
$n^{th}$ comprehension.  Using the $n^{th}$ implication (3) on (5):

\noindent $\dz \Gamma; \Delta, \Xi_n; \Xi_N, \Xi_1, \dotsc, \Xi_{n-1}; \Gamma_{N1}, \Gamma_1,
\dotsc, \Gamma_{n-1}; \Delta_{N1}, \Delta_1, \dotsc, \Delta_{n-1}; B, \compz{0}{A}{B},
\Omega \rightarrow \outsem$ \\

\noindent Using $\dz \lolli$ and the matching proposition (2) on (6), the $A \lolli B$
implication is reconstructed:

\noindent $\dz \Gamma; \Delta, \Xi_n; \Xi_N, \Xi_1, \dotsc, \Xi_{n-1}; \Gamma_{N1},
   \Gamma_1, \dotsc, \Gamma_{n-1}; \Delta_{N1}, \Delta_1, \dotsc, \Delta_{n-1};
A \lolli B, \compz{0}{A}{B}, \Omega \rightarrow \outsem$ \\

\noindent Now, $\compz{1}{A}{B}$ is rebuilt by applying $\dz \otimes$ followed by $\dz
\m{comp}^N$:

\noindent $\dz \Gamma; \Delta, \Xi_n; \Xi_N, \Xi_1, \dotsc, \Xi_{n-1}; \Gamma_{N1},
\Gamma_1, \dotsc, \Gamma_{n-1}; \Delta_{N1}, \Delta_1, \dotsc, \Delta_{n-1};
\compz{1}{A}{B}, \Omega \rightarrow \outsem$ \\

\noindent The last 4 steps are then applied $n-1$ times to get:

\noindent $\dz \Gamma; \Delta, \Xi_1, \dotsc, \Xi_n; \Xi_N; \Gamma_{N1}; \Delta_{N1};
\compz{n}{A}{B}, \Omega \rightarrow \outsem$ \\

\noindent Finally, to construct the conclusion and finish the proof, $\dz \m{comp}^*$ needs to
be applied:

\noindent $\dz \Gamma; \Delta, \Xi_1, \dotsc, \Xi_n; \Xi_N; \Gamma_{N1}; \Delta_{N1};
\compsz{A}{B}, \Omega \rightarrow \outsem$ \\

\noindent This finishes the sub-case for comprehensions.

\paragraph{Aggregates} Apply Lemma~\ref{thm:aggregates} on the assumption to get
$n$ applications of the aggregate. Assume that $\Delta_N = \Delta, \Xi_1,
\dotsc, \Xi_n$, where $\Xi_i$ are the facts consumed and $\Gamma_i, \Delta_i$
the facts derived by the $i^{th}$ application.  The lemma proves the following:

\begin{itemize}[leftmargin=*]
   \item $n$ values $\Sigma = V_n :: \dotsb :: V_1 :: \cdot$
   \item $\done \Gamma; \Delta; \Xi_N, \Xi_1, \dotsc, \Xi_n; \Gamma_{N1},
   \Gamma_1, \dotsc, \Gamma_n; \Delta_{N1}, \Delta_1, \dotsc, \Delta_n; (\lambda
         x. C)\Sigma, (\Omega_N \rightarrow \outsem$ \hfill (1)
   \item $n$ propositions $\mz \Gamma; \Xi_i \rightarrow A$ \hfill (2)
   \item $n$ implications\\
   $\forall_{\Omega_x}.($ if $\dz \Gamma; \Delta, \Xi_{i+1}, \dotsc,
         \Xi_{n}; \Xi_N, \Xi_1,
         \dotsc, \Xi_i; \Gamma_{N1}, \Gamma_1, \dotsc, \Gamma_i; \Delta_{N1},
         \Delta_1, \dotsc, \Delta_i; \Omega_x \rightarrow \outsem$ then $\dz \Gamma; \Delta, \Xi_{i+1}, \dotsc, \Xi_n; \Xi_N, \Xi_1,
         \dotsc,
         \Xi_i; \Gamma_{N1}, \Gamma_1, \dotsc, \Gamma_{i-1}; \Delta_{N1},
         \Delta_1, \dotsc, \Delta_{i-1}; B, \Omega_x \rightarrow \outsem)$ \hfill (3) \\
\end{itemize}

\noindent From (1) we apply the inductive hypothesis since $C$ is smaller than
the original aggregate:\\
$\dz \Gamma; \Delta; \Xi_N, \Xi_1, \dotsc, \Xi_n; \Gamma_{N1}, \Gamma_1,
\dotsc, \Gamma_n; \Delta_{N1}, \Delta_1, \dotsc, \Delta_n; (\lambda x. C)\Sigma, \Omega \rightarrow
\outsem$ \\

\noindent Since we are building the proof tree backwards, starting from the final
derivation result, we first need to derive $\aggz{0}{A}{B}{C}{\Sigma}$ by applying rules
$\dz \m{agg}^0$:\\
$\dz \Gamma; \Delta; \Xi_N, \Xi_1, \dotsc, \Xi_n; \Gamma_{N1}, \Gamma_1,
\dotsc, \Gamma_n; \Delta_{N1}, \Delta_1, \dotsc, \Delta_n;
\aggz{0}{A}{B}{C}{\Sigma}, \Omega \rightarrow
\outsem$
\\

\noindent From result (4), we first rebuild the matching and derivation process of the
$n^{th}$ aggregate.  Using the $n^{th}$ implication (3) on (5):

\noindent $\dz \Gamma; \Delta, \Xi_n; \Xi_N, \Xi_1, \dotsc, \Xi_{n-1}; \Gamma_{N1}, \Gamma_1,
\dotsc, \Gamma_{n-1}; \Delta_{N1}, \Delta_1, \dotsc, \Delta_{n-1}; B,
\aggz{0}{A}{B}{C}{\Sigma},
\Omega \rightarrow \outsem$ \\

\noindent Using $\dz \lolli$ and the matching proposition (2) on (6), the $A \lolli B$
implication is reconstructed:

\noindent $\dz \Gamma; \Delta, \Xi_n; \Xi_N, \Xi_1, \dotsc, \Xi_{n-1}; \Gamma_{N1},
   \Gamma_1, \dotsc, \Gamma_{n-1}; \Delta_{N1}, \Delta_1, \dotsc, \Delta_{n-1};
A \lolli B, \aggz{0}{A}{B}{C}{\Sigma}, \Omega \rightarrow \outsem$ \\

\noindent Next, we package the implication and the aggregate using $\dz
\otimes$:

\noindent $\dz \Gamma; \Delta, \Xi_n; \Xi_N, \Xi_1, \dotsc, \Xi_{n-1}; \Gamma_{N1},
\Gamma_1, \dotsc, \Gamma_{n-1}; \Delta_{N1}, \Delta_1, \dotsc, \Delta_{n-1};
(A \lolli B) \otimes \aggz{0}{A}{B}{C}{\Sigma}, \Omega \rightarrow \outsem$ \\

\noindent Now, we apply $\dz \forall$ to include the whole term and deconstruct $\Sigma$
into $x :: V_{n-1} :: \dotsb :: V_1 :: \cdot$ since $V_n$ is the $x$ variable:

\noindent $\dz \Gamma; \Delta, \Xi_n; \Xi_N; \Xi_1, \dotsc, \Xi_{n-1};
\Gamma_{N1}, \Gamma_1, \dotsc, \Gamma_{n-1}; \Delta_{N1}, \Delta_1, \dotsc,
\Delta_{n-1}; \forall_x. ((A \lolli B) \otimes \aggz{0}{A}{B}{C}{x :: V_{n-1} ::
   \dotsb :: V_1 :: \cdot}), \Omega \rightarrow \outsem$ \\

\noindent This last expression can be folded into $\aggz{1}{A}{B}{C}{V_{n-1} ::
   \dotsb :: V_1 :: \cdot}$:

\noindent $\dz \Gamma; \Delta, \Xi_n; \Xi_N; \Xi_1, \dotsc, \Xi_{n-1};
\Gamma_{N1}, \Gamma_1, \dotsc, \Gamma_{n-1}; \Delta_{N1}, \Delta_1, \dotsc,
\Delta_{n-1}; \aggz{1}{A}{B}{C}{V_{n-1} :: \dotsb :: V_1 :: \cdot}, \Omega
   \rightarrow \outsem$ \\

\noindent The last 5 steps are then applied $n-1$ times to get:

\noindent $\dz \Gamma; \Delta, \Xi_1, \dotsc, \Xi_n; \Xi_N; \Gamma_{N1}; \Delta_{N1};
\aggz{n}{A}{B}{C}{\cdot}, \Omega \rightarrow \outsem$ \\

\noindent Finally, to construct the conclusion and finish the proof, $\dz \m{agg}^*$ needs to
be applied:

\noindent $\dz \Gamma; \Delta, \Xi_1, \dotsc, \Xi_n; \Xi_N; \Gamma_{N1}; \Delta_{N1};
\aggsz{A}{B}{C}, \Omega \rightarrow \outsem$ \\

\noindent This completes the sub-case for aggregates.

\end{proof}

\subsection{Wrapping-up}

In order to bring everything together, we need to use the Head derivation
soundness lemma (Lemma~\ref{thm:head_derivation_soundness}) and the Body match
soundness lemma (Lemma~\ref{thm:body_match}).  We first prove that if we run one
step in the LLD semantics then there exists one rule where matching was
successful. Then, we prove that the application of such rule is sound in
relation to HLD.

\begin{theorem}[One rule]\label{thm:one_rule}
If $\doo \Gamma; \Delta; \Phi \rightarrow \outsem$ then
$\exists_{R \in \Phi}. \doo \Gamma; \Delta; R \rightarrow \outsem$.
\end{theorem}
\begin{proof}
Induction on the size of $\Phi$.

Inverting the assumption twice, we get $\mo \Gamma; \Delta; \cdot; A; B; \cdot;
(\Phi', \Delta) \rightarrow \outsem$, where $\Phi = A \lolli B, \Phi'$. Applying Lemma~\ref{thm:body_match} (body match soundness), we have two sub-cases:

\begin{itemize}[leftmargin=*]
   \item Match fails: \\ $\cont \cdot; H; (\Phi', \Delta); \Gamma \rightarrow
   \outsem$ \hfill (1)\\
   $\doo \Gamma; \Delta; \Phi' \rightarrow \outsem$ \hfill (2) inversion of (1) \\
   $\doo \Gamma; \Delta; R' \rightarrow \outsem$ \hfill (3) i.h. on (2) where $R' \in \Phi'$ \\
   \item Match succeeds: \\
   $\mo \Gamma; \Delta; \cdot; A; B; \cdot; (\cdot, \Delta) \rightarrow \outsem$ \hfill (1)\\
   $\ao \Gamma; \Delta; A \lolli B; (\cdot, \Delta) \rightarrow \outsem$ \hfill (2) rule $\ao \m{rule}$ on (1) \\
   $\doo \Gamma; \Delta; A \lolli B \rightarrow \outsem$ \hfill (3) rule $\doo \m{rule}$ on (2) \\
\end{itemize}
\end{proof}

\begin{theorem}[Soundness]\label{thm:soundness}
If $\doo \Gamma; \Delta; \Phi \rightarrow \outsem$ then $\exists_{R \in \Phi}.
\az \Gamma ; \Delta ; R \rightarrow \outsem$.
\end{theorem}

\begin{proof}
As follows:\\
$\doo \Gamma; \Delta; \Phi \rightarrow \outsem$ \hfill (1) assumption \\
$\doo \Gamma; \Delta; R \rightarrow \outsem$ \hfill (2) One rule theorem (\ref{thm:one_rule}) on (1), where $R \in \Phi$ \\
$\mo \Gamma; \Delta; \cdot; A; B; \cdot; R \rightarrow \outsem$ \hfill (3) inversion of (2) \\
$\mo \Gamma; \Delta_1; \Delta_2; \cdot; B; C; R \rightarrow \outsem$ \hfill (4) Body match soundness lemma (\ref{thm:body_match}) on (3), where $\Delta = \Delta_1, \Delta_2$ \\
$\mz \Gamma; \Delta_2 \rightarrow A$ \hfill (5) from same lemma \\
$\done \Gamma; \Delta_1; \Delta_2; \cdot; \cdot; B \rightarrow \outsem$ \hfill (6) inversion of (4) \\
$\dz \Gamma; \Delta_1; \Delta_2; \cdot; \cdot; B \rightarrow \outsem$ \hfill (7) Head derivation soundness lemma (\ref{thm:head_derivation_soundness}) on (6) \\
$\az \Gamma ; \Delta_1, \Delta_2 ; A \lolli B \rightarrow \outsem$ \hfill (8) rule $\az \m{rule}$ on (5) and (7)\\
\end{proof}
