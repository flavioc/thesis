Aggregates can be seen as a more general case of comprehensions. The major
differences are: (1) aggregates accumulate a list of values for each combination
and (2) aggregates have a second head to be derived before deriving the
remaining head terms.

The match judgment $\ma{AG} \Gamma; \Delta; \Xi_N; \Gamma_{N1};
\Delta_{N1}; \Xi; \Omega; \lstack{C}; \lstack{P}; \Omega_N; \Delta_N; \Sigma \rightarrow
\outsem$ is as follows:

\begin{enumerate}

   \item[$\Delta$] multi-set of linear facts remaining up to this point in the
   matching process;

   \item[$\Xi_N$] multi-set of linear facts used during the matching process of
   the body of the rule and all the previous aggregates;

   \item[$\Gamma_{N1}$] set of persistent facts derived up to this point in the
   head of the rule;

   \item[$\Delta_{N1}$] multi-set of linear facts derived up to this point in
   the head of the rule;

   \item[$\Xi$] multi-set of linear facts consumed up to this point;

   \item[$\Omega$] ordered list of terms that need to be matched for the
   comprehension to be applied;

   \item[$\lstack{C}$] continuation stack that contains both linear and persistent
   frames. The first frame must be linear;

   \item[$\lstack{P}$] initial part of the continuation stack with only persistent
   frames;

   \item[$AG$] aggregate $\aggsz{A}{B}{C}$ that is being matched;

   \item[$\Omega_N$] ordered list of remaining terms of the head of the rule to
   be derived;

   \item[$\Delta_N$] multi-set of linear facts that were still available after
   matching the body of the rule and all the previous aggregates. Note that
   $\Delta, \Xi = \Delta_N$;

   \item[$\Sigma$] the list of aggregated values.

\end{enumerate}

The judgment $\ma{AG}$is similar to $\mc{AB}$but it has the extra $\Sigma$ argument that
represents the list of aggregated values. We extend the $\Psi$ context to
include triplets $x : M : \tau$ (variable, term and type) instead of pairs $M :
\tau$ in order to be able to retrieve bound variables for all $\ma{AG}$judgments.
Remember that $\Psi$ is used for the quantification connectives in \fragment.

\subsubsection{Linear fact expressions}


\begin{multline}
\transx{
   \matstatea{\deltan}{\lstack{C};
      \lstack{P}}{\Gamma}{\Delta, p_1, \Delta''}{p, \Omega}{\Delta' \rightarrow
         \Omega'}{\Sigma}
}
{
   \matstatea{\deltan}{\lframe{\Delta,
   p_1}{\Delta''}{p}{\Omega}{\Delta'}{\Omega'}, \lstack{C}; \lstack{P}}{\Gamma}{\Delta,
      \Delta''}{\Omega}{\Delta', p \rightarrow \Omega' \otimes
      p}{\Sigma}\tag{agg match p ok}
}
\end{multline}

\[
\trans{
   \matstatea{\deltan}{\lstack{C}; \lstack{P}}{\Gamma}{\Delta}{p,
      \Omega}{\Delta' \rightarrow \Omega'}{\Sigma}
}
{
   \contstatea{\deltan}{\lstack{C} ; \lstack{P}}{\Gamma}{\Sigma}
}\tag{agg match p fail}
\]


\subsubsection{Persistent fact expressions}



\[
\trans{
   \matstatea{\Delta_N}{\cdot;
      \lstack{P}}{\Gamma, p_1, \Gamma''}{\Delta}{\bang p, \Omega}{\Delta' \rightarrow
         \Omega'}{\Sigma}
}
{
   \matstatea{\Delta_N}{\cdot; \pframe{\Gamma''}{\Delta}{\bang
   p}{\Omega}{\Delta'}{\Omega'}, \lstack{P}}{\Gamma, p_1, \Gamma''}{\Delta}{\Omega}
   {\Delta' \rightarrow \Omega' \otimes \bang p}{\Sigma}
}
\]

\[
\trans{
   \matstatea{\Delta_N}{\lstack{C};
      \lstack{P}}{\Gamma, p_1, \Gamma''}{\Delta}{\bang p, \Omega}{\Delta' \rightarrow
         \Omega'}{\Sigma}
}
{
   \matstatea{\Delta_N}{\pframe{\Gamma''}{\Delta}{\bang
   p}{\Omega}{\Delta'}{\Omega'}, \lstack{C} ; \lstack{P}}{\Gamma, p_1, \Gamma''}{\Delta}{\Omega}
   {\Delta' \rightarrow \Omega' \otimes \bang p}{\Sigma}
}
\]


\[
\trans{
   \matstatea{\Delta_N}{\lstack{C}; \lstack{P}}{\Gamma}{\Delta}{\bang p,
      \Omega}{\Delta' \rightarrow \Omega'}{\Sigma}
}
{
   \contstatea{\Delta_N}{\lstack{C} ; \lstack{P}}{\Gamma}{\Sigma}
}
\]



\subsubsection{Deconstruct body}


\begin{multline}
\transx{
   \matstatea{\deltan}{\lstack{C};
      \lstack{P}}{\Gamma}{\Delta}{X \otimes Y, \Omega}{\Delta' \rightarrow
         \Omega'}{\Sigma}
}
{
   \matstatea{\deltan}{\lstack{C};
      \lstack{P}}{\Gamma}{\Delta}{X, Y, \Omega}{\Delta' \rightarrow
         \Omega'}{\Sigma}
} \tag{agg match $\otimes$}
\end{multline}

\begin{multline}
\transx{
   \matstatea{\deltan}{\lstack{C};
      \lstack{P}}{\Gamma}{\Delta}{\one, \Omega}{\Delta' \rightarrow
         \Omega'}{\Sigma}
}
{
   \matstatea{\deltan}{\lstack{C};
      \lstack{P}}{\Gamma}{\Delta}{\Omega}{\Delta' \rightarrow
         \Omega'}{\Sigma}
      } \tag{agg match $\one$}
\end{multline}


\subsubsection{Successful match}

When the aggregate body finally matches, we retrieve the term for variable $x$
(the aggregate variable) and add it to the list $\Sigma$.

\[
\infer[\ma{AG} \m{end}]
{\ma{AG} \Psi; \Gamma; \Delta; \Xi_N; \Gamma_{N1}; \Delta_{N1}; \Xi; \cdot;
   \lstack{C}; \lstack{P}; \Omega_N; \Delta_N; \Sigma \rightarrow \outsem}
{\fixa{AG} \Gamma; \Xi_N; \Gamma_{N1}; \Delta_{N1}; \Xi; \lstack{C}; \lstack{P}; \Omega_N;
   \Delta_N; V :: \Sigma \rightarrow \outsem & x : V : \tau \in \Psi}
\]


\subsubsection{Continuation stack update}

After matching a single aggregate, the stack is updated as if it was a
comprehension: we drop all but the first linear continuation frame and then fix
the contexts of the remaining stack.  The judgment that updates the stack has
the form $\fixa{AG} \Gamma; \Delta; \Xi_N; \Gamma_{N1}; \Delta_{N1}; \Xi; \lstack{C};
\lstack{P}; \Omega_N; \Delta_N; \Sigma \rightarrow \outsem$ and every argument
has the usual meaning.

\subsubsection{Remove linear continuation frames}


{\tiny
\[
\infer[\fixa ~\m{end~linear}]
{\fixa \Gamma; \Xi_N; \Gamma_{N1}; \Delta_{N1}; \Xi; (\Delta_x; \Delta''; \cdot;
      p; \Omega; \cdot; \Upsilon); P;  \aggsz{A}{B}{C}; \Omega_N; \Delta_N; T \rightarrow \Xi'; \Delta'; \Gamma'}
{\begin{split}\strans &\Xi; P; P' \\ \da \Gamma; \Xi_N, \Xi; \Gamma_{N1};
   \Delta_{N1}; B; (\Delta_x - \Xi; \Delta'' - \Xi; \cdot;& p; \Omega; \cdot;
         \Upsilon) ; P' ; \aggsz{A}{B}{C}; \Omega_N; (\Delta_N - \Xi); T &\rightarrow \Xi'; \Delta'; \Gamma'\end{split}}
\]
}

\[
\infer[\fixa \m{more}]
{\fixa \Gamma; \Xi_N; \Gamma_{N1}; \Delta_{N1}; \Xi; \_, X, C; P; AG; \Omega_N; \Delta_N; T \rightarrow \Xi'; \Delta'; \Gamma'}
{\fixa \Gamma; \Xi_N; \Gamma_{N1}; \Delta_{N1}; \Xi; X, C; P; AG; \Omega_N; \Delta_N; T \rightarrow \Xi'; \Delta'; \Gamma'}
\]

{\footnotesize
\[
\infer[\fixa \m{end~empty}]
{\fixa \Gamma; \Xi_N; \Gamma_{N1}; \Delta_{N1}; \Xi; \cdot; P; \aggsz{A}{B}{C}; \Omega_N; \Delta_N; T \rightarrow \Xi'; \Delta'; \Gamma'}
{\begin{split}\strans &\Xi; P; P' \\ \da \Gamma; \Xi_N, \Xi; \Gamma_{N1};
   \Delta_{N1}; B; \cdot ; P' ; &\aggsz{A}{B}{C}; \Omega_N; (\Delta_N - \Xi); T &\rightarrow \Xi'; \Delta'; \Gamma'\end{split}}
\]
}


\subsubsection{Aggregate continuation}

If the aggregate match fails, we need to backtrack. The judgment for
backtracking has the form $\conta{AG} \Gamma; \Delta_N; \Xi_N; \Delta_{N1};
\lstack{C}; \lstack{P}; \Omega_N; \Sigma \rightarrow \outsem$.

\paragraph{Using the $\lstack{C}$ stack}


\begin{multline}
\transx{
   \contstatea{\deltan}{\lframe{\Delta}{p_1, \Delta''}{p}{\Omega}{\Delta'}{\Omega'}, \lstack{C} ; \lstack{P}}{\Gamma}{\Sigma}
}
{
   \matstatea{\deltan}{\lframe{\Delta,
      p_1}{\Delta''}{p}{\Omega}{\Delta'}{\Omega'}, \lstack{C}; \lstack{P}}{\Gamma}{\Delta}{p,
      \Omega}{\Delta', p_1 \rightarrow \Omega' \otimes p}{\Sigma}
} \tag{agg next p $\lstack{C}$}
\end{multline}

\begin{multline}
\transx{
   \contstatea{\deltan}{\pframe{p_1, \Gamma''}{\Delta}{\bang
   p}{\Omega}{\Delta'}{\Omega'}, \lstack{C} ; \lstack{P}}{\Gamma}{\Sigma}
}
{
   \matstatea{\deltan}{\pframe{\Gamma''}{\Delta}{\bang p}
      {\Omega}{\Delta'}{\Omega'}, \lstack{C}; \lstack{P}}{\Gamma}{\Delta}{p,
      \Omega}{\Delta' \rightarrow \Omega' \otimes \bang p}{\Sigma}
} \tag{agg next \bang p $\lstack{C}$}
\end{multline}

\[
\trans{
   \contstatea{\deltan}{\lframe{\Delta}{\cdot}{p}{\Omega}{\Delta'}{\Omega'}, \lstack{C} ; \lstack{P}}{\Gamma}{\Sigma}
}
{
   \contstatea{\deltan}{\lstack{C} ; \lstack{P}}{\Gamma}{\Sigma}
} \tag{agg  next frame $\lstack{C}$}
\]

\[
\trans{
   \contstatea{\deltan}{\pframe{\cdot}{\Delta}{\bang
   p}{\Omega}{\Delta'}{\Omega'}, \lstack{C} ; \lstack{P}}{\Gamma}{\Sigma}
}
{
   \contstatea{\deltan}{\lstack{C} ; \lstack{P}}{\Gamma}{\Sigma}
} \tag{agg next \bang frame $\lstack{C}$}
\]


\paragraph{Using the $\lstack{P}$ stack}

\[
\infer[\conta{AG} \m{next}~\lstack{P}~\bang p]
{\conta{AG} \Gamma; \Delta_N; \Xi_N; \Gamma_{N1}; \Delta_{N1}; \cdot; f, \lstack{P}; \Omega_N; \Sigma \rightarrow \outsem}
{\begin{gathered}
   f = [p_1, \Gamma'; \Delta_N; \cdot; \bang p; \Omega; \cdot; \Upsilon] \\
   f' = [\Gamma'; \Delta_N; \cdot; \bang p; \Omega; \cdot; \Upsilon] \\
   \ma{AG} \Gamma; \Delta_N; \Xi_N; \Gamma_{N1}; \Delta_{N1}; \cdot; \Omega; \cdot;
      f', \lstack{P}; \Omega_N; \Delta_N; \Sigma \rightarrow \outsem
 \end{gathered}
}
\]

\[
\infer[\conta{AG} \m{next}~\lstack{P}~\m{empty}~\bang p]
{\conta{AG} \Gamma; \Delta_N; \Xi_N; \Gamma_{N1}; \Delta_{N1}; \cdot; f, \lstack{P}; \Omega_N; \Sigma
   \rightarrow \outsem}
{\begin{gathered}
   f =  [\cdot; \Delta_N; \cdot; \bang p; \Omega; \cdot; \Upsilon] \\
   \conta{AG} \Gamma; \Delta_N; \Xi_N; \Gamma_{N1}; \Delta_{N1}; \cdot; \lstack{P};
      \Omega_N; \Sigma \rightarrow \outsem
 \end{gathered}
}
\]


\paragraph{Aggregate done}

\[
\infer[\conta{\aggsz{A}{B}{C}} \m{end}]
{\conta{\aggsz{A}{B}{C}} \Gamma; \Delta_N; \Xi_N; \Gamma_{N1}; \Delta_{N1}; \cdot; \cdot;
   \Omega; \Sigma \rightarrow \outsem}
{\done \Gamma; \Delta_N; \Xi_N; \Gamma_{N1}; \Delta_{N1}; (\lambda x. C
      x)\Sigma,
   \Omega \rightarrow \outsem}
\]


\subsubsection{Aggregate Derivation}

\[
\infer[\da{AG} p]
{\da{AG} \Gamma; \Delta_N; \Xi_N; \Gamma_1; \Delta_1; p, \Omega; \lstack{C}; \lstack{P}; \Omega_N;
   \Sigma \rightarrow \outsem}
{\da{AG} \Gamma; \Delta_N; \Xi_N; \Gamma_1; \Delta_1, p; \Omega; \lstack{C}; \lstack{P}; \Omega_N;
   \Sigma \rightarrow \outsem}
\]

\[
\infer[\da{AG} \bang p]
{\da{AG} \Gamma; \Delta_N; \Xi_N; \Gamma_1; \Delta_1; \bang p, \Omega; \lstack{C};
   \lstack{P}; \Omega_N; \Sigma \rightarrow \outsem}
{\da{AG} \Gamma; \Delta_N; \Xi_N; \Gamma_1, p; \Delta_1; \Omega; \lstack{C}; \lstack{P}; \Omega_N;
   \Sigma \rightarrow \outsem}
\]

\[
\infer[\da{AG} \otimes]
{\da{AG} \Gamma; \Delta_N; \Xi_N; \Gamma_1; \Delta_1; A \otimes B, \Omega; \lstack{C}; \lstack{P}; \Omega_N;
   \Sigma \rightarrow \outsem}
{\da{AG} \Gamma; \Delta_N; \Xi_N; \Gamma_1; \Delta_1; A, B, \Omega; \lstack{C}; \lstack{P}; \Omega_N;
   \Sigma \rightarrow \outsem}
\]

\[
\infer[\da{AG} \m{end}]
{\da{AG} \Gamma; \Delta_N; \Xi_N; \Gamma_1; \Delta_1; \cdot; \lstack{C}; \lstack{P}; \Omega_N;
   \Sigma \rightarrow \outsem}
{\conta{AG} \Gamma; \Delta_N; \Xi_N; \Gamma_1; \Delta_1; \lstack{C}; \lstack{P}; \Omega_N; \Sigma
   \rightarrow \outsem}
\]



This completes the specification of LLD.
