The most intricate part of the derivation process is processing comprehensions
and aggregates. For both of them, we need to perform as many derivations as the
database allows, therefore we need to deterministically check the contents of
the database until no more derivations are possible.  The matching process is
then similar to the process used for matching the body of the rule presented in
Section~\ref{sec:lld_body_match}, however we use two continuation stacks,
$\lstack{C}$ and $\lstack{P}$. In $\lstack{P}$, we put all the initial
persistent frames and in $\lstack{C}$ we put the first linear frame and then
everything else.

In order to reuse the stacks $\lstack{C}$ and $\lstack{P}$, we need to update
them by removing all the frames in $\lstack{C}$ pushed after the first linear
continuation frame.  If we tried to use those frames, we would assumed that the
linear facts used by the other frames were still in the database, but that is
not true because they have been consumed during the first application of the
comprehension.  For example, if the body is $\bang \mathtt{a(X)} \otimes
\mathtt{b(X)} \otimes \mathtt{c(X)}$ and the continuation stack has three frames
(one per fact), we cannot backtrack to the frame of $\mathtt{c(X)}$ because, at
that point, the matching process was assuming that the previous \texttt{b(X)}
linear fact was still available.  Moreover, we also need to remove the consumed
linear facts from the frames of \texttt{b(X)} and $\bang$\texttt{a(X)} in order
to make the stack fully consistent with the new database. We will see later on
how to do that.

Each aggregate derivation also needs to accumulate a list of values for each
combination. Once all combinations are performed, then the main head of the
aggregate is derived using the combined value.

The matching state for aggregates is 
$\matstatea{\Delta_N}{\lstack{C};
   \lstack{P}}{\Gamma}{\Delta}{\Omega}{\Delta' \rightarrow \Omega'}{\Sigma}$

\begin{enumerate}
   \item[$\Omega_N$] ordered list of remaining terms of the head of the rule to
   be derived;

   \item[$\Delta_N$] multi-set of linear facts that were still available after
   matching the body of the rule and all the previous aggregates. Note that
   $\Delta, \Xi = \Delta_N$;

   \item[$\Xi$] multi-set of linear facts used during the matching process of
   the body of the rule and all the previous aggregates;

   \item[$\Gamma_{1}$] set of persistent facts derived up to this point in the
   head of the rule;

   \item[$\Delta_{1}$] multi-set of linear facts derived up to this point in
   the head of the rule;

   \item[$\Delta'$] multi-set of linear facts consumed up to this point;

   \item[$\Omega'$] terms matched using $\Delta'$ up to this point;

   \item[$\m{agg}$] aggregate that is being matched;

   \item[$\Sigma$] the list of aggregated values;

   \item[$\lstack{C}$] continuation stack that contains both linear and persistent
   frames. The first frame must be linear;

   \item[$\lstack{P}$] initial part of the continuation stack with only persistent
   frames;

   \item[$\Delta$] multi-set of linear facts remaining up to this point in the
   matching process;

   \item[$\Omega$] ordered list of terms that need to be matched for the
   comprehension to be applied.

\end{enumerate}

Since aggregates accumulate values (from specific variables), we extend the
$\Psi$ context to include triplets $x : M : \tau$ (variable, term and type)
instead of pairs $M : \tau$ in order to be able to retrieve bound variables.
Remember that $\Psi$ is used for the quantification connectives in the sequent
calculus.

\subsubsection{Linear fact expressions}

The following two transitions deal with the case when there is a linear
fact expression in the body of the aggregate.

\[
\infer[\ma{AG} p~\m{first}]
{\ma{AG} \Gamma; \Delta, p_1, \Delta''; \Xi_N; \Gamma_{N1}; \Delta_{N1}; \cdot; p,
   \Omega; \cdot; \cdot; \Omega_N; \Delta_N; \Sigma \rightarrow \outsem}
{
   \begin{gathered}
      p_1, \Delta'' \prec p \\
      f = (\Delta, p_1; \Delta''; \cdot; p; \Omega;
            \cdot; \cdot) \\
      \ma{AG} \Gamma; \Delta, \Delta''; \Xi_N; \Gamma_{N1};
         \Delta_{N1}; \Xi, p_1; \Omega; f; \cdot; \Omega_N; \Delta_N; \Sigma \rightarrow \outsem
   \end{gathered}
}
\]

\[
\infer[\ma{AG} p~\m{on}~q]
{\ma{AG} \Gamma; \Delta, p_1, \Delta''; \Xi_N; \Gamma_{N1}; \Delta_{N1}; \Xi; p,
   \Omega; C_1, \lstack{C}; \lstack{P}; \Omega_N; \Delta_N; \Sigma \rightarrow \outsem}
{
   \begin{gathered}
      p_1, \Delta'' \prec p \\
      f = (\Delta_{old}; \Delta'_{old}; \Xi_{old}; q; \Omega_{old}; \Lambda; \Upsilon) \\
      f' =  (\Delta, p_1; \Delta''; \Xi; p; \Omega; q, \Lambda; \Upsilon) \\
      \ma{AG} \Gamma; \Delta, \Delta''; \Xi_N; \Gamma_{N1};
         \Delta_{N1}; \Xi, p_1; \Omega; f', f, \lstack{C}; \lstack{P}; \Omega_N;
         \Delta_N; \Sigma \rightarrow \outsem
   \end{gathered}
}
\]

\[
\infer[\ma{AG} p~\m{on}~\bang q~\lstack{C}]
{\ma{AG} \Gamma; \Delta, p_1, \Delta''; \Xi_N; \Gamma_{N1}; \Delta_{N1}; \Xi; p,
   \Omega; C_1, \lstack{C}; \lstack{P}; \Omega_N; \Delta_N; \Sigma \rightarrow \outsem}
{
   \begin{gathered}
      p_1, \Delta'' \prec p \\
      f = [\Gamma_{old}; \Delta_{old}; \Xi_{old}; q;
         \Omega_{old}; \Lambda; \Upsilon]\\
      f' = (\Delta, p_1; \Delta''; \Xi; p; \Omega; \Lambda; q, \Upsilon) \\
      \ma{AG} \Gamma; \Delta, \Delta''; \Xi_N; \Gamma_{N1};
         \Delta_{N1}; \Xi, p_1; \Omega;
         f', f, \lstack{C}; \lstack{P}; \Omega_N; \Delta_N;
         \Sigma \rightarrow \outsem
   \end{gathered}
}
\]
\[
\infer[\ma{AG} p~\m{on}~\bang q~\lstack{P}]
{\ma{AG} \Gamma; \Delta, p_1, \Delta''; \Xi_N; \Gamma_{N1}; \Delta_{N1}; \cdot; p,
   \Omega; \cdot; f, \lstack{P}; \Omega_N; \Delta_N; \Sigma \rightarrow \outsem}
{
   \begin{gathered}
      p_1, \Delta'' \prec p \\
      f = [\Gamma_{old}; \Delta_N; \cdot; q; \Omega_{old}; \cdot; \Upsilon]\\
      f' = (\Delta, p_1; \Delta''; \cdot; p; \Omega; \cdot; q, \Upsilon) \\
      \ma{AG} \Gamma; \Delta, \Delta''; \Xi_N;
            \Gamma_{N1}; \Delta_{N1}; p_1; \Omega; f'; f, \lstack{P}; \Omega_N;
            \Delta_N; \Sigma \rightarrow \outsem
   \end{gathered}
}
\]

\[
\infer[\ma{AG} p~\m{fail}]
{\ma{AG} \Gamma; \Delta; \Xi_N; \Gamma_{N1}; \Delta_{N1}; \Xi; p, \Omega;
   \lstack{C}; \lstack{P}; \Omega_N; \Delta_N; \Sigma \rightarrow \outsem}
{\conta{AG} \Gamma; \Delta_N; \Xi_N; \Gamma_{N1}; \Delta_{N1}; \lstack{C};
   \lstack{P}; \Omega_N;
   \Sigma \rightarrow \outsem}
\]



\subsubsection{Persistent fact expressions}

The transitions for dealing with persistent facts are similar to the previous
ones.


\begin{multline}
\transx{
   \matstatea{\Delta_N}{\cdot;
      \lstack{P}}{\Gamma, p_1, \Gamma''}{\Delta}{\bang p, \Omega}{\Delta' \rightarrow
         \Omega'}{\Sigma}
}
{
   \matstatea{\Delta_N}{\cdot; \pframe{\Gamma''}{\Delta}{\bang
   p}{\Omega}{\Delta'}{\Omega'}, \lstack{P}}{\Gamma, p_1, \Gamma''}{\Delta}{\Omega}
   {\Delta' \rightarrow \Omega' \otimes \bang p}{\Sigma}
} \tag{agg match \bang p ok $\lstack{P}$}
\end{multline}

\begin{multline}
\transx{
   \matstatea{\Delta_N}{\lstack{C};
      \lstack{P}}{\Gamma, p_1, \Gamma''}{\Delta}{\bang p, \Omega}{\Delta' \rightarrow
         \Omega'}{\Sigma}
}
{
   \matstatea{\Delta_N}{\pframe{\Gamma''}{\Delta}{\bang
   p}{\Omega}{\Delta'}{\Omega'}, \lstack{C} ; \lstack{P}}{\Gamma, p_1, \Gamma''}{\Delta}{\Omega}
   {\Delta' \rightarrow \Omega' \otimes \bang p}{\Sigma}
} \tag{agg match \bang p ok $\lstack{C}$}
\end{multline}

\[
\trans{
   \matstatea{\Delta_N}{\lstack{C}; \lstack{P}}{\Gamma}{\Delta}{\bang p,
      \Omega}{\Delta' \rightarrow \Omega'}{\Sigma}
}
{
   \contstatea{\Delta_N}{\lstack{C} ; \lstack{P}}{\Gamma}{\Sigma}
} \tag{agg match \bang p fail}
\]



\subsubsection{Deconstruct body}


\begin{multline}
\transx{
   \matstatea{\Delta_N}{\lstack{C};
      \lstack{P}}{\Gamma}{\Delta}{X \otimes Y, \Omega}{\Delta' \rightarrow
         \Omega'}{\Sigma}
}
{
   \matstatea{\Delta_N}{\lstack{C};
      \lstack{P}}{\Gamma}{\Delta}{X, Y, \Omega}{\Delta' \rightarrow
         \Omega'}{\Sigma}
} \tag{agg match $\otimes$}
\end{multline}

\begin{multline}
\transx{
   \matstatea{\Delta_N}{\lstack{C};
      \lstack{P}}{\Gamma}{\Delta}{\one, \Omega}{\Delta' \rightarrow
         \Omega'}{\Sigma}
}
{
   \matstatea{\Delta_N}{\lstack{C};
      \lstack{P}}{\Gamma}{\Delta}{\Omega}{\Delta' \rightarrow
         \Omega'}{\Sigma}
      } \tag{agg match $\one$}
\end{multline}



\subsubsection{Successful match}

When the aggregate body finally matches, we retrieve the term for variable $x$
(the aggregate variable) and add it to the list $\Sigma$.

\[
\trans{
   \matstatea{\Delta_N}{\lstack{C};
      \lstack{P}}{\Gamma}{\Delta}{\cdot}{\Delta' \rightarrow
         \Omega'}{\Sigma}
}
{
   \fixstatea{\Delta}{\Xi; \Delta'}{\lstack{C}; \lstack{P}}{\Gamma}{\Sigma}
}
\]


\subsubsection{Continuation stack update}

As we said before, to update the continuation stacks, we need remove to all the
frames except the first linear frame and remove the consumed linear facts from
the remaining frames so that they are still valid for the next application of
the aggregate.  The judgment that updates the stack has the form
$\fixstatea{\Delta}{\Xi; \Delta'}{\lstack{C};
   \lstack{P}}{\Gamma}{\Sigma}$, where:

\begin{enumerate}
   \item[$\Omega_N$] ordered list of remaining terms of the head of the rule to
   be derived;
   \item[$\Delta$] multi-set of linear facts that were still available after
   matching the body of the rule and the body of the aggregate;
   \item[$\Xi$] multi-set of linear facts used during the matching process of
   the body of the rule and all the previous aggregates;
   \item[$\Delta'$] multi-set of linear facts consumed by the aggregate body;
   \item[$\Gamma_{1}$] set of persistent facts derived by the head of the rule
   and all the previous aggregates;
   \item[$\Delta_{1}$] multi-set of linear facts derived by the head of the
   rule and all the previous aggregates;
   \item[$\m{agg}$] the current aggregate;
   \item[$\Sigma$] list of accumulated values;
   \item[$\lstack{C}, \lstack{P}$] continuation stacks for the comprehension;
   \item[$\Gamma$] set of usable persistent facts.
\end{enumerate}

\subsubsection{Remove linear continuation frames}

To remove all linear continuation frames except the first one, we simply go
through all the frames in the stack $\lstack{C}$ until only one frame remains.
This last frame and stack $\lstack{P}$ are then updated by removing $\Delta'$
from its contexts.

\[
\trans{
   \fixstatea{\Delta}{\Xi; \Delta'}{\_, f, \lstack{C}; \lstack{P}}{\Gamma}{\Sigma}
}
{
   \fixstatea{\Delta}{\Xi; \Delta'}{f, \lstack{C}; \lstack{P}}{\Gamma}{\Sigma}
} \tag{agg fix rec}
\]

\[
\underset{
   \begin{gathered}
   \Pi(\m{agg}) = \forall_{\widehat{v}, \Sigma'}.
   (\defstwo{agg}{\widehat{v}}{\Sigma'} \lolli ((\lambda x. C x)\Sigma' \with (\forall_{\widehat{x}, \sigma}.
                                                (A \lolli B \otimes
                                                 \defstwo{agg}{\widehat{v}}{\sigma
                                                 ::\Sigma'})))) \\
                                                 f' = \texttt{remove}(f, \Delta') \\
                                                 \lstack{P'} = \texttt{remove}(\lstack{P}, \Delta') \\
   V = \Psi(\sigma)
   \end{gathered}
}
{
   \transx{
      \fixstatea{\Delta}{\Xi; \Delta'}{f; \lstack{P}}{\Gamma}{\Sigma}
   }
   {
      \derstatea{\Delta}{\Xi; \Delta'}{\gammanew}{\deltanew}{V :: \Sigma}{f';
         \lstack{P'}}{B\{\Psi(\widehat{x}), V / \widehat{x}, \sigma \}}
   }
}
   \tag{agg fix end1}
\]

\[
\underset{
   \begin{gathered}
   \Pi(\m{agg}) = \forall_{\widehat{v}, \Sigma'}.
   (\defstwo{agg}{\widehat{v}}{\Sigma'} \lolli ((\lambda x. C x)\Sigma' \with (\forall_{\widehat{x}, \sigma}.
                                                (A \lolli B \otimes
                                                 \defstwo{agg}{\widehat{v}}{\sigma
                                                 ::\Sigma'})))) \\
                                                 \lstack{P'} = \texttt{remove}(\lstack{P}, \Delta') \\
   V = \Psi(\sigma)
   \end{gathered}
}
{
   \trans{
      \fixstatea{\Delta}{\Xi; \Delta'}{\cdot; \lstack{P}}{\Gamma}{\Sigma}
   }
   {
      \derstatea{\Delta}{\Xi, \Delta'}{\Gamma_{N1}}{\Delta_{N1}}{V :: \Sigma}{\cdot;
         \lstack{P}'}{B\{\Psi(\widehat{x}), V / \widehat{x}, \sigma \}}
   }
} \tag{agg fix end2}
\]


\subsubsection{Aggregate backtracking}

If the aggregate match fails, we need to backtrack to the next candidate fact.
The backtracking state 
has the form
$\contstatea{\Delta_N}{\lstack{C} ; \lstack{P}}{\Gamma}{\Sigma}$, where:

\begin{enumerate}
   \item[$\Omega_N$] ordered list of remaining terms of the head of the rule to
   be derived;
   \item[$\Delta_N$] multi-set of linear facts that were still available after
   matching the body of the rule and the body of the aggregate;
   \item[$\Xi$] multi-set of linear facts used during the matching process of
   the body of the rule and all the previous aggregates;
   \item[$\Gamma_{1}$] set of persistent facts derived by the head of the rule
   and all the previous aggregates;
   \item[$\Delta_{1}$] multi-set of linear facts derived by the head of the
   rule and all the previous aggregates;
   \item[$\m{agg}$] the current aggregate;
   \item[$\Sigma$] list of accumulated values.
   \item[$\lstack{C}, \lstack{P}$] continuation stacks for the comprehension;
   \item[$\Gamma$] set of usable persistent facts.
\end{enumerate}

\paragraph{Using the $\lstack{C}$ stack}

The following 4 state transitions use the $\lstack{C}$ stack, the stack where the
first continuation frame is linear, to perform backtracking.

{\footnotesize
\[
\infer[\conta \m{next}~C~p]
{\conta \Gamma; \Delta_N; \Xi_N; \Gamma_{N1}; \Delta_{N1}; (\Delta; p_1, \Delta''; \Xi; p; \Omega; \Lambda; \Upsilon), C; P; AG; \Omega_N; T \rightarrow \Xi'; \Delta'; \Gamma'}
{\ma \Gamma; \Delta; \Xi_N; \Gamma_{N1}; \Delta_{N1}; \Xi; \Omega; (\Delta, p_1; \Delta''; \Xi; p; \Omega; \Lambda; \Upsilon), C; P; AG; \Omega_N; \Delta_N; T \rightarrow \Xi'; \Delta'; \Gamma'}
\]

\[
\infer[\conta \m{next}~C~\bang p]
{\conta \Gamma; \Delta_N; \Xi_N; \Gamma_{N1}; \Delta_{N1}; [p_1, \Gamma'; \Delta; \Xi; \bang p; \Omega; \Lambda; \Upsilon], C; P; AG; \Omega_N; T \rightarrow \Xi'; \Delta'; \Gamma'}
{\ma \Gamma; \Delta; \Xi_N; \Gamma_{N1}; \Delta_{N1}; \Xi; \Omega; [\Gamma'; \Delta; \Xi; \bang p; \Omega; \Lambda; \Upsilon], C; P; AG; \Omega_N; \Delta_N; T \rightarrow \Xi'; \Delta'; \Gamma'}
\]

\[
\infer[\conta \m{next}~C~\m{empty}~p]
{\conta \Gamma; \Delta_N; \Xi_N; \Gamma_{N1}; \Delta_{N1}; (\Delta; \cdot; \Xi; p; \Omega; \Lambda; \Upsilon), C; P; AG; \Omega_N; T \rightarrow \Xi'; \Delta'; \Gamma'}
{\conta \Gamma; \Delta_N; \Xi_N; \Gamma_{N1}; \Delta_{N1}; C; P; AG; \Omega_N; T \rightarrow \Xi'; \Delta'; \Gamma'}
\]

\[
\infer[\conta \m{next}~C~\m{empty}~\bang p]
{\conta \Gamma; \Delta_N; \Xi_N; \Gamma_{N1}; \Delta_{N1}; [\cdot; \Delta; \Xi; \bang p; \Omega; \Lambda; \Upsilon], C; P; AG; \Omega_N; T \rightarrow \Xi'; \Delta'; \Gamma'}
{\conta \Gamma; \Delta_N; \Xi_N; \Gamma_{N1}; \Delta_{N1}; C; P; AG; \Omega_N; T \rightarrow \Xi'; \Delta'; \Gamma'}
\]
}



\paragraph{Using the $\lstack{P}$ stack}

The following 2 state transitions rules use the $\lstack{P}$ stack instead, the stack where all
continuation frames are persistent.


\begin{multline}
\transx{
   \contstatea{\deltan}{\cdot ; \pframe{p_1, \Gamma''}{\Delta}{\bang
   p}{\Omega}{\Delta'}{\Omega'}, \lstack{P}}{\Gamma}{\Sigma}
}
{
   \matstatea{\deltan}{\cdot; \pframe{\Gamma''}{\Delta}{\bang p}
      {\Omega}{\Delta'}{\Omega'}, \lstack{P}}{\Gamma}{\Delta}{p,
      \Omega}{\Delta' \rightarrow \Omega' \otimes \bang p}{\Sigma}
} \tag{agg next \bang p $\lstack{P}$}
\end{multline}

\[
\trans{
   \contstatea{\deltan}{\cdot; \pframe{\cdot}{\Delta}{\bang
   p}{\Omega}{\Delta'}{\Omega'}, \lstack{P}}{\Gamma}{\Sigma}
}
{
   \contstatea{\deltan}{\cdot ; \lstack{P}}{\Gamma}{\Sigma}
} \tag{agg next \bang frame $\lstack{P}$}
\]


\paragraph{Aggregate done}

If both the $\lstack{C}$ and $\lstack{P}$ stacks are empty, backtracking is
impossible and the aggregate is done. The final head of the aggregate is then
derived along with the rest of the rule's head.


\[
\underset{
   \Pi(\m{agg}) = \forall_{\widehat{v}, \Sigma'}.
   (\defstwo{agg}{\widehat{v}}{\Sigma'} \lolli ((\lambda x. C x)\Sigma' \with (\forall_{\widehat{x}, \sigma}.
                                                (A \lolli B \otimes
                                                 \defstwo{agg}{\widehat{v}}{\sigma
                                                 ::\Sigma'}))))
}
{
\trans{
   \contstatea{\Delta_N}{\cdot ; \cdot}{\Gamma}{\Sigma}
}
{
   \derstatex{\Gamma}{\Delta_N}{\Xi}{\Gamma_{N1}}{\Delta_{N1}}{(\lambda x.
         C\{\Psi(\widehat{v})/\widehat{v}\} x) \Sigma,
      \Omega_N}
}
}
\]


\subsubsection{Aggregate Derivation}

After updating the continuation stacks, the subhead of the aggregate is derived.
The derivation state has the form
$\derstatea{\Delta}{\Xi}{\Gamma_1}{\Delta_1}{\Sigma}{\lstack{C};
   \lstack{P}}{\Omega}$, where:

\begin{enumerate}
   \item[$\Omega_N$] ordered list of remaining terms of the head of the rule to
   be derived;
   \item[$\Delta$] multi-set of remaining linear facts that can be used for
   the next aggregate applications.
   \item[$\Xi$] multi-set of linear facts consumed both by the body of the rule
   and previous aggregate applications;
   \item[$\Gamma_1$] set of persistent facts derived by the head of the rule,
   previous aggregates and current derivation;
   \item[$\Delta_1$] multi-set of linear facts derived by the head of the rule,
   previous aggregates and current derivation;
   \item[$\m{agg}$] current aggregate symbol;
   \item[$\Sigma$] accumulated list of values of the aggregate;
   \item[$\lstack{C}, \lstack{P}$] new continuation stacks;
   \item[$\Gamma$] set of persistent facts;
   \item[$\Omega$] ordered list of terms to derive.
\end{enumerate}


\[
\trans{
   \derstatea{\Delta}{\Xi}{\gammanew}{\deltanew}{\Sigma}{\lstack{C};
      \lstack{P}}{p, \Omega}
}
{
   \derstatea{\Delta}{\Xi}{\gammanew}{\deltanew, p}{\Sigma}{\lstack{C};
      \lstack{P}}{\Omega}
} \tag{agg new p}
\]

\[
\trans{
   \derstatea{\Delta}{\Xi}{\gammanew}{\deltanew}{\Sigma}{\lstack{C};
      \lstack{P}}{\bang p, \Omega}
}
{
   \derstatea{\Delta}{\Xi}{\gammanew, p}{\deltanew}{\Sigma}{\lstack{C};
      \lstack{P}}{\Omega}
} \tag{agg new \bang p}
\]

\[
\trans{
   \derstatea{\Delta}{\Xi}{\gammanew}{\deltanew}{\Sigma}{\lstack{C};
      \lstack{P}}{X \otimes Y, \Omega}
}
{
   \derstatea{\Delta}{\Xi}{\gammanew, p}{\deltanew}{\Sigma}{\lstack{C};
      \lstack{P}}{X, Y, \Omega}
} \tag{agg new $\otimes$}
\]

\[
\trans{
   \derstatea{\Delta}{\Xi}{\gammanew}{\deltanew}{\Sigma}{\lstack{C};
      \lstack{P}}{\one, \Omega}
}
{
   \derstatea{\Delta}{\Xi}{\gammanew, p}{\deltanew}{\Sigma}{\lstack{C};
      \lstack{P}}{\Omega}
} \tag{agg new $\one$}
\]

\[
\trans{
   \derstatea{\Delta}{\Xi}{\gammanew}{\deltanew}{\Sigma}{\lstack{C};
      \lstack{P}}{\cdot}
}
{
   \contstatea{\Delta}{\lstack{C} ; \lstack{P}}{\Gamma}{\Sigma}
} \tag{agg next}
\]


This completes the specification of LLD.
