
Linear Meld (LM) is a \emph{forward chaining} logic programming language in the
style of Datalog~\cite{Ullman:1990:PDK:533142}. The program is defined as a
\emph{database of facts} and a set of \emph{derivation rules}.  Initially, we
populate the database with the program's initial facts and then determine which
derivation rules can be applied by using the current database. Once a rule is
applied, we derive new facts, which are then added to the database.  If a rule
uses linear facts, they are consumed and thus deleted from the database.  The
program stops when we reach \emph{quiescence}, that is, when we can no longer
able to obtain new facts through rule derivation.

LM departs from Datalog by making the database of facts a graph data structure
where each node contains a fraction of the database. Derivation rules are then
restricted so that they are only allowed to manipulate facts belonging to the
same node. This restriction allows multiple nodes of the graph to perform
independent rule derivations at the node level, thus introducing concurrency in
the program. Communication between nodes arises when a derivation rule derives a
fact that is in another node.

\section{A Taste Of LM}

In order to understand how LM programs are written, we now present and discuss
three basic LM programs.

\subsection{First Example: Message Routing}

Figure~\ref{code:language:message} shows the first LM program, a message routing
program that simulates message transmission through a network of nodes.
Lines~\ref{line:language:message_pred1}-\ref{line:language:message_pred2}
declare the predicates used in the program's rules. Note that the first argument
of every predicate must be typed as \code{node} because the first argument
indicates where the fact lives in the graph. Predicate \code{edge} is a
\emph{persistent predicate} while \code{message} and \code{processed} are
\emph{linear predicates}. Persistent predicates model facts that are never
retracted from the database, while linear predicates model linear facts which
are retracted when used in rules. To improve readability of LM rules, persistent
predicates are preceded with a \code{!} symbol. Predicate \code{edge} represents
the connections between nodes, predicate \code{message} contains the message
content and the route list, and predicate \code{processed} keeps count of the
number of messages routed at each node. Along with the type, a predicate
argument can also be named (e.g., see \code{Neighbor} and \code{Content}) for
documentation purposes.

\begin{figure}[h!]
\begin{LineCode}[commandchars=\*\{\}]
type edge(node, node Neighbor).*label{line:language:message_pred1}*hfill// Predicate declaration
type linear message(node, string Content, list node Routing).
type linear processed(node, int Total).*label{line:language:message_pred2}

message(A, Content, [B | L]),*label{line:language:message_first1}*hfill// Rule 1
!edge(A, B),
processed(A, N)
   -o processed(A, N + 1),
      message(B, Content, L).*label{line:language:message_first2}

message(A, Content, []),*label{line:language:message_second1}*hfill// Rule 2
processed(A, N)
   -o processed(A, N + 1).*label{line:language:message_second2}
\end{LineCode}
\mycap{LM code for routing messages in a graph.}
\label{code:language:message}
\end{figure}

The message routing program in Fig.~\ref{code:language:message} implements two
rules. An LM rule has the form $L_1, \cdots, L_n \mathtt{-o} \; R_1, \cdots,
R_m$, where $L_1, \cdots, L_n$ is the \emph{left-hand side}~(LHS) of the rule
and $R_1, \cdots, R_m$ is \emph{right-hand side}~(RHS) of the rule. The meaning
of a rule is then as follows: if facts $L_1, \cdots, L_n$ are present in the
database then consume all the facts $L_i$ that are linear facts and derive the
facts $R_1, \cdots, R_m$ from the RHS. Note that in the rules, the LHS of each
rule uses only facts from the same node (represented by \code{A} in both rules
of Fig.~\ref{code:language:message}), but the rule's RHS may derive facts that
belong to other nodes (case of \code{B} in the first rule) if the variable is
instantiated in the LHS.

The first rule
(lines~\ref{line:language:message_first1}-\ref{line:language:message_first2})
grabs the head node \code{B} in the route list (third argument of \code{message}
using the syntax \code{[B | L]} to represent the head and tail of a list) and
ensures that a communication edge exists (through \code{edge(A,~B)}). If so,
the number of processed messages is increased by consuming \code{processed(A,~N)}
and deriving \code{processed(A,~N~+~1)}, along with a new message to
the head node \code{B}.  When the route list is empty, the message has reached
its destination and thus it is simply consumed (second rule in lines
\ref{line:language:message_second1}-\ref{line:language:message_second2}).

The initial facts of the program are presented in
Fig.~\ref{code:language:message_facts}. The \code{edge} facts describe the
structure of the graph, where the node in the first argument has a direct
connection with the node in the second argument. Node literals are represented
using the syntax \code{@N}, where \code{N} is an non-negative number. We also
declare the message that needs to be routed with the content \code{"hello
world"} and a route list with the nodes \code{@3} and \code{@4}.

\begin{figure}[h!]
\begin{LineCode}[commandchars=\*\{\}]
!edge(@1, @2). !edge(@2, @3).
!edge(@3, @4). !edge(@1, @3).
processed(@1, 0). processed(@2, 0).
processed(@3, 0). processed(@4, 0).
message(@1, "hello world", [@3, @4]).
\end{LineCode}
\mycap{Initial facts for the message routing program. There is only one message ("hello
world") to route through nodes \code{@3} and \code{@4}.}
\label{code:language:message_facts}
\end{figure}

In Fig.~\ref{fig:message_trace} we present an execution trace of the message
routing program. The database is represented as a graph structure where the
edges represent the \code{edge} initial facts. To simplify the figure, we
dropped the first argument of each fact since the first argument corresponds to
the node where the fact is placed. In Fig.~\ref{fig:message_trace}(a) the
database is initialized with the program's initial facts. Note that the initial
\code{message} fact is instantiated at node \code{@1}. After applying rule 1, we
get the database represented in Fig.~\ref{fig:message_trace}(b), where the
message has been derived at node \code{@3}. After applying rule 1 again, the
message is then routed to node \code{@4} (Fig.~\ref{fig:message_trace}(c))
where it will be consumed (Fig.~\ref{fig:message_trace}(d)).

\newcommand{\visitsize}{0.38}

\begin{figure}[h]
        \centering
        \begin{subfigure}[b]{\visitsize\textwidth}
                \includegraphics[width=\textwidth]{figures/message/message_trace1}
                \mycap{Initial database.}
                \label{fig:message_trace1}
        \end{subfigure}%
        ~
        \begin{subfigure}[b]{\visitsize\textwidth}
                \includegraphics[width=\textwidth]{figures/message/message_trace2}
                \mycap{After applying rule 1 at node \code{@1}.}
                \label{fig:message_trace2}
        \end{subfigure}\\
        \begin{subfigure}[b]{\visitsize\textwidth}
                \includegraphics[width=\textwidth]{figures/message/message_trace3}
                \mycap{After applying rule 1 at node \code{@3}.}
                \label{fig:message_trace3}
        \end{subfigure}%
        ~
        \begin{subfigure}[b]{\visitsize\textwidth}
                  \includegraphics[width=\textwidth]{figures/message/message_trace4}
                  \mycap{After applying rule 2 at node \code{@4}.}
                  \label{fig:message_trace4}
          \end{subfigure}
        \mycap{An execution trace for the message routing program. The message "hello
        world" travels from node \code{@1} to node \code{@4}.}\label{fig:message_trace}
\end{figure}

The attentive reader will wonder how much concurrency is available in this
particular routing implementation. For a single message, there is no concurrency
because the message follows a pre-determined path as it travels from node to
node. However, concurrency arises when the program needs to route multiple
messages. The amount of concurrency is then dependent on the messages routing
lists being non-overlapping. If messages travel in different paths, then there
is more concurrency because more nodes are routing messages at the same time.


\subsection{Second Example: Key/Value Dictionary}\label{sec:language:key_value}

\begin{figure}[ht]
\begin{Verbatim}[numbers=left,fontsize=\codesize,commandchars=\*\{\}]
type left(node, node Child).*label{line:language:dict_header1}*hfill // Predicate declaration
type right(node, node Child).
type linear value(node, int Key, string Value).
type linear replace(node, int Key, string Value).*label{line:language:dict_header2}

replace(A, K, New),*label{line:language:dict_first1}*hfill// Rule 1: we found our key
value(A, K, Old)
   -o value(A, K, New).*label{line:language:dict_first2}

replace(A, RKey, RValue),*label{line:language:dict_second1}*hfill// Rule 2: go left
value(A, Key, Value),
RKey < Key,
!left(A, B)
   -o value(A, Key, Value),
      replace(B, RKey, RValue).*label{line:language:dict_second2}

replace(A, RKey, RValue),*label{line:language:dict_third1}*hfill// Rule 3: go right
value(A, Key, Value),
RKey > Key,
!right(A, B)
   -o value(A, Key, Value),
      replace(B, RKey, RValue). *label{line:language:dict_third2}
\end{Verbatim}
\mycap{LM program for replacing a key's value in a BST dictionary.}
\label{code:language:btree_replace}
\end{figure}

Our second example, shown in Fig.~\ref{code:language:btree_replace}, implements
the key update operation for a binary search tree~(BST) represented as a
key/value dictionary. Each LM node represents a binary tree node. We first
declare all the predicates in
lines~\ref{line:language:dict_header1}-\ref{line:language:dict_header2}.
Predicates \code{left} and \code{right} represent the child nodes of each BST
node. Linear predicate \code{value} stores the key/value pair of a node, while
the linear predicate \code{replace} represents an update operation where the key
in the second argument is to be updated to the value in the third argument.

The algorithm uses three rules for the three possible cases of updating a key's
value. The first rule
(lines~\ref{line:language:dict_first1}-\ref{line:language:dict_first2}) performs
the update by removing \code{replace(A, K, New)} and \code{value(A, K, Old)} and
deriving \code{value(A, K, New)}. The second rule
(lines~\ref{line:language:dict_second1}-\ref{line:language:dict_second2})
recursively picks the left branch for the update operation by deleting
\code{replace(A, RKey, RValue)} and re-deriving it at node \code{B}. Similarly,
third rule
(lines~\ref{line:language:dict_third1}-\ref{line:language:dict_third2})
recursively descends the right branch. The derivation of \code{replace} facts on
node \code{B} can also be seen as a implicit \emph{message passing} from
\code{A} to \code{B}, since \code{B} is a different node than \code{A} and the
LHS of each rule can only manipulate facts from the same node.

The initial facts of the program are presented in
Fig.~\ref{code:language:btree_replace_initial} and describe the initial binary
tree configuration, including keys and values, and the \code{replace(@1, 6,
"x")} fact instantiated at the root node \code{@1} that manifests the intent to
change the value of key 6 to "x".

\begin{figure}[ht]
\begin{Verbatim}[numbers=left,fontsize=\codesize,commandchars=\*\{\}]
!left(@1, @2).
!right(@1, @3).
!left(@2, @4).
!right(@2, @5). 
!left(@3, @6).
!right(@3, @7).

value(@1, 3, "a").
value(@2, 1, "b").
value(@3, 5, "c").
value(@4, 0, "d").
value(@5, 2, "e").
value(@6, 4, "f").
value(@7, 6, "g").

// Update key 6 to value "x".
replace(@1, 6, "x").
\end{Verbatim}
\mycap{Initial facts for replacing a key's value in a BST dictionary.}
\label{code:language:btree_replace_initial}
\end{figure}

Figure~\ref{fig:language:btree_trace} represents the trace of the algorithm. The program
database is partitioned by the seven nodes using the first argument of each
fact. In Fig.~\ref{fig:language:btree_trace}~(a), we present the database filled with the
program's initial facts. Next, we follow the right branch using rule 3 since $6
> 3$ (Fig.~\ref{fig:language:btree_trace}~(b)).  We use the same rule again in
Fig.~\ref{fig:language:btree_trace}~(c) where we finally reach the key 6. Here, we apply
rule 1 and \code{value(@7, 6, "g")} is updated to \code{value(@7, 6, "x")}.

\begin{figure}[h]
        \centering
        \begin{subfigure}[b]{0.5\textwidth}
                \includegraphics[width=\textwidth]{figures/btree/btree_trace1}
                \mycap{Initial database.}
                \label{fig:language:btree_trace1}
        \end{subfigure}%
        ~
        \begin{subfigure}[b]{0.5\textwidth}
                \includegraphics[width=\textwidth]{figures/btree/btree_trace2}

                \mycap{After applying rule 3 at node \code{@1}.}

                \label{fig:language:btree_trace2}
        \end{subfigure}\\
        \begin{subfigure}[b]{0.5\textwidth}
                \includegraphics[width=\textwidth]{figures/btree/btree_trace3}
                \mycap{After applying rule 3 at node \code{@3}.}
                \label{fig:language:btree_trace3}
        \end{subfigure}%
        ~
        \begin{subfigure}[b]{0.5\textwidth}
                  \includegraphics[width=\textwidth]{figures/btree/btree_trace4}
                  \mycap{After applying rule 1 at node \code{@7}.}
                  \label{fig:language:btree_trace4}
          \end{subfigure}
        \mycap{An execution trace for the binary tree dictionary
           algorithm. The first argument of each fact was dropped and the
           address of the node was placed beside it.}\label{fig:language:btree_trace}
\end{figure}

Like the message routing example shown in the previous section, the amount of
concurrency present in this example depends on the number of \code{replace}
facts. If there are many \code{replace} operations to perform on different parts
of the BST, then there is more concurrency. On the other hand, if only a few BST
nodes are being updated, then concurrency is reduced. This is the same behavior
that one would expect from an parallel implementation of the BST data structure
in an imperative language.


\subsection{Third Example: Graph Visit}

\begin{figure}[h!]
\begin{Verbatim}[numbers=left,fontsize=\codesize,commandchars=\*\[\]]
type edge(node, node).*hfill// Predicate declaration
type linear visit(node).
type linear unvisited(node).
type linear visited(node).

visit(A), *label[line:language:visit_first1]*hfill // Rule 1: visit node
unvisited(A) -o
   visited(A),
   {B | !edge(A, B) -o visit(B)}.*label[line:language:visit_first2]*label[line:language:visit_comprehension]

visit(A),*label[line:language:visit_second1]*hfill // Rule 2: node already visited
visited(A)
   -o visited(A).*label[line:language:visit_second2]
\end{Verbatim}
  \mycap{LM code for the graph visit program.}
  \label{code:language:visit}
\end{figure}

Our third example, shown in Fig.~\ref{code:language:visit}, presents another LM
program that, for a given graph of nodes, performs a visit to all nodes
reachable from node \code{@1}. The first rule of the program
(lines~\ref{line:language:visit_first1}-\ref{line:language:visit_first2}) visits
node \code{A} for the first time: fact \code{visited(A)} is derived and a
\emph{comprehension} construct is used to go through all the edge facts and then
derive \code{visit(B)} at each neighbor node \code{B}. The second rule of the
program
(lines~\ref{line:language:visit_second1}-\ref{line:language:visit_second2})
applies when the node is already visited more than once: we keep the
\code{visited} fact and delete \code{visit}. The initial facts shown in
Fig.~\ref{code:language:visit_initial} use the \code{visit(@1)} fact to start
the program.

\begin{figure}[h!]
\begin{Verbatim}[numbers=left,fontsize=\codesize,commandchars=\*\[\]]
!edge(@1, @2).
!edge(@1, @4).
!edge(@2, @3).
!edge(@2, @4).
!edge(@2, @1).
!edge(@3, @2).
!edge(@4, @1).
!edge(@4, @2).

unvisited(@1).
unvisited(@2).
unvisited(@3).
unvisited(@4).

visit(@1).
\end{Verbatim}
  \mycap{Initial facts for the graph visit program. Nodes reachable from node \code{@1} will be marked as \code{visited}.}
  \label{code:language:visit_initial}
\end{figure}

Fig.~\ref{fig:exec_trace} shows a possible execution trace for the visit
program. After applying the first rule at node \code{@1} we get the database in
Fig~\ref{fig:exec_trace}~(b).  Note that node \code{@1} is now visited and nodes
\code{@2} and \code{@4} now have a \code{visit} fact. At this point, we could
either apply rule 1 at node \code{@2} or at node \code{@4}.  For this specific
trace, we apply the rule at node \code{@2}, resulting in
Fig.~\ref{fig:exec_trace}~(c). Node \code{@4} now has two \code{visit} facts,
thus we can apply rule 1 followed by rule 2, therefore consuming both
\code{visit} facts and deriving \code{visited}. In addition, we can also apply
rule 1 at node \code{@3} to reach the state of Fig.~\ref{fig:exec_trace}~(d).

\begin{figure}[h]
        \centering
        \begin{subfigure}[b]{0.45\textwidth}
                \includegraphics[width=\textwidth]{figures/visit/trace1}
                \mycap{Initial database.}
                \label{fig:exec_trace1}
        \end{subfigure}%
        ~ %add desired spacing between images, e. g. ~, \quad, \qquad etc.
          %(or a blank line to force the subfigure onto a new line)
        \begin{subfigure}[b]{0.45\textwidth}
                \includegraphics[width=\textwidth]{figures/visit/trace2}
                \mycap{After applying rule 1 at node \code{@1}.}
                \label{fig:exec_trace2}
        \end{subfigure}\\
        \begin{subfigure}[b]{0.45\textwidth}
                \includegraphics[width=\textwidth]{figures/visit/trace3}
                \mycap{After applying rule 1 at node \code{@2}.}
                \label{fig:exec_trace3}
        \end{subfigure}%
        ~ %add desired spacing between images, e. g. ~, \quad, \qquad etc.
         %(or a blank line to force the subfigure onto a new line)
        \begin{subfigure}[b]{0.45\textwidth}
                  \includegraphics[width=\textwidth]{figures/visit/trace4}

                  \mycap{After applying rule 1 and 2 at node \code{@4} and
                  rule 1 at node \code{@3}.}

                  \label{fig:exec_trace4}
          \end{subfigure}
        \mycap{A possible execution trace for the visit
           program. Note that the \code{edge} facts were omitted for simplicity.}
        \label{fig:exec_trace}
\end{figure}

The graph visit has the potential to have plenty of concurrency. Once the visit
starts from the initial node, it expands to other nodes, allowing several nodes
to derive rules concurrently. At some point, the amount of concurrency is
reduced because more and more nodes have been visited. The level of concurrency
depends on the structure of the graph. In the worst case, if the graph is a
chain of nodes then the program becomes effectively sequential.  In the best
case, if the graph is densily connected (each node connects to most nodes in the
graph), then it is possible to run the program on most nodes at the same time.

The goal of introducing higher-level declarative languages is to facilitate
reasoning about the properties of a program. In the case of the visit program,
the most important goal is to prove that, if the graph is connected, then all
the nodes will become \code{visited}, regardless of the order in which we apply
the rules. First, we define a visit graph:

\begin{definition}[Visit graph]
A visit graph is an ordered pair $G = (N, E)$ comprising a set $N$ of nodes together
with a set $E$ of edges. Set $E$ contains pairs $(A, B)$ that correspond to
facts $\bang \mathtt{edge}(A, B)$. For every pair $(A, B) \in E$ there is also a
pair $(B, A) \in E$, representing an undirected edge.
\end{definition}

To prove the correctness property of the program, we first define the main
\emph{invariant} of the program:

\begin{invariant}[Node state]
A node is either \code{visited} or \code{unvisited} but not both.
\end{invariant}

\begin{proof}

From the initial facts we check if all nodes start as \code{unvisited}. This is
clearly true for the previous example. Rule 1 changes a node from
\code{unvisited} to \code{visited}, while rule 2 keeps the node \code{visited},
proving the invariant.

\end{proof}

Invariants are conditions that never change, no matter which rules are derived
during execution. With this invariant, it is now possible to classify nodes of
the graph $G$ according to their state:

\begin{definition}[Node sets] \code{visited} nodes are contained in set $V$,
while \code{unvisited} nodes are in set $U$. From the node state invariant, we
know that $V \cup U = N$ and $V \cap U = \emptyset$.
\end{definition}

We can now prove an important lemma about sets $V$ and $U$:

\begin{invariant}[Visited set]

After each rule derivation, visited set $V$ always increases or stays the same
size. The inverse is true for set $U$.

\end{invariant}
\begin{proof}
Initially, $V = \emptyset$ and $U = N$.
By rule 1, $V$ increases by 1 while $U$ decreases by 1. With rule 2, set
membership remains unchanged.
\end{proof}

In turn, since set membership changes from $U$ to $V$, we now prove the
following:

\begin{lemma}[Edge visits]\label{lemma:language:edge_visits}
The program generates at most one \code{visit} per directed edge and for a node
$a \in N$ that receives a \code{visit} fact, then for all $b \in N$ where $(a,
b) \in E$, exactly one \code{visit} fact is generated at $b$.
\end{lemma}
\begin{proof}
From the visited set invariant, we know that once nodes become members of set $V$,
they no longer return to set $U$, therefore rule 1 applies once per
node. This rule generates a \code{visit} fact per neighbor node.
\end{proof}

In order to prove that all the nodes in the graph are visited, we need to make
sure that the graph is connected.

\begin{definition}[Connected graph]
A connected graph is a graph where every pair of nodes has a path between them.
\end{definition}

Finally, we prove that all nodes will become \code{visited}.

\begin{theorem}[Graph visit correctness]
If graph $G$ is connected, set $V$ will eventually include all nodes in $N$,
while $U = \emptyset$.
\end{theorem}
\begin{proof}
Proof by induction.

\begin{itemize}

   \item Base case: initial fact \code{visit(@1)} adds node \code{@1} to $V$.
      By Lemma~\ref{lemma:language:edge_visits}, a \code{visit} fact is generate
      for all edges of \code{@1}.

   \item Inductive case: assume visited set $V'$ and unvisited set $U'$.
  Since the graph
   is connected, there must be a node $a \in V'$ that is connected to a node $b
   \in U'$. Using the Edge visits lemma, a $\mathtt{visit}(b)$ fact is generated,
   swapping $b$ from $U'$ to $V'$.
\end{itemize}

Eventually, set $V$ will include all nodes in $N$. Otherwise, there would be
unreachable nodes in the graph and that would be a contradiction since the graph
is connected.
\end{proof}



\section{LM Syntax}

\begin{table}[h]
\centering
\begin{tabular}{ l l c l }
  Program & $Prog$ & $::=$ & $\Sigma, D$ \\
  List Of Rules & $\Sigma$ & $::=$ & $\cdot \; | \; \Sigma, R$\\
  Database & $D$ & $::=$ & $\Gamma; \Delta$ \\
  Rule & $R$ & $::=$ & $BE \lolli HE \; | \; \forall_{x}. R \; | \; \selector{S}{y}{BE}{HE}$ \\
  Body Expression & $BE$ & $::=$ & $L \; | \; P \; | \; C \; | \; BE, BE \; | \;
\exists_{x}. BE \; | \; \one$\\
  Head Expression & $HE$ & $::=$ & $L \; | \; P \; | \; HE, HE \; | \; EE \; |
  \; CE \; | \; AE \; | \; \one$\\
  
  Linear Fact & $L$ & $::=$ & $l(\hat{x})$\\
  Persistent Fact & $P$ & $::=$ & $\bang p(\hat{x})$\\
  Constraint & $C$ & $::=$ & $c(\hat{x})$ \\
  Selector Operation & $S$ & $::=$ & $\mathtt{min} \; | \; \mathtt{max} \; | \; \mathtt{random}$\\
  
  Exists Expression & $EE$ & $::=$ & $\existsc{\widehat{x}}{SH}$ \\
  Comprehension & $CE$ & $::=$ & $\comprehension{\widehat{x}}{SB}{SH}$ \\

  Aggregate & $AE$ & $::=$ & $\aggregate{A}{y}{\widehat{x}}{SB}{SH_1}{SH_2}$ \\
  Aggregate Operation & $A$ & $::=$ & $\mathtt{min} \; | \; \mathtt{max} \; | \;
\mathtt{sum} \; | \; \mathtt{count} \; | \; \mathtt{collect}$ \\
  
  Sub-Body & $SB$ & $::=$ & $L \; | \; P \; | \; SB, SB \; | \; \exists_{x}. SB$\\
  Sub-Head & $SH$ & $::=$ & $L \; | \; P \; | \; SH, SH \; | \; \one$\\
  
  Known Linear Facts & $\Delta$ & $::=$ & $\cdot \; | \; \Delta, l(\hat{t})$ \\
  Known Persistent Facts & $\Gamma$ & $::=$ & $\cdot \; | \; \Gamma, \bang p(\hat{t})$ \\
\end{tabular}
\caption{Abstract syntax of LM.}\label{tbl:ast}
\end{table}

Table~\ref{tbl:ast} shows the abstract syntax for rules in LM.  A LM program
$Prog$ consists of a list of derivation rules $\Sigma$ and a database $D$.  Each
derivation rule $R$ can be written as $BE \lolli HE$ where $BE$ is the body of a
rule and $HE$ is the head. Rules without bodies are allowed in LM and they are
called \emph{initial facts}
(lines~\ref{line:language:dict_axioms1}-\ref{line:language:dict_axioms2} in
Fig.~\ref{code:btree_replace}). Rules without heads are also allowed.

The body of a rule, $BE$, may contain linear ($L$) and persistent ($P$)
\emph{fact expressions} and constraints ($C$). Fact expressions are template
facts that instantiate variables (from facts in the database) such as
\code{visit(A)} in line~\ref{line:language:visit_second} in
Fig.~\ref{code:visit}. Variables can be used again in the body for matching and
also in the head when instantiating facts.  Constraints are boolean expressions
that must be true in order for the rule to be fired. Constraints use variables
from fact expressions and are built using a small functional language that
includes mathematical operations, boolean operations, external functions and
literal values.

The head of a rule, $HE$, contains linear ($L$) and persistent ($P$) \emph{fact
templates} which are uninstantiated facts and will derive new facts. The head
can also have \emph{exists expressions} ($EE$), \emph{comprehensions} ($CE$)
and \emph{aggregates} ($AE$). All those expressions may use all the variables
instantiated in the body.

\subsubsection{Graph visit using abstract syntax}\label{visit:ast}

In order to show how programs are represented using the abstract syntax
presented in Table~\ref{tbl:ast}, we present the 2 rules in the graph visit
program shown in Fig.~\ref{code:visit}:

\nopagebreak

\begin{align}
\forall_A. \mathtt{visit}(A), \; \mathtt{unvisited}(A) \lolli & \;
\mathtt{visited}(A), \; \{ B; \; \bang \mathtt{edge}(A, B); \;
\mathtt{visit}(B)\}\\
\forall_A. \mathtt{visit}(A), \; \mathtt{visited}(A) \lolli & \;
\mathtt{visited}(A)
\end{align}

Finally, initial facts are represented using an empty body:

\nopagebreak

\begin{align}
\one \lolli & \bang \mathtt{edge}(@1, @2) \\
\one \lolli & \bang \mathtt{edge}(@2, @3) \\
\one \lolli & \bang \mathtt{edge}(@1, @4) \\
\one \lolli & \bang \mathtt{edge}(@2, @4) \\
\one \lolli & \bang \mathtt{unvisited}(@1)  \\
\one \lolli & \bang \mathtt{unvisited}(@2) \\
\one \lolli & \bang \mathtt{unvisited}(@3) \\
\one \lolli & \bang \mathtt{unvisited}(@4) \\
\one \lolli & \bang \mathtt{visit}(@1)
\end{align}

\subsection{Predicates And Facts}

Each fact is an association between a \emph{predicate} and a tuple of values. A
predicate is a pair with a name and a tuple of types (the argument types). LM
rules are type-checked using the predicate declarations in the header of the
program. LM has a simple type system that includes the following simples types:
\emph{node}, \emph{int}, \emph{float}, \emph{string}, \emph{bool}. Recursive
types such as \emph{list X} and \emph{pair X; Y} are also allowed.

LM allows definition of new type names from simpler types using the declaration
\code{type simple-type new-type.} in the header of the program. The type
\code{new-type} can then be used as any other type. Note that LM uses
\emph{structural equivalence} to check if two types are the same, therefore
\code{simple-type} and \code{new-type} are type equivalent.

Node subtypes can be introduced using the declaration \code{node subnode.},
allowing the programmer to use nodes with separate predicates.  The
\emph{subnode} type becomes a subtype of \emph{node}, that is, $subnode <:
node$. If initial facts are declared using the form \code{f(A)}, where \code{A} is a
\emph{subnode}, then the axiom is selectively instantiated on nodes with type
\emph{subnode}. Otherwise, if \code{A} was a \emph{node} then \code{f(A)} would
be instantiated in every node.

\subsection{Selectors}

When a rule body is instantiated using facts from the database, facts are picked
non-deterministically. While our system uses an implementation dependent order
for efficiency reasons, sometimes it is important to sort facts by one of the
arguments because linearity imposes commitment during rule derivation. The
abstract syntax for this construct is $\selector{S}{y}{BE}{HE}$, where $S$ is
the selection operation and $y$ is the variable in the body $BE$ that represents
the value to be selected according to $S$. An example using concrete syntax is
as follows:

\begin{Verbatim}[fontsize=\codesize]
[min => W | !edge(A, B), weight(A, B, W)]
                              -o picked(A, B, W).
\end{Verbatim}

In this case, we will order the \code{weight} facts by \code{W} in ascending
order and then try to match them. Other operations available are \code{max}
and \code{random} (to force no pre-defined order at the implementation level).

\subsection{Exists Expression}

Exist constructs ($EE$) are based on the linear logic construct of the same name
and are used to create new node addresses. We can use the new address to
instantiate new facts for the new node.  The following example illustrates the
use of the exists construct, where we derive \code{perform-work} at a new node
\code{B}.

\begin{Verbatim}[fontsize=\codesize]
   work(A, W) -o exists B. (perform-work(B, W)).
\end{Verbatim}

\subsection{Comprehensions}

Sometimes we need to consume a linear fact and then immediately generate several
facts depending on the contents of the database. To solve this particular need,
we created the concept of comprehensions, which are sub-rules that are applied
with all possible combinations of facts from the database. In a comprehension
$\comprehension{\widehat{x}}{SB}{SH}$, $\widehat{x}$ is a list of variables,
$SB$ is the body of the comprehension and $SH$ is the head.  The body $SB$ is
used to generate all possible combinations for the head $SH$, according to the
facts in the database.  We have already seen an example of comprehensions in the
visit program (Fig.~\ref{code:visit}
line~\ref{line:language:visit_comprehension}). Here, we match \code{!edge(A,
B)} using all the combinations available in the database and for each
combination we derive \code{visit(B)}.

\subsection{Aggregates}

Another useful feature in logic programs is the ability to reduce several facts
into a single fact.  In LM we have aggregates ($AE$), a special kind of sub-rule
that works very similarly to comprehensions.  In the abstract syntax
$\aggregate{A}{y}{\widehat{x}}{SB}{SH_1}{SH_2}$, $A$ is the aggregate operation,
$\widehat{x}$ is the list of variables introduced in $BE$, $SH_1$ and $SH_2$
and $y$ is the variable in the body $SB$ that represents the values to be
aggregated using $A$. Like comprehensions, we use $\widehat{x}$ to try all
the combinations of $SB$, but, in addition to deriving $SH_1$ for each
combination, we aggregate the values represented by $y$ into a new $y$
variable that is used inside the head $SH_2$.

To better understand aggregates, let's consider a database with the following
facts and a rule:

\begin{Verbatim}[fontsize=\codesize]
price(@1, 3).
price(@1, 4).
price(@1, 5).
count-prices(@1).
count-prices(A) -o [sum => P | . | price(A, P) | 1 | total(A, P)].
\end{Verbatim}

By applying the rule, we consume \code{count-prices(@1)} and derive the
aggregate which consumes all the \code{price(@1, P)} facts.  These are summed
and \code{total(@1,~12)} is derived. LM provides several aggregate operations,
including the \code{min} (minimum value), \code{max} (maximum value),
\code{sum} (add all numbers), \code{count} (count combinations) and
\code{collect} (collect items into a list).

\section{Operational Semantics}

As said before, the first argument of every predicate must be typed as a
\emph{node}.  For distribution purposes, the body of all derivation rules can
only use facts from the same node in order to avoid concurrency issues.
However, the facts in the head may refer to other nodes, as long as those nodes
are instantiated in the body of the rule.  We drew some inspiration from the
Linda system~\cite{linda} mentioned early on, where the tuple space contains
the data and is used by the processors to communicate and do computation.  This
differs from imperative languages, since in those languages data and computation
are two separate entities.

The execution is performed at the node level and happens non-deterministically
(i.e., any node can be picked to run). This means that the programmer cannot
expect that facts coming from different nodes will be considered as a whole or
partially since the process is non-deterministic. The operational semantics
promises that rule derivations are performed atomically, therefore if a rule
sends many facts to a node then the target node will receive them all at once.
Under these restrictions, computation can then be parallelized by processing
many nodes concurrently.

Each rule in LM has a defined priority that is inferred from its position in the
source file.  Rules at the beginning of the file have higher priority. At the
node level, we consider all the new facts that have been not consider before to
create a priority queue of \emph{candidate rules}.  The queue of candidate rules
is then applied (by priority) and updated as new facts are derived or consumed.
Section~\ref{sec:implementation:rule_engine} gives details in how our
implementation manages the set of candidate rules.

\section{Applications}
In this section, we present solutions to well-known problems. We start with
straightforward graph-based problems such as bipartitness checking and two
versions of the PageRank program. Next, we present the LM version of the
QuickSort algorithm, which from a first impression may not fit well under the
programming paradigm offered by LM. Informal correctness and termination proofs
are also included to further show that such important properties are relatively
easy to prove for programs written in LM.

\subsection{Bipartiteness Checking}

The problem of checking if a graph is bipartite can be seen as a 2-color graph
coloring problem.  The code for this algorithm is shown in
Fig.~\ref{language:code:bichecking}. All nodes in the graph start as
\texttt{uncolored},
because they do not have a color yet. The axiom \texttt{visit(@1, 1)} is
instantiated at node \texttt{@1} (line 9) in order to color it with color 1.

If a node is \texttt{uncolored} and needs to be marked with a color \texttt{P}
then the rule in lines 11-12 is applied. We consume the \texttt{uncolored} fact
and derive a \texttt{colored(A, P)} to effectively color the node with
\texttt{P}. We also derive \texttt{visit(B, next(P))} in neighbor nodes to color
them with the other color. Line 

The coloring can fail if a node is already colored with a color \texttt{P} and
needs to be colored with a different color (line 15) or if it has already failed
(line 16).

\begin{figure}[h!]
\begin{Verbatim}[numbers=left,fontsize=\scriptsize]
type route edge(node, node).
type linear visit(node, int).
type linear uncolored(node).
type linear colored(node, int).
type linear fail(node).

fun next(int X) : int = if X <> 1 then 1 else 2 end.

visit(@1, 1).

visit(A, P), uncolored(A)
   -o {B | !edge(A, B) | visit(B, next(P))}, colored(A, P).

visit(A, P), colored(A, P) -o colored(A, P).
visit(A, P1), colored(A, P2), P1 <> P2 -o fail(A).
visit(A, P), fail(A) -o fail(A).
\end{Verbatim}
  \caption{Bipartiteness Checking program.}
  \label{language:code:bichecking}
\end{figure}

\subsubsection{Proof Of Correctness}

In order to show that the code in Fig.~\ref{language:code:bichecking} works as
intended, we first setup some invariants that hold throughout the execution of
the program. Assume that the set of nodes in the graph is represented as $N$.

\begin{invariant}[Node state]
Set of nodes $N$ is partitioned into 4 different states that represent the 4
possible states that a node can be in, namely:

\begin{itemize}
   \item $U$ (\texttt{uncolored} nodes)
   \item $F$ (\texttt{fail} nodes)
   \item $C_{true}$ (\texttt{colored(A, 1)} nodes)
   \item $C_{false}$ (\texttt{colored(A, 2)} nodes)
\end{itemize}
\end{invariant}
\begin{proof}
Initially, all nodes start in set $U$. All the 4 rules of the programs either
keep the node in the same set or exchange the node with another set.
\end{proof}

A bipartite graph is one where in every edge $a \leftrightarrow b$, there is a
valid assignment that makes $a$ member of set $C_{true}$ or $C_{false}$ and node
$b$ member of either $C_{false}$ or $C_{true}$ respectively.

\begin{lemma}[Bipartiteness
   Convergence]\label{language:lemma:bipartite_convergence}
   We now reason from the application of the program rules. After each
   application of an inference rule, one of the following will happen:

   \begin{itemize}
      \item Set $U$ will decrease and set $C_{true}$ or $C_{false}$ will
         increase, with a potential increase in the number of \texttt{visit}
         facts.
      \item Set $C_{true}$ or $C_{false}$ will stay the same, while the number
         of \texttt{visit} facts will be reduced.

      \item Set $C_{true}$ or $C_{false}$ will decrease and set $F$ will
         increase, while the number of \texttt{visit} facts will be reduced.

      \item Set $F$ will stay the same, while the number of \texttt{visit} facts
         decreases.
   \end{itemize}

\end{lemma}
\begin{proof}
Directly from the rules.
\end{proof}

From this invariant, it can be inferred that set $U$ never increases in size
and in a node transition from \texttt{uncolored} to \texttt{colored}, the
database may increase in size. For every other rule application, the database of
facts always decreases. This also means that the program will eventually
terminate, since it is limited by the number of \texttt{visit} facts that can be
generated.

\begin{theorem}[Bipartiteness Correctness]
If the graph is connected and bipartite then the nodes will be partitioned into
sets $C_{true}$ and $C_{false}$, while sets $F$ and $U$ are empty.
\end{theorem}
\begin{proof}
   By induction, we prove that uncolored nodes become part of either $C_{true}$
   and $C_{false}$ and, if there is an edge between nodes in the two sets then
   they have different colors.

   In the base case, we start with empty sets but node \texttt{@1} is made
   member of $C_{true}$. Rule 1 sends \texttt{visit} facts to the neighbors of
   \texttt{@1}, forcing them to be members of $C_{false}$.

   In the inductive case, we have sets $C'_{true}$ and $C'_{false}$ with some
   nodes already colored. From Lemma~\ref{language:lemma:bipartite_convergence},
   we know that $U$ always decreases. Since the graph is bipartite, events 3 and
   4 never happen since there is a possible partitioning of nodes. With event 1,
   we have set $C_{true} = C'_{true}, n$, (or $C_{false}$) where $n$ is the
   node and with event 2, the sets remain the same. Since the graph is
   connected, every node will be colored, therefore event 1 will happen for
   every node of the graph.
\end{proof}

\subsection{PageRank}

PageRank~\cite{Page:2001:MNR} is a well known graph algorithm that is used to
compute the relative relevance of web pages.  The code for a synchronous version
of the algorithm is shown in Fig.~\ref{code:pagerank}.  As the name indicates,
the pagerank is computed for a certain number of iterations. The initial
pagerank is the same for every page and is initialized in the first rule (lines
15-16) along with an accumulator.

The second rule of the program (lines 17-19) propagates a newly computed
pagerank value to all neighbors. The fact \texttt{neighbor-pagerank} informs the
neighbor node about the pagerank value of node \texttt{A} for iteration
\texttt{Iter + 1}. For every iteration, each node will accumulate the all the
\texttt{neighbor-pagerank} facts into the \texttt{accumulator} fact (lines
27-28). When all inbound neighbor pagerank values are accumulated, the third
rule (lines 21-25) is fired and a pagerank value is derived for iteration
\texttt{Iter}.

\begin{figure}[h!]
\begin{Verbatim}[numbers=left,fontsize=\scriptsize]
type outbound(node, node, float).
type linear pagerank(node, float, int).
type numInbound(node, int).
type linear accumulator(node, float Acc, int Left, int Iteration).
type linear neighbor-pagerank(node, node Neighbor, float Rank, int Iteration).
type linear start(node).

const damping = 0.85. // probability of user following a link in the current page.
const iterations = str2int(@arg1). // iterations to compute.
const pages = @world. // number of pages in the graph.

start(A).

start(A), !numInbound(A, T)
   -o accumulator(A, 0.0, T, 1), pagerank(A, 1.0 / float(pages), 0).

pagerank(A, V, Iter), // propagate new pagerank value
Iter < iterations
   -o {B, W | !outbound(A, B, W) | neighbor-pagerank(B, A, V * W, Iter + 1)}.

accumulator(A, Acc, 0, Iter), // new pagerank
!numInbound(A, T),
V = damping + (1.0 - damping) * Acc,
Iter <= iterations
   -o pagerank(A, V, Iter), accumulator(A, 0.0, T, Iter + 1).
	
neighbor-pagerank(A, B, V, Iter), accumulator(A, Acc, T, Iter)
   -o accumulator(A, Acc + V, T - 1, Iter).
\end{Verbatim}
\caption{Synchronous PageRank program.}
\label{language:code:pagerank}
\end{figure}

\subsubsection{Asynchronous PageRank}

We also have an asynchronous version of the algorithm that trades correctness
with convergence speed since it does not synchronize between iterations.
Figure~\ref{language:code:async_pagerank} shows the LM code for this particular
version, where two major differences can be observed: (1) there is a linear fact
\texttt{neighbor-pagerank} containing the most up-to-date pagerank value of a
neighbor node; (2) a new \texttt{update} fact that forces the node to re-compute
its pagerank by processing the currently available \texttt{neighbor-pagerank}
facts. Rules in lines 13-21 update the \texttt{neighbor-pagerank} values, while
rule in lines 23-29 asynchronously update the current pagerank value. This last
rule derives multiple \texttt{new-neighbor-rank} that is used to inform the
neighbor about the new pagerank value.

\begin{figure}[h!]
\begin{Verbatim}[numbers=left,fontsize=\scriptsize]
type outbound(node, node, float).
type linear pagerank(node, float, int).
type numInbound(node, int).
type linear neighbor-pagerank(node, node Neighbor, float Rank, int Iteration).
type linear new-neighbor-rank(node, node Neighbor, float Rank, int Iteration).
type linear update(A).
type linear sum-ranks(node, float).

pagerank(A, 1.0 / float(pages), 0).
update(A).
neighbor-pagerank(A, B, 1.0 / float(pages), 0). // pagerank of B is ...

// save incoming pagerank value.
new-neighbor-rank(A, B, New, Iteration),
neighbor-pagerank(A, B, Old, OldIteration),
Iteration > OldIteration
   -o neighbor-pagerank(A, B, New, Iteration).
new-neighbor-rank(A, B, New, Iteration),
neighbor-pagerank(A, B, Old, OldIteration),
Iteration <= OldIteration
   -o neighbor-pagerank(A, B, Old, OldIteration).

sum-ranks(A, Acc),
NewRank = damping/float(pages) + (1.0 - damping) * Acc,
pagerank(A, OldRank, Iteration)
      -o pagerank(A, NewRank, Iteration + 1),
         {B, W, Delta, Iter | !outbound(A, B, W), Delta = fabs(NewRank -
               OldRank) * W | new-neighbor-rank(B, A, NewRank, Iteration + 1),
               if Delta > bound then update(B) end}.

update(A), update(A) -o update(A).

update(A),
!numInbound(A, T)
   -o [sum => V | B, W, Val, Iter | neighbor-pagerank(A, B, Val, Iter)
         V = Val/float(T) | neighbor-pagerank(A, B, Val, Iter) | sum-ranks(A, V)].
\end{Verbatim}
\caption{Asynchronous PageRank program.}
\label{language:code:async_pagerank}
\end{figure}

\subsubsection{Proof Of Correctness}

To build the proof of correctness, we must again prove several program
invariants. This will help us prove that this partifcular program is similar to
a computation on a nonnegative matrix of of unit spectral radius, which has been
proven that it converges~\cite{DBLP:journals/corr/abs-cs-0606047,
Lubachevsky:1986:CAA:4904.4801}.

\begin{invariant}[Page Invariant]
Each page/node has a single \texttt{pagerank(A, Value, Iteration)} and:
\begin{itemize}
   \item For each outbound link, a single \texttt{\bang outbound(A, B, W)}.
   \item For each inbound link, a single \texttt{neighbor-pagerank(A, B, V, Iter)}.
   \item For each \texttt{\bang outbound(A, B, W)}, a \texttt{neighbor-pagerank(A,
      B, V, Iter}.
\end{itemize}
\end{invariant}

\begin{proof}

All axioms validate the 3 conditions of the variant. Note that the third
condition is also validated by the axioms, although not all axioms are shown in
the code.

In relation to rule application:

\begin{itemize}
   \item Rule 1: inbound link re-derived.
   \item Rule 2: inbound link re-derived.
   \item Rule 3: \texttt{pagerank/2} re-derived.
   \item Rule 4: Nothing happens.
   \item Rule 5: inbound links re-derived in the comprehension.
\end{itemize}
\end{proof}

\begin{lemma}[Neighbor rank lemma]
Given a fact \texttt{neighbor-pagerank(A, B, V, Iter)} and a set of facts
\texttt{new-neighbor-rank(A, B, New, Iter2)}, we end up with a single
\texttt{neighbor-pagerank(A, B, V', Iter')}, where \texttt{Iter} is the greater of
\texttt{Iter} or all of \texttt{Iter2'}.
\end{lemma}
\begin{proof}
By induction on the number of \texttt{new-neighbor-rank} facts.

Base case: \texttt{neighbor-pagerank} remains.

Inductive case: given one \texttt{new-neighbor-rank} fact:

\begin{itemize}
   \item Rule 1: the new iteration is older and thus \texttt{neighbor-pagerank}
   is replaced. By applying induction, we know that we will select either the
   new best iteration or a better iteration from the remaining set of
   \texttt{new-neighbor-rank} facts.
   \item Rule 2: the new iteration is not older and we keep the old
   \texttt{neighbor-pagerank} fact. By induction, we select the best from either
   the current iteration or some other (from the set).
\end{itemize}
\end{proof}

\begin{lemma}[Update lemma]
Given at least 1 \texttt{update/1} fact, rule 7 will run.
\end{lemma}
\begin{proof}
By induction on the number of \texttt{update} facts.

Base case: rule 5 will run.

Inductive case: rule 4 will run first because it has a higher priority, reducing
the number of \texttt{update} facts by one. By induction, we know that by
using the remaining \texttt{update} facts, rule 7 will run.
\end{proof}

\begin{lemma}[Pagerank update lemma]
(1) Given at least one \texttt{update} fact, the \texttt{pagerank(A, $V_{I}$,
I)} fact will be updated to become \texttt{pagerank(A, $V_{I + 1}$, I +
1)}, where \texttt{$V_{I + 1} = damping / P + (1.0 - damping)\sum_{B,
I} (W_{B} \times  N_{I,B})$}.

where $W_B = 1.0/T$ ($T$ from \texttt{\bang numInbound(A, $T$)})
and $N_{I,B}$ from \texttt{neighbor-pagerank(A, B, $N_{I, B}$, $I$)}.

(2) For all \texttt{B} outbound nodes (represented using \texttt{\bang outbound(A, B,
W)}, a \texttt{new-neighbor-rank(B, A, $V_{I+1}$, $I + 1$)} is generated.

(3) For all \texttt{B} outbound nodes (represented using \texttt{\bang outbound(A, B,
W)}), a \texttt{update(B)} is generated if 
$fabs(V_{I + 1} - V_{I}) \times W > bound$.
\end{lemma}
\begin{proof}
Using the Update lemma, rule 5 will necessarily run.

It derives \texttt{sum-ranks(A, $\sum_{B, I} (W_B \times N_{I,B})$)} and
fulfills (3).

\texttt{sum-ranks/2} will necessarily fire rule 6,
computing $V_{I+1}$ and updating \texttt{pagerank}. (2) and (3) are fulfilled
through the comprehension of rule 6.
\end{proof}

\begin{invariant}[New neighbor rank equality]
All \texttt{new-neighbor-rank(A, B, V, I)} facts are generated from a corresponding
\texttt{pagerank(B, V, I)} fact, therefore the iteration of any
\texttt{new-neighbor-rank} is at least the same or less than the iteration of
the current pagerank.
\end{invariant}
\begin{proof}
No axioms to prove.

\begin{itemize}
   \item Rule 3: true, new fact is generated.
   \item Rule 6: the fact is kept.
\end{itemize}
\end{proof}

\begin{invariant}[Neighbor rank equality]
All \texttt{neighbor-pagerank(A, B, V, I)} facts have one corresponding
\texttt{pagerank(B, V, I)} fact and the iteration of the \texttt{neighbor-pagerank}
is the same or less than the current iteration of the corresponding
\texttt{pagerank}.
\end{invariant}
\begin{proof}
By analyzing axioms and rules.

Axioms: true.

Rule cases:

\begin{itemize}
   \item Rule 1: uses \texttt{new-neighbor-rank} fact (use new neighbor rank
         equality invariant).
   \item Rule 2: same fact is re-derived.
\end{itemize}
\end{proof}

\begin{theorem}[Pagerank convergence]
The program will compute the pagerank of all nodes that is within \texttt{bound} error
of an asynchronous pagerank computation.
\end{theorem}
\begin{proof}

Using the program axioms, we start with the same pagerank value for all nodes.
The \texttt{\bang outbound(A, B, W)} fact forms a $n \times n$ square matrix (number
of nodes) and is the so-called "Google Matrix".  All the initial pagerank values
can be seen as a vector that adds up to $1$.

The pagerank computation from the "Pagerank update lemma" computes $V_{I + 1} =
damping / P + (1.0 - damping)\sum_{B, I'} (W_{B} \times N_{I',B})$, where $I' <=
I$
(from Neighbor rank equality invariant).

Consider that each node contains a column $G_i$ of the Google matrix. The
pagerank computation can then be represented as: \newline


$V_{I + 1} = G_i fix([N_{I_1, B_1}, ..., N_{I_p, B_p}])$ \hfill (1) \\


Where $p$ is the number of inbound links and $N_{I_j, B_j}$ is the value of
the \texttt{neighbor-pagerank(A, $B_j$, $N_{I_j, B_j}$, $I_j$)}. The $fix()$
function takes the neighbor vector and expands it with zeros corresponding to
nodes that are not inbound links.

From~\cite{DBLP:journals/corr/abs-cs-0606047, Lubachevsky:1986:CAA:4904.4801} we
know that equation (1) will improve (converge) the pagerank value, given that some new
neighbor pagerank values are sent to node $i$ and by the fact that $G_i$ is a
nonnegative matrix of unit spectral radius. Let's use induction by assuming that there
is at least one \texttt{update/1} fact that
schedules a node to improve its pagerank. We want to prove that such fact will
not only improve the node's pagerank but also the pagerank vector.
If the pagerank vector is now close enough (within \texttt{bound}), then the
program will terminate.

\begin{itemize}
   \item Base case: Since we have an \texttt{update} fact as an axiom, rule 7
   will compute a new pagerank (Pagerank update lemma) for all nodes that is
   improved (from equation (1)). From these updates, a new \texttt{update}
   fact is generated that correspond to nodes that have inbound links from the
   source node and need to update their pagerank. These \texttt{update} facts
   may not be generated if the pagerank vector is close enough to its real
   value.

   \item The induction hypothesis tells us that there is at least one node that
   has an \texttt{update} fact. From pagerank update lemma, this
   generates \texttt{new-neighbor-rank} facts if the new value differs
   significantly from the older value. When this happens and using the ``Neighbor
   rank lemma'', the target node will update its \texttt{neighbor-pagerank} fact
   with the newest iteration and then execute a valid pagerank computation that
   brings the pagerank vector close to its solution.

\end{itemize}

\end{proof}


\subsection{Quick-Sort}

The quick-sort algorithm is a divide and conquer sorting algorithm that works by splitting
a list of items into two sublists and then recursively sorting the two sublists.
To split the list, we pick a pivot element and put the items that are smaller than the pivot
into the first sublist and items greater than the pivot into the second list.

The quick-sort algorithm is interesting because it does not map immediately to the graph-based
model of LM. Our approach considers that the program starts with a single node where
the initial list is located. Then we split the list as usual and create two nodes
that will recursively sort the sublists. Interestingly, this will create a tree
that will look similar to a call tree in a functional language.

Fig.~\ref{code:quicksort} presents the code for the quick-sort algorithm in LM.
For each sublist to sort, we start with a \texttt{down} fact that must be (eventually)
transformed into an \texttt{up} fact, where the sublist in the \texttt{up} fact is sorted.
In line 11 we start with the initial list at node \texttt{@0}. Lines 13-16 will immediately
sort the list when the number of items is very small. Otherwise, we apply the rule in line 17.
\texttt{buildpivot} will first split the list using the pivot \texttt{X} using rules in
lines 23-26. When there is nothing more to split, we apply the rule in lines 19-21
that uses an exist construct to create nodes \texttt{B} and \texttt{C}. The sublists
are then sent to these nodes using \texttt{down} facts. Note, however, that we also
derive \texttt{back} facts, that will be used to send the sorted list back using the rule
in line 40.

When the sublists are finally sorted, we get two \texttt{sorted} facts that will match
against \texttt{waitpivot} in the rule located in lines 28-31. The sorted sublists
are appended and then an \texttt{up} fact is finally derived (line 37).

\begin{figure}[h!]
\small\begin{Verbatim}[numbers=left]
type route back(node, node).
type linear down(node, list int).
type linear up(node, list int).
type linear sorted(node, node, list int).
type linear buildpivot(node, list int, int, list int, list int).
type linear waitpivot(node, node, node, int).
type linear append(node, list int, list int).
type linear reverse(node, list int, list int, list int).
type linear reverse2(node, list int, list int).

down(@0, tosort).

down(A, []) -o up(A, []).
down(A, [X]) -o up(A, [X]).
down(A, [X, Y]), X < Y -o up(A, [X, Y]).
down(A, [X, Y]), X >= Y -o up(A, [Y, X]).
down(A, [X | L]) -o buildpivot(A, L, X, [], []).

buildpivot(A, [], X, Smaller, Greater)
   -o exists B, C. (back(B, A), down(B, Smaller),
            back(C, A), down(C, Greater), waitpivot(A, B, C, X)).

buildpivot(A, [Y | L], X, Smaller, Greater), Y <= X
   -o buildpivot(A, L, X, [Y | Smaller], Greater).
buildpivot(A, [Y | L], X, Smaller, Greater), Y > X
   -o buildpivot(A, L, X, Smaller, [Y | Greater]).
   
waitpivot(A, NodeSmaller, NodeGreater, Pivot),
sorted(A, NodeSmaller, Smaller),
sorted(A, NodeGreater, Greater)
   -o append(A, Smaller, [Pivot | Greater]).

append(A, L1, L2) -o reverse(A, L1, L2, []).

reverse(A, [], L2, L3) -o reverse2(A, L3, L2).
reverse(A, [X | L], L2, L3) -o reverse(A, L, L2, [X | L3]).
reverse2(A, [], Result) -o up(A, Result).
reverse2(A, [X | L1], L2) -o reverse2(A, L1, [X | L2]).

up(A, L), back(A, B) -o sorted(B, A, L).
\end{Verbatim}
  \caption{Quick-Sort program.}
  \label{code:quicksort}
\end{figure}
\normalsize




\section{Related Work}\label{section:language:related}
\subsection{Graph-Based Programming Models}

Many programming systems have designed for writing graph-based programs.  Some
good examples are the Dryad, Pregel, GraphLab, Ligra, Grace, Galois and
SociaLite systems.

The Dryad system~\cite{Isard:2007:DDD:1272996.1273005} combines computational
vertices with communication channels (edges) to form a data-flow graph. The
program is scheduled to run on multiple processors or cores and data is
partitioned during runtime. Routines that run on computational vertices are
sequential, with no synchronization. Dryad is better suited for a wider range of
problem domains than LM, however Dryad is not a language but a framework upon
which languages such as LM could be built.

The Pregel system~\cite{Malewicz:2010:PSL:1807167.1807184} is also graph based,
although programs have a more strict structure. They must be represented as a
sequence of iterations where each iteration is composed of computation and
message passing. Pregel is especially suited for large graphs since it aims to
scale to large architectures. LM does not impose iteration restrictions on
programs, however it is not as well suited for large graphs as Pregel.

GraphLab~\cite{GraphLab2010} is a C++ framework for developing parallel machine
learning algorithms. While Pregel uses message passing, GraphLab allows nodes to
have read/write access to different scopes through different concurrent access
models in order to balance performance and data consistency. While some programs
only require access to the local node's data, others may need to update edge
information. Each consistency model will provide different guarantees that are
better adapted to some algorithms. GraphLab also provides different schedulers
that dictate the order in which nodes are computed. The downside of GraphLab is
that GraphLab has a steep learning curve and programs must be written in C++,
requiring a lot of boilerplate code. LM programs tend to be smaller and easier to
reason about since they are written at a higher abstraction level.

Ligra~\cite{Shun:2013:LLG:2517327.2442530} is a lightweight framework for large
scale graph processing on a single multicore machine. Ligra exploits the fact
that most huge graph datasets available today can be made to fit in the main
memory of commodity servers. Ligra is a simple framework that exposes two main
interfaces: \texttt{EdgeMap} and \texttt{VertexMap}. The former applies a
function to a subset of edges of the graph, while the latter applies a function
to a subset of vertices. The functions passed as arguments are applied to either
a single edge or a single vertex and the user must ensure that the function can
be executed in parallel. The framework allows the use of
\emph{Compare-and-Swap}~(CAS) instructions when implementing functions in order
to avoid race conditions.

Grace~\cite{wang:asynchronous} is another graph-based framework for multicore
machines. Unlike Ligra, Grace programs are implemented from the point of view of
a vertex. Each vertex and edge can be customized with different data depending
on the target application. By default, programs are executed iteratively: for
each iteration the vertex program reads incoming edge messages, performs
computation and sends messages to the outbound edges. Since iterative programs
require synchronization after each iteration, Grace allows the user to relax
these constraints and implement customizable execution policies by implementing
code for describing which vertices are allowed to run and in which order. The
order is dictated by assigning a \emph{scheduling priority value}.

Galois~\cite{Pingali:2011:TPA:1993316.1993501} is a parallel programming model
with irregular applications based on graphs, trees and sets. A Galois parallel
algorithm is viewed as a parallel application of an \emph{operator} over an
irregular data structure which generate \emph{activities} on the data structure.
Such operator may, for instance, be applied to the node of the graph in order to
change its data or change the structure of its neighborhood, allowing for data
structure changes. In Galois, \emph{active elements} are nodes of the graph
where computation needs to be performed. Each operator performed on an active
element needs to acquire the \emph{neighborhood}, which are the nodes connected
to the active element that are involved in the operator's computation.
Furthermore, operators are selected for execution according to some specific
\emph{ordering}. From the point of view of the programmer, the active elements
are represented in a work-list, while operators can be implemented on top of
work-list's iterators. Galois supports speculative execution by allowing
operator rollback when an operator requires a node that is held by another
operator.

Ligra, Grace and Galois are not programming language but frameworks built on top
of other programming languages such as C and Java. Naturally, programs written
in these frameworks need to take into account several implementation details
such as \emph{compare-and-set} instructions, work-lists, and scheduling order.
LM programs are more abstract since reasoning is performed around logical rules
and logical facts which are much closer to the problem domain of graph-based
programs.

SociaLite~\cite{Seo:2013:DSD:2556549.2556572} is a Datalog-based language for
writing algorithms for social-network analysis. SociaLite takes Datalog programs
and compiles them to efficient distributed Java code. The language places some
restrictions on rules in order to make distribution possible, however, the
programmer is free to tell the system how to shard database's facts. Like LM,
these restrictions deal with the first argument of each fact, however, the first
argument is not related to a graph abstraction but is instead related to fact
distribution, allowing programmers to optimize how messages are sent between
machines.

\subsection{Sensor Network Programming Languages}

Many programming languages have been designed to help developers write programs
that run on sensor networks, which are networks that can be represented as
graphs.  Programming languages such as
Hood~\cite{Whitehouse:2004:HNA:990064.990079},
Tinydb~\cite{Madden:2005:TAQ:1061318.1061322} or
Regiment~\cite{Newton:2007:RMS:1236360.1236422} have been proposed, providing
support for data collection and aggregation over the network. These systems
assume that the network remains static and nodes stay in place.

Other languages such as Pleiades~\cite{Kothari:2007:REP:1250734.1250757},
LDP~\cite{4543691} or Proto~\cite{Beal:2006:IEE:1137236.1137354} go beyond
static networks and support dynamic reconfiguration. In Pleiades, the
programmer writes an application from the point of view of the whole
sensor network and the compiler transforms it into code that can be run on
each individual node.  LDP is a language derived from a method for
distributed debugging, that allows it to efficiently detect conditions on
variably-sized groups of nodes. It is based on the tick model, generating
a new set of condition matchers throughout the ensemble on each tick.
Like Pleiades and LDP, Proto also compiles global programs into locally
executed code.

Finally, the original Meld~\cite{ashley-rollman-iclp09} is also a language
designed for dynamic networks, namely, ensembles of robots. As we have seen
before, Meld is a logic programming language where programs are sets of logical
rules that infer over the state of the ensemble. Meld supports action facts and
sensing facts, which allow the robots to act on the world or sense the world,
respectively. Like Pleiades programs, Meld programs are also written from the
point of the view of the whole ensemble.

\subsection{Constraint Handling Rules}

Since LM is a bottom-up linear logic programming language, it also shares
similarities with Constraint Handling Rules
(CHR)~\cite{Betz:2005kx,Betz:2013:LBA:2422085.2422086}.  CHR is a concurrent
committed-choice constraint language used to write constraint solvers. A CHR
program is a set of rules and a set of constraints. Constraints can be consumed
or generated during the application of rules. Unlike LM, in CHR there is no
concept of rule priorities, but there is an extension to CHR that supports
them~\cite{DeKoninck:2007:URP:1273920.1273924}. Finally, there is also a CHR
extension that adds persistent constraints and it has been proven to be sound
and complete~\cite{DBLP:journals/corr/abs-1007-3829} in relation to the standard
formulation of CHR.

\subsection{Graph Transformation Systems}

Graph Transformation Systems (GTS)~\cite{Ehrig:2004vn}, commonly used to model
distributed systems, perform rewriting of graphs through a set of graph
productions. GTS also introduces concepts of concurrency, where it may be
possible to apply several transformations at the same time. In principle, it
should be possible to model LM programs as graph transformations: we directly
map the LM graph of nodes to GTS's initial graph and consider logical facts as
nodes that are connected to LM's nodes. Each LM rule is then a graph production
that manipulates the node's neighbors (the database) or sends new facts to other
nodes. On the other hand, it is also possible to embed GTS inside
CHR~\cite{Raiser:2011:AGT:1972935.1972938}.

\section{Chapter Summary}

In this chapter we gave an overview of the LM language, including its syntax and
operational semantics.  We also explained how to write programs using all the
facilities provided by LM, including linear facts and comprehensions. We also
explained how to informally prove the correctness of several LM programs.  The implicit
parallelism of LM arising from database partitioning and rule restriction allows
programs to be run in a concurrent fashion.
