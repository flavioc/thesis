
In addition to the node locks, namely the \emph{Node Lock} and \emph{Db Lock},
which were described in Section~\ref{sec:data_structures}, we also have the
\emph{Queue Lock}, which is a per queue lock used to protect the work queue from
simultaneous access. Note that all those locks are implemented as ticket
spin-locks.

We summarize the synchronization hotspots of the VM by describing how the locks
are used in order to implement several operations:

\begin{itemize}

   \item \textbf{New Facts}: We use the node's \emph{Main Lock} and then attempt
      to lock the \emph{DB Lock}. If the \emph{Db Lock} cannot be used, then the
      new facts are added to the \emph{Fact Buffer}, otherwise the node data
      structure is updated with new facts. If the target node is not currently
      in any work queue, we lock the destination work queue and then add the
      node and change the state of the node to \textbf{active}. Finally, if the
      target thread that owns the target node is \textbf{idle}, we activate it
      by updating its state flag to \textbf{active}.

   \item \textbf{Node Stealing}: For node stealing, we acquire the lock of the
      target thread's queue and then copy the stolen node pointers to a temporary
      buffer. For each node, we use the \emph{Main Lock} to update its
      \emph{Owner} attribute and then add it to the thread's work queue.

   \item \textbf{Next Node}: When a thread is \textbf{active} state, it locks
      its work queue by acquiring the \emph{Queue Lock} and then removing the
      first from the queue. The \emph{Main Lock} is then acquired momentarily in
      order to update the node state flag.

   \item \textbf{Node Computation}: The \emph{DB Lock} is acquired before
      any rule is executed on the node since node computation manipulates the
      database. The lock is released when all candidate rules are executed.

   \item \textbf{Node Completion}: Once node computation is completed and all
      candidate rules have been executed, the \emph{Main Lock} is acquired in
      order to change the state flag. Note that if newer facts have been derived
      by other nodes, computation is resumed on the current node instead of
      using another node from the work queue.

\end{itemize}

In order to manipulate the state flag of each thread (see
Section~\ref{sec:implementation:parallelism}) we do not use locks but instead
manipulate the state flag using lock-free \emph{compare-and-swap} operations to
implement a state machine.

