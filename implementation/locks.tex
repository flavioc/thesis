
We now summarize the synchronization hotspots of the VM by describing how the locks
are used in order to implement different operations:

\begin{itemize}

   \item \textbf{New Facts}: We use the node's \emph{State Lock} and then attempt
      to lock the \emph{DB Lock}. If the \emph{Db Lock} cannot be used, then the
      new facts are added to the \emph{Incoming Fact Buffer}, otherwise the node database
      structures are updated with new facts. If the target node is not currently
      in any work queue, we lock the destination work queue and then add the
      node and change the state of the node to \textbf{active}. Finally, if the
      target thread that owns the target node is \textbf{idle}, we activate it
      by updating its state flag to \textbf{active}. The complete pseudo-code
      for this operation is shown in Fig.~\ref{alg:multicore:new_fact}.

\begin{figure}
\begin{algorithm}[H]
   \KwData{Target Node N, Fact F}
   \eIf{$N.db\_lock.try\_lock()$}{
      \tcc{New Fact (1)}
      $N.DB.add\_fact(F)$\;
      $N.rule\_engine.new\_fact(F)$\;
      $N.db\_lock.unlock()$\;
      \tcc{Still need to check if node is active}
      $N.state\_lock.lock()$\;
   }{
      \tcc{New Fact (2)}
      $N.state\_lock.lock()$\;
      $N.FactBuffer.add\_fact(F)$\;
   }
   \tcc{Activate node}
   $TTH \longleftarrow N.Owner$\;
   \If{$N.state == inactive$}{
      $TTH.work\_queue.lock()$\;
      $TTH.work\_queue.push(N)$\;
      $TTH.work\_queue.unlock()$\;
      $N.state \longleftarrow active$\;
      \If{$TTH.State == idle$}{
         $TTH.become\_active()$\;
      }
   }
   $N.state\_lock.unlock()$\;
\end{algorithm}
\caption{Synchronization code for sending a fact to another node.}
 \label{alg:multicore:new_fact}
\end{figure}


   \item \textbf{Node Stealing}: For node stealing, we acquire the lock of the
      target thread's queue and then copy the stolen node pointers to a
      temporary buffer. For each node, we use the \emph{State Lock} to update its
      \emph{Owner} attribute and then add it to the thread's work queue.
      Figure~\ref{alg:multicore:steal_nodes} presents the pseudo-code for this
      synchronization mechanism. Note that when using $pop\_half(stealing)$, the
      \emph{Work Queue} will (unsafely) change the state of the nodes to
      \textbf{stealing}. When adding the new functionality presented in
      Chapter~\ref{chapter:coordination}, the node stealing loop in the
      pseudo-code must have an extra check for cases when the node's owner
      is updated using coordination facts.

\begin{figure}
\begin{algorithm}[H]
   \KwData{Source Thread TH, Target Thread TTH}
   $TTH.work\_queue.lock()$\;
   $nodes \longleftarrow TTH.work\_queue.pop\_half(stealing)$\;
   $TTH.work\_queue.unlock()$\;
   \For{$node \; in \; nodes$}{
      $node.state\_lock.lock()$\;
      \If{$node.state != stealing$}{
         \tcc{Node was put back into the queue, therefore we give up}
         $node.state\_lock.unlock()$\;
         continue \\
      }
      $node.owner \longleftarrow TH$\;
      $node.state \longleftarrow active$\;
      $node.state\_lock.unlock()$\;
   }
   $TH.work\_queue.push(nodes)$\;
\end{algorithm}
\caption{Synchronization code for sending a fact to another node.}
 \label{alg:multicore:steal_nodes}
\end{figure}

   \item \textbf{Node Computation}: The \emph{DB Lock} is acquired before any
      rule is executed on the node since node computation manipulates the
      database. The lock is released when all candidate rules are executed. The
      initial pseudo-code for the \code{process\_node} procedure in
      Fig.~\ref{alg:multicore:process_node} shows this synchronization protocol.

   \item \textbf{Node Completion}: Once node computation is completed and all
      candidate rules have been executed, the \emph{State Lock} is acquired in
      order to change the state flag. Note that if newer facts have been derived
      by other nodes, computation is resumed on the current node instead of
      using another node from the \emph{Work Queue}. The final section of
      \code{process\_node} in Fig.~\ref{alg:multicore:process_node} shows this
      synchronization protocol.

\end{itemize}

All the locks in the VM are implemented using Mellor-Crummey and Scott~(MCS)
spinlocks~\cite{Mellor-Crummey:1991}, which is a fair and contention-free
spinlock implementation that uses a FIFO queue to implement spinlock operations.

In order to manipulate the \emph{State} flag of each thread (see
Section~\ref{sec:implementation:parallelism}) we do not use locks but instead
manipulate the state flag using lock-free \emph{compare-and-swap} operations to
implement a state machine.

