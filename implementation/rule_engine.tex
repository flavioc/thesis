The rule engine decides which rules may need to be executed while taking into
account rule priorities. The rule engine is composed of 3 data structures:

\begin{itemize}

   \item \emph{Rule Queue} is a bitmap representing the rules scheduled to run.
      When the bit $i$ in the \emph{Rule Queue} is set, it means that the rule
      $i$ is scheduled to run. A rule to be made to run depends on fact
      availability (we have all the facts required to run a rule $i$). Rules are
      executed by order by fetching the least significant bit of the bitmap,
      unsetting that bit and then executing the corresponding rule. This
      operation is accomplished by using the \emph{bit scan forward (bsf)}
      assembly instruction available on x86/x86-64 machines;

   \item \emph{Rule Counter} is a counter of the number of predicates that exist
      in the database and that are needed by the rule. For example, a rule
      \code{a, e(1) -o b} needs predicates \code{a} and \code{e} and the
      rule counter is thus at the most 2 (where the rule can be executed). The
      \emph{Rule Queue} bitmap is updated when the rule counter
      increments/decrements to/from the maximum possible value;

   \item \emph{Predicate Bitmap} is a bitmap representing the existence of facts
      for a given predicate in the database. When a predicate becomes available
      then the \emph{Rule Counter} is updated to take into account the existence
      of new facts.

\end{itemize}


\begin{figure}[t]
   \begin{center}
\begin{BVerbatim}[numbers=left,fontsize=\codesize]
a, e(1) -o b.  // Rule 1.
a -o c.        // Rule 2.
b -o d.        // Rule 3.
e(0) -o f.     // Rule 4.
c -o e(1).     // Rule 5.

a.
e(0).
\end{BVerbatim}
\end{center}
\vspace{5mm}
   \includegraphics[width=0.96\textwidth]{figures/implementation/rule_queue.pdf}

   \mycap{Example program and corresponding rule engine data structures. The
      initial state is represented in (a), where the rules scheduled to run are
      1, 2 and 4. After attempting rule 1, bit 0 is unset from the \emph{Rule
      Queue}, resulting in (b). Figure (c) is the result of applying rule 2,
      \code{a -o c}, which marks rule 5 in the \emph{Rule Queue} since the rule
      is now \emph{available} in the \emph{Rule Counter}.}

   \label{fig:implementation:rule_engine}
\end{figure}

To better understand how our rule engine works,
Fig.~\ref{fig:implementation:rule_engine} shows an example program and the
corresponding rule engine data structures. Since we have facts for predicates
\code{a} and \code{e}, the \code{Rule Counter} starts with rules 1, 2 and 4 with
2, 1, and 1 predicate counts. Since these rules have the required counter to be
applied, the \emph{Rule Queue} bitmap starts with the same three rules
(Fig.~\ref{fig:implementation:rule_engine}(a)). In order to pick rules for
execution, we take the rule corresponding to the least significant bit from the
\emph{Rule Queue} bitmap, initially the first rule \code{a, e(1) -o b}.
However, since we don't have fact \code{e(1)}, this rule fails and its bit in
\emph{Rule Queue} must be set to 0.
Figure~\ref{fig:implementation:rule_engine}(b) shows the rule engine data
structures at that point.

The next rule in \emph{Rule Queue} is the second rule \code{a -o c}.  Because
this rule succeeds, fact \code{a} is consumed and fact \code{c} is derived.  We
thus update \code{Predicates Bitmap} accordingly, and decrease the counters for
the first and second rules in \emph{Rule Counter} since such rules are no longer
applicable (\code{a} was consumed), and increase the counter for the fifth rule
since \code{c} was derived. Finally, to update the \emph{Rule Queue}, we must
schedule the fifth rule since its counter has been increased to the required
number (we have all predicates).  Figure~\ref{fig:implementation:rule_engine}(b)
shows the rule engine data structures at that point.  In the continuation, the
rule engine will schedule the fourth and fifth rules to run.

Note that every node in the program has the same set of data structures
presented in Fig.~\ref{fig:implementation:rule_engine}. We use 64 bit integers
to implement the 2 bitmaps and an array of 16 bits integers for the \code{Rule
Counter}.

For persistent facts, we do a small optimization to reduce the number of
derivations and, for that, we divide the program rules into two sets:
\emph{persistent rules} and \emph{non persistent rules}. Persistent rules are
rules where only persistent facts are involved. We compile such rules
incrementally, i.e., we attempt to fire all rules when a new persistent fact is
derived. This is called the \emph{pipelined semi-naive} evaluation and it
originated in the P2 system~\cite{Loo-condie-garofalakis-p2}. This evaluation
method avoids excessive re-derivations of the same fact. The order of derivation
does not matter for those rules, since only persistent facts are used.

