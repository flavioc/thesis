\subsection{Virtual Machines}

Virtual machines are a popular technique for implementing interpreters for high
level programming languages. Due to the increased availability of parallel
and distributed architectures, several machines have been developed with
parallelism in mind~\cite{Kara:1997:AMM:265274}. One good example is
the Parallel Virtual Machine~(PVM)~\cite{Sunderam90pvm:a}, which servers as an
abstraction to program heterogeneous computers as a single machine. Another
important machine is the Threaded Abstract
Machine~(TAM)~\cite{CullerGSvE93,goldstein-tr94}, which defines a self-scheduled
machine language of parallel threads where a program is represented using
conventional control flow.

Prolog, the most prominent logic programming language, has a rich history of
virtual machine research centered extending/adapting the Warren Abstract
Machine~(WAM)~\cite{AICPub641:1983}.  Because Prolog programs tend to be
naturally parallel, much research has been done to either parallelize the WAM or
create new parallel machines. Two types of parallelism are possible in Prolog:
\emph{OR-parallelism}, where several clauses for the same goal can be executed,
and \emph{AND-parallelism}, where goals in the same clause are tried in
parallel. For OR-parallelism, we have several models such as: the SRI
model~\cite{Warren:1987:OEM:67683.67699}, the Argonne
model~\cite{ButlerDLOOS88}, the MUSE model~\cite{Ali:1990fk} and the BC
machine~\cite{Ali88}. For AND-parallelism, different implementations were built
on top of the WAM~\cite{Hermenegildo:1986:AMB:913061,Lin:1988:AEL:900478},
however more general models such as the Andorra
model~\cite{Haridi:1990:KAP:87961.87964} were developed which allows both AND
and OR-parallelism.

\subsection{Datalog Execution}

The Datalog language is a forwards-chaining logic
programming and therefore requires different evaluation strategies than those
used by Prolog, which is a backwards-chaining logic programming language.
Arguably, the most well-known strategy for Datalog programs with recursive rules
is the \emph{Semi-Na\"{\i}ve Fixpoint}
algorithm~\cite{Balbin:1987:GDA:34657.34661}, where the computation is split
into iterations and the facts generated in the previous generation are used as
inputs to derive the facts in the next iteration. The advantage of this
mechanism is that no redundant computations are performed, which could happen in
the case of recursive rules but is avoided when the next iteration only uses new
facts derived by the previous iteration.
   
In the context of the P2 system~\cite{Loo-condie-garofalakis-p2}, which was
created for developing declarative networking programs, the previous strategy is
not suitable since it is centralized, therefore a new strategy, called
\emph{Pipelined Semi-Na\"{\i}ve}~(PSN) evaluation, was developed for distributed
computation. PSN evaluation evaluates one fact at the time by firing any rule
that is derivable using the new fact and older facts. If the new fact is already
in the database, then the fact is not used for derivation since it would derive
redundant facts.

LM's evaluation strategy is similar to PSN, however, it considers the whole
database of facts when firing rules due to the existence of rule priorities.
For instance, when a node fires a rule, new facts added for the local node are
considered as a whole in order to select the next inference rule. In the case of
PSN, each new fact would be considered separately. Still, PSN and LM are both
asynchronous strategies which take advantage of the nature of distributed and
parallel architectures, respectively. Finally, an important distinction that
should be made between PSN and LM is that PSN is for logic programs with only
persistent facts, which results in deterministic results, however because LM
uses linear facts, it follows a \emph{don't care} or \emph{committed choice}
non-determinism, which may result in different results depending on the order of
computations.

\subsection{CHR implementations}

Many basic optimizations used in the LM compiler such as join optimizations and
the use of different data structures for indexing facts were inspired in work
done on CHR~\cite{DBLP:journals/corr/cs-PL-0408025}.  Wuille et al.~\cite{42866}
have described a CHR to C compiler that follows some of the ideas presented in
this chapter and De Koninck et al.~\cite{chrp} showed how to compile CHR
programs with dynamic priorities into Prolog. The novelty of our work focuses on
supporting a combination of comprehensions, aggregates and rule priorities.

