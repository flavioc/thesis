We now describe the linear logic fragment used as a basis for LM.  Note that in
this thesis we follow the intuitionistic approach and use the sequent
calculus~\cite{gen35} to specify the logic. Our initial sequent is written as
$\Psi; \seqx{\Gamma}{\Delta}{C}$ and can be read as "assuming persistent
resources $\Gamma$ and linear resources $\Delta$ then $C$ is true".  More
specifically, $\Psi$ is the typing context which contains unique variables,
$\Gamma$ is a multi-set of persistent resources, $\Delta$ is a multi-set of
linear resources while $C$ is the proposition we want to prove. The sequent can
also be decomposed into a \emph{succedent} ($C$) and the \emph{antecedents}
(contexts that appear before $\seq$).

We now present the connectives and their associated rules for the linear logic fragment.
First, we have the \emph{simultaneous conjunction} $A \otimes B$ that packages
linear resources together. In the right rule, $A \otimes B$ is true if both $A$
and $B$ are true, and, in the left rule, it is possible to split $A \otimes B$
apart.


\[
\infer[\otimes R]
{\Psi ; \seqx{\Gamma}{\Delta, \Delta'}{A \otimes B}}
{\Psi ; \seqx{\Gamma}{\Delta}{A} & \Psi ; \seqx{\Gamma}{\Delta}{B}}
\tab
\infer[\otimes L]
{\Psi ; \seqx{\Gamma}{\Delta, A \otimes B}{C}}
{\Psi ; \seqx{\Gamma}{\Delta, A, B}{C}}
\]



Note that the inference rules above can be decomposed into premises (the
sequents above the separator line) and conclusion (the sequent below the line).

Next, we have the \emph{additive conjunction} $A \with B$ that allows us to
select between $A$ or $B$. In the right rule we must prove $A$ and $B$ using
the same resources, while in the left rule, we can select one of the
resources.

\[
   \infer[\with L_1]
   {\Psi; \seqx{\Gamma}{\Delta, A \with B}{C}}
   {\Psi; \seqx{\Gamma}{\Delta, A}{C}}
   \tab
   \infer[\with L_2]
   {\Psi; \seqx{\Gamma}{\Delta, A \with B}{C}}
   {\Psi; \seqx{\Gamma}{\Delta, B}{C}}
   \tab
   \infer[\with R]
   {\Psi; \seqx{\Gamma}{\Delta}{A \with B}}
   {\Psi; \seqx{\Gamma}{\Delta}{A} & \Psi; \seqx{\Gamma}{\Delta}{B}}
\]



To express inference, we introduce the \emph{linear implication} connective
written as $A \lolli B$. For the right rule, we prove $A \lolli B$ by assuming
$A$ and then proving $B$, while in the left rule, we obtain $B$ by using some
linear resources to prove $A$.


\[
\infer[\lolli R]
{\Psi ; \seqx{\Gamma}{\Delta}{A \lolli B}}
{\Psi ; \seqx{\Gamma}{\Delta, A}{B}}
\tab
\infer[\lolli L]
{\Psi; \seqx{\Gamma}{\Delta, \Delta', A \lolli B}{C}}
{\Psi ; \seqx{\Gamma}{\Delta}{A} &
   \Psi ; \seqx{\Gamma}{\Delta', B}{C}}
\]


Next, we introduce persistent resources written as $\bang A$. For the right
rule, we prove $\bang A$ by proving it without any linear resources. Likewise,
to use a persistent resource, we simply drop the $
\bang$. There is also a $\m{copy}$ rule that moves persistent resources from
$\Gamma$ to $\Delta$. Remember that $\Gamma$ contains persistent resources.


\[
\infer[\bang R]
{\Psi ; \seqx{\Gamma}{\cdot}{\bang A}}
{\Psi ; \seqx{\Gamma}{\cdot}{A}}
\tab
\infer[\bang L]
{\Psi ; \seqx{\Gamma}{\Delta, \bang A}{C}}
{\Psi ; \seqx{\Gamma, A}{\Delta}{C}}
\tab
\infer[\m{copy}]
{\Psi ; \seqx{\Gamma, A}{\Delta}{C}}
{\Psi ; \seqx{\Gamma, A}{\Delta, A}{C}}
\]


Another useful connective is the \emph{multiplicative unit} of the $\otimes$
connective. It is written as $\one$ and is best understood as something that
does not need any resource to be proven.


\[
\infer[\one R]
{\Psi ; \seqx{\Gamma}{\cdot}{\one}}
{}
\tab
\infer[\one L]
{\Psi ; \seqx{\Gamma}{\Delta, \one}{C}}
{\Psi ; \seqx{\Gamma}{\Delta}{C}}
\]


Next, we introduce the \emph{quantification} connectives, namely \emph{universal
quantification} $\forall_{n:\tau}. A$ and \emph{existential quantification}
$\exists_{n:\tau}. A$ ($n:\tau$ means that $n$ has type $\tau$).
These connectives use the typing context $\Psi$ to introduce and read term variables.
The right and left rules of those two connectives are dual.

\[
\infer[\forall R]
{\Psi ; \seqx{\Gamma}{\Delta}{\forall_{n:\tau}. A}}
{\Psi, m:\tau ; \seqx{\Gamma}{\Delta}{A\{m/n\}}}
\tab
\infer[\forall L]
{\Psi ; \seqx{\Gamma}{\Delta, \forall_{n:\tau}. A}{C}}
{\Psi \vdash M : \tau & \Psi ; \seqx{\Gamma}{\Delta, A\{M/n\}}{C}}
\]

\[
\infer[\exists R]
{\Psi ; \seqx{\Gamma}{\Delta}{\exists_{n: \tau}. A}}
{\Psi \vdash M : \tau &
   \Psi ; \seqx{\Gamma}{\Delta}{A \{M/n\}}}
\tab
\infer[\exists L]
{\Psi ; \seqx{\Gamma}{\Delta, \exists_{n:\tau}. A}{C}}
{\Psi, m:\tau ; \seqx{\Gamma}{\Delta, A\{m/n\}}{C}}
\]


The judgment $\Psi \vdash M : \tau$ introduces a new term $M$ with type $\tau$
that does not depend on $\Gamma$ or $\Delta$ but may depend on the variables in
$\Psi$. In rules $\forall R$ and $\exists L$, the new $m$ variable introduced in
$\Psi$ must always be \emph{fresh}. We complete the linear logic system with the
\emph{cut rules} and the \emph{identity rule}:

\[
   \infer[cut_A]
   {\Psi; \seqx{\Gamma}{\Delta, \Delta'}{C}}
   {\Psi; \seqx {\Gamma}{\Delta}{A} & \Psi ; \seqx{\Gamma}{\Delta', A}{C}}
   \tab
   \infer[cut\bang_A]
   {\Psi; \seqx{\Gamma}{\Delta}{C}}
   {\Psi; \seqx{\Gamma}{\cdot}{A} & \Psi; \seqx{\Gamma, A}{\Delta}{C}}
\]

\[
\infer[id_{A}]
{\Psi ; \seqx{\Gamma}{A}{A}}
{}
\]

