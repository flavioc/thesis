Logic, as classically understood, treats true propositions as \emph{persistent
truth}. When a persistent proposition is needed to prove other propositions, it
can be reused as many times as we wish because it is true indefinitely. This is
also true in the constructive or intuitionistic school of logic.  Linear logic
is a \emph{substructural logic} (lacks weakening and contraction) developed by
Girard~\cite{Girard95logic:its} that extends persistent logic with linear
propositions which can be understood as ephemeral resources that can be used
only once to prove other propositions.  Due to the resource interpretation of
the logic, linear logic presents a good basis for programming languages with a
structured way to manage state~\cite{Miller85anoverview}. Beyond programming
languages, linear logic has also been used for game
semantics~\cite{lafont1991games,blass1992game}, concurrent
programming~\cite{lopez2005monadic,mazurak2010lolliproc,perez-2012}, knowledge
representation~\cite{bos2011survey}, or narrative
generation~\cite{chris-int7,martens2013linear}.

In the context of the Curry-Howard correspondence~\cite{howard:tfatnoc}, linear
logic has been applied in programming languages as a mechanism to implement
\emph{linear types}. Linear types are types that force objects to be used
exactly once. Surprisingly, such types add mutable state to functional languages
because they enforce a linear view of state, allowing the language to naturally
support concurrency, input/output and data structure's updates.  Arguably, the
most popular language that features uniqueness types is the Clean programming
language~\cite{JFP:1349748}.  Monads~\cite{Wadler:1997:DI:262009.262011}, made
popular with the Haskell programming language, are another interesting way to
add state to functional languages. While monads tend to be more powerful than
linear types, they also ensure equational reasoning in the presence of mutable
data structures and I/O effects.

Linear logic programming is a different approach than either monads or linear
types.  While the latter are mechanisms that enhance functional programming with
state, the former uses state as a foundation, since the program's database is
both the state and the program, because it drives the computation forward
through rule application.

Traditional forward-chaining logic programming languages like
Datalog only use persistent logic, however many ad-hoc
extensions~\cite{Liu98extendingdatalog,Ludascher95alogical} have been devised
in to support state updates, but most are extra-logical which makes it harder
to reason about programs. LM uses linear logic as its foundation, therefore
state updates are natural to the language.

In linear logic, truth is treated as a resource that is consumed once used. For
instance, in the graph visit program in Fig.~\ref{code:language:visit}, the
\texttt{unvisited(A)} and \texttt{visit(A)} linear facts are consumed in order
to prove \texttt{visit(A)}. If those facts were persistent, then the rule would
make no sense, because the node would be \texttt{visited} and \texttt{unvisited}
at the same time!

\subsection{Sequent Calculus}

We now describe the linear logic fragment used as a basis for LM.  Note that in
this thesis we follow the intuitionistic approach and use the sequent
calculus~\cite{gen35} to specify the logic. Our initial sequent is written as
$\Psi; \seqx{\Gamma}{\Delta}{C}$ and can be read as "assuming persistent
resources $\Gamma$ and linear resources $\Delta$ then $C$ is true".  More
specifically, $\Psi$ is the typing context which contains unique variables,
$\Gamma$ is a multi-set of persistent resources, $\Delta$ is a multi-set of
linear resources while $C$ is the proposition we want to prove.

We first have the \emph{simultaneous conjunction} $A \otimes B$ that packages
linear resources together. In the right rule, $A \otimes B$ is true if both $A$
and $B$ are true, and, in the left rule, it is possible to split $A \otimes B$
apart.


\[
\infer[\otimes R]
{\Psi ; \seqx{\Gamma}{\Delta, \Delta'}{A \otimes B}}
{\Psi ; \seqx{\Gamma}{\Delta}{A} & \Psi ; \seqx{\Gamma}{\Delta}{B}}
\tab
\infer[\otimes L]
{\Psi ; \seqx{\Gamma}{\Delta, A \otimes B}{C}}
{\Psi ; \seqx{\Gamma}{\Delta, A, B}{C}}
\]



Next, we have the \emph{additive conjunction} $A \with B$ that allows us to
select between $A$ or $B$. In the right rule we must prove $A$ and $B$ using
the same resources, while in the left rule, we can select one of the
resources.

\[
   \infer[\with L_1]
   {\Psi; \seqx{\Gamma}{\Delta, A \with B}{C}}
   {\Psi; \seqx{\Gamma}{\Delta, A}{C}}
   \tab
   \infer[\with L_2]
   {\Psi; \seqx{\Gamma}{\Delta, A \with B}{C}}
   {\Psi; \seqx{\Gamma}{\Delta, B}{C}}
   \tab
   \infer[\with R]
   {\Psi; \seqx{\Gamma}{\Delta}{A \with B}}
   {\Psi; \seqx{\Gamma}{\Delta}{A} & \Psi; \seqx{\Gamma}{\Delta}{B}}
\]



To express inference, we introduce the \emph{linear implication} connective
written as $A \lolli B$. For the right rule, we prove $A \lolli B$ by assuming
$A$ and then proving $B$, while in the left rule, we obtain $B$ by using some
linear resources to prove $A$.


\[
\infer[\lolli R]
{\Psi ; \seqx{\Gamma}{\Delta}{A \lolli B}}
{\Psi ; \seqx{\Gamma}{\Delta, A}{B}}
\tab
\infer[\lolli L]
{\Psi; \seqx{\Gamma}{\Delta, \Delta', A \lolli B}{C}}
{\Psi ; \seqx{\Gamma}{\Delta}{A} &
   \Psi ; \seqx{\Gamma}{\Delta', B}{C}}
\]


Next, we introduce persistent resources written as $\bang A$. For the right
rule, we prove $\bang A$ by proving it without any linear resources. Likewise,
to use a persistent resource, we simply drop the $
\bang$. There is also a $\m{copy}$ rule that moves persistent resources from
$\Gamma$ to $\Delta$. Remember that $\Gamma$ contains persistent resources.


\[
\infer[\bang R]
{\Psi ; \seqx{\Gamma}{\cdot}{\bang A}}
{\Psi ; \seqx{\Gamma}{\cdot}{A}}
\tab
\infer[\bang L]
{\Psi ; \seqx{\Gamma}{\Delta, \bang A}{C}}
{\Psi ; \seqx{\Gamma, A}{\Delta}{C}}
\tab
\infer[\m{copy}]
{\Psi ; \seqx{\Gamma, A}{\Delta}{C}}
{\Psi ; \seqx{\Gamma, A}{\Delta, A}{C}}
\]


Another useful connective is the \emph{multiplicative unit} of the $\otimes$
connective. It is written as $\one$ and is best understood as something that
does not need any resource to be proven.


\[
\infer[\one R]
{\Psi ; \seqx{\Gamma}{\cdot}{\one}}
{}
\tab
\infer[\one L]
{\Psi ; \seqx{\Gamma}{\Delta, \one}{C}}
{\Psi ; \seqx{\Gamma}{\Delta}{C}}
\]


Next, we introduce the \emph{quantification} connectives, namely \emph{universal
quantification} $\forall_{n:\tau}. A$ and \emph{existencial quantification}
$\exists_{n:\tau}. A$. These connectives use the typing context $\Psi$ because
they can introduce and read term variables from the context. The right and left duals of
those two connectives are dual.

\[
\infer[\forall R]
{\Psi ; \seqx{\Gamma}{\Delta}{\forall_{n:\tau}. A}}
{\Psi, m:\tau ; \seqx{\Gamma}{\Delta}{A\{m/n\}}}
\tab
\infer[\forall L]
{\Psi ; \seqx{\Gamma}{\Delta, \forall_{n:\tau}. A}{C}}
{\Psi \vdash M : \tau & \Psi ; \seqx{\Gamma}{\Delta, A\{M/n\}}{C}}
\]

\[
\infer[\exists R]
{\Psi ; \seqx{\Gamma}{\Delta}{\exists_{n: \tau}. A}}
{\Psi \vdash M : \tau &
   \Psi ; \seqx{\Gamma}{\Delta}{A \{M/n\}}}
\tab
\infer[\exists L]
{\Psi ; \seqx{\Gamma}{\Delta, \exists_{n:\tau}. A}{C}}
{\Psi, m:\tau ; \seqx{\Gamma}{\Delta, A\{m/n\}}{C}}
\]


The judgment $\Psi \vdash M : \tau$ introduces a new term $M$ with type $\tau$
that does not depend on $\Gamma$ or $\Delta$ but may depend on the variables in
$\Psi$. In rules $\forall R$ and $\exists L$, the new $m$ variable introduced in
$\Psi$ must always be \emph{fresh}. We complete the linear logic system with the
\emph{cut rules} and the \emph{identity rule}:

\[
   \infer[cut_A]
   {\Psi; \seqx{\Gamma}{\Delta, \Delta'}{C}}
   {\Psi; \seqx {\Gamma}{\Delta}{A} & \Psi ; \seqx{\Gamma}{\Delta', A}{C}}
   \tab
   \infer[cut\bang_A]
   {\Psi; \seqx{\Gamma}{\Delta}{C}}
   {\Psi; \seqx{\Gamma}{\cdot}{A} & \Psi; \seqx{\Gamma, A}{\Delta}{C}}
\]

\[
\infer[id_{A}]
{\Psi ; \seqx{\Gamma}{A}{A}}
{}
\]



\subsection{From The Sequent Calculus To LM}

\begin{table*}
\begin{center}
\resizebox{16cm}{!}{
    \begin{tabular}{ | l | l || l | l | l |}
    \hline
    Connective                   & Description                                      & LM Syntax                                  & LM Place     & LM Example                                  \\ \hline \hline
    $\emph{fact}(\hat{x})$       & Linear atomic propositions.                      & $fact(\hat{x})$                               & LHS or RHS    & \texttt{path(A, P)}                            \\ \hline
    $\bang \emph{fact}(\hat{x})$ & Persistent atomic propositions.                  & $\bang fact(\hat{x})$                         & LHS or RHS    & \texttt{$\bang$edge(X, Y, W)}                  \\ \hline
    $1$                          & Represents rules with an empty RHS.              & $1$                                           & RHS            & \texttt{1}                                     \\ \hline
    $A \otimes B$                & Connect two expressions.                         & $A, B$                                        & LHS and RHS   & \texttt{path(A, P), edge(A, B, W)}             \\ \hline
    $\forall x. A$               & To represent variables defined inside the rule.  & Please see $A \lolli B$                       & Rule            & \texttt{path(A, B) $\lolli$ reachable(A, B)}   \\ \hline
    $\exists x. A$               & Instantiates new node variables.  & $\existsc{\widehat{x}}{B}$                  & RHS            & \texttt{exists A.(path(A, P))}                 \\ \hline
    $A \lolli B$                 & $\lolli$ means "linearly implies".               & $A \lolli B$                                  & Rule            & \texttt{path(A, B) $\lolli$ reachable(A, B)}   \\
                                 & $A$ is the rule's LHS and $B$ is the RHS.             &                                               &                 &                                                \\ \hline
    $\bang C$                    & Constraint.                                      & $A = B$                     & LHS & \texttt{A = B} \\ \hline
                                 $\defstwo{comp}{\widehat{V}}{M}$               & For comprehensions
    ($M$ is not used).  & $\comprehension{\widehat{x}}{A}{B}$  & RHS & \texttt{\{B | !edge(A, B) -o visit(B)\}}        \\
    & For aggregates ($M$ accumulates). $\widehat{V}$ captures rule variables.          &                                               &                 &                                                \\ \hline
    \end{tabular}
}
\end{center}
\caption{Connectives from linear logic and their use in LM.}
\label{table:linear}
\end{table*}

The connections between LM and the sequent calculus fragment presented in the
Table~\ref{table:linear}. In the table, we show how each connective is
translated to LM's abstract syntax and then to LM programs. In order to
understand how LM rules are related to the sequent calculus, consider the first
rule of the graph visit program shown in Fig~\ref{code:language:visit}:

\begin{Verbatim}[numbers=left,fontsize=\codesize,commandchars=\*\[\]]
visit(A),
unvisited(A)
   -o visited(A),
      {B | !edge(A, B) -o visit(B)}.
\end{Verbatim}

This rule is translated to a sequent calculus proposition, as follows:

\begin{align}
\forall_A. (\texttt{visit}(A) \otimes \texttt{unvisited}(A) \lolli
   \texttt{visited}(A) \otimes \defsone{comp}{A})
\end{align}

First, the rule's variable $A$ is included using the $\forall$ connective. The
rule's LHS and RHS are connected using the $\lolli$ connective. The
comprehension is transformed into $\defsone{comp}{A}$, which is a
\emph{recursive} term that is assigned to an unique name, namely, $\m{comp}$.
This name is related to the following persistent term:

\begin{align}
\bang \forall_A. (\defsone{comp}{A} \lolli (\one \with
         (\forall_B. (\bang \texttt{edge}(A, B) \lolli
                                             \texttt{visit}(B)) \otimes
          \defsone{comp}{A})))
\end{align}

Notice that the enclosing $\forall$ includes all the arguments of the unique
name in order to pass around variables from outside the definition of the
comprehension, in this case variable $A$. The persistent term allows the
implication of the comprehension to be derived as many times as needed.
However, the argument list can also be used to implement aggregates.  Recall the
PageRank aggregate example shown before:

\begin{Verbatim}[fontsize=\codesize]
  update(A),
  !numInbound(A, T)
     -o [sum => V; B, Val, Iter | neighbor-pagerank(A, B, Val, Iter), V =
           Val/float(T) -o neighbor-pagerank(A, B, Val, Iter) -> sum-ranks(A, V)].
\end{Verbatim}

This rule is translated into a linear logic proposition as shown next:

\begin{align}
\forall_{A}. \forall_{T}. (\texttt{update}(A) \otimes \bang \texttt{numInbound}(A, T) \lolli
\defstwo{agg}{A, T}{0})
\end{align}

The persistent term for $\texttt{agg}$ is defined as follows:

\begin{multline}
\bang \forall_A. \forall_T. \forall_S. (\defstwo{agg}{A, T}{S} \lolli \texttt{sum-ranks}(A, S) \with\\
(\forall_V. \forall_B. \forall_{Val}. \forall_{Iter}.
   (\texttt{neighbor-pagerank}(A, B, Val, Iter) \otimes \bang V = Val/\texttt{float}(T) -o \\\texttt{neighbor-pagerank}(A, B, Val, Iter) \otimes \defstwo{agg}{A, T}{S + V})))
\end{multline}

The argument $S$ of $\defsz{agg}$ accumulates the PageRank values of the
neighborhood by consuming $\texttt{neighbor-pagerank}$ and re-deriving a new
$\defsz{agg}$ with $S + V$. Once the aggregate is complete, we simply select
$\texttt{sum-ranks}(A, S)$ instead. As an side, note how the constraint are
translated to a persistent term of the form $\bang V = Val/\texttt{float}(T)$
since it does not require any fact to be proven true. This recursive mechanism
is inspired in Baelde's work on \emph{fix points}~\cite{BaeldeM07,Baelde:2012}, which allow
the introduction of recursive definitions into a consistent fragment of linear
logic.
