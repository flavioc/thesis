Linear logic has been used in the past as a basis for logic-based programming
languages~\cite{Miller85anoverview}, including bottom-up and top-down
programming languages. Lolli, a programming language presented
in~\cite{Hodas94logicprogramming}, is based on a fragment of intuitionistic
linear logic and proves goals by lazily managing the context of linear resources
during top-down proof search.  LolliMon~\cite{Lopez:2005:MCL:1069774.1069778} is
a concurrent linear logic programming language that integrates both bottom-up
and top-down search, where top-down search is done sequentially but bottom-up
computations, which are encapsulated inside a monad, can be performed
concurrently. Programs start by performing top-down search but this can be
suspended in order to perform bottom-up search. This concurrent bottom-up search
stops until a fix-point is achieved, after which top-down search is resumed.
LolliMon is derived from the concurrent logical framework called
CLF~\cite{Watkins:2004uq,Cervesato02aconcurrent,Watkins03aconcurrent}.                   

