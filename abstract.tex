\begin{abstract}

Parallel programs are known to be difficult to write and reason about. Programs
written in low level parallel constructs commonly available in imperative
programming languages tend to be problematic to understand and debug.
Declarative programming is a step in the right direction because it moves the
details of parallelization from the programmer to the runtime system.  However,
these paradigms tend to remove most scheduling decisions from the programmer
resulting in few opportunities for optimization.

We propose a declarative logic programming language called Linear Meld that is
suited to run graph-based programs in parallel.  The foundation of our language
is linear logic, a powerful logical system where logical facts can be asserted
and removed. Linear logic gives the language a structured way to manage state
while remaining fully declarative. We envision the program as a communicating
graph data structure where each processing unit performs work on a different
subset of the graph. Logical rules are constrained to facts in the local node so
that computation happens locally.

Although Linear Meld is declarative, it also minimizes the problem of programmer
control by introducing coordination directives.  Coordination directives in
Linear Meld exist in two main forms: (1) as sensing facts with information about
the state of the runtime system and (2) as action facts that can be derive in
order to inform the scheduling decisions of the system.  The interplay between
regular facts, sensing facts and action facts results in faster execution time
and improved parallelism because regular facts affect how action facts are
derived and, conversely, action facts may affect which regular facts are
derived.

We have written graph algorithms, search algorithms and machine learning
algorithms and have seen good results on multicores, although our runtime system
can be easily extended to other distributed architectures.  \end{abstract}
