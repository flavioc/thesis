Parallel programming is known to be difficult to apply, exploit and reason
about. Programs written using low level parallel constructs tend to be
problematic to understand and debug. Declarative programming is a step towards
the right direction because it moves the details of parallelization from the
programmer to the runtime system. However, this paradigm leaves the programmer
with little opportunities to coordinate the execution of the program, resulting
in suboptimal declarative programs.  We propose a new declarative programming
language, called Linear Meld~(LM), that provides a solution to this problem by
supporting data-driven dynamic coordination mechanisms that are semantically
equivalent to regular computation.

LM is a logic programming language designed for programs that operate on graphs
and supports coordination and structured manipulation of mutable state.
Coordination is achieved through two mechanisms: (i) coordination facts, which
allow the programmer to control how computation is scheduled and how data is
laid out, and (ii) thread facts, which allow the programmer to reason about the
state of the underlying parallel architecture.  The use of coordination allows the
programmer to combine the inherent implicit parallelism of LM with a form of
declarative explicit parallelism provided by coordination, allowing the
development of complex coordinated programs that run faster than regular
programs. Furthermore, since coordination is indistinguishable from regular
computation, it allows the programmer to reason both about the problem at hand
and also about parallel execution.

We have written several graph algorithms, search algorithms and machine learning
algorithms in LM. For some programs, we have written informal proofs of
correctness to show that programs are easily proven correct. We have also
engineered a compiler and runtime system that is able to run LM programs on
multi core architectures with decent performance and scalability.
