This chapter provides an overview of the proof theoretic basis behind LM
and the dynamic semantics of the language. First, we will present the subset of
linear logic from which LM is built on. Second, we present the high level
dynamic semantics - how rules are evaluated and node communication - followed by
the low level dynamics, a close representation of how the virtual machine runs.
Finally, we prove that the low level dynamic semantics are sound in relation to
the high level dynamic semantics.

\section{Linear Logic}

Logic, as \emph{classically} understood, treats true propositions as
\emph{persistent truth}. When a persistent proposition is needed to prove other
propositions, it can be reused as many times as we wish because it is true
indefinitely. This is also true in the \emph{constructive} or
\emph{intuitionistic} school of logic.  Linear logic is a \emph{substructural
logic} (lacks weakening and contraction) developed by
Girard~\cite{Girard95logic:its} extends \emph{persistent logic} with linear
propositions which can be understood as ephemeral resources that can be used
only once to prove other propositions.  Naturally, linear logic is well
suited for modeling computing systems that deal with state, of which LM is
one of them.  Traditional forward-chaining logic programming languages like
Datalog only use persistent logic, however many ad-hoc
extensions~\cite{Liu98extendingdatalog,Ludascher95alogical} have been devised
in to support state updates, but most are extra-logical which makes it harder
to reason about programs. LM uses linear logic as its foundation, therefore
state updates are natural to the language.

In linear logic, truth is treated as a resource that is consumed once used. For
instance, in the graph visit program in Fig.~\ref{code:visit}, the
\texttt{unvisited(A)} and \texttt{visit(A)} linear facts are consumed in order
to prove \texttt{visit(A)} and the comprehension. If those facts were
persistent, then the rule would make no sense, because the node would be
\texttt{visited} and \texttt{unvisited} at the same time!

\subsection{Sequent calculus of \fragment}

We now describe the linear logic fragment used as a basis for LM.  Note that in
this thesis we follow the intuitionistic approach and use the sequent
calculus~\cite{gen35} to specify the logic. Our initial sequent is written as
$\Psi; \seqx{\Gamma}{\Delta}{C}$ and can be read as "assuming persistent
resources $\Psi$ and linear resources $\Delta$ then $C$ is true".  More
specifically, $\Psi$ is the typing context, $\Gamma$ is a multi-set of
persistent resources, $\Delta$ is a multi-set of linear resources while $C$ is
the proposition we want to prove.

We first have the \emph{simultaneous conjunction} $A \otimes B$ that packages
linear resources together. In the right rule, $A \otimes B$ is true if both $A$
and $B$ are true, and, in the left rule, it is possible to split $A \otimes B$
apart.

\[
\infer[\otimes R]
{\Psi ; \seqx{\Gamma}{\Delta, \Delta'}{A \otimes B}}
{\Psi ; \seqx{\Gamma}{\Delta}{A} & \Psi ; \seqx{\Gamma}{\Delta}{B}}
\tab
\infer[\otimes L]
{\Psi ;\seqx{\Gamma}{\Delta, A \otimes B}{C}}
{\Psi ; \seqx{\Gamma}{\Delta, A, B}{C}}
\]

Next, we have the \emph{additive conjunction} $A \with B$ that allows us to
select between $A$ or $B$. In the right rule we must prove $A$ and $B$ using
the same resources, while in the left rule, we can select one of the
resources.

\[
\infer[\with R]
{\Psi ; \seqx{\Gamma}{\Delta}{A \with B}}
{\Psi ; \seqx{\Gamma}{\Delta}{A} & \Psi ; \seqx{\Gamma}{\Delta}{B}}
\]

\[
\infer[\with L_1]
{\Psi ; \seqx{\Gamma}{\Delta, A \with B}{C}}
{\Psi ; \seqx{\Gamma}{\Delta, A}{C}}
\tab
\infer[\with L_2]
{\Psi ; \seqx{\Gamma}{\Delta, A \with B}{C}}
{\Psi ; \seqx{\Gamma}{\Delta, B}{C}}
\]


To express inference, we introduce the \emph{linear implication} connective
written as $A \lolli B$. For the right rule, we prove $A \lolli B$ by assuming
$A$ and then proving $B$, while in the left rule, we obtain $B$ by using some
linear resources to prove $A$.

\[
\infer[\lolli R]
{\Psi ; \seqx{\Gamma}{\Delta}{A \lolli B}}
{\Psi ; \seqx{\Gamma}{\Delta, A}{B}}
\tab
\infer[\lolli L]
{\seqx{\Gamma}{\Delta, \Delta', A \lolli B}{C}}
{\Psi ; \seqx{\Gamma}{\Delta}{A} &
   \Psi ; \seqx{\Gamma}{\Delta', B}{C}}
\]

Next, we introduce persistent resources written as $\bang A$. For the right
rule, we prove $\bang A$ by proving it without any linear resources. Likewise,
to use a persistent resource, we simply drop the $
\bang$. There is also a $\m{copy}$ rule that moves persistent resources from
$\Gamma$ to $\Delta$. Remember that $\Gamma$ contains persistent resources.

\[
\infer[\bang R]
{\Psi ; \seqx{\Gamma}{\cdot}{\bang A}}
{\Psi ; \seqx{\Gamma}{\cdot}{A}}
\tab
\infer[\bang L]
{\Psi ; \seqx{\Gamma}{\Delta, \bang A}{C}}
{\Psi ; \seqx{\Gamma, A}{\Delta}{C}}
\tab
\infer[\m{copy}]
{\Psi ; \seqx{\Gamma, A}{\Delta}{C}}
{\Psi ; \seqx{\Gamma, A}{\Delta, A}{C}}
\]

Another useful connective is the \emph{multiplicative unit} of the $\otimes$
connective. It is written as $\one$ and is best understood as something that
does not need any resource to be proven.

\[
\infer[\one R]
{\Psi ; \seqx{\Gamma}{\cdot}{\one}}
{}
\tab
\infer[\one L]
{\Psi ; \seqx{\Gamma}{\Delta, \one}{C}}
{\Psi ; \seqx{\Gamma}{\Delta}{C}}
\]

Next, we introduce the \emph{quantification} connectives, namely \emph{universal
quantification} $\forall_{n:\tau}. A$ and \emph{existencial quantification}
$\exists_{n:\tau}. A$. These connectives use the typing context $\Psi$ because
they can introduce and read terms from the context. The right and left duals of
those two connectives are dual.

\[
\infer[\forall R]
{\Psi ; \seqx{\Gamma}{\Delta}{\forall_{n:\tau}. A}}
{\Psi, m:\tau ; \seqx{\Gamma}{\Delta}{A\{m/n\}}}
\tab
\infer[\forall L]
{\Psi ; \seqx{\Gamma}{\Delta, \forall_{n:\tau}. A}{C}}
{\Psi \vdash M : \tau & \Psi ; \seqx{\Gamma}{\Delta, A\{M/n\}}{C}}
\]

\[
\infer[\exists R]
{\Psi ; \seqx{\Gamma}{\Delta}{\exists_{n: \tau}. A}}
{\Psi \vdash M : \tau &
   \Psi ; \seqx{\Gamma}{\Delta}{A \{M/n\}}}
\tab
\infer[\exists L]
{\Psi ; \seqx{\Gamma}{\Delta, \exists_{n:\tau}. A}{C}}
{\Psi, m:\tau ; \seqx{\Gamma}{\Delta, A\{m/n\}}{C}}
\]

Finally, we complete the linear logic system with the \emph{cut rules} and the
\emph{identity rule}:

\[
\infer[cut_A]
{\Psi ; \seqx{\Gamma}{\Delta, \Delta'}{C}}
{\Psi ; \seqx{\Gamma}{\Delta}{A} & \Psi ; \seqx{\Gamma}{\Delta', A}{C}}
\tab
\infer[cut\bang_A]
{\Psi ; \seqx{\Gamma}{\Delta}{C}}
{\Psi ; \seqx{\Gamma}{\cdot}{A} & \Psi ; \seqx{\Gamma, A}{\Delta}{C}}
\]

\[
\infer[id_A]
{\Psi ; \seqx{\Gamma}{A}{A}}
{}
\]


\subsection{From the \fragment sequent calculus to LM}


\begin{table*}
\begin{center}
\resizebox{16cm}{!}{
    \begin{tabular}{ | l | l || l | l | l |}
    \hline
    Connective                   & Description                                      & LM Syntax                                  & LM Place     & LM Example                                  \\ \hline \hline
    $\emph{fact}(\hat{x})$       & Linear facts.                                    & $fact(\hat{x})$                               & Body or Head    & \texttt{path(A, P)}                            \\ \hline
    $\bang \emph{fact}(\hat{x})$ & Persistent facts.                                & $\bang fact(\hat{x})$                         & Body or Head    & \texttt{$\bang$edge(X, Y, W)}                  \\ \hline
    $1$                          & Represents rules with an empty head.             & $1$                                           & Head            & \texttt{1}                                     \\ \hline
    $A \otimes B$                & Connect two expressions.                         & $A, B$                                        & Body and Head   & \texttt{path(A, P), edge(A, B, W)}             \\ \hline
    $\forall x. A$               & To represent variables defined inside the rule.  & Please see $A \lolli B$                       & Rule            & \texttt{path(A, B) $\lolli$ reachable(A, B)}   \\ \hline
    $\exists x. A$               & Instantiates new node variables.  & $\existsc{\widehat{x}}{B}$                  & Head            & \texttt{exists A.(path(A, P))}                 \\ \hline
    $A \lolli B$                 & $\lolli$ means "linearly implies".               & $A \lolli B$                                  & Rule            & \texttt{path(A, B) $\lolli$ reachable(A, B)}   \\
                                 & $A$ is the body and $B$ is the head.             &                                               &                 &                                                \\ \hline
    $\iters{name}{\widehat{V}}$               & For comprehensions
    ($\widehat{V}$ is empty).  & $\comprehension{\widehat{x}}{A}{B}$  & Head            & \texttt{\{B | !edge(A, B) | visit(B)\}}        \\
                                 & For aggregates ($\widehat{V}$ accumulates).          &                                               &                 &                                                \\ \hline
    \end{tabular}
}
\end{center}
\caption{Connectives from \fragment and their use in LM.}
\label{table:linear}
\end{table*}

The connections between LM and \fragment presented in the previous section
are somewhat obvious. We summarize the connection connectives of the system and
the LM syntax in Table~\ref{table:linear}. As an example, we translate the
abstract syntax of the first rule of the graph visit program shown
in~\ref{visit:ast} to a proposition in \fragment:

\begin{align}
\forall_A. (\mathtt{visit}(A) \otimes \mathtt{unvisited}(A) \lolli
   \mathtt{visited}(A) \otimes \itersz{comp} A)
\end{align}

The translation is fairly straightforward, except for the comprehension. Each
comprehension of a LM program must be assigned to an unique name and its
corresponding terms. For the iterative definition $comp$, it is defined as
following:

\begin{align}
\iterz{comp}{0} A & \defeq \one\\
\iterz{comp}{N} A & \defeq \forall_B.((\bang \mathtt{edge}(A, B) \lolli
\mathtt{visit}(B)) \otimes \iterz{comp}{N-1} A)
\end{align}

Notice that the argument list of the iterative definition is being used
to pass around terms from outside the definition, in this case, the variable
$A$. However, the argument list can also be used to implement aggregates.
Recall the aggregate example shown before:

{\small
\begin{Verbatim}
count-prices(A) -o [sum => P | . | price(A, P) | 1 | total(A, P)].
\end{Verbatim}
}

This rule is translated into a linear logic proposition as shown next:

\begin{align}
\forall A. (\mathtt{count-prices}(A) \lolli \iters{agg} A \; 0)
\end{align}

The iterative definition $\mathtt{agg}$ is defined as follows:

\begin{align}
\iterz{agg}{0} A \; P' & \defeq \mathtt{total}(A, P') \\
\iterz{agg}{N} A \; P' & \defeq \forall_P. ((\mathtt{price}(A, P) \lolli \one)
      \otimes \iterz{agg}{N-1} A \; (P' + P))
\end{align}

\section{High Level Dynamic Semantics}


In this section, we present the high level dynamic~(HLD) semantics of LM.  HLD
formalizes the mechanism of matching rules and deriving new facts.  HLD
semantics present a simplified overview of the dynamics of the language that are
closer to \fragment (Section~\ref{sec:fragment}) presented before.  than the
implementation principles of our virtual machine. The low level dynamic~(LLD)
semantics are much closer to a real implementation and represents the
operational semantics of the language.

Note that both HLD and LLD do not model the use of variable bindings when
matching facts from the database. The formalization of bindings tends to
complicate the formal system and it is not necessary for a good understanding of
the system. Instead, we assume that all facts of type $\emph{fact}(\hat{x})$ do
not have the argument $\hat{x}$.

Starting from \fragment presented earlier, we consider $\Gamma$ and $\Delta$ the database
of our program. $\Gamma$ contains the database of persistent facts while $\Delta$ the database of linear
facts. We assume that the rules of the program are persistent linear implications of the form
$\bang (A \lolli B)$ that can be used several times. However, we do not put the rules in the $\Gamma$
context but in a separate context $\Phi$.

The main idea of the dynamic semantics is to ignore the right side of the
sequent calculus and use \emph{chaining} and \emph{inversion} on the $\Delta$
and $\Gamma$ contexts so that we only have atomic facts (e.g., the database of
facts).  To apply rules we use
\emph{focusing}~\cite{Andreoli92logicprogramming} on one of the derivation rules
in $\Phi$. Note that in the focusing process we assume that all the atoms
(facts) are positive thus the chaining proceeds in a \emph{forward chaining}
fashion.

\subsection{Step}\label{sec:step_hld}

Operationally, LM proceeds in \emph{steps}. A step happens at some node $i$ and
proceeds by picking one rule to apply, matching the body of the rule against the
database, removing all those facts from the database and then deriving all the
constructs in the head of the rule. We assume the existence of $n$ nodes in the
program and that $\Delta$ and $\Gamma$ are split into $\Delta_1, \dotsc, \Delta_n$
and $\Gamma_1, \dotsc, \Gamma_n$ respectively. After each step, the database of
each fact is updated accordingly.

Steps are defined as $\stepz \Gamma; \Delta; \Phi \Longrightarrow \Gamma';
\Delta'$, where $\Gamma'$ and $\Delta'$ is the new database. The step rule is as
follows:

{\footnotesize
\[
\infer[\stepz]
{\stepz [\Gamma_1 .. \Gamma_i .. \Gamma_n]; [\Delta_1 .. \Delta_i ..
   \Delta_n]; \Phi \Longrightarrow [\Gamma_1, \Gamma'_1; .. \Gamma_i,
   \Gamma'_i; .. \Gamma_n, \Gamma'_n]; [\Delta_1, \Delta'_1; .. (\Delta_i -
         \Xi'), \Delta'_i; .. \Delta_n, \Delta'_n]}
{\doz \Gamma_i; \Delta_i; \Phi \rightarrow \Xi'; \Delta'_1 .. \Delta'_n;
   \Gamma'_1 .. \Gamma'_n}
\]
}


\subsection{Application}

A step is performed through $\doz \Gamma; \Delta; \Phi \rightarrow \Xi';
\Delta'; \Gamma'$.  $\Gamma$, $\Delta$ and $\Phi$ have the meaning explained
before, while $\Xi'$, $\Delta'$ and $\Gamma'$ are output multi-sets from
applying one of the rules in $\Phi$. $\Xi'$ is the set of consumed linear
resources, $\Delta'$ is the set of derived linear facts and $\Gamma'$ is the set
of derived persistent facts. Note that for HLD semantics there is no concept of
rule priority, therefore a rule is picked non-deterministically.

The judgment $\az \Gamma ; \Delta ; A \lolli B \rightarrow \Xi'; \Delta';
\Gamma'$ will attempt to apply the derivation rule $A \lolli B$. To do this, it
splits the $\Delta$ context into $\Delta_1$ and $\Delta_2$, namely the set of
linear resources consumed to match the body of the rule ($\Delta_1$) and the
remaining linear facts ($\Delta_2$).  Again, the set of resources needed to
match the body of the rule is guessed. LLD semantics will deterministically
calculate $\Delta_1$.

\[
\infer[\az \m{rule}]
{\az \Gamma ; \Delta_1, \Delta_2 ; A \lolli B \rightarrow \Xi' ; \Delta' ; \Gamma'}
{\mz \Gamma ; \Delta_1 \rightarrow A & \dz \Gamma ; \Delta_2; \Delta_1; \cdot ; \cdot ; B \rightarrow \Xi' ; \Delta' ; \Gamma'}
\]

\[
\infer[\doz \m{rule}]
{\doz \Gamma ; \Delta ; R, \Phi \rightarrow \Xi' ; \Delta' ; \Gamma'}
{\az \Gamma ; \Delta ; R \rightarrow \Xi' ; \Delta' ; \Gamma'}
\]


\subsection{Match}

The $\mz \Gamma ; \Delta \rightarrow C$ judgment uses the right ($R$) rules of
\fragment in order to match (prove) the body $C$ using $\Gamma$
and $\Delta$. We must consume all the linear facts in the multi-set $\Delta$
when matching $C$. The context $\Gamma$ may be used to match persistent terms in
$C$ but such facts are never consumed since they are persistent.

\[
\infer[\mzname \one]
{\mz{\Gamma}{\cdot}{\one}}
{}
\]
\[
\infer[\mzname p]
{\mz{\Gamma}{p}{p}}
{}
\tab
\infer[\mzname \bang p]
{\mz{\Gamma, p}{\cdot}{\bang p}}
{}
\]

\[
\infer[\mzname \otimes]
{\mz{\Gamma}{\Delta_1, \Delta_2}{A \otimes B}}
{\mz{\Gamma}{\Delta_1}{A} & \mz{\Gamma}{\Delta_2}{B}}
\]


\subsection{Derivation}

After successfully matching the body of the rule, we next derive the head of the
rule. The derivation judgment has the form $\dz \Gamma ; \Delta ; \Xi ; \Gamma_1
; \Delta_1 ; \Omega \rightarrow \Xi'; \Delta'; \Gamma'$ with the following
meaning:

\begin{enumerate}

   \item[$\Gamma$] the multi-set of persistent resources in the database;

   \item[$\Delta$] the multi-set of linear resources in the database not yet
   consumed;

   \item[$\Xi$] the multi-set of linear resources that have been consumed while
   matching the body of the rule, matching comprehensions or aggregates;

   \item[$\Gamma_1$] the multi-set of persistent facts that have been derived
   using the current rule;

   \item[$\Delta_1$] the multi-set of linear facts that have been derived using
   the current rule;

   \item[$\Omega$] an ordered list contain the terms of the head of rule that
   still need to be derived. We start with the head of the rule $B$ that is
   continuously deconstructed to derive all the facts of the rule;

   \item[$\Xi'$] the consumed linear facts to apply this rule;

   \item[$\Delta'$] the derived linear facts;

   \item[$\Gamma'$] the derived persistent facts.

\end{enumerate}

The following derivation rules are a direct translation from \fragment:

\[
\infer[\dz p]
{\dz \Gamma ; \Delta ; \Xi ; \Gamma_1 ; \Delta_1 ; p, \Omega \rightarrow \Xi' ; \Delta' ; \Gamma'}
{\dz \Gamma ; \Delta ; \Xi ; \Gamma_1 ; p, \Delta_1 ; \Omega \rightarrow \Xi' ; \Delta' ; \Gamma'}
\]

\[
\infer[\dz \bang p]
{\dz \Gamma ; \Delta ; \Xi ; \Gamma_1 ; \Delta_1 ; \bang p, \Omega \rightarrow \Xi' ; \Delta' ; \Gamma'}
{\dz \Gamma ; \Delta ; \Xi ; \Gamma_1, p ; \Delta_1 ; \Omega \rightarrow \Xi' ; \Delta' ; \Gamma'}
\]

\[
\infer[\dz \otimes]
{\dz \Gamma ; \Delta ; \Xi ; \Gamma_1 ; \Delta_1 ; A \otimes B, \Omega \rightarrow \Xi' ; \Delta' ; \Gamma'}
{\dz \Gamma ; \Delta ; \Xi ; \Gamma_1 ; \Delta_1 ; A, B, \Omega \rightarrow \Xi' ; \Delta' ; \Gamma'}
\]

\[
\infer[\dz \one]
{\dz \Gamma ; \Delta ; \Xi ; \Gamma_1; \Delta_1 ; 1, \Omega \rightarrow \Xi' ; \Delta' ; \Gamma'}
{\dz \Gamma ; \Delta ; \Xi ; \Gamma_1; \Delta_1 ; \Omega \rightarrow \Xi' ; \Delta' ; \Gamma'}
\]

\[
\infer[\dz end]
{\dz \Gamma ; \Delta ; \Xi' ; \Gamma' ; \Delta' ; \cdot \rightarrow \Xi' ; \Delta' ; \Gamma'}
{}
\]


We did not include the aggregates here because they are similar to comprehensions.
The starting rule for deriving comprehensions is $\dz comp^*$. It
deterministically picks a number $N$ that then can be unfolded $N$ times to get
$A \lolli B$. The HLD semantics do not take into account the contents of the
database to determine how many times a comprehension should be applied.

\[
\infer[\dz \m{comp}^*]
{\dz \Gamma ; \Delta ; \Xi ; \Gamma_1 ; \Delta_1 ; \compsz{A}{B}, \Omega \rightarrow \Xi' ; \Delta' ; \Gamma'}
{\dz \Gamma ; \Delta ; \Xi ; \Gamma_1 ; \Delta_1 ; \compz{N}{A}{B}, \Omega \rightarrow \Xi' ; \Delta' ; \Gamma'}
\]

\[
\infer[\dz \m{comp}^N]
{\dz \Gamma ; \Delta ; \Xi ; \Gamma_1 ; \Delta_1 ; \compz{N}{A}{B}, \Omega \rightarrow \Xi' ; \Delta' ; \Gamma'}
{\dz \Gamma ; \Delta ; \Xi ; \Gamma_1 ; \Delta_1 ; \compunfold{N-1}{A}{B}, \Omega \rightarrow \Xi' ; \Delta' ; \Gamma'}
\]

\[
\infer[\dz \m{comp}^0]
{\dz \Gamma ; \Delta ; \Xi ; \Gamma_1 ; \Delta_1 ; \compz{0}{A}{B}, \Omega \rightarrow \Xi' ; \Delta' ; \Gamma'}
{\dz \Gamma ; \Delta ; \Xi ; \Gamma_1 ; \Delta_1 ; \compunfoldz, \Omega \rightarrow \Xi' ; \Delta' ; \Gamma'}
\]


These rules mirror the rules for the iterative definition in \fragment.  Rules
for aggregates are similar:

\[
\infer[\dz \m{agg}^*]
{\dz \Gamma ; \Delta ; \Xi ; \Gamma_1 ; \Delta_1 ; \aggsz{A}{B}{C}, \Omega \rightarrow \Xi' ; \Delta' ; \Gamma'}
{\dz \Gamma ; \Delta ; \Xi ; \Gamma_1 ; \Delta_1 ; \aggz{N}{A}{B}{C}{0}, \Omega \rightarrow \Xi' ; \Delta' ; \Gamma'}
\]

{\small
\[
\infer[\dz \m{agg}^N]
{\dz \Gamma ; \Delta ; \Xi ; \Gamma_1 ; \Delta_1 ; \aggz{N}{A}{B}{C}{V}, \Omega \rightarrow \Xi' ; \Delta' ; \Gamma'}
{\dz \Gamma ; \Delta ; \Xi ; \Gamma_1 ; \Delta_1 ; \aggunfold{N-1}{A}{B}{C}{V}, \Omega \rightarrow \Xi' ; \Delta' ; \Gamma'}
\]
}

\[
\infer[\dz \m{agg}^0]
{\dz \Gamma ; \Delta ; \Xi ; \Gamma_1 ; \Delta_1 ; \aggz{0}{A}{B}{C}{V}, \Omega \rightarrow \Xi' ; \Delta' ; \Gamma'}
{\dz \Gamma ; \Delta ; \Xi ; \Gamma_1 ; \Delta_1 ; \aggunfoldz{C}{V}, \Omega \rightarrow \Xi' ; \Delta' ; \Gamma'}
\]


Finally, because the comprehensions and aggregates create implications $A \lolli
B$, we add a final derivation rule $\dz \lolli$:

\[
\infer[\dzname \lolli]
{\dz{\Gamma}{\Pi}{\Delta_a, \Delta_b}{\Xi}{\Gamma_1}{\Delta_1}{A \lolli B,
   \Omega}{\outsem}}
{\mz{\Gamma}{\Delta_a}{A} & \dz{\Gamma}{\Pi}{\Delta_b}{\Xi, \Delta_a}
   {\Gamma_1}{\Delta_1}{B, \Omega}{\outsem}}
\]

\[
\infer[\dzname \forall]
{\dz{\Gamma}{\Pi}{\Delta}{\Xi}{\Gamma_1}{\Delta_1}{\forall_x. A,
   \Omega}{\outsem}}
{\dz{\Gamma}{\Pi}{\Delta}{\Xi}{\Gamma_1}{\Delta_1}{A\{V/x\},
   \Omega}{\outsem}}
\]



\section{Low Level Dynamic Semantics}
The Low Level Dynamic~(LLD) semantics remove all the non-deterministic choices
in the previous dynamics and makes them deterministic. The new semantics will do
the following:

\begin{itemize}

   \item Match rules by priority order;

   \item Determine the set of linear facts needed to match either the rule's LHS
      or the LHS of comprehensions/aggregates without guessing;

   \item Apply as many comprehensions as the database allows.

   \item Apply as many aggregates as the database allows.

\end{itemize}

While the implementation presented in Chapter~\ref{chapter:local} follows the
LLD semantics, there are several optimizations not implemented in LLD, such as:

\begin{itemize}
   \item Indexing: the implementation uses indexing for looking up facts using a
      specific argument;
   \item Better candidate rules: when selecting a rule to execute, the
      implementation filters out rules which do not have enough facts to be
      derived;
   \item Multiple rule derivation: the LLD semantics only execute one rule at
      the time, while the implementation is able to derive a rule multiple times
      when there are no conflicting rule;
   \item Matching and substitution: in the implementation, matching is done
      implicitly using variables and comparisons, while LLD uses the $\Psi$
      context to hold substitutions.
\end{itemize}

The complete set of inference rules for the LLD semantics are presented in
Appendix~\ref{sec:lld}.

LLD is specified as an \emph{abstract machine} and is represented as a sequence
of state transitions of the form $\trans{S_1}{S_2}$. HLD had many different
proof trees for a given triplet $\Gamma; \Delta; \Phi$ because HLD allows
choices to be made during the inference rules. For instance, in HLD any rule
could be selected to be executed. In LLD there is only one state sequence
possible for a given $\Gamma; \Delta; \Phi$ since there is no guessing involved.
LLD semantics present a complete step by step mechanism that is needed to
correctly evaluate a LM program. For instance, when LLD tries to apply a rule,
it will check if there are enough facts in the database and backtrack until a
rule can be applied.

\subsection{Application}

LLD shares exactly the same inputs and outputs as HLD. The inputs correspond to
the $\Gamma$ and $\Delta$ fact contexts and the list of rules $\Phi$, while the
outputs correspond to the newly asserted facts in $\Gamma'$ and $\Delta'$ and
the retracted facts which are put in the $\Xi'$ context.

The first difference between LLD and HLD start when picking a rule to derive.
Instead of guessing, LLD treats the list of rules as a stack and picks the first
rule $R_1$ to execute (the rule with the highest priority). The remaining rules
are stored as a \emph{continuation}. If $R_1$ cannot be matched because there
are not enough facts in the database, we backtrack and use the rule continuation
to pick the next rule and so on, until one rule can be successfully applied.

The machine starts with a database $(\Gamma; \Delta)$ and a list of rules
$\Phi$. The initial state is always $\dostate{\Delta}{\Phi}{\Gamma}{\Pi}$.
We start by picking the first rule $R_1$ from $\Phi$:


\[
\trans{\dostate{\Delta}{R_1, \Phi}{\Gamma}{\Pi}}
{\appstate{\cdot}{\Delta}{\Phi}{\Pi}{\Gamma}{R}} \tag{select rule}
\]


If, after trying all the rules, there are no remaining candidate rules, the
machine enters into the $\m{next}$ state, which means that no more rules are
possible for this node and the machine should perform local computation on
another node.


\[
\trans{\dostate{\Delta}{\cdot}{\Gamma}{\Pi}}
   {\failstate{\Gamma}{\Delta}} \tag{fail}
\]


In order to try a particular rule, we either need to unfold the $\forall$
connective, by adding its variable to the $\Psi$ context, or, initiate the matching
process when reaching the $\lolli$ connective. The variables in the $\Psi$
context, which are initially assigned to an unknown value $\_$, will later be
assigned to a concrete value as the matching process goes forward.


\[
   \trans{\appstate{\Psi}{\Delta}{\Phi}{\Pi}{\Gamma}{\forall_{x : \tau}. A}}
   {\appstate{\Psi, x : \_ : \tau}{\Delta}{\Phi}{\Pi}{\Gamma}{A}}
                                                             \tag{open rule}
\]


\[
   \trans{\appstate{\Psi}{\Delta}{\Phi}{\Pi}{\Gamma}{A \lolli B}}
   {\matstateb{A \lolli B}{(\Delta; \Phi)}{\cdot}{\Gamma}{\Delta}{A}{\cdot \rightarrow
   \one}{\Psi}} \tag{init
                                                            rule}
\]



\subsection{Continuation Frames}

The most interesting aspects introduced by the LLD machine are the
\emph{continuation frame} and the \emph{continuation stack}. A continuation
frame acts as a choice point that is created during rule matching whenever we
try to match a fact expression against the database.  The frame considers all
the facts relevant to the expression given the current context $\Psi$.

The frame contains enough state to resume the matching process at the time of
its creation, therefore we can easily backtrack to the choice point and select
the next candidate fact from the database.  We keep the continuation frames in a
continuation stack for backtracking purposes. If, at some point there are no
candidate facts because the current variable assignments are not usable, we
update the top frame to try the next candidate fact. If all candidates are
exhausted, we pop the top frame and continue with the next available frame.

By using this match mechanism, we determine which facts need to be used to match
a rule.  Our LM implementation works like LLD, by iterating over the available
facts at each choice point and then committing to the rule if the matching
process succeeds. However, while the implementation only attempts to match rules
when the database has all the facts required by the rule's LHS, LLD is more
na\"{i}ve in this aspect because it tries all rules in order.


\subsection{Structure of Continuation Frames}

We have two continuation frame types, depending on the type of the candidate
facts.\footnote{All continuation frames have an implicit $\Psi$ context that
models variable assignments, including variable names, values and their
locations in the terms. This is important if we want to model variable
assignments and matchings.}

\subsubsection{Linear Continuation Frames}

There are two types of continuation frames. Linear frames use the form
$\lframe{\Delta}{\Delta''}{p(\widehat{x})}{\Omega; \Psi}{\Delta'}{\Omega'}$, where:

\begin{description}

   \item[$p(\widehat{x})$] atomic proposition that created this
      frame. The predicate for the proposition is $p$;

   \item[$\Delta$] multi-set of linear facts that are not of predicate $p$ plus
      all the other candidate facts of the predicate $p$ we have already
      tried, including a fact $p$, which is the current candidate fact;

   \item[$\Delta''$] facts of predicate $p$ that match $p(\widehat{x})$ which we
      haven't tried yet. It is a multi-set of linear facts;


   \item[$\Omega$] ordered list of remaining terms needed to match;

   \item[$\Delta'$] multi-set of linear facts we have consumed to reach this point;

   \item[$\Omega'$] terms matched already using $\Delta'$ and $\Gamma$;
   \item[$\Psi$] dictionary of variable assignments (includes variable and
      value).
\end{description}

\subsubsection{Persistent Continuation Frame}

Persistent frames are slightly different since they only need to keep track of
remaining persistent candidates. They are structured as
$\pframe{\Gamma''}{\Delta}{\bang
   p(\widehat{x})}{\Omega; \Psi}{\Delta'}{\Omega'}$:

\begin{description}
   \item[$\bang p(\widehat{x})$] persistent atomic proposition that created
      this frame;
   \item[$\Gamma''$] remaining candidate facts that match $\bang p(\widehat{x})$;
   \item[$\Delta$] multi-set of linear facts not consumed yet;

   \item[$\Omega$] ordered list of terms needed to match past this
   frame;

   \item[$\Delta'$] multi-set of linear facts consumed up-to this frame;
   \item[$\Omega'$] terms matched up-to this point using $\Delta'$ and $\Gamma$;
   \item[$\Psi$] dictionary of variable assignments (includes variable and value).
\end{description}


\subsection{Match}\label{sec:lld_body_match}

The matching state for the LLD machine uses the continuation stack to try
different combinations of facts until a match is achieved.  The state is
structured as $\matstate{A \lolli
   B}{\rulestk}{\lstack{C}}{\Gamma}{\Delta}{\Omega}{\Delta' \rightarrow
      \Omega'}$, where:

\begin{description}
   \item[$A \lolli B$] rule being matched: $A$ is the rule's LHS and $B$ the RHS;

   \item[$\rulestk$] rule continuation to be used if the current rule fails.
   Contains the original $\Delta_N$ and the rest of the rules $\Phi$;

   \item[$\lstack{C}$] ordered list of frames representing the continuation
   stack used for matching $A$;

   \item[$\Delta$] multi-set of linear facts still available to complete the
   matching process;

   \item[$\Omega$] ordered list of deconstructed RHS terms to match;

   \item[$\Delta'$] multi-set of linear facts from the original $\Delta_N$ that
   were already consumed ($\Delta', \Delta = \Delta_N$);

   \item[$\Omega'$] parts of $A$ already matched. They are in the form $P_1
   \otimes \dotsb \otimes P_n$. The idea is to use term equivalence and the fact
   that $\feq{\Omega, \Omega'}{A}$ to justify $\mz{\Gamma}{\Delta'}{A}$ when the
   matching process completes.

\end{description}

Not shown in the matching state is the context $\Psi$ that maps variables to
values. At the start of matching, the $\widehat{x}$ variables are set as
\emph{undefined}. Matching then uses facts from $\Delta$ and $\Gamma$ to match
the terms of the rule's LHS represented as $\Omega$. During the process
continuation frames are pushed into $\lstack{C}$ and if the matching process
fails, we use $\lstack{C}$ to restore the process using different candidate
facts. New facts also update the variables in the $\Psi$ context by assigning
them concrete values.

\subsubsection{Linear fact expression}

The first 2 state transitions are used when the head of $\Omega$ is a linear fact
expression $p$.

In the first transition we find $p_1$ and $\Delta''$ as facts from the database
that match $p$'s hidden and partially initialized arguments.  Context $\Delta''$
is stored in the second argument of the new continuation frame but is passed
along with $\Delta$ since the facts have not been consumed yet (just $p_1$).

The second transition deals with the case where there are no candiate facts and
thus a different machine state is used for enabling backtracking.

Note that the proposition $p_1, \Delta'' \prec p$ indicates that facts
$\Delta'', p_1$ satisfy the constraints of $p$ while $\Delta \npreceq p$
indicates that no fact in $\Delta$ satisfies $p$. Both propositions use the
omitted variable context $\Psi$ in order to replace the variables of $p$.


\begin{multline}
\transx{\matstateb{A \lolli B}{\rulestk}{\lstack{C}}{\Gamma}{\Delta, p_1,
\Delta''}{p(\widehat{x}),
   \Omega}{\Delta' \rightarrow \Omega'}{\Psi}}
{\matstateb{A \lolli B}{\rulestk}{\lframe{\Delta,
p_1}{\Delta''}{p(\widehat{x})}{\Omega; \m{extend}(\Psi, \theta)}{\Delta'}{\Omega'}, \lstack{C}}{\Gamma}{\Delta,
   \Delta''}{\Omega}{\Delta', p_1 \rightarrow \Omega' \otimes
      p(\widehat{x}\theta)}{\m{extend}(\Psi, \theta)}} \\
   \;\;\; (p_1,
   \Delta'' \prec p(\widehat{x}) \;\;\; \Delta \npreceq p(\widehat{x}))
   \tag{match p ok}
\end{multline}

\begin{align}
   \trans{\matstate{A \lolli
   B}{\rulestk}{\lstack{C}}{\Gamma}{\Delta}{p(\widehat{x}),
   \Omega}{\Delta' \rightarrow \Omega'}}
{\contstate{A \lolli B}{\rulestk}{\lstack{C}}{\Gamma}} \;\;\; (\Delta \npreceq
p(\widehat{x})) \tag{match p fail}
\end{align}


\subsubsection{Persistent fact expressions}

The next 2 state transitions are used when the head of $\Omega$ contains a
persistent fact expression $\bang p$. They are identical to the previous 2
transitions but they deal with the persistent context $\Gamma$.


\[
\infer[\mo \bang p~\m{first}]
{\mo \Gamma, p_1, \Gamma'' ; \Delta; \Xi; \bang p, \Omega; H; \cdot; \lstack{R}
   \rightarrow \outsem}
{
   \begin{gathered}
      p_1, \Gamma'' \prec \bang p \\
      \mo \Gamma, p_1, \Gamma'' ; \Delta; \Xi; \Omega;
      H; [\Gamma''; \Delta; \bang p ; \Omega; \Xi; \cdot; \cdot]; \lstack{R} \rightarrow \outsem
   \end{gathered}
}
\]

\[
\infer[\mo \bang p~\m{on}~q]
{\mo \Gamma, p_1, \Gamma'' ; \Delta; \Xi; \bang p, \Omega; H; f, \lstack{C};
   \lstack{R}
   \rightarrow \outsem}
{
   \begin{gathered}
      p_1, \Gamma'' \prec \bang p \\
      f = (\Delta_{old}; \Delta'_{old};
         q; \Omega_{old}; \Xi_{old}; \Lambda; \Upsilon) \\
      \mo \Gamma, p_1,
         \Gamma'' ; \Delta; \Xi; \Omega; H; [\Gamma''; \Delta; \bang p ; \Omega; \Xi; q,
      \Lambda; \Upsilon], f, \lstack{C}; \lstack{R} \rightarrow \outsem
   \end{gathered}
}
\]


\[
\infer[\mo \bang p~\m{on}~\bang q]
{\mo \Gamma, p_1, \Gamma'' ; \Delta; \Xi; \bang p, \Omega; H; f, \lstack{C};
   \lstack{R}
   \rightarrow \outsem}
{
   \begin{gathered}
      p_1, \Gamma'' \prec \bang p \\
      f = [\Gamma_{old}; \Delta_{old}; \bang q; \Omega_{old}; \Xi_{old}; \Lambda; \Upsilon] \\
      \mo \Gamma, p_1, \Gamma'' ; \Delta; \Xi; \Omega; H; [\Gamma''; \Delta;
      \bang p ; \Omega; \Xi; \Lambda; q, \Upsilon], f, \lstack{C}; \lstack{R} \rightarrow \outsem
   \end{gathered}
}
\]

\[
\infer[\mo \bang p~\m{fail}]
{\mo \Gamma ; \Delta; \Xi; \bang p, \Omega; H; \lstack{C}; \lstack{R} \rightarrow \outsem}
{\Gamma \npreceq \bang p & \cont \lstack{C}; H; \lstack{R}; \Gamma \rightarrow \outsem}
\]


\subsubsection{Other expressions}

Finally, we have the transitions that deconstruct the other LHS terms and a
final transition to initiate the RHS derivation.


\begin{align}
\trans{\matstate{A \lolli B}{\rulestk}{\lstack{C}}{\Gamma}{\Delta}{\one,
   \Omega}{\Delta' \rightarrow \Omega'}}
{\matstate{A \lolli B}{\rulestk}{\lstack{C}}{\Gamma}{\Delta}{\Omega}{\Delta'
   \rightarrow \Omega'}} \tag{match $\one$}
\end{align}

\begin{align}
\trans{\matstate{A \lolli B}{\rulestk}{\lstack{C}}{\Gamma}{\Delta}{X \otimes Y,
   \Omega}{\Delta' \rightarrow \Omega'}}
{\matstate{A \lolli B}{\rulestk}{\lstack{C}}{\Gamma}{\Delta}{X, Y,
   \Omega}{\Delta' \rightarrow \Omega;}} \tag{match $\otimes$}
\end{align}

\begin{align}
\trans{\matstate{A \lolli
   B}{\rulestk}{\lstack{C}}{\Gamma}{\Delta}{\cdot}{\Delta' \rightarrow \Omega'}}
{
   \derstatex{\Gamma}{\Delta}{\Delta'}{\cdot}{\cdot}{B}
} \tag{match end}
\end{align}


\subsection{Backtracking}\label{sec:lld_match_cont}

The backtracking state of the machine reads the top of the continuation stack
$\lstack{C}$ and restores the matching process with a different candidate fact
from the continuation frame. The state is written as $\contstate{A \lolli
B}{\rulestk}{\lstack{C}}{\Gamma}$, where:

\begin{description}
   \item[$A \lolli B$] the rule being matched;
   \item[$\rulestk$] next available rules if the current rule does not match;
   the rule continuation;
   \item[$\lstack{C}$] the continuation stack for matching $A$;
\end{description}

\subsubsection{Linear continuation frames}

The next two state transitions handle linear continuation frames on the top of the
continuation stack. The first transition selects the next candidate fact $p_1$ from the
second argument of the linear frame and updates the frame. Otherwise, if we have
no more candidate facts, we pop the continuation frame and backtrack to the
remaining continuation stack.

\begin{align}
\trans{\contstate{A \lolli B}{\rulestk}{\lframe{\Delta}{p_2,
   \Delta''}{p}{\Omega}{\Delta'}{\Omega'}, \lstack{C}}{\Gamma}}
{
   \matstate{A \lolli B}{\rulestk}{\lframe{\Delta,
      p_2}{\Delta''}{p}{\Omega}{\Delta'}{\Omega'},
   \lstack{C}}{\Gamma}{\Delta}{\Omega}{\Delta', p_2 \rightarrow \Omega' \otimes p}}
   \tag{next p}
\end{align}

\begin{align}
\trans{\contstate{A \lolli
   B}{\rulestk}{\lframe{\Delta}{\cdot}{p}{\Omega}{\Delta'}{\Omega'},
      \lstack{C}}{\Gamma}}
{
   \contstate{A \lolli B}{\rulestk}{\lstack{C}}{\Gamma}} \tag{next frame}
\end{align}


\subsubsection{Persistent continuation frames}

We also have the same two kinds of inference rules for persistent continuation
frames.

\[
\infer[\cont \bang p~\m{next}]
{\cont [p_1, \Gamma'; \Delta; \bang p, \Omega; \Xi; \Lambda; \Upsilon],
   \lstack{C}; H; \lstack{R};
   \Gamma \rightarrow \outsem}
{\mo \Gamma; \Delta; \Xi; \Omega; H; [\Gamma'; \Delta; \bang p, \Omega; \Xi;
   \Lambda; \Upsilon], \lstack{C}; \lstack{R} \rightarrow \outsem}
\]

\[
\infer[\cont \bang p~\m{no~more}]
{\cont [\cdot; \Delta; \bang p, \Omega; \Xi; \Lambda; \Upsilon], \lstack{C}; H;
   \lstack{R}; \Gamma
   \rightarrow \outsem}
{\cont \lstack{C}; H; \lstack{R}; \Gamma \rightarrow \outsem}
\]


\subsubsection{Empty continuation stack}

Finally, if the continuation stack is empty, we simply force execution to try
the next inference rule in $\Phi$.

\[
\infer[\cont \m{next~rule}]
{\cont \cdot; H; (\Phi, \Delta); \Gamma \rightarrow \Xi'; \Delta'; \Gamma'}
{\doo \Gamma; \Delta; \Phi \rightarrow \Xi'; \Delta'; \Gamma'}
\]


\subsection{Derivation}

Once the list of terms $\Omega$ of the LHS is exhausted, we derive the rule's
RHS. The derivation state simply iterates over $B$, the rule's RHS, and derives
terms into the corresponding new contexts. The state is represented as
$\derstatex{\Gamma}{\Delta}{\Xi}{\Gamma_1}{\Delta_1}{\Omega}$ with the following
meaning:

\begin{enumerate}
   \item[$\Gamma$] set of persistent facts;

   \item[$\Delta$] multi-set of remaining liner facts;

   \item[$\Xi$] multi-set of linear facts consumed up-to this point;

   \item[$\Gamma_1$] set of persistent facts derived;

   \item[$\Delta_1$] multi-set of linear facts derived;

   \item[$\Omega$] remaining terms to derive as an ordered list. We start with
   $B$ if the original rule is $A \lolli B$.

\end{enumerate}

\subsubsection{Fact templates}

When deriving either $p$ or $\bang p$ we have the following two inference rules:

{\footnotesize
\[
\infer[\done p]
{\done \Gamma ; \Delta; \Xi; \Gamma_1 ; \Delta_1; p, \Omega \rightarrow \Xi'; \Delta'; \Gamma'}
{\done \Gamma ; \Delta; \Xi; \Gamma_1 ; p, \Delta_1; \Omega \rightarrow \Xi'; \Delta'; \Gamma'}
\tab
\infer[\done \bang p]
{\done \Gamma ; \Delta ; \Xi; \Gamma_1 ; \Delta_1; \bang p, \Omega \rightarrow \Xi'; \Delta'; \Gamma'}
{\done \Gamma ; \Delta ; \Xi; \Gamma_1, p; \Delta_1; \Omega \rightarrow \Xi'; \Delta'; \Gamma'}
\]
}


\subsubsection{RHS deconstruction}

The following two inference rules deconstruct the RHS list $\Omega$ from terms
created using either $\one$ or $\otimes$.

\[
\infer[\done 1]
{\done \Gamma; \Delta; \Xi; \Gamma_1 ; \Delta_1; 1, \Omega \rightarrow \outsem}
{\done \Gamma; \Delta; \Xi; \Gamma_1 ; \Delta_1; \Omega \rightarrow \outsem}
\tab
\infer[\done \otimes]
{\done \Gamma ; \Delta; \Xi; \Gamma_1; \Delta_1; A \otimes B, \Omega \rightarrow
   \outsem}
{\done \Gamma ; \Delta; \Xi; \Gamma_1; \Delta_1; A, B, \Omega \rightarrow
   \outsem}
\]


\subsubsection{Aggregates}

We also have a transition for aggregates. The aggregate starts with a set of
values $\widehat{V}$ and an accumulator initialized as $\cdot$. The second state
initiates the matching process of the LHS $A$ of the aggregate (explained in
the next section).

\[
\infer[\done \m{agg}]
{\done \Gamma; \Delta ; \Xi; \Gamma_1; \Delta_1; \aggsz{A}{B}{C}, \Omega
   \rightarrow \outsem}
{\ma \Gamma; \Delta; \Xi; \Gamma_1; \Delta_1; \cdot; A ; \cdot; \cdot;
   \aggsz{A}{B}{C}; \Omega; \Delta; \cdot \rightarrow \outsem}
\]


\subsubsection{Successful rule}

Finally, if the ordered list $\Omega$ is exhausted, then the whole execution
process is done.  Note how the output arguments match the input arguments of the
$\done$judgment.

\[
\infer[\done \m{end}]
{\done \Gamma; \Delta; \Xi; \Gamma_1; \Delta_1; \cdot \rightarrow \Xi; \Delta_1; \Gamma_1}
{}
\]


\subsection{Aggregates}

The most intricate part of the derivation process is processing comprehensions
and aggregates. For both of them, we need to perform as many derivations as the
database allows, therefore we need to deterministically check the contents of
the database until no more derivations are possible.  The matching process is
then similar to the process used for matching the body of the rule presented in
Section~\ref{sec:lld_body_match}, however we use two continuation stacks,
$\lstack{C}$ and $\lstack{P}$. In $\lstack{P}$, we put all the initial
persistent frames and in $\lstack{C}$ we put the first linear frame and then
everything else.

In order to reuse the stacks $\lstack{C}$ and $\lstack{P}$, we need to update
them by removing all the frames in $\lstack{C}$ pushed after the first linear
continuation frame.  If we tried to use those frames, we would assumed that the
linear facts used by the other frames were still in the database, but that is
not true because they have been consumed during the first application of the
comprehension.  For example, if the body is $\bang \mathtt{a(X)} \otimes
\mathtt{b(X)} \otimes \mathtt{c(X)}$ and the continuation stack has three frames
(one per fact), we cannot backtrack to the frame of $\mathtt{c(X)}$ because, at
that point, the matching process was assuming that the previous \texttt{b(X)}
linear fact was still available.  Moreover, we also need to remove the consumed
linear facts from the frames of \texttt{b(X)} and $\bang$\texttt{a(X)} in order
to make the stack fully consistent with the new database. We will see later on
how to do that.

Each aggregate derivation also needs to accumulate a list of values for each
combination. Once all combinations are performed, then the main head of the
aggregate is derived using the combined value.

The matching state for aggregates is 
$\matstatea{\Delta_N}{\lstack{C};
   \lstack{P}}{\Gamma}{\Delta}{\Omega}{\Delta' \rightarrow \Omega'}{\Sigma}$

\begin{enumerate}
   \item[$\Omega_N$] ordered list of remaining terms of the head of the rule to
   be derived;

   \item[$\Delta_N$] multi-set of linear facts that were still available after
   matching the body of the rule and all the previous aggregates. Note that
   $\Delta, \Xi = \Delta_N$;

   \item[$\Xi$] multi-set of linear facts used during the matching process of
   the body of the rule and all the previous aggregates;

   \item[$\Gamma_{1}$] set of persistent facts derived up to this point in the
   head of the rule;

   \item[$\Delta_{1}$] multi-set of linear facts derived up to this point in
   the head of the rule;

   \item[$\Delta'$] multi-set of linear facts consumed up to this point;

   \item[$\Omega'$] terms matched using $\Delta'$ up to this point;

   \item[$\m{agg}$] aggregate that is being matched;

   \item[$\Sigma$] the list of aggregated values;

   \item[$\lstack{C}$] continuation stack that contains both linear and persistent
   frames. The first frame must be linear;

   \item[$\lstack{P}$] initial part of the continuation stack with only persistent
   frames;

   \item[$\Delta$] multi-set of linear facts remaining up to this point in the
   matching process;

   \item[$\Omega$] ordered list of terms that need to be matched for the
   comprehension to be applied.

\end{enumerate}

Since aggregates accumulate values (from specific variables), we extend the
$\Psi$ context to include triplets $x : M : \tau$ (variable, term and type)
instead of pairs $M : \tau$ in order to be able to retrieve bound variables.
Remember that $\Psi$ is used for the quantification connectives in the sequent
calculus.

\subsubsection{Linear fact expressions}

The following two transitions deal with the case when there is a linear
fact expression in the body of the aggregate.

\[
\infer[\ma{AG} p~\m{first}]
{\ma{AG} \Gamma; \Delta, p_1, \Delta''; \Xi_N; \Gamma_{N1}; \Delta_{N1}; \cdot; p,
   \Omega; \cdot; \cdot; \Omega_N; \Delta_N; \Sigma \rightarrow \outsem}
{
   \begin{gathered}
      p_1, \Delta'' \prec p \\
      f = (\Delta, p_1; \Delta''; \cdot; p; \Omega;
            \cdot; \cdot) \\
      \ma{AG} \Gamma; \Delta, \Delta''; \Xi_N; \Gamma_{N1};
         \Delta_{N1}; \Xi, p_1; \Omega; f; \cdot; \Omega_N; \Delta_N; \Sigma \rightarrow \outsem
   \end{gathered}
}
\]

\[
\infer[\ma{AG} p~\m{on}~q]
{\ma{AG} \Gamma; \Delta, p_1, \Delta''; \Xi_N; \Gamma_{N1}; \Delta_{N1}; \Xi; p,
   \Omega; C_1, \lstack{C}; \lstack{P}; \Omega_N; \Delta_N; \Sigma \rightarrow \outsem}
{
   \begin{gathered}
      p_1, \Delta'' \prec p \\
      f = (\Delta_{old}; \Delta'_{old}; \Xi_{old}; q; \Omega_{old}; \Lambda; \Upsilon) \\
      f' =  (\Delta, p_1; \Delta''; \Xi; p; \Omega; q, \Lambda; \Upsilon) \\
      \ma{AG} \Gamma; \Delta, \Delta''; \Xi_N; \Gamma_{N1};
         \Delta_{N1}; \Xi, p_1; \Omega; f', f, \lstack{C}; \lstack{P}; \Omega_N;
         \Delta_N; \Sigma \rightarrow \outsem
   \end{gathered}
}
\]

\[
\infer[\ma{AG} p~\m{on}~\bang q~\lstack{C}]
{\ma{AG} \Gamma; \Delta, p_1, \Delta''; \Xi_N; \Gamma_{N1}; \Delta_{N1}; \Xi; p,
   \Omega; C_1, \lstack{C}; \lstack{P}; \Omega_N; \Delta_N; \Sigma \rightarrow \outsem}
{
   \begin{gathered}
      p_1, \Delta'' \prec p \\
      f = [\Gamma_{old}; \Delta_{old}; \Xi_{old}; q;
         \Omega_{old}; \Lambda; \Upsilon]\\
      f' = (\Delta, p_1; \Delta''; \Xi; p; \Omega; \Lambda; q, \Upsilon) \\
      \ma{AG} \Gamma; \Delta, \Delta''; \Xi_N; \Gamma_{N1};
         \Delta_{N1}; \Xi, p_1; \Omega;
         f', f, \lstack{C}; \lstack{P}; \Omega_N; \Delta_N;
         \Sigma \rightarrow \outsem
   \end{gathered}
}
\]
\[
\infer[\ma{AG} p~\m{on}~\bang q~\lstack{P}]
{\ma{AG} \Gamma; \Delta, p_1, \Delta''; \Xi_N; \Gamma_{N1}; \Delta_{N1}; \cdot; p,
   \Omega; \cdot; f, \lstack{P}; \Omega_N; \Delta_N; \Sigma \rightarrow \outsem}
{
   \begin{gathered}
      p_1, \Delta'' \prec p \\
      f = [\Gamma_{old}; \Delta_N; \cdot; q; \Omega_{old}; \cdot; \Upsilon]\\
      f' = (\Delta, p_1; \Delta''; \cdot; p; \Omega; \cdot; q, \Upsilon) \\
      \ma{AG} \Gamma; \Delta, \Delta''; \Xi_N;
            \Gamma_{N1}; \Delta_{N1}; p_1; \Omega; f'; f, \lstack{P}; \Omega_N;
            \Delta_N; \Sigma \rightarrow \outsem
   \end{gathered}
}
\]

\[
\infer[\ma{AG} p~\m{fail}]
{\ma{AG} \Gamma; \Delta; \Xi_N; \Gamma_{N1}; \Delta_{N1}; \Xi; p, \Omega;
   \lstack{C}; \lstack{P}; \Omega_N; \Delta_N; \Sigma \rightarrow \outsem}
{\conta{AG} \Gamma; \Delta_N; \Xi_N; \Gamma_{N1}; \Delta_{N1}; \lstack{C};
   \lstack{P}; \Omega_N;
   \Sigma \rightarrow \outsem}
\]



\subsubsection{Persistent fact expressions}

The transitions for dealing with persistent facts are similar to the previous
ones.


\begin{multline}
\transx{
   \matstatea{\Delta_N}{\cdot;
      \lstack{P}}{\Gamma, p_1, \Gamma''}{\Delta}{\bang p, \Omega}{\Delta' \rightarrow
         \Omega'}{\Sigma}
}
{
   \matstatea{\Delta_N}{\cdot; \pframe{\Gamma''}{\Delta}{\bang
   p}{\Omega}{\Delta'}{\Omega'}, \lstack{P}}{\Gamma, p_1, \Gamma''}{\Delta}{\Omega}
   {\Delta' \rightarrow \Omega' \otimes \bang p}{\Sigma}
} \tag{agg match \bang p ok $\lstack{P}$}
\end{multline}

\begin{multline}
\transx{
   \matstatea{\Delta_N}{\lstack{C};
      \lstack{P}}{\Gamma, p_1, \Gamma''}{\Delta}{\bang p, \Omega}{\Delta' \rightarrow
         \Omega'}{\Sigma}
}
{
   \matstatea{\Delta_N}{\pframe{\Gamma''}{\Delta}{\bang
   p}{\Omega}{\Delta'}{\Omega'}, \lstack{C} ; \lstack{P}}{\Gamma, p_1, \Gamma''}{\Delta}{\Omega}
   {\Delta' \rightarrow \Omega' \otimes \bang p}{\Sigma}
} \tag{agg match \bang p ok $\lstack{C}$}
\end{multline}

\[
\trans{
   \matstatea{\Delta_N}{\lstack{C}; \lstack{P}}{\Gamma}{\Delta}{\bang p,
      \Omega}{\Delta' \rightarrow \Omega'}{\Sigma}
}
{
   \contstatea{\Delta_N}{\lstack{C} ; \lstack{P}}{\Gamma}{\Sigma}
} \tag{agg match \bang p fail}
\]



\subsubsection{Deconstruct body}


\begin{multline}
\transx{
   \matstatea{\Delta_N}{\lstack{C};
      \lstack{P}}{\Gamma}{\Delta}{X \otimes Y, \Omega}{\Delta' \rightarrow
         \Omega'}{\Sigma}
}
{
   \matstatea{\Delta_N}{\lstack{C};
      \lstack{P}}{\Gamma}{\Delta}{X, Y, \Omega}{\Delta' \rightarrow
         \Omega'}{\Sigma}
} \tag{agg match $\otimes$}
\end{multline}

\begin{multline}
\transx{
   \matstatea{\Delta_N}{\lstack{C};
      \lstack{P}}{\Gamma}{\Delta}{\one, \Omega}{\Delta' \rightarrow
         \Omega'}{\Sigma}
}
{
   \matstatea{\Delta_N}{\lstack{C};
      \lstack{P}}{\Gamma}{\Delta}{\Omega}{\Delta' \rightarrow
         \Omega'}{\Sigma}
      } \tag{agg match $\one$}
\end{multline}



\subsubsection{Successful match}

When the aggregate body finally matches, we retrieve the term for variable $x$
(the aggregate variable) and add it to the list $\Sigma$.

\[
\trans{
   \matstatea{\Delta_N}{\lstack{C};
      \lstack{P}}{\Gamma}{\Delta}{\cdot}{\Delta' \rightarrow
         \Omega'}{\Sigma}
}
{
   \fixstatea{\Delta}{\Xi; \Delta'}{\lstack{C}; \lstack{P}}{\Gamma}{\Sigma}
}
\]


\subsubsection{Continuation stack update}

As we said before, to update the continuation stacks, we need remove to all the
frames except the first linear frame and remove the consumed linear facts from
the remaining frames so that they are still valid for the next application of
the aggregate.  The judgment that updates the stack has the form
$\fixstatea{\Delta}{\Xi; \Delta'}{\lstack{C};
   \lstack{P}}{\Gamma}{\Sigma}$, where:

\begin{enumerate}
   \item[$\Omega_N$] ordered list of remaining terms of the head of the rule to
   be derived;
   \item[$\Delta$] multi-set of linear facts that were still available after
   matching the body of the rule and the body of the aggregate;
   \item[$\Xi$] multi-set of linear facts used during the matching process of
   the body of the rule and all the previous aggregates;
   \item[$\Delta'$] multi-set of linear facts consumed by the aggregate body;
   \item[$\Gamma_{1}$] set of persistent facts derived by the head of the rule
   and all the previous aggregates;
   \item[$\Delta_{1}$] multi-set of linear facts derived by the head of the
   rule and all the previous aggregates;
   \item[$\m{agg}$] the current aggregate;
   \item[$\Sigma$] list of accumulated values;
   \item[$\lstack{C}, \lstack{P}$] continuation stacks for the comprehension;
   \item[$\Gamma$] set of usable persistent facts.
\end{enumerate}

\subsubsection{Remove linear continuation frames}

To remove all linear continuation frames except the first one, we simply go
through all the frames in the stack $\lstack{C}$ until only one frame remains.
This last frame and stack $\lstack{P}$ are then updated by removing $\Delta'$
from its contexts.

\[
\trans{
   \fixstatea{\Delta}{\Xi; \Delta'}{\_, f, \lstack{C}; \lstack{P}}{\Gamma}{\Sigma}
}
{
   \fixstatea{\Delta}{\Xi; \Delta'}{f, \lstack{C}; \lstack{P}}{\Gamma}{\Sigma}
} \tag{agg fix rec}
\]

\[
\underset{
   \begin{gathered}
   \Pi(\m{agg}) = \forall_{\widehat{v}, \Sigma'}.
   (\defstwo{agg}{\widehat{v}}{\Sigma'} \lolli ((\lambda x. C x)\Sigma' \with (\forall_{\widehat{x}, \sigma}.
                                                (A \lolli B \otimes
                                                 \defstwo{agg}{\widehat{v}}{\sigma
                                                 ::\Sigma'})))) \\
                                                 f' = \texttt{remove}(f, \Delta') \\
                                                 \lstack{P'} = \texttt{remove}(\lstack{P}, \Delta') \\
   V = \Psi(\sigma)
   \end{gathered}
}
{
   \transx{
      \fixstatea{\Delta}{\Xi; \Delta'}{f; \lstack{P}}{\Gamma}{\Sigma}
   }
   {
      \derstatea{\Delta}{\Xi; \Delta'}{\gammanew}{\deltanew}{V :: \Sigma}{f';
         \lstack{P'}}{B\{\Psi(\widehat{x}), V / \widehat{x}, \sigma \}}
   }
}
   \tag{agg fix end1}
\]

\[
\underset{
   \begin{gathered}
   \Pi(\m{agg}) = \forall_{\widehat{v}, \Sigma'}.
   (\defstwo{agg}{\widehat{v}}{\Sigma'} \lolli ((\lambda x. C x)\Sigma' \with (\forall_{\widehat{x}, \sigma}.
                                                (A \lolli B \otimes
                                                 \defstwo{agg}{\widehat{v}}{\sigma
                                                 ::\Sigma'})))) \\
                                                 \lstack{P'} = \texttt{remove}(\lstack{P}, \Delta') \\
   V = \Psi(\sigma)
   \end{gathered}
}
{
   \trans{
      \fixstatea{\Delta}{\Xi; \Delta'}{\cdot; \lstack{P}}{\Gamma}{\Sigma}
   }
   {
      \derstatea{\Delta}{\Xi, \Delta'}{\Gamma_{N1}}{\Delta_{N1}}{V :: \Sigma}{\cdot;
         \lstack{P}'}{B\{\Psi(\widehat{x}), V / \widehat{x}, \sigma \}}
   }
} \tag{agg fix end2}
\]


\subsubsection{Aggregate backtracking}

If the aggregate match fails, we need to backtrack to the next candidate fact.
The backtracking state 
has the form
$\contstatea{\Delta_N}{\lstack{C} ; \lstack{P}}{\Gamma}{\Sigma}$, where:

\begin{enumerate}
   \item[$\Omega_N$] ordered list of remaining terms of the head of the rule to
   be derived;
   \item[$\Delta_N$] multi-set of linear facts that were still available after
   matching the body of the rule and the body of the aggregate;
   \item[$\Xi$] multi-set of linear facts used during the matching process of
   the body of the rule and all the previous aggregates;
   \item[$\Gamma_{1}$] set of persistent facts derived by the head of the rule
   and all the previous aggregates;
   \item[$\Delta_{1}$] multi-set of linear facts derived by the head of the
   rule and all the previous aggregates;
   \item[$\m{agg}$] the current aggregate;
   \item[$\Sigma$] list of accumulated values.
   \item[$\lstack{C}, \lstack{P}$] continuation stacks for the comprehension;
   \item[$\Gamma$] set of usable persistent facts.
\end{enumerate}

\paragraph{Using the $\lstack{C}$ stack}

The following 4 state transitions use the $\lstack{C}$ stack, the stack where the
first continuation frame is linear, to perform backtracking.

{\footnotesize
\[
\infer[\conta \m{next}~C~p]
{\conta \Gamma; \Delta_N; \Xi_N; \Gamma_{N1}; \Delta_{N1}; (\Delta; p_1, \Delta''; \Xi; p; \Omega; \Lambda; \Upsilon), C; P; AG; \Omega_N; T \rightarrow \Xi'; \Delta'; \Gamma'}
{\ma \Gamma; \Delta; \Xi_N; \Gamma_{N1}; \Delta_{N1}; \Xi; \Omega; (\Delta, p_1; \Delta''; \Xi; p; \Omega; \Lambda; \Upsilon), C; P; AG; \Omega_N; \Delta_N; T \rightarrow \Xi'; \Delta'; \Gamma'}
\]

\[
\infer[\conta \m{next}~C~\bang p]
{\conta \Gamma; \Delta_N; \Xi_N; \Gamma_{N1}; \Delta_{N1}; [p_1, \Gamma'; \Delta; \Xi; \bang p; \Omega; \Lambda; \Upsilon], C; P; AG; \Omega_N; T \rightarrow \Xi'; \Delta'; \Gamma'}
{\ma \Gamma; \Delta; \Xi_N; \Gamma_{N1}; \Delta_{N1}; \Xi; \Omega; [\Gamma'; \Delta; \Xi; \bang p; \Omega; \Lambda; \Upsilon], C; P; AG; \Omega_N; \Delta_N; T \rightarrow \Xi'; \Delta'; \Gamma'}
\]

\[
\infer[\conta \m{next}~C~\m{empty}~p]
{\conta \Gamma; \Delta_N; \Xi_N; \Gamma_{N1}; \Delta_{N1}; (\Delta; \cdot; \Xi; p; \Omega; \Lambda; \Upsilon), C; P; AG; \Omega_N; T \rightarrow \Xi'; \Delta'; \Gamma'}
{\conta \Gamma; \Delta_N; \Xi_N; \Gamma_{N1}; \Delta_{N1}; C; P; AG; \Omega_N; T \rightarrow \Xi'; \Delta'; \Gamma'}
\]

\[
\infer[\conta \m{next}~C~\m{empty}~\bang p]
{\conta \Gamma; \Delta_N; \Xi_N; \Gamma_{N1}; \Delta_{N1}; [\cdot; \Delta; \Xi; \bang p; \Omega; \Lambda; \Upsilon], C; P; AG; \Omega_N; T \rightarrow \Xi'; \Delta'; \Gamma'}
{\conta \Gamma; \Delta_N; \Xi_N; \Gamma_{N1}; \Delta_{N1}; C; P; AG; \Omega_N; T \rightarrow \Xi'; \Delta'; \Gamma'}
\]
}



\paragraph{Using the $\lstack{P}$ stack}

The following 2 state transitions rules use the $\lstack{P}$ stack instead, the stack where all
continuation frames are persistent.


\begin{multline}
\transx{
   \contstatea{\deltan}{\cdot ; \pframe{p_1, \Gamma''}{\Delta}{\bang
   p}{\Omega}{\Delta'}{\Omega'}, \lstack{P}}{\Gamma}{\Sigma}
}
{
   \matstatea{\deltan}{\cdot; \pframe{\Gamma''}{\Delta}{\bang p}
      {\Omega}{\Delta'}{\Omega'}, \lstack{P}}{\Gamma}{\Delta}{p,
      \Omega}{\Delta' \rightarrow \Omega' \otimes \bang p}{\Sigma}
} \tag{agg next \bang p $\lstack{P}$}
\end{multline}

\[
\trans{
   \contstatea{\deltan}{\cdot; \pframe{\cdot}{\Delta}{\bang
   p}{\Omega}{\Delta'}{\Omega'}, \lstack{P}}{\Gamma}{\Sigma}
}
{
   \contstatea{\deltan}{\cdot ; \lstack{P}}{\Gamma}{\Sigma}
} \tag{agg next \bang frame $\lstack{P}$}
\]


\paragraph{Aggregate done}

If both the $\lstack{C}$ and $\lstack{P}$ stacks are empty, backtracking is
impossible and the aggregate is done. The final head of the aggregate is then
derived along with the rest of the rule's head.


\[
\underset{
   \Pi(\m{agg}) = \forall_{\widehat{v}, \Sigma'}.
   (\defstwo{agg}{\widehat{v}}{\Sigma'} \lolli ((\lambda x. C x)\Sigma' \with (\forall_{\widehat{x}, \sigma}.
                                                (A \lolli B \otimes
                                                 \defstwo{agg}{\widehat{v}}{\sigma
                                                 ::\Sigma'}))))
}
{
\trans{
   \contstatea{\Delta_N}{\cdot ; \cdot}{\Gamma}{\Sigma}
}
{
   \derstatex{\Gamma}{\Delta_N}{\Xi}{\Gamma_{N1}}{\Delta_{N1}}{(\lambda x.
         C\{\Psi(\widehat{v})/\widehat{v}\} x) \Sigma,
      \Omega_N}
}
}
\]


\subsubsection{Aggregate Derivation}

After updating the continuation stacks, the subhead of the aggregate is derived.
The derivation state has the form
$\derstatea{\Delta}{\Xi}{\Gamma_1}{\Delta_1}{\Sigma}{\lstack{C};
   \lstack{P}}{\Omega}$, where:

\begin{enumerate}
   \item[$\Omega_N$] ordered list of remaining terms of the head of the rule to
   be derived;
   \item[$\Delta$] multi-set of remaining linear facts that can be used for
   the next aggregate applications.
   \item[$\Xi$] multi-set of linear facts consumed both by the body of the rule
   and previous aggregate applications;
   \item[$\Gamma_1$] set of persistent facts derived by the head of the rule,
   previous aggregates and current derivation;
   \item[$\Delta_1$] multi-set of linear facts derived by the head of the rule,
   previous aggregates and current derivation;
   \item[$\m{agg}$] current aggregate symbol;
   \item[$\Sigma$] accumulated list of values of the aggregate;
   \item[$\lstack{C}, \lstack{P}$] new continuation stacks;
   \item[$\Gamma$] set of persistent facts;
   \item[$\Omega$] ordered list of terms to derive.
\end{enumerate}


\[
\trans{
   \derstatea{\Delta}{\Xi}{\gammanew}{\deltanew}{\Sigma}{\lstack{C};
      \lstack{P}}{p, \Omega}
}
{
   \derstatea{\Delta}{\Xi}{\gammanew}{\deltanew, p}{\Sigma}{\lstack{C};
      \lstack{P}}{\Omega}
} \tag{agg new p}
\]

\[
\trans{
   \derstatea{\Delta}{\Xi}{\gammanew}{\deltanew}{\Sigma}{\lstack{C};
      \lstack{P}}{\bang p, \Omega}
}
{
   \derstatea{\Delta}{\Xi}{\gammanew, p}{\deltanew}{\Sigma}{\lstack{C};
      \lstack{P}}{\Omega}
} \tag{agg new \bang p}
\]

\[
\trans{
   \derstatea{\Delta}{\Xi}{\gammanew}{\deltanew}{\Sigma}{\lstack{C};
      \lstack{P}}{X \otimes Y, \Omega}
}
{
   \derstatea{\Delta}{\Xi}{\gammanew, p}{\deltanew}{\Sigma}{\lstack{C};
      \lstack{P}}{X, Y, \Omega}
} \tag{agg new $\otimes$}
\]

\[
\trans{
   \derstatea{\Delta}{\Xi}{\gammanew}{\deltanew}{\Sigma}{\lstack{C};
      \lstack{P}}{\one, \Omega}
}
{
   \derstatea{\Delta}{\Xi}{\gammanew, p}{\deltanew}{\Sigma}{\lstack{C};
      \lstack{P}}{\Omega}
} \tag{agg new $\one$}
\]

\[
\trans{
   \derstatea{\Delta}{\Xi}{\gammanew}{\deltanew}{\Sigma}{\lstack{C};
      \lstack{P}}{\cdot}
}
{
   \contstatea{\Delta}{\lstack{C} ; \lstack{P}}{\Gamma}{\Sigma}
} \tag{agg next}
\]


This completes the specification of LLD.



\section{Soundness Proof}

Now that we have presented both the HLD and LLD semantics, we are in position to
start building our soundness theorem.  The soundness theorem proves that if a
rule was successfully derived in the LLD semantics then it can also be derived
in the HLD semantics. Since the HLD semantics are so close to linear logic, we
prove that our language has a determined, correct, proof search behavior when
executing programs. However, the completeness theorem cannot be proven since LLD
lacks the non-determinism inherent in HLD.

First and foremost, we need to prove some auxiliary theorems and definitions
that will be used during the soundness theorem. Note that $\outsem$ is a
short-hand for the output contexts of LLD and HLD.

\subsection{Term equivalence}

The first definition defines the equality between two multi-sets of terms.  Two
multi-sets $A$ and $B$ are equal, $\feq{A}{B}$, when they have the same
constituent atoms.

\[
\infer[\equiv p]
{\feq{p, A}{p, B}}
{\feq{A}{B}}
\tab
\infer[\equiv \bang p]
{\feq{\bang p, A}{\bang p, B}}
{\feq{A}{B}}
\tab
\infer[\equiv 1~L]
{\feq{1, A}{B}}
{\feq{A}{B}}
\tab
\infer[\equiv 1~R]
{\feq{A}{1, B}}
{\feq{A}{B}}
\]

\[
\infer[\equiv \cdot]
{\feq{\cdot}{\cdot}}
{}
\tab
\infer[\equiv \otimes~L]
{\feq{A \otimes B, C}{D}}
{\feq{A, B, C}{D}}
\tab
\infer[\equiv \otimes~R]
{\feq{A}{B \otimes C, D}}
{\feq{A}{B, C, D}}
\]

\begin{theorem}[Match equivalence]
If two multi-sets are equivalent, $\feq{A_1, \dotsc, A_n}{B_1, \dotsc, B_m}$,
   and we can match $A_1 \otimes \dotsb \otimes A_n$ in HLD such that $\mz
   \Gamma ; \Delta \rightarrow A_1 \otimes \dotsb \otimes A_n$ then $\mz \Gamma;
   \Delta \rightarrow B_1 \otimes \dotsb \otimes B_m$ is also true.
\end{theorem}
\begin{proof}
By straightforward induction on the first assumption.
\end{proof}

\subsection{Well-formed continuation frames}

We now define the concept of a well-formed frame given initial linear and
persistent contexts and a term $A$ that needs to be matched.

\begin{definition}[Well-formed frame]

Consider a triplet $A; \Gamma; \Delta_{N}$ where $A$ is a term, $\Gamma$ is a
multi-set of persistent resources and $\Delta_{N}$ a multi-set of linear
resources. A frame $f$ is well-formed iff:

\begin{enumerate}[leftmargin=*]
   \item Linear frame $f = (\Delta, p_1; \Delta'; \Xi_1, \dotsc, \Xi_m; p;
         \Omega_1, \dotsc, \Omega_n; \Lambda_1, \dotsc, \Lambda_m; \Upsilon_1,
         \dotsc, \Upsilon_k)$

   \begin{enumerate}
      \item $\feq{p, \Omega_1, \dotsc, \Omega_n, \Lambda_1, \dotsc, \Lambda_m,
         \Upsilon_1, \dotsc, \Upsilon_k}{A}$ (the remaining terms and already
               matched terms are equivalent to the initial body $A$);
      \item $\mz \Xi_1, \dotsc, \Xi_m \rightarrow \Lambda_1 \otimes \dotsb \otimes
      \Lambda_m$ and $\mz \Xi_i \rightarrow \Lambda_i$ for every $i$;

      \item $\Delta, \Delta', \Xi, p_1 = \Delta_{N}$ (available facts, candidate
            facts for $p$, consumed facts and the linear fact used for $p$,
            respectively, are the same as the initial $\Delta_{N}$);

      \item $\mz \Gamma; \cdot \rightarrow \Upsilon_1 \otimes \dotsb \otimes
      \Upsilon_k$ (past persistent facts can be matched with $\Gamma$).

   \end{enumerate}
   \item Persistent frame $f = [\Gamma'; \Delta; \Xi_1, \dotsc, \Xi_m; \bang p;
         \Omega_1, \dotsc, \Omega_n; \Lambda_1, \dotsc, \Lambda_m; \Upsilon_1,
         \dotsc, \Upsilon_k]$

      \begin{enumerate}
         \item $\feq{\bang p, \Omega_1, \dotsc, \Omega_n, \Lambda_1, \dotsc,
                     \Lambda_m, \Upsilon_1, \dotsc, \Upsilon_k}{A}$;
         \item $\mz \Xi_1, \dots, \Xi_m \rightarrow \Lambda_1 \otimes \dotsb \otimes
                     \Lambda_m$ and $\mz \Xi_i \rightarrow \Lambda_i$ for every $i$;
         \item $\Delta, \Xi = \Delta_{N}$;
         \item $\mz \Gamma; \cdot \rightarrow \bang p \otimes \Upsilon_1 \otimes
                     \dotsb \otimes \Upsilon_k$;
         \item $\Gamma' \subset \Gamma$ (remaining candidates are a subset of
                     $\Gamma$).
      \end{enumerate}
\end{enumerate}
\end{definition}


\begin{definition}[Well-formed stack]
A continuation stack $C$ is well-formed iff every frame is well-formed.
\end{definition}

Given the previous definitions, we can now define what it means for a matching
judgment to be well-formed.

\begin{definition}[Well-formed body match]

$\mo \Gamma; \Delta; \Xi; \Omega; H; C; R \rightarrow \outsem$ is well-formed in relation to a triplet $A; \Gamma; \Delta_{N}$ iff:

\begin{itemize}[leftmargin=*]
   \item $\Delta, \Xi = \Delta_{N}$
   \item $C$ is well-formed in relation to $A; \Gamma; \Delta_{N}$ and:
   \begin{itemize}[leftmargin=\secondm]
      \item If $C = \cdot$
   
      $\feq{\Omega}{A}$.
   
      \item If $C = (\Delta_a, p_1; \Delta_b; \Xi''; p; \Omega'; \Lambda_1,
            \dotsc, \Lambda_m; \Upsilon_1, \dotsc, \Upsilon_k), C'$
   
      \begin{itemize}[leftmargin=\thirdm]
         \item $\feq{\Omega'}{\Omega}$;
         \item $p_1 \in \Xi$ and $\mz \Gamma; p_1 \rightarrow p$;
         \item $\Xi = \Xi'', p_1$;
         \item $\Delta = \Delta_a, \Delta_b$.
      \end{itemize}
      \item If $C = [\Gamma'; \Delta''; \Xi''; \bang p; \Omega'; \Lambda_1,
      \dotsc, \Lambda_m; \Upsilon_1, \dotsc, \Upsilon_k], C'$
      \begin{itemize}[leftmargin=\thirdm]
         \item $\feq{\Omega}{\Omega'}$;
         \item $\Xi = \Xi''$;
         \item $\Delta = \Delta''$.
      \end{itemize}
   \end{itemize}
\end{itemize}

\end{definition}

\begin{definition}[Well-formed comprehension match]
$\mc \Gamma; \Delta; \Xi_N; \Gamma_{N1}; \Delta_{N1}; \Xi; \Omega; C; P;
\compsz{A}{B}; \Omega_N; \Delta_N \rightarrow \outsem$ is
well-formed in relation to a triplet $A; \Gamma; \Delta_{N}$ iff:

\begin{itemize}[leftmargin=*]
   \item $P$ is composed solely of persistent frames.
   \item $C$ is composed of either linear or persistent frames, but the first
   frame is linear.
   \item $\Delta, \Xi = \Delta_{N}$
   \item $C$ and $P$ are well-formed in relation to $A; \Gamma; \Delta_{N}$ and
   follow the same rules presented before in "Well-formed body match" as a stack
   $C, P$.
\end{itemize}
\end{definition}

\begin{definition}[Well-formed aggregate match]
$\ma \Gamma; \Delta; \Xi_N; \Gamma_{N1}; \Delta_{N1}; \Xi; \Omega; C; P;
\aggsz{A}{B}{C}; \Omega_N; \Delta_N; T \rightarrow \outsem$ is
well-formed in relation to a triplet $A; \Gamma; \Delta_{N}$ iff the rules in
"Well-formed comprehension match" also apply.

\end{definition}


\subsection{Soundness of matching}

The soundness theorem will be proven into two main steps. First, we prove that
performing a rule match at LLD is sound in relation to HLD and then we prove
that the derivation of head terms in LLD is also sound.

In order to prove the soundness of matching, we want to reconstitute a valid
$\mz$in HLD from a valid $\mo$in LLD. However, LLD may fail during matching,
therefore our body match lemma needs to handle unsuccessful matches. In order to
be able to use induction, we must assume a matching proposition $\mo$that
already contains some continuation frames in stack $C$ that is well-formed in
relation to the rule's body $A$ and initial database.

Our lemma also needs to apply to our continuation judgment $\contlld$, because when inverting some of
the matching assumptions, we get a continuation proposition. Apart from an unsuccessful match, we deal
with two situations during a successful match: (1) we succeed without needing to backtrack to a frame
in stack $C$ or (2) we need to backtrack to a frame in $C$. The complete lemma is stated and proven below.

\begin{lemma}[Body match soundness]\label{thm:body_match}
   
Given a rule $A \lolli H$, consider a triplet $T = A; \Gamma; \Delta_{N}$ and a context $\Delta_{N} = \Delta_1, \Delta_2, \Xi$.

If $\mo \Gamma; \Delta_1, \Delta_2; \Xi; \Omega; H; C; R \rightarrow \outsem$ is well-formed in relation to $T$ then either:

\begin{itemize}[leftmargin=*]
   \item Match fails:
   \begin{itemize}[leftmargin=\secondm]
      \item $\cont \cdot; H; R; \Gamma \rightarrow \outsem$
   \end{itemize}

   \item Match succeeds with no backtracking to frames of stack $C$:
   \begin{itemize}[leftmargin=\secondm]
      \item $\mz \Gamma; \Xi, \Delta_2 \rightarrow A$
      \item $\mo \Gamma; \Delta_1; \Xi, \Delta_2; \cdot; H; C'', C; R
         \rightarrow \outsem$ (well-formed in relation to $T$)
      \item $\mo \Gamma; \Delta_1; \Xi, \Delta_2; \Omega; H; C; (\cdot, \Delta_N) \rightarrow \outsem$ (well-formed in relation to $T$)
   \end{itemize}

   \item Match succeeds with backtracking to a linear frame:
   \begin{itemize}[leftmargin=\secondm]
      \item $\mz \Gamma; \Xi_1, \dotsc, \Xi_m, p_2, \Xi_c \rightarrow A$
      \item $\exists_{f \in C}. f = (\Delta_a; \Delta_{b_1}, p_2, \Delta_{b_2}; p;
            \Omega_1, \dotsc, \Omega_k; \Xi_1 .. \Xi_m; \Lambda_1, \dotsc,
            \Lambda_m; \Upsilon_1, \dotsc, \Upsilon_n)$

      \item $C = C', f, C''$

      \item $f$ turns into $f' = (\Delta_a, \Delta_{b_1}, p_2;
            \Delta_{b_2}; p; \Omega_1, \dotsc, \Omega_k; \Xi_1, \dotsc, \Xi_m;
            \Lambda_1, \dotsc, \Lambda_m; \Upsilon_1, \dotsc, \Upsilon_n)$

      \item $\mo \Gamma; \Delta_c; \Xi_1, \dotsc, \Xi_m, p_2, \Xi_c; \cdot; H;
            C''', f', C''; R \rightarrow \outsem$ (well-formed in
                  relation to $T$)
      \item $\Delta_c = (\Delta_1, \Delta_2, \Xi) - (\Xi_1, \dotsc, \Xi_m, p_2, \Xi_c)$
   \end{itemize}

   \item Match succeeds with backtracking to a persistent frame:
   \begin{itemize}[leftmargin=\secondm]
      \item $\mz \Gamma; \Xi_1, \dotsc, \Xi_m, \Delta_{c_2} \rightarrow A$
      \item $\exists_{f \in C}. = f = [\Gamma_1, p_2, \Gamma_2; \Delta_{c_1}, \Delta_{c_2}; \Xi_c; \bang
         p; \Omega_1, \dotsc, \Omega_k; \Lambda_1, \dotsc, \Lambda_m;
         \Upsilon_1, \dotsc, \Upsilon_n]$
      \item $C = C', f, C''$
      \item $f$ turns into $f' = [\Gamma_2; \Delta_{c_1}, \Delta_{c_2}; \Xi_1, \dotsc,
         \Xi_m; \bang p; \Omega_1, \dotsc, \Omega_k; \Lambda_1, \dotsc,
         \Lambda_m; \Upsilon_1, \dotsc, \Upsilon_n]$
      \item $\mo \Gamma; \Delta_{c_1}; \Xi_1, \dotsc, \Xi_m, \Delta_{c_2};
         \cdot; H; C'', f', C''; R \rightarrow \outsem$ (well-formed in
            relation to $T$)
      \item $\Delta_{c_1}, \Delta_{c_2} = (\Delta_1, \Delta_2,
            \Xi) - (\Xi_1, \dotsc, \Xi_m)$
   \end{itemize}
\end{itemize}

If $\cont C; H; R; \Gamma \rightarrow \outsem$ and $C$ is well-formed in relation to $T$ then either:

\begin{itemize}[leftmargin=*]
   \item Match fails:
   \begin{itemize}[leftmargin=\secondm]
      \item $\cont \cdot; H; R; \Gamma \rightarrow \outsem$
   \end{itemize}

   \item Match succeeds with backtracking to a linear frame:
   \begin{itemize}[leftmargin=\secondm]
      \item $\mz \Gamma; \Xi_1, \dotsc, \Xi_m, p_2, \Xi_c \rightarrow A$
      \item $\exists_{f \in C}. f = (\Delta_a; \Delta_{b_1}, p_2, \Delta_{b_2}; p;
            \Omega_1, \dotsc, \Omega_k; \Xi_1 .. \Xi_m; \Lambda_1, \dotsc,
            \Lambda_m; \Upsilon_1, \dotsc, \Upsilon_n)$

      \item $C = C', f, C''$

      \item $f$ turns into $f' = (\Delta_a, \Delta_{b_1}, p_2;
            \Delta_{b_2}; p; \Omega_1, \dotsc, \Omega_k; \Xi_1, \dotsc, \Xi_m;
            \Lambda_1, \dotsc, \Lambda_m; \Upsilon_1, \dotsc, \Upsilon_n)$

      \item $\mo \Gamma; \Delta_c; \Xi_1, \dotsc, \Xi_m, p_2, \Xi_c; \cdot; H;
            C''', f', C''; R \rightarrow \outsem$ (well-formed in
                  relation to $T$)
      \item $\Delta_c = (\Delta_1, \Delta_2, \Xi) - (\Xi_1, \dotsc, \Xi_m, p_2, \Xi_c)$
   \end{itemize}

   \item Match succeeds with backtracking to a persistent frame:
   \begin{itemize}[leftmargin=\secondm]
      \item $\mz \Gamma; \Xi_1, \dotsc, \Xi_m, \Delta_{c_2} \rightarrow A$
      \item $\exists_{f \in C}. = f = [\Gamma_1, p_2, \Gamma_2; \Delta_{c_1}, \Delta_{c_2}; \Xi_c; \bang
         p; \Omega_1, \dotsc, \Omega_k; \Lambda_1, \dotsc, \Lambda_m;
         \Upsilon_1, \dotsc, \Upsilon_n]$
      \item $C = C', f, C''$
      \item $f$ turns into $f' = [\Gamma_2; \Delta_{c_1}, \Delta_{c_2}; \Xi_1, \dotsc,
         \Xi_m; \bang p; \Omega_1, \dotsc, \Omega_k; \Lambda_1, \dotsc,
         \Lambda_m; \Upsilon_1, \dotsc, \Upsilon_n]$
      \item $\mo \Gamma; \Delta_{c_1}; \Xi_1, \dotsc, \Xi_m, \Delta_{c_2};
         \cdot; H; C'', f', C''; R \rightarrow \outsem$ (well-formed in
            relation to $T$)
      \item $\Delta_{c_1}, \Delta_{c_2} = (\Delta_1, \Delta_2,
            \Xi) - (\Xi_1, \dotsc, \Xi_m)$
   \end{itemize}
\end{itemize}
\end{lemma}

\begin{proof}
   Proof by mutual induction. In $\mo$on the size of $\Omega$ and on $\contlld$, first on the size of the second argument of the frame ($\Delta''$ and $\Gamma''$) and then on the size of the stack $C$. Sub-cases:
   
\begin{itemize}[leftmargin=*]
   \item $\mo p~\m{first}$, $\mo p~\m{on}~q$, $\mo p~\m{on}~\bang q$, $\mo \bang p~\m{first}$ $\mo \bang p~\m{on}~q$, $\mo \bang p~\m{on}~\bang q$, $\mo \otimes$
   
   When inverting the assumption, the well-formedness of the stack and match are
   proven straightforwardly using the well-formedness of the assumption and the
   match equivalence theorem. The induction hypothesis is then applied
   straightforwardly.
   
   \item $\mo \m{end}$
   
   Trivial.
   
   \item $\mo p~\m{fail}$, $\mo \bang p~\m{fail}$
   
   Invert the assumption and apply induction hypothesis on the $\cont$assumption.
   
   \item $\cont \m{next}~\m{rule}$
   
   Match fails.
   
   \item $\cont p~\m{next}, \cont \bang p~\m{next}$
   
   When inverting the assumption, we get a $\mo$proposition that is trivially
   proven to be well-formed in relation to $T$. Using induction hypothesis on
   this assumption, we have 3 sub-cases:
   
   \begin{itemize}[leftmargin=\secondm]
      \item Match fails: trivial.
      \item Match succeeds with no backtracking: the frame that we updated is the successful frame to backtrack to.
      \item Match succeeds with backtracking: $f \in C$ from the new assumption is the frame we need.
   \end{itemize}
   
   \item $\cont p~\m{no}~\m{more}$, $\cont \bang p~\m{no}~\m{more}$
   
   Invert the assumption to apply the induction hypothesis.
\end{itemize}
\end{proof}

For the induction hypothesis to be applicable in in Lemma~\ref{thm:body_match} there must be
a relation between the judgments $\mo$and $\contlld$.
We can define a lexicographic ordering $A \prec B$, meaning that proposition $A$ has a smaller proof than proposition $B$ (potentially $A$ is sub-proof of $B$),
or alternatively, $A$ is "executed later" than $B$.
The specific ordering is as follows:

\begin{enumerate}[leftmargin=*]
   \item $\cont C; H; R; \Gamma \rightarrow \outsem \prec \cont C', C; H; R; \Gamma \rightarrow \outsem$\\
   The continuation must use the top of the stack $C'$ before using $C$;
   \item $\cont C', (\Delta, \Delta_1; \Delta_2; \Xi; p; \Omega; \Lambda; \Upsilon), C; H; R; \Gamma \rightarrow \outsem$\\
   \hspace*{1cm}$\prec \cont C'', (\Delta; \Delta_1, \Delta_2; \Xi; p; \Omega; \Lambda; \Upsilon), C; H; R; \Gamma \rightarrow \outsem$\\
   A continuation frame with more candidates has more steps to do than a frame with less candidates;
   \item $\cont C', [\Gamma_1; \Delta; \Xi; \bang p; \Omega; \Lambda; \Upsilon], C; H; R; \Gamma \rightarrow \outsem$\\
   \hspace*{1cm} $\prec \cont C'', [\Gamma_2, \Gamma_3; \Delta; \Xi; \bang p; \Omega; \Lambda; \Upsilon], C; H; R; \Gamma \rightarrow \outsem$\\
      Same as the previous one;
   \item $\cont C; H; R; \Gamma \rightarrow \outsem \prec \mo \Gamma; \Delta; \Xi; \Omega; H; C', C; R \rightarrow \outsem$\\
   Same as (1);
   \item $\mo \Gamma; \Delta; \Xi; \Omega; H; C; R \rightarrow \outsem \prec \cont C', C; H; R; \Gamma \rightarrow \outsem$\\
   Same as the previous one;
   \item $\mo \Gamma; \Delta''; \Xi''; \Omega'; H; C', C; R \rightarrow \outsem \prec \mo \Gamma; \Delta; \Xi; \Omega; H; C; R \rightarrow \outsem$ as long as $\Omega' \prec \Omega$\\
   Adding continuation frames to the stack makes the proof smaller as long as $\Omega$ is also smaller; 
   \item $\mo \Gamma; \Delta; \Xi; \Omega; H; C', (\Delta, \Delta_1; \Delta_2;
         \Xi; p; \Omega; \Lambda; \Upsilon), C; R \rightarrow \outsem$\\
   \hspace*{1cm} $\prec \mo \Gamma; \Delta''; \Xi''; \Omega'; C'', (\Delta; \Delta_1, \Delta_2; \Xi; p; \Omega; \Lambda; \Upsilon), C; R \rightarrow \outsem$\\
   Same as (2);
   \item $\mo \Gamma; \Delta; \Xi; \Omega; H; C', [\Gamma_1; \Delta; \Xi; \bang p; \Omega; \Lambda; \Upsilon], C; R \rightarrow \outsem$\\
   \hspace*{1cm} $\prec \mo \Gamma; \Delta''; \Xi''; \Omega'; C'', [\Gamma_2, \Gamma_3; \Delta; \Xi; \bang p; \Omega; \Lambda; \Upsilon], C; R \rightarrow \outsem$\\
   Same as (3).
\end{enumerate}

\subsection{Soundness of derivation}

Proving that the derivation of the head of the rule is sound is trivial except
for comprehensions and aggregates. LLD deterministically computes the number of
available comprehensions to apply while HLD "guesses" the number and then
performs the derivation. In the next two sections, we show how to prove the
soundness of comprehensions and aggregates. The strategy for proving for proving
both is identical due to their inherient similarities.

\subsection{Comprehension soundness}

Proving that deriving a comprehension in LLD is sound in relation to HLD is
built from 4 results: (1) proving that matching the body of a comprehension is
sound in relation to HLD; (2) proving that updating the continuation stacks
makes them suitable for use in the next comprehension applications; (3) proving
that deriving the head of the comprehension is sound in relation to HLD; (4)
proving that we can apply as many comprehensions as the database allows.

\begin{lemma}[Comprehension body match]\label{thm:comprehension_body_match}
Given a comprehension $AB = \compsz{A}{B}$, consider a triplet $T = A; \Gamma; \Delta_{N}$ and a context $\Delta_{N} = \Delta_1, \Delta_2, \Xi$.

If $\mc \Gamma; \Delta_1, \Delta_2; \Xi_N; \Gamma_{N1}; \Delta_{N1}; \Xi;
\Omega; C; P; \compsz{A}{B}; \Omega_N; \Delta_N \rightarrow \outsem$ is well-formed in relation to $T$ then either:

   \begin{itemize}[leftmargin=*]
      \item Match fails:
      \begin{itemize}[leftmargin=\secondm]
         \item $\done \Gamma; \Delta_N; \Xi_N; \Gamma_{N1}; \Delta_{N1}; \Omega_N \rightarrow \outsem$
      \end{itemize}
      
      \item Match succeeds with no backtracking to frames of stack $C$ or $P$
      ($C \neq \cdot$):

      \begin{itemize}[leftmargin=\secondm]
         \item $\mz \Gamma; \Delta_2 \rightarrow A$
         \item $\mc \Gamma; \Delta_1; \Xi_N; \Gamma_{N1}; \Delta_{N1}; \Xi,
            \Delta_2; \cdot; C', C; P; AB; \Omega_N; \Delta_N
            \rightarrow \outsem$ (well-formed in relation to
                  $T$)
      \end{itemize}

      \item Match succeeds with no backtracking to frames of stack $P$ ($C =
            \cdot$):
      \begin{itemize}[leftmargin=\secondm]
         \item $\mz \Gamma; \Delta_2 \rightarrow A$
         \item $\mc \Gamma; \Delta_1; \Xi_N; \Gamma_{N1}; \Delta_{N1}; \Xi,
            \Delta_2; \cdot; C'; P', P; AB; \Omega_N; \Delta_N
            \rightarrow \outsem$ (well-formed in relation to
                  $T$)
      \end{itemize}

      \item Match succeeds with backtracking to a linear continuation frame in
      stack $C$ ($C \neq \cdot$):

      \begin{itemize}[leftmargin=\secondm]
         \item $\mz \Gamma; \Xi_1, \dotsc, \Xi_m, p_2, \Xi_c$
         \item $\exists_{f \in C}. f = (\Delta_a; \Delta_{b_1}, p_2,
               \Delta_{b_2}; p; \Xi_1, \dotsc, \Xi_m; \Omega_1, \dotsc,
               \Omega_k; \Lambda_1, \dotsc, \Lambda_m; \Upsilon_1, \dotsc,
               \Upsilon_n)$
         \item $C = C', f, C''$
         \item $f$ turns into $f' = (\Delta_a, \Delta_{b_1}, p_2;
               \Delta_{b_2}; p; \Xi_1, \dotsc, \Xi_m;
               \Omega_1, \dotsc, \Omega_k; \Lambda_1, \dotsc, \Lambda_m;
               \Upsilon_1, \dotsc, \Upsilon_n)$
         \item $\mc \Gamma; \Delta_c; \Xi_N; \Gamma_{N1}; \Delta_{N1}; \Xi_1,
            \dotsc, \Xi_m, p_2, \Xi_c; \cdot; C''', f', C''; P;
            AB; \Omega_N; \Delta_N \rightarrow \outsem$ (well-formed in relation to $T$)
         \item $\Delta_c = (\Delta_1, \Delta_2, \Xi) - (\Xi_1, \dotsc, \Xi_m,
               p_2, \Xi_c)$
      \end{itemize}

      \item Match succeeds with backtracking to a persistent continuation frame
      in stack $C$ ($C \neq \cdot$):
      \begin{itemize}[leftmargin=\secondm]
         \item $\mz \Gamma; \Delta_{c_2}, \Xi_1, \dotsc, \Xi_m \rightarrow A$
         \item $\exists_{f \in C}. f = [\Gamma_1, p_2, \Gamma_2; \Delta_{c_1},
            \Delta_{c_2}; \Xi_1, \dotsc, \Xi_m; \bang p; \Omega_1, \dotsc, \Omega_k;
            \Lambda_1, \dotsc, \Lambda_m; \Upsilon_1, \dotsc, \Upsilon_n]$
         \item $C = C', f, C''$
         \item $f$ turns into $f' = [\Gamma_2; \Delta_{c_1}, \Delta_{c_2};
            \Xi_1, \dotsc, \Xi_m; \bang p; \Omega_1, \dotsc, \Omega_k; \Lambda_1,
            \dotsc, \Lambda_m; \Upsilon_1, \dotsc, \Upsilon_n]$
         \item $\mc \Gamma; \Delta_{c_1}; \Xi_N; \Gamma_{N1}; \Delta_{N1};
            \Delta_{c_2}, \Xi_1, \dotsc, \Xi_m; \cdot; C''', f', C''; P;
            AB; \Omega_N; \Delta_N \rightarrow \outsem$ (well-formed in relation to $T$)
         \item $\Delta_{c_1}, \Delta_{c_2} = (\Delta_1, \Delta_2, \Xi) - (\Xi_1, \dotsc, \Xi_m)$
      \end{itemize}

      \item Match succeeds with backtracking to a persistent continuation frame
      in stack $P$ ($C = \cdot$):
      \begin{itemize}[leftmargin=\secondm]
         \item $\mz \Gamma; \Delta_{c_2}, \Xi_1, \dotsc, \Xi_m \rightarrow A$
         \item $\exists_{f \in P}. f = [\Gamma_1, p_2, \Gamma_2; \Delta_{c_1}, \Delta_{c_2};
      \Xi_1, \dotsc, \Xi_m; \bang p; \Omega_1, \dotsc, \Omega_k; \Lambda_1,
      \dotsc, \Lambda_m; \Upsilon_1, \dotsc, \Upsilon_n]$
         \item $P = P', f, P''$
         \item $f$ turns into $f' = [\Gamma_2; \Delta_{c_1},
            \Delta_{c_2}; \Xi_1, \dotsc, \Xi_m; \bang p; \Omega_1, \dotsc,
            \Omega_k; \Lambda_1, \dotsc, \Lambda_m; \Upsilon_1, \dotsc, \Upsilon_n]$
         \item $\mc \Gamma; \Delta_{c_1}; \Xi_N; \Gamma_{N1}; \Delta_{N1};
         \Delta_{c_2}, \Xi_1, \dotsc, \Xi_m; \cdot; C'; P''', f', P'';
         AB; \Omega_N; \Delta_N \rightarrow \outsem$
         (well-formed in relation to $T$)
         \item $\Delta_{c_1}, \Delta_{c_2} = (\Delta_1, \Delta_2, \Xi) - (\Xi_1, \dotsc,
               \Xi_m)$
      \end{itemize}
   \end{itemize}
   
If $\contc \Gamma; \Delta_{N}; \Xi_{N}; \Gamma_{N1}; \Delta_{N1}; C; P;
AB; \Omega_N \rightarrow \outsem$ and $C$ and $P$ are
well-formed in relation to $T$ then either:

\begin{itemize}[leftmargin=*]
   \item Match fails:
   \begin{itemize}[leftmargin=\secondm]
      \item $\done \Gamma; \Delta_N; \Xi_N; \Gamma_{N1}; \Delta_{N1}; \Omega_N \rightarrow \outsem$
   \end{itemize}

   \item Match succeeds with backtracking to a linear continuation frame in
   stack $C$ ($C \neq \cdot$):

   \item Match succeeds: $\mz \Gamma; \Delta_x \rightarrow A$ (where
         $\Delta_x \subseteq \Delta_N$) and either:
   \begin{itemize}[leftmargin=\secondm]
      \item $\mz \Gamma; \Xi_1, \dotsc, \Xi_m, p_2, \Xi_c$
      \item $\exists_{f \in C}. f = (\Delta_a; \Delta_{b_1}, p_2,
            \Delta_{b_2}; p; \Xi_1, \dotsc, \Xi_m; \Omega_1, \dotsc,
            \Omega_k; \Lambda_1, \dotsc, \Lambda_m; \Upsilon_1, \dotsc,
            \Upsilon_n)$
      \item $C = C', f, C''$
      \item $f$ turns into $f' = (\Delta_a, \Delta_{b_1}, p_2;
            \Delta_{b_2}; p; \Xi_1, \dotsc, \Xi_m;
            \Omega_1, \dotsc, \Omega_k; \Lambda_1, \dotsc, \Lambda_m;
            \Upsilon_1, \dotsc, \Upsilon_n)$
      \item $\mc \Gamma; \Delta_c; \Xi_N; \Gamma_{N1}; \Delta_{N1}; \Xi_1,
         \dotsc, \Xi_m, p_2, \Xi_c; \cdot; C''', f', C''; P;
         AB; \Omega_N; \Delta_N \rightarrow \outsem$ (well-formed in relation to $T$)
      \item $\Delta_c = (\Delta_1, \Delta_2, \Xi) - (\Xi_1, \dotsc, \Xi_m,
            p_2, \Xi_c)$
   \end{itemize}

   \item Match succeeds with backtracking to a persistent continuation frame
   in stack $C$ ($C \neq \cdot$):
   \begin{itemize}[leftmargin=\secondm]
      \item $\mz \Gamma; \Delta_{c_2}, \Xi_1, \dotsc, \Xi_m \rightarrow A$
      \item $\exists_{f \in C}. f = [\Gamma_1, p_2, \Gamma_2; \Delta_{c_1},
         \Delta_{c_2}; \Xi_1, \dotsc, \Xi_m; \bang p; \Omega_1, \dotsc, \Omega_k;
         \Lambda_1, \dotsc, \Lambda_m; \Upsilon_1, \dotsc, \Upsilon_n]$
      \item $C = C', f, C''$
      \item $f$ turns into $f' = [\Gamma_2; \Delta_{c_1}, \Delta_{c_2};
         \Xi_1, \dotsc, \Xi_m; \bang p; \Omega_1, \dotsc, \Omega_k; \Lambda_1,
         \dotsc, \Lambda_m; \Upsilon_1, \dotsc, \Upsilon_n]$
      \item $\mc \Gamma; \Delta_{c_1}; \Xi_N; \Gamma_{N1}; \Delta_{N1};
         \Delta_{c_2}, \Xi_1, \dotsc, \Xi_m; \cdot; C''', f', C''; P;
         AB; \Omega_N; \Delta_N \rightarrow \outsem$ (well-formed in relation to $T$)
      \item $\Delta_{c_1}, \Delta_{c_2} = (\Delta_1, \Delta_2, \Xi) - (\Xi_1, \dotsc, \Xi_m)$
   \end{itemize}

   \item Match succeeds with backtracking to a persistent continuation frame
   in stack $P$ ($C = \cdot$):
   \begin{itemize}[leftmargin=\secondm]
      \item $\mz \Gamma; \Delta_{c_2}, \Xi_1, \dotsc, \Xi_m \rightarrow A$
      \item $\exists_{f \in P}. f = [\Gamma_1, p_2, \Gamma_2; \Delta_{c_1}, \Delta_{c_2};
   \Xi_1, \dotsc, \Xi_m; \bang p; \Omega_1, \dotsc, \Omega_k; \Lambda_1,
   \dotsc, \Lambda_m; \Upsilon_1, \dotsc, \Upsilon_n]$
      \item $P = P', f, P''$
      \item $f$ turns into $f' = [\Gamma_2; \Delta_{c_1},
         \Delta_{c_2}; \Xi_1, \dotsc, \Xi_m; \bang p; \Omega_1, \dotsc,
         \Omega_k; \Lambda_1, \dotsc, \Lambda_m; \Upsilon_1, \dotsc, \Upsilon_n]$
      \item $\mc \Gamma; \Delta_{c_1}; \Xi_N; \Gamma_{N1}; \Delta_{N1};
      \Delta_{c_2}, \Xi_1, \dotsc, \Xi_m; \cdot; C'; P''', f', P'';
      AB; \Omega_N; \Delta_N \rightarrow \outsem$
      (well-formed in relation to $T$)
      \item $\Delta_{c_1}, \Delta_{c_2} = (\Delta_1, \Delta_2, \Xi) - (\Xi_1, \dotsc,
            \Xi_m)$
   \end{itemize}
   
\end{itemize}
\end{lemma}

Proving that matching the body of a comprehension is sound in relation to HLD
follows the structure of the Lemma~\ref{thm:body_match}. The lemma uses mutual
induction on the recursive judgments $\mc$and $\contc$and considers the three
possible results of matching: failure, success with no backtracking and success
with backtracking.

In order to apply a comprehension again, we need to reuse the continuation
stacks. However, in order to use $C$ and $P$ safely, we need to prove that $C$
will have at most one updated linear continuation frame and $P$ will have all
its frames updated to account the consumption of the facts from the previous
application of the comprehension.

We first prove some auxiliary theorems.

\begin{theorem}[Full stack update]\label{thm:stack_update}
If $\strans \Xi; P; P'$ then $P'$ will be the transformation of stack $P$ where
every frame $f \in P$, where $f = [\Gamma'; \Delta_N; \cdot; \bang p; \Omega; \cdot;
      \Upsilon])$, will turn into $f' = [\Gamma'; \Delta_N - \Xi; \cdot;
      \bang p; \Omega; \cdot; \Upsilon]$, where $f' \in P'$.
\end{theorem}
\begin{proof}
Straightforward induction on the size of $P$.
\end{proof}

\begin{theorem}[From update to derivation]\label{thm:from_update_to_derivation}
If $\fix \Gamma; \Xi_N; \Gamma_{N1}; \Delta_{N1}; \Xi; C; P; AB; \Omega_N;
\Delta_N \rightarrow \outsem$ then\\
\texttab$\dc \Gamma; \Xi_N, \Xi;
\Gamma_{N1}; \Delta_{N1}; B; C' ; P'; AB; \Omega_N; (\Delta_N - \Xi) \rightarrow
\outsem$, where:

\begin{itemize}[leftmargin=*]
   \item If $C = \cdot$ then $C' = \cdot$

   \item If $C = C_1, (\Delta_a; \Delta_b; \cdot; p; \Omega; \cdot; \Upsilon)$
   then $C' = (\Delta_a - \Xi; \Delta_b - \Xi; \cdot; p; \Omega; \cdot;
         \Upsilon)$

   \item $P'$ is the transformation of stack $P$, where for every frame $f \in
   P$ of the form $[\Gamma'; \Delta_N; \cdot; \bang p; \Omega; \cdot; \Upsilon]$
   will turn into $f' = [\Gamma';\Delta_N-\Xi;\cdot;\bang p;\Omega;\cdot;\Upsilon]$

\end{itemize}
\end{theorem}
\begin{proof}
Use induction on the size of the stack $C$ to get the result of $C'$ and then
apply Theorem~\ref{thm:stack_update} to get $P'$.
\end{proof}


Now we prove that a match of a comprehension's body implies the start of a
derivation of the comprehension's head with correct continuation stacks. Note
that $\Omega = \cdot$ in $\matchlldc$, so there is nothing left to match.

\begin{corollary}[Match to derivation]\label{thm:match_to_derivation}
If $\mc \Gamma; \Delta; \Xi_N; \Gamma_{N1}; \Delta_{N1}; \Xi; \cdot; B; C; P;
AB;\Omega_N; \Delta_N \rightarrow \outsem$ then\\
\texttab$\dc \Gamma; \Xi_N, \Xi; \Gamma_{N1}; \Delta_{N1}; B; C'; P'; AB; \Omega_N; (\Delta_N - \Xi) \rightarrow \outsem$ where:
   
\begin{itemize}[leftmargin=*]
   \item If $C = \cdot$ then $C' = \cdot$
   \item If $C = C_1, (\Delta_a; \Delta_b; \cdot; p; \Omega; \cdot; \Upsilon)$ then $C' = (\Delta_a - \Xi; \Delta_b - \Xi; \cdot; p; \Omega; \cdot; \Upsilon)$ then \linebreak $C' = (\Delta_a - \Xi; \Delta_b - \Xi; \cdot; p; \Omega; \cdot; \Upsilon)$
   \item $P'$ is the transformation of stack $P$, where for every frame $f \in
   P$ of the form $[\Gamma'; \Delta_N; \cdot; \bang p; \Omega; \cdot; \Upsilon]$
   will turn into $f' = [\Gamma';\Delta_N-\Xi;\cdot;\bang p;\Omega;\cdot;\Upsilon]$
\end{itemize}
\end{corollary}

\begin{proof}
Invert the assumption and then apply Theorem~\ref{thm:from_update_to_derivation}.
\end{proof}


\paragraph{Comprehension Derivation}

We also need to prove that deriving the head of a comprehension is sound in
relation to HLD.  With the results of the next theorem we can reuse the
continuation stacks to start the comprehension process all over again, but now
with a non-empty continuation stack.

\begin{theorem}[Comprehension derivation soundness]\label{thm:comprehension_derivation}
If $\dc \Gamma; \Delta; \Xi_N; \Gamma_{N1}; \Delta_{N1}; \Omega_1, \dotsc, \Omega_n; C; P;
AB; \Omega_N; \Delta_N \rightarrow \outsem$ then:

\begin{itemize}[leftmargin=*]
   \item $\dc \Gamma; \Delta; \Xi_N; \Gamma_{N1}, \Gamma_1, \dotsc, \Gamma_n; \Delta_{N1},
   \Delta_1, \dotsc, \Delta_n; \cdot; C; P; AB; \Omega_N; \Delta_N \rightarrow
   \outsem$;

   \item $\forall_{\Omega_x}.($ if $\dz \Gamma; \Delta; \Xi_N;
   \Gamma_{N1}, \Gamma_1, \dotsc, \Gamma_n; \Delta_{N1}, \Delta_1, \dotsc,
   \Delta_n; \Omega_x \rightarrow \outsem$ then

   $\dz \Gamma; \Delta; \Xi_N; \Gamma_{N1}; \Delta_{N1}; \Omega_1, \dotsc,
   \Omega_n, \Omega_x \rightarrow \outsem)$.

\end{itemize}
\end{theorem}

\begin{proof}
Straightforward induction on $\Omega_1, \dotsc, \Omega_n$.
\end{proof}

The second result of this theorem is the soundness result we need because it will allow us to reconstruct the derivation tree in HLD.


\paragraph{Multiple Comprehension Derivation} We are interested in proving that
if we start with a given comprehension match $\matchlldc$ then we can apply the
comprehension several times.

\begin{theorem}[Multiple comprehension derivation]\label{thm:multiple_comprehension_derivation}
Consider a triplet $T = A; \Gamma; \Delta_{N}$ and a comprehension $AB =
\compsz{A}{B}$. Assume that there exists $n \geq 0$ applications of $AB$
where the $i_{th}$ application is related to the following contexts:
\begin{description}
   \item[$\Delta_i$]: context of derived linear facts;
   \item[$\Gamma_i$]: context of derived persistent facts;
   \item[$\Xi_i$]: context of consumed linear facts.
\end{description}

Since each application consumes $\Xi_i$ then the initial context $\Delta_N =
\Delta, \Xi_1, \dotsc, \Xi_n$. We now define the two main implications of the
theorem.

\begin{itemize}[leftmargin=*]
   \item Assume that $\Delta_N = \Delta_a, \Delta_b$, $\Delta_b =
   \Delta'_b, p_1$ and there is a frame $f = (\Delta_a, p_1; \Delta'_b; \cdot;
         p; \Omega; \cdot; \Upsilon)$.

   If $\mc \Gamma; \Delta_a, \Delta'_b; \Xi_N; \Gamma_{N1}; \Delta_{N1}; p_1;
   \Omega; f; P; AB; \Omega_N; \Delta, \Xi_1, \dotsc, \Xi_n \rightarrow \outsem$ (well-formed in relation to $T$) then:

   \begin{itemize}[leftmargin=\secondm]
      \item $n$ comprehensions are derived:\\
      $\done \Gamma; \Delta_N; \Xi_N, \Xi_1, \dotsc, \Xi_n; \Gamma_{N1},
      \Gamma_1, \dotsc, \Gamma_n; \Delta_{N1}, \Delta_1, \dotsc, \Delta_n; \Omega_N \rightarrow \outsem$
      \item $n$ $\mz$propositions for the $n$ comprehension matches:
      \begin{itemize}[leftmargin=\thirdm]
         \item $\mz \Gamma; \Xi_1 \rightarrow A$
         \item $\dots$
         \item $\mz \Gamma; \Xi_n \rightarrow A$
      \end{itemize}
      \item $n$ derivation implications for HLD: \\
      $\forall_{\Omega_x}.($ if $\dz \Gamma; \Delta, \Xi_{i+1}, \dotsc, \Xi_{n}; \Xi_N, \Xi_1,
            \dotsc, \Xi_i; \Gamma_{N1}, \Gamma_1, \dotsc, \Gamma_i; \Delta_{N1},
            \Delta_1, \dotsc, \Delta_i; \Omega_x \rightarrow \outsem$ then $\dz \Gamma; \Delta, \Xi_{i+1}, \dotsc, \Xi_{n}; \Xi_N, \Xi_1,
            \dotsc,
            \Xi_i; \Gamma_{N1}, \Gamma_1, \dotsc, \Gamma_{i-1}; \Delta_{N1},
            \Delta_1, \dotsc, \Delta_{i-1}; B, \Omega_x \rightarrow \outsem)$
   \end{itemize}

   \item If $\mc \Gamma; \Delta_N; \Xi_N; \Gamma_{N1}; \Delta_{N1}; \cdot; \Omega;
      \cdot; P; AB; \Omega_N; \Delta, \Xi_1, \dotsc, \Xi_n \rightarrow \outsem$ (well-formed in relation to $T$) then:

   \begin{itemize}[leftmargin=\secondm]
      \item $n$ comprehensions are derived:\\
      $\done \Gamma; \Delta_N; \Xi_N, \Xi_1, \dotsc, \Xi_n; \Gamma_{N1},
      \Gamma_1, \dotsc, \Gamma_n; \Delta_{N1}, \Delta_1, \dotsc, \Delta_n; \Omega_N \rightarrow \outsem$

      \item $n$ $\mz$propositions for the $n$ comprehension matches:
      \begin{itemize}[leftmargin=\thirdm]
         \item $\mz \Gamma; \Xi_1 \rightarrow A$
         \item \dots
         \item $\mz \Gamma; \Xi_n \rightarrow A$
      \end{itemize}

      \item $n$ derivation implications for HLD: \\
      $\forall_{\Omega_x}.($ if $\dz \Gamma; \Delta, \Xi_{i+1}, \dotsc, \Xi_{n}; \Xi_N, \Xi_1,
            \dotsc, \Xi_i; \Gamma_{N1}, \Gamma_1, \dotsc, \Gamma_i; \Delta_{N1},
            \Delta_1, \dotsc, \Delta_i; \Omega_x \rightarrow \outsem$ then $\dz \Gamma; \Delta, \Xi_{i+1}, \dotsc, \Xi_{n}; \Xi_N, \Xi_1,
            \dotsc,
            \Xi_i; \Gamma_{N1}, \Gamma_1, \dotsc, \Gamma_{i-1}; \Delta_{N1},
            \Delta_1, \dotsc, \Delta_{i-1}; B, \Omega_x \rightarrow \outsem)$
   \end{itemize}

\end{itemize}
   
\end{theorem}
\begin{proof}

By mutual induction, first on either the size of $\Delta'_b$ (second argument of
the linear continuation frame) or $\Gamma'$ (second argument of the
persistent frame in $P$) and then on the size of $C, P$.  We only show
how to prove the first implication since the second implication is proven
in a similar way.

$\mc \Gamma; \Delta_a, \Delta'_b; \Xi_N; \Gamma_{N1}; \Delta_{N1}; p_1;
\Omega; f; P; AB; \Omega_N; \Delta, \Xi_1, \dotsc, \Xi_n \rightarrow \outsem$ \hfill (1) assumption\\

By applying Lemma~\ref{thm:comprehension_body_match} to (1), we get:

\begin{itemize}[leftmargin=*]
   \item Failure:
   
   $\done \Gamma; \Delta_N; \Xi_N; \Gamma_{N1}; \Delta_{N1}; \Omega_N
   \rightarrow \outsem$ \hfill (2) from lemma, thus $n = 0$\\
   
   \item Success with no backtracking to frames of stack $C$ or $P$:
   
      $\mz \Gamma; \Xi_1 \rightarrow A$ \hfill (2) from lemma \\

      $\Xi_1 = \Xi'_1, p_1$ \hfill (3) from the well-formedness of (1) \\
      $f = (\Delta_a, p_1; \Delta'_b; \cdot; p; \Omega; \cdot; \Upsilon)$ \\

      $\mc \Gamma; \Delta, \Xi_2, \dotsc, \Xi_n; \Xi_N; \Gamma_{N1};
            \Delta_{N1}; p_1, \Xi'_1; \cdot; C', f; P; AB; \Omega_N; \Delta_N \rightarrow
            \outsem$ \\
      \dots \hfill (4) from lemma (1) \\

      $f' = (\Delta_a, p_1 - \Xi_1; \Delta_b - \Xi_1; \cdot; p; \Omega; \cdot;
            \Upsilon)$ \\

      $\dc \Gamma; \Xi_N, \Xi_1; \Gamma_{N1}; \Delta_{N1}; B; f'; P'; AB;
            \Omega_N; \Delta, \Xi_2, \dotsc, \Xi_n \rightarrow \outsem$ \\
      \dots \hfill (5) using Corollary~\ref{thm:match_to_derivation} on (4) \\

      $\dc \Gamma; \Xi_N, \Xi_1; \Gamma_{N1}, \Gamma_1; \Delta_{N1}, \Delta_1;
            \cdot; f'; P; AB; \Omega_N; \Delta, \Xi_2, \dotsc, \Xi_n \rightarrow \outsem$
      \\ \dots \hfill (6) applying Theorem~\ref{thm:comprehension_derivation} on (5)

      $\forall_{\Omega_x}. ($ if $\dz \Gamma; \Delta, \Xi_2, \dotsc, \Xi_n; \Xi_N, \Xi_1;
            \Gamma_{N1}, \Gamma_1; \Delta_{N1}, \Delta_1; \Omega_x \rightarrow
            \outsem$ then \\ \hspace*{0.5cm} $\dz \Gamma;
            \Delta, \Xi_2, \dotsc, \Xi_n; \Xi_N, \Xi_1; \Gamma_{N1}; \Delta_{N1}; B, \Omega_x
            \rightarrow \outsem)$ \hfill (7) from
      Theorem~\ref{thm:comprehension_derivation} on (5) \\

      $\contc \Gamma; \Delta, \Xi_2, \dotsc, \Xi_n; \Xi_N, \Xi_1; \Gamma_{N1},
         \Gamma_1; \Delta_{N1}, \Delta_1; f'; P'; AB; \Omega_N
         \rightarrow \outsem$\\ \dots \hfill (8) inversion of (6) \\
        
        By inverting (8) we either fail (thus $n = 1$) or we get a new match.
        For the latter case, we apply mutual induction to get the remaining $n -
        1$ comprehensions.
      
   \item With backtracking to the linear continuation frame of stack $C$:
      
      $\mz \Gamma; \Xi_1 \rightarrow A$ \hfill (2) from lemma \\

      $f = (\Delta_a, p_1; \Delta'_b; \cdot; p; \Omega; \cdot; \Upsilon)$ \hfill (3) frame to backtrack to \\
      turns into $f' = (\Delta_a, p_1, \Delta'''_b, p_2; \Delta''_b; \cdot; p; \Omega; \cdot; \Upsilon)$ \hfill (4) resulting frame \\

      $\mc \Gamma; \Delta, \Xi_2, \dotsc, \Xi_n; \Xi_N; \Gamma_{N1};
\Delta_{N1}; p_2, \Xi'_1; \cdot; C', f'; P; AB; \Omega_N; \Delta_N \rightarrow
\outsem$\\ \dots \hfill (5) from lemma (1) \\
      
      Use the same approach as the case with no backtracking.
      
   \item With backtracking to a persistent continuation frame of stack $P$:

      $\mz \Gamma; \Xi_1 \rightarrow A$ \hfill (2) from lemma \\

      $f = [\Gamma''_1, p_2, \Gamma''_2; \Delta_N; \cdot; \bang p; \Omega; \cdot; \Upsilon]$ \hfill (4) from theorem \\
      turns into $f' = [\Gamma''_2; \Delta_N; \cdot; \bang p; \Omega; \cdot;
      \Upsilon]$ \hfill (5) from theorem \\

      $\mc \Gamma; \Delta, \Xi_2, \dotsc, \Xi_n; \Xi_N; \Gamma_{N1};
\Delta_{N1}; \Xi_1; \cdot; C'; P', f', P''; AB; \Omega_N; \Delta_N \rightarrow
\outsem$ \\ \dots \hfill (6) from theorem \\
         
      Use the same approach as the case with no backtracking.
      
\end{itemize}
\end{proof}

For this theorem, we derive three important propositions for HLD: (1) the final
derivation proposition; (2) the matching propositions for each comprehension
application; (2) derivation implications to get from (1) to a derivation
judgment without any derivations of the comprehension. However, the theorem
starts from an initial stack with frames and the comprehension process starts
with an empty stack. We need another theorem that gives us one application of
the comprehension plus the other $n$ that we get from this theorem.

\begin{lemma}[All comprehensions]\label{thm:comprehension}
Consider a triplet $T = A; \Gamma; \Delta_{N}$ and a comprehension $AB =
\compsz{A}{B}$. Assume that there exists $n \geq 0$ applications of $AB$
where the $i_{th}$ application is related to the following contexts:
\begin{description}
   \item[$\Delta_i$]: context of derived linear facts;
   \item[$\Gamma_i$]: context of derived persistent facts;
   \item[$\Xi_i$]: context of consumed linear facts.
\end{description}

Since each application consumes $\Xi_i$ then the initial context $\Delta_N =
\Delta, \Xi_1, \dotsc, \Xi_n$.

If $\mc \Gamma; \Delta, \Xi_1, \dotsc, \Xi_n;
\Xi_N; \Gamma_{N1}; \Delta_{N1}; \cdot; A; \cdot; \cdot; AB; \Omega_N;
\Delta, \Xi_1, \dotsc, \Xi_n \rightarrow \outsem$ (well-formed in
relation to $T$) then:

\begin{itemize}[leftmargin=*]
   \item $n$ comprehensions are derived:\\
   $\done \Gamma; \Delta_N; \Xi_N, \Xi_1, \dotsc, \Xi_n; \Gamma_{N1},
   \Gamma_1, \dotsc, \Gamma_n; \Delta_{N1}, \Delta_1, \dotsc, \Delta_n; \Omega_N \rightarrow \outsem$
   \item $n$ $\mz$propositions for the $n$ comprehension matches:
   \begin{itemize}[leftmargin=\secondm]
      \item $\mz \Gamma; \Xi_1 \rightarrow A$
      \item $\dots$
      \item $\mz \Gamma; \Xi_n \rightarrow A$
   \end{itemize}
   \item $n$ derivation implications for HLD: \\
   $\forall_{\Omega_x}.($ if $\dz \Gamma; \Delta, \Xi_{i+1}, \dotsc, \Xi_n; \Xi_N, \Xi_1,
         \dotsc, \Xi_i; \Gamma_{N1}, \Gamma_1, \dotsc, \Gamma_i; \Delta_{N1},
         \Delta_1, \dotsc, \Delta_i; \Omega_x \rightarrow \outsem$ then $\dz \Gamma; \Delta, \Xi_{i+1}, \dotsc, \Xi_n; \Xi_N, \Xi_1,
         \dotsc,
         \Xi_i; \Gamma_{N1}, \Gamma_1, \dotsc, \Gamma_{i-1}; \Delta_{N1},
         \Delta_1, \dotsc, \Delta_{i-1}; B, \Omega_x \rightarrow \outsem)$
\end{itemize}
\end{lemma}

\begin{proof}
Apply Lemma~\ref{thm:comprehension_body_match} to get two sub-cases:
   
\begin{itemize}[leftmargin=*]
   \item Match fails:
   
   
   $\done \Gamma; \Delta_N; \Xi_N; \Gamma_{N1}; \Delta_{N1}; \Omega_N
   \rightarrow \outsem$\\
   \dots \hfill (1) no comprehension application was possible ($n = 0$)\\
   
   \item Match succeeds:
   
   $\mc \Gamma; \Xi_2, \dotsc, \Xi_n; \Xi_N; \Gamma_{N1}; \Delta_{N1}; \Xi_1; \cdot; C; P; AB; \Omega_N; \Delta_N \rightarrow \outsem$
   
   \dots \hfill (1) result from Lemma~\ref{thm:comprehension_body_match}
   
   $\mz \Gamma; \Xi_1 \rightarrow A$
   \hfill (2) also from Lemma~\ref{thm:comprehension_body_match}
   
   $\dc \Gamma; \Xi_N, \Xi_1; \Gamma_{N1}; \Delta_{N1}; B; C'; P'; AB;
   \Omega_N; \Delta, \Xi_2, \dotsc, \Xi_n \rightarrow \outsem$
   
   \dots \hfill (3) applying Corollary~\ref{thm:match_to_derivation} on (1)
   
   $\dc \Gamma; \Xi_N, \Xi_1; \Gamma_{N1}, \Gamma_1; \Delta_{N1}, \Delta_1;
   \cdot; C'; P'; AB; \Omega_N; \Delta, \Xi_2, \dotsc, \Xi_n \rightarrow \outsem$
   
   \dots \hfill (4) using Theorem~\ref{thm:comprehension_derivation} on (3)\\
   
   $\forall_{\Omega_x}. ($ if $\dz \Gamma; \Delta, \Xi_2, \dotsc, \Xi_n; \Xi_N, \Xi_1; \Gamma_{N1}, \Gamma_1; \Delta_{N1}, \Delta_1; \Omega_x \rightarrow \outsem$ then
   
    \hspace*{0.5cm} $\dz \Gamma; \Delta, \Xi_2, \dotsc, \Xi_n; \Xi_N, \Xi_1; \Gamma_{N1};
    \Delta_{N1}; B, \Omega_x \rightarrow \outsem)$ \\ \dots \hfill (5)
   from the theorem applied in (4)\\
   
   $\contc \Gamma; \Delta, \Xi_2, \dotsc, \Xi_n; \Xi_N, \Xi_1; \Gamma_{N1},
   \Gamma_1; \Delta_{N1}, \Delta_1; C'; P'; AB; \Omega_N \rightarrow \outsem$
   
   \dots \hfill (6) inversion of (5)\\
   
   Invert (6) to get either $n = 1$ application of the comprehension or apply Theorem~\ref{thm:multiple_comprehension_derivation} to the inversion to get the remaining $n-1$. 
\end{itemize}
\end{proof}

If the previous lemma, the comprehension is applied for as many times as the
database allows. We now have to map these $n$ applications to HLD by rebuilding
the proof tree for these $n$ matches and derivations and then using
$n$ when "guessing" the number of iterative definitions in HLD.

\subsection{Soundness of derivation}

We are finally ready to prove that the derivation of terms of the head of a rule
is sound in relation to HLD.

\begin{lemma}[Head derivation soundness]\label{thm:head_derivation_soundness}
If $\done \Gamma; \Delta_N; \Xi_N; \Gamma_{N1}; \Delta_{N1}; \Omega \rightarrow \outsem$ then
$\dz \Gamma; \Delta_N; \Xi_N; \Gamma_{N1}; \Delta_{N1}; \Omega \rightarrow \outsem$.
\end{lemma}

\begin{proof}\label{sec:derivation_theorem} Induction on $\Omega$. Most of the
sub-cases can be proven using the induction hypothesis or by straightforward
rule inference. The sub-cases for the comprehensions and aggregates are
complicated and are proved beflow.

\paragraph{Comprehensions} Apply Lemma~\ref{thm:comprehension} on the assumption
to get $n$ applications of the comprehension. Assume that 
$\Delta_N = \Delta, \Xi_1, \dotsc, \Xi_n$, where $\Xi_i$ are the facts consumed
and $\Gamma_i, \Delta_i$ the facts derived by the $i^{th}$ application.
The lemma proves the following:

\begin{itemize}[leftmargin=*]
   \item $\done \Gamma; \Delta; \Xi_N, \Xi_1, \dotsc, \Xi_n; \Gamma_{N1},
   \Gamma_1, \dotsc, \Gamma_n; \Delta_{N1}, \Delta_1, \dotsc, \Delta_n;
\Omega_N \rightarrow \outsem$ \hfill (1)
   \item $n$ propositions $\mz \Gamma; \Xi_i \rightarrow A$ \hfill (2)
   \item $n$ implications\\
   $\forall_{\Omega_x}.($ if $\dz \Gamma; \Delta, \Xi_{i+1}, \dotsc,
         \Xi_{n}; \Xi_N, \Xi_1,
         \dotsc, \Xi_i; \Gamma_{N1}, \Gamma_1, \dotsc, \Gamma_i; \Delta_{N1},
         \Delta_1, \dotsc, \Delta_i; \Omega_x \rightarrow \outsem$ then $\dz \Gamma; \Delta, \Xi_{i+1}, \dotsc, \Xi_n; \Xi_N, \Xi_1,
         \dotsc,
         \Xi_i; \Gamma_{N1}, \Gamma_1, \dotsc, \Gamma_{i-1}; \Delta_{N1},
         \Delta_1, \dotsc, \Delta_{i-1}; B, \Omega_x \rightarrow \outsem)$ \hfill (3) \\
\end{itemize}

\noindent From (1) we apply the inductive hypothesis since $\Omega$ gets
smaller:\\
$\dz \Gamma; \Delta; \Xi_N, \Xi_1, \dotsc, \Xi_n; \Gamma_{N1}, \Gamma_1,
\dotsc, \Gamma_n; \Delta_{N1}, \Delta_1, \dotsc, \Delta_n; \Omega \rightarrow
\outsem$ \\

\noindent Since we are building the proof tree backwards, starting from the final
derivation result, we first need to derive $\compz{0}{A}{B}$ by applying rules
$\dz \one$ and $\dz \m{comp}^0$:\\
$\dz \Gamma; \Delta; \Xi_N, \Xi_1, \dotsc, \Xi_n; \Gamma_{N1}, \Gamma_1,
\dotsc, \Gamma_n; \Delta_{N1}, \Delta_1, \dotsc, \Delta_n; \compz{0}{A}{B}, \Omega \rightarrow
\outsem$
\\

\noindent From result (4), we first rebuild the matching and derivation process of the
$n^{th}$ comprehension.  Using the $n^{th}$ implication (3) on (5):

\noindent $\dz \Gamma; \Delta, \Xi_n; \Xi_N, \Xi_1, \dotsc, \Xi_{n-1}; \Gamma_{N1}, \Gamma_1,
\dotsc, \Gamma_{n-1}; \Delta_{N1}, \Delta_1, \dotsc, \Delta_{n-1}; B, \compz{0}{A}{B},
\Omega \rightarrow \outsem$ \\

\noindent Using $\dz \lolli$ and the matching proposition (2) on (6), the $A \lolli B$
implication is reconstructed:

\noindent $\dz \Gamma; \Delta, \Xi_n; \Xi_N, \Xi_1, \dotsc, \Xi_{n-1}; \Gamma_{N1},
   \Gamma_1, \dotsc, \Gamma_{n-1}; \Delta_{N1}, \Delta_1, \dotsc, \Delta_{n-1};
A \lolli B, \compz{0}{A}{B}, \Omega \rightarrow \outsem$ \\

\noindent Now, $\compz{1}{A}{B}$ is rebuilt by applying $\dz \otimes$ followed by $\dz
\m{comp}^N$:

\noindent $\dz \Gamma; \Delta, \Xi_n; \Xi_N, \Xi_1, \dotsc, \Xi_{n-1}; \Gamma_{N1},
\Gamma_1, \dotsc, \Gamma_{n-1}; \Delta_{N1}, \Delta_1, \dotsc, \Delta_{n-1};
\compz{1}{A}{B}, \Omega \rightarrow \outsem$ \\

\noindent Steps (5) through (8) are then applied $n-1$ times to get:

\noindent $\dz \Gamma; \Delta, \Xi_1, \dotsc, \Xi_n; \Xi_N; \Gamma_{N1}; \Delta_{N1};
\compz{n}{A}{B}, \Omega \rightarrow \outsem$ \\

\noindent Finally, to construct the conclusion and finish the proof, $\dz \m{comp}^*$ needs to
be applied:

\noindent $\dz \Gamma; \Delta, \Xi_1, \dotsc, \Xi_n; \Xi_N; \Gamma_{N1}; \Delta_{N1};
\compsz{A}{B}, \Omega \rightarrow \outsem$ \\

\noindent This finishes the sub-case for comprehensions.

\end{proof}

\section{Summary}

In this chapter we presented the proof theoretic foundations of LM.  First, we
introduced \fragment, the linear logic fragment that supports LM. We then
presented HLD, the high level dynamic semantics that was created by interpreting
the linear logic fragment using focusing and chaining. Next, we designed LLD,
the low level dynamic semantics that mimics the execution of rules in our
virtual machine minus small details.  Finally, we proved that LLD is sound
in relation to HLD, thus showing a connection from LLD to \fragment.

