
In this section, we present the high level dynamic~(HLD) semantics of LM.  HLD
formalizes the mechanism of matching rules and deriving new facts.  HLD
semantics present a simplified overview of the dynamics of the language that are
closer to \fragment (Section~\ref{sec:fragment}) presented before.  than the
implementation principles of our virtual machine. The low level dynamic~(LLD)
semantics are much closer to a real implementation and represents the
operational semantics of the language.

Note that both HLD and LLD do not model the use of variable bindings when
matching facts from the database. The formalization of bindings tends to
complicate the formal system and it is not necessary for a good understanding of
the system. Instead, we assume that all facts of type $\emph{fact}(\hat{x})$ do
not have the argument $\hat{x}$.

Starting from \fragment presented earlier, we consider $\Gamma$ and $\Delta$ the database
of our program. $\Gamma$ contains the database of persistent facts while $\Delta$ the database of linear
facts. We assume that the rules of the program are persistent linear implications of the form
$\bang (A \lolli B)$ that can be used several times. However, we do not put the rules in the $\Gamma$
context but in a separate context $\Phi$.

The main idea of the dynamic semantics is to ignore the right side of the
sequent calculus and use \emph{chaining} and \emph{inversion} on the $\Delta$
and $\Gamma$ contexts so that we only have atomic facts (e.g., the database of
facts).  To apply rules we use
\emph{focusing}~\cite{Andreoli92logicprogramming} on one of the derivation rules
in $\Phi$. Note that in the focusing process we assume that all the atoms
(facts) are positive thus the chaining proceeds in a \emph{forward chaining}
fashion.

\subsection{Step}\label{sec:step_hld}

Operationally, LM proceeds in \emph{steps}. A step happens at some node $i$ and
proceeds by picking one rule to apply, matching the body of the rule against the
database, removing all those facts from the database and then deriving all the
constructs in the head of the rule. We assume the existence of $n$ nodes in the
program and that $\Delta$ and $\Gamma$ are split into $\Delta_1, \dotsc, \Delta_n$
and $\Gamma_1, \dotsc, \Gamma_n$ respectively. After each step, the database of
each fact is updated accordingly.

Steps are defined as $\stepz \Gamma; \Delta; \Phi \Longrightarrow \Gamma';
\Delta'$, where $\Gamma'$ and $\Delta'$ is the new database. The step rule is as
follows:

\[
\infer[\stepz]
{\begin{split}
   \stepz [\Gamma_1, \dotsc, \Gamma_i, \dotsc, \Gamma_n]&; [\Delta_1, \dotsc,
   \Delta_i, \dotsc, \Delta_n]; \Phi \\
   \Longrightarrow& \\ [\Gamma_1, \Gamma'_1, \dotsc, \Gamma_i,
   \Gamma'_i, \dotsc, \Gamma_n, \Gamma'_n]; & [\Delta_1, \Delta'_1, \dotsc, (\Delta_i -
         \Xi'), \Delta'_i, \dotsc, \Delta_n, \Delta'_n]
\end{split}
}
{\doz{\Gamma_i}{\Delta_i}{\Phi}{\Pi}{\Xi'}{\Gamma'_1, \dotsc,
   \Gamma'_n}{\Delta'_1, \dotsc, \Delta'_n}
}
\]


\subsection{Application}

A step is performed through $\doz \Gamma; \Delta; \Phi \rightarrow \Xi';
\Delta'; \Gamma'$.  $\Gamma$, $\Delta$ and $\Phi$ have the meaning explained
before, while $\Xi'$, $\Delta'$ and $\Gamma'$ are output multi-sets from
applying one of the rules in $\Phi$. $\Xi'$ is the set of consumed linear
resources, $\Delta'$ is the set of derived linear facts and $\Gamma'$ is the set
of derived persistent facts. Note that for HLD semantics there is no concept of
rule priority, therefore a rule is picked non-deterministically.

The judgment $\az \Gamma ; \Delta ; A \lolli B \rightarrow \Xi'; \Delta';
\Gamma'$ will attempt to apply the derivation rule $A \lolli B$. To do this, it
splits the $\Delta$ context into $\Delta_1$ and $\Delta_2$, namely the set of
linear resources consumed to match the body of the rule ($\Delta_1$) and the
remaining linear facts ($\Delta_2$).  Again, the set of resources needed to
match the body of the rule is guessed. LLD semantics will deterministically
calculate $\Delta_1$.

\[
\infer[\az \m{rule}]
{\az \Gamma ; \Delta_1, \Delta_2 ; A \lolli B \rightarrow \Xi' ; \Delta' ; \Gamma'}
{\mz \Gamma ; \Delta_1 \rightarrow A & \dz \Gamma ; \Delta_2; \Delta_1; \cdot ; \cdot ; B \rightarrow \Xi' ; \Delta' ; \Gamma'}
\]

\[
\infer[\doz \m{rule}]
{\doz \Gamma ; \Delta ; R, \Phi \rightarrow \Xi' ; \Delta' ; \Gamma'}
{\az \Gamma ; \Delta ; R \rightarrow \Xi' ; \Delta' ; \Gamma'}
\]


\subsection{Match}

The $\mz \Gamma ; \Delta \rightarrow C$ judgment uses the right ($R$) rules of
\fragment in order to match (prove) the body $C$ using $\Gamma$
and $\Delta$. We must consume all the linear facts in the multi-set $\Delta$
when matching $C$. The context $\Gamma$ may be used to match persistent terms in
$C$ but such facts are never consumed since they are persistent.

\[
\infer[\mz \one]
{\mz \Gamma; \cdot \rightarrow \one}
{}
\]
\[
\infer[\mz p]
{\mz \Gamma; p \rightarrow p}
{}
\tab
\infer[\mz \bang p]
{\mz \Gamma, p; \cdot \rightarrow \bang p}
{}
\]

\[
\infer[\mz \otimes]
{\mz \Gamma; \Delta_1, \Delta_2 \rightarrow A \otimes B}
{\mz \Gamma; \Delta_1 \rightarrow A & \mz \Gamma; \Delta_2 \rightarrow B}
\]


\subsection{Derivation}

After successfully matching the body of the rule, we next derive the head of the
rule. The derivation judgment has the form $\dz \Gamma ; \Delta ; \Xi ; \Gamma_1
; \Delta_1 ; \Omega \rightarrow \Xi'; \Delta'; \Gamma'$ with the following
meaning:

\begin{enumerate}

   \item[$\Gamma$] the multi-set of persistent resources in the database;

   \item[$\Delta$] the multi-set of linear resources in the database not yet
   consumed;

   \item[$\Xi$] the multi-set of linear resources that have been consumed while
   matching the body of the rule, matching comprehensions or aggregates;

   \item[$\Gamma_1$] the multi-set of persistent facts that have been derived
   using the current rule;

   \item[$\Delta_1$] the multi-set of linear facts that have been derived using
   the current rule;

   \item[$\Omega$] an ordered list contain the terms of the head of rule that
   still need to be derived. We start with the head of the rule $B$ that is
   continuously deconstructed to derive all the facts of the rule;

   \item[$\Xi'$] the consumed linear facts to apply this rule;

   \item[$\Delta'$] the derived linear facts;

   \item[$\Gamma'$] the derived persistent facts.

\end{enumerate}

The following derivation rules are a direct translation from \fragment:

\[
\infer[\dz p]
{\dz \Gamma ; \Delta ; \Xi ; \Gamma_1 ; \Delta_1 ; p, \Omega \rightarrow \Xi' ; \Delta' ; \Gamma'}
{\dz \Gamma ; \Delta ; \Xi ; \Gamma_1 ; p, \Delta_1 ; \Omega \rightarrow \Xi' ; \Delta' ; \Gamma'}
\]

\[
\infer[\dz \bang p]
{\dz \Gamma ; \Delta ; \Xi ; \Gamma_1 ; \Delta_1 ; \bang p, \Omega \rightarrow \Xi' ; \Delta' ; \Gamma'}
{\dz \Gamma ; \Delta ; \Xi ; \Gamma_1, p ; \Delta_1 ; \Omega \rightarrow \Xi' ; \Delta' ; \Gamma'}
\]

\[
\infer[\dz \otimes]
{\dz \Gamma ; \Delta ; \Xi ; \Gamma_1 ; \Delta_1 ; A \otimes B, \Omega \rightarrow \Xi' ; \Delta' ; \Gamma'}
{\dz \Gamma ; \Delta ; \Xi ; \Gamma_1 ; \Delta_1 ; A, B, \Omega \rightarrow \Xi' ; \Delta' ; \Gamma'}
\]

\[
\infer[\dz \one]
{\dz \Gamma ; \Delta ; \Xi ; \Gamma_1; \Delta_1 ; 1, \Omega \rightarrow \Xi' ; \Delta' ; \Gamma'}
{\dz \Gamma ; \Delta ; \Xi ; \Gamma_1; \Delta_1 ; \Omega \rightarrow \Xi' ; \Delta' ; \Gamma'}
\]

\[
\infer[\dz end]
{\dz \Gamma ; \Delta ; \Xi' ; \Gamma' ; \Delta' ; \cdot \rightarrow \Xi' ; \Delta' ; \Gamma'}
{}
\]


We did not include the aggregates here because they are similar to comprehensions.
The starting rule for deriving comprehensions is $\dz comp^*$. It
deterministically picks a number $N$ that then can be unfolded $N$ times to get
$A \lolli B$. The HLD semantics do not take into account the contents of the
database to determine how many times a comprehension should be applied.

\[
\infer[\dz \m{comp}^*]
{\dz \Gamma ; \Delta ; \Xi ; \Gamma_1 ; \Delta_1 ; \compsz{A}{B}, \Omega
   \rightarrow \outsem}
{\dz \Gamma ; \Delta ; \Xi ; \Gamma_1 ; \Delta_1 ; \compz{N}{A}{B}, \Omega
   \rightarrow \outsem}
\]

\[
\infer[\dz \m{comp}^N]
{\dz \Gamma ; \Delta ; \Xi ; \Gamma_1 ; \Delta_1 ; \compz{N}{A}{B}, \Omega
   \rightarrow \outsem}
{\dz \Gamma ; \Delta ; \Xi ; \Gamma_1 ; \Delta_1 ; \compunfold{N-1}{A}{B},
   \Omega \rightarrow \outsem}
\]

\[
\infer[\dz \m{comp}^0]
{\dz \Gamma ; \Delta ; \Xi ; \Gamma_1 ; \Delta_1 ; \compz{0}{A}{B}, \Omega
   \rightarrow \outsem}
{\dz \Gamma ; \Delta ; \Xi ; \Gamma_1 ; \Delta_1 ; \compunfoldz, \Omega
   \rightarrow \outsem}
\]


These rules mirror the rules for the iterative definition in \fragment.  Rules
for aggregates are similar:

\[
\infer[\dz \m{agg}^*]
{\dz \Gamma ; \Delta ; \Xi ; \Gamma_1 ; \Delta_1 ; \aggsz{A}{B}{C}, \Omega
   \rightarrow \outsem}
{\dz \Gamma ; \Delta ; \Xi ; \Gamma_1 ; \Delta_1 ; \aggz{N}{A}{B}{C}{0}, \Omega
   \rightarrow \outsem}
\]

{\small
\[
\infer[\dz \m{agg}^N]
{\dz \Gamma ; \Delta ; \Xi ; \Gamma_1 ; \Delta_1 ; \aggz{N}{A}{B}{C}{V}, \Omega
   \rightarrow \outsem}
{\dz \Gamma ; \Delta ; \Xi ; \Gamma_1 ; \Delta_1 ; \aggunfold{N-1}{A}{B}{C}{V},
   \Omega \rightarrow \outsem}
\]
}

\[
\infer[\dz \m{agg}^0]
{\dz \Gamma ; \Delta ; \Xi ; \Gamma_1 ; \Delta_1 ; \aggz{0}{A}{B}{C}{V}, \Omega
   \rightarrow \outsem}
{\dz \Gamma ; \Delta ; \Xi ; \Gamma_1 ; \Delta_1 ; \aggunfoldz{C}{V}, \Omega
   \rightarrow \outsem}
\]


Finally, because the comprehensions and aggregates create implications $A \lolli
B$, we add a final derivation rule $\dz \lolli$:

\[
\infer[\dz \lolli]
{\dz \Gamma ; \Delta_a, \Delta_b ; \Xi ; \Gamma_1 ; \Delta_1 ; A \lolli B,
   \Omega \rightarrow \outsem}
{\mz \Gamma ; \Delta_a \rightarrow A & \dz \Gamma ; \Delta_b ; \Xi, \Delta_a ;
   \Gamma_1 ; \Delta_1 ; B, \Omega \rightarrow \outsem}
\]

