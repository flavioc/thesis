
The last decade has seen a priority shift for processor manufactures. If clock
frequency was once the main metric for performance, today computing power is
measured in number of cores in a single chip.  For software developers and
computer scientists, once focused in developing sequential programs, newer
hardware usually meant faster programs without any change to the source code.
Today, the free lunch is over. Multicore processors are now forcing the
development of new software methodologies that take advantage of increasing
processing power through parallelism.  However, parallel programming is
difficult, usually because programs are written in imperative and stateful
programming languages that make use of low level synchronization primitives such
as locks, mutexes and barriers. This tends to make the task of managing
multithreaded execution quite intricate and error-prone, resulting in race
hazards and deadlocks.  In the future, \emph{many-core} processors will make
this task look even more daunting.

Advances in network speed and bandwidth are making distributed computing more
appealing. For instance, \emph{cloud computing} is a new emerging paradigm that
wants to make every computer connected to the Internet as a part of a pool of
computing power, where data can be retrieved and computation performed. From the
perspective of high performance computing, the \emph{computer cluster} is a well
established paradigm that uses fast local area networks to improve performance
and solve problems that would take a long time with a single computer.

Developments in parallel and distributed programming have given birth to several
programming models.  At one end of the spectrum are lower-level programming
abstractions such as \emph{message passing} (e.g.,
MPI~\cite{gabriel04-open-mpi}) and \emph{shared memory} (e.g.,
Pthreads~\cite{Butenhof:1997:PPT:263953} or
OpenMP~\cite{Chapman-2007-UOP-1370966}).  While such abstractions are
very expressive and enable the programmer to write high performant code,
program APIs are very hard to use and debug, which makes it difficult to
prove that a program is correct.  On the opposite end, we have many
declarative programming models that can be run in
parallel~\cite{Blelloch:1996:PPA:227234.227246}.  While those declarative
paradigms tend to make programs easier to reason about, they tend to offer
little or no control to the programmer for managing parallel execution
which may result in suboptimal performance.

In the context of the Claytronics project~\cite{goldstein-computer05},
Ashley-Rollman et al.~\cite{ashley-rollman-iclp09,
ashley-rollman-derosa-iros07wksp} has created Meld, a logic programming
language suited to program massively distributed systems made of modular
robots with a dynamic topology.  Meld programs can derive actions that are
used to make the robots act on the outside world. The distribution of
computation is done by first partitioning the program state across the
robots and then making the computation local to the node. Because Meld
programs are sets of logical clauses, they are more amenable to proof.

However, Meld and other declarative programming models give very little control
to the programmer since they are stateless languages.  This is a clear
disadvantage against lower-level abstractions, since its difficult to change how
programs are scheduled by the runtime system and how the system manages
parallelism.

In this proposal, we present Linear Meld (LM), a new language for parallel
programming over graph data structures that extends the original Meld with
linear logic and coordination.

Linear logic gives the language a structured way to manage state, allowing the
programmer to derive and delete logical facts. In turn, this gives the
programmer more expressiveness since linear facts can be used optionally where
state makes more sense. Some problems found in the original Meld, like the
proliferation of persistent facts, can be circumvented with the judicious use of
linear facts.

While the new language retains the declarative aspects of Meld, it also adds
explicit programmer control and opportunities for optimization that arise with
stateful programs.  We introduce the concept of \emph{coordination facts}, which
are logical facts used for scheduling and data partitioning purposes.
Coordination facts can be split into \emph{action facts} and \emph{sensing
facts}, a concept already present in the original Meld. Coordination facts
allow the programmer to declaratively reason about the state of computation
and the state of the runtime system in order to perform scheduling and data
partitioning decisions. In turn, these decisions allow the programmer to
reduce program run time, increase data locality, improve partitioning and
parallel scalability in a declarative fashion. Since coordination facts are
semantically equivalent to computation and a first class entity of Linear
Meld, we are able to reason about the coordination even in the presence of
coordination.

Our goal with LM is to efficiently execute provably correct logical graph-based
programs on multicore machines and then improve their execution with
coordination, while retaining the declarativeness and correctness of the
original algorithms. To show this, we wrote many graph-based algorithms, proved
important properties like termination and correctness and developed a compiler
and runtime system where we have seen good experimental results when executing
these programs. Finally, we have also coordinated some of the programs and have
seen some interesting improvements in terms of run time, data locality and
scalability.

\section{Thesis Statement}


In this thesis, we describe a new linear logic programming language, called
Linear Meld~(LM), designed to write \textbf{declarative parallel graph based
programs} for multicore architectures. Our goal with LM is to prove the
following thesis:

\begin{quote}
Linear logic with coordination provides a high level language for expressing
parallel graph-based programs that are both easy to reason about and implement
efficiently. Furthermore, modeling programs as graphs allows the runtime system
to solve the granularity problem by grouping nodes into sub-graphs that are
processed independently.
\end{quote}


LM is a novel programming language that makes \textbf{coordination} a
first-class programming construct that is \textbf{semantically equivalent to
computation}. Coordination allows the programmer to write declarative code that
reasons about the underlying parallel architecture in order to improve the run
time and scalability of programs. Since LM is based on logical foundations,
programs are amenable to reasoning, even in the presence of coordination.

We support our thesis through seven major contributions:

\begin{description}
   
   \item[Linear Logic]

   We integrated linear logic into our language, so that program state can be
   encoded naturally. The original Meld was fully based on classical logic where
   everything that is derived is true forever. Linear logic turns some facts
   into resources that can be consumed when a rule is applied.  To the best of
   our knowledge, LM is the first linear logic based language implementation
   that attempts to solve real world problems.

   \item[Coordination Facts]
   
   LM offers execution control to the programmer through the use of coordination
   facts. These coordination facts change how the runtime system schedules
   computation and partitions data and is semantically equivalent to standard
   computation facts. We can increase the priority of certain nodes according
   to the state of the computation and to the state of the runtime in order to
   make better scheduling decisions so that programs can be more scalable and
   run faster.

   \item[Thread-Based Facts]

   While LM can be classified as a programming language with implicit
   parallelism, it introduces thread facts to support some form of declarative
   explicit parallelism. Thread facts allow the programmer to reason about the
   state of the underlying parallel architecture, namely the execution thread,
   by managing facts about each thread. Thread facts can be used along with
   regular facts belonging to nodes and coordination facts, allowing for the
   implementation of customized parallel scheduling algorithms. We show how some
   programs take advantage of this facility to create optimal scheduling
   algorithms that are not possible in the limited context of implicit
   parallelism.
   
   \item[Provability]
   
   We leveraged the logical foundations of LM to show how to prove the
   correctness of programs. We also show that coordination does not change those
   correctness proofs but only improves run time, scalability or memory usage.

   \item[Efficient Runtime and Compilation]

   We have implemented a runtime system with support for efficient data
   structures for handling linear facts and a compiler that is designed to
   transform inference rules into C++ code that make use of those data
   structures. We have achieved a sequential performance that is less than one
   order of magnitude slower than hand-written sequential C++ programs and is
   competitive with some more mature frameworks such as
   GraphLab~\cite{GraphLab2010} or Ligra~\cite{Shun:2013:LLG:2517327.2442530}.

   \item[Multicore Parallelism]
   
   The logical facts of the program are partitioned across the graph of the
   nodes. Since the logical rules only make use of facts from a single node,
   computation can be performed locally, independently of other nodes of the
   graph. We view applications as a communicating graph data structure where
   each processing unit performs work on a different subset of the graph, thus
   enabling concurrency. This allows LM to solve the granularity problem by
   allowing computation to be grouped into sub-graphs that can be processed by
   different processing cores. Even if there is poor work imbalance between
   cores, it is possible to easily move nodes between cores to achieve better
   load balancing and scheduling.

   \item[Experimental Results]

   We have implemented a compiler and a virtual machine prototype from scratch
   that executes on multicore machines.  We have implemented programs such as
   belief propagation, belief propagation with residual splash, PageRank, graph
   coloring, N queens, shortest path, diameter estimation, MapReduce, game of
   life, quick-sort, neural network training, among others. Our experiments
   performed on a machine with 32 cores shows that our implementation provides
   good scalability with up to 32 threads of execution.
      
\end{description}


