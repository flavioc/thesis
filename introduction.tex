Multicore architectures have become more widespread recently and are forcing the
development of new software methodologies that enable developers to take
advantage of increasing processing power through parallelism. However, parallel
programming is difficult, usually because programs are written in imperative and
stateful programming languages that make use of low level synchronization
primitives such as locks, mutexes and barriers. This tends to make the task of
managing multithreaded execution complicated and error-prone, resulting in race
hazards and deadlocks. In the future, \emph{many-core} processors will make this
task even more daunting.

Past developments in parallel and distributed programming have given birth to
several programming models. At one end of the spectrum are the lower-level
programming abstractions such as \emph{message passing} (e.g.,
MPI~\cite{gabriel04-open-mpi}) and \emph{shared memory} (e.g.,
Pthreads~\cite{Butenhof:1997:PPT:263953} or
OpenMP~\cite{Chapman-2007-UOP-1370966}). While such abstractions give a lot of
control to the programmer and provide excellent performance, program APIs are
very hard to use and debug, which makes it difficult to prove that a program is
correct, for instance. On the opposite end, we have many declarative programming
models~\cite{Blelloch:1996:PPA:227234.227246} that can exploit some form of
implicit parallelism.  While those declarative paradigms tend to make programs
easier to reason about, they offer little or no control to the programmer over
how parallel execution is scheduled or how data is laid out, making it hard to
improve efficiency. This introduces performance issues because even if the
runtime system is able to reasonably parallelize the program using a general
algorithm, there is a lack of specific information about the program that a
compiler cannot easily deduce. Such information could make execution better in
terms of run time, memory usage, or scalability.

In the context of the Claytronics project~\cite{goldstein-computer05},
Ashley-Rollman et al.~\cite{ashley-rollman-iclp09,
ashley-rollman-derosa-iros07wksp} created Meld, a logic programming language
suited to program massively distributed systems made of modular robots with a
dynamic topology. Meld programs can derive actions that are used to make the
robots act on the outside world. The distribution of computation is done by
first partitioning the program state across the robots and then by making the
computation local to the robot. Because Meld programs are sets of logical
clauses, they are more amenable to proofs.

In this thesis, we present Linear Meld (LM), a new language for parallel
programming over graph data structures that extends the original Meld with
linear logic and coordination. Linear logic gives the language a structured way
to manage state, allowing the programmer to assert and retract logical facts.
While the original Meld sees a running program as an ensemble of robots, LM sees
the program as a graph of node data structures, where each node performs
computation independently and is able to communicate with other nodes.

LM introduces a new mechanism, called coordination, that is semantically
equivalent to regular computation and allows the programmer to reason about
parallel execution. Coordination introduces the concept of \emph{coordination
facts}, which are logical facts used for scheduling and data partitioning
purposes, and \emph{thread facts}, which allow the programmer to reason about
the state of the underlying parallel architecture. The use of these new
facilities moves the LM language from the paradigm of implicit parallelism to
some form of declarative explicit parallelism, but without the pitfalls of
imperative parallel programming. In turn, this makes LM a novel declarative
language that allows the programmer to optionally control how execution and data
is managed by the execution system.

Our goal with LM is to efficiently execute provably correct declarative
graph-based programs on multicore machines. To show this, we wrote many
graph-based algorithms, proved program correctness and developed a compiler and
runtime system where we have seen good experimental results when executing these
programs. Finally, we have also used coordination in some of these programs and
we were able to see interesting improvements in terms of run time, memory usage
and scalability.

\section{Thesis Statement}


In this thesis, we describe a new linear logic programming language, called
Linear Meld~(LM), designed to write \textbf{declarative parallel graph based
programs} for multicore architectures. Our goal with LM is to prove the
following thesis:

\begin{quote}
Linear logic with coordination provides a high level language for expressing
parallel graph-based programs that are both easy to reason about and implement
efficiently. Furthermore, modeling programs as graphs allows the runtime system
to solve the granularity problem by grouping nodes into sub-graphs that are
processed independently.
\end{quote}


LM is a novel programming language that makes \textbf{coordination} a
first-class programming construct that is \textbf{semantically equivalent to
computation}. Coordination allows the programmer to write declarative code that
reasons about the underlying parallel architecture in order to improve the run
time and scalability of programs. Since LM is based on logical foundations,
programs are amenable to reasoning, even in the presence of coordination.

We support our thesis through seven major contributions:

\begin{description}
   
   \item[Linear Logic]

   We integrated linear logic into our language, so that program state can be
   encoded naturally. The original Meld was fully based on classical logic where
   everything that is derived is true forever. Linear logic turns some facts
   into resources that can be consumed when a rule is applied.  To the best of
   our knowledge, LM is the first linear logic based language implementation
   that attempts to solve real world problems.

   \item[Coordination Facts]
   
   LM offers execution control to the programmer through the use of coordination
   facts. These coordination facts change how the runtime system schedules
   computation and partitions data and is semantically equivalent to standard
   computation facts. We can increase the priority of certain nodes according
   to the state of the computation and to the state of the runtime in order to
   make better scheduling decisions so that programs can be more scalable and
   run faster.

   \item[Thread-Based Facts]

   While LM can be classified as a programming language with implicit
   parallelism, it introduces thread facts to support some form of declarative
   explicit parallelism. Thread facts allow the programmer to reason about the
   state of the underlying parallel architecture, namely the execution thread,
   by managing facts about each thread. Thread facts can be used along with
   regular facts belonging to nodes and coordination facts, allowing for the
   implementation of customized parallel scheduling algorithms. We show how some
   programs take advantage of this facility to create optimal scheduling
   algorithms that are not possible in the limited context of implicit
   parallelism.
   
   \item[Provability]
   
   We leveraged the logical foundations of LM to show how to prove the
   correctness of programs. We also show that coordination does not change those
   correctness proofs but only improves run time, scalability or memory usage.

   \item[Efficient Runtime and Compilation]

   We have implemented a runtime system with support for efficient data
   structures for handling linear facts and a compiler that is designed to
   transform inference rules into C++ code that make use of those data
   structures. We have achieved a sequential performance that is less than one
   order of magnitude slower than hand-written sequential C++ programs and is
   competitive with some more mature frameworks such as
   GraphLab~\cite{GraphLab2010} or Ligra~\cite{Shun:2013:LLG:2517327.2442530}.

   \item[Multicore Parallelism]
   
   The logical facts of the program are partitioned across the graph of the
   nodes. Since the logical rules only make use of facts from a single node,
   computation can be performed locally, independently of other nodes of the
   graph. We view applications as a communicating graph data structure where
   each processing unit performs work on a different subset of the graph, thus
   enabling concurrency. This allows LM to solve the granularity problem by
   allowing computation to be grouped into sub-graphs that can be processed by
   different processing cores. Even if there is poor work imbalance between
   cores, it is possible to easily move nodes between cores to achieve better
   load balancing and scheduling.

   \item[Experimental Results]

   We have implemented a compiler and a virtual machine prototype from scratch
   that executes on multicore machines.  We have implemented programs such as
   belief propagation, belief propagation with residual splash, PageRank, graph
   coloring, N queens, shortest path, diameter estimation, MapReduce, game of
   life, quick-sort, neural network training, among others. Our experiments
   performed on a machine with 32 cores shows that our implementation provides
   good scalability with up to 32 threads of execution.
      
\end{description}


