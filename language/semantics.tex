
Nodes are selected to computation non-deterministically (i.e., any node can be
picked to run). This means that the programmer cannot expect that facts coming
from different nodes will be considered as a whole since the process is
non-deterministic. The operational semantics promises that rule derivations are
performed atomically, therefore if a rule derives many facts belonging to a node
then that node will receive them all at once. Under these restrictions,
computation can then be parallelized by processing nodes concurrently.

Each rule in LM has a defined priority that is inferred from its position in the
source file. Rules at the beginning of the file have higher priority. At the
node level, we consider all the new facts that have not been considered yet to
create a queue of \emph{candidate rules}. The queue of candidate rules
is then applied (in priority order) and updated as new facts are derived or
consumed. As an example, consider the following three rules:

\begin{Code}
f(A), g(A) -o f(A).

h(A) -o g(A).

g(A) -o 1.
\end{Code}

If the database contains the facts \code{h(@1)} and \code{f(@1)}, then the
second rule is applied, retracting \code{h(@1)} and deriving \code{g(@1)}. Next,
the first and third rules are candidate rules, but since the first rule has
higher priority, it is applied first, resulting in a database with a single fact
\code{f(@1)}, where no rule can be applied. Later, in
Section~\ref{sec:implementation:rule_engine}, we give more details about how our
implementation manages the set of candidate rules.

