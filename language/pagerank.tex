PageRank~\cite{Page:2001:MNR} is a well known graph algorithm that is used to
compute the relative relevance of web pages.  The code for a synchronous version
of the algorithm is shown in Fig.~\ref{language:code:pagerank}. As the name
indicates, the pagerank is computed for a certain number of iterations. The
initial pagerank is the same for every page and is initialized in the first rule
(lines
\ref{line:language:spagerank_first1}-\ref{line:language:spagerank_first2}) along
with an accumulator.

The second rule of the program
(lines~\ref{line:language:spagerank_second1}-\ref{line:language:spagerank_second2})
propagates a newly computed pagerank value to all neighbors. The fact
\code{neighbor-pagerank} informs the neighbor node about the pagerank value of
node \code{A} for iteration \code{Iter + 1}. For every iteration, each node
will accumulate the all the \code{neighbor-pagerank} facts into the
\code{accumulator} fact (lines
\ref{line:language:spagerank_fourth1}-\ref{line:language:spagerank_fourth2}).
When all inbound neighbor pagerank values are accumulated, the third rule
(lines~\ref{line:language:spagerank_third1}-\ref{line:language:spagerank_third2})
is derived and a pagerank value is generated for iteration \code{Iter}.

\begin{figure}[h!]
\begin{Verbatim}[numbers=left,fontsize=\codesize,commandchars=\\\[\]]
type outbound(node, node, float).
type linear pagerank(node, float, int).
type numInbound(node, int).
type linear accumulator(node, float Acc, int Left, int Iteration).
type linear neighbor-pagerank(node, node Neighbor, float Rank, int Iteration).
type linear start(node).

const damping = 0.85. // probability of user following a link in the current page.
const iterations = str2int(@arg1). // iterations to compute.
const pages = @world. // number of pages in the graph.

start(A).

start(A), !numInbound(A, T)\label[line:language:spagerank_first1]
   -o accumulator(A, 0.0, T, 1), pagerank(A, 1.0 / float(pages), 0).\label[line:language:spagerank_first2]

pagerank(A, V, Iter), // propagate new pagerank value\label[line:language:spagerank_second1]
Iter < iterations
   -o {B, W | !outbound(A, B, W) -o neighbor-pagerank(B, A, V * W, Iter + 1)}.\label[line:language:spagerank_second2]

accumulator(A, Acc, 0, Iter), // new pagerank\label[line:language:spagerank_third1]
!numInbound(A, T),
V = damping + (1.0 - damping) * Acc,
Iter <= iterations
   -o pagerank(A, V, Iter), accumulator(A, 0.0, T, Iter + 1).\label[line:language:spagerank_third2]
	
neighbor-pagerank(A, B, V, Iter), accumulator(A, Acc, T, Iter)\label[line:language:spagerank_fourth1]
   -o accumulator(A, Acc + V, T - 1, Iter).\label[line:language:spagerank_fourth2]
\end{Verbatim}
\caption{Synchronous PageRank program.}
\label{language:code:pagerank}
\end{figure}

\subsubsection{Asynchronous PageRank}

We also have an asynchronous version of the algorithm that trades precision for
convergence speed since it does not synchronize between iterations.
Figure~\ref{language:code:async_pagerank} shows the LM code for this particular
version, where two major differences can be observed: (1) there is a linear fact
\code{neighbor-pagerank} containing the most up-to-date pagerank value of a
neighbor node; (2) a new \code{update} fact that forces the node to re-compute
its pagerank by processing the currently available \code{neighbor-pagerank}
facts. Rules in
lines~\ref{line:language:apagerank_update1}-\ref{line:language:apagerank_update2}
update the \code{neighbor-pagerank} values, while rule in
lines~\ref{line:language:apagerank_new1}-\ref{line:language:apagerank_new2}
asynchronously update the current pagerank value. This last rule derives
multiple \code{new-neighbor-rank} that is used to inform the neighbor about
the new pagerank value.

\begin{figure}[h!]
\begin{Verbatim}[numbers=left,fontsize=\codesize,commandchars=\\\#\&]
type outbound(node, node, float).
type linear pagerank(node, float, int).
type numInbound(node, int).
type linear neighbor-pagerank(node, node Neighbor, float Rank, int Iteration).
type linear new-neighbor-rank(node, node Neighbor, float Rank, int Iteration).
type linear update(A).
type linear sum-ranks(node, float).

pagerank(A, 1.0 / float(pages), 0).
update(A).
neighbor-pagerank(A, B, 1.0 / float(pages), 0). // pagerank of B is ...

// save incoming pagerank value.\label#line:language:apagerank_update1&
new-neighbor-rank(A, B, New, Iteration),
neighbor-pagerank(A, B, Old, OldIteration),
Iteration > OldIteration
   -o neighbor-pagerank(A, B, New, Iteration).
new-neighbor-rank(A, B, New, Iteration),
neighbor-pagerank(A, B, Old, OldIteration),
Iteration <= OldIteration
   -o neighbor-pagerank(A, B, Old, OldIteration).\label#line:language:apagerank_update2&

sum-ranks(A, Acc),
NewRank = damping/float(pages) + (1.0 - damping) * Acc,
pagerank(A, OldRank, Iteration)
      -o pagerank(A, NewRank, Iteration + 1),
         {B, W, Delta, Iter | !outbound(A, B, W), Delta = fabs(NewRank -
               OldRank) * W -o new-neighbor-rank(B, A, NewRank, Iteration + 1),
               if Delta > bound then update(B) end}.

update(A), update(A) -o update(A).

update(A),\label#line:language:apagerank_new1&
!numInbound(A, T)
   -o [sum => V; B, W, Val, Iter | neighbor-pagerank(A, B, Val, Iter),
         V = Val/float(T) -o neighbor-pagerank(A, B, Val, Iter) -> sum-ranks(A, V)].\label#line:language:apagerank_new2&
\end{Verbatim}
\caption{Asynchronous PageRank program.}
\label{language:code:async_pagerank}
\end{figure}

\subsubsection{Proof of correctness}

To build the proof of correctness, we must again prove several program
invariants. This will help us prove that this particular program corresponds to
a computation on a nonnegative matrix of of unit spectral radius, which has been
proven to converge~\cite{DBLP:journals/corr/abs-cs-0606047,
Lubachevsky:1986:CAA:4904.4801}.

\begin{invariant}[Page Invariant]
Each page/node has a single \code{pagerank(A, Value, Iteration)} and:
\begin{itemize}
   \item For each outbound link, a single \code{\bang outbound(A, B, W)}.
   \item For each inbound link, a single \code{neighbor-pagerank(A, B, V, Iter)}.
   \item For each \code{\bang outbound(A, B, W)}, a
      \code{neighbor-pagerank(A, B, V, Iter)}.
\end{itemize}
\end{invariant}

\begin{proof}

All initial facts validate the 3 conditions of the variant. Note that the third
condition is also validated by the initial facts, although not all facts are shown in
the code.

In relation to rule application:

\begin{itemize}
   \item Rule 1: inbound link re-derived.
   \item Rule 2: inbound link re-derived.
   \item Rule 3: \code{pagerank} re-derived.
   \item Rule 4: Nothing happens.
   \item Rule 5: inbound links re-derived in the comprehension.
\end{itemize}
\end{proof}

\begin{lemma}[Neighbor rank lemma]

Given a fact \code{neighbor-pagerank(A, B, V, Iter)} and a set of facts
\code{new-neighbor-rank(A, B, New, Iter2)}, we end up with a single
\code{neighbor-pagerank(A, B, V', Iter')}, where \code{Iter} is the greater of
\code{Iter} and \code{Iter2'}.

\end{lemma}
\begin{proof}
By induction on the number of \code{new-neighbor-rank} facts.

Base case: \code{neighbor-pagerank} remains.

Inductive case: given one \code{new-neighbor-rank} fact:

\begin{itemize}
   \item Rule 1: the new iteration is older and thus \code{neighbor-pagerank}
   is replaced. By applying induction, we know that we will select either the
   new best iteration or a better iteration from the remaining set of
   \code{new-neighbor-rank} facts.
   \item Rule 2: the new iteration is not older and we keep the old
   \code{neighbor-pagerank} fact. By induction, we select the best from either
   the current iteration or some other (from the set).
\end{itemize}
\end{proof}

\begin{lemma}[Update lemma]
Given at least 1 \code{update} fact, rule 7 will run.
\end{lemma}
\begin{proof}
By induction on the number of \code{update} facts.

Base case: rule 5 will run.

Inductive case: rule 4 will run first because it has a higher priority, reducing
the number of \code{update} facts by one. By induction, we know that by
using the remaining \code{update} facts, rule 7 will run.
\end{proof}

\begin{lemma}[Pagerank update lemma]
(1) Given at least one \code{update} fact, the \code{pagerank(A, $V_{I}$,
I)} fact will be updated to become \code{pagerank(A, $V_{I + 1}$, I +
1)}, where \code{$V_{I + 1} = damping / P + (1.0 - damping)\sum_{B,
I} (W_{B} \times  N_{I,B})$}.

where $W_B = 1.0/T$ ($T$ from \code{\bang numInbound(A, $T$)})
and $N_{I,B}$ from \code{neighbor-pagerank(A, B, $N_{I, B}$, $I$)}.

(2) For all \code{B} outbound nodes (represented using \code{\bang outbound(A, B,
W)}, a \code{new-neighbor-rank(B, A, $V_{I+1}$, $I + 1$)} is generated.

(3) For all \code{B} outbound nodes (represented using \code{\bang outbound(A, B,
W)}), a \code{update(B)} is generated if 
$fabs(V_{I + 1} - V_{I}) \times W > bound$.
\end{lemma}
\begin{proof}
Using the Update lemma, rule 5 will necessarily run.

It derives \code{sum-ranks(A, $\sum_{B, I} (W_B \times N_{I,B})$)} and
fulfills (3).

\code{sum-ranks} will necessarily fire rule 6,
computing $V_{I+1}$ and updating \code{pagerank}. (2) and (3) are fulfilled
through the comprehension of rule 6.
\end{proof}

\begin{invariant}[New neighbor rank equality]
All \code{new-neighbor-rank(A, B, V, I)} facts are generated from a corresponding
\code{pagerank(B, V, I)} fact, therefore the iteration of any
\code{new-neighbor-rank} is at least the same or less than the iteration of
the current pagerank.
\end{invariant}
\begin{proof}
No initial facts to prove.

\begin{itemize}
   \item Rule 3: true, new fact is generated.
   \item Rule 6: the fact is kept.
\end{itemize}
\end{proof}

\begin{invariant}[Neighbor rank equality]
All \code{neighbor-pagerank(A, B, V, I)} facts have one corresponding
\code{pagerank(B, V, I)} fact and the iteration of the \code{neighbor-pagerank}
is the same or less than the current iteration of the corresponding
\code{pagerank}.
\end{invariant}
\begin{proof}
By analyzing initial facts and rules.

Axioms: true.

Rule cases:

\begin{itemize}
   \item Rule 1: uses \code{new-neighbor-rank} fact (use new neighbor rank
         equality invariant).
   \item Rule 2: same fact is re-derived.
\end{itemize}
\end{proof}

\begin{theorem}[Pagerank convergence]
The program will compute the pagerank of all nodes that is within \code{bound} error
of an asynchronous pagerank computation.
\end{theorem}
\begin{proof}

Using the program initial facts, we start with the same pagerank value for all nodes.
The \code{\bang outbound(A, B, W)} fact forms a $n \times n$ square matrix (number
of nodes) and is the so-called "Google Matrix".  All the initial pagerank values
can be seen as a vector that adds up to $1$.

The pagerank computation from the "Pagerank update lemma" computes $V_{I + 1} =
damping / P + (1.0 - damping)\sum_{B, I'} (W_{B} \times N_{I',B})$, where $I'
\leq I$
(from Neighbor rank equality invariant).

Consider that each node contains a column $G_i$ of the Google matrix. The
pagerank computation can then be represented as: \newline


$V_{I + 1} = G_i fix([N_{I_1, B_1}, ..., N_{I_p, B_p}])$ \hfill (1) \\


Where $p$ is the number of inbound links and $N_{I_j, B_j}$ is the value of
the \code{neighbor-pagerank(A, $B_j$, $N_{I_j, B_j}$, $I_j$)}. The $fix()$
function takes the neighbor vector and expands it with zeros corresponding to
nodes that are not inbound links.

From~\cite{DBLP:journals/corr/abs-cs-0606047, Lubachevsky:1986:CAA:4904.4801} we
know that equation (1) will improve (converge) the pagerank value, given that some new
neighbor pagerank values are sent to node $i$ and by the fact that $G_i$ is a
nonnegative matrix of unit spectral radius. Let's use induction by assuming that there
is at least one \code{update} fact that
schedules a node to improve its pagerank. We want to prove that such fact will
not only improve the node's pagerank but also the pagerank vector.
If the pagerank vector is now close enough (within \code{bound}), then the
program will terminate.

\begin{itemize}

   \item Base case: since we have an \code{update} fact as an axiom, rule 7 will
      compute a new pagerank (Pagerank update lemma) for all nodes that is
      improved (from equation (1)). From these updates, a new \code{update} fact
      is generated that correspond to nodes that have inbound links from the
      source node and need to update their pagerank. These \code{update} facts
      may not be generated if the pagerank vector is close enough to its real
      value.

   \item The induction hypothesis tells us that there is at least one node that
      has an \code{update} fact. From pagerank update lemma, this generates
      \code{new-neighbor-rank} facts if the new value differs significantly from
      the older value. When this happens and using the ``Neighbor rank lemma'',
      the target node will update its \code{neighbor-pagerank} fact with the
      newest iteration and then execute a valid pagerank computation that brings
      the pagerank vector close to its solution.

\end{itemize}

\end{proof}

