The Quick-Sort algorithm is a divide and conquer sorting algorithm that works by
splitting a list of items into two sublists and then recursively sorting the two
sublists. To split the list, we pick a pivot element and put the items that are
smaller than the pivot into the first sublist and items greater than the pivot
into the second list.

The Quick-Sort algorithm is interesting because it does not map immediately to
the graph-based model of LM. Our approach considers that the program starts with
a single node where the initial list is located. Then the list is split as usual
and two nodes are created that will recursively sort the sublists.
Interestingly, this creates a tree that will look similar to a call tree in a
functional programming language.

Figure~\ref{language:code:quicksort} presents the code for the quick-sort
algorithm in LM. For each sublist to sort, we start with a \code{down} fact that
must be (eventually) transformed into an \code{up} fact, where the sublist in
the \code{up} fact is sorted.  In line~\ref{line:language:qs_axiom} we start
with the initial list at node \code{@0}. If the list has a small number of
items, then lines~\ref{line:language:qs_small1}-\ref{line:language:qs_small2}
will immediately sort it, otherwise the rule in
line~\ref{line:language:qs_complex} is applied.  \code{split} first splits the
list using the pivot \code{Pivot} using rules in
lines~\ref{line:language:qs_split1}-\ref{line:language:qs_split2}.  When there
is nothing more to split, the rule in
lines~\ref{line:language:qs_exists1}-\ref{line:language:qs_exists2} uses an
exists construct to create nodes \code{B} and \code{C}. The sublists are then
sent to nodes \code{B} and \code{C} using \code{down} facts.  Note, however,
that \code{back} facts are also derived and are going to be used to send the
sorted list back to the parent using the rule in
line~\ref{line:language:qs_back}.

When the sublists are finally sorted, two \code{sorted} facts are derived that
must be matched against \code{waitpivot} in the rule in
lines~\ref{line:language:qs_sorted1}-\ref{line:language:qs_sorted2}. The sorted
sublists are appended and then an \code{up} fact is finally derived
(line~\ref{line:language:qs_up}).

\begin{figure}[h!]
\begin{Verbatim}[numbers=left,fontsize=\codesize,commandchars=\*\{\}]
type linear back(node, node).
type linear down(node, list int).
type linear up(node, list int).
type linear sorted(node, node, list int).
type linear split(node, int list int, list int, list int).
type linear waitpivot(node, int, node, node).

down(@0, tosort).*label{line:language:qs_axiom}

down(A, []) -o up(A, []).*label{line:language:qs_small1}
down(A, [X]) -o up(A, [X]).
down(A, [X, Y]), X < Y -o up(A, [X, Y]).
down(A, [X, Y]), X >= Y -o up(A, [Y, X]).*label{line:language:qs_small2}
down(A, [Pivot | Xs]) -o split(A, Pivot, Xs, [], []).*label{line:language:qs_complex}

split(A, Pivot, [], Smaller, Greater) -o*label{line:language:qs_exists1}
   exists B, C. (back(B, A), back(C, A),
                 down(B, Smaller), down(C, Greater), waitpivot(A, Pivot, B, C)).*label{line:language:qs_exists2}

split(A, Pivot, [X | Xs], Smaller, Greater), X <= Pivot*label{line:language:qs_split1}
   -o split(A, Pivot, Xs, [Y | Smaller], Greater).
split(A, Pivot, [X | Xs], Smaller, Greater), X > Pivot
   -o split(A, Pivot, Xs, Smaller, [Y | Greater]).*label{line:language:qs_split2}
   
waitpivot(A, Pivot, NodeSmaller, NodeGreater),*label{line:language:qs_sorted1}
sorted(A, NodeSmaller, Smaller),
sorted(A, NodeGreater, Greater)
   -o up(A, Smaller ++ [Pivot | Greater]).*label{line:language:qs_sorted2}*label{line:language:qs_up} // Append the lists.

up(A, L), back(A, B) -o sorted(B, A, L).*label{line:language:qs_back}
\end{Verbatim}
  \caption{Quick-Sort program written in LM.}
  \label{language:code:quicksort}
\end{figure}

The Quick-Sort program shows that it is possible for applications to manipulate
and change the structure of the graph during run time. This is possible with the
use of the exists construct, which allows the programmer to create new nodes
where facts can be derived. In the case of the Quick-Sort, it allows the program
to create a tree of nodes where sorting can take place concurrently.

\subsubsection{Proof of correctness}

The proof of correctness for Quick-Sort will follow a different style than what
we have done up to this point. Instead of proving invariants, we prove what
happens to the database given the presence of some logical facts.

\begin{lemma}[Split lemma]

If fact \code{split(A, Pivot, L, Small, Great)} exists then it will be consumed
and \code{split(A, Pivot, [], Small' ++ Small, Great' ++ Great)} will be
derived, where elements of \code{Small'} are \code{<=} \code{Pivot} and elements
of \code{Great'} are \code{> Pivot}.

\end{lemma}
\begin{proof}
By induction on the size of \code{L}.
\end{proof}

\begin{theorem}[Sort theorem]
If fact \code{down(A, L)} exists then it will be consumed and a \code{up(A,
L')} fact will be derived, where \code{L'} is the sorted list.
\end{theorem}
\begin{proof}
By induction on the size of \code{L}.

The base cases are proven trivially.

In the inductive case we have:
\begin{Verbatim}[fontsize=\codesize]
down(A, [Pivot | Xs]) -o split(A, Pivot, Xs, [], []).
\end{Verbatim}

We necessarily derive \code{split(A, Pivot, Xs, [], [])}. Using the split lemma
we get \code{split(A, Pivot, [], Smaller, Greater)}. With this linear fact we
can only use the rule:

\begin{Verbatim}[fontsize=\codesize]
split(A, Pivot, [], Smaller, Greater) -o
   exists B, C. (back(B, A), back(C, A),
                 down(B, Smaller), down(C, Greater), waitpivot(A, Pivot, B, C)).
\end{Verbatim}

We derive \code{back(B, A)}, \code{back(C, A)}, \code{down(B, Smaller)},
\code{down(C, Greater)} and \code{waitpivot(A, Pivot, B, C)}. The semantics of
LM ensure that \code{B} and \code{C} are fresh node addresses, therefore those
new facts will be derived on nodes with no facts. We know that \code{Smaller}
and \code{Greater} are both smaller (in size) than \code{L}, so, if we use the
induction hypothesis, we get \code{up(B, Smaller')} and \code{up(C, Greater')}.
These last two facts will be applied using the following rule:

\begin{Verbatim}[fontsize=\codesize]
up(A, L), back(A, B) -o sorted(B, A, L).
\end{Verbatim}

Generating \code{sorted(A, B, Smaller')} and \code{sorted(A, C, Greater')}.
Furthermore, we can use the only rule that accepts \code{sorted} and
\code{waitpivot} facts:

\begin{Verbatim}[fontsize=\codesize]
waitpivot(A, Pivot, NodeSmaller, NodeGreater),
sorted(A, NodeSmaller, Smaller),
sorted(A, NodeGreater, Greater)
   -o up(A, Smaller ++ [Pivot | Greater]). // Append the lists.
\end{Verbatim}

Returning \code{up(A, Smaller' ++ [Pivot | Greater'])}.  We know that
\code{Smaller' ++ [Pivot | Greater']} is sorted since \code{Small'} contains the
sorted list of elements \code{<= Pivot} and \code{Greater'} the elements \code{>
Pivot}.

\end{proof}

