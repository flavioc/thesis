The Quick-Sort algorithm is a divide and conquer sorting algorithm that works by
splitting a list of items into two sublists and then recursively sorting the two
sublists. To split a list, it picks a pivot element and puts the items that are
smaller than the pivot into the first sublist and the items greater than the pivot
into the second list.

The Quick-Sort algorithm is interesting because it does not map immediately to
the graph-based model of LM and will demonstrate that LM supports applications
with dynamic graphs. The LM program starts with a single node where the initial
unsorted list is located. Then the list is split as usual and two nodes are
created that will recursively sort the sublists. Interestingly, this looks
similar to a call tree in a functional programming language.

Figure~\ref{language:code:quicksort} presents the code for the Quick-Sort
algorithm in LM. The code uses six predicates described as follows: \code{down}
represents a list that needs to be sorted; \code{up} is the result of sorting
an \code{down} list; \code{sorted} represents a sorted sublist; \code{back}
connects a node that is sorting a sublist to its parent node; \code{split} is
used for splitting a list into two sublists using a pivot element; and
\code{waitpivot} waits for two sorted sublists.

For each sublist, we start with a \code{down} fact that later
must be, eventually, transformed into an \code{up} fact with the sublist sorted.
In line~\ref{line:language:qs_axiom} we start with the initial list at node
\code{@0}. If the list has a small number of items (two or less), then rules at
lines~\ref{line:language:qs_small1}-\ref{line:language:qs_small2} will
immediately sort it, otherwise the rule in line~\ref{line:language:qs_complex}
is applied to split the list in 2 sublists. The fact \code{split} first splits
the list using the pivot \code{Pivot} using rules in
lines~\ref{line:language:qs_split1}-\ref{line:language:qs_split2}.  When there
is nothing else to split, the rule in
lines~\ref{line:language:qs_exists1}-\ref{line:language:qs_exists2} uses an
exists construct to create nodes \code{B} and \code{C} and then the sublists are
sent to nodes \code{B} and \code{C} using \code{down} facts.  \code{back} facts
are also derived to be used to send the sorted list back to the parent node
using the rule in line~\ref{line:language:qs_back}.

When two sublists are sorted, two \code{sorted} facts are derived that must be
matched against \code{waitpivot} in the rule in
lines~\ref{line:language:qs_sorted1}-\ref{line:language:qs_sorted2}. The sorted
sublists are appended and send \code{up} to the parent node via the derivation
of an \code{up} fact (line~\ref{line:language:qs_up}).

\begin{figure}[h!]
\begin{LineCode}[commandchars=\*\{\}]
type linear down(node, list int).*hfill// Predicate declaration
type linear up(node, list int).
type linear sorted(node, node, list int).
type linear back(node, node).
type linear split(node, int list int, list int, list int).
type linear waitpivot(node, int, node, node).

down(A, []) -o up(A, []).*label{line:language:qs_small1}*hfill// Rule 1: empty list

down(A, [X]) -o up(A, [X]).*hfill// Rule 2: single element list

down(A, [X, Y]), X < Y -o up(A, [X, Y]).*hfill// Rule 3: two element list

down(A, [X, Y]), X >= Y -o up(A, [Y, X]).*label{line:language:qs_small2}*hfill// Rule 4: two element list

down(A, [Pivot | Xs])*label{line:language:qs_complex}*hfill// Rule 5: lists with more than two elements
   -o split(A, Pivot, Xs, [], []).

split(A, Pivot, [], Smaller, Greater) -o*label{line:language:qs_exists1}*hfill// Rule 6: create nodes to sort sublists
   exists B, C. (back(B, A), back(C, A),
                 down(B, Smaller), down(C, Greater), waitpivot(A, Pivot, B, C)).*label{line:language:qs_exists2}

split(A, Pivot, [X | Xs], Smaller, Greater), X <= Pivot*label{line:language:qs_split1}*hfill// Rule 7: split case 1
   -o split(A, Pivot, Xs, [Y | Smaller], Greater).

split(A, Pivot, [X | Xs], Smaller, Greater), X > Pivot*hfill// Rule 8: split case 2
   -o split(A, Pivot, Xs, Smaller, [Y | Greater]).*label{line:language:qs_split2}
   
waitpivot(A, Pivot, NodeSmaller, NodeGreater),*label{line:language:qs_sorted1}*hfill// Rule 9: merge sublists
sorted(A, NodeSmaller, Smaller),
sorted(A, NodeGreater, Greater)
   -o up(A, Smaller ++ [Pivot | Greater]).*label{line:language:qs_sorted2}*label{line:language:qs_up}

up(A, L), back(A, B) -o sorted(B, A, L).*label{line:language:qs_back}*hfill// Rule 10: send list to parent

down(@0, initial_list).*label{line:language:qs_axiom}*hfill// Initial facts
\end{LineCode}
  \mycap{Quick-Sort program written in LM.}
  \label{language:code:quicksort}
\end{figure}

The use of the exists construct allows the programmer to create new nodes where
facts can be derived. In the case of the Quick-Sort, it allows the program to
create a tree of nodes where sorting can take place concurrently.

The amount of concurrency available in the Quick-Sort program depends on the
quality of the selected pivot. If the pivot splits the list in equal parts, then
there is more concurrency because it is now possible to work on the two halves of
the list concurrently. If a bad pivot is selected, then we may end up in
situations where the pivot is the smallest (or largest) element of the list,
splitting the list into an empty list and a list with $n-1$ elements. It is
clear that the amount of work required to sort the empty list is much smaller
than the work required to sort the larger list.  Repeatedly choosing a bad pivot
will effectively turn Quick-Sort into a sequential algorithm. This is not
surprising since it is directly related to the Quick-Sort's average and worst
case complexity, $\mathcal{O}(n \log{n})$ and $\mathcal{O}(n^2)$, respectively.

The proof of correctness for Quick-Sort follows a different style than the
proofs done so far. Instead of proving invariants, we prove what happens to the
database given the presence of some logical facts.

\begin{lemma}[Split lemma]

If a $\mathtt{split}(A, Pivot, L, Small, Great)$ fact exists then it will be
consumed to derive a $\mathtt{split}(A, Pivot, [], Small' ++ Small, Great' ++
Great)$ fact, where the elements of $Small'$ are lesser or equal than $Pivot$
and the elements of $Great'$ are greater than $Pivot$.

\end{lemma}
\begin{proof}
By induction on the size of $L$.
\end{proof}

\begin{theorem}[Sort theorem]

If a $\mathtt{down}(A, L)$ fact exists then it will be consumed and a
$\mathtt{up}(A, L')$ fact will be derived, where $L'$ is the sorted list of $L$.

\end{theorem}
\begin{proof}
By induction on the size of $L$.

The base cases are proven trivially (rules 1-4).

In the inductive case, only rule 5 applies:
\begin{Code}
down(A, [Pivot | Xs]) -o split(A, Pivot, Xs, [], []).
\end{Code}

\noindent which necessarily derives a $\mathtt{split}(A, Pivot, Xs, [], [])$
fact. By applying the split lemma, a $\mathtt{split}(A, Pivot, [], Smaller, Greater)$
fact is generated, from which only rule 6 can be used:

\begin{Code}
split(A, Pivot, [], Smaller, Greater) -o
   exists B, C. (back(B, A), back(C, A),
                 down(B, Smaller), down(C, Greater), waitpivot(A, Pivot, B, C)).
\end{Code}

\noindent which necessarily derives $\mathtt{back}(B, A)$, $\mathtt{back}(C,
A)$, $\mathtt{down}(B, Smaller)$, $\mathtt{down}(C, Greater)$ and also a
$\mathtt{waitpivot}(A, Pivot, B, C)$ fact. The semantics of LM ensure that $B$
and $C$ are fresh node addresses, therefore those new facts will be derived on
nodes with no facts. The lists $Smaller$ and $Greater$ are both smaller (in
size) than $L$, so, by the the induction hypothesis, an $\mathtt{up}(B,
Smaller')$ and an $\mathtt{up}(C, Greater')$ facts are derived. These last two
facts will be used in the following rule:

\begin{Code}
up(A, L), back(A, B) -o sorted(B, A, L).
\end{Code}

\noindent which generates a $\mathtt{sorted}(A, B, Smaller')$ and a $\mathtt{sorted}(A, C,
Greater')$ facts. In the continuation, there is only one rule that accepts
\code{sorted} and \code{waitpivot} facts:

\begin{Code}
waitpivot(A, Pivot, NodeSmaller, NodeGreater),
sorted(A, NodeSmaller, Smaller),
sorted(A, NodeGreater, Greater)
   -o up(A, Smaller ++ [Pivot | Greater]).
\end{Code}

\noindent returning $\mathtt{up}(A, Smaller' ++ [Pivot | Greater'])$.  We know that
$Smaller' ++ [Pivot | Greater']$ is sorted since $Small'$ contains the sorted
list of elements lesser or equal than $Pivot$ and $Greater'$ the elements
greater than $Pivot$.

\end{proof}

