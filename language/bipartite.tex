The problem of checking if a graph is bipartite can be seen as a 2-color graph
coloring problem. The code for this algorithm is shown in
Fig.~\ref{language:code:bichecking}. The code declares five predicates, namely:
\code{edge}, to specify the structure of the graph; \code{uncolored}, to mark
nodes as uncolored; \code{colored}, to mark nodes as colored and the node's
color; \code{fail}, to mark an invalid bipartite graph; and \code{visit} to
perform the coloring of the graph.  Initially, all nodes in the graph start as
\code{uncolored}, because they do not have a color yet. The initial fact
\code{visit(@1, 1)} is instantiated at node \code{@1}
(line~\ref{line:language:bc_axiom}) in order to start the coloring process by
assigning it with color 1.

If a node is \code{uncolored} and needs to be marked with a color \code{P} then
the rule in lines~\ref{line:language:bc_first1}-\ref{line:language:bc_first2} is
applied. We consume the \code{uncolored} fact and derive a \code{colored(A, P)}
to effectively color the node with \code{P}. We also derive \code{visit(B,
next(P))} in neighbor nodes to color them with the other color.  The coloring
can fail if a node is already colored with a color \code{P} and needs to be
colored with a different color (line~\ref{line:language:bc_third}) or if it has
already failed (line~\ref{line:language:bc_fourth}).

\begin{figure}[h!]
\begin{Verbatim}[numbers=left,fontsize=\codesize,commandchars=\*\[\]]
type edge(node, node).*hfill// Predicate declaration
type linear uncolored(node).
type linear colored(node, int).
type linear fail(node).
type linear visit(node, int).

fun next(int X) : int = if X <> 1 then 1 else 2 end.*hfill// Function declaration

visit(A, P), uncolored(A)*label[line:language:bc_first1]*hfill// Rule 1: coloring a node
   -o {B | !edge(A, B) -o visit(B, next(P))},
      colored(A, P).*label[line:language:bc_first2]

visit(A, P), colored(A, P) -o colored(A, P).*label[line:language:bc_second]*hfill// Rule 2: node is already colored

visit(A, P1), colored(A, P2), P1 <> P2 -o fail(A).*label[line:language:bc_third]*hfill// Rule 3: graph is not bipartite

visit(A, P), fail(A) -o fail(A).*label[line:language:bc_fourth]*hfill// Rule 4: graph is still not bipartite

visit(@1, 1).*label[line:language:bc_axiom]*hfill// Initial facts
unvisited(A).*label[line:language:bc_unvisited]
\end{Verbatim}
  \caption{Bipartiteness Checking program.}
  \label{language:code:bichecking}
\end{figure}

In order to show that the code in Fig.~\ref{language:code:bichecking} works as
intended, we first setup some invariants that hold throughout the execution of
the program. Assume that the set of nodes in the graph is represented as $N$.

\begin{invariant}[Node state]
The set of nodes $N$ is partitioned into 4 different states that represent the 4
possible states that a node can be in, namely:

\begin{itemize}
   \item $U$ (\code{uncolored} nodes)
   \item $F$ (\code{fail} nodes)
   \item $C_{1}$ ($\mathtt{colored}(A, 1)$ nodes)
   \item $C_{2}$ ($\mathtt{colored}(A, 2)$ nodes)
\end{itemize}
\end{invariant}
\begin{proof}
Initially, all nodes start in set $U$ (line~\ref{line:language:bc_unvisited}
of Fig.~\ref{language:code:bichecking}. All the 4 rules of the programs either
keep the node in the same set or exchange the node with another set.
\end{proof}

A bipartite graph is one where in every $\mathtt{edge}(a, b)$, there is a valid
   assignment that makes $a$ member of set $C_{1}$ or $C_{2}$ and node
   $b$ member of either $C_{2}$ or $C_{1}$ respectively.

\begin{variant}[Bipartiteness
   Convergence]\label{language:lemma:bipartite_convergence}

   We now reason from the application of the program rules. After each
   application of an inference rule, one of the following events will happen:

   \begin{enumerate}
      \item Set $U$ will decrease and set $C_{1}$ or $C_{2}$ will
         increase, with a potential increase in the number of \code{visit}
         facts.

      \item Set $C_{1}$ or $C_{2}$ will stay the same, while the number
         of \code{visit} facts will be reduced.

      \item Set $C_{1}$ or $C_{2}$ will decrease and set $F$ will
         increase, while the number of \code{visit} facts will be reduced.

      \item Set $F$ will stay the same, while the number of \code{visit} facts
         decreases.
   \end{enumerate}

\end{variant}
\begin{proof}
Trivially from the rules.
\end{proof}

From this variant, it can be inferred that set $U$ never increases in size
and in a node transition from \code{uncolored} to \code{colored}, the
database may increase in size. For every other rule application, the database of
facts always decreases. This also means that the program will eventually
terminate, since it is limited by the number of \code{visit} facts that can be
generated.

\begin{theorem}[Bipartiteness Correctness]

If the graph is connected and bipartite then the nodes will be partitioned into sets
$C_{1}$ and $C_{2}$, while sets $F$ and $U$ will be empty.

\end{theorem}
\begin{proof}
   By induction, we prove that uncolored nodes become part of either $C_{1}$
   or $C_{2}$ and, if there is an edge between nodes in the two sets then
   they have different colors.

   In the base case, we start with empty sets but node \code{@1} is made
   member of $C_{1}$. Rule 1 sends \code{visit} facts to the neighbors of
   \code{@1}, forcing them to be members of $C_{2}$.

   In the inductive case, we have sets $C'_{1}$ and $C'_{2}$ with some nodes
   already colored. From Variant~\ref{language:lemma:bipartite_convergence}, we
   know that $U$ always decreases. Since the graph is bipartite, events 3 and 4
   never happen since there is a possible partitioning of nodes. With event 1,
   we have set $C_{1} = C'_{1} \cup \{n\}$, (or $C_{2}$) where $n$ is the new
   colored node. With event 2, the sets remain the same. Since the graph is
   connected, every node will be colored, therefore event 1 will happen for
   every node of the graph.

\end{proof}

\begin{theorem}[Bipartiteness Failure]
If the graph is connected but not bipartite then some nodes will be part of set
$F$.
\end{theorem}
\begin{proof}

Assume that the algorithm completely partitions the nodes into sets $C_{1}$ and
$C_{2}$ and thus $F$ is empty. Since the graph is connected, we know that the
algorithm tries to build a valid partitioning represented by $C_{1}$ and
$C_{2}$.  This is a contradiction because the graph is not bipartite (by
definition) and thus at least one node will be part of set $F$ with rule 3.

\end{proof}

