\subsection{Graph-Based Programming Models}

Many programming systems model programs as graphs where computation is
performed.  Some good examples are the Dryad, Pregel, GraphLab, Ligra, Grace and
Galois systems.

The Dryad system~\cite{Isard:2007:DDD:1272996.1273005} combines computational
vertices with communication channels (edges) to form a data-flow graph. The
program is scheduled to run on multiple processors or cores and data is
partitioned during runtime. Routines that run on computational vertices are
sequential, with no synchronization.

The Pregel system~\cite{Malewicz:2010:PSL:1807167.1807184} is also graph based,
although programs have a more strict structure. They must be represented as
a sequence of iterations where each iteration is composed of computation and
message passing.  Pregel is specially suited to work on big graphs and to
scale to large architectures.

GraphLab~\cite{GraphLab2010} is a C++ framework for developing parallel machine
learning algorithms. While Pregel uses message passing, GraphLab allows nodes to
have read/write access to different scopes through different concurrent access
models in order to balance performance and data consistency. While some programs
only need to access the local node's data, others may need to update edge
information. Each consistency model will provide different guarantees that are
better adapted to some algorithms. GraphLab also provides different schedulers
that dictate the order in which node's are computed.

Ligra~\cite{Shun:2013:LLG:2517327.2442530} is lightweight framework for large scale
graph processing on a single multicore machine. Ligra exploits the fact that
most huge graph datasets available today can be made to fit in the main memory
of commodity servers. Ligra is a simple framework that exposes 
two main interfaces: \texttt{EdgeMap} and \texttt{VertexMap}. The former applies
a function to a subset of edges of the graph, while the latter applys a function
to a subset of vertices. The functions passed as arguments are applied to either
a single edge or a single vertex and the user must ensure that the function can
be executed in parallel. The framework allows the use of \emph{Compare-and-Swap}
instructions when implementing functions in order to avoid race conditions.

Grace~\cite{wang:asynchronous} is another graph-based framework for multicore
machines. Unlike Ligra, Grace programs are implemented from implemented from the
point of view of a vertex. Each vertex and edge can be customized with different
data depending on the target application. By default, programs are executed
iteratively: for each iteration the vertex program reads incoming messages
located in the edges, performs computation and sends messages to the outbound
edges. Since iterative programs require synchronization after each iteration,
Grace allows the user to relax these constraints and implement customizable
execution policies by implementing code for describing which vertices are
allowed to run and in which order. The order is dictated by assigning a
\emph{scheduling priority value}.

Galois~\cite{Pingali:2011:TPA:1993316.1993501} is a parallel programming model
with irregular applications based on graphs, trees and sets. A Galois parallel
algorithm is viewed as a parallel application of an \emph{operator} over an
irregular data structure which generate \emph{activies} on the data structure.
Such operator may, for instance, be applied on the
node of the graph in order to change its data or change the structure of its
neighborhood, allowing for data structure changes. In Galois, we consider
\emph{active elements}, which are nodes of the graph where computation needs to
be performed, the \emph{neighborhood} containing the nodes of the graph required
to perform an operator, and \emph{ordering} which dictactes the order in which
operators are applied. From the point of view of the programer, the active
elements are represented in a worklist, while operators can be implemented on
top of iterators of the worklist. Galois supports speculative execution by
allowing operator rollback when an operator requires the use of a node that is
held by another operator.

\subsubsection{Languages for Sensor Networks}

Sensor nets are another interesting domain for programming languages that deal
with graph-like networks.  Programming languages such as
Hood~\cite{Whitehouse:2004:HNA:990064.990079},
Tinydb~\cite{Madden:2005:TAQ:1061318.1061322} or
Regiment~\cite{Newton:2007:RMS:1236360.1236422} have been proposed for this
particular domain, providing support for data collection and aggregation over
the network.  These systems assume that the network remains static and nodes
stay in place.

Other languages such as Pleiades~\cite{Kothari:2007:REP:1250734.1250757},
LDP~\cite{4543691} or Proto~\cite{Beal:2006:IEE:1137236.1137354} go beyond
static networks and support dynamic reconfiguration. In Pleiades, the
programmer writes an application from the point of view of the whole
sensor network and the compiler transforms it into code that can be run on
each individual node.  LDP is a language derived from a method for
distributed debugging, that allows it to efficiently detect conditions on
variably-sized groups of nodes. It is based on the tick model, generating
a new set of condition matchers throughout the ensemble on each tick.
Like Pleiades and LDP, Proto also compiles global programs into locally
executed code.

\subsection{Constraint Handling Rules}

Since LM is a bottom-up linear logic programming language, it also shares
similarities with Constraint Handling Rules
(CHR)~\cite{Betz:2005kx,Betz:2013:LBA:2422085.2422086}.  CHR is a concurrent
committed-choice constraint language used to write constraint solvers. A CHR
program is a set of rules and a set of constraints. Constraints can be consumed
or generated during the application of rules.  Unlike LM, in CHR there is no
concept of rule priorities, but there is an extension to CHR that supports
them~\cite{DeKoninck:2007:URP:1273920.1273924}.  Finally, there is also a CHR
extension that adds persistent constraints and it has been proven to be sound
and complete~\cite{DBLP:journals/corr/abs-1007-3829}.

\subsection{Graph Transformation Systems}

Graph Transformation Systems (GTS)~\cite{Ehrig:2004vn}, commonly used to model
distributed systems, perform rewriting of graphs through a set of graph
productions. GTS also introduces concepts of concurrency, where it may be
possible to apply several transformations at the same time. In principle, it
should be possible to model LM programs as a graph transformation: we directly
map the LM graph of nodes to GTS's initial graph and consider logical facts as
nodes that are connected to LM's nodes. Each LM rule is then a graph production
that manipulates the node's neighbors (the database) or sends new facts to
other nodes. On the other hand, it is also possible to embed GTS inside
CHR~\cite{Raiser:2011:AGT:1972935.1972938}.
