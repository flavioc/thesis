In this section, we present solutions to well-known problems. We start with
straightforward graph-based problems such as bipartitness checking and two
versions of the PageRank program. Next, we present the LM version of the
Quick-Sort algorithm, which from a first impression may not fit well under the
programming paradigm offered by LM. Informal correctness and termination proofs
are also included to further show that such important properties are relatively
easy to prove for programs written in LM.

\subsection{Bipartiteness Checking}

The problem of checking if a graph is bipartite can be seen as a 2-color graph
coloring problem.  The code for this algorithm is shown in
Fig.~\ref{language:code:bichecking}. All nodes in the graph start as
\texttt{uncolored}, because they do not have a color yet. The axiom
\texttt{visit(@1, 1)} is instantiated at node \texttt{@1}
(line~\ref{line:language:bc_axiom}) in order to color it with color 1.

If a node is \texttt{uncolored} and needs to be marked with a color \texttt{P}
then the rule in
lines~\ref{line:language:bc_first1}-\ref{line:language:bc_first2} is applied. We
consume the \texttt{uncolored} fact and derive a \texttt{colored(A, P)} to
effectively color the node with \texttt{P}. We also derive \texttt{visit(B,
next(P))} in neighbor nodes to color them with the other color. 

The coloring can fail if a node is already colored with a color \texttt{P} and
needs to be colored with a different color (line~\ref{line:language:bc_third})
or if it has already failed (line~\ref{line:language:bc_fourth}).

\begin{figure}[h!]
\begin{Verbatim}[numbers=left,fontsize=\codesize,commandchars=\*\[\]]
type route edge(node, node).
type linear visit(node, int).
type linear uncolored(node).
type linear colored(node, int).
type linear fail(node).

fun next(int X) : int = if X <> 1 then 1 else 2 end.

visit(@1, 1).*label[line:language:bc_axiom]

visit(A, P), uncolored(A)*label[line:language:bc_first1]
   -o {B | !edge(A, B) -o visit(B, next(P))},
      colored(A, P).*label[line:language:bc_first2]

visit(A, P), colored(A, P) -o colored(A, P).*label[line:language:bc_second]
visit(A, P1), colored(A, P2), P1 <> P2 -o fail(A).*label[line:language:bc_third]
visit(A, P), fail(A) -o fail(A).*label[line:language:bc_fourth]
\end{Verbatim}
  \caption{Bipartiteness Checking program.}
  \label{language:code:bichecking}
\end{figure}

\subsubsection{Proof of correctness}

In order to show that the code in Fig.~\ref{language:code:bichecking} works as
intended, we first setup some invariants that hold throughout the execution of
the program. Assume that the set of nodes in the graph is represented as $N$.

\begin{invariant}[Node state]
Set of nodes $N$ is partitioned into 4 different states that represent the 4
possible states that a node can be in, namely:

\begin{itemize}
   \item $U$ (\texttt{uncolored} nodes)
   \item $F$ (\texttt{fail} nodes)
   \item $C_{true}$ (\texttt{colored(A, 1)} nodes)
   \item $C_{false}$ (\texttt{colored(A, 2)} nodes)
\end{itemize}
\end{invariant}
\begin{proof}
Initially, all nodes start in set $U$. All the 4 rules of the programs either
keep the node in the same set or exchange the node with another set.
\end{proof}

A bipartite graph is one where in every edge $a \leftrightarrow b$, there is a
valid assignment that makes $a$ member of set $C_{true}$ or $C_{false}$ and node
$b$ member of either $C_{false}$ or $C_{true}$ respectively.

\begin{lemma}[Bipartiteness
   Convergence]\label{language:lemma:bipartite_convergence}
   We now reason from the application of the program rules. After each
   application of an inference rule, one of the following will happen:

   \begin{itemize}
      \item Set $U$ will decrease and set $C_{true}$ or $C_{false}$ will
         increase, with a potential increase in the number of \texttt{visit}
         facts.
      \item Set $C_{true}$ or $C_{false}$ will stay the same, while the number
         of \texttt{visit} facts will be reduced.

      \item Set $C_{true}$ or $C_{false}$ will decrease and set $F$ will
         increase, while the number of \texttt{visit} facts will be reduced.

      \item Set $F$ will stay the same, while the number of \texttt{visit} facts
         decreases.
   \end{itemize}

\end{lemma}
\begin{proof}
Directly from the rules.
\end{proof}

From this invariant, it can be inferred that set $U$ never increases in size
and in a node transition from \texttt{uncolored} to \texttt{colored}, the
database may increase in size. For every other rule application, the database of
facts always decreases. This also means that the program will eventually
terminate, since it is limited by the number of \texttt{visit} facts that can be
generated.

\begin{theorem}[Bipartiteness Correctness]
If the graph is connected and bipartite then the nodes will be partitioned into
sets $C_{true}$ and $C_{false}$, while sets $F$ and $U$ are empty.
\end{theorem}
\begin{proof}
   By induction, we prove that uncolored nodes become part of either $C_{true}$
   and $C_{false}$ and, if there is an edge between nodes in the two sets then
   they have different colors.

   In the base case, we start with empty sets but node \texttt{@1} is made
   member of $C_{true}$. Rule 1 sends \texttt{visit} facts to the neighbors of
   \texttt{@1}, forcing them to be members of $C_{false}$.

   In the inductive case, we have sets $C'_{true}$ and $C'_{false}$ with some
   nodes already colored. From Lemma~\ref{language:lemma:bipartite_convergence},
   we know that $U$ always decreases. Since the graph is bipartite, events 3 and
   4 never happen since there is a possible partitioning of nodes. With event 1,
   we have set $C_{true} = C'_{true}, n$, (or $C_{false}$) where $n$ is the
   node and with event 2, the sets remain the same. Since the graph is
   connected, every node will be colored, therefore event 1 will happen for
   every node of the graph.
\end{proof}

\subsection{PageRank}\label{section:language:pagerank}
PageRank~\cite{Page:2001:MNR} is a well known graph algorithm that is used to
compute the relative relevance of web pages. The standard formulation of the
PageRank algorithm uses three $n \times n$ matrices (where $n$ is the number of
pages):

\begin{itemize}

   \item An adjacency matrix $A$, where $A_{ij}$ is $1$ if page $i$ has an
      outgoing link to $j$;

   \item A transition matrix $P$, where $P_{ij} = A_{ij}/deg(i)$ and $\deg(i)$
      is the number of outgoing links for page $i$;

   \item A \scare{Google matrix} $G$, where $G = \alpha P + (1-\alpha)I$, where
      $\alpha$ is the \emph{damping factor} and $I$ is a $n \times n$ matrix
      where $I_{ij}$ is $1$. The damping factor is the probablity of a user
      jumping to a random page instead of following the links of the current
      page.

\end{itemize}

The PageRank vector $x$ of size $1 \times n$ is the solution to the following
linear system:

\begin{align}
x = G x
\end{align}

To solve this problem, it is possible to use an iterative method by starting
with an initial vector $x(0)$ where all pages start with the same value (that
adds up to 1) and then perform the following computation:

\begin{align}
x(t + 1) = G x(t)
\end{align}

In order to distribute this calculation, we have the following formula for page
$i$:

\begin{align}
x_{i}(t + 1) = G_{i} x(t)\label{eq:language:pagerank}
\end{align}

Where $G_{i}$ is the row of the $G$ matrix that corresponds to the inbound links
of page $i$. Note that we also do not need to have the complete $x(t)$ vector
most elements of $G_{i}$ are $0$.

The code for this iterative computation is shown in
Fig.~\ref{language:code:pagerank} and starts by declaring six predicates,
namely: \code{outbound}, to declare outbound links where the third argument
represents a value in the transition matrix $P$; \code{numInbound}, with the
number of inbound links; \code{pagerank}, to represent the PageRank of a page;
\code{accumulator}, to accumulate the incoming neighbor's PageRank values;
\code{neighbor-pagerank}, to represent an incoming PageRank value; and
\code{start} to initialize a node.  Since this program is asynchronous, the
PageRank values of all nodes must be computed before the next PageRank is
computed.

We use constant definitions provided by LM
(lines~\ref{line:language:spagerank_const1}-\ref{line:language:spagerank_const2})
to refer to constant values used throughout the program. The \code{damping}
constant is the damping factor $\alpha$ used in the PageRank calculations. The
constant \code{iterations} reads the number of iterations to execute from the
program's input arguments and \code{pages} is assigned to \code{@world}, a
special constant that evaluates to the number of nodes in the program (in this
case, the number of pages).

The initial PageRank representing $x_i(0)$ is initialized in the first rule
(lines
\ref{line:language:spagerank_first1}-\ref{line:language:spagerank_first2}) along
with the accumulator. All the initial PageRank values form the initial $x(0)$
vector. The second rule of the program
(lines~\ref{line:language:spagerank_second1}-\ref{line:language:spagerank_second2})
propagates a newly computed PageRank value to all neighbors and represents a
step in the iterative method for a column in the $G$ matrix. The fact
\code{neighbor-pagerank} informs the neighbor node about the PageRank value of
node \code{A} for iteration \code{Iter + 1}. For every iteration, each node will
accumulate the all the \code{neighbor-pagerank} facts into the
\code{accumulator} fact (lines
\ref{line:language:spagerank_fourth1}-\ref{line:language:spagerank_fourth2}).
When all inbound neighbor PageRank values are accumulated, the third rule
(lines~\ref{line:language:spagerank_third1}-\ref{line:language:spagerank_third2})
is derived and a PageRank value is generated for iteration \code{Iter}.

\begin{figure}[h!]
\begin{Verbatim}[numbers=left,fontsize=\codesize,commandchars=\\\[\]]
type outbound(node, node, float).
type numInbound(node, int).
type linear pagerank(node, float, int).
type linear accumulator(node, float Acc, int Left, int Iteration).
type linear neighbor-pagerank(node, node Neighbor, float Rank, int Iteration).
type linear start(node).

const damping = 0.85. // probability of user following a link in the current page.\label[line:language:spagerank_const1]
const iterations = str2int(@arg1). // iterations to compute.
const pages = @world. // number of pages in the graph.\label[line:language:spagerank_const2]

start(A).

start(A), !numInbound(A, T)\label[line:language:spagerank_first1]
   -o accumulator(A, 0.0, T, 1), pagerank(A, 1.0 / float(pages), 0).\label[line:language:spagerank_first2]

pagerank(A, V, Iter), // propagate new pagerank value\label[line:language:spagerank_second1]
Iter < iterations
   -o {B, W | !outbound(A, B, W) -o neighbor-pagerank(B, A, V * W, Iter + 1)}.\label[line:language:spagerank_second2]

accumulator(A, Acc, 0, Iter), // new pagerank\label[line:language:spagerank_third1]
!numInbound(A, T),
V = damping + (1.0 - damping) * Acc,
Iter <= iterations
   -o pagerank(A, V, Iter), accumulator(A, 0.0, T, Iter + 1).\label[line:language:spagerank_third2]
	
neighbor-pagerank(A, B, V, Iter), accumulator(A, Acc, T, Iter)\label[line:language:spagerank_fourth1]
   -o accumulator(A, Acc + V, T - 1, Iter).\label[line:language:spagerank_fourth2]
\end{Verbatim}
\caption{Synchronous PageRank program.}
\label{language:code:pagerank}
\end{figure}

The synchronous version of the PageRank algorithm has a fairly high level of
concurrency. First, the program starts on all nodes of the graph, which makes
the program trivial to parallelize. However, because this is a synchronous
algorithm, there is a need to synchronize because there is a data dependency
between PageRank iterations. Furthermore, nodes can only compute their next
iteration after they receive all the neighbor's PageRank values from the
previous iteration.

Appendix~\ref{section:appendix:pagerank} further expands this section with the
asynchronous version of PageRank and also includes a proof of correctness.

\subsection{Quick-Sort}
The Quick-Sort algorithm is a divide and conquer sorting algorithm that works by
splitting a list of items into two sublists and then recursively sorting the two
sublists. To split a list, it picks a pivot element and puts the items that are
smaller than the pivot into the first sublist and the items greater than the pivot
into the second list.

The Quick-Sort algorithm is interesting because it does not map immediately to
the graph-based model of LM and will demonstrate that LM supports applications
with dynamic graphs. The LM program starts with a single node where the initial
unsorted list is located. Then the list is split as usual and two nodes are
created that will recursively sort the sublists. Interestingly, this looks
similar to a call tree in a functional programming language.

Figure~\ref{language:code:quicksort} presents the code for the Quick-Sort
algorithm in LM. The code uses six predicates described as follows: \code{down}
represents a list that needs to be sorted; \code{up} is the result of sorting
an \code{down} list; \code{sorted} represents a sorted sublist; \code{back}
connects a node that is sorting a sublist to its parent node; \code{split} is
used for splitting a list into two sublists using a pivot element; and
\code{waitpivot} waits for two sorted sublists.

For each sublist, we start with a \code{down} fact that later
must be, eventually, transformed into an \code{up} fact with the sublist sorted.
In line~\ref{line:language:qs_axiom} we start with the initial list at node
\code{@0}. If the list has a small number of items (two or less), then rules at
lines~\ref{line:language:qs_small1}-\ref{line:language:qs_small2} will
immediately sort it, otherwise the rule in line~\ref{line:language:qs_complex}
is applied to split the list in 2 sublists. The fact \code{split} first splits
the list using the pivot \code{Pivot} using rules in
lines~\ref{line:language:qs_split1}-\ref{line:language:qs_split2}.  When there
is nothing else to split, the rule in
lines~\ref{line:language:qs_exists1}-\ref{line:language:qs_exists2} uses an
exists construct to create nodes \code{B} and \code{C} and then the sublists are
sent to nodes \code{B} and \code{C} using \code{down} facts.  \code{back} facts
are also derived to be used to send the sorted list back to the parent node
using the rule in line~\ref{line:language:qs_back}.

When two sublists are sorted, two \code{sorted} facts are derived that must be
matched against \code{waitpivot} in the rule in
lines~\ref{line:language:qs_sorted1}-\ref{line:language:qs_sorted2}. The sorted
sublists are appended and send \code{up} to the parent node via the derivation
of an \code{up} fact (line~\ref{line:language:qs_up}).

\begin{figure}[h!]
\begin{LineCode}[commandchars=\*\{\}]
type linear down(node, list int).*hfill// Predicate declaration
type linear up(node, list int).
type linear sorted(node, node, list int).
type linear back(node, node).
type linear split(node, int list int, list int, list int).
type linear waitpivot(node, int, node, node).

down(A, []) -o up(A, []).*label{line:language:qs_small1}*hfill// Rule 1: empty list

down(A, [X]) -o up(A, [X]).*hfill// Rule 2: single element list

down(A, [X, Y]), X < Y -o up(A, [X, Y]).*hfill// Rule 3: two element list

down(A, [X, Y]), X >= Y -o up(A, [Y, X]).*label{line:language:qs_small2}*hfill// Rule 4: two element list

down(A, [Pivot | Xs])*label{line:language:qs_complex}*hfill// Rule 5: lists with more than two elements
   -o split(A, Pivot, Xs, [], []).

split(A, Pivot, [], Smaller, Greater) -o*label{line:language:qs_exists1}*hfill// Rule 6: create nodes to sort sublists
   exists B, C. (back(B, A), back(C, A),
                 down(B, Smaller), down(C, Greater), waitpivot(A, Pivot, B, C)).*label{line:language:qs_exists2}

split(A, Pivot, [X | Xs], Smaller, Greater), X <= Pivot*label{line:language:qs_split1}*hfill// Rule 7: split case 1
   -o split(A, Pivot, Xs, [Y | Smaller], Greater).

split(A, Pivot, [X | Xs], Smaller, Greater), X > Pivot*hfill// Rule 8: split case 2
   -o split(A, Pivot, Xs, Smaller, [Y | Greater]).*label{line:language:qs_split2}
   
waitpivot(A, Pivot, NodeSmaller, NodeGreater),*label{line:language:qs_sorted1}*hfill// Rule 9: merge sublists
sorted(A, NodeSmaller, Smaller),
sorted(A, NodeGreater, Greater)
   -o up(A, Smaller ++ [Pivot | Greater]).*label{line:language:qs_sorted2}*label{line:language:qs_up}

up(A, L), back(A, B) -o sorted(B, A, L).*label{line:language:qs_back}*hfill// Rule 10: send list to parent

down(@0, initial_list).*label{line:language:qs_axiom}*hfill// Initial facts
\end{LineCode}
  \mycap{Quick-Sort program written in LM.}
  \label{language:code:quicksort}
\end{figure}

The use of the exists construct allows the programmer to create new nodes where
facts can be derived. In the case of the Quick-Sort, it allows the program to
create a tree of nodes where sorting can take place concurrently.

The amount of concurrency available in the Quick-Sort program depends on the
quality of the selected pivot. If the pivot splits the list in equal parts, then
there is more concurrency because it is now possible to work on the two halves of
the list concurrently. If a bad pivot is selected, then we may end up in
situations where the pivot is the smallest (or largest) element of the list,
splitting the list into an empty list and a list with $n-1$ elements. It is
clear that the amount of work required to sort the empty list is much smaller
than the work required to sort the larger list.  Repeatedly choosing a bad pivot
will effectively turn Quick-Sort into a sequential algorithm. This is not
surprising since it is directly related to the Quick-Sort's average and worst
case complexity, $\mathcal{O}(n \log{n})$ and $\mathcal{O}(n^2)$, respectively.

The proof of correctness for Quick-Sort follows a different style than the
proofs done so far. Instead of proving invariants, we prove what happens to the
database given the presence of some logical facts.

\begin{lemma}[Split lemma]

If a $\mathtt{split}(A, Pivot, L, Small, Great)$ fact exists then it will be
consumed to derive a $\mathtt{split}(A, Pivot, [], Small' ++ Small, Great' ++
Great)$ fact, where the elements of $Small'$ are lesser or equal than $Pivot$
and the elements of $Great'$ are greater than $Pivot$.

\end{lemma}
\begin{proof}
By induction on the size of $L$.
\end{proof}

\begin{theorem}[Sort theorem]

If a $\mathtt{down}(A, L)$ fact exists then it will be consumed and a
$\mathtt{up}(A, L')$ fact will be derived, where $L'$ is the sorted list of $L$.

\end{theorem}
\begin{proof}
By induction on the size of $L$.

The base cases are proven trivially (rules 1-4).

In the inductive case, only rule 5 applies:
\begin{Code}
down(A, [Pivot | Xs]) -o split(A, Pivot, Xs, [], []).
\end{Code}

\noindent which necessarily derives a $\mathtt{split}(A, Pivot, Xs, [], [])$
fact. By applying the split lemma, a $\mathtt{split}(A, Pivot, [], Smaller, Greater)$
fact is generated, from which only rule 6 can be used:

\begin{Code}
split(A, Pivot, [], Smaller, Greater) -o
   exists B, C. (back(B, A), back(C, A),
                 down(B, Smaller), down(C, Greater), waitpivot(A, Pivot, B, C)).
\end{Code}

\noindent which necessarily derives $\mathtt{back}(B, A)$, $\mathtt{back}(C,
A)$, $\mathtt{down}(B, Smaller)$, $\mathtt{down}(C, Greater)$ and also a
$\mathtt{waitpivot}(A, Pivot, B, C)$ fact. The semantics of LM ensure that $B$
and $C$ are fresh node addresses, therefore those new facts will be derived on
nodes with no facts. The lists $Smaller$ and $Greater$ are both smaller (in
size) than $L$, so, by the the induction hypothesis, an $\mathtt{up}(B,
Smaller')$ and an $\mathtt{up}(C, Greater')$ facts are derived. These last two
facts will be used in the following rule:

\begin{Code}
up(A, L), back(A, B) -o sorted(B, A, L).
\end{Code}

\noindent which generates a $\mathtt{sorted}(A, B, Smaller')$ and a $\mathtt{sorted}(A, C,
Greater')$ facts. In the continuation, there is only one rule that accepts
\code{sorted} and \code{waitpivot} facts:

\begin{Code}
waitpivot(A, Pivot, NodeSmaller, NodeGreater),
sorted(A, NodeSmaller, Smaller),
sorted(A, NodeGreater, Greater)
   -o up(A, Smaller ++ [Pivot | Greater]).
\end{Code}

\noindent returning $\mathtt{up}(A, Smaller' ++ [Pivot | Greater'])$.  We know that
$Smaller' ++ [Pivot | Greater']$ is sorted since $Small'$ contains the sorted
list of elements lesser or equal than $Pivot$ and $Greater'$ the elements
greater than $Pivot$.

\end{proof}


