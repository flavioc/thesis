
Fig.~\ref{code:language:message} shows the first LM program, a message routing
program that simulates message transmission through a network of nodes. In
lines~\ref{line:language:message_pred1}-\ref{line:language:message_pred2} we
declare the predicates used in the program's rules. Note that the first argument
of every predicate must be typed as \code{node} because the first argument
indicates where the node lives in the graph.  Predicate \code{edge} is a
\emph{persistent predicate} while \code{message} and \code{processed} are
\emph{linear predicates}. Persistent predicates model facts that are never
retracted from the database, while linear predicates model linear facts which
are retracted when used in rules. To improve readability of LM rules, persistent
predicates are preceded with a \code{!} symbol. Predicate \code{edge} represents
the connections between nodes, predicate \code{message} contains the message
content and the route list, and predicate \code{processed} keeps count of the
number of messages routed at each node.

\begin{figure}[h!]
\begin{Verbatim}[numbers=left,commandchars=\*\{\},fontsize=\codesize]
type edge(node, node Neighbor).*label{line:language:message_pred1}
type linear message(node, string Content, list node Routing).
type linear processed(node, int Total).*label{line:language:message_pred2}

!edge(A, B),*label{line:language:message_first1}
message(A, Content, [B | L]),
processed(A, N)
   -o message(B, Content, L),
      processed(A, N + 1).*label{line:language:message_first2}

message(A, Content, []),*label{line:language:message_second1}
processed(A, N)
   -o processed(A, N + 1).*label{line:language:message_second2}

!edge(@1, @2). !edge(@2, @3). !edge(@3, @4). !edge(@1, @3).
processed(@1, 0). processed(@2, 0). processed(@3, 0). processed(@4, 0).
message(@1, "hello world", [@3, @4]).*label{line:language:message_message}
\end{Verbatim}
\caption{Code for routing messages in a graph. There is only one message ("hello
world") to route through nodes \code{@3} and \code{@4}.}
\label{code:language:message}
\end{figure}

The message routing program in Fig.~\ref{code:language:message} implements two
rules in
lines~\ref{line:language:message_first1}-\ref{line:language:message_first2}. An
LM rule has the form $L_1, \cdots, L_n \mathtt{-o} \; R_1, \cdots, R_m$, where
$L_1, \cdots, L_n$ is the \emph{left-hand side}~(LHS) of the rule while $R_1,
\cdots, R_m$ is \emph{right-hand side}~(RHS) of the rule. The meaning of a rule
is then as follows: if facts $L_1, \cdots, L_n$ are present in the database then
retract facts $L_i$ that are linear facts and derive the facts $R_1, \cdots,
R_m$ from the RHS. Note that in all the rules, the LHS of each rule uses facts
from the same node (in this particular case, \code{A}), but the rule's RHS may
derive facts in other nodes (\code{B} in the first rule) because a node variable
was instantiated in the LHS.

The first rule
(lines~\ref{line:language:message_first1}-\ref{line:language:message_first2})
grabs the next node in the route list (third argument of \code{message/3}) and
ensures that a communication edge exists (through \code{edge(A,~B)}). We
increase the number of processed messages by consuming \code{processed(A,~N)}
and deriving \code{processed(A,~N+1)}.  When the route list is empty, the
message has reached its destination and thus it is consumed (rule in lines
\ref{line:language:message_first1}-\ref{line:language:message_first2}).  Note
that we only need to send one message since there is only one \code{message}
axiom (line~\ref{line:language:message_message}).

In Fig.~\ref{fig:message_trace} we present an execution trace of the message
routing program.  The database is represented as a graph structure where the
edges represent the \code{edge/2} initial facts. In
Fig.~\ref{fig:message_trace}~(a) the database is initialized with the program's
initial facts. Note that the initial \code{message/3} fact is instantiated at
node \code{@1}. After applying rule 1, we get the database represented in
Fig.~\ref{fig:message_trace}~(b), where the message has been derived at node
\code{@3}. After applying rule 1 again, the message is then routed to node
\code{@4} (Fig.~\ref{fig:message_trace}~(c)) where it will be consumed
(Fig.~\ref{fig:message_trace}~(d)).

\begin{figure}[h]
        \centering
        \begin{subfigure}[b]{0.4\textwidth}
                \includegraphics[width=\textwidth]{figures/message/message_trace1}
                \caption{Initial database.}
                \label{fig:message_trace1}
        \end{subfigure}%
        ~
        \begin{subfigure}[b]{0.4\textwidth}
                \includegraphics[width=\textwidth]{figures/message/message_trace2}
                \caption{After applying rule 1 at node \code{@1}.}
                \label{fig:message_trace2}
        \end{subfigure}\\
        \begin{subfigure}[b]{0.4\textwidth}
                \includegraphics[width=\textwidth]{figures/message/message_trace3}
                \caption{After applying rule 1 at node \code{@3}.}
                \label{fig:message_trace3}
        \end{subfigure}%
        ~
        \begin{subfigure}[b]{0.4\textwidth}
                  \includegraphics[width=\textwidth]{figures/message/message_trace4}
                  \caption{After applying rule 2 (nodes \code{@4}).}
                  \label{fig:message_trace4}
          \end{subfigure}
        \caption{An execution trace for the message program. The message "hello
        world" travels from node \code{@1} to node \code{@4}.}\label{fig:message_trace}
\end{figure}

